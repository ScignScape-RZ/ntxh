\p{Most CyberPhysical Systems are connected to a software hub 
which takes responsibility for monitoring, validating, and 
documenting the state of the system's networked devices.  
Developing robust, user-friendly central software is an
essential project in any CyberPhysical Systems deployment.  
In this chapter, I will refer to systems' central software 
as their \q{software hub}.  Implementing software hubs 
introduces technical challenges which are distinct from 
manufacturing CyberPhysical devices themselves \mdash{} in particular, 
devices are usually narrowly focused on a particular 
kind of data and measurement, while software hubs are 
multi-purpose applications that need to understand and 
integrate data from a variety of different kinds of 
devices.  CyberPhysical software hubs also 
present technical challenges that are different from 
other kinds of software applications, even if 
these hubs are one specialized domain in the larger 
class of user-focused software applications.
}
\p{Any software application provides human users with tools to 
interactively and visually access data and computer 
files, either locally (data encoded on the \q{host} computer running 
the software) or remotely (data accessed over a network).  
Computer programs can be generally classified as 
\i{applications} (which are designed with a priority 
to User Experience) and \i{background processes} 
(which often start and maintain their state automatically 
and have little input or visibility to human users, 
except for special troubleshooting circumstances).  
Applications, in turn, can be generally classified as 
\q{web applications} (where users usually see one 
resource at a time, such as a web page displaying some 
collection of data, and where data is usually stored 
on remote servers) and \q{native applications} 
(which typically provide multiple windows and 
Graphical User Interface components, and 
which often work with data and files saved 
locally \mdash{} i.e., saved on the filesystem 
of the host computer).  
Contemporary software design also recognizes
\q{hybrid} applications which combine features 
of web and of native (desktop) software.   
}
\p{Within this taxonomy, the typical CyberPhysical 
software hub should be classified as a native, 
desktop-style application, representing the 
state of networked devices through 
special-purpose Graphical User Interface 
(\GUI{}) components.  Networked CyberPhysical 
devices are not necessarily connected to the 
Internet, or communicate via Internet 
protocols.  In many cases, software hubs will 
access device data through securitized, closed-circuit 
mechanisms which (barring malicious intrusion) ensure 
that only the hub application can read or alter 
devices' state.  Accordingly, an application reading 
device data is fundamentally different than a 
web application obtaining information from an Internet 
server.\footnote{It may be appropriate for some device data \mdash{} either 
in real time or retroactively \mdash{} to be shared 
with the public via Internet connections, but 
this is an additional feature complementing 
the monitoring software's primary oversight roles.
}  CyberPhysical networks are designed to 
prioritize real-time connections between device and 
software points, and minimize network latency.  
Ideally, human monitors should be able 
(via the centralized software) to alter device state 
almost instantaneously.  Moreover, in contrast to 
Internet communications with the \TCP{} protocol, 
data is canonically shared between devices and 
software hubs in complete units \mdash{} rather than 
in packets which the software needs to reassemble.  
These properties of CyberPhysical networks imply 
that software design practices for monitoring 
CyberPhysical Systems are technically different 
than requirements for web-based components, such as 
\HTTP{} servers.    
}
\p{At the same time, we can assume that an increasing quantity of 
CyberPhysical data \i{will} be shared via the World Wide Web.  
This reflects a confluence of societal and technological 
forces: public demand is increasing for access both to 
conventional medical information and to real-time health-related 
data (often via \q{wearable} sensors and other technologies that, 
when properly deployed, can promote health-conscious lifestyles).  
Similarly, the public demands greater transparency for 
civic and environmental data, and science is learning how to 
use CyberPhysical technology to track ecological conditions and 
urban infrastructure \mdash{} analysis of traffic patterns, for 
instance, can help civic planners optimize public transit routes 
(which benefit both the public and the environment).
}
\p{Meanwhile, parallel to the rise of accessible health or civic data,
companies are bringing to market an increasing 
array of software products and \q{apps} which access and 
leverage this data.  These applications do not necessarily 
fit the profile of \q{hub software}.  Nevertheless, 
it is still useful to focus attention on the design and 
securitization of hub software, because hub-software
methodology can provide a foundation for the design of 
other styles of application that access CyberPhysical data.  
Over time, we may realize that relatively \q{light-weight} 
portals like web sites and phone apps are suboptimal 
for interfacing with CyberPhysical networks \mdash{} too 
vulnerable and/or too limited in User Interface features.  
In that scenario, software used by the general public 
may adopted many of the practices and implementations of 
mainframe hub applications. 
}
\p{As I argued, software hubs have different design 
principles than web or phone apps.  
Because they deal with raw 
device data (and not, for example, primarily 
with local filesystem files), software hubs also 
have different requirements than conventional 
desktop applications.  As CyberPhysical Systems 
become an increasingly significant part of 
our Information Technology ecosystem, it will 
be necessary for engineers to developed 
rigorous models and design workflows modeled 
expressly around the unique challenges and 
niche specific to CyberPhysical software hubs.
}
\p{Hubs have at least three key responsibilities: 
\begin{enumerate}[leftmargin=5mm] 
\item{} To present device and system data for human 
users, in graphical, interactive formats suitable 
for humans to oversee the system and intervene 
as needed.
\item{} To validate device and system data ensuring 
that the system is behaving correctly and predictably.
\item{} To log data (in whole or in part) for subsequent 
analysis and maintenance.
\end{enumerate}
Prior to each of those capabilities is of course receiving 
data from devices and pooling disparate data profiles into 
a central runtime-picture of device and system state.   
It may be, however, that direct device connection is 
proper not to the software hub itself but to 
drivers and background processes that are computationally 
distinct from the main application.   Therefore, a 
theoretical model of hub software design should assume 
that there is an intermediate layer of background 
processes separating the central application from 
the actual physical devices.  Engineers can 
assume that these background processes communicate 
information about device state either by exposing 
certain functions which the central application 
can call (analogous to system kernel functions) 
or by sending signals to the central application 
when devices' state changes.  I will discuss these 
architectural stipulations more rigorously later in 
this chapter. 
}
\p{Once software receives device data, it needs to 
marshal this information between different formats, 
exposing the data in the different contexts of 
\GUI{} components, database storage, and 
analytic review.  Consider the example of a 
temperature reading, with \GPS{} device location and 
timestamp data (therefore a four-part structure 
giving temperature at one place and time).  
The software needs, in a typical scenario, to do 
several things with this information: it has 
to check the data to ensure it fits within 
expected ranges (because malformed data can indicate 
physical malfunction in the devices or the network).  
It may need to show the temperature reading to a 
human user via some visual or textual indicator.  
And it may need to store the reading in a database 
for future study or troubleshooting.  In these 
tasks, the original four-part data structure is 
transformed into new structures which are 
suitable for verification-analytics, \GUI{} programming, 
and database persistence, respectively.     
}
\p{The more rigorously that engineers understand and document 
the morphology of information across these different  
software roles, the more clearly we can 
define protocols for software design and user expectations.  
Careful design requires answering many technical questions: 
how should the application respond if it encounters 
unexpected data?  How, in the presence of erroneous data, 
can we distinguish device malfunction from coding error?  
How should application users and/or support staff 
be notified of errors?  What is the optimal Interface Design 
for users to identify anomalies, or identify situations 
needing human intervention, and then be able to 
perform the necessary actions via the software?  
What kind of database should hold system data retroactively, 
and what kind of queries or analyses should engineers 
be able to perform so as to study system data, to access the 
system's past states and performance? 
}
\p{I believe that the software development community has neglected 
to consider general models of CyberPhysical software
which could answer these kinds of questions in a rigorous, 
theoretically informed manner.  There is of course a robust 
field of cybersecurity and code-safety, which establishes 
Best Practices for different kinds of computing projects.  
Certainly this established knowledge can and does influence 
the implementation of software connected to CyberPhysical 
systems no less than any other kind of software.  But 
models of programming Best Practices are often associated 
with specific coding paradigms, and therefore reflect 
implementations' programming environment more than they 
reflect the empirical domain targeted by a particular 
software project.
}
\p{For example, Object-Oriented Programming, 
Functional Programming, and Web-Based Programming present 
different capabilities and vulnerabilities and therefore
each have their own \q{Best Practices}.  As a result, 
our understanding of how to deploy robust, well-documented 
and well-tested software tends to be decentralized
among distinct programming styles and development 
environments.  External analysis of a code base \mdash{} e.g., searching 
for security vulnerabilities (attack routes for malicious code) 
\mdash{} are then separate disciplines with their own methods 
and paradigms.  Such dissipated wisdom is unfortunate if 
we aspire to develop integrated, broadly-applicable models 
of CyberPhysical safety and optimal application 
design, models which transcend paradigmatic 
differences between coding styles and roles 
(treating implementation, testing, and code 
review as distinct technical roles, for instance).
}
\p{It is also helpful to distinguish cyber 
\i{security} from \i{safety}.  When these concepts are 
separated, \i{security} generally refers to
preventing \i{deliberate}, \i{malicious} intrusion into 
CyberPhysical networks.  Cyber \i{safety} refers to preventing 
unintended or dangerous system behavior due to innocent human 
error, physical malfunction, or incorrect programming.  
Malicious attacks \mdash{} in particular the risks of 
\q{cyber warfare} \mdash{} are prominent in the 
public imagination, but innocent coding errors or design 
flaws are equally dangerous.  Incorrect data readings, 
for example, led to recent Boeing 737 MAX jet accidents 
causing over 200 fatalities (plus the worldwide grounding
of that airplane model and billions of dollars in losses 
for the company).  Software failures either 
in runtime maintenance or anticipatory risk-assessment 
have been identified as contributing factors to 
high-profile accidents like Chernobyl \cite{MikhailMalko} 
and the Fukushima nuclear reactor 
meltdown \cite{JoonEonYang}.
A less tragic but noteworthy 
case was the 1999 crash of NASA's US \$125 million 
Mars Climate Orbiter.  This crash was caused by 
software malfunctions which in turn were caused 
by two different software components producing 
incompatible data \mdash{} in particular, using 
incompatible scales of measurement 
(resulting in an unanticipated mixture of 
imperial and metric units).  In general, it 
is reasonable to assume that coding errors 
are among the deadliest and costliest sources 
of man-made injury and property damage. 
}
\p{Given the risks of undetected data corruption, seemingly 
mundane questions about how CyberPhysical applications verify 
data \mdash{} and respond to apparent anomalies \mdash{} 
become essential aspects of planning and development.  
Consider even a simple data aggregate like 
blood pressure (combining systolic and 
diastolic measurements).  Empirically, systolic pressure is 
always greater than diastolic.  Software systems 
need to agree on a protocol for encoding the number to 
ensure that they are in the correct order, and that they 
represent biologically plausible measurements.  
How should a particular software component test that 
received blood pressure data is accurate?  Should it 
always test that the systolic quantity is indeed 
greater than the diastolic, and that both numbers 
fall in medically possible ranges?  How should the 
component report data which fails this test?  If 
such data checking is not performed \mdash{} on the 
premise that the data will be proofed elsewhere 
\mdash{} then how can this assumption be 
justified?  How can engineers identify, in a 
large and complex software system, all the points 
where data is subject to validation tests; and 
then by modeling the overall system in term 
of these check-points ensure that all needed 
verifications are performed at least one time?  
To take the blood-pressure example, 
how would a software procedure that \i{does} 
check the integrity of the systolic/diastolic 
pair indicate for the overall system model 
that it performs that particular verification?  
Conversely, how would a procedure which does 
\i{not} perform that verification indicate 
that this verification must be performed 
elsewhere in the system to guarantee that 
the procedure's assumptions are satisfied?    
} 
\p{These questions are important not only for objective, 
measurable assessments of software quality, but 
also for people's more subjective trust in the reliability 
of software systems.  In the modern world we 
allow software to be a determining factor, in places
where malfunction can be fatal \mdash{} airplanes, hospitals, 
electricity grids, trains carrying toxic chemicals, 
highways and city streets, etc.  
Consider the model of \q{Ubiquitous Computing} pertinent to the
book series to which this volume (and hence
this chapter) belongs.  As explained in the
series introduction: 
\begin{displayquote}U-healthcare systems ... will allow physicians to remotely diagnose, access, and monitor critical patient's symptoms and will enable real time communication with patients.  [This] 
series will contain systems based on the four future ubiquitous sensing for healthcare (USH) principles, namely i) proactiveness, where healthcare data transmission to healthcare providers has to be done proactively to enable necessary interventions, ii) transparency, where the healthcare monitoring system design should transparent, iii) awareness, where monitors and devices should be tuned to the context of the wearer, and iv) trustworthiness, where the personal health data transmission over a wireless medium requires security, control and authorize access.\footnote{\url{https://sites.google.com/view/series-title-ausah/home?authuser=0}
}  
\end{displayquote}
Observe that in this scenario, patients will have to 
place a level of trust in Ubiquitous Health technology comparable 
to the trust that they place in human doctors and other 
health professionals.   
}
\p{All of this should cause software engineers and developers to 
take notice.  Modern society places trust in doctors 
for well-rehearsed and legally scrutinized reasons: 
physicians need to rigorously prove their competence 
before being allowed to practice medicine, and 
this right can be revoked due to malpractice.  Treatment 
and diagnostic clinics need to be licenced, 
and pharmaceuticals (as well as medical equipment) are subject 
to rigorous testing and scientific investigation before being 
marketable.  Notwithstanding \q{free market} ideologies, 
governments are aggressively involved in regulating 
medical practices; commercial practices (like marketing) are 
constrained, and operational transparency 
(like reporting adverse outcomes) is mandated, more so 
than in most other sectors of the economy.  This level of 
oversight \i{causes} the public to trust that clinicians' 
recommendations are usually correct, or that medicines are 
usually beneficial more than harmful.  
}
\p{The problem, as software becomes an increasingly central feature 
of the biomedical ecosystem, is that no commensurate oversight 
framework exists in the software world.
Biomedical \IT{} regulations tend to be ad-hoc and narrowly domain-focused. 
For example, code bases in the United States which manage HL-7 
data (the current federal Electronic Medical Record format) must 
meet certain requirements, but there is no comparable framework 
for software targeting other kinds of health-care information.  
This is not only \mdash{} or not primarily \mdash{} an issue of 
lax government oversight.  The deeper problem is that 
we do not have a clear picture, in the framework of 
computer programming and software development, of 
what a robust regulatory framework would look like: what 
kind of questions it would ask; what steps a company could 
follow to demonstrate regulatory compliance; what indicators 
the public should consult to check that any software 
that could affect their medical outcomes is properly vetted.  
And, outside the medical arena, similar comments could be 
made regarding software in CyberPhysical settings like 
transportation, energy (power generation and electrical 
grids), physical infrastructure, environmental protections, 
government and civic data, and so forth 
\mdash{} settings where software errors threaten personal
and/or property damages.
}
\p{In the case of personal medical data, as one example, there is 
general agreement that data should be accessed when it is
medically necessary \mdash{} say, in an emergency room 
\mdash{} but that each patient should mostly control how and whether 
their data is used.  When data is pooled for epidemiological 
or meta-analytic studies, we generally believe that such information 
should be anomalized so that socioeconomic or \q{cohort} data is 
considered, whereas unique \q{personal} data 
remains hidden.  These seem like common-sense requirements.  
However, they rely on concepts which we may intuitively 
understand, but whose precise definitions are elusive 
or controversial.  What exactly does it mean to distinguish 
uniquely \i{personal} data, that is indelibly fixed 
to one person and therefore particularly sensitive as a matter 
of due privacy, from \i{demographic} data which is also personal 
but which, tieing patients to a cohort of their peers, is 
of potential public interest insofar as race, gender, and 
other social qualities can sometimes be statistically significant?  
How do privacy rights intersect with the legitimate desire to 
identify all scientific factors that can affect epidemiological trends 
or treatment outcomes?  More deeply, how should we 
actually demarcate \i{demographic} from \i{personal} data?  What 
details indicate that some part of some data structure is one or 
the other?
}
\p{More fundamentally, what exactly is data sharing?  What are the 
technical situations such that certain software operations are 
to be \i{sharing} data in a fashion that triggers concerns 
about privacy and patient oversight?  Although again we may 
intuitively picture what \q{data sharing} entails, producing a 
rigorous definition is surprisingly difficult.
}
\p{In short, the public has a relatively inchoate 
idea of issues related to cyber safety, security, and 
privacy: we (collectively) have an informal impression that 
current technology is failing to meet the public's desired 
standards, but there is no clear picture of what 
\IT{} engineers can or should do to improve the technology 
going forward.  Needless to say, software should prevent 
industrial catastrophes, and private financial data 
should not be stolen by crime syndicates.  But, beyond these 
obvious points, it is not clearly defined how 
the public's expectations for safer and more secure 
technology translates to low-level programming practices.  
How should developers earn public trust, and 
when is that trust deserved?  Maxims like \q{try to avoid 
catastrophic failure} are too self-evident to be useful.  
We need more technical structures to identify 
which coding practices are explicitly recommended, 
in the context of a dynamic where engineers need 
to earn the public trust, but also need to define 
the parameters for where this trust is warranted.    
Without software safety models rooted in low-level 
computer code, software safety can only be ex-post-facto 
engineered, imposing requirements relatively late in the 
development cycle and checking code externally, via 
code review and analysis methods that are beyond
the core development process.  While such secondary 
checking is important, it cannot replace software built 
with an eye to safety from the ground up. 
}
\p{This chapter, then, is written from the viewpoint that 
cyber safety practices have not been clearly articulated 
at the level of software implementation itself, 
separate and apart from institutional or governmental oversight.  
Regulatory oversight is only effective in proportion to 
scientific clarity \visavis{} desired outcomes and how 
technology promotes them.  Drugs and treatment protocols, 
for instance, can be evaluated through \q{gold standard} 
double-blind clinical trials \mdash{} alongside statistical 
models, like \q{five-sigma} criteria, which measure 
scientists' confidence that trial results are truly 
predictive, rather than results of random chance.  This package 
of scientific methodology provides a framework which can 
then be adopted in legal or legislative contexts.  
Continuing the example, policy makers can stipulate that 
pharmaceuticals need to be tested in double-blind trials, 
with statistically verifiable positive results, before 
being approved for general-purpose clinical use.  Such a  
well-defined policy approach \i{is only possible} because 
there are biomedical paradigms which define how treatments 
can be tested to maximize the chance that positive test 
results predict similar results for the 
general patient population.
}
\p{Analogously, a general theory of cyber safety has 
to be a software-design issue before it becomes a 
policy or contractual issue.  It is at the level of 
low-level software design \mdash{} of actual source code 
in its local implementation and holistic integration 
\mdash{} that engineers can develop technical \q{best practices} 
which then provide substance to regulative oversight.  
Stakeholders or governments can recommend (or require) that 
certain practices adopted, but only if engineers 
have identified practices which are believed, 
on firm theoretical ground, to effectuate safer, 
more robust software.  
}
\p{This chapter, then, considers code-safety from the 
perspective of computer code outward; it is grounded 
on code-writing practice and in the theoretical 
systems which have historically been linked to 
programming (like type theory and lambda 
calculus), yielding its scientific basis.  
I assume that formal safety models 
formulated in this low-level context can propagate 
to institutional and governmental stake-holders, 
but discussion of the legal or contractual 
norms that can guide software practice are 
outside the chapter's central scope. 
}
\p{In the CyberPhysical context, I assume here 
that the most relevant software projects 
are hub applications; and that the preeminent 
issues in cyber safety are validating 
data and responding safely and predictably 
to incorrect or malformed data.  
Here we run into gaps between proper safety 
protocols and common programming practice 
and programming language design.  In particular, 
most mainstream languages have limited 
\i{language-level} support for foundational 
safety practices such as dimensional checking 
(ensuring that algorithms do not work with 
incommensurate measurement axes) or range checking 
(ensuring that inaccurate CyberPhysical data is 
properly identified as such \mdash{} in the hopes 
of avoiding cases like the Boeing 737 crashes, 
where onboard software failed to recognize 
inaccurate data from angle-of-attack sensors).  
More robust safety models are often implicit in 
software libraries, outside the core language; however, 
to the degree that such libraries are considered 
experimental, or tangential to core language features, 
they are not likely to \q{propagate} outside the 
narrow domain of software development proper.  
To put it differently, no safety model appears to have 
been developed in the context of any mainstream 
programming language far enough that the 
very existence of such a model provides a concrete 
foundation for stakeholders to define requirements 
that developers can then follow.
}
\thindecoline{}
\p{This chapter's discussion will be oriented toward 
the \Cpp{} programming language, which is arguably the most
central point from which to consider the integration 
of concerns \mdash{} \GUI{}, device networking, analytics 
\mdash{} characteristic of CyberPhysical hub software.  
In practice, low-level code that interfaces with 
devices (or their drivers) might be written in 
\CLang{} rather than \Cpp{}; likewise, there is often a role 
for functional programming languages \mdash{}  
even theorem-proving systems \mdash{} in mission-critical 
data checking and system design validation.  But 
\Cpp{} is unique in having extensive resources 
traversing various programming domains, like 
native \GUI{} components alongside low-level networking 
and logically rigorous data verification.  For 
this reason \Cpp{} is a reasonable default language 
for examining how these various concerns interoperate.   
}
\p{In that spirit, then, the \Cpp{} core language is a 
good case-study in language-level 
cyber-safety support (and the lack thereof).  
There are numerous \Cpp{} libraries, mostly from 
scientific computing, which provide features that would 
be essential to a robust cyber safety model (such as 
bounded number types and unit-of-measurement types).  
If some version of these libraries were adopted 
into a future \Cpp{} standard (analogous to the 
\q{concepts} library, a kind of metaprogramming validator, 
which has been included in \CppTwenty{} after many 
years of preparation), then \Cpp{} coders 
would have a canonical framework for safety-oriented 
programming \mdash{} a specific set of data types and 
core libraries that could become an essential part of 
critical CyberPhysical components.  That specific 
circle of libraries, along with their scientific and 
computational principles, would then become a 
\q{cyber safety model} available to CyberPhysical applications.  
Moreover, the existence of 
such a model might then serve as a concrete 
foundation for defining coding and project requirements.  
Stakeholders should stipulate that developers use those 
specific libraries intrinsic to the cyber safety model,  
or if this is infeasible, alternate libraries offering 
similar features.
}
\p{Of course, the last paragraph was counterfactual \mdash{} \i{without} 
such a canonical \q{cyber safety model}, there is no 
firm foundation for identifying stakeholder priorities.  
We may have generic guidelines \mdash{} try to protect 
against physical error; try to restrict access to private data 
\mdash{} but we do not have a canonical model, 
integrated with a core language, against which compliant code 
can be designed.  I believe this is a reasonable claim 
to make in the context of \Cpp{}, and most or all other 
mainstream programming languages as well.
}
\p{Having said that, we should not \q{blame} software language engineers 
for gaps in mainstream languages.  It turns out that 
such features as dimensional-unit types and bounded numerics 
are surprisingly difficult to implement, particularly at 
the core language level where such types must seamlessly 
interoperate with all other language features 
(examine the code \mdash{} or even documentation \mdash{} for 
the \boostUnits{} library for a sense of the technical 
intricacies these implementations involve).
Consequently, progress toward core-language cyber safety 
features will be advanced with methodological progress in 
software language design and engineering itself.   
}
\p{But this situation also implies that language designers and 
library developers can play a lead role in establishing a 
safety-oriented CyberPhysical foundation.  Insofar as 
this foundation lies in programming languages and 
software engineering \mdash{} in data types, procedural 
implementations, and code analytics \mdash{} then the responsibility 
for developing a safety-oriented theory and practice 
lies with the software community, not with CyberPhysical device 
makers or with civic or institutional stakeholders.  
The core principles of a next-generation CyberPhysical  
architecture would then be worked out in the context 
of software language design and software-based data 
modeling.  My goal in this chapter is accordingly 
to define what I believe are fundamental and canonical 
structures for theorizing data structures and the 
computer code which operates on them, with an 
eye toward cyber safety and Software Quality Assurance. 
}
\p{In general, software requirements can be studied 
either from the perspective of computer code, 
or from the perspective of data models.
Consider again the requirement that systolic blood pressure 
must always be a greater quantity than diastolic: 
we can define this as a precondition for any code 
which displays, records, or performs computations on 
blood pressure (e.g. comparing a patient's pressure at 
different times).  Such code is only operationally 
well-defined if it is provided data conforming to 
the systolic-over-diastolic mandate.  The code 
\i{should not} execute if this mandate fails.  
Design and testing should therefore guarantee that 
the code \i{will not} execute inappropriately.  
Conversely, these same requirements can be expressed 
within a data model: a structure representing blood 
pressure is only well-formed if its component part 
(or \q{field}) representing systolic pressure 
measures greater than its field representing diastolic pressure.  
}
\p{These perspectives are complementary: 
a database which tracks blood pressure should be 
screened to ensure that all of its data is well-formed 
(including systolic-over-diastolic).  At the same time, 
an application which works with medical data should 
double-check data when relevant procedures are 
called (e.g., those working with blood pressure), 
particularly if the data is from uncertain provenance.
Data could certainly come from multiple databases, 
or perhaps directly from CyberPhysical devices, 
and developers cannot be sure that all sources check 
their data with sufficient rigor (moreover, in the case 
of CyberPhysical sensors, validation in the device 
itself may be impossible).   
}
\p{Conceptually, however, validation through data models and 
code requirements represent distinct methodologies with 
distinct theoretical backgrounds.  This chapter will therefore 
consider both perspectives, as practically aligned but
conceptually \i{sui generis}.  I will also, however, 
argue that certain theoretical foundations \mdash{} particularly 
hypergraph-based data representation, and type systems 
derived from that basis \mdash{} serve as a unifying element.  
I will therefore trace a construction of 
\i{hypergraph-based} type theory across both data- and 
code-modeling methodologies.
}
\p{}
