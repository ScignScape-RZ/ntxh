\documentclass[11pt,twocolumn]{article}
%\usepakage{newfile}

\usepackage{hyperref}

\usepackage{etoolbox}

\usepackage{zref-user}

\newwrite\sdiFile
\immediate\openout\sdiFile=\jobname.sdi.txt

\newcommand{\p}[1]{

\vspace{10pt}#1}

\newif\iftabng
\tabngfalse


\usepackage{letltxmacro}
\LetLtxMacro{\oldmmsemi}{\;}
\LetLtxMacro{\oldtbplus}{\+}
\LetLtxMacro{\oldtbgt}{\>}
\LetLtxMacro{\oldmmgt}{\+}

\newcommand{\+}{\iftabng\oldtbplus\else\sss\fi}

\renewcommand{\>}{\iftabng\oldtbplus\else
\ifmmode\oldmmgt\else\sse\sss\fi\fi}

%\renewcommand{\>}{\sse\sss}

\renewcommand{\;}{\relax\ifmmode\oldmmsemi\else\sse\fi}

\newcommand{\writeSDI}[1]{\immediate\write\sdiFile#1}

\newcommand{\emblink}[2]{\href{\#sdi:#1--#2}{\#sdi:#1--#2}}

%\newcount\sdiCounter
%\def\advsdiCounter{\global\advance\sdiCounter by1}

%\newcount\sdiCounterP
%\def\advsdiCounterP{\global\advance\sdiCounterP by1}

%\newcounter{sdiCounter}
\newcounter{sdiCounterP}[page]
\newcounter{sdiCounter}

\def\topt#1{\expandafter\the\dimexpr\dimexpr#1sp\relax\relax}

\makeatletter
%\catcode`\*=10
\newcommand{\sss}{%
\stepcounter{sdiCounterP}
\stepcounter{sdiCounter}
\pdfsavepos\write\sdiFile{!/ SDI_Sentence_Start} 
\write\sdiFile\expandafter{\expandafter$%
\expandafter i\expandafter:%
\expandafter\space\the\c@sdiCounter}
\write\sdiFile\expandafter{\expandafter$%
\expandafter o\expandafter:%
\expandafter\space\the\c@sdiCounterP}
\write\sdiFile\expandafter{\expandafter$%
\expandafter p\expandafter:%
\expandafter\space\thepage^^J%
$x: \topt\pdflastxpos^^J%
$y: \topt\pdflastypos^^J%
/!^^J%
<<>^^J%
}}
%\catcode`\%=14
\makeatother

\makeatletter
\newcommand{\sse}{%
\pdfsavepos\write\sdiFile{!/ SDI_Sentence_End} 
\write\sdiFile\expandafter{\expandafter$%
\expandafter i\expandafter:%
\expandafter\space\the\c@sdiCounter}
\write\sdiFile\expandafter{\expandafter$%
\expandafter o\expandafter:%
\expandafter\space\the\c@sdiCounterP}
\write\sdiFile\expandafter{\expandafter$%
\expandafter p\expandafter:%
\expandafter\space\thepage^^J%
$x: \topt\pdflastxpos^^J%
$y: \topt\pdflastypos^^J%
/!^^J%
<<>^^J%
}}
\makeatother


\begin{document}\title{Hypergraph-Based Type Theory for Requirements Engineering and
Generalized Lambda Calculus (with a case study in safety protocols 
for biomedical devices)}
\author{Nathaniel Christen}
\twocolumn[\begin{@twocolumnfalse}
\maketitle{}
\begin{abstract}This chapter will explore the integration of several methodologies 
related to source code analysis and software Requirements 
Engineering.  In particular, I will review Semantic Web 
and general graph-based representations of source code, 
alongside applied type theory (for expressing 
programming languages' type systems).  The purpose of this 
integration is to develop a systematic (albeit not 
rigorously mathematical) account of foundational programming 
elements such as functions/procedures, function calls, and 
inter-procedural information flows.  The principal relatively new 
representational device I suggest here involves a theory 
of \q{channels} which ties together models associated with 
lambda calculi, type theory, and graph-based code representation.  
I will argue that the perspective afforded by channels permits 
succinct protocols for describing procedures, procedural 
semantics, and holistic program behavior.  In practice, I 
outline how the proposed techniques support documentation and 
verification of procedural, data type, and holistic 
specifications \mdash{} implementational assumptions on procedures  
and/or modeling assumption on types.  I argue that 
we need new methodologies for indicating and testing 
programming assumptions/specifications insofar as 
there are several options for handling requirements in 
a software context, each with their own trade-offs.  
Requirements Engineering protocols should therefore 
be developed within the foundational planning and 
prototyping stages of a project, particularly 
in CyberPhysical contexts which tend to exhibit both 
empirically-driven data types (with granular 
modeling parameters) and strong robustness mandates.
An open-source accompanying data set (at 
\url{https://github.com/scignscape/ntxh}) 
demonstrates code libraries concretizing 
techniqes outlined here. 
\end{abstract} 
%\begin{flushright}%\usebox{\qboxi}
%\usebox{\qboxii}
%\end{flushright}
\decoline{}
\vspace{3em}
\end{@twocolumnfalse}]
\addcontentsline{toc}{section}{Hypergraph-Based Type Theory ...}
\p{It is possible to look at CyberPhysical systems in isolation: there is 
one kind of CyberPhysical \i{device}, a sensor or actuator, that 
generates data and/or can effectuate physical change in its 
surrounding environment.  Potentially there are two or more kinds 
of devices which generate overlapping or interconnected kinds 
of data.  CyberPhysical \i{signals} are then sent to a  
\i{software} point where the input is interpreted and transformed for 
practical use: stored in a database, displayed for human users, 
and/or subject to data mining and analysis.  We live in a world 
where more and more kinds of devices proliferate, as do devices 
themselves.  As a result, there are more and more networks generating 
data, and proportionately more access points for people to see and 
act upon that data.
}
\p{CyberPhysical systems can also be seen in a more holistic way.  
As CyberPhysical networks proliferate, we can envision a 
rise in technologies that merge and integrate data from 
many kinds of devices, from many different vendors.  
This eventuality has already been contemplated, including 
in this volume.  Teixera \i{et. al.}, for example, 
argue that 
\begin{dquote}Overall, the increase in sensors, devices, and appliances, in our homes, has transformed it into a rather complex environment with which to interact. This characteristic cannot be merely addressed by a matching set of device-dependent applications, turning the smart home into a set of isolated interactive artifacts. Hence, there is a strong need to unify this experience, blending this diversity into a unique interactive ecosystem. This can be tackled, to a large extent by the proposal of a unique, integrated, ubiquitous distributed smart home application capable of handling a dynamic set of sensors and devices and providing the different house occupants (e.g., children, teenagers, young adults, and elderly) with natural and simple ways of controlling and accessing information. However, the creation of such application presents a challenge, particularly due to the need to support natural and adaptive forms of interaction beyond a simple home dashboard application.
\end{dquote}
In this chapter, I will refer to an \q{application capable of handling a dynamic set of 
sensors and devices} as a \i{hub application}. 
}
\p{Assuming hub applications become popular, this technology may 
fundamentally change how CyberPhysical devices are designed.  
Alongside \q{isolated} access 
points \mdash{} such as Smart Phone apps, which are 
often created by the same companies as market devices 
themselves \mdash{} hub applications would become an alternative  
recipient of CyberPhysical information.  The total 
network \mdash{} encompassing hub applications, along with all 
devices from which they obtain input \mdash{} would 
integrate variegated technologies, offered by many 
different companies.  Standardized formats would need 
to be engineered \mdash{} not only for the stage of 
sending signals between devices and hubs, but for 
the higher-level project of designing application components 
which can understand and display device data.  The essential 
question at this level is not how to transmit \i{raw} data,
but how CyberPhysical components can transform raw 
data into visual, interactive displays for users' benefit. 
}
\p{Hub applications, for their part, must receive data from 
many kinds of devices, but also respond properly to 
data once it is received.  For each kind of device, 
there must then be a corresponding software component,  
part of the hub application, which is specifically 
capable of understanding data generated by \i{that particular} 
device.  For sake of discussion, I will call such 
device-specific software components a \i{hub library}.  
We may assume that hub applications will feature many 
hub libraries, and that implementing hub libraries will 
become an integral step in the process of deploying 
CyberPhysical instruments.  These hub libraies would 
have requirements beyond just low-level \q{driver} code; 
they would also be responsible for validating input, 
confirming that their associated device is working properly, 
and transforming raw data into usable software artifacts.  
In short, hub libraries need to bridge the low-level 
realm of CyberPhysical signals (and the networks that carry them) 
with the high-level realm of software engineering: \GUI{} 
components, data validation, fluid and responsive User Experience, 
and so forth. 
}
\p{In short, hub \i{applications} create a need for hub \i{libraries}, 
which would then become an intrinsic part of device creation.  
A hub library serves as a reference point and prototype for 
device properties: a well implemented library should document 
the kind of data its corresponding device generates; what are its 
structural or numerical properties; and what are the proper algorithms 
for mapping raw data into usable data structures.  Hub libraries, 
in other words, are manifestations of \i{data models} which 
convey technical details about devices; for instance, scientific 
and mathematical details about the proper range and dimensions 
of data fields.   
}
\p{Here, then, hub applications have a significance which may not 
be obvious to their users.  In addition to providing software-based 
access to CyberPhysical information, hub applications would serve 
as a central reference point for conceptualizing how CyberPhysical 
devices operate.  Hub libraries, in particular, would provide a 
concrete artifact that engineers could consult to obtain information 
about device properties, data models, and expected behavior.  
Compared to \q{a set of isolated interactive artifacts}, I believe 
the centralized architecture of hub applications and hub libraries 
is more conducive to rigorous, concrete representation of CyberPhysical 
devices as technical products.  This centralized focus can also 
make CyberPhysical systems more secure and trustworthy \mdash{} 
hub applications can be used for testing and protyping devices and 
their supporting code, even before devices are brought to market.            
}
\p{This chapter is not explicitly about hub applications; 
here I will examine coding and code documentation techniques 
which are applicable in many contexts.  However, CyberPhysical 
hubs are a good case study in programming contexts where   
Requirements Engineering is an intrinsic architectural feature.  
I write this 
chapter, then, from the perspective of a programmer 
creating a \q{hub library} for some form of 
CyberPhysical input.  This programmer needs to express in 
code the physical and computational details specific 
to the device's signals; its functionality and 
capabilities.  The hub library has to verify data integrity 
\i{and also} to profile the shape of data generated 
by properly functioning devices.  The hub library has to serve 
as a reference point, a proxy for the device itself, 
in that engineers may study the library as an indirect 
way of coming to understand the device.  Given 
these requirements, hub libraries need an especially 
rigorous development methodology, one that emphasizes 
strict documentation and verification of coding requirements.    
} 
\p{The first two sections in this chapter will discuss hub 
applications and CyberPhysical systems on a more practical 
level: what are representative examples of the 
data structures and coding requirements that hub 
libraries will need to encapsulate?  I will 
then, in the final two sections, turn to 
computer code at a more theoretical level, outlining 
certain representational paradigms, such as 
Directed Hypergraphs, which I believe can yield more 
expressive and comprehensive models of coding structures 
and requirements.  
}

\section{Gatekeeper Code}
\phantomsection\label{sOne}
\p{There are several design principles which can help ensure 
safety in large-scale, native/desktop-style \GUI{}-based 
applications.  These include:
\begin{enumerate}\item{}  Identify operational relationships between types.  
Suppose \calS{} is a data structure modeled via type \caltypeT{}.  
This type can then be associated with a type (say, 
\typeTp{}) of \GUI{} components which visually display 
values of type \caltypeT{}.  A simple data structure 
may have \GUI{} representation via small \q{widgets} 
embedded in other components (consider a thermometer icon 
to display temperature).  Conversely, if \calS{} has many component 
parts, its corresponding \GUI{} type may need to span its 
own application window, with a collection of nested textual 
or graphical elements.  There may also be a type 
(say, \typeTpp{}) representing \caltypeT{}-values in a format 
suitable for database persistence.  Application code should 
explicitly indicate these sorts of inter-type relationships.
\item{}  Identify coding assumptions which determine the validity 
of typed values and of function calls.  For each 
application-specific data type, consider whether every 
computationally possible instance of that type is actually 
meaningful for the real-world domain which the type represents.  
For instance, a type representing blood pressure has a subset 
of values which are biologically meaningful \mdash{} where systolic 
pressure is greater than diastolic and where both numbers are 
in a sensible range.  Likewise, for every procedure defined 
on application-specific data types, consider whether the procedure 
might receive arguments that are computationally feasible but 
empirically nonsensical.  Then, establish a protocol for 
acting upon erroneous data values or procedure parameters.  
How should the error be handled, without disrupting the 
overall application?
\item{}  Identify points in the code base which represent new data 
being introduced into the application, or code which can materially 
affect the \q{outside world}.  Most of the code behind \GUI{} 
software will manage data being transferred between different 
parts of the system, internally.  However, there will be 
specific code sites \mdash{} e.g., specific procedures \mdash{} which 
receive new data from external sources, or respond to 
external signals.  A simple example is, for desktop applications, 
the preliminary code which runs when users click a mouse button.  
In the CyberPhysical context, an example might be code which 
is activated when motion-detector sensors signal something moving 
in their vicinity.  These are the \q{surface} points where data 
\q{enters the system}.
\pseudoIndent{}
Conversely, other code points localize 
the software's capabilities to initiate external effects.  For 
instance, one consequence of users clicking a mouse button might 
be that the on-screen cursor changes shape.  Or, motion detection 
might trigger lights to be turned on.  In these cases the software 
is hooked up to external devices which have tangible capabilities, 
such as activating a light-source or modifying the on-screen cursor.  
The specific code points which leverage such capabilities 
represent data \q{leaving the system}.  
\pseudoIndent{}
In general, it is important to identify points where data 
\q{enters} and \q{leaves} the system, and to distinguish 
these points from sites where data is transferred 
\q{inside} the application.  This helps ensure that 
incoming data and external effects are properly vetted.  
Several mathematical frameworks have been developed 
which codify the intuition of software components as 
\q{systems} with external data sources and effects, 
extending the model of software as self-contained 
information spaces: notably, Functional-Reactive Programming 
(see e.g. \cite{JenniferPaykin}, 
\cite{PaykinKrishnaswami}, 
\cite{WolfgangJeltsch}) and the theory of 
Hypergraph Categories 
(\cite{InteractingConceptualSpaces}, \cite{BrendanFong}, 
\cite{BrendanFongThesis}, \cite{AleksKissinger}). 
\end{enumerate}
Methods I propose in this chapter are applicable to each 
of these concerns, but for purposes of exposition I 
will focus on the second issue: testing 
type instances and procedure parameters for fine-grained 
specifications (more precise than strong typing alone). 
}
\p{Strongly-typed programming language offer some guarantees on 
types and procedures: a function which takes an integer will 
never be called on a value that is \i{not} an integer 
(e.g., the character-string \q{46} instead of the \i{number} 
46).  Likewise, a type where one field is an integer 
(representing someone's age, say), will never be instantiated 
with something \i{other than} an integer in that field.  
Such minimal guarantees, however, are too coarse for 
safety-conscious programming.  Even the smallest 
(8-bit) unsigned integer type would permit someone's age to 
be 255 years, which is surely an error.  So any 
safety-conscious code dealing with ages needs to check that 
the numbers fall in a range narrower than built-in 
types allow on their own, or to ensure that such checks are 
performed ahead of time.   
}
\p{The central technical challenge of safety-conscious coding 
is therefore to \i{extend} or \i{complement} each programming 
languages' built-in type system so as to represent 
more fine-grained assumptions and specifications.  
While individual tests may seem straightforward on a 
local level, a consistent 
data-verification architecture \mdash{} how this coding dimension 
integrates with the totality of software features and 
responsibility \mdash{} can be much more complicated.  
Developers need to consider several overarching questions, 
such as: 
\begin{itemize}\item{} Should data validation be included in the same 
procedures which operate on (validated) data, or 
should validation be factored into separate procedures?
\item{} Should data validation be implemented at the type 
level or the procedural level?  That is, should specialized 
data types be implemented that are guaranteed only to 
hold valid data?  Or should procedures work with more 
generic data types, and perform validations on a case-by-case 
basis?
\item{} How should incorrect data be handled?  In CyberPhysical software, 
there may be no obvious way to abort an operation in the 
presence of corrupt data.  Terminating the application may not be 
an option; silently canceling the desired operation or trying to substitute 
\q{correct} or \q{default} data may be unwise; and 
presenting technical error messages to human users may be confusing.  
\end{itemize}
These questions do not have simple answers.  As such, we 
should develop a rigorous theoretical framework so as to 
codify the various options involved \mdash{} what architectural 
decisions can be made, and what are the strengths and weaknesses 
of different solutions.
}
\p{This section will sketch an overview of the data-validation 
issues from the broader vantage of planning and stakeholder 
expectations, before addressing narrower programming concerns
in subsequent sections.
} 
\subsection{Gatekeeper Code and Fragile Code}
\p{I will use the term \i{gatekeeper code} for any code which checks 
programming assumptions more fine-grained than strong typing 
alone allows \mdash{} for example, that someone's age is not reported 
as 255 years, or that systolic pressure is not recorded as 
less than diastolic.  I will use the term \i{fragile code} for
code which \i{makes} programming assumptions \i{without itself} 
verifying that such assumptions are obeyed.  Fragile code is 
especially consequential when incorrect data would cause the 
code to fail significantly \mdash{} to crash the application, 
enter an infinite loop, or any other nonrecoverable scenario.
}
\p{Note that \q{fragile} is not a term of criticism \mdash{} some algorithms 
simply work on a restricted space of values, and it is inevitable 
that code implementing such algorithms will only behave properly 
when provided values with the requisite properties.  It is necessary 
to ensure that such algorithms are \i{only} called with 
correct data.  But insofar as testing of the data lies outside 
the algorithms themselves, the proper validation has to occur 
\i{before} the algorithms commence.  In short, \i{fragile} and
\i{gatekeeper} code often has to be paired off: for each 
segment of fragile code which \i{makes} assumptions, there should 
be a corresponding segment of gatekeeper code which
\i{checks} those assumptions.  
}
\p{In that general outline, however, there is room for a variety 
of coding styles and paradigms.  Perhaps these can be broadly 
classified into three groups: 
\begin{enumerate}\item{} Combine gatekeeper and fragile code in one procedure.
\item{} Separate gatekeeper and fragile code into different procedures.
\item{} Implement narrower types so that gatekeeper code is 
called when types are first instantiated.
\end{enumerate}
Consider a function which calculates the difference between 
systolic and diastolic blood pressure, returning an unsigned
integer.  If this code were called with malformed data wherein 
systolic and diastolic are inverted, the difference would 
be a negative number, which (under binary conversion to an 
unsigned integer) would come out as a potentially 
extremely large positive number (as if the patient had 
blood pressure in, say, the tens-of-thousands).  This nonsensical 
outcome indicates that the basic calculation is fragile.  
We then have three options: test \q{systolic-greater-than 
diastolic} \i{within the procedure}; require that this test 
be performed prior to the procedure being called; 
or use a special data structure configured such that 
systolic-over-diastolic can be confirmed as soon as 
any blood-pressure value is constructed in the system.
}
\p{There are strengths and weaknesses of each option.  
Checking parameters at the start of a procedure makes 
code more complex and harder to maintain, and also 
makes updating the code more difficult.  The 
blood-pressure case is a simple example, but in real 
situations there may be more complex data-validation 
requirements, and separating code which \i{checks} 
data from code which \i{uses} data, into different 
procedures, may simplify subsequent code maintenance.
If the \i{validation} code needs to be modified 
\mdash{} and if it is factored into its its own procedure \mdash{}  
this can be done without modifying the 
code which actually works on the data (reducing the 
risk of new coding errors).  In short, factoring 
\i{gatekeeper} and \i{fragile} code into separate 
procedures exemplifies the programming principle of 
\q{separation of concerns}.  On the other hand, 
such separation creates a new problem of ensuring that 
the gatekeeping procedure is always called.  
Meanwhile, using special-purpose, narrowed data types 
adds complexity to the overall software if these data types
are unique to that one code base, and therefore 
incommensurate with data provided by external sources.  
In these situations the software must transform data between 
more generic and more specific representations before 
sharing it (as sender or receiver), which makes 
the code more complicated.  
} 
\p{In this preliminary discussion I refrain from any concrete 
analysis of the coding or type-theoretic models that 
can shed light on these options; I merely want to 
identify the kinds of questions which need to be 
addressed in preparation for a software project, 
particularly in the CyberPhysical domain.  
Ideally, protocols for pairing up fragile and 
gatekeeper code should be consistent through the code base. 
}
\p{In the specific CyberPhysical context, gatekeeping is especially 
important when working with device data.  Such data is 
almost always constrained by the physical construction of 
devices and the kinds of physical quantities they measure 
(if they are sensors) or their physical capabilities 
(if they are \q{actuators}, devices that cause changes in their 
environments).  For sensors, it is an empirical question what 
range of values can be expected from properly functioning 
devices (and therefore what validations can check that the 
device is working as intended).  For actuators, it should 
be similarly understood what range of values guarantee 
safe, correct behavior.  For any device then we can 
construct a \i{profile} \mdash{} an abstract, mathematical 
picture of the space of \q{normal} values associated with 
proper device performance.  Gatekeeping code can 
then ensure that data received from or sent to devices 
fits within the profile.  Defining device profiles, and 
explicitly notating the corresponding gatekeeping code, 
should therefore be an essential pre-implementation planning 
step for CyberPhysical software hubs.  
}
%\spsubsection{Proactive Design}
\subsection{Proactive Design}
\p{I have thus far argued that applications
which process CyberPhysical data need to rigorously organize their functionality
around specific devices' data profiles.  The procedures that directly interact
with devices \mdash{} receiving data from and perhaps sending instructions
to each one \mdash{} will in many instances be \q{fragile} in the sense
I invoke in this chapter.  Each of these procedures may make assumptions
legislated by the relevant device's
specifications, to the extent that using any one procedure too broadly
constitutes a system error.  Furthermore, CyberPhysical devices that are
not full-fledged computers may
exhibit errors due to mechanical malfunction, hostile attacks,
or one-off errors in electrical-computing operations, causing
performance anomalies which look like software mistakes even if the code is
entirely correct (see \cite{MichaelEngel} and
\cite{LavanyaRamapantulu}, for example).  As a
consequence, \i{error classification} is especially
important \mdash{} distinguishing kinds of software errors
and even which problems are software errors to begin with.
}
\p{To cite concrete examples,
a heart-rate sensor generates continuously-sampled integer values
whose understood Dimension of Measurement is in \q{beats per minute}
and whose maximum sensible range (inclusive of both
rest and exercise) corresponds roughly
to the \ftytwoh{} interval.  Meanwhile, an accelerometer
presents data as voltage changes in two or three directional
axes, data which may only produce signals when a change occurs
(and therefore is not continuously varying), and which is
mathematically converted to yield information about physical
objects' (including a person's) movement and incline.  The
pairwise combination of heart-rate and acceleration data
(common in wearable devices) is then a mixture of these
two measurement profiles \mdash{} partly continuous and
partly discrete sampling, with variegated axes and
inter-axial relationships.
}
\p{These data profiles need to be integrated with CyberPhysical code from a
perspective that cuts across multiple dimensions of project scale and
lifetime.  Do we design for biaxial or triaxial accelerometers, or both,
and may this change?  Is heart rate to be sampled in a context where
the range considered normal is based on \q{resting} rate or is it
expanded to factor in subjects who are exercising?  These kinds
of questions point to the multitude of subtle and project-specific
specifications that have to be established when implementing and then
deploying software systems in a domain like Ubiquitous Computing.
It is unreasonable to expect that all relevant standards will be
settled \i{a priori} by sufficiently monolithic and comprehensive
data models.  Instead,
developers and end-users need to acquire trust in a development process
which is ordered to make standardization questions become apparent
and capable of being followed-up in system-wide ways.
}
\p{For instance, the hypothetical questions I pondered in
the last paragraph \mdash{} about biaxial vs.
triaxial accelerometers and about at-rest vs. exercise
heart-rate ranges \mdash{} would not
necessarily be evident to software engineers or project architects when the
system is first conceived.  These are the kind of modeling questions that tend
to emerge as individual procedures and datatypes are
implemented.  For this reason, code development serves a role beyond just
concretizing a system's deployment software.
The code at fine-grained scales also reveals questions that need to be
asked at larger scales, and then the larger answers reflected back in the
fine-grained coding assumptions, plus annotations
and documentation.  The overall
project community needs to recognize software implementation as a crucial
source for insights into the specifications that have to be established
to make the deployed system correct and resilient.
}
\p{For these reasons, code-writing \mdash{} especially at the smallest scales \mdash{}
should proceed via paradigms disposed to maximize
the \q{discovery of questions} effect
(see also, as a case study, \cite[pages 6-10]{Arantes}).  Systems in operation will be
more trustworthy when and insofar as their software bears witness to a project
evolution that has been well-poised to unearth questions
that could otherwise diminish the system's trustworthiness.
Lest this seem like common sense and unworthy of being emphasized
so lengthily, I'd comment that literature on Ubiquitous Sensing for 
Healthcare (\USH{}), for
example, appears to place much greater emphasis on Ontologies or Modeling
Languages whose goal is to predetermine software design at such
detail that the actual code merely enacts a preformulated schema,
rather than incorporate subjects (like type Theory and
Software Language Engineering) whose insights can
help ensure that code development plays a more proactive role.
}
\p{\q{Proactiveness}, like transparency and trustworthiness, has been
identified as a core \USH{} principle, referring (again in
the series intro, as above)
to \q{data transmission to healthcare providers
... \i{to enable necessary interventions}} (my emphasis).  In
other words \mdash{} or so this language implies, as an
unstated axiom \mdash{}
patients need to be confident in deployed \USH{} products
to such degree that they are comfortable with clinical/logistical
procedures \mdash{} the functional design of medical spaces; decisions about
course of treatment \mdash{} being grounded in part on data generated from
a \USH{} ecosystem.  This level of trust, or so I would argue,
is only warranted if patients feel
that the preconceived notions of a \USH{} project have been vetted against
operational reality \mdash{} which can happen through the interplay between
the domain experts who germinally envision a project and the programmers
(software and software-language engineers) who, in the end, produce its
digital substratum.
}
\p{\q{Transparency} in this environment means that \USH{} code needs
to explicitly declare its operational assumptions, on the
zoomed-in procedure-by-procedure scale, and also exhibit its
Quality Assurance strategies, on the zoomed-out system-wide scale.  It
needs to demonstrate, for example, that the code base has sufficiently
strong typing and thorough testing that devices are always matched to
the proper processing and/or management functions: e.g., that there are no
coding errors or version-control mismatches which might cause situations
where functions are assigned to the wrong devices, or the wrong
versions of correct devices.  Furthermore, insofar as most \USH{} data
qualifies as patient-centered information that may be personal and
sensitive, there needs to be well-structured transparency concerning
how sensitive data is allowed to \q{leak} across the system.  Because
functions handling \USH{} devices are inherently fragile,
the overall system needs extensive and openly documented
gatekeeping code that both validates their input/output and controls
access to potentially sensitive patient data.
}
\thindecoline{}
\p{Fragile code is not necessarily a sign of poor design.  Sometimes
implementations can be optimized for special circumstances, and
optimizations are valuable and should be used wherever possible.  Consider an
optimized algorithm that works with two lists that must be the same size.
Such an algorithm should be preferred over a less efficient
one whenever possible \mdash{} which is to say, whenever dealing with two
lists which are indeed the same size.  Suppose this algorithm is
included in an open-source library intended to be shared among many different
projects.  The library's engineer might, quite reasonably, deliberately
choose not to check that the algorithm is invoked on same-sized lists
\mdash{} checks that would complicate the code, and sometimes slow the
algorithm unnecessarily.  It is then the responsibility of code that
\i{calls} whatever procedure implements the algorithm to ensure that it
is being employed correctly \mdash{} specifically, that this
\q{client} code does \i{not} try
to use the algorithm with \i{different-sized} lists.  Here \q{fragility} is
probably well-motivated: accepting that algorithms are sometimes
implemented in fragile code can make the code cleaner, its intentions
clearer, and permits their being optimized for speed.
}
\p{The opposite of fragile code is sometimes called \q{robust} code.
While robustness is desirable in principle, code which simplistically
avoids fragility may be harder to maintain than deliberately fragile but
carefully documented code.  Robust code often has to check for many
conditions to ensure that it is being used properly, which can make
the code harder to maintain and understand.  The hypothetical
algorithm that I contemplated last paragraph
could be made robust by \i{checking}
(rather than just \i{assuming}) that it is invoked with same-sized lists.
But if it has other requirements \mdash{} that the lists are non-empty,
and so forth \mdash{} the implementation can get padded with a chain of
preliminary \q{gatekeeper} code.  In such cases the gatekeeper
code may be better factored into a different procedure, or expressed
as a specification which engineers must study before attempting to
use the implementation itself.
}
\p{Such transparent declaration of coding assumptions and specifications can
inspire developers using the code to proceed attentively,
which can be safer in the long run than trying to avoid fragile code
through engineering alone.  The takeaway is that while \q{robust} is
contrasted with \q{fragile} at the smallest scales (such as
a single procedure), the overall goal is systems and components that are robust at the
largest scale \mdash{} which often means accepting \i{locally} fragile
code.  Architecturally, the ideal design may combine
individual, \i{locally fragile} units with rigorous documentation and gatekeeping.
So defining and declaring specifications is
an intrinsic part of implementing code bases which are both robust
and maintainable.
}
\p{Unfortunately, specifications are often created
only as human-readable documents, which might have a semi-formal
structure but are not actually machine-readable.
There is then a disconnect between features \i{in the code itself} that
promote robustness, and specifications intended for \i{human} readers
\mdash{} developers and engineers.  The code-level and
human-level features promoting robustness will tend to overlap partially
but not completely, demanding a complex evaluation of where gatekeeping
code is needed and how to double-check via
unit tests and other post-implementation examinations.  This is the
kind of situation \mdash{} an impasse, or partial but incomplete overlap,
between formal and semi-formal specifications \mdash{} which many programmers
hope to avoid via strong type systems.
}
\p{Most programming language will provide some basic (typically relatively
coarse-grained) specification semantics, usually
through type systems and straightforward code observations
(like compiler warnings about unused or uninitialized variables).
For sake of discussion, assume that all languages have distinct
compile-time and run-time stages (though these may be opaque to
the codewriter).  We can therefore distinguish compile-time
tests/errors from run-time tests and errors/exceptions.
Via Software Language Engineering, we can study
questions like: how
should code requirements be expressed?  How and to
what extent should requirements be tested by the language
engine itself \mdash{} and beyond that how can the language help coders implement
more sophisticated gatekeepers than the language natively offers?
What checks can and should be compile-time or run-time?  How
does \q{gatekeeping} integrate with the overall semantics and
syntax of a language?
}
\p{Given the maxim that
procedures should have single and narrow roles \mdash{} \q{separation 
of concerns} \mdash{} note that \i{validating} input
is actually a different role than \i{doing} calculations.  This is 
why procedures with fine requirements might be split into two: a
gatekeeper that validates input before a fragile procedure is called,
separate and apart from that procedure's own implementation.
A related idea is overloading fragile procedures: for example,
a function which takes one value can be overloaded in terms
of whether the value fits in some prespecified range.  These two
can be combined: gatekeepers can test inputs and call one of several
overloaded functions, based on which overload's specifications are
satisfied by the input.
}
\p{But despite their potential elegance, mainstream programming languages
do not supply much language-level support for expressing
groups of fine-grained functions along these lines.  Advanced 
type-theoretic constructs \mdash{} including Dependent Types,
typestate, and effect-systems \mdash{} model requirements with more precision
than can be achieved via conventional type systems alone.  Integrating these
paradigms into core-language type systems permits data validation 
to be integrated with general-purpose type checking, without the need for
static analyzers or other \q{third party} tools (that is, projects maintained
orthogonally to the actual language engineering; i.e., to
compiler and runtime implementations).  Unfortunately, 
these advanced type systems are also more complex to implement.  
If software language engineers aspire to make Dependent Types and 
similar advanced constructs part of their core language, 
creating compilers and runtime engines for these languages 
becomes proportionately more difficult.
} 
\p{If these observations are correct, I maintain that it is a worthwhile
endeavor to return to the theoretical drawing board, with the goal 
of improving programming language technology itself.  
Programming languages are, at one level, artificial 
\i{languages} \mdash{} they allow humans to communicate 
algorithms and procedures to computer processors, and 
to one another.  But programming languages are also 
themselves engineering artifacts.  It is a complex
project to transform textual source-code \mdash{} which is 
human-readable and looks a little bit like natural 
language \mdash{} into binary instructions that computers 
can execute.  For each language, there is a stack 
of tools \mdash{} parsers, compilers, and/or runtime libraries 
\mdash{} which enable source code to be executed 
according to the language specifications.  
Language design is therefore constrained by 
what is technically feasible for these supporting tools.  
Practical language design, then, is an interdisciplinary 
process which needs to consider both the dimension of 
programming languages as communicative media and 
as digital artifacts with their own engineering challenges 
and limitations.
}
%\spsubsection{Core Language vs. External Tools}
\subsection{Core Language vs. External Tools}
\p{Because of programming languages' engineering limitations, 
such as I just outlined, software projects should not 
necessarily rely on core-language features for 
responsible, safety-conscious programming.
Academic and experimental languages tend to have 
more advanced features, and to embody more 
cutting-edge language engineering, compared to mainstream 
programming languages.  However, it is not always feasible 
or desirable to implement important software with 
experimental, non-mainstream languages.  By their nature, 
such projects tend to produce code that must be understood by 
many different developers and must remain usable years into 
the future.  These requirements point toward 
well-established, mainstream languages \mdash{} and 
mainstream development techniques overall \mdash{} as opposed to 
unfamiliar and experimental methodologies, even if those 
methodologies have potential for safer, more productive 
coding in the future.   
}
\p{In short, methodologies for safety-conscious coding can be 
split between those which depend on core-language features, 
and those which rely on external, retroactive analysis 
of sensitive code.  On the one hand, some languages and projects
prioritize specifications that are intrinsic to the language and integrate
seamlessly and operationally into the language's foundational
compile-and-run sequence.  Improper code (relative to specifications)
should not compile, or, as a last resort, should fail gracefully at run-time.
Moreover, in terms of programmers' thought processes, the
description of specifications should be intellectually continuous
with other cognitive processes involved in composing code, such
as designing types or implementing algorithms.  For sake of 
discussion, I will call this paradigm \q{internalism}.  
}
\p{The \q{internalist} mindset seeks to integrate data 
validation seamlessly with other language features.  
Malformed data should be flagged via similar mechanisms 
as code which fails to type-check; and errors should 
be detected as early in the development process as possible.   
Such a mindset is evident in passages like this (describing
the Ivory programming language):
\begin{dquote}Ivory's type system is shallowly embedded within Haskell's
type system, taking advantage of the extensions provided by [the
Glasgow Haskell Compiler].  Thus, well-typed Ivory programs
are guaranteed to produce memory safe executables, \i{all without
writing a stand-alone type-checker} [my emphasis].  In contrast, 
the Ivory syntax is \i{deeply} embedded within Haskell.
This novel combination of shallowly-embedded types and 
deeply-embedded syntax permits ease of development without sacrificing
the ability to develop various back-ends and verification tools [such as]  
a theorem-prover back-end.  All these back-ends share the
same AST [Abstract Syntax Tree]: Ivory verifies what it compiles.
\cite[p. 1]{ivory}.
\end{dquote}   In other words, the creators of Ivory are promoting the
fact that their language buttresses via its type system 
\mdash{} and via a mathematical precision suitable for 
proof engines \mdash{} 
code guarantees that for most languages require external
analysis tools.
}
\p{Contrary to this \q{internalist} philosophy, other approaches
(perhaps I can call them \q{externalist}) favor a neater separation
of specification, declaration and testing from the core language,
and from basic-level coding activity.  In particular \mdash{} according to 
the \q{externalist} mind-set \mdash{} most of the more important or complex
safety-checking does not natively integrate with the
underlying language, but instead requires
either an external source code analyzer, or 
regulatory runtime libraries, or some combination of the two.  
Moreover, it is unrealistic
to expect all programming errors to be avoided with enough proactive planning,
expressive typing, and safety-focused paradigms: any complex
code base requires some retroactive design, some combination
of unit-testing and mechanisms (including those
third-party to both the language and the projects whose code is
implemented in the language) for externally
analyzing, observing, and higher-scale testing for the code,
plus post-deployment monitoring.
}
\p{As a counterpoint to the features cited as benefits to the
Ivory language, which I identified as representing the 
\q{internalist} paradigm, consider Santanu Paul's Source Code Algebra (\SCA{})
system described in \cite{SantanuPaul} and
\cite{GiladMishne}, \cite{TillyEtAl}:
\begin{dquote}Source code files are processed using
tools such as parsers, static analyzers, etc. and the necessary information
(according to the SCA data model) is stored in a repository.  A user interacts
with the system, in principle, through a variety of high-level languages, or
by specifying SCA expressions directly.  Queries are mapped to SCA expressions,
the SCA optimizer tries to simplify the expressions, and finally, the SCA
evaluator evaluates the expression and returns the results to the user.\nl{}
We expect that many source code queries will be expressed using high-level
query languages or invoked through graphical user interfaces.  High-level queries
in the appropriate form (e.g., graphical, command-line, relational, or
pattern-based) will be translated into equivalent SCA expressions.  An SCA
expression can then be evaluated using a standard SCA evaluator, which
will serve as a common query processing engine.  The analogy from
relational database systems is the translation of SQL to expressions based on
relational algebra. \cite[p. 15]{SantanuPaul}
\end{dquote}
So the \i{algebraic} representation of source code is favored
here because it makes computer code available
as a data structure that can be processed via \i{external}
technologies, like \q{high-level languages}, query languages, and
graphical tools.  The vision of an optimal development environment
guiding this kind of project is opposite, or at least
complementary, to a project like Ivory: the whole point
of Source Code Algebra is to pull code verification \mdash{} the
analysis of code to build trust in its safety and robustness
\mdash{} \i{outside} the language itself and into the surrounding
Development Environment ecosystem.
}
\p{These philosophical differences (what I dub \q{internalist} vs. \q{externalist}) 
are normative as well as descriptive:
they influence programming language design, and how languages in turn influence
coding practices.  One goal of language design is to produce languages 
which offer rigorous guarantees \mdash{} fine-tuning the languages' 
type system and compilation model to maximize the level of detail 
guaranteed for any code which type-checks and compiles.  
Another goal of language design is to define syntax and 
semantics permitting valid source code to be analyzed 
as a data structure in its own right.  Ideally, 
languages can aspire to both goals.  In practice, however, 
achieving both equally can be technically difficult.  
The internal representations conducive to strong type and 
compiler guarantees are not necessarily amenable to 
convenient source-level analysis, and vice-versa.    
}
\p{Language engineers, then, have to work with
two rather different constituencies.  One community of
programmers tends to prefer that specification and validation be
integral to/integrated with the language's type system and
compile-run cycle (and standard runtime environment); whereas
a different community prefers to treat code evaluation
as a distinct part of the development process, something logically, operationally,
and cognitively separate from hand-to-screen codewriting
(and may chafe at languages restricting certain code constructs
because they can theoretically produce coding errors, even when
the anomalies involved are trivial enough to be tractable for
even barely adequate code review).  One challenge for language engineers is
accordingly to serve both communities.  We can, for example, aspire to
implement type systems which are sufficiently
expressive to model many specification, validation, and
gatekeeping scenarios, while also anticipating that language code
should be syntactically and semantic designed to be
useful in the context of external tools (like
static analyzers) and models (like Source Code
Algebras and Source Code Ontologies).
}
\p{The techniques I discuss here work toward these goals on two levels.  First, I
propose a general-purpose representation of computer code in terms
of Directed Hypergraphs, sufficiently rigorous to codify a
theory of functional types as types whose values are (potentially) initialized from
formal representations of source code \mdash{} which is to say, in the present
context, code graphs.  Next, I
analyze different kinds of \q{lambda abstraction} \mdash{} the idea of
converting closed expressions to open-ended formulae by asserting that
some symbols are \q{input parameters} rather than fixed values, as in
Lambda Calculus \mdash{} from the perspective of
axioms regulating
how inputs and outputs may be passed to and obtained from
computational procedures.  I bridge these topics \mdash{} Hypergraphs
and Generalized Lambda Calculi \mdash{} by taking abstraction as a
feature of code graphs wherein some hypernodes are singled out
as procedural
\q{inputs} or \q{outputs}.  The basic form of this model
\mdash{} combining what are essentially two otherwise unrelated
mathematical formations, Directed Hypergraphs and
(typed) Lambda Calculus \mdash{} is laid out in
Sections \sectsym{}\hyperref[sTwo]{\ref{sTwo}}
and \sectsym{}\hyperref[sThree]{\ref{sThree}}.
}
\p{Following that sketch-out, I engage a more rigorous study of
code-graph hypernodes as \q{carriers} of runtime values, some of
which collectively form \q{channels} concerning values which
vary at runtime between different executions of a function body.
Carriers and channels piece together to form
\q{Channel Groups} that describe structures with meaning both
within source code as an organized system (at \q{compile time}
and during static code analysis) and at runtime.  Channel Groups
have four different semantic interpretations, varying via the
distinctions between runtime and compile-time and between
\i{expressions} and (function) \i{signatures}.
I use the framework of Channel Groups to identify
design patterns that achieve many goals of
\q{expressive} type systems while being implementationally
feasible given the constraints of mainstream programming
languages and compilers (with an emphasis on \Cpp{}).
}
\p{After this mostly theoretical prelude, I conclude this
chapter with a discussion of code annotation, particularly
in the context of CyberPhysical Systems.  Because CyberPhysical applications
directly manage physical devices, it is especially important that they be
vetted to ensure that they do not convey erroneous instructions
to devices, do not fail in ways that leave devices uncontrolled, and
do not incorrectly process the data obtained from devices.
Moreover, CyberPhysical devices are intrinsically \i{networked},
enlarging the \q{surface area} for vulnerability, and often worn by people
or used in a domestic setting, so they tend carry personal (e.g., location)
information, making network security protocols especially important
(\cite{RonaldAshri}, \cite{LalanaKagal}, \cite{AbhishekDwivedi},
\cite{TakeshiTakahashi}, \cite{BhavaniThuraisingham},
\cite{MozhganTavakolifard}).  The dangers
of coding errors and software vulnerabilities, in CyberPhysical
Systems like the Internet of Things (\IoT{}), are even more pronounced
than in other application domains.  While it is
unfortunate if a software crash causes someone to lose data,
for example, it is even more serious if a CyberPhysical \q{dashboard}
application were to malfunction and leave physical, networked
devices in a dangerous state.
}
\p{To put it differently, computer code which directly interacts with
CyberPhysical Systems will typically have many fragile pieces, which
means that applications providing user portals to maintain and control
CyberPhysical Systems need a lot of gatekeeping code.  Consequently,
code verification is an important part of preparing CyberPhysical Systems
for deployment.  The \q{Channelized Hypergraph} framework I develop here
can be practically expressed in terms of code annotations that benefit
code-validation pipelines.  This use case is shown in demo code
published as a data set alongside this chapter (available for
download at \url{https://github.com/scignscape/ntxh}).
These techniques are not designed to substitute for Test Suites or
Test-Driven Development,
though they can help to clarify the breadth of coverage of
a test suite \mdash{} in other
words, to justify claims about tests being thorough enough that
the code base passing all tests actually does argue for the code
being safe and reliable.  Nor are code annotations intended to
automatically verify that code is safe or
standards-compliant, or to substitute for
more purely mathematical code analysis using proof-assistants.
But the constructions presented here,
I claim, can be used as part of a
code-review process that will enhance stakeholders' trust
in safety-critical computer code, in cost-effective, practically
effective ways.
}

\section{Types' Internal Structure and \lrgNCFour{} Type Theory}
\p{When data modeling is conducted in a \q{software-centric} 
milieu, notions of \i{type systems} and \i{typed values} 
become especially important.  In the context of 
text-based serialization languages, like \XML{}, 
data structures can be assembled without mandating conformance 
to predetermined schema.  Assuming there is some core 
set of \q{basic} types \mdash{} such as integers, floating-point 
numbers, and Unicode character strings (to represent 
names and phrases in natural languages) \mdash{} data aggregates 
can be assembled in a \q{semi-structured} manner, with 
different types and groupings juxtaposed in no particular 
order.  This indeterminacy is possible because 
each part of a data structure is embodied via a textual 
encoding, and text-streams can be packaged in a relatively 
free-form manner, with no \i{a priori} length or structure. 
}
\p{Computer software, by contrast, works with binary 
resources \mdash{} i.e., with \i{binary} (rather than textual) 
encodings of data structures; in the general case 
we can see binary encodings as strings of 8-bit 
integers (i.e., strings of bytes, with values in 
the range \lclc{0-255}).  Binary resources have to 
registered in fixed-size, pre-allocated segments of 
computer memory.  It is possible to emulate more free-form 
text-encoded structures via software memory, but this requires 
more complex implementations; it is not the underlying 
representational pattern which is canonical to binary 
data.  As a result, software-centric data models should 
be grounded on structures consistent with the constraints 
imposed by binary representation at the base level, 
and then generalize to less restrictive aggregates via 
higher-level, multidimensional representations.  
}
\p{Whereas the building blocks of data structures in a context 
like \XML{} are therefore semi-structured data complexes, 
the analogous foundation for software-centric data models 
are \i{typed values}, or binary resources associated with 
a type \ty{}, which in turn belongs to a \i{type system} 
\TyS{}.  Data models in this sense are closely tied 
to the details of the relevant type systems: 
in a given \TyS{}, what constitutes a type in the first 
place, how types are combined into aggregates, and so forth.   
}
\p{This section will outline a model of type systems which I 
believe is conducive to an overall project of unifying 
Hypergraph Category based grammar with Conceptual Space 
semantics.  I will build off of work I have published 
elsewhere, so some details will be skipped over (see 
\cite{MeHBTT} for more extensive treatment of the 
underyling theory).   
}
\subsectiontwoline{Cocyclic Types, Precyclic and Endocyclic Tuples}
\p{In a hypergraph-based modeling environment, \i{hyperedges} 
may span three or more nodes (ordinary graph edge connect
exactly two nodes).  For \i{directed} hypergraphs (\DH{}s), 
hyperedges have a \i{head set} and a \i{tail set}, each 
collections of one or more nodes.  The term \i{hypernode} 
can be used to designate node-sets which are either the 
head or tail of a directed hyperedge; to avoid confusion 
the nodes inside a hypernode can then be called \i{hyponodes}.  
Directed hypgraphs which are \i{labeled} (generalizing 
ordinary labeled graphs, which are the basis 
of Semantic Web data, such as \RDF{}) allow information to be associated 
with connections between hypernodes.  Each labeled hyperedge 
then asserts that a certain kind of relationship exists between 
the entitites or sets of entities grouped on either side 
of the hyperedge (head or tail).\footnote{The relation is assumed to be intransitive, in the head-to-tail 
direction, thereby generalizing \q{Subject/Predicate/Object} triples 
in \RDF{}
} 
}
\p{Labeled \DH{}s thereby have two different formations 
for aggregating information: first, how hyponodes are grouped 
into hypernodes; and, second, how hypernodes are interrelated 
via labele connections (hyperedges).  This duality allows 
hypegraphs to combine the paradigms associated with 
ordinary labeled graphs and with data tuples or \q{records} 
(e.g., Relational Databases).  So \DH{}s evince a step toward 
a universal, expressive framework for data representation 
which is structurally rigorous but not tied down to 
simplified modeling paradigms.    
}
\p{Analysis of hypergraph models can bifurcate into two branches, then, 
depending on whether we attend to the formation of hypernodes 
from hypernodes or to the assertion of inter-hypernode connections, 
via hyperedges.  I will focus on the first alternative.
}
\subsubsection{Cocyclic Types for Hypernodes}
\p{I assume we operate in a context where a type system 
\TyS{} is employed in conjunction with hypegraphs, so 
both hypo- and hyper-nodes receive type attributions.  
We can then consider what sorts of types should be 
representable in \TyS{} to adequately model the 
spectrum of hypernodes which may appear in a hypergraph.  
Without undue loss of generality, we can assume that 
nonidentical hypernodes do not overlap (i.e., 
no hyponode is covered by more than one hypernode).\footnote{This restriction \mdash{} which I call \q{disjointness} \mdash{} 
can actually be weakened somewhat; see 
\cite[p. ?]{MeHBTT} for details.
}  Any given graph, then, seen as a static 
data structure, will have some fixed list of 
hyponodes for each hypernode (since directed hyperedges 
are ordered, we can assume that there is an ordering on 
their head and tail sets, and therefore that hyponodes 
have a fixed order in their covering hypernodes).\footnote{Assume that hypernode identity is affected by hyponode order; 
so the same set of hyponodes cannot appear as a head-set or 
tail-set in two different edgers where their order 
would be permutated, since that would violate disjointness. 
}  
}
\p{In the models I propose, however, I want to focus on \q{Procedural} Hypergraphs, 
which are not necessarily static structues.  Instead, each hypergraph 
has certain evolutionary possibilities, i.e., certain regulated 
operations by which it can be modified, such as (potentially) 
adding a hyponode to a hypernode (if that is compatible with 
the hypernode's type).  We want, then, a more flexible type mechanism 
whereby hypernodes can cover a varying number of hyponodes.  
On the other hand, we also want these hyponodes to have types 
according to some fixed pattern, to preserve a usable 
type-attribution mechanism for hypernodes.  In other words, 
if we allow hypernodes to cover an unconstrained list of 
hyponodes with any type, there ceases to be any means of 
sorting hypernodes into different types.  The problem 
is then to free up type \q{tupling} as far as possible while 
preserving a strong type system at the hypernode level.
}
\p{The concept of \q{cocyclic} types, then, is intended to convey a 
pattern among hyponode types which is flexible but still 
constrained by strong typing.  Let \tColl{} be any ordered 
sequence of types in a \TyS{}.  I will say that \tColl{} is 
\i{cyclic} if the sequence repeats: every \nth{} type 
is the same, for some \nNum{}.  I will call a \tColl{} 
\i{cocyclic} if it comprises a cyclic sequence preceded 
by a fixed-width tuple of types.  A cocyclic \i{type} is 
then a product-type in \TyS{} whose instances are 
hypernodes wherein their contained hyponodes have types which, 
listed as a sequence of \TyS{} types, comprise a 
cocyclic sequence.  The fixed-width tuple at the start of 
the hyponode-list I will call the \i{precyclic} part of 
the hypernode, while the type-tuple that repeats over the 
rest of the sequence I will call the \i{endocyclic} part.   
}
\p{This definition of cocyclic types can be extended outside the 
context of hypergraph type-attributions by considering 
\q{data fields} or other components of product-types in lieu of 
hyponodes.  In practice, the Cocyclic model is implemented 
via the \q{PhaonGraph} library included with the dataset accompanying 
this paper.  Note that any fixed-length product type 
(whose instances are fixed-length tuples of values, or, 
in the hypergraph context, hyponodes) is a cocyclic type 
with no encyclic part.  Likewise, a \q{list} or \q{collections} 
type built on a single \TyS{} type \mdash{} a list, stack, queue, 
or deque of \ty{}s (meaning a list of \ty{}s which grows 
and/or shrinks from one or 
another end, or both) \mdash{} is a cocyclic type with no precyclic 
part (although an implementation might model these instead with 
precyclic field tracking data such as the current length of the list).  
An \q{associative array} (a.k.a. \q{dictionary}) using one type to index a second 
\mdash{} where the \i{keys} to the dictionary come from a \tyOne{} and 
the values from a \tyTwo{} \mdash{} is similarly (representable as) an 
endocycle alternating between \tyOne{} and \tyTwo{}.  
}
\p{The purpose of 
this framework is generality \mdash{} similar data structures can be 
emulated in a type system where tuples have to have fixed 
lengths, or where varying-length tuples have to contain 
only one single types: in these cases (what I call) \q{proxies} 
(hyponodes uniquely designating hypernodes they are not part of) 
can approximate the layout of cocyclic types, by analogy to 
programming languages using pointers or nested tuples to 
represent dynamically-sized collections types.  However, 
the cocyclic type paradigm yields less indirection 
\mdash{} less gap between the conceptual pattern represented 
by a data model and its software implementation \mdash{} 
without diminishing the model's computational realizability.  
The \q{PhaonGraph} library represents one implementational 
strategy for modeling hypernodes via cocyclic 
types as an underlying data representation.\footnote{In a nutshell, PhaonGraph, written in \Cpp{}, allocates 
memory in fixed-sized chunks corresponding to some 
fixed array of hyponodes, and then maintains a separate 
list of pointers to the chunks thus far allocated in order.
A related \q{PhaonLib} library uses similar techniques to 
provide collections values such as stacks, deques, and 
queues, built around a uniform list-interface providing 
common iterator and visitor functionality. 
}  Of course, any individual \TyS{} type 
(in the case of a hypernode with one single hyponode) can 
be seen as a cocyclic type with a length-one precycle.   
}
\p{Against this background, then, I assume that for any 
\TyS{} with fixed-length types we can generalize to a 
related system wherein all types are cocylic.  
From here on then I assume that any \TyS{} under 
discussion is \q{cocyclic}, meaning each 
\ty{} in \TyS{} is cocyclic. 
}
\subsection{Channelized Types and Channel Algebra}
\p{In any reasonably advanced type system, some types in 
\TyS{} are \q{function-like}: they represent computations 
which, in some sense, take \i{inputs} of some type or 
types in \TyS{} and produce \i{outputs}.  Programmers 
sometimes talk of \q{pure functions} as computations 
that map inputs to outputs with no side effects.  
In most programming languages, however, the description of 
function-like types has to be more complex.  In particular, 
languages can have variant sorts of input \mdash{} in 
Object Oriented programming, for example, 
some (or in some languages all) functions 
(called \q{methods}) have a special Object or \q{\this{}} 
input which, in various technical ways, is treated differently 
than the method's other input parameters.  Likewise, 
procedures have modes of \q{output} other than 
returning values \mdash{} they can throw exceptions,  
modify input values, or have other side-effects in 
addition to (or in place of) returning values.  
Procedures implemented in most programming 
languages, in short, have multiple mechanisms 
for \q{communicating with the outside world} 
\mdash{} for getting data with which to 
complete their given task, and for sharing 
the results of their operations, or otherwise 
effectuating some change beyond just computing a 
result.  These alternatives generally get some 
representation in languages' type systems.  
For example, in \Cpp{}, a method (which 
takes an object as a special \this{} value) 
has a different type than a function where 
that same object would be passed as a normal argument.     
}
\p{We therefore should assume that \TyS{} allows us, in principle, 
to differentiate function-like types on the basis of 
multiple \i{kinds} of input and output, and/or side-effects.  
It is not sufficient to represent \TyS{} as permitting, say, 
given a single input \tyOne{} (or list of input types) and 
output \tyTwo{} (or again a list of output types), 
the identificatio of a type \tyOneTotyTwo{} representing 
functions from \tyOne{} to \tyTwo{}.  Instead, \TyS{} has 
to model multiple input and output modes.  This variation 
does not necessarily yield distinct types 
\mdash{} for instance, in \Cpp{}, whether or not functions 
throw exceptions is not normally considered part of 
their type (you can assign the address of a functon which 
\throw{}s to a function-pointer whose declared type makes 
no mention of exceptions).  However, differences in 
input/output options \i{could} potentially herald 
differences in type attributions (e.g., a type system 
\i{could} stipulate that procedures which do not 
throw exceptions may never be deemed an instance of 
the same type as a procedure which \i{does} \throw{}).  
}
\p{In \cite{MeHBTT} I introduced the idea of \q{Channel Algebra} to 
model different forms of input and output (an alternative outline 
of this framework can be found in \cite{MePGVM}, which is 
distributed alongside this paper).  In essence, each channel is a 
separate mode of input and output, and procedures are assigned 
types based on grouping together one or more channels.  
I say that a type system is \i{Channelized} if function-like types 
in \TyS{} can be described by describing the \q{channel complex}, 
or sums of channels group together, specifying the 
kinds of inputs and outputs a given procedure recognizes 
(along with the types of values passed in to or returned 
from each procedure).  I use the term \i{carrier} to designate 
the resource (e.g. a source-code token, or a hypernode in a graph 
used to model computer code) holding a value in the 
context of a procedure.  Channels are then sequences of one or more 
(or potentially zero or more) carriers.  Carriers are distinct from values 
(and from types) because they have states separate and 
apart from the values they hold: when a procedure throws an 
exception rather than return a value, for instance, the 
carrier(s) in that procedures \q{\returnch{}} channel 
will hold no value, which on this theory is a valid carrier-state.     
}
\p{In \cite{MePGVM} I also discuss channels and carriers, but from a 
more graph-oriented perspective.  There I propose channels as 
extra structures defined on hypergraphs, grouping together 
multiple hyperedges as aggregate units.  The two models 
fit together insofar as the definition of Channelized 
Type Systems is one application of the construction of 
channels on hypergraps, where procedures' types are 
established via \q{code-graphs} exemplifying the procedure's 
signature (channel complexes) and calls \i{to} the 
procedure (which I call channel \i{packages}).  
Hypergraphs conveying the form of channel complexes and 
packages thereby apply the general notion of hypergraph 
channels to the specific project of modeling 
procedure output and input kinds via code-graphs. 
}
\p{The rationale for introducing channels into hypergraphs, 
\visavis{} models of procedures expressed in computer 
code, is evident similar to the hypegraph formations defined 
in recent work on Hypergraph Categories (providing the 
mathematical background to the syntactic side of 
the \q{\HCS{}} synthesis).  Making connections between 
formalisms developed in a mathematical (in this case, 
Category-Theoretic) context, and those based more 
on applied computer programming, is not fully rigorous 
\mdash{} to some degree I would be interpreting or 
hypothesizing about the practical intent, or 
possible applications, behind the specific 
Hypergraph formulations chosen by the researchers 
behind the \HCS{} idea.  With that caveat, though, 
it certainly \i{seems} as if those mathematicians 
generalize from graphs to hypergraphs to capture 
some of the same procedural generalizations as 
motivated my \q{Channel Algebra} proposal.   
}
\p{More exactly, the Hypergraph model specifically laid out 
for \HCS{} embodies (what we can call) procedures 
\mdash{} the Category-Theoretic setting prefers to talk 
of \q{morphisms} instead\footnote{For technical reasons 
why I prefer different terminology \mdash{} and to 
reserve \i{morphism} for a limited space of 
intertype functions \mdash{} see \cite[p. ?]{MeHBTT}} \mdash{} 
in hypernodes, and their inputs and outputs in 
edges incident to those nodes.  An unadorned 
input-versus-output distinction does not actually 
mandate \i{hypergraph} treatment: ordinary directed 
graphs have the structure to distinguish edges 
coming \i{in} to a node from those going \i{out} of 
a node.  In my \q{channel} theory, I work 
within a hypergraph framework because we need a more 
fine-grained edge-classification than just input and 
output; as I outlined above, there are multiple 
\i{kinds} of input and output channels.  Channels 
therefore group edges together in a fashion that 
introduces extra structural components outside the 
definition of ordinary labeled graphs.     
}
\p{In the case of \HCS{} Hypergraph categories, the motivation 
for adopting hypergraphs is oriented more toward the 
idea that inter-procedural connections represent 
\q{information flows}.  In this framework, computer 
software can be thought of as an interconnected system 
whose architecture can be summarized by graphs: with 
procedures as nodes, an edge exists between procedures 
when the output of one procedure becomes an input for a 
second.  Moreover, the theory uses nodes \i{also} to 
represent information \q{entering} the system \mdash{} in 
effect, data being presented to a procedure which does 
not arise as the result of some other procedure, but 
rather as information obtained via some measurement 
or observation of external states.
}
\p{Moreover, nodes also represent side-effects which can be 
effectuated by a software system.\footnote{Hypergraph Categories are not specifically about software, 
but reasoning about software behavior is cited as a possible 
application and used as a hypothetical case-study.
}  Suppose a procedure concludes by formulating 
an instruction that a rectangle of a given size and color 
is to be drawn on a computer screen.  The values describing 
that desired effect are understood to be \q{outputs} of 
the procedure, but instead of their being passed to a further 
procedure they are somehow translated to tangible effects in 
the software's external environment (in practice, this would 
presumably happen by calling some system kernel function, but 
in the abstract sense we can treat this as an output 
which is \q{absorbed} by the computer rather than passed on 
as an input).    
}
\p{Taking these two ideas together \mdash{} \i{inputs} which are 
measurements of external states and \i{outputs} which are 
effects \i{on} external states \mdash{} in the corresponding 
graph representations we then have directed edges 
which have target nodes but no source nodes (for state-inputs) 
or which have source nodes but no target nodes 
(for effect-outputs).  Authors like Cocke \i{et. al.} 
then use hypergraphs to model these pattern by 
leveraging a generalization wherein Directed Hypergraphs' 
cardinality, for either head or tail, can be any quantity 
(including zero).  That is, Hypegraphs in this 
Category-Theoretic contexts allow for hyperedges with no 
source or no target, as well as hyperedges with multiple 
sources and/or targets.  
}
\p{The state/effect systems thereby 
represented (by head- or tail-empty hyperedges) have a 
correspinding construction, in practical software, via 
techniques generally described as \q{reactive}, as in  
\q{Functional Reactive Programming}.  In this paper I will 
point particularly to the so-called \i{signal/slot} 
mechanism used in \Cpp{} within the \Qt{} libraries.  
A head-empty \q{state} edge, on this analogy, 
corresponds to a \i{signal} which triggers a 
\q{slot} procedure; and a tail-empty 
\q{effect} edge corresponds to an operation of 
\q{emitting} a signal.  I will discuss this analogy 
in more detail below.
}
\p{There are various routes toward generalizing graphs to hypergraphs; 
Hypergraph Categories are only one example.  Also, technical 
presentations of hypergraphs are not exclusively mathematical: 
certain software libraries and graph databases also embody 
formal hypergraph models, with varying features 
(for instance, some allow hyponodes to also be 
hypernodes; some support undirected hyperedges; 
some allow circular hyperedges wherein the head set is 
also the tail set).  The \q{Channelized Hypegraph} 
(\CH{}) setup I outlined in \cite{MeHBTT} is different 
in some details than Hypergraph Categories \mdash{} including 
by introducing channels as an extra construction on 
graphs \mdash{} but I believe the frameworks are sufficiently 
close that the Channelized Hypergraph constructions 
(and therefore Channelized Types) are a plausible 
extension of Hypergraph Categories from the viewpoint of 
integration with Conceptual Space Theory.  
}
\subsection{Constructors and Carrier States}
\p{The \HCS{} version of Hypergraph Category Theory makes the 
trenchant point that graphs modeling values only as they pass 
between procedures \mdash{} outputs becoming inputs \mdash{} are 
necessarily incomplete, because values have to original 
from \i{somewhere}.  In practice, some data handled by a 
software component comes from external sources \mdash{} files, 
packets sent over a network, CyberPhysical devices, and 
so forth.  Otherwise, often values come from computer code directly: 
all programming languages have some notion of \i{source code literals} 
which permit values (at least, those of the most basic types) to 
be initialized from character strings in source code.  
For example, the literal token \q{\codeTextr{99}} becomes 
the \i{number} \codeTextr{99}.  One question to be 
addressed for an applied type theory \mdash{} i.e., to 
specify the nature of an applied type system \TyS{} \mdash{} 
is how and which types can be constructed from source 
code literals in this manner.  
}
\p{Representing values passed between procedures can be seen as 
a \q{syntactic} gloss on computer code, because programming language 
grammars are built around how expressions in each language describe 
the seqeuence of function-calls specified to enact some algorithm 
or calculation.  But understanding how different typed values 
interact with function-calls is also a \i{semantic} matter, 
characterizing the semantics of individual types.  Given a 
specific instance \vVar{} of type \ty{}, we may analyze \mdash{} what 
sort of procedures can produce \vVar{}?  Is \vVar{} obtained from 
some other \ty{}-value, or can it be initialized via a source-code literal?  
Is \vVar{} a \q{default} value for \ty{} which can be created \i{ab initio}, 
with no further input?  If \vVar{} is derived by modiying some alternative 
\ty{}-value, can we identify this prior value, thereby \q{deconstructing} 
the construction which yielded \ty{}?   
}
\p{Suppose, for instance, that \ty{} is a list of signed 32-bit integers.  The 
\q{default value} for such a type is almost always defined as an empty 
list.  New \ty{}-values are created by appending a number to the end 
of the list.  Given a non-empty list \vVar{}, we can always \q{deconstruct} 
\vVar{} by noting that \vVar{} is derived by adding some number 
\nNum{} to a shorter-by-one list \vVarPrime{}.  Algorithms which 
depend on examing the whole list \mdash{} finding its largest element, or 
counting how many times a given number appears, or how many elements 
are larger than some target \mdash{} can be conveniently expressed 
by examining the list recursively, each time stripping the 
last number and repeating the test on the shorter-by-one outcome.  
To count how many numers are positive, say, check each \nNum{} at 
the end of the list, increase the result by one if \nNumGtZero{}, 
and repeat that process with the smaller-by-one list obtained 
by \q{deconstructing} the current \vVar{} into \vVarPrime{} and 
\nNum{}.  The style of recursive algorithm is endemic to Functional 
Programming, where it yields procedures that do not need to 
employ \i{iterators} which loop over the elements of a data 
collection.  Recursive procedures can eliminate the loop initializations 
and tests that can make non-recursive, iterator-based code more cluttered 
and obscure.   
}
\p{However, this recursive style of programming is only possible if 
specific metadata is embedded with \ty{} values which allows 
\q{deconstructing} \ty{} instances into construction \i{patterns} 
and allows \ty{}s to be reused in a recursive fashion (e.g., 
repeatedly calling a procedure on shorter-by-one lists).  
This functionality is not available for simpler in-memory 
representations of data structures, like \q{linked lists} 
(a sequence of pointers to each value paired with pointers to the next 
value) or \CLang{}-style zero-terminated arrays.  In order for \ty{} 
to support recursive algorithms in lieu of iterators, it needs to be 
expressly implemented in anticipation of this pattern.  
Conversely, types also need extra functionality 
(e.g. \mbegin{} and \mend{} methods in \Cpp{}) to support iterators, 
and extra functionality (like \operatorB{} in \Cpp{}) to support 
array-based access.   
}
\p{In effect, the semantics of types is much more detailed than 
simply describing the kind of values \ty{} may instantiate.  
A \q{list of numbers} may have one abstract profile, but there 
are a broad range of practical implementations which 
build, traverse, and read numbers from the list in different ways.  
Two types which have mathematically the same \q{space} of 
values may be distinct types with very different programming interfaces.  
While it is abstractly true that any non-empty list can 
be \q{deconstructed} into a single element and a shorter-by-one 
predecessor with that element removed, a given list implementation 
may not support procedures to compute that deconstruction in an 
efficient manner.  As a result, mathematical properties of 
types' sets of possible values have only limited applicability 
to types' operational semantics. 
}
\p{This point also reinforces the insight that types, in applied 
type systems such programming languages', are not really mathematical 
entities \mdash{} they are digital artifacts designed by an implementer 
to be programmatically employed in specific ways.  When analyzing 
types we therefore have to explicate the usage patterns 
that are facilitated by their implementations.  Type systems can 
expedite this process by allowing types to be described 
in ways that clarify their intended and expected use-cases.
}
\p{The first step in descriging types' usage, moreover, is to 
account for their \i{constructors}, i.e., for the 
procedures which create new \ty{} values.  Constructors 
are different from other procedures which output \ty{} values 
because constructors are internal to how the type is 
designed; they are in a sense \q{part of} the type.  
In most cases, any programmer can write procedures which 
return instances of \ty{}s in \TyS{}, but most type 
systems have restrictions on where \ty{} \i{constructors} 
are implemented.  Constructors are intrinsic to types in 
that redesigning constructros for \ty{} is understood to 
modify \ty{}'s interface, whereas simply writing a 
procedure which returns a \ty{} is not normally seen as 
\q{changing} \ty{}.  Moreover, any \q{external} procedure 
which returns \ty{} is understood to call a \i{constructor} 
for \ty{} to obtain the value to be the external procedures 
outcome, or at least to call some other procedure which 
calls a \ty{}-constructor, etc. \mdash{} whenever a \ty{} is 
the \i{outcome} of a procedure, at some point, during some 
(maybe nested) procedure, there is a call to a \ty{}-constructors.  
Constructors then become landmarks for identifying the properties 
of \ty{}, because all \ty{}-values originate from 
\ty{}-constructors at some point. 
}
\p{In \cite{MeHBTT} I also introduce the idea of \q{co-constructors}, which 
are conceptually similar to constructors but which wrap the \q{real} constructors 
in separate procedures which can present a streamlined type interface.  
The technical details of co-constructors vs. normal constructors are 
not pertinent to this paper, but suffice it to say that a type system 
may choose to make \i{co-constructors} the basic building-blocks of 
a type interface.  On this strategy, code which is not part of 
the \q{core} implementation of \ty{} would not call \ty{}'s constructors 
directly, but instead would call \ty{} co-constructors.\footnote{The same applies for \TyS{} understood not as the system embraced 
by a \i{language}, but rather the norms adopted by a library or 
application: in \Cpp{}, say, developers could enforce a framework 
of co-constructors by strategically excluding (what \Cpp{} would
call) actual constructors from classes' public interface.
}
}
\p{Co-constructors are similar to what some programmers 
call \q{factory methods} or \q{object factories}, and 
are similar in intent.  Actual constructors have 
some language-limitations or peculiarities: in 
\Cpp{}, for example, you cannot take a pointer to a 
constructor.  On the other hand, insofar as co-constructors 
are ordinary (non-member) procedures one can take their 
address, e.g. for a lookup table mapping strings to 
co-constructor pointers; i.e., co-constructors are more 
amenable to \q{reflection} whereby programmers can dynamically 
invoke a procedure by supplying its name (which in turn is useful 
for allowing applications to be fine-tuned via scripts, or 
constructing objects at runtime from a database).  Constructors 
are also sometimes \q{default-implemented} by compilers behind 
the scenes.  Co-constructors or \q{factories}, then, are 
in some contexts more precise representations of 
types' intended usage patterns than the actual constructors 
as recognized by compilers.    
}
\p{In particular, implementers of a type \ty{} may use co-constructors to 
document and differentiate patterns in how \ty{} values are created.  
Constructors for a \ty{} can be classified into several patterns, such as: 
\begin{itemize}\item{} Default constructors which require no further inputs.  Default-constructed 
values may be deemed conceptually valid instances of their type (e.g. \Zero{} is a 
valid integer) or may also be \q{special} values indicating missing 
data (like \null{} pointers or \q{\NaN{}}).   
\item{} Literal constructors which initialize \ty{} values from literal strings.
\item{} Binary constructors which initialize \ty{} values from binary resources 
holding preexisting instances; in the simplest case simply copying the 
bytes in a \vVar{} to initialize a \vVarPrime{}.
\item{} Pattern-based constructors which initialize \ty{} values from an 
aggregate data structure which may (but need not) include other \ty{} values.  
This would include building a list \vVarPrime{} from a shorter-by-one 
list \vVar{} by appending an element to the list, if that procedure is 
exposed as a (co-)constructor.
\item{} Resursive constructors allow values obtained 
by pattern-based constructions to be \q{deconstructed} and 
used for recursive algorithms, as outlined earlier in the case of lists.
\end{itemize}
I could add further details to this breakdown \mdash{} (co-) constructors which 
validate their input, for instance \mdash{} but the basic idea is that 
types are often designed with implicit assumptions about how they are 
to be used, and these assumptions become manifest in what sorts of 
constructors are provided.  These design patterns can be 
made more rigorous or explicit by consciously notating and classifying 
what sort of use-case is envisioned; one way to achieve this is by 
making object factories or co-constructors the basic public interface 
for a type, and then supplementing co-constructors with metadata 
that describes the type interface in a systematic manner.   
}
\p{Assuming this methodology, we then have an additional set of 
tools for reasoning about \ty{} values.  Upon enumerating 
various \i{kinds} of (co-)constructors, as above, we 
can specify whether a \ty{}-value \vVar{} could be the 
product of a co-constructor of a given kind \mdash{} whether 
\vVar{} could be default-constructed, say, or 
constructed from a source-code literal.  Intuitively 
we then have an idea of \q{partitioning} the space of a 
type into regions based on the kind of construction that results 
in the curresponding \ty{}-values.  This picture is hard 
to make fully rigorous because is not automatically given 
how we should think of type-instances as a \q{space}.  
For some types, we can neatly list all their possible 
values (say, signed bytes are every number from \codeTextrr{7pt}{-127} 
to \codeTextrr{7pt}{127}), but in many varying-size types the actual 
set of values that could be represented at any moment, in any 
particular computing environment, will dynamically depend on 
factors like available memory.  It is impossible to say, 
for instance, just how long a list can become before it 
requires too much memory, which in turn would result in the 
proposed list failing to be constructed.   
}
\p{In short, we need analytic methodology which does not treat 
types as if they were \q{sets of values}.  In the framework 
of channels and carriers, I try to achieve this by reasoning 
about types through the carriers which hold type-instances, 
and by defining carrier \i{states} (including states orthogonal 
to any type system, e.g. a carrier which \i{doesn't} hold a 
value).  With this foundation we can talk about the 
\q{space} of type-instances in terms of \i{states} on carriers.  
Suppose a carrier \cCar{} holds a \ty{} value produced by a co-constructor 
of a given kind (\kVar{}, say).  We can then introduce \kVar{} as a 
state on \cCar{}: \cCar{} is in the state of holding a value 
emerging from a \kVar{}-co-constructor.  This provides 
potentially useful information.  If \kVar{} corresponds 
to a default-initialized \q{sentinel} value 
\mdash{} i.e., a fallback like \null{} for unavailable data 
\mdash{} then such a state corresponds to holding a 
conceptually \q{invalid} \ty{}-instance.  Or, if 
\kVar{} corresponds to a pattern-based construction 
suitable for recursion, \vVar{} could be used 
in recursive algorithm. 
}
\p{Note also that many types have a notion of a \q{fallback} or 
\q{default} value, which may or may not be \i{valid}.  
For numeric types, that value is usually zero, but 
the meaning of \Zero{} can depend on context.  Consider a 
procedure which checks a database to learn someone's age: 
the default \Zero{} may be intended to mean that this 
information was not found or not provided.  However, 
in (say) a medical context, \Zero{} may also be the 
(real) age of an infant.  Analogously, an empty string 
might mean that someone's middle name is not known; 
or that someone does not \i{have} a middle name.    
}
\p{To avoid confusion, types often are built around two 
\q{special} values, or are engineered so that a default 
\q{sentinel} \null{} value cannot be confused with 
a conceptually meaningful value.  In computer graphics, consider the case 
of color: a reasonable default value (for drawing a line, say) 
would be the color black \mdash{} which also, in \RGB{} encoding, 
corresponds to vector of three \Zero{}s, so it is consistent 
with default-to-zero conventions.  On the other hand, a system 
may need to identify situations where a color is unknown or 
not specified (analogous to an unknown age vs. a baby's \Zero{} 
years), so a type representing colors may have some extra 
value meaning \q{no color} \mdash{} which would not be confused 
with or deemed equal to black.
}
\p{Analogously, most 
programming allow the (technically negative) number \negOne{} 
to be used in an \i{unsigned} context, as a special 
\q{unknown} value.  If someone's age is given as \negOne{}, then, 
it would be clear that the intent is to report that the age is 
not known, with no confusion \visavis{} a child before their 
first birthday.  Numerically, \negOne{} would actually have a 
binary encoding (most likely) as \codeTextrr{7pt}{255}, which would never 
be confused with someone's real age.  Similarly, types 
representing textual strings sometimes distinguish 
\i{empty} strings (like when someone is known not to have a middle name) 
from \i{null} strings (representing missing or unknown data). 
}
\p{In practice, accounting for cases of default or missing data 
is an essential part of designing types, qua digital artifacts.  
If a \ty{} value is not known or not provided, should a 
(valid) default value be used instead?  Black, say, is 
a reaonable default for colors (though what about color-systems 
with transparency: should the default be fully opaque colors, 
with no transparency effects, or fully transparent and therefore 
invisible colors)?  Conversely, \Zero{} (when it also means 
zero years, not yet one yeal old) is probably not reaonable 
default for someone's age.  When data is missing, should 
default values be used; if not, how should the problem 
be represented?  These decisions influence our conceptual 
understanding of types' spectrum of values and their 
expected uses.  A default and conceptually valid 
value (like black in the realm of colors) is 
conceptually different than a default value which 
is \i{not} conceptually a \q{real-life} instance of the 
type (like \negOne{} for someone's age) \mdash{} 
let's call these \i{meaningful} and \i{meaningless} defaults, 
respectivel \mdash{} and that 
in turn is different from an invalid value which should 
generally not be passed between procedures (which I'll 
likewise call an \i{invalid} default).   
For an example of this last distinction, a sentinel \q{\NaN{}}  
should rarely actually be passed to procedures expecting a number 
\mdash{} on the premise that any calculation on \NaN{} should yield 
\NaN{}, so the call would be superfluous \mdash{} while 
it is quite common to pass \null{} pointers in \Cpp{} even though 
their conceptual meaning is that they do \i{not} point 
to any memory address (so, conceptually, they are not \q{real} pointers). 
In short, \i{black} is a meaningful default, \null{} is a 
meaningless but not invalid default for pointers, and 
\NaN{} is typically an invalid default for numbers \mdash{} or at 
least this summarizes typical usage patterns.
}
\p{Such conceptual patterns in how we think about types are, at least in 
part, formal analogs to the conceptual semantics of Natural Language.  
Here I will in fact argue that unpacking the conceptual variations between 
different type-instances \mdash{} meaningful, meaningless and 
invalid defaults, for instance, or literal-intitializable values 
\mdash{} gives rise to a sense of types' \q{conceptual spaces} 
which is to some degree analogous to Conceptual Space Theory.  
Before launching into that discussion, however, I will 
review some more details in the \q{theory of constructors} 
with an eye toward defining very general (while still 
formally rigorous) \TyS{} systems.  
}
\subsection{Nonconstructive Type Theory}
\p{Thus far I have suggected that types' conceptual 
and operational profiles can be defined in part by 
describing the system of constructors (or co-constructors) 
through which their values may be created.  The process 
by which a particular \ty{}-value \vVar{} has been 
created can be a factor in how \vVar{} may be used.  
For example, most functional programming languages 
allow procedures to be implemented via \q{pattern matching}, 
which means splitting the procedure into different 
versions or different routines based on the nature of 
an input value, which can be determined by how it 
was constructed.  A canonical example is procedures 
defined on lists: suppose \vVar{} can be 
\q{deconstructed} into a smaller-by-one sublist 
\vVarPrime{} and a single element \eVar{} \mdash{} 
\vVar{} is \vVarPrime{} appended by \eVar{}.  
The right-hand side (\vVarPrime{} with \eVar{}) can 
be called a \i{construction pattern} which 
yields, or defines the provenance of, \vVar{}.  
On that basis, a procedure which operates 
on \vVar{} could equally well, at least logically 
or conceptually, be seem as operating on 
the \vVarPrime{} and \eVar{} pair.  On the 
other hand, if \vVar{} is an empty list, 
then algorithms need to proceed differently 
than for \vVar{} non-empty.  In combination, this 
yields an outline for procedures as follows: 
differentiate empty from non-empty \vVar{}; 
for the latter, allow the procedure to 
operate on a \vVarPrime{} and \eVar{} pattern, 
rather than on \vVar{} directly.     
}
\p{Programming languages which want to support these 
\q{pattern matching} features need two capabilities: 
they need to be able to implement procedures which 
bifurcate based on which pattern matches; and 
they need to take a multi-part structure as 
a procedural input, like \vVarPrime{} \i{and} \eVar{}, 
in the place of a single carrier like \vVar{}.  
One straightfoward way to achieve this is to 
differentiate procedures based on the construction-patterns 
evinced by their arguments.  For example, we can 
consider a procedure which \i{only} operates 
on \i{empty} lists, paired with a procedure 
which \i{only} operates on \i{non-empty} lists.
}
\p{We then have to investigate how these distinctions 
intersect with the relevant type system. 
Should \TyS{} stipulate that procedures for 
empty lists have a different \i{type} than procedures 
for non-empty lists?  Note that in general the 
empty/non-empty distinction does not yield 
different type-attributions: a non-empty list is 
not a different \i{type} than an empty list (assuming 
compatibility in the type of elements declared to 
go in the list).  There are nonetheless frameworks 
which would allow a \i{function} taking empty 
\ty{}s (for list-type \ty{}) being considered a 
different type than ones taking non-empty \ty{}s.  
For procedures taking non-empty lists, moroever, 
their argument can (potentially) be converted into a 
construction-pattern (like \vVarPrime{} and \eVar{}), 
so that the single input to (this verson of the) 
procedure actually becomes two different inputs.  
Enabling procedures to be split and designed 
in this manner \mdash{} split and paired off by pattern-matching 
and taking compound inputs \mdash{} requires \TyS{} types to 
be implemented with the requisite capabilities 
(e.g., calculating the proper construction-pattern for a 
given \ty{}-value).  Because this sense of 
\q{pattern matching} is a common idiom in functional 
programming languague, it certainly needs to be accommodate 
in a broad-based type theory attuned to the type 
systems of different kinds of formal languages.\footnote{Note that there are other, unrelated uses of the phrase 
\q{pattern matching} in programming and computer science 
\mdash{} e.g., in the context of regular expressions, which is 
essentially entirely different from the current context.
}  
}
\p{Conversely, though, it is just as common for types 
to \i{lack} the infrastructure for pattern-matching in 
this sense, so we need to these features as 
\i{possible} but not \i{necessary} aspects of types.  
For sake of discussion, I will call types amenable 
to pattern-matching \i{constructive}; and otherwise 
\i{non-constructive}.  There is no need here to 
formalize a technical definition of the comparison, 
but semi-formally I'll say that a \i{constructive} type 
has (or can be provisioned with) a procedure which, 
given any instance \vVar{}, can return a construction-pattern 
which reciprocates the construction of \vVar{} from other 
values.  In this context I treat the functionality to 
calculate patterns as an ordinary procedural interface on 
a type: for constructive \ty{} there will be an associated 
\q{construction pattern} type (usually a type \i{different} from 
\ty{}) and procedures to map \ty{}s to their corresponding patterns.  
Pattern-matching can then be achieved, or at least emulated, 
by defining procedures on the construction-pattern types 
for \ty{} rather than \ty{} itself.  
}
\p{Implementing constructive types introduces some complications 
in conjunction with an overall type interface, which might not 
be immediately apparent.  For instance, consider types 
which have (what I earlier called) \q{binary constructors} 
\mdash{} e.g., a \ty{} which can initialize values by copying 
the bit-pattern of some other \ty{}-instance.  For many 
\TyS{} this option results in the theoretical possibility 
that \ty{}s could be created with any bit-pattern whatsoever.  
In \Cpp{}, for example, a pointer could potentially point to 
some random area of memory (after an error in neglecting 
to initialize the pointer, say), from which a dereferenced yields 
a \ty{} value manifest in a random sequence of bits.  Such 
randomness is almost always an error, but this does not 
preclude code having to accept the possibility that 
\ty{}-values might be \q{randomly} constructed.  In this 
case, the data which could be used derive construction-patterns 
may well become corrupted.  
}
\p{Suppose a type offers an interface to return construction-patterns 
for all of its values, but makes assumptions about these values' 
binary layout when deriving those patterns.  For instance, a 
list type might be based on a list-pointer together with fields 
indexing the start and end of the list.  With this arrangement, 
one single list pointer \mdash{} i.e. one list in memory \mdash{} can 
be the basis for multiple \ty{}-values, by varying their 
start and/or endpoints.  A construction pattern for a list 
\vVar{} could then be efficiently obtained by noting the 
element \eVar{} at the start or end of \vVar{}, and producing a 
\vVarPrime{}-\eVar{} pair by constructing \vVarPrime{} as a 
variant on \vVar{} with its start or end index advanced (respectively 
decreased) by one.  This is a plausible \q{deconstructing} scheme 
because all the intermediate list values obtained in construction-patterns, 
and then potentially used in recursive procedures, share the same 
underlying memory; there is no copying or modifying of in-memory 
data.  A list is then considered \i{empty} if its start and 
end indices are the same.  A recursive algorithm 
would work with a sequence of construction-patterns, with 
the two indices getting closer together, until the recursion 
would be broken by final list being empty. 
}
\p{The problem here is that this design only works if the start 
and end indices for the original \vVar{} are in the correct order 
(the start must be less than or equal to the end).  
If that requirement cannot be guaranteed, then the 
above derivation of construction-patterns cannot be guaranteed 
to work properly; in particular, a recursive algorithm might 
loop infinitely.  As this example shows, a constructive type 
may need to double-check all values at their point of construction 
to ensure that the type's various fields and internal data 
are configured properly to support features like pattern-matching.  
A constructive \ty{}, in short, may need to ensure that 
no \ty{}-values are constructed without certain validation checks being 
performed (this would be one use-case for a co-constructor).  
For instance, simply copying bit-patterns from one place to another 
may have to be disabled as a tactic for copy-constructing 
values, unless the newly constructed data structure is validated.   
}
\p{Given these considerations, it is certainly possible that some 
types will be non-constructive \mdash{} i.e., they decline 
to provide procedures that return construction-patterns, 
and to provision the support needed to use construction-patterns \mdash{} 
even if the logical characteristics of the types' values would seem 
to support a pattern-based interface.  In practice, 
programming languages that enable pointer-based 
access to values, and pointer-dereferencing, tend not to 
natively recognize construction-patterns, and vice-versa.  
}
\p{The idea of proxying \ty{}-values via construction-patterns 
has mathematical foundations, emanating from \q{constructive} 
mathematics.  In particular, consider \q{recursive} construction 
patterns, where a \vVar{} is \q{deconstructable} into 
a \vVarPrime{}/\eVar{} pattern, and then \vVarPrime{} is further 
substitutable by a further pattern based on a \vVarDblPrime{}, 
and so on.  In many cases, for any \ty{}-instance \vVar{} there is 
then a determinate sequence of constructions which 
eventually produces \vVar{}, and likewise an \q{inverse} 
seqeunce of construction patterns which \q{undoes} those 
constructions.  For instance, any list of numbers can 
be seen as the product of numerous constructions which 
begins with an empty list, and produces successively 
larger lists by appending one number at a time, 
eventually arriving at an end-result \vVar{}.  
In a language like Haskell, the notions of 
\i{list of elements} and \i{sequence of constructions 
(appending elements to a prior list)} are understood to be 
conceptually indistinguishable.  That is to say, 
\ty{} values are understood to be internally inseparable 
from the progression of steps which provide 
a recipe for constructing those \ty{}s.  
}
\p{Mathematically, an analogous assumption is that 
the space of \ty{}-values is isomorphic to the 
space of construction-sequences which yield 
\ty{}s.  We can also say that the space of these 
sequences is a \i{model} for \ty{}.  A type mathematically 
representing \i{lists of integers}, say 
\mdash{} meaning in this context a logical specification 
of some mathematical space \mdash{} can be said to be 
modeled by the space of programs which 
produce these lists by starting from an empty list and 
progressively appending numbers.  This \q{space of programs} 
is a \i{model} for the type insofar as it satisfies the 
types' logical requirements.  This is one example of 
a project for analyzing mathematical spaces in terms 
of \i{finite constructions} which yield elements of those 
spaces.  In constructive mathematics, proofs of 
propositions on such \q{finite constructions} is considered to 
be easier, or more logically sounder, than proofs 
which engage with infinite spaces and rely on 
logical indirections, like the \q{law of the excluded middle}.    
}
\p{Constructive types, then, inherit this mathematical 
backstory insofar as such types allow us to 
deem types' space of values as, in essence, logically 
indistinguishable from the space of construction-sequence 
that yield those values.  A \i{constructive} type theory 
would be one which treats all types as constructive, perhaps  
on the basis of logical or philosophical reasoning: 
we can say in the abstract, for instance, that any list 
is \i{logically} isomorphic to a sequence of sublists 
building up to it.  In practice, though, I consider a 
type constructive only if it \i{explicitly} 
provides the infrastructure needed (or if that happens 
automatically given the relevant implementation language) 
so that \q{constructive mathematics} intuitions can be 
concretely leveraged.  I will say that a type system 
is \i{non-constructive} if it does not \i{assume} that its 
types are constructive; a non-constructive \TyS{} may still 
have \i{some} constructive types. 
}
\p{In short, a non-constructive \TyS{} actually \i{generalizes} 
constructive frameworks: by allowing both constructive and 
non-constructive types it presents a superset subsuming 
\TyS{}s failing to properly model non-constructive 
types, as well as \TyS{}s failing to properly model \i{constructive} 
types.  I claim therefore that non-constructive \TyS{}s 
are the requisite framework to represent the spectrum of 
applied type systems in greatest generality. 
}
\p{So, in this section I have presented several features of 
type systems which, I believe, allow us to achieve the greatest 
breadth in covering the diversity of possible \TyS{}s 
while staying within the bounds of software-centric, 
implementational rigor: type systems which are 
\i{non-constructive}, \i{channelized}, and 
\i{co-cyclic}, from which I derive the \q{\NCFour{}} moniker.  
In this paper, then, I want to use \NCFour{} type systems 
as a foundation for studying the semantics of 
formal languages, in the hopes of deriving an \HCS{} theory 
\mdash{} a theory of unifying Hypergraphs with Conceptual Spaces 
\mdash{} for these languages which can be usefully paired with analogous 
unifications for \i{natural} language.     
}

\section{Directed Hypergraphs and Generalized Lambda Calculus}
\phantomsection\label{sThree}
\p{Thus far in this chapter, I have written in general terms about 
architectural features related to Cyber-Physical software;
especially, verifying coding assumptions concerning 
individual data types and/or procedures.  My comments 
were intended to summarize the relevant territory, 
so that I can add some theoretical details or suggestions 
from this point forward.  In particular, I will explore 
how to model software components at different scales 
so as to facilitate robust, safety-conscious coding practices.   
}
\p{Note that almost all non-trivial software is in some 
sense \q{procedural}: the total package of 
functionality provided by each software component 
is distributed among many individual, interconnected 
procedures.  Each procedure, in general, 
implements its functionality by calling \i{other} procedures 
in some strategic order.  Of course, often 
inter-procedure calls are \i{conditional} \mdash{} a calling 
procedure will call one (or some sequence of) procedures 
when some condition holds, but call alternate procedures
when some other conditions hold.  In any case, computer 
code can be analyzed as a graph, where connections exist 
between procedures insofar as one procedure calls, 
or sometimes calls, the other.
}
\p{This general picture is only of only limited applicability 
to actual applications, however, because the basic concept 
of \q{procedure} varies somewhat between different 
programming languages.  As a result, it takes some effort 
to develop a comprehensive model of computer code which 
accommodates a representative spectrum of coding styles 
and paradigms. 
}
\p{There are perhaps three different perspectives for 
such a comprehensive theory.  One perspective is 
to consider source code as a data structure in its 
own right, employing a Source Code Algebra or 
Source Code Ontology to assert properties of source code 
and enable queries against source code, qua information 
space.  A second option derives from type theory: to 
consider procedures as instances of functional types,
specified by tuples of input and output types.  
A procedure is then a transform which, in the presence 
of (zero or more) inputs having the proper types, produces 
(one or more) outputs with their respective types.   
(In practice, some procedures do not return values, but they 
\i{do} have some kind of side-effect, which can be analyzed as a 
variety of \q{output}.)  Finally, third, procedures can be 
studied via mathematical frameworks such as Lambda Calculus, 
allowing notions of functions on typed parameters, and 
of functional application \mdash{} applying functions to concrete 
values, which is analogous to calling procedures with 
concrete input arguments \mdash{} to be made formally rigorous.
}
\p{I will briefly consider all three of these perspectives \mdash{} Source Code 
Ontology, type-theoretic models, and Lambda Calculus \mdash{} in 
this section.  I will also propose a new 
model, based on the idea of \q{channels}, which combines 
elements of all three. 
}
\vspace{-.1em}
\subsection{Generalized Lambda Calculus}
\p{Lambda (or \mOldLambda{}-) Calculus emerged in the early 
20th Century as a formal model of mathematical 
functions and function-application.  There are 
many mathematical constructions which can be 
subsumed under the notion of \q{function-application}, 
but these have myriad notations and conventions 
(compare the visual differences between mathematical 
notations \mdash{} integrals, square roots, super- and 
sub-scripted indices, and so forth \mdash{} to the much 
simpler alphabets of mainstream programming languages).  
But the early 20th century was a time of 
great interest in \q{mathematical foundations}, seeking 
to provide philosophical underpinnings for mathematical 
reasoning in general, unifying disparate mathematical 
methods and subdisciplines.  One consequence of 
this foundational program was an attempt to capture 
the formal essence of the concept of \q{function} and 
of functions being applied to concrete values.       
}
\p{A related foundational concern is how mathematical 
formulae can be nested, yielding new formulae.  
For example, the volume of a sphere 
(expressed in terms of its radius \rRad{}) is \VolSphere{}.  
The symbol \rRad{} is just a mnemonic which could be replaced with a different symbol, 
without the formula being different.  But it can also be replaced by a more complex 
expression, to yield a new formula.  In this case, 
substituting the formula for a cube's half-diagonal \mdash{} \crVOverRTwo{}
where \vVol{} is its volume \mdash{} for \rRad{}, in the first formula, 
yields \volSphCube{}: a formula for the sphere's volume in terms 
of the volume of the largest cube that can fit inside it     
(\cite{KennethAnderson} has similar interesting examples in the 
context of code optimization).  This kind of tinkering with equations is 
of course a bread-and-butter of mathematical discovery.  In terms 
of foundations research, though, observe that the derivation depended on 
two givens: that the \rRad{} symbol is \q{free} in the first formula 
\mdash{} it is a place-holder rather than the designation of a concrete 
value, like \piSym{} \mdash{} and that free symbols (like \rRad{}) can be 
bound to other formulae, yielding new equations. 
}
\p{From cases like these \mdash{} relatively simple geometric 
expressions \mdash{} mathematicians began to ask foundational 
questions about mathematical formulae: what are all formulae 
that can be built up from a set of core equations via 
repeatedly substituting nested expressions for
free symbols?  This question turns out to be related to 
the issue of finite calculations: in lieu of 
building complex formulae out of simpler parts, we can 
proceed in the opposite direction, replacing nested 
expressions with values.  Formulae are constructed 
in terms of unknown values; when we have concrete measurements 
to plug in to those formulae, the set of unknowns decreases.
If \i{all} values are known, then a well-constructed formula 
will converge to a (possibly empty) set of outcomes.  
This is roughly analogous to a computation which 
terminates in real time.  On the other hand, 
a \i{recursive} formula \mdash{} an expression nested inside 
itself, such as a continued fraction \mdash{} is analogous to 
a computation which loops indefinitely.\nobrfootnote{Although there are sometimes techniques for converting formulae 
like Continued Fractions into \q{closed form} equations 
which do \q{terminate}.
% It may be desirable to write this as "nobrfootnote" ...
}
}
\p{In the early days of computer programming, it was natural to 
turn to \mOldLambda{}-Calculus as a formal model of 
computer procedures, which are in some ways analogous to 
mathematical formulae.  As a mathematical subject, 
\mOldLambda{}-Calculus predates digital computers as 
we know them.  While there were no digital computers at the time,
there \i{was} a growing interest in mechanical computing 
devices, which led to the evolution of 
cryptographic machines used during the Second World War.  
So there was indeed a practical interest in 
\q{computing machines}, which eventually led to 
John von Neumann's formal prototypes for digital 
computers.  
}
\p{Early on, though, \mOldLambda{}-Calculus was less about blueprints
for calculating machines and more about \i{abstract} formulation of
calculational processes.  Historically, the original 
purpose of \mOldLambda{}-Calculus was largely a mathematical \i{simulation} of
computations, which is not the same as a mathematical \i{prototype}
for computing machines.  Mathematicians in the decades before WWII
investigated logical properties of computations, with particular
emphasis on what sort of problems could always be solved in
finite time, or what kinds of procedures can be guaranteed to
terminate \mdash{} a \q{Computable Number}, for example, is a number
which can be approximated to any degree of precision by a terminating
function.  Similarly, a Computable Function is a function from
input values to output values that can be associated with an
always-terminating procedure which necessarily calculates the desired
outputs from a set of inputs.  The space of Computable Functions
and Computable Numbers are mathematical objects whose properties
can be studied through mathematical techniques \mdash{} for instance,
Computable Numbers are known to be a
countable field within the real numbers.
These mathematical properties are proven using a formal
description of \q{any computer whatsoever}, which has no
concern for the size and physical design of the \q{computers}
or the time required for its \q{programs}, so long
as they are finite.  Computational procedures in
this context are not actual implementations but rather
mathematical distillations that can stand in for
calculations for the purpose of mathematical analysis
(interesting and representative contemporary articles
continuing these perspectives include, e.g., \cite{MartinEscardo},
\cite{MasahitoHasegawa}, \cite{TuckerZucker}).
}
\p{It was only after the emergence of modern digital computers 
that \mOldLambda{}-Calculus become reinterpreted as a 
model of \i{concrete} computing machines.  In its 
guise as a Computer Science (and not just Mathematical 
Foundations) discipline, \mOldLambda{}-Calculus has been 
most influential not in its original form but in 
a plethora of more complex models which track the 
evolution of programming languages.  Many programming 
languages have important differences which are not 
describable on a purely mathematical basis: two 
languages which are both \q{Turing complete} are 
abstractly interchangeable, but it is important to
represent the contrast between, say,
Object-Oriented and Functional programming.  
In lieu of a straightforward, mathematical model 
of formulae as procedures which map inputs to 
outputs, modern programming languages add 
may new constructs which determine different
mechanisms whereby procedures can read and 
modify values: objects, exceptions, closures, 
mutable references, side-effects, signal/slot 
connections, and so forth.  Accordingly, 
new programming constructions have inspired 
new variants of \mOldLambda{}-Calculus, analyzing
different features of modern programming languages \mdash{} Object
Orientation, Exceptions, call-by-name, call-by-reference,
side effects, polymorphic type systems, lazy evaluation 
\mdash{} in the hopes of deriving formal proofs of 
program behavior insofar as computer code uses the 
relevant constructions.  In short, a reasonable
history can say that \mOldLambda{}-Calculus mutated from being an
abstract model for studying Computability as a mathematical concept,
to being a paradigm for prototype-specifications of concretely
realized computing environments.
}
\p{Modern programming languages have many different ways of handing-off
values between procedures.  The \q{inputs} to a function can be \q{message receivers}
as in Object-Oriented programming, or lexically scoped
values \q{captured} in an anonymous function that inherits
values from the lexical scope (loosely, the area of source code)
where its body is composed.  Procedures can also \q{receive} data indirectly
from pipes, streams, sockets, network connections, database connections, or files.
All of these are potential \q{input channels} whereby a function implementation
may access a value that it needs.  In addition, procedures can \q{return} values
not just by providing a final result but by throwing exceptions, writing
to files or pipes, and so forth.  To represent these myriad
\q{channels of communication} computer scientists have invented a menagerie
of extensions to \mOldLambda{}-Calculus \mdash{} a noteworthy 
example is the \q{Sigma} calculus to model Object-Oriented
Programming; but parallel extensions represent call-by-need evaluation, 
exceptions, by-value and by-reference capture, etc.
}
\p{Rather than study each system in isolation, in this 
chapter I propose an integrated strategy for 
unifying disparate \mOldLambda{}-Calculus extensions 
into an overarching framework.  The 
\q{channel-based} tactic I endorse here may not be 
optimal for a \i{mathematical} calculus which has 
formal axioms and provable theorems, but I believe 
it can be useful for the more practical goal 
of modeling computer code and software 
components, to establish recommended design patterns 
and to document coding assumptions.  
}
\p{In this perspective, different extensions
or variations to \mOldLambda{}-Calculus model different 
\i{channels}, or data-sources through which procedures 
receive and/or modify values.  Different channels 
have their own protocols and semantics
for passing values to functions.
We can generically discuss \q{input} and
\q{output} channels, but programming languages have different specifications
for different genres of input/output, which we can model via
different channels.  For a particular channel, we can recognize
language-specific limitations on how values passed in to or
received from those channels are used, and how the symbols carrying those
values interact with other symbols both in function call-sites and in the 
body of procedure implementations.  
For example, procedures can output values by throwing exceptions, but
exceptions are unusual values which have to be handled
in specific ways \mdash{} languages use exceptions to signal possible
programming errors, and they are engineered to interrupt
normal program flow until or unless exceptions are \q{caught}.
}
\p{Computer scientists have explored these more complex programming paradigms
in part by inventing new variations on \mOldLambda{}-calculi.  
Here I will develop one theory representing
code in terms of Directed Hypergraphs, which are subject to multiple kinds
of lambda abstraction \mdash{} in principle, unifying multiple 
\mOldLambda{}-Calculus extensions.
The following subsection will lay out the details of this form of Directed Hypergraph
and how \mOldLambda{}-calculi
can be defined on its foundation, while the last subsection 
summarizes an expanded type theory which follows organically
from this approach.
}
\p{Many concepts outlined here are reflected in the accompanying code set
(which includes a \Cpp{} Directed Hypergraph library).
My strategy for unifying multiple \mOldLambda{}-calculi depends 
in turn on hypergraph code representations, which is a theme 
in the umbrella of graph-based data modeling, 
to which I now turn. 
}
\vspace{-.1em}
\subsectiontwolinerepl{Directed Hypergraphs and \q{Channel Abstractions}}%
{Directed Hypergraphs and 'Channel Abstractions'}
\p{A \i{hypergraph} is a graph whose edges (a.k.a. \q{hyperedges}) can span
more than two nodes (\cite[e.g. volume 2, p. 24]{BenGoetzel},
\cite{HaishanLiu}, \cite{MarkMinas}, \cite{BalintMolnar},
\cite{AlexandraPoulovassilis}, \cite{JohnStell},
\cite{JohnStellFCA}).
A \i{directed} hypergraph (\q{\DH{}}) is a hypergraph
where each edge has a \i{head set} and
\i{tail set} (both possibly empty).  Both of these are sets of nodes
which (when non-empty) are called \i{hypernodes}.  A hypernode can
also be thought of as a hyperedge whose tail-set
(or head-set) is empty.  Note that a typical hyperedge
connects two hypernodes (its head- and tail-sets), so if
we consider just hypernodes, a hypergraph potentially
reduces to a directed ordinary graph.  While
\q{edge} and \q{hyperedge} are formally equivalent,
I will use the former term when attending more to the
edge's representational role as linking two hypernodes,
and use the latter term when focusing more on its tuple
of spanned nodes irrespective of their partition into
\i{head} and \i{tail}.
}
\p{I assume that hyperedges always span an \i{ordered} node-tuple
which induces an ordering in the head- and tail-sets: so a
hypernode is an \i{ordered list} of nodes, not just a
\i{set} of nodes.  I will say that
two hypernodes \i{overlap} if they
share at least one node; they are \i{identical}
if they share exactly the same nodes in the same order; and
\i{disjoint} if they do not overlap at all.  I call a
Directed Hypergraph
\q{reducible} if all hypernodes are either disjoint or
identical.  The information in reducible \DH{}s can be factored
into two \q{scales}, one a directed graph whose nodes are the
original hypernodes, and then a table of all nodes
contained in each hypernode.  Reducible \DH{}s allow
ordinary graph traversal algorithms when hypernodes
are treated as ordinary nodes on the coarser scale
(so that their internal information \mdash{} their list
of contained nodes \mdash{} is ignored).\footnote{A weaker restriction on \DH{} nodes is that two
non-identical hypernodes \i{can} overlap, but
must preserve node-order: i.e., if the first
hypernode includes nodes \nodeNOne{}, and \nodeNTwo{}
immediately after, and the second hypernode
also includes \nodeNOne{}, then the second
hypernode must also include \nodeNTwo{} immediately
thereafter.  Overlapping hypernodes
can not \q{permute} nodes
\mdash{} cannot include them in different orders or in a way
that \q{skips} nodes.
Trivially, all reducible \DH{}s meet this condition.  
Any graphs discussed here are assumed to meet 
this condition.    
}
}
\p{To avoid confusion, I will hereafter use the word \q{hyponode} in place
of \q{node}, to emphasize the container/contained relation between
hypernodes and hyponodes.  I will use \q{node} as an informal word
for comments applicable to both hyper- and hypo-nodes.  Some
Hypergraph theories and/or implementations
allow hypernodes to be nested: i.e., a hypernode can contain
another hypernode.  In these theories, in the general case any node
is potentially both a hypernode and a hyponode.  For this chapter,
I assume the converse: any \q{node} (as I am hereafter using the term) is
\i{either} hypo- or hyper-.  However, multi-scale Hypergraphs can be approximated
by using hyponodes whose values are proxies to hypernodes.
}
\p{Here I will focus on a class of \DH{}s which (for reasons to emerge)
I will call \q{Channelizable}. Channelizable Hypergraphs
(\CH{}s) have these properties:
\begin{enumerate}\item{}  They have a Type System \TyS{} and all hyponodes and hypernodes are assigned
exactly one canonical type (they may also be considered instances of super- or subtypes
of that type).
\item{}  All hyponodes can have (or \q{express}) at most one value, an instance of its
canonical type, which I will call a \i{hypovertex}.  Hypernodes, similarly,
can have at most one \i{hypervertex}.  Like \q{node} being an informal
designation for hypo- and hyper-nodes, \q{vertex} will be a general term
for both hypo- and hyper-vertices.  Nodes which do have a vertex
are called \i{initialized}.  The hypovertices \q{of} a hypernode are those
of its hyponodes.
\item{}  Two hyponodes are \q{equatable} if they express the same value of the same
type.  Two (possibly non-identical) hypernodes are \q{equatable} if all of their
hyponodes, compared one-by-one in order, are equatable.  I will also say that values
are \q{equatable} (rather than just saying \q{equal}) to emphasize that
they are the respective values of equatable nodes.
\item{}  There may be a stronger relation, defined on equatable non-equivalent hypernodes,
whereby two hypernodes are \i{inferentially equivalent} if any inference justified via
edges incident to the first hypernode can be freely combined with inferences
justified via edges incident to the second hypernode.  Equatable nodes are not
necessarily inferentially equivalent.
\item{}  Hypernodes can be assumed to be unique in each graph, but it is
unwarranted to assume (without type-level semantics) that two equatable
hypernodes in different graphs are or are not inferentially equivalent.
Conversely, even if graphs are uniquely labeled \mdash{} which would
appear to enable a formal distinction between hypernodes in one
graph from those in another, \CH{}
semantics does not permit the assumption that this separation alone
justifies inferences presupposing that their hypernodes
\i{are not} inferentially equivalent.
\item{}  All hypo- and hypernodes have a \q{proxy}, meaning there is a type in
\TyS{} including, for each node, a unique identifier designating
that node, that can be expressed in other hyponodes.
\item{}  There are some types (including these proxies) which may only be expressed
in hyponodes.  There may be other types which may only be expressed
in hypernodes.  Types can then be classified as \q{hypotypes} and \q{hypertypes}.
The \TyS{} may stipulate that all types are \i{either} hypo or hyper.  In
this case, it is reasonable to assume that each hypotype maps to a unique
hypertype, similar to \q{boxing} in a language which recognizes \q{primitive}
types (in Object-Oriented languages, boxing allows non-class-type
values to be used as if they were objects).
\item{}  Types may be subject to the restriction that any hypernode which has that
type can only be a tail-set, not a head-set; call these \i{tail-only} types.
\item{}  Hyponodes may not appear in the graph outside of hypernodes.  However, a
hypernode is permitted to contain only one hyponode.
\item{}  Each edge, separate and apart from the \CH{}'s actual graph structure,
is associated with a distinct hypernode, called its \i{annotation}.  This
annotation cannot (except via a proxy) be associated with any other hypernode
(it cannot be a head- or tail-set in any hypernode).
The first hyponode in its annotation I will 
dub a hyperedge's \i{classifier}.  The outgoing edge-set of a hypernode can
always be represented as an associative array indexed by the classifier's vertex.
\item{}  A hypernode's type may be subject to restrictions such that there is a
single number of hyponodes shared by all instances.  However, other types may be
expressed in hypernodes whose size may vary.  In this case the
hyponode types cannot be random; there must be some pattern linking
the distribution of hyponode types evident in hypernodes (with the same
hypernode types) of different sizes.  For example, the hypernodes
may be dividable into a fixed-size, possibly empty sequence of hyponodes,
followed by a chain of hyponode-sequences repeating the same type pattern.
The simplest manifestation of this structure is a hypernode all of whose
hyponodes are the same type.
\item{}  Call a \i{product-type transform} of a hypernode to be a different
hypernode whose hypovertices are tuples of values equatable to those from the first hypernode,
typed in terms of product types (i.e., tuples).  For example, consider two
different representations of semi-transparent colors: as a 4-vector
\vecrgbt{}, or as an \vecrgb{} three-vector paired with a transparency magnitude.
The second representation is a product-type transform of the first, because the first
three values are grouped into a three-valued tuple.  We can
assert the requirement in most contexts that \CH{}s whose hypernodes are
product-type transforms of each other contain \q{the same information}
and as sources of information are interchangeable.
\item{}  The Type System \TyS{} is \i{channelized}, i.e., closed under a
Channel Algebra, as will be discussed below.
\end{enumerate}
}
\p{These definitions allude to two strategies for computationally representing
\CH{}s.  One, already mentioned, is to reduce them to directed graphs
by treating hypernodes as integral units (ignoring their internal structure).
A second is to model hypernodes as a \q{table of associations} whose
keys are the values of the classifier hyponodes on each of their edges.
A \CH{} can also be transformed into an \i{undirected} hypergraph by
collapsing head- and tail- sets into an overarching tuple.  All of these
transformations may be useful in some analytic/representational contexts,
and \CH{}s are flexible in part by morphing naturally into these various
forms.\phantomsection\label{unplug}
}
\spinctc{unplug}{Unplugging a Node.}{fig:unplug}
\p{Notice that information present \i{within} a hypernode can also be expressed as
relations \i{between} hypernodes.  For example, consider the information that
I (Nathaniel), age \FourtySix{}, live in Brooklyn as a registered Democrat.  This may be
represented as a hypernode with hyponodes \NathFF{}, connected to a hypernode
with hyponodes \BrookDem{}, via a hyperedge whose classifier encodes the
concept \q{lives in} or \q{is a resident of}.  However, it may also be
encoded by \q{unplugging} the \q{age} attribute so the first hypernode becomes
just \Nath{} and it acquires a new edge, whose tail has a single
hyponode \ageFF{} and a classifier (encoding the concept) \q{age}
(see the comparison in Diagram \hyperref[fig:unplug]{\ref{fig:unplug}}).
This construction can work in reverse:
information present in a hyperedge can be refactored so that it \q{plugs in}
to a single hypernode.
}
\p{These alternatives are not redundant.  Generally, representing information
via hyperedges connecting two hypernodes implies that this information is
somehow conceptually apart from the hypernodes themselves, whereas representing
information via hyponodes \i{inside} hypernodes implies that this information
is central and recurring (enforced by types), and that the data
thereby aggregated forms a recurring logical unit.  In a political survey,
people's names may \i{always} be joined to their age, and
likewise their district of
residence \i{always} joined to their political affiliation.  The left-hand side
representation of the info (seen as an undirected hyperedge) \NathFFBD{}
in Diagram \hyperref[fig:unplug]{\ref{fig:unplug}} 
captures this semantics better because it describes
the name/age and \mbox{place/party} pairings as
\i{types} which require analogous
node-tuples when expressed by other hypernodes.  For example, any two
hypernodes with the same type as \NathFF{} will necessarily have
an \q{age} hypovertex
and so can predictably be compared along this one axis.  By contrast, the
right-hand (\q{unplugged}) version in Diagram \hyperref[fig:unplug]{\ref{fig:unplug}}
implies no guarantees that the \q{age}
data point is present as part of a recurring pattern.
}
\itclfig{initializing-hypernodes}{fig:initializinghypernodes}
\p{The two-tiered \DH{} structure is also a factor when integrating 
serialized or shared data structures with runtime data values.  
In the demo \DH{} library, for example, it is assumed that 
each node can be associated with a runtime, binary data allocation 
(practically speaking, a pointer to user data).  Hypernodes' internal 
structure can therefore be represented \i{either} via hyponodes explicit 
in the graph content \i{or} by internal structure in the user 
data (or some combination).  Graph deserialization can then be 
a matter of mapping hyponodes to fields in the \q{internal} data 
allocations, before then mapping inter-hypernode relations to 
the proper hypervertex-relations.  Code sample 
\ref{lst:initializing-hypernodes} demonstrates the pattern 
of hypervertex construction as \Cpp{} objects that get wrapped 
in new nodes ({\OneOverlay}-{\TwoOverlay}), 
along with obtaining nodes already registered in a runtime graph 
({\ThreeOverlay}) and then inserting the new nodes (with stated 
relationships) alongside prior ones into the runtime graph 
({\ThreeOverlay}).       
}
\p{In general, graph representations like \CH{} and \RDF{} serve two goals: first,
they are used to \i{serialize} data structures (so that
they may be shared between
different locations; such as, via the internet);
and, second, they provide
formal, machine-readable descriptions of information content, allowing for
analyses and transformations, to infer new information or produce new data
structures.  The design and rationale of representational paradigms is
influenced differently by these two goals, as I will review now with an eye
in part on drawing comparisons between \CH{} and \RDF{}.
}
\subsection{Channelized Hypergraphs and \largeRDF{}}
\phantomsection\label{RDF}
\p{The Resource Description Framework (\RDF{}) models information
via directed graphs (\cite{MadalinaCroitoru}, \cite{ErnestoDamiani},
\cite{AnglesGuttierez}, and \cite{RodriguezWatkins} are good discussions of
Semantic Web technologies from a graph-theoretic perspective),
whose edges are labeled with concepts that,
in well-structured contexts, are drawn from published Ontologies
(these labels play a similar role to \q{classifiers} in \CH{}s).
In principle, all data expressed via \RDF{} graphs is defined
by unordered sets of labeled edges, also called \q{triples}
(\q{\SPO{}}, where the \q{Predicate} is the label).  In practice,
however, higher-level \RDF{} notation such as \TTL{} (\Turtle{} or
\q{Terse \RDF{} Triple Language}) and Notation3 (\NThree{})
deal with aggregate groups of data, such as \RDF{} containers and
collections.\phantomsection\label{lived}
}
\spinctc{lived}{CH vs. RDF Collections.}{fig:lived}
\p{For example, imagine a representation of the
fact \q{(A/The person named) Nathaniel, \FourtySix{}, has lived in Brooklyn,
Buffalo, and Montreal} (shown in Diagram \hyperref[fig:lived]{\ref{fig:lived}} as both
a \CH{} and in \RDF{}).  If we consider \Turtle{} or \NThree{} as \i{languages} and
not just \i{notations}, it would appear as if their semantics is built
around hyperedges rather than triples.  It would seem that these
languages encode many-to-many or one-to-many assertions, graphed as
edges having more than one subject and/or predicate.  Indeed,
Tim Berners-Lee himself suggests that
\q{Implementations may treat list as a data type rather than just
a ladder of rdf:first and rdf:rest properties} \cite[p. 6]{TimBernersLee}.
That is, the specification for
\RDF{} list-type data structures invites us to consider that
they \i{may} be regarded integral units rather than
just aggregates that get pulled apart in semantic interpretation.
}
\p{Technically, perhaps, this is an illusion.  Despite their higher-level
expressiveness, \RDF{} expression languages are, perhaps,
supposed to be deemed \q{syntactic sugar}
for a more primitive listing of triples: the \i{semantics} of
\Turtle{} and \NThree{} are conceived to be defined by translating
expressions down to the triple-sets that they logically imply
(see also \cite{YurickWilks}).
This intention accepts the paradigm that providing semantics
for a formal language is closely related to defining which
propositions are logically entailed by its statements.
}
\p{There is, however, a divergent tradition in formal semantics that is oriented to
type theory more than logic.  It is consistent with this alternative
approach to see a different semantics for a language like \Turtle{},
where larger-scale aggregates become \q{first class} values.
So, \NathFF{} can be seen as a (single, integral)
\i{value} whose \i{type} is a \nameAge{} pair.  Such a value has an
\q{internal structure} which subsumes multiple data-points.  The
\RDF{} version is organized, instead, around a \i{blank node} which
ties together disparate data points, such as my name and my age.
This blank node is also connected to another blank node which
ties together place and party.  The blank nodes
play an organizational role, since nodes are grouped together
insofar as they connect to the same blank node.  But the
implied organization is less strictly entailed; one might
assume that the \BrookDem{} nodes could just as readily
be attached individually to the \q{name/age} blank
(i.e., I live in Brooklyn, \i{and} I vote Democratic).
}
\p{Why, that is, are Brooklyn and Democratic grouped together?
What concept does this fusion model?
There is a presumptive rationale for the name/age blank
(i.e., the fusing name/age by joining them to a blank
node rather than allowing them to take edges independently):
conceivably there are multiple \FourtySix{}-year-olds named Nathaniel,
so \i{that} blank node plays a key semantic role
(analogous to the quantifier in \q{\i{There is} a Nathaniel,
age \FourtySix{}...}); it provides an unambiguous nexus so that
further predicates can be attached to \i{one specific}
\FourtySix{}-year-old Nathaniel rather than any old \NathFF{}.  But there is no
similarly suggested semantic role for the \q{place/party} grouping.  The name
cannot logically be teased apart from the name/age blank (because there
are multiple Nathaniels); but there seems to be no \i{logical}
significance to the \mbox{place/party} grouping.  Yet pairing
these values \i{can} be motivated by a modeling convention
\mdash{} reflecting that geographic and party affiliation data
are grouped together in a data set or data model.  The logical
semantics of \RDF{} make it harder to express these kinds of modeling
assumptions that are driven by convention more than logic
\mdash{} an abstracting from data's modeling environment that can be desirable
in some contexts but not in others.
}
\p{So, why does the Semantic Web community effectively insist on
a semantic interpretation of \Turtle{} and \NThree{} as \i{just} a
notational convenience for \NTrips{} rather than as higher-level
languages with a different higher-level semantics \mdash{} and despite
statements like the above Tim Berners-Lee quote insinuating that
an alternative interpretation has been contemplated even by
those at the heart of Semantic Web specifications?  
Moreover, defining hierarchies of material composition or 
structural organization \mdash{} and so by extension, 
potentially, distinct scales of modeling resolution \mdash{} 
has been identified as an intrinsic part of domain-specific 
Ontology design (see \cite{Aranda}, \cite{BittnerSmithDonnelly},
\cite{BittnerSmith}, \cite{MaureenDonnelly},
\cite{Fabrikant}, \cite{PetitotSmith}, \cite{SegevGal},
\cite{BarrySmithBlood},
or \cite{PietroRamellini}).
Semantic Web advocates have not however promoted multitier 
structure as a feature \i{of} Semantic models fundamentally, 
as opposed to criteriology \i{within} specific Ontologies.        
To the degree that this has an explanation, it probably has something to
do with reasoning engines: the tools that evaluate \SPARQL{} queries
operate on a triplestore basis.  So the \q{reductive} semantic
interpretation is arguably justified via the warrant that the
definitive criteria for Semantic Web representations are not their
conceptual elegance \visavis{} human judgments but their utility in
cross-Ontology and cross-context inferences.
}
\p{As a counter-argument,
however, note that many inference engines in Constraint Solving,
Computer Vision, and so forth, rely on specialized algorithms
and cannot be reduced to a canonical query format.  Libraries such
as \GeCODE{} and \ITK{} are important because problem-solving
in many domains demands fine-tuned application-level engineering.
We can think of these libraries as supporting \i{special} or
domain-specific reasoning engines, often built for specific
projects, whereas \OWL{}-based reasoners like \FactPP{} are
\i{general} engines that work on general-purpose \RDF{} data
without further qualification.  In order to
apply \q{special} reasoners to \RDF{}, a contingent of nodes must
be selected which are consistent with reasoners' runtime requirements.
}
\p{Of course, special reasoners cannot be expected to run on the domain of
the entire Semantic Web, or even on \q{very large} data sets in general.
A typical analysis will subdivide its problem into smaller parts
that are each tractable to custom reasoners \mdash{} in radiology, say,
a diagnosis may proceed by first selecting a medical
image series and then performing
image-by-image segmentation.  Applied to \RDF{}, this
two-step process can be considered a combination of general and special
reasoners: a general language like \SPARQL{} filters many nodes down to a smaller
subset, which are then mapped/deserialized to domain-specific representations
(including runtime memory).  For example, \RDF{} can link a patient to a
diagnostic test, ordered on a particular date by a particular doctor, whose
results can be obtained as a suite of images \mdash{} thereby selecting the
particular series relevant for a diagnostic task.  General reasoners
can \i{find} the images of interest and then pass them to
special reasoners (such as segmentation algorithms)
to analyze.  Insofar as this architecture is in effect, Semantic Web
data is a site for many kinds of reasoning engines.  Some of these engines
need to operate by transforming \RDF{} data and resources to an optimized,
internal representation.  Moreover, the semantics of these representations
will typically be closer to a high-level \NThree{} semantics taken as
\suigeneris{}, rather than as interpreted reductively as a notational
convenience for lower-level formats like \NTrip{}.  This appears
to undermine the justification for reductive semantics in terms of
\OWL{} reasoners.
}
\p{Perhaps the most accurate paradigm is that Semantic Web data has two
different interpretations, differing in being consistent with
special and general semantics, respectively.  It makes sense to
label these the \q{special semantic interpretation} or
\q{semantic interpretation for special-purpose reasoners}
(\SSI{}, maybe) and the \q{general semantic interpretation}
(\GSI{}), respectively.  Both these interpretations should be deemed
to have a role in the \q{semantics} of the Semantic Web.
}
\p{Another order of considerations involve the
semantics of \RDF{} nodes and \CH{} hypernodes
particularly with respect to uniqueness.  Nodes in \RDF{} fall into three classes:
blank nodes; nodes with values from a small set of basic types like strings and
integers; and nodes with \URL{}s which are understood to be unique across the
entire World Wide Web.  There are no blank nodes in \CH{}; and intrinsically
no \URL{}s either, although one can certainly define a \URL{} \i{type}.
There is nothing in the semantics of \URL{}s
which guarantees that each \URL{} designates a distinct internet resource;
this is just a convention which essentially, \i{de facto},
fulfills itself because
it structures a web of commercial and legal practices, not just digital
ones; e.g. ownership is uniquely granted for each internet domain name.
In \CH{}, a data type may be structured to reflect institutional
practices which guarantee the uniqueness of values in some context:
books have unique \ISBN{} codes; places have distinct \GIS{} locations,
etc.  These uniqueness requirements, however, are not intrinsically
part of \CH{}, and need to be expressed with additional axioms.  In
general, a \CH{} hypernode is a tuple of relatively simple values
and any additional semantics are determined by type
definitions (it may be useful to see \CH{} hypernodes as roughly analogous to
\CStruct{}s \mdash{} which have no \i{a priori} uniqueness mechanism).
}
\p{Also, \RDF{} types are less intrinsic to \RDF{} semantics than in \CH{}
(see \cite{HeikoPaulheim}).  The
foundational elements of \CH{} are value-tuples (via nodes expressing values,
whose tuples in turn are hypernodes).  Tuples are indexed by position, not by
labels: the tuple \NathFF{} does not in itself draw in the labels \q{name} or
\q{age}, which instead are defined at the type-level (insofar as type-definitions
may stipulate that the label \q{age} is an alias for the node in its
second position, etc.).  So there is no way to ascertain the semantic/conceptual
intent of hypernodes without considering both hyponode and hypernode types.  Conversely,
\RDF{} does not have actual tuples (though these can be represented as collections,
if desired); and nodes are always joined to other nodes via labeled connectors
\mdash{} there is no direct equivalent to the \CH{}
modeling unit of a hyponode being included in a hypernode
by position.
}
\p{At its core, then, \RDF{} semantics are built on the proposition that many
nodes can be declared globally unique by fiat.  This does not need to be
true of all nodes \mdash{} \RDF{} types like integers and floats are more
ethereal; the number \FourtySix{} in one graph is indistinguishable from \FourtySix{} in
another graph.  This can be formalized by saying that some nodes can be
\i{objects} but never \i{subjects}.  If such restrictions were not enforced,
then \RDF{} graphs could become in some sense overdetermined, implying
relationships by virtue of quantitative magnitudes devoid of semantic
content.  This would open the door to bizarre judgments like
\q{my age is non-prime} or \q{I am older than Mohamed Salah's
2018 goal totals}.
One way to block these inferences is to prevent nodes like
\q{the number \FourtySix{}} from being subjects as well as objects.
But nodes which are not primitive values \mdash{} ones, say, designating
Mohamed Salah himself rather than his goal totals \mdash{} are justifiably
globally unique, since we have compelling reasons to adopt a model
where there is exactly one thing which is \i{that} Mohamed Salah.
So \RDF{} semantics basically marries some primitive types which are
objects but never subjects with a web of globally unique but internally
unstructured values which can be either subject or object.
}
\p{In \CH{} the \q{primitive} types are effectively hypotypes; hyponodes
are (at least indirectly)
analogous to object-only \RDF{} nodes insofar as
they can only be represented via inclusion inside
hypernodes.  But \CH{} hypernodes are neither (in themselves) globally
unique nor lacking in internal structure.  In essence, an \RDF{}
semantics based on guaranteed uniqueness for atom-like
primitives is replaced by a semantics based on structured building-blocks
without guaranteed uniqueness.  This alternative may be considered in the
context of general versus special reasoners: since general reasoners
potentially take the entire Semantic Web as their domain, global
uniqueness is a more desired property than internal structure.
However, since special reasoners only run on specially selected data,
global uniqueness is less important than efficient mapping
to domain-specific representations.  It is not computationally
optimal to deserialize data by running \SPARQL{} queries.
}
\p{Finally, as a last point in the comparison between
\RDF{} and \CH{} semantics,
it is worth considering the distinction 
between \q{declarative knowledge} and \q{procedural knowledge} 
(see e.g. 
\cite[pages 182-197]{BenGoetzel}).  According
to this distinction, canonical \RDF{} data exemplifies \i{declarative} knowledge
because it asserts apparent facts without explicitly trying to interpret or
process them.  Declarative knowledge circulates among software in canonical,
reusable data formats, allowing individual components to use or make inferences from
data according to their own purposes.
}
\p{Counter to this paradigm, return to
hypothetical \USH{} examples, such as 
the conversion of Voltage data to acceleration data, which is a
prerequisite to accelerometers' readings being useful in most contexts.  Software
possessing capabilities to process accelerometers therefore reveals
what can be called \i{procedural} knowledge, because software
so characterized not only receives data
but also processes such data in standardized ways.
}
\p{The declarative/procedural distinction perhaps fails to capture how
procedural transformations may be understood as intrinsic to some semantic
domains \mdash{} so that even the information we perceive as
\q{declarative} has a procedural element.  For example, the
very fact that \q{accelerometers} are not
called \q{Voltmeters} (which are something else) suggests how the
Ubiquitous Computing community perceives voltage-to-acceleration
calculations as intrinsic to accelerometers' data.  But strictly speaking
the components which participate in \USH{} networks are not just
engaged in data sharing; they are functioning parts of the network because
they can perform several widely-recognized
computations which are understood to be central to the relevant
domain \mdash{} in other words, they have (and share with their
peers) a certain \q{procedural knowledge}.
}
\p{\RDF{} is structured as if static data sharing were the sole arbiter of
semantically informed interactions between different components,
which may have a variety of designs and rationales \mdash{} which
is to say, a
Semantic Web.  But a thorough account of formal communication semantics
has to reckon with how semantic models are informed by the implicit, sometimes
unconscious assumption that producers and/or consumers of data will
have certain operational capacities: the dynamic processes anticipated as
part of sharing data are hard to conceptually separate from the static
data which is literally transferred.  To continue the accelerometer
example, designers can
think of such instruments as \q{measuring acceleration} even though
\i{physically} this is not strictly true; their
output must be mathematically transformed for it to be interpreted in
these terms.  Whether represented via \RDF{} graphs or Directed Hypergraphs,
the semantics of shared data is incomplete unless the operations
which may accompany sending and receiving data are recognized as
preconditions for legitimate semantic alignment.
}
\p{While Ontologies are valuable for coordinating and integrating
disparate semantic models, the Semantic Web has perhaps influenced
engineers to conceive of semantically informed data sharing
as mostly a matter of presenting static data conformant to published
Ontologies (i.e., alignment of \q{declarative knowledge}).  In reality,
robust data sharing also needs an \q{alignment of \i{procedural}
knowledge}: in an ideal Semantic Network, procedural capabilities
are circled among components, promoting an emergent \q{collective
procedural knowledge} driven by transparency about code and libraries as
well as about data and formats.  The \CH{} model arguably supports this
possibility because it makes type assertions fundamental to semantics.
Rigorous typing both lays a foundation for procedural alignment
and mandates that procedural capabilities be factored in to assessments
of network components, because a type attribution has no meaning
without adequate libraries and code to construct and interpret
type-specific values.
}
\thindecoline{}
\p{Despite their differences, the Semantic Web, on the one hand, 
and Hypergraph-based frameworks,
on the other, both belong to the overall
space of graph-oriented semantic models.
Hypergraphs can be emulated in \RDF{}, 
and \RDF{} graphs can be organically mapped to 
a Hypegraph representation (insofar as 
Directed Hypegraphs with annotations are a 
proper superspace of Directed Labeled Graphs). 
Semantic Web Ontologies for computer source code can 
thus be modeled by suitably typed \DH{}s as well, 
even while we can also formulate Hypergraph-based 
Source Code Ontologies as well.  So, we are justified 
in assuming that a sufficient Ontology exists 
for most or all programming languages. 
This means that, for any given
procedure, we can assume that there is a 
corresponding \DH{} representation
which embodies that procedure's implementation.
}
\p{\phantomsection\label{detachedeval}
Procedures, of course, depend on \i{inputs} which are fixed for 
each call, and produce \q{outputs} once they terminate.  
In the context of a graph-representation, this implies that some
hypernodes represent and/or express values that are \i{inputs}, while 
others represent and/or express its \i{outputs}.  These
hypernodes are \i{abstract} in the sense (as in Lambda Calculus) that they
do not have a specific assigned value within the body, \i{qua} formal
structure.  Instead, a \i{runtime manifestation} of a \DH{}
(or equivalently a \CH{}, once channelized types are introduced) populates
the abstract hypernodes with concrete values, which in turn allows
expressions described by the \CH{} to be evaluated.
}
\p{These points suggest a strategy for unifying 
Lambda Calculi with Source Code Ontologies.
The essential construct in \mOldLambda{}-calculi is
that mathematical formulae include 
\q{free symbols} which are \i{abstracted}: sites 
where a formula can give rise to a concrete 
value, by supplying values to unknowns; or 
give rise to new formulae, via nested expressions.  
Analogously, nodes in a graph-based source-code 
representation are effectively \mOldLambda{}-abstracted 
if they model input parameters, which are 
given concrete values when the procedure runs.  
Connecting the output of one procedure to the 
input of another \mdash{} which can be modeled as a 
graph operation, linking two nodes \mdash{} is then 
a graph-based analog to embedding a complex expression 
into a formula (via a free symbol in latter).    
}
\p{Carrying this analogy further, I earlier mentioned different 
\mOldLambda{}-Calculus extensions inspired by programming-language 
features such as Object-Orientation, exceptions, and 
by-reference or by-value captures.  
These, too, can be incorporated into a Source Code Ontology: 
e.g., the connection between a node holding a value passed to an 
input parameter node, in a procedure signature, is semantically 
distinct from the nodes holding \q{Objects} which are 
senders and receivers for \q{messages}, in Object-Oriented 
Parlance.  Variant input/output protocols, including 
Objects, captures, and exceptions, are certainly semantic 
constructs (in the computer-code domain) which 
Source Code Ontologies should recognize.  So we can see 
a convergence in the modeling of multifarious input/output protocols 
via \mOldLambda{}-Calculus and via Source Code Ontologies.  
I will now discuss a corresponding expansion in the 
realm of applied Type Theory, with the goal of 
ultimately folding type theory into this convergence as well.   
}
\vspace{-.1em}
\subsectiontwoline{Procedural Input/Output Protocols via Type Theory}
\p{\label{types}Parallel to the historical evolution where \mOldLambda{}-Calculus
progressively diversified and re-oriented toward concrete
programming languages, there has been an analogous (and
to some extent overlapping) history in Type Theory.
When there are multiple ways of passing input to a
function, there are at potentially multiple kinds
of function types.  For instance, Object-Orientation inspired
expanded \mOldLambda{}-calculi that distinguish function
inputs which are \q{method receivers} or \q{\this{} objects} from
ordinary (\q{lambda}) inputs.  Simultaneously, Object-Orientation also
distinguishes \q{class} from \q{value} types
and between function-types which are \q{methods} versus ordinary
functions.  So, to take one example, a function telling
us the size of a list can exhibit two different types, depending
on whether the list itself is passed in as a method-call target
(\listsize{} vs. \sizelist{}).
}
\p{One way to systematize the diversity of type systems
is to assume that, for any particular type system, there
is a category \tCat{} of types conformant to that system.  This requires
modeling important type-related concepts as \q{morphisms} or maps
between types.  Another useful concept is an \q{endofunctor}:
an \q{operator} which maps elements in a category
to other (or sometimes the same) elements.  In a \tCat{} an endofunctor
selects (or constructs) a type \tyTwo{} from a type \tyOne{} \mdash{} note how this is
different from a morphism which maps \i{values of} \tyOne{} to \tyTwo{}.
Type systems are then built up from a smaller set of \q{core} types via
operations like products, sums, enumerations, and forming \q{function-like} types.
}
\p{We may think of the
\q{core} types for practical programming as number-based
(booleans, bytes, and larger integer types), with everything else built up by aggregation
or encodings (like \ascii{} and \unicode{}, allowing types to include text and 
alphabets; or pixel-coordinates and colors, 
allowing for graphical/visual components).\footnote{In other contexts, however, non-mathematical core types may be appropriate: for example,
the grammar of natural languages can be modeled in terms of a type system whose core are
the two types \tyNoun{} and \tyProposition{} and which also includes
function types (maps) between pairs or tuples of types (verbs,
say, map \tyNoun{}s \mdash{} maybe multiple nouns, e.g. direct objects
\mdash{} to \tyProposition{}s).
}  Ultimately, a type system \tCat{} is characterized
(1) by which are its core types and
(2) by how aggregate types are built from simpler ones
(which essentially involves endofunctors and/or products).
}
\p{In Category Theory, a Category \cCat{} is called \q{Cartesian Closed} if
for every pair of elements \eOne{} and \eTwo{} in \cCat{} there is an
element \eOneToeTwo{} representing (for some relevant notion of
\q{function}) all functions from \eOne{} to \eTwo{} \cite{RBrown}.  The stipulation that
a type system \TyS{} include function-like types is roughly equivalent, then,
to the requirement that \TyS{}, seen as a Category, is Cartesian-Closed.
The historical basis for this concept (suggested by the terminology)
is that the construction to form function-types is an \q{operator}, something that
creates new types out of old.  A type system
\TyS{} may then be \q{closed} under products:
if \tOne{} and \tTwo{} are in \TyS{} then \tOneTimesTTwo{} must be as well.
Analogously, \TyS{} supports function-like types if it 
is closed under a kind of \q{functionalization} operator \mdash{} if the
\tOneTimesTTwo{} product can be mapped onto a function-like type
\tyOneTotyTwo{}.
}
\p{In general, more sophisticated type systems \TyS{} are described by
identifying new kinds of inter-type operators and studying those
type systems which are closed under these operators: if \tyOne{} and
\tyTwo{} are in \TyS{} then so is the combination of \tyOne{} and
\tyTwo{}, where the meaning of \q{combination} depends on the
operator being introduced.  Expanded \mOldLambda{}-calculi \mdash{} which
define new ways of creating functions \mdash{} are correlated with new
type systems, insofar as \q{new ways of creating functions}
also means \q{new ways of combining types into function-like types}.
}
\p{Furthermore, \q{expanded} \mOldLambda{}-calculi generally involve
\q{new kinds of abstraction}: new ways that the building-blocks
of functional expressions, whether these be mathematical formulae
or bodies of computer code, can be \q{abstracted}, treated as
inputs or outputs rather than as fixed values.  In this chapter, I attempt to
make the notion of \q{abstraction} rigorous by analyzing it against
the background of \DH{}s that formally model computer code.
So, given the correlations I have just described between
\mOldLambda{}-calculi and type systems \mdash{} specifically, on
\TyS{}-closure stipulations \mdash{} there are parallel correlations
between type systems and \i{kinds of abstraction defined on
Channelized Hypergraphs}.  I will now discuss this further.
}
\subsubsection{Kinds of Abstraction}
\p{The \q{abstracted} nodes in a \CH{} are loosely classifiable as
\q{input} and \q{output}, but in practice there are various paradigms
for passing values into and out of functions, each with their own semantics.
For example, a \q{\this{}} symbol in \Cpp{} is an abstracted, \q{input}
hypernode with special treatment in terms of overload resolution and access
controls.  Similarly, exiting a function via \returnct{} presents
different semantics than exiting via \throw{}.  As mentioned earlier,
some of this variation in semantics has been formally modeled
by different extensions to \mOldLambda{}-Calculus.
}
\p{So, different hypernodes in a \CH{} are subject to different kinds of abstraction.
Speaking rather informally, hypernodes can be grouped into \i{channels} based on
the semantics of their kind of abstraction.  More precisely,
channels are defined initially on \i{symbols}, which are associated with hypernodes:
in any \q{body} (i.e., an \q{implementation graph}) hypernodes can be grouped
together by sharing the same symbol, and correlatively sharing the
same value during a \q{runtime manifestation} of the \CH{}.  Therefore,
the \q{channels of abstraction} at work in a procedure can be identified
by providing a name representing the \i{kind} of channel and a list of
symbols affected by that kind of abstraction.  In the notation I adopt here,
conventional lambda-abstraction like \lXY{} would be written as \CHlXY{}.
}
\p{I propose \q{Channel Algebra} as a tactic for capturing the 
semantics of channels, so as to model programming languages' 
conventions and protocols with respect to calls between 
procedures.  Once we get beyond the basic contrast between 
\q{input} and \q{output} parameters, it becomes necessary to 
define conditions on channels' size, and on how 
channels are associated with different procedures that may 
share values.  Here are several examples: 
\begin{itemize}\item{} In most Object-Oriented languages, any procedure can 
have at most one \this{} (\q{message receiver}) object.  
Let \sCh{} model a \q{Sigma} channel, as in \q{Sigma Calculus}
(written as \sigmaCalculus{}: see e.g. \cite{MartinAbadi},
\cite{CamposVasconcelos},
\cite{KathleenFisher}, \cite{EdwardZalta}, etc.).
We then have the requirement than any procedure's 
\sCh{} channel can carry at most one value.
\item{} \label{retexc} In all common languages which have exceptions, 
procedures can \i{either} throw an exception \i{or} 
return a value.  If \return{} and \exception{} 
model the channels carrying standard returns and 
thrown exceptions, respectively, this convention 
translates to a requirement that the two channels 
cannot both be non-empty.
\item{} A thrown exception cannot be handled as an ordinary 
value.  The whole point of throwing exceptions is to 
disrupt ordinary program flow, which means the exception 
value is only accessible in special constructs, like a 
\catch{} block.  One way to model this restriction is 
to forbid \exception{} channels from transferring values to 
other channels.  Instead, 
exception values are bound (in \catch{} blocks) to 
lexically-scoped symbols (I will discuss channel-to-symbol 
transfers below).
\item{} Suppose a procedure is an Object-Oriented method 
(it has a non-empty \q{\sCh{}} channel).  Any other methods 
called from that procedure will \mdash{} at least in the 
conventional Object-Oriented protocol \mdash{} automatically 
receive the enclosing method's \sigmach{} channel unless 
a different object for the called method is supplied expressly. 
\item{} \phantomsection\label{chaining} In the object-oriented 
technique known as \q{method chaining},
one procedures' \return{} channel is transferred to a 
subsequent procedures' \sCh{} channel.  The pairing of 
\return{} and \sCh{} thereupon gives rise to one 
function-composition operator.  With suitable restrictions 
(on channel size), \return{} and \lambda{} channels engender a 
different function-composition operator.  So channels can be 
used to define operators between procedures which yield new 
function-like values (i.e., instances of function-like 
types).  In some cases, function-like values defined via 
inter-function operators can be used in lieu of those 
instantiated from implemented procedures (although the 
specifics of this substitutability \mdash{} an example of 
so-called \q{eta ($\eta{}$) equivalence} \mdash{} varies by language). 
\end{itemize}
}
\p{The above examples represent possible combinations or 
interconnections (sharing values) between channels, 
together with semantic restrictions on when 
such connections are possible.  In this chapter, I 
assume that notations describing these connections and 
restrictions can be systematized into a \q{Channel Algebra}, 
and then used to model programming language conventions and 
computer code.  A basic example of inter-channel 
aggregation would be how a \lambda{} channel, combined 
with a \return{} channel, associated with one procedure, 
yields a conventional input/output pairing.  
One particular channel formation \mdash{} \lambdaPLUSreturn{}, 
say \mdash{} therefore models 
the basic \mOldLambda{}-Calculus and, simultaneously, 
a minimal definition of function-like types.
Notionally, a procedure is, in the simplest
conceptualization, the unification of an
input channel and an output channel 
\mdash{} written, say, \chaOnePluschaTwo{} 
(with the \chplus{} possibly holding extra 
stipulations, like \cChaOne{} and \cChaTwo{}
cannot both be non-empty).
So a \q{channel sum} creates the basic
foundation for a procedure, analogous to
how input and output graph elements
yield the foundations for morphisms
in Hypergraph Categories.
More complex channel combinations and protocols 
can then model more complex 
variations on \mOldLambda{}-Calculi and on 
programming language type systems. 
}
\subsubsection{Channelized Type Systems}
\p{Collectively, to summarize my discussion to this point,
I will say that formulations 
describing channel kinds, their restrictions, and
their interrelationships describe a \i{Channel Algebra}, 
which express how channels combine to
describe possible function signatures \mdash{} and 
accordingly to describe functional \i{types}.  The
purpose of a Channel Algebra is, among other things, to elucidate 
how formal languages (like programming languages) formulate functions 
and procedures, and
the rules they put in place for inputs and outputs.  If \Chi{} is a
Channel Algebra, a language adequately described by its formulations 
(channel kinds, restrictions, and interrelationships) can be called a
\Chi{}-language.  The basic \mOldLambda{}-Calculus can be described as a
\Chi{}-language for the algebra defined by a minimal 
\lambdaPLUSreturn{} combination (with \return{} channels 
restricted to at most one element).
Analogously, a type system \TyS{} is 
a \q{\Chi{}-type-system}, and is \q{closed} with respect to \Chi{}, 
if valid signatures described using channel kinds in
\Chi{} correspond to types found in \TyS{}.  Types may be less granular than
signatures: as a case in point,
functions differing in signature only by whether they
throw exceptions may or may not be deemed the same type.  But a channel
construction on types in \TyS{} must also yield a type in \TyS{}.
}
\p{I say that a type system is \i{channelized} if it is
closed with respect to some Channel Algebra.
Channelized Hypergraphs are then \DH{}s whose type system is Channelized.
We can think of channel constructions as operators which combine
groups of types into new types.
Once we assert that a \CH{} is Channelized, we know that there is a mechanism
for describing some Hypergraphs or subgraphs as \q{procedure 
implementations} some of whose hypernodes are subject to
kinds of abstraction present in the relevant Channel Algebra.  
Channel formulae and signatures describe 
source-code norms which could also be expressed via more conventional
Ontologies.  So Channel Algebra can be seen as a generalization of
(\RDF{}-environment) Source Code Ontology
(of the kinds studied for example by
\cite{ImanKeivanloo}, \cite{WernerKlieber},
\cite{JohnathanLee}, \cite{TurnerEden},
\cite{ReneWitte}, \cite{PornpitWongthongtham}).  Given the relations between
\RDF{} and Directed Hypergraphs (despite differences I have discussed here),
Channel Algebras can also be seen as adding to Ontologies governing
Directed Hypergraphs.  Such is the perspective I will take
for the remainder of this chapter.
}
\p{For a Channel Algebra \Chi{} and a \Chi{}-closed type system
(written, say) \TySChi{}, \Chi{} extends \TyS{} because function-signatures
conforming to \Chi{} become types in \TyS{}.  At the same time,
\TyS{} also extends \Chi{}, because the elements that
populate channels in \Chi{} have types within \TyS{}.  Assume that for
any type system, there is a
partner \q{Type Expression Language} (\TXL{}) which governs how type
descriptions (especially for aggregate types that do not have a
single symbol name) can be composed consistent with the logic of
the system.  The \TXL{} for a type-system \TyS{} can be
notated as \TXLTyS{}.  If \TyS{} is channelized then its
\TXL{} is also channelized \mdash{} say, \TXLTySChi{} for some \Chi{}.
}
\p{Similarly, we can then develop for Channel Algebras a \i{Channel
Expression Language}, or \CXL{}, which can indeed be integrated with
appropriate \TXL{}s.  Formal declarations of channel axioms 
\mdash{} e.g., restrictions on channel sizes, alone or in combination \mdash{} 
are examples of terms that should be representable in a \CXL{}.   
However, whereas the \CXL{} expressions I have described so far
describe the overall shape of channels
\mdash{} which channels exist in a given context and their sizes
\mdash{} \CXL{} expressions can also add details concerning the \i{types} of
values that can or do populate channels.
\CXL{} expressions with these extra specifications then become
function signatures, and as such type-expressions in the
relevant \TXL{}.  A channelized \TXL{} is then a
superset of a \CXL{}, because it adds \mdash{} to \CXL{} expressions
for function-signatures \mdash{} the stipulation that a particular
signature does describe a \i{type}; so \CXL{} expressions
become \TXL{} expressions when supplemented with a proviso
that the stated \CXL{} construction describes a
function-like type's signature.  With such a proviso, descriptions of
channels used by a function qualifies as a type attribution,
connecting function symbol-names
to expressions recognized in the \TXL{} as describing a type.
}
\p{Some \TXL{} expressions
designate function-like types, but not all, since there are many types
(\int{}, etc.) which do not have channels at all.
While a \TXL{} lies \q{above} a \CXL{} by adding provisos that
yield type-definition semantics from \CXL{} expressions,
the \TXL{} simultaneously in a sense lies \q{beneath} the
\CXL{} in that it provides expressions for the non-functional
types which in the general case are the basis for \CXL{}
expressions of functional types,
since most function parameters \mdash{} the input/output values
that populate channels \mdash{} have non-functional types.
Section \sectsym{}\hyperref[sFive]{\ref{sFive}} will discuss the elements that \q{populate}
channels (which I will call \q{carriers}) in more detail.
}
\p{In the following sections I will sketch a 
Channel Algebra that codifies the graph-based representation
of functions as procedures whose inputs and
outputs are related to other functions by variegated semantics
(semantics that can be catalogued in a Source Code
Ontology).  With this foundation, I will argue that Channel-Algebraic
type representations can usefully model higher-scale
code segments (like statements and code blocks)
within a type system, and also how type interpretations
can give a rigorous interpretation to modeling
constructs such as code specifications and
\q{gatekeeping} code.  I will start this
discussion, however, by expanding on the idea of 
employing code-graphs \mdash{} hypergraphs annotated according 
to a Source Code Ontology \mdash{} to represent 
procedure implementations, and therefore to model 
procedures as instances of function-like types.
}

\input{section3a.ngml}
\input{section3b.ngml}
\section{Modeling Procedures via Channelized Hypergraphs}
\phantomsection\label{sFour}
\p{Assuming we have a suitable Source Code Ontology, software 
procedures can be seen from two perspectives.  On the 
one hand, they are examples of well-formed code graphs: 
annotated graph structures convey the lexical symbols, 
input/output parameters (via different \q{abstractions}, 
in the sense of \mOldLambda{}-abstraction, subject to 
relevant channel protocols), and calls to other procedures, 
through which any given procedure's functionality is 
achieved.  On the other hand, we can see procedures as 
instances of function-like types, where the types carried 
in each channel determine the type of the procedure itself, 
as a functional value.  Although these two perspectives are 
usually mutually consistent, the notion of functional 
values is more general than procedures which are expressly 
implemented in computer code.  In particular, as I briefly 
mentioned earlier, sometimes functional values are denoted 
via inter-function operators (like the composition 
\fOfG{}) rather than by giving an explicit implementation.  
We can say that functions defined via operators 
(like \Ofop{}) lack a \q{function body}.  
}
\p{Going forward, I will generally use the term \i{procedure} 
with reference to function-like type instances that are 
defined \i{with} function bodies: that is, they are 
associated with sections of code that supply the procedure's 
implementation, and can be represented via code-graphs.  
I will use the term \i{function} more generally for 
instances of function-like types, irregardless of their 
provenance.  In particular, functions are \i{values} 
\mdash{} instances of types in a relevant type-system 
\TyS{} \mdash{} whereas I will not usually discuss procedures 
as \q{values}.  On the other hand, code-graphs capture 
the implementations through which function-like types 
are (mostly) populated with concrete values.   
}
\p{To model the general maxim that any coding assumptions 
made (but not verified) by one procedure \mdash{} say, \ProcOne{} \mdash{} 
should be tested by other procedures which call \ProcOne{}, we need 
a systematic outline capturing the notion of procedures calling 
other procedures, in the course of their own implementation.  
Here I propose to model these details via channels and 
interrelationships between channels.  Moreover, channels 
can be seen as structures on \i{graphs}, as well as 
runtime information flows, so that channels are applicable 
for both static and dynamic program analysis.    
}
\p{One consequence of my graph-oriented 
approach is that the technical distinctions between 
procedures and function-values (in general) have to be 
duly observed.  There are some relevant complications appertaining to 
the general picture of source-code segments instantiating 
function-like types.  I will briefly review these issues now, 
before pivoting to more macro-scale themes concerning 
Requirements Engineering via code models.
}
\vspace{-.1em}
%\spsubsection{Initializing Function-Typed Values}
\subsection{Initializing Function-Typed Values}
\p{Although in general function-typed values are \i{initialized}
from code-graphs that blueprint their implementation,
this glosses over several different mechanisms by which
function-typed values may be defined:
\begin{enumerate}\eli{}  \phantomsection\label{funconstr}In the simplest case, there is
a one-to-one relationship
between a code graph and an implemented function (\fFun{}, say).
If \fFun{} is polymorphic, in this case, it must be an example
of subtype (or \q{runtime}) polymorphism where the declared types of \fFun{}'s
parameters are actually instantiated, at runtime, by values
of their subtypes.
\eli{}  A different situation (\q{compile-time} polymorphism) applies
to generic code as in \Cpp{}
templates.  Here, a single code-graph generates multiple function
bodies, which differ only by virtue of their expected types.
For example, a templated \sortfn{} function will generate
multiple function bodies \mdash{} one for integers, say, one for strings,
etc.  These functions may be structurally similar, but they have
different signatures by virtue of working with different types.  This
means that symbols used in the function-bodies may refer to
different functions even though the symbols themselves do not vary
between function-bodies (since, after all, they come from the
same node in a single code-graph).  That is, the code-graphs
rely on symbol-overloading for function names
to achieve a kind of polymorphism, where one code-graph
yields multiple bodies.
\pseudoIndent{}
In this compile-time polymorphism,
symbols are resolved to the proper overload-implementation
at compile-time, whereas in runtime polymorphism this
decision is deferred until the runtime-polymorphic function
is actually being executed.  The key difference is that
runtime-polymorphic functions are \i{one} function-typed value,
which can work for diverse types only via subtyping \mdash{}
or via more exotic forms of indirection, like
using function-pointers in place of function symbols; whereas
compile-time-polymorphic (i.e., templated) functions are
\i{multiple} values, which share the same code-graph
representation but are otherwise unrelated.
\eli{}  \label{ops}A third possibility for producing function-like 
values is to define operators on function-like types themselves, which transform
function-like values to other function-like values, by analogy
to how arithmetic operations transform numbers to other
numbers.  As will be discussed below, this may or may not be
different from initializing function-like values via code-graphs.
For instance, given the composition operator \Ofop{},
\fDotOfg{} may or may not be treated as only a convenient
shorthand for a code graph spelling out something like \fgx{}.
\eli{}  \label{Curry}Finally, as a special case of operators on function-typed values,
one function may be obtained from another by \q{Currying}, that is,
fixing the value of one or more of the original function's
arguments.  For example, the \inc{} (\q{increment}) function which adds
\litOne{} to a value is a special case of addition, where the added value
is always \litOne{}.  Here again, Currying may or may not be
treated as a function-value-initialization process different from ones
starting from code-graphs.
\end{enumerate}
}
\p{The differences between how languages may process the \i{initialization}
of function-type values, which I alluded to in (\ref{ops}) and (\ref{Curry}), 
reflect differences in how function-like values are internally represented.
We \i{might} treat all initializations of these
values as via code-graphs (in practice, compiled down via an 
Abstract Syntax Tree or graph to some Intermediate Representation or byte-code).  
Suppose we have an \addFun{} function
and want to define an \inc{} function, as in \incimpl{}.  Even if a language has
a special Currying notation, that notation could translate behind-the-scenes to
an explicit function body, like the code at the end of the last sentence.
Alternatively, however, a language engine may also note that \inc{} is derived from \addFun{}
and can be wholly described by a handle denoting \addFun{}
(a pointer, say) along with a designation of the fixed value: in
other words, \addOne{}.  Instead of initializing \inc{} from a code-graph,
the language can represent it via a two-part data structure like
\addOne{} \mdash{} but only if the language \i{can} represent
function-typed values as compound data structures.
}
\p{Let's assume a language can always represent \i{some} function-typed values,
ones that are obtained from code-graphs, via pointers to
(or some other unique identifier for) an internal
memory area where at least \i{some} compiled function bodies are stored.
The interesting question is whether \i{all}  function-typed values
are represented in this manner and, in either case, the
consequences for the semantics of functional types \mdash{} semantic
issues such as \fOfg{} composition operators and Currying
(and also, as I will argue, Dependent Types).
}
%\spsubsection{Addressability and Implementation}
\subsubsection{Addressability and Implementation}
\p{Talk about polymorphism in a language like
\Cpp{} covers several distinct language features: achieving
code reuse by templating on type symbols is internally very different
from using virtual methods calls.  The key difference \mdash{} highlighted
by the contrast between runtime- and compile-time polymorphism \mdash{} is
that there are some function implementations which actually
compile to \i{single} functions, meaning in
particular that their compiled code has a single place in memory and
that they may be invoked through function pointers.  Conversely,
what appears in written code as one function body may actually be
duplicated, somewhere in the compiler workflow, generating multiple
function-like values.  The most common cases of such duplication
are templated code as discussed above (though there are
more exotic options, e.g. via \Cpp{} macros and/or
repeated file \codeinclude{}s).  Implementations of the first sort I will
call \q{addressable}, whereas those of the second 
produce multiple addressable values.  These concepts prove to be consequential
in the abstract theory of types, although for non-obvious reasons.
}
\p{To see why, consider first that type systems are intrinsically
pluralistic: there are numerous details whereby the type system
underlying one computing environment can differ from those employed
by other environments.  So there is no single, universal
\q{Type Expression Language}.  One role of any given
\TXL{} is to model what its corresponding language
recognizes as a type, or \mdash{} better \mdash{} a
\i{potential} type.  A \TXL{} expression which designates
a (unique) type is well-formed if it unambiguously
describes a type that \i{could} exist.  Such an
expression does not, however, implement the
type on its own, or mandate that the type be implemented;
it would merely affirm that the type so designated
is implementable within the target language.
}
\p{As a concrete example, consider a type described in
English as: \q{the type inhabited by functions 
which take, as one parameter, a Unicode string, and,
as the second parameter, an unsigned integer less than the
length of the string}.  A \TXL{} version of this specification
would only be valid if the requirements thereby described
can be satisfied, in the target language, via type-checking
alone.
}
\p{For a more in-depth example, if in \Cpp{} I
assert \q{\templateTMyList{}}, it would then be consistent with
a \Cpp{}-specific \TXL{}
to describe a type as \MyListInt{} (assume this will be implemented 
as a list of integers).  However,
the type \MyListInt{} is not, without further code, actually implemented.
It is a \i{possible} type because its description conforms to a relevant
\TXL{}, but not an \i{actual} type.  If a programmer supplies
a templated implementation for \TMyList{},
then the compiler can derive a \q{specialization} of the
template for a specific \TType{} \mdash{} or the programmer can specialize
\MyList{} on \int{} (or any other chosen type) manually.  
But in either case the actualization of
\TMyList{} will depend on an implementation (either a templated implementation
that works for multiple types or a specialization for a
single type); this is separate and apart from \TMyList{} being
a valid \i{expression} denoting a \i{possible} type.
}
\p{Templates and specialization add complexity to discussions
about types, because compilers may automatically instantiate
concrete types from templated code \i{unless} programmers
supply specializations which deviate from the template.
As a result, in a local segment of a source file it may be impossible
to know whether or not the code concretizing a templated type 
is automatically generated from a template.
Another complication is that compilers may derive
\i{default implementations} of types' constructors, unless
these are coded explicitly.  Taking these two considerations
together, it can be difficult in a code base to, 
given a type, find which code-segments correspond to
that types' constructors.
}
\p{As an analytic device, here I assume that every implementable
type can be associated with a procedure I will call a
\i{co-constructor}, whose role is to wrap constructor-calls
in a readily identifiable code body.  Co-constructors are
\q{ordinary} procedures in the sense that they are 
\q{addressable}.  Specifically, addressable procedures 
have these properties:
\begin{enumerate}\item{} You can take their address (assuming we are dealing
with a language that supports function pointers in the
first place).
\item{} They have a corresponding (possibly templated) location in
source code (and therefore a code-graph).  For co-constructors,
this location can be marked as
such \mdash{} it should be straightforward to identify all co-constructor
implementations in a code base.
\item{} They can be exposed to scripting engines and
runtime reflection; so co-constructors enable type-instances
to be created via scripts and other
runtime-introspection capabilities.
\end{enumerate}
Operationally, co-constructors are similar to
\i{factory procedures} or \i{object factories}
(see e.g. \cite[esp. pages 32-35]{ChochlikNaumann},
\cite[esp. pages 35-36]{JeremiahDangler},
\cite{DawidIreno}, \cite{McNattBieman}),
which similarly delegate to constructors but
can be used in contexts where constructors
cannot, e.g. where it is necessary to address
the factory through a pointer (note that in \Cpp{}
you may not take the address of an actual constructor).
}
\p{Insofar as co-constructors are \i{addressable}, they
provide an indirect mechanism for designating their
corresponding type.  I will use the term \i{preconstructor}
to mean a function-pointer holding the address of a
co-constructor, or some similar data structure which
uniquely identifies a co-constructor.  A preconstructor thereby
holds a compact value which is associated with exactly
one type.  A valid preconstructor, in particular, serves as proof
that a given type is implemented \mdash{} it confirms the
existence of at least one fully implemented constructor
for that type, indicating that the type is \i{actual} and
not just \i{potential}.
}
\p{Suppose certification
requires that the function which displays the gas level on a car's dashboard
never attempts to display a value above \litOH{} (intended to mean \q{One Hundred percent},
or completely full).  One way to ensure this specification is to declare
the function as taking a \i{type} which, by design, will only ever include
whole numbers in the range \ZeroToOneHundred{}.  Thus, a type system may support
such a type by including in its \TXL{} notation for \q{range-delimited} types,
types derived from other types by declaring a fixed range of allowed values.
A notation might be, say, \IntZToOH{}, for integers in the \ZeroToOneHundred{}
range \mdash{} or, more generally expressions like \TVOneToVTwo{}, meaning a \i{type} derived
from \TType{} but restricted to the range spanned by \VOne{} and \VTwo{} (assumed to be
values of \TType{} \mdash{} notice that a \TXL{} supporting this notation must
consequently support some notation of specific values, like numeric literals).
}
\p{However, merely describing range-delimited types' desired space of
values does not provide a full implementation specification.
What should happen if
someone tries to construct an \IntZToOH{} value with the number, say,
\litOHO{}?  What about with values taken from an external
source, like a web \API{}, where it cannot be formally
proven that the values fall in the proper range?  These
question point to implementation choices that transcend
formal designations.  This is why \TXL{} expressions
should be seen as just articulating \i{potential} types,
because bringing types into actuality will usually
call for engineering choices that transcend type
theory \i{per se}.  Once types \i{are} implemented,
co-constructors serve as tangible witness to
types' actualization, and preconstructors are
convenient proxies referring to those types.\footnote{Similar issues are sometimes addressed by a
\i{modal} type theory (cf., e.g., \cite{MurdochGabbay})
where (in one interpretation) a \i{logical}
assertion about a type may be \i{possible} but not necessary
(the modality ranging over \q{computing environments}, which
act like \q{possible worlds}).
}
}
\p{Reasoning abstractly about functions and types needs to be differentiated from
reasoning about available, implemented types (and functions defined 
on them).  Consider function pointers: what is the address of \fofg{}
if that expression is interpreted in and of itself
as evaluating to a functional value?\footnote{\label{fofgplausible}In my perspective
here, \fofg{} may be a \i{plausible} value, but it is
not an \i{actual} value without being implemented,
whether via a code graph (spelling out the equivalent of \lambdaxfgx{})
or some indirect/behavioral description (analogous to \inc{}
represented as \addOne{}).
}  This suggests
that a composition operator does not work in function-like
types quite like arithmetic operators in numeric types
(which is not unexpected insofar as functional values,
internally, are more like pointers than numbers-with-arithmetic).\footnote{Of course, languages are free to implement
functions behind the scenes to expand (say) \fofg{}, but
then \fofg{} is just syntactic sugar (even if its purpose
is not just to neaten source code, but also to inspire programmers
toward thinking of function-composition in quasi-arithmetic ways).
}  To put it differently, an \addressOf{} operator
\i{may} be available for \fofg{} if it is available for \fFun{} and
\gFun{}, but this depends on language design; it is not an
abstract property of type systems.
}
\p{A similar discussion applies to \q{Currying} \mdash{} the proposal
that types \tOnetoTwotoThree{} and  \tOnetoTwoTOThree{} are
equivalent, in that fixing one value as argument to a
binary function yields a new unary function.  Again,
since the Curried function is not necessarily implemented,
there is a \i{modal} difference between \tOnetoTwotoThree{}
and  \tOnetoTwoTOThree{}.  Languages \i{may} be engineered
to silently Curry any function on demand, but purported
\tOnetoTwotoThree{} and \tOnetoTwoTOThree{} 
equivalence is not a \i{necessary} feature of type systems.
}
\p{To the extent that both mathematical and programming concepts have a place here, we
find a certain divergence in how the word \q{function} is used.  If I say that
\q{there exists a function from \tOne{} to \tTwo{}}, where \tOne{} and \tTwo{} are
(not necessarily different) types, then this statement has two possible interpretations.
One is that, mathematically, I can assume the existence of a \tOneTotTwo{} mapping
by appeal to some sort of logic; the other is that a \tOneTotTwo{} function actually
exists in code.  This is not just a \q{metalanguage} difference projected
from how the discourse of mathematical type theory is used to different ends than discourses
about engineered programming languages, which are social as well as digital-technical
artifacts.  Instead, we can make the difference exact: when a function-value 
is keyed to a procedure, it is bound to a segment of code subject to 
analysis and to alternative representations (such as code graphs).  
}
\p{Since co-constructors are \i{addressable}, they 
cannot \mdash{} at least not within the framework I have 
discussed thus far \mdash{} be \q{temporary} 
function-values analogous to \fOfg{}.  
This means that \i{types} cannot be temporary values.  
More precisely, a type system may be constrained 
by the proposition that \i{no type can be created} 
whose co-constructors would have to be temporary 
values \mdash{} or, to put it differently, 
no type can be created whose co-constructors are 
not procedures that can be mapped to source-code 
segments (and thereby to code-graphs).   
}
\p{Notice that co-constructors then are not just 
function-like values; co-constructors have to be 
in that subspace of function-like values initialized 
via code-graphs, rather than via some quasi-arithmetic 
inter-function operator like \fOfG{}.
This then limits what we can do with 
Dependent Types, typestate, and other \q{expressive} 
type mechanisms.  I will call this the 
\q{metaconstructor} problem: insofar as co-constructors 
are function-like values, they (in principle) need 
their own constructors \mdash{} call these \q{metaconstructors}.  
We can stipulate that metaconstructors \mdash{} constructors of 
co-constructors \mdash{} have to be derived from code 
graphs (they cannot be temporary values), but 
this renders certain advanced type-theoretic 
features inaccessible to our applied type systems.  
Conversely, we can accept the idea of constructors 
being (potentially) temporary values, but this 
interferes with the idea of preconstructors being 
referential proxies for types themselves 
(unless types also are, potentially, temporary 
constructs, which creates a new set of problems).  
I will now explain this choice in greater depth.
}

\vspace{-.1em}
\subsection{Dependent Types and Co-Constructors}
\p{To see why the metaconstructor problem 
determines how extensively Dependent Types are 
supported in a type system, consider a variation 
on the range \ZeroToOneHundred{} type.  
In lieu of a fixed range, consider a procedure 
taking a (variable) \Tvar{}-range \rRan{} and a number \xSym{}
\i{which must be in that range}.  Here \xSym{}
\q{depends} on \rRan{} \mdash{} its \i{type} is \rRan{}
seen as its own \TVOneToVTwo{} type \mdash{} so 
\xSym{} can vary among many 
range-types, only being fixed at runtime.  Defining a 
\i{type} for procedures meeting those 
\i{specifications} is a classic problem of 
Dependent Type theory.
}
\p{Using the \rRan{}-type as before, the type of \fFun{}'s second parameter
would then be \Tvar{} restricted to the \rRan{} interval, but here
\rRan{} is not fixed in \fFun{}'s declaration but rather passed in to
\fFun{} as a parameter.  Unless we know \i{a priori} that only
a specific set of \rRan{}s in the first parameter will ever be encountered,
the compiler has to be prepared for \xSym{} being assigned any one 
of many different range types, depending on the \fFun{}'s first argument.  
In particular, the compiler cannot know ahead of time which 
constructor to call for \xSym{}.  More precisely, it is impossible 
for the compiler to have \i{separate} constructors for millions 
of possible range types.  Instead, the compiler must either 
\q{create} a constructor \q{on the fly} or else have 
some generic constructor which services many range-types, 
but then requires extra information to establish 
\i{which} range is desired. 
}
\p{Assuming we use co-constructors to wrap constructors, 
these two options for compiler writers correspond to 
the choice of \i{either} creating ad-hoc co-constructors 
\i{or} designing co-constructors as a 
compound data structure.  We could certainly write a function that takes a range and a value and
ensures that the value fits the range \mdash{} perhaps by throwing an
exception if not, or mapping the value to the range's closest point.
Such a function would provide common functionality for a family of
constructors each associated with a given range.  But a function (\cfFun{}, say)
providing \q{common functionality} for value constructors is not necessarily
itself a value constructor.\footnote{Here I say \q{value constructor} to clarify that I am not 
commenting on \i{type constructors}, which derive specialized 
types from generic ones.
}
To treat such a function as a
\i{real} value constructor we would have to add contextual modifiers:
\cfFun{} is a value constructor for range-type \rRan{} in the 
presence of a \Tvar{}-pair to specify \rRan{} at runtime.
The co-constructor for a range type \TrRan{} is accordingly the
\q{common functionality} base function \i{plus} \Tvar{}'s 
passed to it \mdash{} some sort of \Cfr{} compound data structure,
again by analogy to \inc{} and
\addOne{} (see footnote \ref{fofgplausible}, above).  Here again, though, 
the co-constructor is a temporary data structure, 
created on-the-fly to model the desired value constructor 
for an \xSym{} whose type (and therefore whose constructor) 
is not known until runtime.  I contend, on examples like 
these, that Dependant Typing for a type system \TyS{} is thus logically 
equivalent to the possibility of \TyS{} co-constructors 
being temporary values.   
}
\p{But value constructors (and by extension co-constructors) 
are not just any function-value: they have a privileged
status \visavis{} types, and may be invoked whenever an appropriately-typed
value is used.  Many constructors are called behind-the-scenes: 
in \Cpp{}, the standard function-call mechanism is
\q{pass by value}, wherein values are \i{copied} when passed 
between procedures; but any copy can potentially 
invoke a so-called \q{copy constructor}.  Indeed, programmers 
use certain constructors as \q{hooks} to silently 
insert logic into normal program flow (usually this is 
to make complex types behave like built-in-types from 
client code's point of view).  Allowing large type families (like one type
for each \int{} or each two-number range \rRan{} \mdash{} similar to \q{inductive typing} as
discussed by Edwin Brady in the context of the Idris language
\cite[p. 14]{EdwinBradyImpl}) \mdash{} could easily conflict with 
user-defined constructor overrides: users (meaning, in this context, 
library developers) would need not only to write their own 
(e.g., copy) constructors, but to hook into a complex 
run-time mechanism for creating constructors ad-hoc as temporary values.
Conversely, forcing co-constructors to be
addressable prohibits \q{large} type families \mdash{} like types indexed
over other (non-enumerative) types
(see e.g. \cite[p. 4]{BernardyEtAl}) \mdash{} at least as \i{actual} types.
This apparently precludes full-fledged Dependent Types, since
dependent-typed values invariably require in general some extra
contextual data \mdash{} not just a function-pointer \mdash{} to designate the
desired value constructor at the point where a value,
attributed to the relevant dependent type,
is needed.  It may be infeasible to add the requisite contextual
information at every point where a dependent-typed value has to be constructed
\mdash{} unless, perhaps, a description of the context can be packaged and
carried around with the value, sharing the value's lifetime.
}
\p{As I will now review, this analysis in the realm of 
Dependent Types carries over into \i{typestate}, 
which is another mechanism intended to model 
coding requirements via type-checkable specifications.
}

\vspace{-.1em}
\subsubsection{Dependent Types and Typestate}
\p{Typestates are finer-grained classifications than types.  
A canonical example of typestate is restricting how functions are
called which operate on files.  A single \q{file} type actually covers several
cases, including files that are open or closed, and even files that
are nonexistent \mdash{} they may be described by a path on a filesystem
which does not actually point to a file (perhaps in preparation for
creating such a file).  Instead of \i{one} type covering
each of these cases, we can envision \i{different} types for nonexistent,
closed, or open files.  With these more detailed types, constraints
like \q{don't try to create an already-existing file}
or \q{don't try to modify a closed or nonexistent file} are enforced by
type-checking.
}
\p{While this kind of gatekeeping is valuable in theory, it raises
questions in practice.  Reifying \q{cases} \mdash{} i.e., \i{typestates}
like open, closed, or nonexistent \mdash{} to distinct \i{types}
implies that a \q{file} value can go through different
types between construction and destruction.  If this is literally true, it
violates the convention that types are an intrinsic and fixed aspect of
typed values.  It is true that, as part of a type cast, values can be
reinterpreted (like treating an \int{} as a \float{}), but this
typically assumes a mathematical
overlap where one type can be considered as subsumed by a different type
for some calculation, \i{without this changing anything}:
any integer is equally a ratio with unit denominator, say.
\q{Casting} a closed file to an open one is the opposite effect,
using disjunctures between types to capture the fact that state
\i{has} changed; to capture a trajectory of states for one
value \mdash{} which must then have different types at
different times, since this is the whole point of modeling successive
states via alternations in type-attribution.
}
\p{An alternative interpretation is that the \q{trajectory} is not a
single mutated value but a chain of interrelated
values, wherein each successive value is obtained via a state-change
from its predecessor.  But a weakness of this chain-of-values
model is that it assumes only one value in the chain is currently
correct: a file can't be both open and closed, so if one value
with type \q{closed file} is succeeded by a different value with
type  \q{opened file}, the latter value will be correct only
if the file was in fact opened, and the former otherwise \mdash{} but
a compiler can't know which is which, \i{a priori}.  Or,
instead of a \q{chain} of differently-typed values we can employ a
single general \q{file} type and then \q{cast} the value to
an \q{open file} type when a function needs specifically
an \i{open} file, and so forth.   The effect in that case is to
insert the cast operator as a \q{gatekeeper} function preventing
the function receiving the casted value from getting nonconformant
input.  Again, though, the compiler cannot make any assumptions about
whether the \q{casts} will work (e.g., whether the attempt to open
a file will succeed).
}
\p{In short, typestate forces us to modify some basic assumptions about 
the relationship between types and values: either values can 
change types mid-stream, or a lexical scope can subsume a sequence 
of value \q{holders} which share the same symbol-name (and maybe the same type) 
but differ in state (some holding values unrelated to actual program 
state).  Both options upend normal programming expectations. 
This situation can be juxtaposed with the \q{metaconstructor problem}, 
i.e., how Dependent Types force a rethink on basic value-constructor 
theory.
}
\p{A good real-world example of the overlap between Dependent Types and
typestate (also grounded on file input/output) comes
from the \q{Dependent Effects} tutorial from the Idris
(programming language) documentation \cite{IdrisEffects}:
\begin{dquote}A practical use for dependent effects is in specifying resource usage
protocols and verifying that they are executed correctly.  For example,
file management follows a resource usage protocol with ... requirements
[that] can be expressed formally in [Idris] by creating a
\idrisText{FILE\_IO} effect parameterised over a file handle state,
which is either empty, open for reading, or open for writing.
In particular, consider the type of [a function
to open files]: This returns a \idrisText{Bool} which indicates
whether opening the file was successful.  The resulting state
depends on whether the
operation was successful; if so, we have a file handle open for the
stated purpose, and if not, we have no file handle.  By case analysis
on the result, we continue the protocol accordingly. ...
If we fail to follow the protocol correctly (perhaps by forgetting to
close the file, failing to check that open succeeded, or opening the
file for writing [when given a read-only file handle]) then we will
get a compile-time error.
\end{dquote}
So how does Idris mitigate the type-vs.-typestate conundrum?  Apparently
the key notion is that there is one single \tyFile{} \i{type}, but
a more fine-grained type-\i{state}; and, moreover, an
\i{effect system \q{parametrized over} these typestates}.
In other words, the \i{effect} of \tyFile{} operations is to
modify \i{typestates} (not types) of a \tyFile{} value.
Moreover, Dependent Typing ensures that functions cannot be called
sequentially in ways which \q{violate the protocol}, because
functions are prohibited from having effects that are incompatible
with the potentially affected values' current states.
This elegant syntheses of Dependent Types, typestate, and
Effectual Typing brings together three of the key
features of \q{fine-grained} or \q{very expressive} type systems.
}
\p{But the synthesis achieved by Idris relies on Dependent Typing:
typestate can be enforced because Idris functions 
{\sadded}may{\eadded}{\sgapped}can{\egapped} support
restrictions which \i{depend} on values' current typestate to
satisfy effect-requirements in a type-checking way.  In effect,
Idris requires that all possible variations in values' unfolding
typestate are handled by calling code, because otherwise the
handlers will not type-check.  An analogous tactic in 
\Cpp{} would be to provide an \q{open file} function only with a
signature that takes two callbacks, one for when the \openFn{}
succeeds and a second for when it fails (to mimic the Idris tutorial's
\q{case analysis}).  But that \Cpp{} version still requires convention
to enforce that the two callbacks behave differently: via Dependent Types
Idris can confirm that the \q{open file} callback, for example, is only
actually supplied as a callback for files that have indeed been
opened.  A better \Cpp{} approximation to this design would be
to cast files to separate types \mdash{} not only
typestates \mdash{} after all, but only when passing these values to the callback
functions (or, as I will discuss later, using a \q{passkey} 
to vouch that a callback's file argument \i{can} be thus cast). 
}
\p{In the case of Idris, Dependent Types are feasible because the final
\q{reduction} of expressions to evaluable representations occurs at
runtime.  In the language of the Idris tutorial:
\begin{dquote}In Idris, types are first class, meaning that they can be computed and
manipulated (and passed to functions) just like any other language construct.
For example, we could write a function which computes a type [and]
use this function to calculate a type anywhere that a type can be used.
For example, it can be used to calculate a return type [or]
to have varying input types.
\end{dquote}
More technically, Edwin Brady (and, here, Mat\'u\v{s} Teji\v{s}\v{c}\'ak)
elaborate that
\begin{dquote}\i{Full-spectrum} dependent types ... treat types as first-class language
constructs.  This means that types can be built by computation just like
any other value, leading to powerful techniques for generic programming.
Furthermore, it means that types can be parameterised on values,
meaning that strong, explicit, checkable relationships can be stated
between values and used to verify properties of programs at compile-time.
This expressive power brings new challenges, however, when
compiling programs. ... The challenge, in short, is to identify a
phase distinction between compile-time and run-time objects.
Traditionally, this is simple: types are compile-time only, values are
run-time, and erasure consists simply of erasing types. In a dependently
typed language, however, erasing types alone is not
enough \cite[page 1]{BradyMatus}.
\end{dquote}
To summarize, Idris works by \q{erasing} some, but not all, of the
extra contextual detail needed to ensure that dependent-typed
functions are used (i.e., called) correctly (see also \cite{Christiansen},
and \cite[page 195]{RichardEisenberg}).  This means that a lot of 
contextual detail is \i{not} erased; Idris provides machinery to
join executable code and user specifications onto
\i{types} so that they take effect whenever affected types' values are
constructed or passed to functions.
}
\p{Despite a divergent technical 
background, the net result is arguably not vastly
different from an Aspect-Oriented approach wherein constructors
and function calls are \q{pointcuts} setting anchors upon 
source locations or logical run-points, where extra code can be
added to program flow (see e.g. \cite{DineshGopalani},
\cite{KatharinaMehner}, \cite{CharlesZhang}).  Recall my contrast of
\q{internalist} and \q{externalist} paradigms, sketched at the
top of this chapter: Aspect-Oriented Programming involves
extra code added by external tools (that \q{modify} code by
\q{weaving} extra code providing extra features or gatekeeping).
Implementations like Idris pursue what often are in effect similar
ends from a more \q{internalist} angle, using the type system
to host added code and specifications without resorting to some
external tool that introduces this code in a manner orthogonal
to the language proper.  But Idris relies on Haskell 
to provide its operational environment; it is not clear
how Idris's strategies (or those of other Haskell and \ML{}-style
Dependent Type languages) for attaching runtime expressions to
type constructs would work in an imperative or
Object-Oriented environment, like \Cpp{} as a host language.
}
\subsubsection{Simulating Dependent Types with Preconstructors}
\p{Because Dependent Types and typestate recognize fine-grained 
requirements on which values may be passed to which functions, 
it might seem as if they are a logical continuation of the 
telos toward granular data modeling.  If our goal is to 
provide the most expressive data models possible within 
the bounds of computational tractability (I will return to 
this in the conclusion), we should certainly allow for 
Dependent Types and any other constructions which 
logically imply them (essentially, any formulation 
wherein types or their constructors are ad-hoc temporary 
values).    
}
\p{However, Dependent Types have the technical consequence of 
leaving pre-runtime values (or whatever construct we 
recognize as \q{holding} or \q{carrying} values) 
either \i{without} types, or with \i{different} types 
than they have at runtime.
In this case it becomes difficult to query
code \i{outside} runtime, which arguably \i{subtracts} 
expressiveness from the framework.  In short, we 
are free to explore some foundation for emulating 
Dependent Types \i{without} giving up on 
static reflection; the resulting system would 
not necessarily be expressively lesser than 
a \TyS{} with full-fledged Dependent Type support. 
}
\p{An elegant compromise might seem to be allowing each value to
have two potential type attributions \mdash{} one \q{static}
and one at runtime, potentially changing with each
call to the surrouding function-body.  After all, it is often
consequential that a value of type \tOne{} may be
coerced to, or reinterpreted as an instance of, \tTwo{}.  This
means that a \tOne{}'s specific value is consistent with also
being a \tTwo{}: it falls into the \q{space} where \tOne{} and
\tTwo{} overlap; or there is an available conversion that
creates a \tTwo{} from the \tOne{}.  However, this convertibility
is usually a factor when (using this chapter's terms)
the \tOne{} is transferred to a channel in a place for a
\tTwo{}.  Bear in mind that types are not \q{sets};
it's not as if we can regard two types as indistinguishable
within the collection of values where they can be interconverted.
In my treatment, types are manifest first and foremost as
recipes for values to be \q{handed off} between channels.
}
\p{So, a \q{second} or \q{dynamic} type for some value would only
be operational if it corresponds to the \q{static} type in
some receiving channel (this is how subtyping works).  But
then we are no closer to dealing with \q{temporary} types,
because the \q{metaconstructor} problem simply reappears
within the new channel.  This is not to rule out
the \i{receiving} context having its own duality
of static and dynamic types, where the hand-off
has to match requirements on both static-to-static
and dynamic-to-dynamic type-pairs.  In that case,
however, the dynamic types are not really
\q{second} types to which the initial values are
cast; they are more like logical preconditions which
must be satisfied at both ends of a channel-transfer.
}
\p{Indeed, type-attributions do two different things:
first they establish that some value is suitable
for transfer between procedures, but, second, they
affirm certain predicates \visavis{} that value.
With dynamically-recognized, temporarily \q{constructed}
types we are actually dealing with the second
salience: proof that values \i{can} be attributed
to some (maybe temporary and context-specific) type
establishes facts about that value.  But in this
case we are not interested in using that
second type as the infrastructure for a carrier-transfer;
we are instead trying to employ the type-attribution
\i{logically} as a transfer precondition.  Perhaps
a credible analogy is the post office only accepting boxes
within a certain size and weight (manifesting logistical
constraints in how the boxes can be handled) vs.
only accepting boxes which their sender certifies
not to contain dangerous contents (establishing contractual
rules that transcend the raw logistics of transporting packages).
Channel packages-to-complexes
{\sadded} 
(I will explain this terminology in the next section) 
{\eadded}
is an analogous \q{binary}
transporting which can be factored into an underlying
digital logistics and a more nuanced accounting of
package/complex compatibility, wherein we desire to
reject certain package/complex pairs which ordinary
type systems would allow (e.g. an index parameter on
the package side incompatible with the size of
a list parameter).  The problem is how to
achieve semantics modeling Dependent Types within
a framework that situates type theory itself
in a channel-transfer (and graph-oriented)
context: types only \q{exist} insofar as they
regulate inter-channel handoffs.  A given type
therefore only exists if there are capabilities
in the system to test package/complex
matches against the proposed type's logical posits.
}
\p{The solution I suggest to effectuate this compromise
involves using preconstructors as witnesses that a given 
value construction \i{could} be performed 
\mdash{} so that a given value (or values) is/are \i{logically consistent} 
with being construed as instance of some (perhaps ad-hoc) type, 
but are not \i{literally} assigned to that type.  The preconstructor 
can then be passed in to functions as an extra parameter, 
which when valid (e.g., when not a null function pointer) vouches 
for the co-construction it references being permissible.
That is, the preconstructor become a \q{passkey} parameter 
returned from a gatekeeper procedure and then passed 
on as evidence of the gatekeeping validation.  
}
\p{As a concrete example, suppose a procedure requires two numbers 
where the second is greater than the first (the inverse of 
the systolic-over-diastolic mandate): 
\fxy{} where \xlty{}.  How can we express the \xlty{} condition
within \fFun{}'s signature, assuming the signature can only
express semantics pertinent to \fFun{}'s type attribution?
On the face of it, we know that the desired \q{increasing}
condition is equivalent to \yVal{} having a type like
\RangeGTValx{} \mdash{} a range bounded (only) from below \mdash{} 
where this \q{\xVal{}} is the \xVal{} preceding
\yVal{} in \fFun{}'s signature.  But using such 
directly as \yVal{}'s type-attribution means that
from the perspective of \fFun{}'s own type-attribution,
\yVal{} does not have a single, fixed type; its type
varies according to the value of \xVal{}.  Here again
we encounter a \q{metaconstructor} problem: in order
for the \xlty{} condition to be modeled \i{directly} by
\yVal{}'s type-attribution we would need the
constructor for \yVal{}'s value-constructor to
be some operation that produces a temporary function-value
\mdash{} not simply the compilation of a code-graph to an 
addressable, non-temporary implementation.
}
\p{These issues go away if, instead of working with a function
taking \i{two} integers, we instead consider a function taking \i{one}
value which is a monotone-increasing pair (something like
\fmipair{}).  A type like
\MIpair{}, based on ordered pairs \xCommaY{} of \int{}s, solves the
metaconstructor problem for \yVal{} because \xVal{} and \yVal{}
are no longer distinct \fFun{} parameters with distinct value-constructors;
they are subsumed into one pair, whose own value-constructor
can check the \xlty{} condition.  The \i{requirements} for the
original (two-valued) \fFun{} may then be \i{described} as
\xVal{} and \yVal{} being convertible into a pair \prVal{}
which is an instance of \MIpair{} (so that \xlty{}).  This
\i{description} is not a \i{type}, but elevating the
description to type level can be at least approximated with
a wrapper like \fxypreqMIpairxy{}, so \fFun{} when used as
\fxy{} will silently call the \MIpair{} constructor.  This is only
approximate because it allows anomalies like \fxypreqMIpairZeroOne{},
 \mdash{} taking \MIpair{}s on anything \i{but} \xVal{} and \yVal{}
defeats the purpose \mdash{} but at least we can
approximate a Dependent Type signature with a passkey protocol 
that is not difficult to enforce via calling conventions
(client code should never call the three-argument form directly,
which could be sequestered to a file-scoped or private member function).
}
\p{Now, notice that we do not actually need the third argument; we 
just want to know that the \MIpair{} constructor \i{would} accept 
the \xCommaY{} pair.  So we can replace the \i{actual} \MIpair{} constructor 
with a \i{preconstructor} that \i{could} be used as a factory 
for \MIpair{} instances if needed, but can \i{also} serve to 
certify that a certain set of arguments (here a pair of numbers) 
meets the logical preconditions which an actual constructor 
would check off.
}
\p{For this to work, the 
\MIpair{} type would need to be implemented with a static 
procedure that returns a valid preconstructor for valid inputs
\mdash{} plus, assuming the preconstructor/co-constructor pattern I
am advocating, a co-constructor whose address would be 
the basis of the preconstructor value.  These are obviously 
not features of \Cpp{} (or any other language I know of) 
but could readily be an implementational norm for 
data types used in a safety-conscious project.
In effect, consistent use of preconstructors for 
fine-grained types is one strategy for siting 
gatekeeping code broadly throughout a code base.
}
\p{Further discussion of preconstructors is outside the scope 
of this paper, but concrete examples of range-checking via 
preconstructor passkeys can be found in the demo.  
Here I'll make the further point that \mdash{} if we accept a 
Channel Algebra which expands beyond present programming 
languages \mdash{} we can move preconstructor passkeys to a 
separate channel, thereby approximating Dependent Types 
more eloquently.  Co-constructors may be identified via
a dedicated co-constructing channel \mdash{} \coconch{} 
\mdash{} which signals that a return value is not 
\i{any} procedure returning the associated type, but 
a constructing procedure which is part of the 
type's interface and helps to demarcate its space 
of values.  A \coconch{} channel paired with a 
special \preconch{} channel, for preconstructor passkeys, 
provides a metamodel wherein Dependent Types, typestate, 
and many effect-systems can be reasonably encoded. 
}
\p{This last case also points to how a theory of channels 
adds semantic expressiveness to code models: 
we can achieve via descriptions of inter-procedure 
information flows \mdash{} including distinguishing distinct 
roles such as passkeys vs. ordinary parameters, 
and constructing returns vs. ordinary procedures 
happening to return a given type \mdash{} a semantic exactitude 
that is implementationally harder (from a language engineering
perspective) to achieve directly within a type system.  
Channel Algebras are not limited to channels 
actually recognized by existing languages \mdash{} they could 
be the basis for new languages, and/or new analytic tools 
isolating patterns in existing code.  With this flexibly 
channels can be lifted into a construct recognized within 
data and code modeling paradigms \mdash{} as well as an 
added structural layer within hypergraphs \mdash{} in general.  
These possibilities may become clearer as I present a 
theory of channels in more detail next section.    
}

%\section{Modeling Procedures via Channelized Hypergraphs}
\phantomsection\label{sFour}
\p{Assuming we have a suitable Source Code Ontology, software 
procedures can be seen from two perspectives.  On the 
one hand, they are examples of well-formed code graphs: 
annotated graph structures convey the lexical symbols, 
input/output parameters (via different \q{abstractions}, 
in the sense of \mOldLambda{}-abstraction, subject to 
relevant channel protocols), and calls to other procedures, 
through which any given procedure's functionality is 
achieved.  On the other hand, we can see procedures as 
instances of function-like types, where the types carried 
in each channel determine the type of the procedure itself, 
as a functional value.  Although these two perspectives are 
usually mutually consistent, the notion of functional 
values is more general than procedures which are expressly 
implemented in computer code.  In particular, as I briefly 
mentioned earlier, sometimes functional values are denoted 
via inter-function operators (like the composition 
\fOfG{}) rather than by giving an explicit implementation.  
We can say that functions defined via operators 
(like \Ofop{}) lack a \q{function body}.  
}
\p{Going forward, I will generally use the term \i{procedure} 
with reference to function-like type instances that are 
defined \i{with} function bodies: that is, they are 
associated with sections of code that supply the procedure's 
implementation, and can be represented via code-graphs.  
I will use the term \i{function} more generally for 
instances of function-like types, irregardless of their 
provenance.  In particular, functions are \i{values} 
\mdash{} instances of types in a relevant type-system 
\TyS{} \mdash{} whereas I will not usually discuss procedures 
as \q{values}.  On the other hand, code-graphs capture 
the implementations through which function-like types 
are (mostly) populated with concrete values.   
}
\p{To model the general maxim that any coding assumptions 
made (but not verified) by one procedure \mdash{} say, \ProcOne{} \mdash{} 
should be tested by other procedures which call \ProcOne{}, we need 
a systematic outline capturing the notion of procedures calling 
other procedures, in the course of their own implementation.  
Here I propose to model these details via channels and 
interrelationships between channels.  Moreover, channels 
can be seen as structures on \i{graphs}, as well as 
runtime information flows, so that channels are applicable 
for both static and dynamic program analysis.    
}
\p{One consequence of my graph-oriented 
approach is that the technical distinctions between 
procedures and function-values (in general) have to be 
duly observed.  There are some relevant complications appertaining to 
the general picture of source-code segments instantiating 
function-like types.  I will briefly review these issues now, 
before pivoting to more macro-scale themes concerning 
Requirements Engineering via code models.
}
\vspace{-.1em}
%\spsubsection{Initializing Function-Typed Values}
\subsection{Initializing Function-Typed Values}
\p{Although in general function-typed values are \i{initialized}
from code-graphs that blueprint their implementation,
this glosses over several different mechanisms by which
function-typed values may be defined:
\begin{enumerate}\eli{}  \phantomsection\label{funconstr}In the simplest case, there is
a one-to-one relationship
between a code graph and an implemented function (\fFun{}, say).
If \fFun{} is polymorphic, in this case, it must be an example
of subtype (or \q{runtime}) polymorphism where the declared types of \fFun{}'s
parameters are actually instantiated, at runtime, by values
of their subtypes.
\eli{}  A different situation (\q{compile-time} polymorphism) applies
to generic code as in \Cpp{}
templates.  Here, a single code-graph generates multiple function
bodies, which differ only by virtue of their expected types.
For example, a templated \sortfn{} function will generate
multiple function bodies \mdash{} one for integers, say, one for strings,
etc.  These functions may be structurally similar, but they have
different signatures by virtue of working with different types.  This
means that symbols used in the function-bodies may refer to
different functions even though the symbols themselves do not vary
between function-bodies (since, after all, they come from the
same node in a single code-graph).  That is, the code-graphs
rely on symbol-overloading for function names
to achieve a kind of polymorphism, where one code-graph
yields multiple bodies.
\pseudoIndent{}
In this compile-time polymorphism,
symbols are resolved to the proper overload-implementation
at compile-time, whereas in runtime polymorphism this
decision is deferred until the runtime-polymorphic function
is actually being executed.  The key difference is that
runtime-polymorphic functions are \i{one} function-typed value,
which can work for diverse types only via subtyping \mdash{}
or via more exotic forms of indirection, like
using function-pointers in place of function symbols; whereas
compile-time-polymorphic (i.e., templated) functions are
\i{multiple} values, which share the same code-graph
representation but are otherwise unrelated.
\eli{}  \label{ops}A third possibility for producing function-like 
values is to define operators on function-like types themselves, which transform
function-like values to other function-like values, by analogy
to how arithmetic operations transform numbers to other
numbers.  As will be discussed below, this may or may not be
different from initializing function-like values via code-graphs.
For instance, given the composition operator \Ofop{},
\fDotOfg{} may or may not be treated as only a convenient
shorthand for a code graph spelling out something like \fgx{}.
\eli{}  \label{Curry}Finally, as a special case of operators on function-typed values,
one function may be obtained from another by \q{Currying}, that is,
fixing the value of one or more of the original function's
arguments.  For example, the \inc{} (\q{increment}) function which adds
\litOne{} to a value is a special case of addition, where the added value
is always \litOne{}.  Here again, Currying may or may not be
treated as a function-value-initialization process different from ones
starting from code-graphs.
\end{enumerate}
}
\p{The differences between how languages may process the \i{initialization}
of function-type values, which I alluded to in (\ref{ops}) and (\ref{Curry}), 
reflect differences in how function-like values are internally represented.
We \i{might} treat all initializations of these
values as via code-graphs (in practice, compiled down via an 
Abstract Syntax Tree or graph to some Intermediate Representation or byte-code).  
Suppose we have an \addFun{} function
and want to define an \inc{} function, as in \incimpl{}.  Even if a language has
a special Currying notation, that notation could translate behind-the-scenes to
an explicit function body, like the code at the end of the last sentence.
Alternatively, however, a language engine may also note that \inc{} is derived from \addFun{}
and can be wholly described by a handle denoting \addFun{}
(a pointer, say) along with a designation of the fixed value: in
other words, \addOne{}.  Instead of initializing \inc{} from a code-graph,
the language can represent it via a two-part data structure like
\addOne{} \mdash{} but only if the language \i{can} represent
function-typed values as compound data structures.
}
\p{Let's assume a language can always represent \i{some} function-typed values,
ones that are obtained from code-graphs, via pointers to
(or some other unique identifier for) an internal
memory area where at least \i{some} compiled function bodies are stored.
The interesting question is whether \i{all}  function-typed values
are represented in this manner and, in either case, the
consequences for the semantics of functional types \mdash{} semantic
issues such as \fOfg{} composition operators and Currying
(and also, as I will argue, Dependent Types).
}
%\spsubsection{Addressability and Implementation}
\subsubsection{Addressability and Implementation}
\p{Talk about polymorphism in a language like
\Cpp{} covers several distinct language features: achieving
code reuse by templating on type symbols is internally very different
from using virtual methods calls.  The key difference \mdash{} highlighted
by the contrast between runtime- and compile-time polymorphism \mdash{} is
that there are some function implementations which actually
compile to \i{single} functions, meaning in
particular that their compiled code has a single place in memory and
that they may be invoked through function pointers.  Conversely,
what appears in written code as one function body may actually be
duplicated, somewhere in the compiler workflow, generating multiple
function-like values.  The most common cases of such duplication
are templated code as discussed above (though there are
more exotic options, e.g. via \Cpp{} macros and/or
repeated file \codeinclude{}s).  Implementations of the first sort I will
call \q{addressable}, whereas those of the second 
produce multiple addressable values.  These concepts prove to be consequential
in the abstract theory of types, although for non-obvious reasons.
}
\p{To see why, consider first that type systems are intrinsically
pluralistic: there are numerous details whereby the type system
underlying one computing environment can differ from those employed
by other environments.  So there is no single, universal
\q{Type Expression Language}.  One role of any given
\TXL{} is to model what its corresponding language
recognizes as a type, or \mdash{} better \mdash{} a
\i{potential} type.  A \TXL{} expression which designates
a (unique) type is well-formed if it unambiguously
describes a type that \i{could} exist.  Such an
expression does not, however, implement the
type on its own, or mandate that the type be implemented;
it would merely affirm that the type so designated
is implementable within the target language.
}
\p{As a concrete example, consider a type described in
English as: \q{the type inhabited by functions 
which take, as one parameter, a Unicode string, and,
as the second parameter, an unsigned integer less than the
length of the string}.  A \TXL{} version of this specification
would only be valid if the requirements thereby described
can be satisfied, in the target language, via type-checking
alone.
}
\p{For a more in-depth example, if in \Cpp{} I
assert \q{\templateTMyList{}}, it would then be consistent with
a \Cpp{}-specific \TXL{}
to describe a type as \MyListInt{} (assume this will be implemented 
as a list of integers).  However,
the type \MyListInt{} is not, without further code, actually implemented.
It is a \i{possible} type because its description conforms to a relevant
\TXL{}, but not an \i{actual} type.  If a programmer supplies
a templated implementation for \TMyList{},
then the compiler can derive a \q{specialization} of the
template for a specific \TType{} \mdash{} or the programmer can specialize
\MyList{} on \int{} (or any other chosen type) manually.  
But in either case the actualization of
\TMyList{} will depend on an implementation (either a templated implementation
that works for multiple types or a specialization for a
single type); this is separate and apart from \TMyList{} being
a valid \i{expression} denoting a \i{possible} type.
}
\p{Templates and specialization add complexity to discussions
about types, because compilers may automatically instantiate
concrete types from templated code \i{unless} programmers
supply specializations which deviate from the template.
As a result, in a local segment of a source file it may be impossible
to know whether or not the code concretizing a templated type 
is automatically generated from a template.
Another complication is that compilers may derive
\i{default implementations} of types' constructors, unless
these are coded explicitly.  Taking these two considerations
together, it can be difficult in a code base to, 
given a type, find which code-segments correspond to
that types' constructors.
}
\p{As an analytic device, here I assume that every implementable
type can be associated with a procedure I will call a
\i{co-constructor}, whose role is to wrap constructor-calls
in a readily identifiable code body.  Co-constructors are
\q{ordinary} procedures in the sense that they are 
\q{addressable}.  Specifically, addressable procedures 
have these properties:
\begin{enumerate}\item{} You can take their address (assuming we are dealing
with a language that supports function pointers in the
first place).
\item{} They have a corresponding (possibly templated) location in
source code (and therefore a code-graph).  For co-constructors,
this location can be marked as
such \mdash{} it should be straightforward to identify all co-constructor
implementations in a code base.
\item{} They can be exposed to scripting engines and
runtime reflection; so co-constructors enable type-instances
to be created via scripts and other
runtime-introspection capabilities.
\end{enumerate}
Operationally, co-constructors are similar to
\i{factory procedures} or \i{object factories}
(see e.g. \cite[esp. pages 32-35]{ChochlikNaumann},
\cite[esp. pages 35-36]{JeremiahDangler},
\cite{DawidIreno}, \cite{McNattBieman}),
which similarly delegate to constructors but
can be used in contexts where constructors
cannot, e.g. where it is necessary to address
the factory through a pointer (note that in \Cpp{}
you may not take the address of an actual constructor).
}
\p{Insofar as co-constructors are \i{addressable}, they
provide an indirect mechanism for designating their
corresponding type.  I will use the term \i{preconstructor}
to mean a function-pointer holding the address of a
co-constructor, or some similar data structure which
uniquely identifies a co-constructor.  A preconstructor thereby
holds a compact value which is associated with exactly
one type.  A valid preconstructor, in particular, serves as proof
that a given type is implemented \mdash{} it confirms the
existence of at least one fully implemented constructor
for that type, indicating that the type is \i{actual} and
not just \i{potential}.
}
\p{Suppose certification
requires that the function which displays the gas level on a car's dashboard
never attempts to display a value above \litOH{} (intended to mean \q{One Hundred percent},
or completely full).  One way to ensure this specification is to declare
the function as taking a \i{type} which, by design, will only ever include
whole numbers in the range \ZeroToOneHundred{}.  Thus, a type system may support
such a type by including in its \TXL{} notation for \q{range-delimited} types,
types derived from other types by declaring a fixed range of allowed values.
A notation might be, say, \IntZToOH{}, for integers in the \ZeroToOneHundred{}
range \mdash{} or, more generally expressions like \TVOneToVTwo{}, meaning a \i{type} derived
from \TType{} but restricted to the range spanned by \VOne{} and \VTwo{} (assumed to be
values of \TType{} \mdash{} notice that a \TXL{} supporting this notation must
consequently support some notation of specific values, like numeric literals).
}
\p{However, merely describing range-delimited types' desired space of
values does not provide a full implementation specification.
What should happen if
someone tries to construct an \IntZToOH{} value with the number, say,
\litOHO{}?  What about with values taken from an external
source, like a web \API{}, where it cannot be formally
proven that the values fall in the proper range?  These
question point to implementation choices that transcend
formal designations.  This is why \TXL{} expressions
should be seen as just articulating \i{potential} types,
because bringing types into actuality will usually
call for engineering choices that transcend type
theory \i{per se}.  Once types \i{are} implemented,
co-constructors serve as tangible witness to
types' actualization, and preconstructors are
convenient proxies referring to those types.\footnote{Similar issues are sometimes addressed by a
\i{modal} type theory (cf., e.g., \cite{MurdochGabbay})
where (in one interpretation) a \i{logical}
assertion about a type may be \i{possible} but not necessary
(the modality ranging over \q{computing environments}, which
act like \q{possible worlds}).
}
}
\p{Reasoning abstractly about functions and types needs to be differentiated from
reasoning about available, implemented types (and functions defined 
on them).  Consider function pointers: what is the address of \fofg{}
if that expression is interpreted in and of itself
as evaluating to a functional value?\footnote{\label{fofgplausible}In my perspective
here, \fofg{} may be a \i{plausible} value, but it is
not an \i{actual} value without being implemented,
whether via a code graph (spelling out the equivalent of \lambdaxfgx{})
or some indirect/behavioral description (analogous to \inc{}
represented as \addOne{}).
}  This suggests
that a composition operator does not work in function-like
types quite like arithmetic operators in numeric types
(which is not unexpected insofar as functional values,
internally, are more like pointers than numbers-with-arithmetic).\footnote{Of course, languages are free to implement
functions behind the scenes to expand (say) \fofg{}, but
then \fofg{} is just syntactic sugar (even if its purpose
is not just to neaten source code, but also to inspire programmers
toward thinking of function-composition in quasi-arithmetic ways).
}  To put it differently, an \addressOf{} operator
\i{may} be available for \fofg{} if it is available for \fFun{} and
\gFun{}, but this depends on language design; it is not an
abstract property of type systems.
}
\p{A similar discussion applies to \q{Currying} \mdash{} the proposal
that types \tOnetoTwotoThree{} and  \tOnetoTwoTOThree{} are
equivalent, in that fixing one value as argument to a
binary function yields a new unary function.  Again,
since the Curried function is not necessarily implemented,
there is a \i{modal} difference between \tOnetoTwotoThree{}
and  \tOnetoTwoTOThree{}.  Languages \i{may} be engineered
to silently Curry any function on demand, but purported
\tOnetoTwotoThree{} and \tOnetoTwoTOThree{} 
equivalence is not a \i{necessary} feature of type systems.
}
\p{To the extent that both mathematical and programming concepts have a place here, we
find a certain divergence in how the word \q{function} is used.  If I say that
\q{there exists a function from \tOne{} to \tTwo{}}, where \tOne{} and \tTwo{} are
(not necessarily different) types, then this statement has two possible interpretations.
One is that, mathematically, I can assume the existence of a \tOneTotTwo{} mapping
by appeal to some sort of logic; the other is that a \tOneTotTwo{} function actually
exists in code.  This is not just a \q{metalanguage} difference projected
from how the discourse of mathematical type theory is used to different ends than discourses
about engineered programming languages, which are social as well as digital-technical
artifacts.  Instead, we can make the difference exact: when a function-value 
is keyed to a procedure, it is bound to a segment of code subject to 
analysis and to alternative representations (such as code graphs).  
}
\p{Since co-constructors are \i{addressable}, they 
cannot \mdash{} at least not within the framework I have 
discussed thus far \mdash{} be \q{temporary} 
function-values analogous to \fOfg{}.  
This means that \i{types} cannot be temporary values.  
More precisely, a type system may be constrained 
by the proposition that \i{no type can be created} 
whose co-constructors would have to be temporary 
values \mdash{} or, to put it differently, 
no type can be created whose co-constructors are 
not procedures that can be mapped to source-code 
segments (and thereby to code-graphs).   
}
\p{Notice that co-constructors then are not just 
function-like values; co-constructors have to be 
in that subspace of function-like values initialized 
via code-graphs, rather than via some quasi-arithmetic 
inter-function operator like \fOfG{}.
This then limits what we can do with 
Dependent Types, typestate, and other \q{expressive} 
type mechanisms.  I will call this the 
\q{metaconstructor} problem: insofar as co-constructors 
are function-like values, they (in principle) need 
their own constructors \mdash{} call these \q{metaconstructors}.  
We can stipulate that metaconstructors \mdash{} constructors of 
co-constructors \mdash{} have to be derived from code 
graphs (they cannot be temporary values), but 
this renders certain advanced type-theoretic 
features inaccessible to our applied type systems.  
Conversely, we can accept the idea of constructors 
being (potentially) temporary values, but this 
interferes with the idea of preconstructors being 
referential proxies for types themselves 
(unless types also are, potentially, temporary 
constructs, which creates a new set of problems).  
I will now explain this choice in greater depth.
}

\section{Conclusion}
\p{When Conceptual Space Theory migrated from a natural-language and 
philosophical environment to a more technical and scientific 
foundation \mdash{} as a basis for modeling scientific data 
and analyzing scientific theories and theory-formation 
\mdash{} it also picked up certain evident practical applications.  
For example, \CSML{} was a concrete proposal for technical 
data modeling whose exlplicit goal was to be more 
conceptually expressive and scientifically rigorous than 
conventional \mdash{} or even \q{Semantic Web} \mdash{} data 
sharing tactics.  So one obvious domain for concrete 
applying Conceptual Space Theory lies in the 
communicating and annotating of scientific 
(and other technical research) data.  
This use-case could certainly benefit from the 
added structure of Hypergraph syntactic models 
(which can engender hypergraph serialization formats) 
and hypergraph-based type theories.
 }
\p{So a Hypergraph/Conceptual Space hybrid can readily be 
imagined as a kind of next-generation 
extension of the Semantic Web or reincarnation of 
\CSML{}, with an emphasis on sharing scientific 
data in a format conducive to capturing the 
theoretical context within which data is generated.  
This is still removed from the \i{natural language} 
origins of Conceptual Spaces, but it would mark 
a further step in the evolution of 
\Gardenfors{}'s theory from a linguistic to 
a metascientific paradigm.  
}
\p{But going even a step further, a data-sharing 
framework emerging in the scientific context 
may retroactively be utilized in a more 
humanistic context as well; so an \HCS{} hybrid 
may find applications in the conveying 
of \i{humanities} data \mdash{} natural language 
structures (parse trees or graphs, lexicons, 
and so forth), sociological/demographic 
data sets, digitized artifacts (art, 
archaeology, museum science), etc.    
In this scenario Conceptual Spaces might 
be relevant to, say, Cognitive Linguistics 
on two level \mdash{} a practical, software-oriented 
tool for linguistic research in its digital 
practice, alongside a paradigm for natural language 
semantic at the theoretical level.  These two 
modes of application may not have fully aligned 
theoretical commitments, but they would 
reveal a core Conceptual Space theory diversify, 
branching, and adapting to different practical 
and theoretical requirements. 
}


\begin{thebibliography}{99}
{\fontfamily{lmtt}\selectfont\scriptsize


\bibitem{RaubalAdams}
Benjamin Adams and Martin Raubal, 
\cq{A Metric Conceptual Space Algebra}.
\biburl{https://pdfs.semanticscholar.org/521a/cbab9658df27acd9f40bba2b9445f75d681c.pdf}

\bibitem{RaubalAdamsCSML}
Benjamin Adams and Martin Raubal, 
\cq{Conceptual Space Markup Language (CSML): Towards the Cognitive Semantic Web}.
\biburl{http://idwebhost-202-147.ethz.ch/Publications/RefConferences/ICSC_2009_AdamsRaubal_Camera-FINAL.pdf}


\bibitem{AsherPustejovsky}
Nicholas Asher and James Pustejovsky, 
\cq{A Type Composition Logic for Generative Lexicon}
\biburl{https://www.cs.brandeis.edu/~jamesp/classes/cs216-2009/readings2009/TCLforGL.pdf}


\bibitem{BarkerShanTG}
Chris Barker and Chung-Chieh Shan, 
\cq{Types as Graphs: Continuations in Type Logical Grammar}
\biburl{https://www.nyu.edu/projects/barker/barker-shan-types-as-graphs.pdf}

\bibitem{BernardyEtAl}
Jean-Philippe Bernardy, et. al.,
\cq{Parametricity and Dependent Types}.
\biburl{http://www.staff.city.ac.uk/~ross/papers/pts.pdf}

\bibitem{BittnerSmithDonnelly}
Thomas Bittner, Barry Smith, and Maureen Donnelly,
\cq{The logic of systems of granular partitions.}
\biburl{http://ontology.buffalo.edu/smith/articles/BittnerSmithDonnelly.pdf}

\bibitem{LineBrandt}
Line Brandt, 
\cq{The Communicative Mind.}
Newcastle-upon-Tyne: Cambridge Scholars Publishing, 2013

%
\bibitem{Belohlavek}
Radim B\v{e}lohl\'avek and Vladim{\i\OldI}r Sklen\'a\v{r}, 
\cq{Formal Concept Analysis Constrained by Attribute-Dependency Formulas}
B. Ganter and R. Godin, eds.,: ICFCA 2005, LNCS 3403, pp. 176-191, Berlin, Springer-Verlag, 2005.
\biburl{http://belohlavek.inf.upol.cz/publications/BeSk_Fcacadf.pdf}

%
\bibitem{PerAageBrandt}
Per Aage Brandt, 
\cq{Spaces, Domains, and Meaning: Essays in Cognitive Semiotics}.
Newcastle-upon-Tyne: Cambridge Scholars Publishing, 2013
\biburl{http://semiotics.au.dk/fileadmin/Semiotics/pdf/per-aage-brandt/Per_Aage_Brandt__Spaces__Domains__and_Meaning._Essays_in_Co.pdf}


\bibitem{ChatzikyriakidisLuo}
Stergios Chatzikyriakidis and Zhaohui Luo, 
\cq{Individuation Criteria, Dot-types and Copredication: A View from Modern Type Theories.}
\intitle{Association for Computational Linguistics, Proceedings of the 14th 
Meeting on the Mathematics of Language (MoL 14)}, pp. 39–50, 2015. 
\biburl{http://www.aclweb.org/anthology/W15-2304}

\bibitem{ChoeCharniak}
Do Kook Choe and Eugene Charniak, 
\cq{Parsing as Language Modeling}.
\tinyurl{https://aclweb.org/anthology/D16-1257}


\bibitem{InteractingConceptualSpaces}
Bob Coecke, \i{et. al.}, 
\cq{Interacting Conceptual Spaces I:
Grammatical Composition of Concepts}.
\biburl{https://arxiv.org/pdf/1703.08314.pdf}


\bibitem{Zenker}
Peter \Gardenfors{} and Frank Zenker,  
\cq{Theory Change as Dimensional Change: Conceptual Spaces 
	Applied to the Dynamics of Empirical Theories}.
\intitle{Synthese 190(6)}, pp. 1039-1058, 2013.  
\biburl{http://lup.lub.lu.se/record/1775234}

\bibitem{KennethHolmqvist}
Kenneth Holmqvist,
\cq{Conceptual Engineering: Implementing cognitive semantics}, 
in Jens Allwood and Peter \Gardenfors, eds., 
\intitle{Cognitive Semantics}, pp 153 - 171, Amsterdam,
Philadelphia: John Benjamins, 1999.

\bibitem{HolmqvistDiss}
Kenneth Holmqvist, 
\cq{Implementing Cognitive Semantics: Image schemata, 
valence accommodation, and valence suggestion for  
AI and computational linguistics}.
PhD thesis, Dept. of Cognitive Science, Lund University, Lund, Sweden, 1993.


\bibitem{Johnson}
Mark Johnson, \cq{The Body in the Mind: 
The Bodily Basis of Meaning, Imagination, and Reason}.  %Chicago, 1990
University of Chicago Press, 1990


\bibitem{KubotaLevine}
Yusuke Kubota and Robert Levine,
\cq{The syntax-semantics interface of \sq{respective} predication:
A unified analysis in Hybrid Type-Logical Categorial Grammar}
\biburl{https://www.asc.ohio-state.edu/levine.1/publications/kl-resp.pdf}


\bibitem{LakoffJohnson}
George Lakoff and Mark Johnson, 
\cq{Philosophy in the Flesh: the Embodied Mind and its Challenge to Western Thought}.
New York: Basic Books, 1999.

\bibitem{Langacker87}
Ronald Langacker, 
\cq{Nouns and Verbs}
\intitle{Language}, Vol. 63, No. 1 (March 1987) pp. 53-94.

\bibitem{LangackerFoundations}
Ronald Langacker, 
\cq{Foundations of Cognitive Grammar, vol. 1}.  
Stanford University Press, 1991

\bibitem{ZhaohuiLuo}
Zhaohui Luo,
\cq{Type-Theoretical Semantics with Coercive Subtyping}.
\biburl{https://www.cs.rhul.ac.uk/home/zhaohui/ESSLLI11notes.pdf}

\bibitem{LuoSoloviev}
Zhaohui Luo and Sergei Soloviev,
\cq{Dependent Coercions}.
\biburl{https://www.sciencedirect.com/science/article/pii/S1571066105803147}

\bibitem{MeryMootRetore}
Bruno Mery, Richard Moot, and Christian Retor\'e, \cq{
	Plurals: Individuals and sets
	in a richly typed semantics}.  
\biburl{http://arxiv.org/pdf/1401.0660.pdf} \archiveDate{3 Jan 2014}


\bibitem{EstherPascual}
Esther Pascual, 
\cq{Fictive Interaction: The conversation frame in thought, language, and discourse}.
Philadelphia, John Benjamins, 2014.

\bibitem{EfePeker}
Efe Peker,
\cq{Following 9/11: George W. Bush's Discursive Re-Articulation of American Social Identity}
Master's Dissertation, Link\"oping University, Sweden
\biburl{http://www.diva-portal.org/smash/get/diva2:21407/FULLTEXT01.pdf}

\bibitem{PetitotSyntaxe}
Jean Petitot, \cq{Syntax Topologique et Grammaire Cognitive}.
\intitle{Langages 25.103}, pp. 97-128, 1991. 

%?
\bibitem{JeanPetitot}
Jean Petitot, \cq{The morphodynamical turn of cognitive linguistics}.
\biburl{https://journals.openedition.org/signata/549}

\bibitem{SivaReddy}
Siva Reddy, \i{et. al},
\cq{Transforming Dependency Structures to Logical Forms for Semantic Parsing}
\biburl{https://aclweb.org/anthology/Q16-1010}

\bibitem{TerryRegier}
Terry Regier, 
\i{The Human Semantic Potential: 
Spatial Language and Constrained Connectionism}.
Cambridge, MA: MIT Press, 2014.

\bibitem{MattSelway}
Matt Selway, \i{et. al},
\cq{Configuring Domain Knowledge for Natural Language Understanding}
\biburl{http://ceur-ws.org/Vol-1128/paper9.pdf}


\bibitem{KiyoshiSudo}
Kiyoshi Sudo, \i{et. al},
\cq{An Improved Extraction Pattern Representation Model
for Automatic IE Pattern Acquisition}
\biburl{https://www.aclweb.org/anthology/P03-1029}


\bibitem{TanakaEtAl}
Ribeka Tanaka, et. al., \cq{Factivity and Presupposition
in Dependent Type Semantics}
\biburl{http://www.lirmm.fr/tytles/Articles/Tanaka.pdf}

\bibitem{OrlinVakarelov}
Orlin Vakarelov,
\cq{Pre-cognitive Semantic Information}.
\biburl{https://link.springer.com/article/10.1007/s12130-010-9109-5}

\bibitem{VakarelovAgent}
Orlin Vakarelov, 
\cq{The cognitive agent: Overcoming
informational limits}
\biburl{https://philarchive.org/archive/VAKTCAv1}

%\bibitem{WiegandMereology}
\bibitem{OlavKWiegand}
Olav K. Wiegand, 
\cq{A Formalism Supplementing Cognitive Semantics Based on a New Approach to Mereology}.
Ingvar Johansson, Bertin Klein and Thomas Roth-Berghofer, eds.,  
\intitle{Contributions to the Third International Workshop on Philosophy and Informatics}, 2006. 

\bibitem{WiegandGestalts}
\underlines
\cq{On referring to Gestalts}.
Mirja Hartimo, ed., \intitle{Phenomenology and Mathematics}, pp. 183-211.  Springer, 2010. 

\bibitem{Wille}
Rudolf Wille, 
\cq{Conceptual Graphs and Formal Concept Analysis}.
\intitle{Conceptual Structures: Fulfilling Peirce's Dream
Fifth International Conference on Conceptual Structures}, 1997.

\bibitem{JordanZlatev}
Jordan Zlatev, 
%\underlines
\cq{The Dependence of Language on Consciousness}.
\biburl{http://www.mrtc.mdh.se/~gdc01/work/ARTICLES/2014/4-IACAP\%202014/IACAP14-GDC/pdf/Language-Consciousness-Zlatev.pdf}


}
\end{thebibliography}

\end{document}
