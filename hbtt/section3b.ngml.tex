\spsubsection{Dependent Types and Typestate}
\p{Typestates are finer-grained classifications than types.  However, just
as it is \q{reasonable} to consider each range as its own type \mdash{} in
the sense that a coherent \TXL{} should allow range-value types, even
if in practice a language may limit how such types are actualized \mdash{}
it is also \q{reasonable} to factor typestates into types as well.
A canonical example of typestate is restricting how functions are
called which operate on files.  A single \q{file} type actually covers several
cases, including files that are open or closed, and even files that
are nonexistent \mdash{} they may be described by a path on a filesystem
which does not actually point to a file (perhaps in preparation for
creating such a file).  Instead of \i{one} type covering
each of these cases, we can envision \i{different} types for nonexistent,
closed, or open files.  With these more detailed types, constraints
like \q{don't try to create an already-existing file}
or \q{don't try to modify a closed or nonexistent file} are enforced by
type-checking.
}
\p{While this kind of gatekeeping is valuable in theory, it raises
questions in practice.  Reifying \q{cases} \mdash{} i.e., \i{typestates}
like open, closed, or nonexistent \mdash{} to distinct \i{types}
implies that a \q{file} value can go through different
types between construction and destruction.  If this is literally true, it
violates the convention that types are an intrinsic and fixed aspect of
typed values.  It is true that, as part of a type cast, values can be
reinterpreted (like treating an \int{} as a \float{}), but this
typically assumes a mathematical
overlap where one type can be considered as subsumed by a different type
for some calculation, \i{without this changing anything}:
any integer is equally a ratio with unit denominator, say.
\q{Casting} a closed file to an open one is the opposite effect,
using disjunctures between types to capture the fact that state
\i{has} changed; to capture a trajectory of states for one
value \mdash{} which must then have different types at
different times, since this is the whole point of modeling successive
states via alternations in type-attribution.
}
\p{An alternative interpretation is that the \q{trajectory} is not a
single mutated value but a chain of interrelated
values, wherein each successive value is obtained via a state-change
from its predecessor.  But a weakness of this chain-of-values
model is that it assumes only one value in the chain is currently
correct: a file can't be both open and closed, so if one value
with type \q{closed file} is succeeded by a different value with
type  \q{opened file}, the latter value will be correct only
if the file was in fact opened, and the former otherwise \mdash{} but
a compiler can't know which is which, \i{a priori}.  Or,
instead of a \q{chain} of differently-typed values we can employ a
single general \q{file} type and then \q{cast} the value to
an \q{open file} type when a function needs specifically
an \i{open} file, and so forth.   The effect in that case is to
insert the cast operator as a \q{gatekeeper} function preventing
the function receiving the casted value from receiving nonconformant
input.  Again, though, the compiler cannot make any assumptions about
whether the \q{casts} will work (e.g., whether the attempt to open
a file will succeed).
}
\p{In short, typestate forces us to modify some basic assumptions about 
the relationship between types and values: either values can 
change types mid-stream, or a lexical scope can subsume a sequence 
of carrier which share the same symbol-name (and maybe the same type) 
but differ in state (some holding values unrelated to actual program 
state).  This situation can be juxtaposed with the \q{metaconstructor problem}, 
i.e., how Dependent Types force a rethink on basic value-constructor 
theory.  As with constructors, here I will advocate for techniques 
that emulate typestate without unduly expanding the scope of 
a Channelized type theory.
}
\p{A good real-world example of the overlap between Dependent Types and
typestate (also grounded on file input/output) comes
from the \q{Dependent Effects} tutorial from the Idris
(programming language) documentation \cite{IdrisEffects}:
%http://docs.idris-lang.org/en/latest/effects/depeff.html
\begin{displayquote}A practical use for dependent effects is in specifying resource usage
protocols and verifying that they are executed correctly.  For example,
file management follows a resource usage protocol with ... requirements
[that] can be expressed formally in [Idris] by creating a
\idrisText{FILE\_IO} effect parameterised over a file handle state,
which is either empty, open for reading, or open for writing.
In particular, consider the type of [a function
to open files]: This returns a \idrisText{Bool} which indicates
whether opening the file was successful.  The resulting state
depends on whether the
operation was successful; if so, we have a file handle open for the
stated purpose, and if not, we have no file handle.  By case analysis
on the result, we continue the protocol accordingly. ...
If we fail to follow the protocol correctly (perhaps by forgetting to
close the file, failing to check that open succeeded, or opening the
file for writing [when given a read-only file handle]) then we will
get a compile-time error.
\end{displayquote}
So how does Idris mitigate the type-vs.-typestate conundrum?  Apparently
the key notion is that there is one single \tyFile{} \i{type}, but
a more fine-grained type-\i{state}; and, moreover, an
\i{effect system \q{parametrized over} these typestates}.
In other words, the \i{effect} of \tyFile{} operations is to
modify \i{typestates} (not types) of a \tyFile{} value.
Moreover, Dependent Typing ensures that functions cannot be called
sequentially in ways which \q{violate the protocol}, because
functions are prohibited from having effects that are incompatible
with the potentially affected values' current states.
This elegant syntheses of Dependent Types, typestate, and
Effectual Typing brings together three of the key
features of \q{fine-grained} or \q{very expressive} type systems.
}
\p{But the synthesis achieved by Idris relies on Dependent Typing:
typestate can be enforced because Idris functions can support
restrictions which \i{depend} on values' current typestate to
satisfy effect-requirements in a type-checking way.  In effect,
Idris requires that all possible variations in values' unfolding
typestate are handled by calling code, because otherwise the
handlers will not type-check.  An analogous tactic in a language like
\Cpp{} would be to provide an \q{open file} function only with a
signature that takes two callbacks, one for when the \openFn{}
succeeds and a second for when it fails (to mimic the Idris tutorial's
\q{case analysis}).  But that \Cpp{} version still requires convention
to enforce that the two callbacks behave differently: via Dependent Types
Idris can confirm that the \q{open file} callback, for example, is only
actually supplied as a callback for files that have actually been
opened.  A better \Cpp{} approximation to this design would be
to cast files to separate types \mdash{} not only
typestates \mdash{} after all, but only when passing these values to the callback
functions; this is similar to the approach I endorse
here to approximate Dependent Typing via (sometimes hidden)
channels rather than constructs (like typed-checked
typestate-parametrized effects) which require a language
to implement a full Dependent Type system.
}
\p{In the case of Idris, Dependent Types are feasible because the final
\q{reduction} of expressions to evaluable representations occurs at
runtime.  In the language of the Idris tutorial:
\begin{displayquote}In Idris, types are first class, meaning that they can be computed and
manipulated (and passed to functions) just like any other language construct.
For example, we could write a function which computes a type [and]
use this function to calculate a type anywhere that a type can be used.
For example, it can be used to calculate a return type [or]
to have varying input types.
\end{displayquote}
More technically, Edwin Brady (and, here, Mat\'u\v{s} Teji\v{s}\v{c}\'ak)
elaborate that
\begin{displayquote}\i{Full-spectrum} dependent types ... treat types as first-class language
constructs.  This means that types can be built by computation just like
any other value, leading to powerful techniques for generic programming.
Furthermore, it means that types can be parameterised on values,
meaning that strong, explicit, checkable relationships can be stated
between values and used to verify properties of programs at compile-time.
This expressive power brings new challenges, however, when
compiling programs. ... The challenge, in short, is to identify a
phase distinction between compile-time and run-time objects.
Traditionally, this is simple: types are compile-time only, values are
run-time, and erasure consists simply of erasing types. In a dependently
typed language, however, erasing types alone is not
enough \cite[page 1]{BradyMatus}.
\end{displayquote}
To summarize, Idris works by \q{erasing} some, but not all, of the
extra contextual detail needed to ensure that dependent-typed
functions are used (i.e., called) correctly (see also \cite{Christiansen},
and \cite[page 210]{RichardEisenberg}).  This means that much 
contextual detail is \i{not} erased; Idris provides machinery to
join executable code and user specifications onto
\i{types} so that they take effect whenever affected types' values are
constructed or passed to functions.
}
\p{Despite a divergent technical 
background, the net result is arguably not vastly
different from an Aspect-Oriented approach wherein constructors
and function calls are \q{pointcuts} setting anchors upon 
source locations or logical run-points, where extra code can be
added to program flow (see e.g. \cite{KatharinaMehner},
\cite{DineshGopalani}, \cite{CharlesZhang}).  Recall my contrast of
\q{internalist} and \q{externalist} paradigms, sketched at the
top of this chapter: Aspect-Oriented Programming involves
extra code added by external tools (that \q{modify} code by
\q{weaving} extra code providing extra features or gatekeeping).
Implementations like Idris pursue what often are in effect similar
ends from a more \q{internalist} angle, using the type system
to host added code and specifications without resorting to some
external tool that introduces this code in a manner orthogonal
to the language proper.  But Idris relies on Haskell 
to provide its operational environment; it is not clear
how Idris's strategies (or those of other Haskell and \ML{}-style
Dependent Type languages) for attaching runtime expressions to
type constructs would work in an imperative or
Object-Oriented environment, like \Cpp{} as a host language.
}
\p{This discussion emanated from Idris examples based on file-management
typestate, but it generalizes to other cases, such as range-delimited
types (or analogous requirements manifest as type-states).  
In practice, working with scenarios such as range-bounded values \mdash{} which
in principle exemplifies programming tasks where Dependent Types can be a
useful idiom \mdash{} arguably ends up more like typestate
management as exemplified by gatekeeping \visavis{}, say, files (only
call \i{this} function if \i{that} file is open).  A range-interval 
is typestate-like in that the practical intent is to affix certain gatekeeper
functions to an accompanying value so that it will never be incorrectly used
\mdash{} for instance, that it will never be modified to lie outside
its assigned range.  Conforming to range
restrictions is more like a typestate than a type: indeed, a range-plus-value pair
has an obvious covering via two typestates
(the value is either in or out of its closed interval).\footnote{Although a more detailed range typestate might have a few
more enumeration values: representing \q{on the border} or
distinguishing \q{too large} from \q{too small}.
}  So the semantics of Dependent Types can practically be
captured via a typestate framework \mdash{} and perhaps vice-versa, since
typestates can be seen as type families indexed over (possibly enumerative)
types; e.g., file typestates as a family indexed over \enumleft{}\closed{},
\open{}, \nonexistent{}\enumright{}.
}
\p{Adding validation code at runtime \mdash{} to dynamically
enforce typestate constraints \mdash{} fits the profile of Aspect-Oriented
Programming more than type-system expressiveness, however.  I propose
therefore to outline a framework which in its engineering works effectively
by code-insertion but expresses a more type-theoretic orientation.
To begin, recall that the implementational problem with Dependent Types
is maintaining entire families of value constructors; but we can
perform computations to assess whether a value meeting stated criteria
\i{could} be constructed without actually constructing the value.  In
this spirit, assume that value constructors can potentially be
associated with \i{preconstructors} which run before the constructor proper
and assess whether the proposed construction is reasonable.  These
preconstructors may be binary-valued but may also have a larger result
type, such as a typestate enumeration.  Given a range-bounded type, for example,
the preconstructor can classify a value as \i{in range}, \i{too small}, or
\i{too large}, returning an enumerated value which the actual constructor
then uses to refine its behavior.  The key point about preconstructors
is that they can be used even without a value constructor present to
receive its value: instead, the preconstructor can test that a value
of a narrowly defined type \i{could} be constructed, even if the
actual value used belongs to a broader type.
}
\p{Similar to preconstructors, I will also introduce a
concept of \i{co-channels}, which are \q{behind the
scenes} channels that create values from functions'
channels, but whose values are passed to functions
implicitly; they are not visible at function-call sites.
I will return to the discussion of Preconstructors and
co-channels after developing a theory of Channel Complexes
more substantially.
}
