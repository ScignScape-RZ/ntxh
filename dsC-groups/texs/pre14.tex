\atsp
\begin{frame}{\ft{A Linguistics Annotation System}}
\vspace{-13em}
\hspace*{6pt}\OneQuad
{
\begin{quadblock}{Tools to Facilitate Annotating Linguistic Corpora}
\hspace{1cm}{\parbox{19cm}{\LARGE
\fontseries{b}\selectfont
{The final three screenshots show an example of how a 
custom-signd application can 
facilotat the task of building 
an annotated corpus from a linguistics text.  
The components demonstrated here enable 
several strategies (which can be combined) for dscribing 
parsing structures and the logical 
composition of language samples:
\vspace{1em} 
\begin{description}
\item[S-Expressions] Representing linguistic units as 
semantic and syntactic transformations 
triggered by words assigned to \curlyquote{functional} 
types.
\item[Deepndency Grammar] Representing phrase structures viabinter-word 
syntactic relationships.
\item[Link Grammar] Representing linguistic structure via connectors 
internal to each word-sense.  Inter-word links are activatd when 
each word in the pair has a connector compatible with the 
other word's connector.  Intuitively, a connctor represents 
how one word's meaning or grammatic contribution can 
be \curlyquote{completed} by linking to a separate word.
\end{description}
}
}}
\end{quadblock}
}
\end{frame}
