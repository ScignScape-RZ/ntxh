

\begin{frame}{\ft{Getting Information About Modeling Parameters}}
\vspace{-13em}
\OneQuad
{
\begin{quadblock}{Using Dataset Applications as Pedagogical Tools}
\hspace{1cm}{\parbox{19cm}{\LARGE
\fontseries{b}\selectfont
{
In addition to interactive visualization, 
Dataset Applications are useful tools for understanding 
experimental protocols and research 
methods.  Within Dataset Applications, 
modeling units such as statistical parameters 
and record fields are visible in situ within a GUI 
--- identified by labels, buttons, and other interactive 
micro-controls.  As a result, users encounter 
modeling elements in a structured visual-intractive context.  
To learn more about modeling elements, Dataset Applications 
are equipped with several pedagogical features shown on the 
following screenshots:
\vspace{1em} 


\begin{description}
\item[\curlyquote{About} Dialogs] Brief summaries of research trms and parameters.
\item[XPDF Links] Links back to research articles read in an embedded PDF viewer.
\item[XPDF Enhancements] The XPDF viewer can be customized for each data set 
and included with dataset code, with extra features to integrate article 
or book texts with Dataset Applications.
\end{description}
}
}}
\end{quadblock}
}
\end{frame}
