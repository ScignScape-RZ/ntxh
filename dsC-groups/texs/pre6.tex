
\begin{frame}{\ft{\parbox{22cm}{\vspace{6pt}\centering Using Dataset Applications to Understand 
Experimental Protocols and Research Methods}}}
\vspace{-11em}
\OneQuad
{
\begin{quadblock}{Obtaining Information About Modeling Parameters}
\hspace{1cm}{\parbox{19cm}{\LARGE
\fontseries{b}\selectfont
{
In addition to interactive visualization, 
Dataset Applications are useful pedagogic tools.  Within Dataset Applications, 
modeling units such as statistical parameters 
and record fields are visible in situ within a GUI 
--- identified by labels, buttons, and other interactive 
micro-controls.  As a result, users encounter 
modeling elements in a structured visual-interactive context.  
To help the reader learn more about modeling elements, 
Dataset Applications 
are equipped with several pedagogic features (which are 
shown on the following slides):
\vspace{1em} 


\begin{description}
\item[\curlyquote{About} Dialogs] Brief summaries of research terms and parameters.
\item[XPDF Links] Links back to research articles read in an embedded PDF viewer.
\item[XPDF Enhancements] The XPDF viewer can be customized for each data set 
and included with dataset code, with extra features to integrate article 
or book texts with Dataset Applications.
\end{description}
}
}}
\end{quadblock}
}
\end{frame}
