\atsp
\begin{frame}{\ft{Technological Components of Dataset Creator}}
\section{Components}

\vspace*{33pt}
{\thrulexrev}
\vspace*{-15pt}

%\MyDiamond{}
{\fontsize{18}{22}\fontfamily{ppl}\selectfont
%\begin{center}
\hspace*{25pt}\begin{minipage}{1.2\textwidth}
\vspace{12pt}


%\definecolor{blback}{RGB}{0,100,100}	
%\definecolor{blfront}{RGB}{0,100,50}

%\fcolorbox{lqboutercolor}{lqbinnercolor}{\begin{minipage}{\textwidth}%
		
%\begin{lightquadblockc}{1,0.4,0.1}{\parbox{24cm}{\hspace*{-9pt}}\vspace{8pt}}
%\begin{minipage}{1.1\textwidth}

{\setlength{\leftmargini}{9pt}\begin{enumerate}
\setlength{\labelsep}{13pt}
\dmitem \textbf{\AtR{} (Application-as-a-Resource)}: \hspace{.25em} 
\AtR{} Applications are self-contained, citable resources and tools which 
can conform to \makebox{modern} resource documentation standards, such as the Research Object protocol.  Dataset Applications can use the \AtR{} tools 
and protocol to create custom desktop-style applications 
for viewing and analyzing research data, while bundling the dataset  and application code into a citable Research Object. 
\vspace{19pt}

\dmitem \textbf{HTXN (Hypergraph Text Encoding Protocol)}:  \hspace{.25em}
HTXN is a protocol for encoding documents' character streams  
and document structure via \q{standoff annotation} (i.e.,  
character encoding is fully separate from structural representation).  
HTXN supports diverse kinds of document models, including 
\LaTeX{}, XML, RDF, and Concurrent/Overlapping XML extensions. 
\vspace{19pt}

\dmitem \textbf{MOSAIC (Multiparadigm Ontologies 
	for Scientific and Academic Publishing)}: \hspace{.25em} 
Mosaic provides data-modeling capabilities which 
reflect a diversity of Information Representation 
paradigms, such as Hypergraphs, Conceptual Spaces, 
and Object-Oriented Simulation.  \\
\vspace{19pt}\hspace{11pt}\raisebox{12pt}{\MySquare}\hspace{11pt}\parbox{19cm}{{\color[rgb]{0.3,0,0.1}{Mosaic includes 
the Mosaic/HTXN Semantic Document Infoset (MH-SDI) and 
Mosaic Plugin Framework (MPF) (see slides 19-26).}}}
\vspace{19pt}  

\end{enumerate}
}

%\end{minipage}
%\end{lightquadblockc}
\end{minipage}
}

%}
%\end{minipage}
%\end{center}
%}

\end{frame}
