\atsp
\begin{frame}{\ft{Components of Mosaic}}
\section{Mosaic}

\vspace{-3pt}
{\thrulex}

%\MyDiamond{}
{\fontsize{17}{22}\fontfamily{ppl}\selectfont
%\begin{center}
\hspace*{13pt}\begin{minipage}{.97\textwidth}
\vspace{-6pt}


%\definecolor{blback}{RGB}{0,100,100}	
%\definecolor{blfront}{RGB}{0,100,50}

%\fcolorbox{lqboutercolor}{lqbinnercolor}{\begin{minipage}{\textwidth}%
		
%\begin{lightquadblockc}{1,0.4,0.1}{\parbox{24cm}{\hspace*{-9pt}}\vspace{8pt}}
%\begin{minipage}{1.1\textwidth}

{\setlength{\leftmargini}{3pt}\begin{enumerate}
\vspace{11pt}
\dmitem \textbf{Mosaic/HTXN Semantic Document 
Infoset (MH-SDI)}:  \hspace{.25em} 
The Mosaic/HTXN Infoset is similar to an XML Infoset, 
embodying a machine-readable representation of documents' 
text, structure, and secondary resources which 
can be accessed according to different protocols 
(such as a Document Object Model).  In contrast to 
XML, the MH-SDI supports more detailed semantic 
queries against document structures, such as 
identifying sentence boundaries and matching 
multimedia assets to manuscript locations. 

\vspace{11pt}  
\dmitem \textbf{Mosaic Plugin Framework (MPF)}:  \hspace{.25em} 
The Mosaic Plugin Framework is a protocol for 
embedding plugins or extensions within document 
viewers, scientific applications, and multimedia 
software, with the plugins interoperating to implement 
multi-application networks.  In particular, 
document viewers can launch and send data to 
scientific or multimedia applications so 
that readers can access multimedia content 
embedded in publications.

\end{enumerate}
}

%\end{minipage}
%\end{lightquadblockc}
\end{minipage}
}

%}
%\end{minipage}
%\end{center}
%}

\end{frame}
