%\spsubsectiontwoline{Channels for Dependent Types and
%Larger-Scale Code Structures}
\subsectiontwoline{Channels for Dependent Types and
Larger-Scale Code Structures}
\phantomsection\label{Channelized}
\p{This chapter has focused on modeling specifications captured entirely
via a type system, so at
this point I'll return to that topic in the specific context
of Dependent Types.  Consider first a function which must receive a
signed integer no greater than \litOH{}.  This is a dependent type
in the mild sense that the type expression depends on a value:
for the range criterion to be \TXL{}-expressible we must assume that
the relevant \TXL{} includes type expressions whose
elements can be values which are not themselves types.
For instance \RangeLTEVal{} can be a type class such that
\RangeLTEValV{} is a type-expression (with \vVal{} a
value, not a type), designating types whose
inhabitants all compare \lte{} to \vVal{}.  As I have alluded to,
we do not have an \i{a priori} set-theoretic machinery
at the implementation level; our ability to express criteria
like \q{the set of unsigned \int{}s \lte{} \litOH{}} depends
on an explicit type implementation \mdash{} and, in particular, a
value constructor.  Assuming we have generic programming,
we can implement \RangeLTEValV{} via a constructor/initializer which takes
an arbitrary (hitherto unchecked) \int{} and verifies
that it falls in the range \lteVal{}.
This leaves unspecified what to do if the check fails \mdash{} throw an
exception?  Silently covert the input to \vVal{}, maybe logging a
warning?  But that choice can be left to the discretion of
the type implementation; the fixed criterion is that any carrier
purporting to hold \RangeLTEValV{} values guarantees that any
initialized value has been range-checked.
}
\p{Note that this checking may be mostly runtime: we might not actually
prohibit code like \RangeLTEOHxeqOHone{} since the value constructor
will be called with \litOHO{} (and have to decide what to do with it) only
at runtime.  Compilers can do basic arithmetic \mdash{} they can
warn, say, that in \intXeqOPointFive{} the decimal is
truncated to an integer \mdash{} but compilers for most languages
allow only primitive number-related checks; certainly they cannot
enforce axioms such as an \q{insert} operation on a list
increasing the list's size by one.
That is, enforcing
restrictions like range-checks via gatekeeper value
constructors does not realize the vision of compile-time type-checking
to sniff out anomalies, but it does serve the goal
of guaranteeing properties of values prior to
functions being called so that these do not have to be checked in
the implementation itself.
}
\p{Type classes like  \RangeLTEValV{} also have the feature that they will
only be instantiated in a handful of concrete types; they are
simpler than real dependent-type constructs such as, consider,
a possible function taking two \i{increasing} integer values:
\fxy{} where \xlty{}.  How can we express the \xlty{} condition
within \fFun{}'s signature, assuming the signature can only
express semantics pertinent to \fFun{}'s type attribution?
On the face of it, we know that the desired \q{increasing}
condition is equivalent to \yVal{} having a type like
\RangeGTValx{}, where this \q{\xVal{}} is the \xVal{} preceding
\yVal{} in \fFun{}'s signature.  But using such 
directly as \yVal{}'s type-attribution means that
from the perspective of \fFun{}'s own type-attribution,
\yVal{} does not have a single, fixed type; its type
varies according to the value of \xVal{}.  As a consequence,
the proper value-constructor for \yVal{} cannot be
known at until runtime.  Moreover, for reasons I reviewed
above in the context of \q{indexed} types, \yVal{}'s
value constructor would have to be some temporary value
(you couldn't have directly-addressable constructors
indexed over non-enumerative types like \int{}).  Here again
we encounter a \q{metaconstructor} problem: in order
for the \xlty{} condition to be modeled \i{directly} by
\yVal{}'s type-attribution we would need the
constructor for \yVal{}'s value-constructor to
be some operation that produces a temporary function-value
\mdash{} not simply the compilation of a code-graph to a
non-temporary implementation that can be
directly-addressable via a function-pointer.
}
\p{These issues go away if, instead of working with a function
taking \i{two} integers, we instead consider a function taking \i{one}
value which is a monotone-increasing pair (something like
\fmipair{}).  A type like
\MIpair{}, based on ordered pairs \xCommaY{} of \int{}s, solves the
metaconstructor problem for \yVal{} because \xVal{} and \yVal{}
are no longer distinct \fFun{} parameters with distinct value-constructors;
they are subsumed into one pair, whose own value-constructor
can check the \xlty{} condition.  The \i{requirements} for the
original (two-valued) \fFun{} may then be \i{described} as
\xVal{} and \yVal{} being convertible into a pair \prVal{}
which is an instance of \MIpair{} (so that \xlty{}).  This
\i{description} is not a \i{type}, but elevating the
description to type-level can be at least approximated with
a signature like \fxypreqMIpairxy{}, which when called as
\fxy{} will silently call the \MIpair{} constructor.  This is only
approximate because it allows anomalies like \fxypreqMIpairZeroOne{},
defeating the purpose \mdash{} how well this \q{hidden values} technique
(similar to what is sometimes called \q{passkey} parameters)
approximates dependent-typed protocols depends on how well
client code can be forbidden from calling the
three-argument form directly.
}
\p{A Channel-based solution is to introduce a \i{hidden} channel of values which
are initialized from values in the other channels (rather than passed
in from the call site).  A variation on this theme is to
construct hidden values that classify values within types
more precisely than types themselves.  Such finer classification
is a feature of \q{typestate} systems \mdash{} the distinction between
closed and open files, for instance, being a variation in typestate
within an overall \fileTy{} type.  For typical typestates, it
is possible in principle to implement types whose constructors
only accept values in the associated state \mdash{} e.g., a
constructor for an \q{open file} type that takes \fileTy{} values
but only returns a value if the passed file is open.  Despite their
greater guarantees, these narrower types may have limited
usefulness because it is presumably impossible to
know at compile-time if (say) a file is (going to be)
open.  On the other hand, creating an open-file value is a
way to gatekeep specifications that a function body
only be entered if the relevant file is open.  Placing
tests in value-constructors is a way to express them
through a vehicle more central to a type system and \TXL{}.
}
\p{Aside from typestate as exclusionary \mdash{} preventing
function bodies from running without specific criteria being
met \mdash{} typestates can also \q{overload} functions on
finer-grained criteria (relative to the type system itself).
Consider a function which works on two lists, but needs different
implementation depending on which is larger, or both equal-sized.
Functions cannot typically be overloaded based on such runtime
criteria alone, but via \q{hidden} channels we can assign them
different signatures: one takes a type whose constructor only
succeeds if it is passed two lists of increasing size; another's
hidden type only gets initialized if the sizes are \i{decreasing};
and a third only if the sizes are equal.  But notice that these three
\q{hidden} types can also be interpreted as typestates of a
pair-of-lists type; every \pairOfLists{} value can be sorted into
one of three states (size-increasing, or decreasing, or equal).
That is: apply passkey techniques like those discussed above
for a hidden \MIpair{}
argument, only to test an \q{increasing} condition not on
values but on list-sizes.
Suppose we have three list-pair functions overloaded on variants
of \MIpair{}, which collectively span the size-comparison
typestates of
a \q{pair-of-lists}.  A compiler \mdash{} or perhaps some
supplemental code-generation tool \mdash{} could plausibly default-implement
an associated function (visible to client code,
without the hidden channel) that delegates to
one of these overloads, automating the
size-comparison check without extra
code either at the call site or in the implementations.
More than just a device for inserting runtime checks at
useful \q{pointcuts} (using Aspect-Oriented terms),
such compiler enhancements, working through hidden
channels, allow automatic dispatching to
implementations who may declare but not test assumptions
and make type-systems allowing these functions closer
to bonafide dependent-type engines.
}
\p{These examples do not exhaust the topic of type-system-level
articulation and enforcement of runtime specifications,
but they perhaps pose ideas that with suitable variation can
apply in other contexts and unify subjects I have
discussed: if a larger class of specifications can be integrated
within type systems, then perhaps a larger class of
source code features (blocks, statements, etc.) can be
interpreted as typed values.
}
\p{}
