%\spsubsection{Described Types and Actual Types}
\subsection{Described Types and Actual Types}
\p{The notion of \q{type systems} I adopt here sees types not
logical abstractions but as engineered artifacts (as are
languages themselves).  A logical description of a plausible type
\mdash{} say, the type of all functions that take equal-sized lists
of integers \mdash{} may not correspond to a type that can be
concretely implemented in a given programming languages and
environment.  There may be \i{contingently}
uninhabited types, types which cannot be inhabited
\i{in some particular computing environment} because there are no
constructors implemented; or because the environment has
no compile-time or run-time mechanism for enforcing the
requirements stipulated by the type \mdash{} insofar as they are used
either to describe values or as an element in typing judgments.\footnote{Similar issues are sometimes addressed by a
\i{modal} type theory (cf., e.g., \cite{MurdochGabbay})
where (in one interpretation) a \i{logical}
assertion about a type may be \i{possible} but not necessary
(the modality ranging over \q{computing environments}, which
act like \q{possible worlds}).
}
}
\p{Types are, then, conceptually distinct from sets of values, and 
type theory should be developed with no appeal to sets or 
to set theory.  We therefore need a mechanism to specify and 
individuate types \mdash{} e.g., under what circumstance do 
two different symbols or two different \TXL{} terms 
designate \q{the same} type?  Here I will resolve questions about 
type-identity via functional resolutions: functions are overloaded 
insofar as one symbol can designate multiple functional values, 
which in turn are distinguished by type differences somewhere 
in their signature.  Consequently, types themselves are 
differentiated insofar as they engender distinct signatures.  
If and only if two expressions designate the same type, 
they can be interchanged in a signature without altering 
the function-like type which the signature demarcates. 
}
\p{In this context, then, a \i{type} is conceptually a set of
guarantees on function-call resolution (for overloaded function
symbols) and gatekeeping (for preventing code from executing with
unwarranted assumptions), and a type can only be inhabited if those
guarantees can be met.  In particular, the \q{witness} to a type's
being inhabited is always one (or more) functions \mdash{} either a constructor
(a \q{value constructor}) which creates a value of the type from
other values or from character-string literals; or, for function-like types,
an implemented procedure.
}
\p{A consequence of this framing is that defining what exactly constitutes
a type \mdash{} via an expression notating a type description and
a corresponding type implementation \mdash{} depends intrinsically
on defining what constitutes a \i{function}, and particularly
an \i{implementation} or \i{function body}.  Moreover, since
functions are implemented in terms of other functions,
another primordial concept is a function \i{call}.  Reasoning
abstractly about functions needs to be differentiated from
reasoning about available, implemented functions.\footnote{As a case in point, a common functional-programming idiom is to treat the composition
of two unary functions as itself a function-typed value to pass
to contexts expecting other function-typed values.  In my perspective
here, \fofg{} may be a \i{plausible} value, but it is
not an \i{actual} value without being implemented,
whether via a code graph (spelling out the equivalent of \lambdaxfgx{})
or some indirect/behavioral description (analogous to \inc{}
represented as \addOne{}).
}  
Consider function pointers: what is the address of \fofg{}
if that expression is interpreted in and of itself
as evaluating to a functional value?  This suggests
that a composition operator does not work in function-like
types quite like arithmetic operators in numeric types
(which is not unexpected insofar as functional values,
internally, are more like pointers than numbers-with-arithmetic).\footnote{Of course, languages are free to implement
functions behind the scenes to expand (say) \fofg{}, but
then \fofg{} is just syntactic sugar (even if its purpose
is not just to neaten source code, but also to inspire programmers
toward thinking of function-composition in quasi-arithmetic ways).
}  To put it differently, an \addressOf{} operator
\i{may} be available for \fofg{} if it is available for \fFun{} and
\gFun{}, but this depends on language design; it is not an
abstract property of type systems.
}
\p{A similar discussion applies to \q{Currying} \mdash{} the proposal
that types \tOnetoTwotoThree{} and  \tOnetoTwoTOThree{} are
equivalent, in that fixing one value as argument to a
binary function yields a new unary function.  Again,
since the Curried function is not necessarily implemented,
there is a \i{modal} difference between \tOnetoTwotoThree{}
and  \tOnetoTwoTOThree{}.  Languages \i{may} be engineered
to silently Curry any function on demand, but purported
\tOnetoTwotoThree{} and \tOnetoTwoTOThree{} 
equivalence is not a \i{necessary} feature of type systems.
}
\p{To the extent that both mathematical and programming concepts have a place here, we
find a certain divergence in how the word \q{function} is used.  If I say that
\q{there exists a function from \tOne{} to \tTwo{}}, where \tOne{} and \tTwo{} are
(not necessarily different) types, then this statement has two possible interpretations.
One is that, mathematically, I can assume the existence of a \tOneTotTwo{} mapping
by appeal to some sort of logic; the other is that a \tOneTotTwo{} function actually
exists in code.  This is not just a \q{metalanguage} difference projected
from how the discourse of mathematical type theory is used to different ends than discourses
about engineered programming languages, which are social as well as digital-technical
artifacts.  Instead, we can make the difference exact: when a function-value 
is keyed to a procedure, it is bound to a segment of code subject to 
analysis and to alternative representations (such as code graphs).  
}
\p{Initializing function-values from code-segments is, roughly, 
analogous to initializing simpler types from source-code literals.  
Typically \mdash{} see item \hyperref[funconstr]{1}
on page \pageref{funconstr} \mdash{} procedures are defined
from aggregate code-spans
with semantically and syntactically regulated internal structure.  I assume that
code is written in a specific language and that, in the context of that language,
any sufficiently complete code-span can be compiled to a code-graph 
\mdash{} say, an Abstract Semantic Graph
(or similar graph-like Intermediate Representation), generalizing from 
an Abstract Syntax Tree.  The logic of this
representation may vary among languages and/or type systems \mdash{} optimal
graph representation of source code is an active area of research, not only for
compiler technologies but also for code analyzers and \q{queries} and for
code deployment on the Semantic Web.  In this chapter, I assume that
an adequate graph representation will be isomorphic to a Directed Hypergraph.
So, assume that every code-span
suitable for compilation into a function-implementation can be isolated 
as a separate graph or subgraph, logically adjoined to one or more 
functional values (viz., those functions instanced in procedures 
described by the graph).  Assume also that every source code 
literal is equivalent to a \codeDH{} graph with one hypernode
and no edges.  In that case, initializing values from literals is one example
of initializing values from code-graph instances more generally.
}
\p{This approach \mdash{} using \DH{} or \CH{} graphs as the formal
ground of type-theoretic statements \mdash{} influences how we
can analyze types.  For every \q{function-like} type, 
most instances of the type are given through implementations suitable to
graph representation.  Many nodes in these graphs represent
values which the function receives from and/or passes to
other functions.  Therefore, assertions about functions' behavior
often take the form of assertions about values
functions receive from other functions, and conditions
for their properly sending values to other functions
in turn.  Insofar as we seek to express conditions on
functions' behavior through a type system,
we can then interpret type systems as
\i{leveraging} taxonomies of node-to-node
relations \mdash{} modeled via Source Code Ontologies \mdash{} to define
notations where descriptions of functions'
\i{behavior} can be interpreted or converted
to descriptions of types themselves.
}
\p{Notice that one single literal may initialize values (or, in my 
terminology, carriers) of multiple types;
the number \q{\zeroNum{}} could become a signed or unsigned integer, a float, etc.
Similarly, a code-span can be compiled multiples times, as in \Cpp{} templates.
So, there is not necessarily a one-to-one correspondence
between code-graphs and function values.  Nevertheless, we can assume that
each function implementation is uniquely determined by the function type
together with its function-body implementation-graph, whose potential
\q{template parameters} are fixed according to the produced type.  So, each
function \i{implementation} is fully determined by a type-and-code-graph
pair.  This formally expresses how \i{implemented functions} are different
phenomena than what we might call \q{functions} in   
mathematics.  If there is a
meaningful type Category then we must have nontrivial morphisms, which I
would argue should be those that abstractly capture type-level semantics,
such as predefined conversion operators or the
\q{initialization} morphism.  But morphisms are not affixed to
the code-graph and value-constructor machinery, though of course some morphisms
may coincide with implemented functions.\footnote{Here I depart from the usual account of Hypergraph Categories, because 
I assume that functions between types are not, in the general case, 
recognized as morphisms in types qua objects in a type Category.  
This is partly because Category-theory morphisms have no analogs 
to \q{channels} other than \lambda{} and \return{}.  
}
}
\p{Given this distinction, we can start to explore why some advanced constructs in
Dependent Type Theory, or other features of very expressive type systems,
may be hard to implement in practice.  I suggested earlier in this
section that \q{range-bounded} types are a good case-study in
implementational complications that can befall described types;
I will now return to that discussion and pursue the \mbox{\q{ranged-type}}
case further.
}
\vspace{-.1em}
%\spsubsectiontwoline{Range-Bounded Types, Value Constructors, and Addressability}
\subsectiontwoline{Range-Bounded Types, Value Constructors, and Addressability}
\p{Consider a function \fFun{} which takes
values that must be in a specified range \rRan{} (say, an integer between
\zeroToOH{}).  By extension, suppose we want to overload \fFun{} based on
whether its argument (say, \xSym{}) falls inside or outside \rRan{}.
This is not hard to achieve if \fFun{}'s \i{type} internally references
a \i{fixed} \rRan{}.  Let \Tvar{} be a symbol that quantifies over the typeclass
of types with magnitude/comparison operators, and \rRanOfT{} be a type
formed from \Tvar{} and two  \Tvar{}-values, implementing the semantics
of closed intervals over \Tvar{}: so \rrRanOfTVV{} is a \q{type-expression}
mapped to a type constructor yielding a single type
(not a type family, typeclass, or higher-order type).
}
\p{For any type-with-intervals \Tvar{} and \Tvar{}-range \rRan{}, a compiler
(for instance by specializing a template) can produce a value constructor that takes
\Tvar{}-values and tests (or coerces) them such that the constructor only returns
a value in \rRan{}.  This can then be the input type for a function which
requires \rRan{}-bounded input.  What to do when a values \i{fails} the
range-test is another question which I set aside for now.  We can similarly define
a \q{non-range} type which only accepts values \i{not} in \rRan{}; and, since these
two types (the in-range and the out-of-range) are different, we can overload
\fFun{} on them.  So, assuming we accept \rRan{} being fixed, we can achieve
something semantically \mdash{} \q{overload-wise} \mdash{}
like dependent typing.  Of course, Dependent Types
as a programming construct are more powerful than this: an example
of \q{real} dependent typing would be something like an \fFun{} which takes
\i{two} arguments: the first a range, and the second a value within
the range.  We want the type system to be engineered allowing the condition
on the second argument to be verified as part of \i{typechecking}.
}
%\itcl{range}
\p{Using the \rRan{}-type as before, the type of \fFun{}'s second parameter
would then be \Tvar{} restricted to the \rRan{} interval, but here
\rRan{} is not fixed in \fFun{}'s declaration but rather passed in to
\fFun{} as a parameter; the type of the second parameter depends on the
\i{value} of the first one.  Unless we know \i{a priori} that only
a specific set of \rRan{}s in the first parameter will ever be encountered,
how should a compiler identify the value constructor to use for
\xSym{}?  This evidently demands either that a value constructor be automatically
created at runtime, for each \rRan{} encountered \mdash{} 
\mbox{so, i.e., that the compiler 
has to insert some runtime} mechanism which creates and calls a value constructor
for \xSym{} before \fFun{}'s body is entered \mdash{}  or else that a
single value constructor is used for all \xSym{}s, regardless of \rRan{}.
Ordinarily, we think of value constructors as procedures which 
yield values of their intended types.  Given the bijection 
(once types are fixed) between procedure implementations and code-graphs, 
each (concretely) inhabitable type would then have at least
one value constructor which is unique to that
type: a type-plus-code-graph pair.  Conversely, one code-graph 
can engender multiple procedures by varying carrier types.  
This allows one code graph to be
mapped to multiple value constructors.  But, this 
generalization does not (without supplemental logic) 
support a parallel generalization where 
one value constructor would service many types, although we
can implement functions that would be semantically analogous to
such a value constructor.
}
\p{We could certainly write a function that takes a range and a value and
ensures that the value fits the range \mdash{} perhaps by throwing an
exception if not, or mapping the value to the closest point in the range.
Such a function would provide common functionality for a family of
constructors each associated with a given range.  But a function (\cfFun{}, say)
providing \q{common functionality} for value constructors is not necessarily
itself a value constructor.  If we'd want to treat such a function as a
\i{real} value constructor we'd have to add contextual modifiers:
\cfFun{} is a value constructor for range-type \rRan{} when \rRan{} as a range
is supplied as one parameter.  The value constructor itself would have to
be dependently typed, its result type varying with the value of its arguments
\mdash{} but a result-type-polymorphic value constructor is no longer an actual
value constructor; at best we can say it is a function which can dynamically
\i{create} value constructors.  In the present case, Currying \cf{} on any given
\rRan{} probably does yield a bonafide value constructor, but a function
which when Curried yields a value constructor is not, or at least
not necessarily, a value constructor itself.
}
\p{It appears that language designers \mdash{} at least considering pureblood Dependent
Types \mdash{} have two options: either modify the notion of value constructor
such that one \i{true} value constructor is understood as a possible constructor
for multiple types, and on behalf of which type it is constructing is something
dynamically determined at runtime; or value constructors are allowed to be
transient values created and recycled at runtime.
This is not just an internal-implementation question
because value-constructors also need to be \i{exposed} for reflection (which
in turn involves some notion of addressability: the most straightforward
reflection tactic is to maintain a map of identifiers to function addresses).
Either option complicates the relation between types'
realizability and their value constructor: instead of each inhabitable type
having at least one value constructor which is itself a value, and as such itself results
from a value constructor derived from a code graph, we have to associate types either with
dynamically created temporary value-constructor values or we have to map value
constructors not to singular values but to a compound structure.  For example, if the
purported value constructor for a range type \TrRan{} is to be the
\q{common functionality} base function \i{plus} a range-argument to be
passed to it \mdash{} some sort of \Cfr{} compound data structure,
again by analogy to \inc{} and
\addOne{} \mdash{}
then the \q{value} of the value constructor no longer has a single part,
but becomes a function-and-range pair.
Let me dub this the \q{metaconstructor}
problem: what are allowable \i{value constructors for the value constructors} of
allowable types?
}
\p{If we ignore templates, a reasonable baseline assumption is that
\q{metaconstructors} must only be obtained from one sort of origin: code graphs.
That is, for each metaconstructor \mdash{} again, a de-facto value constructor
whose result is a value constructor whose result is a value of some type \tTy{}
\mdash{} there must be exactly one code-span notating the 
value constructor's implementation.\footnote{I am not specifically assuming some sort of on-the-fly compilation 
where new functions could be created from source code at runtime.  
That is, referring to \q{meta-constructors} as value-constructors may 
be suggestive more than literal, technical terminology. 
On the other hand, if a language \i{can} dynamically read 
code-graphs, this usage should be understood more literally: 
we might assume that there is a specific type representing 
code-graphs, and values of that type are then inputs to 
value-constructors yielding procedure types. 
}
As I just outlined, dependent typing can complicate the picture because
metaconstructors then have to have possible alternative signatures: e.g.
the value constructor for \q{integers between zero and one hundred (inclusive)} has
to combine a \q{common functionality} function body with another
part that specifies the desired
\zToOH{} range.  If we \i{don't} ignore templates, we can speculate that
each actual metaconstructor is a specialization of a template, so each
one goes back to the one-argument-code-graph signature \mdash{} but we then have
an entire family of metaconstructors (or possible metaconstructors) which
share functionality and differ only according to a criterion that varies
over values of a type.  Consider just the simpler case of integer
ranges with lower bound zero: for any \iVal{} of an integer type
(64-bit unsigned, say) there is a reasonable type of \int{}s \le{} \iVal{}.
The collection of \q{reasonable} types formed in this manner is therefore
co-extensive with \int{} itself.  But on both philosophical
and practical grounds, we may argue that \q{reasonable} types are not
the same thing as types \i{full stop}.
}
\p{Philosophically, programming types lie at the intersection of mathematics and
human concepts: a datatype typically avatars in digital environments some
human concepts.  There are particular arithmetic intervals that have
legitimate conceptual status: let's say, 
\zToOH{} for percentages; the maximum speed of
a car; the dial range of a thermostat.  So, conceptually, we can
implement an abstract family of range types which might be concretized
for a handful of conceptually meaningful specializations.  Moreover,
we can conceptualize a general-purpose structure which is a range \rRan{}
together with a range-bounded value, but then we are conceptualizing
\i{one} type, not a whole family of types.  So the basic \q{Ontology}
of Dependent Types \mdash{} of whole type-families indexed over values of some
other type \mdash{} does not correspond with the nature of concepts (see 
e.g. \cite[p. 4]{BernardyEtAl}): while
there is a \i{reasonable type} for intervals \zTon{} for any \nVal{}, there
is not necessarily a corresponding \i{concept}, \i{a priori}
(similarly, we have a capacity to conceptualize any number \mdash{} assuming
it has some distinct conceptual status, like \q{the first nontrivial Fourier
coefficient of the \jFunction{}} \mdash{} but reasonably we do not have a
distinct concept for every number).
}
\p{Meanwhile, practically, it is not computationally feasible to have an
exponential explosion in the order of \i{actual} types \mdash{} such as,
one unique type
for each 64-bit integer.  For example, it is reasonable for a
language engine to assume that most function values support an address-of
operator.  This is one property whereby function values differ from, say,
integers: we cannot take the address of the number \litFive{} (by contrast, 
we \i{could} form a pointer to a file-scoped \CLang{} function that 
just trivially returns \litFive{}).  
But allowing type families to be indexed on 64-bit integers
\i{and} providing a distinct address for each such type's value-constructors
would be mathematically equivalent to providing a unique address for each
64-bit integer.
}
\p{A reasonable language, conversely,
may have \q{non-addressable} function values: for example, suppose an
anonymous function passed to a procedure is defined via an operator, like an
\fOfG{}.  Say, sorting two lists of strings on a comparison which calls a
\q{to lowercase} function before invoking a less-than operator.  This could
be notated with an anonymous-function block, but some languages may allow a more
\q{algebraic} expression, something meaning \q{lower-case then less-than}, with
the idea that function values can be composed by rough analogy to
numbers being added (see item \hyperref[ops]{3} on page \pageref{ops}).
In this case, the \i{value constructor for the
function type} does not take a code graph, at least not one visible
near the \fOfg{} expression, just as the value constructor for \xVal{} in
\xeqyplusz{} is hidden somewhere in the \q{\yplusz{}} implementation.
A language can reasonably forbid taking the address of (or forming pointers to)
\q{temporary} function values derived algebraically from other
functions.  Indeed, the concepts of \q{constructed from a code graph}
and \q{addressable} may coincide: a compiler may allocate long-term memory for
just those function-implementations it has compiled from code-graphs.
}
\p{But value-constructors are not just any function-value: they have a privileged
status \visavis{} types, and may be invoked whenever an appropriately-typed
value is used.  Allowing large type families (like one type
for each \int{} \mdash{} similar to \q{inductive typing} as
discussed by Edwin Brady in the context of the Idris language
\cite[p. 14]{EdwinBradyImpl}) effectively forces a language to accept
non-addressable value-constructors.  Conversely, forcing value constructors to be
addressable prohibits \q{large} type families \mdash{} like types indexed
over other (non-enumerative) types \mdash{} at least as \i{actual} types.
A language engine may declare that value constructors, in short,
cannot be \q{temporary} values.
This apparently precludes full-fledged Dependent Types, since
dependent-typed values invariably require in general some extra
contextual data \mdash{} not just a function-pointer \mdash{} to designate the
desired value constructor at the point where a value,
attributed to the relevant dependent type,
is needed.  It may be infeasible to add the requisite contextual
information at every point where a dependent-typed value has to be constructed
\mdash{} unless, perhaps, a description of the context can be packaged and
carried around with the value, sharing the value's lifetime.
}
\p{In a nutshell: without restricting their expressiveness, Dependent Types can 
only be achieved by modifying foundational assumptions about the relationships 
between types and the procedures which construct their instances.  
We then have a choice between simply accepting an expanded model of 
value-construction or devising strategies for emulating Dependent Types 
within the confines of a code model that preserves, in particular, 
value-constructor addressability.  The techniques I discuss in this 
chapter are adjusted to the latter decision.    
}
\p{A value can, indeed, actually be an aggregate data structure including
functions to call when
the value is created or modified \mdash{} behaving as if it were dependently typed
\mdash{} but this is more a matter of one type supporting a range of different behaviors,
rather than a family of distinct types.  A single range-plus-value
type can behave \i{as if} it were actually instantiating a type belonging
to a family where every possible range corresponds to a different type \mdash{}
at least with respect to value constructors and accessors, which can implement
hidden gatekeeping code.  But the type is still just one type from the
point of view of overloading: the behavioral constraints are code evaluated
behind-the-scenes at runtime, and cannot in themselves be a basis for compile-time
overload decisions.  In other words, they are more like \i{typestate} than type families.
}
\p{Consider a function to remove the \nth{} value from a \listval{}.
For this to work properly, the \nVal{} has to be less than
the size of \listval{}; i.e., it has to be in the range \rlstsize{}.
The relevant range-expression \i{looks} like the example I used
earlier \mdash{} \IntZToOH{} \mdash{} but in place of a \i{fixed} \ztooh{} range, here
we have a range that can potentially be different each time the
function is called (assuming each \listval{} can be a different
size).  So while it may be a well-formed type-expression to
say that \nVal{} has \i{type} \intrlstsize{}, the net result is that
\nVal{}'s type is then not known until runtime.  Since its type is
not known, nor is the proper value constructor to call when
\nVal{} has to be provided to a \lambda{} channel, at least
not \i{a priori}.  Instead, the value constructor has to be
determined on-the-fly.  As such, the \q{constructed} value constructor
acts like a kind of supplemental function called prior to the main function
being called.  But there are several ways of arranging for such
gatekeeping functions to be called, apart from via explicitly
declaring types whose value constructors implement the desired
functionality.  In the current example, there are ways to
ensure that a gatekeeping function is called whose runtime checks
mimic the \intrlstsize{} value constructor without actually
stipulating that \nVal{}'s \i{type} is \intrlstsize{}.  Some
of these involve typestate, which I will now review briefly.
Other options will be discussed later in the chapter.
}
