\section{Modeling Procedures via Channelized Hypergraphs}
\phantomsection\label{sThree}
\p{Assuming we have a suitable Source Code Ontology, software 
procedures can be seen from two perspectives.  On the 
one hand, they are examples of well-formed code graphs: 
annotated graph structures convey the lexical symbols, 
input/output parameters (via different \q{abstractions}, 
in the sense of \mOldLambda{}-abstraction, subject to 
relevant channel protocols), and calls to other procedures, 
through which any given procedure's functionality is 
achieved.  On the other hand, we can see procedures as 
instances of function-like types, where the types carried 
in each channel determine the type of the procedure itself, 
as a functional value.  Although these two perspectives are 
usually mutually consistent, the notion of functional 
values is more general than procedures which are expressly 
implemented in computer code.  In particular, as I briefly 
mentioned earlier, sometime functional values are denoted 
via inter-function operators (like the composition 
\fOfG{}) rather than by giving an explicit implementation.  
Programmers would say that functions defined via operators 
(such as \Ofop{}) lack a \q{function body}.  
}
\p{Going forward, I will generally use the term \i{procedure} 
with reference to function-like type instances that are 
defined \i{with} function bodies: that is, they are 
associated with sections of code that supply the procedure's 
implementation, and can be represented via code-graphs.  
I will use the term \i{function} more generally for 
instances of function-like types, irregardless of their 
provenance.  In particular, functions are \i{values} 
\mdash{} instances of types in a relevant type-system 
\TyS{} \mdash{} whereas I will not usually discuss procedures 
as \q{values}.    
}
\p{Most functions which may be called by a software component 
are defined as procedures with their own function bodies.  
Indeed, the essential software-development process involves 
implementing procedures in code, and then implementing 
other procedures \mdash{} which call those already-created procedures  
(in various combinations) \mdash{} to realize their own functionality.  
That is, source code itself produces instances of 
functional values, via procedures described by the code.  
Accordingly, code-graphs for individual procedures capture 
the implementations through which function-like types 
are (mostly) populated with concrete values.   
}
\p{These various concepts \mdash{} procedures, functions, implementations, 
and function-bodies or code-graphs \mdash{} are important for 
broad architectural topics like Requirements Engineering, because 
they concern the building-blocks from which larger software 
components are assembled.  When discussing procedures' coding assumptions 
(more fine-grained than type constraints alone can model), for example, 
we need a rigorous presentation of what a procedure itself is, 
to identify the relevant entity whose assumptions can be documented 
and tested.  To model the general maxim that any coding assumptions 
made (but not verified) by one procedure \mdash{} say, \ProcOne{} \mdash{} 
should be tested by other procedures which call \ProcOne{}, we need 
a systematic outline capturing the notion of procedures calling 
other procedures, in the course of their own implementation.  
Here I propose to model these details via channels and 
interrelationships between channels.   
}
\p{In my formulation of these representations, channels are conceptually 
integrated with hypergraph code models.  That is, code-graphs are 
a formal device within the theory I present here.  In many other 
context code-graphs would instead be, at most, convenient expository 
devices; for instance, functional programming languages generally do 
not attach much significance to the contrast of procedures 
(with function-bodies) and function-values constructed by 
other means.  As a result, one consequence of my graph-oriented 
approach is that the technical distinctions between 
procedures and function-values (in general) have to be 
duly observed.  There are some relevant complications appertaining to 
the general picture of source-code segments instantiating 
function-like types.  I will briefly review these issues now, 
before pivoting to more macro-scale themes like Requirements Engineering.
}
\vspace{-.1em}
%\spsubsection{Initializing Function-Typed Values}
\subsection{Initializing Function-Typed Values}
\p{Although in general function-typed values are \i{initialized}
from code-graphs that blueprint their implementation,
this glosses over several different mechanisms by which
function-typed values may be defined:
\begin{enumerate}\eli{}  \phantomsection\label{funconstr}In the simplest case, there is
a one-to-one relationship
between a code graph and an implemented function (\fFun{}, say).
If \fFun{} is polymorphic, in this case, it must be an example
of subtype (or \q{runtime}) polymorphism where the declared types of \fFun{}'s
parameters are actually instantiated, at runtime, by values
of their subtypes.
\eli{}  A different situation (\q{compile-time} polymorphism) applies
to generic code as in \Cpp{}
templates.  Here, a single code-graph generates multiple function
bodies, which differ only by virtue of their expected types.
For example, a templated \sortfn{} function will generate
multiple function bodies \mdash{} one for integers, say, one for strings,
etc.  These functions may be structurally similar, but they have
different signatures by virtue of working with different types.  This
means that symbols used in the function-bodies may refer to
different functions even though the symbols themselves do not vary
between function-bodies (since, after all, they come from the
same node in a single code-graph).  That is, the code-graphs
rely on symbol-overloading for function names
to achieve a kind of polymorphism, where one code-graph
yields multiple bodies.
\pseudoIndent{}
In this compile-time polymorphism,
symbols are resolved to the proper overload-implementation
at compile-time, whereas in runtime polymorphism this
decision is deferred until the runtime-polymorphic function
is actually being executed.  The key difference is that
runtime-polymorphic functions are \i{one} function-typed value,
which can work for diverse types only via subtyping \mdash{}
or via more exotic forms of indirection, like
using function-pointers in place of function symbols; whereas
compile-time-polymorphic (i.e., templated) functions are
\i{multiple} values, which share the same code-graph
representation but are otherwise unrelated.
\eli{}  \label{ops}A third possibility for producing function values is to define
operators on function types themselves, which transform
function-typed values to other function-typed values, by analogy
to how arithmetic operations transform numbers to other
numbers.  As will be discussed below, this may or may not be
different from initializing function-typed values via code-graphs,
depending on how the relevant programming language is implemented.
For instance, given the composition operator \Ofop{},
\fDotOfg{} may or may not be treated as only a convenient
shorthand for a code graph spelling out something like \fgx{}.
\eli{}  Finally, as a special case of operators on function-typed values,
one function may be obtained from another by \q{Currying}, that is,
fixing the value of one or more of the original function's
arguments.  For example, the \inc{} (\q{increment}) function which adds
\litOne{} to a value is a special case of addition, where the added value
is always \litOne{}.  Here again, Currying may or may not be
treated as a function-value-initialization process different from ones
starting from code-graphs.
\end{enumerate}
}
\vspace{-1em}
\p{The differences between how languages may process the \i{initialization}
of function-type values, which I alluded to in (3) and (4), reflect
differences in how function-type values are internally represented.
One option is to store these values solely in blocks of
memory, which would correspond to treating all initializations of these
values as via code-graphs.  For example, suppose we have an \addFun{} function
and want to define an \inc{} function, as in \incimpl{}.  Even if a language has
a special Currying notation, that notation could translate behind-the-scenes to
an explicit function body, like the code at the end of the last sentence.
However, a language engine may also note that \inc{} is derived from \addFun{}
and can be wholly described by a handle denoting \addFun{}
(a pointer, say) along with a designation of the fixed value: in
other words, \addOne{}.  Instead of initializing \inc{} from a code-graph,
the language can represent it via a two-part data structure like
\addOne{} \mdash{} but only if the language \i{can} represent
function-typed values as compound data structures.
}
\p{Let's assume a language can always represent \i{some} function-typed values,
ones that are obtained from code-graphs, via pointers to
(or some other unique identifier for) an internal
memory area where at least \i{some} compiled function bodies are stored.
The interesting question is whether \i{all}  function-typed values
are represented in this manner and, in either case, the
consequences for the semantics of functional types \mdash{} semantic
issues such as \fOfg{} composition operators and Currying
\makebox{(and also, as I will argue, Dependent Types)}.
}
%\spsubsection{Addressability and Implementation}
\subsection{Addressability and Implementation}
\p{As \i{Intermediate} Representations, formal code models (including
those based on \DH{}s) are not strictly
identical to actual computer code as seen in source-code files.
In particular, what appears to
be a single function body may actually form multiple
implementations via code templates (or even
preprocessor macros).  Talk about polymorphism in a language like
\Cpp{} covers several distinct language features: achieving
code reuse by templating on type symbols is internally very different
from using virtual methods calls.  The key difference \mdash{} highlighted
by the contrast between runtime- and compile-time polymorphism \mdash{} is
that there are some function implementations which actually
compile to \i{single} functions, meaning in
particular that their compiled code has a single place in memory and
that they may be invoked through function pointers.  Conversely,
what appears in written code as one function body may actually be
duplicated, somewhere in the compiler workflow, generating multiple
function values.  The most common cases of such duplication
are templated code as discussed above (though there are
more exotic options, e.g. via \Cpp{} macros and/or
repeated file \codeinclude{}s).  Implementations of the first sort I will
call \q{addressable}, whereas those of the second I will
dub \q{multi-addressable}.  These concepts prove to be consequential
in the abstract theory of types, although for non-obvious reasons.
}
\p{To see why, consider first that type systems are intrinsically
pluralistic: there are numerous details whereby the type system
underlying one computing environment can differ from those employed
by other environments.  These include differences in how types are
composed from other types.  There is therefore no abstract vocabulary
(including the language of mathematical type theory) that provides a
neutral and complete way of describing types across systems.  Instead,
each system has its own structure of multi-type aggregation, and
so requires its own style of type description (mathematically, there is
no one \q{Category} of types, but rather multiple candidate Categories
with their own logic).  So there is no single, universal
\q{Type Expression Language}: each type system has its
own \TXL{} with subtly different features than others.
}
\p{By intent, I use \TXL{} to mean languages for describing types
which \i{may} be implemented.  For example, if in \Cpp{} I
assert \q{\templateTMyList{}}, it would then be consistent with
a \Cpp{}-specific \TXL{}
to describe a type as \MyListInt{} (which would presumably be
some sort of list of integers, though of course naming hints about the
intended use of a type
have no bearing on how compilers and runtimes process it).  However,
the type \MyListInt{} is not, without further code, actually implemented.
It is a \i{possible} type because its description conforms to a relevant
\TXL{}, but not an \i{actual} type.  If a programmer supplies
a templated implementation for \TMyList{} (intended for multiple
types \TType{}) then the compiler can derive a \q{specialization} of the
template for a specific \TType{} \mdash{} or the programmer can specialize
\MyList{} on a chosen type manually.  But in either case the actualization of
\TMyList{} will depend on an implementation (either a templated implementation
that works for multiple types or a specialization for a
single type); this is separate and apart from \TMyList{} being
a valid \i{expression} denoting a \i{possible} type.
}
\p{Once \TMyList{} is instantiated, for a particular \TType{}, there may be a
constructor or initialization
function that is \i{addressable}, either as one duplicate
of a multi-addressable implementation or as the compilation of
one non-templated function body.  Call a type \i{addressable} if it
has at least one constructor
(i.e., \q{value} or \q{data} constructor, a
function which creates a value of a type from a literal
or a value of a different type); and \i{multi-addressable} if it has
at least one multi-addressable constructor or initializer: these
terms can carry over from functions to types for which
functions classified in these terms are constructors.\footnote{In this discussion I mostly skip over the technical detail that
in \Cpp{}, at least, one cannot actually take the address of a
constructor function; but this is related to \Cpp{}
\q{constructors} having dual roles of allocating memory
and initialization: we \i{can} take the address of an
initialization function that would be analogous to a
\q{pure} value constructor.
}
}
\p{Addressability refers at one level, as the word suggests, to
\q{taking the address} of functions (and accordingly to
function-pointers); but here I also refer to a broader question of
how functions can be designated from vantage points outside their
immediate implementation \mdash{} code searches, scripting engines,
\IDE{}-based code exploration, and other reflection-oriented
use cases.  Language engines should try to minimize their
reliance on \q{temporary function values} which are
opaque to these reflection-oriented features.  And yet
this can complicate the implementation of type-system features.
To reiterate: the goal of
\q{expressive} type systems is to define types,
as necessary, narrowly enough to type-check granular procedure 
requirements.
The problem is that gatekeeper code induced by type-level
expressiveness \mdash{} particularly code which is automatically
generated \mdash{} can be opaque to extra-linguistic environments
like \IDE{}s and scripting engines.
}
\p{For example, suppose certification
requires that the function which displays the gas level on a car's dashboard
never attempts to display a value above \litOH{} (intended to mean \q{One Hundred percent},
or completely full).  One way to ensure this specification is to declare
the function as taking a \i{type} which, by design, will only ever include
whole numbers in the range \ZeroToOneHundred{}.  Thus, a type system may support
such a type by including in its \TXL{} notation for \q{range-delimited} types,
types derived from other types by declaring a fixed range of allowed values.
A notation might be, say, \IntZToOH{}, for integers in the \ZeroToOneHundred{}
range \mdash{} or, more generally expressions like \TVOneToVTwo{}, meaning a \i{type} derived
from \TType{} but restricted to the range spanned by \VOne{} and \VTwo{} (assumed to be
values of \TType{} \mdash{} notice that a \TXL{} supporting this notation must
consequently support some notation of specific values, like numeric literals).
This is a reasonable and, for programmers, potentially convenient type description.
}
\p{For a language designer, however, it raises questions.  What should happen if
someone tries to construct an \IntZToOH{} value with the number, say,
\litOHO{}?  How should the range-test code be exposed for
reflection (is it a separate function; is it a regular
function-typed value or some alternative data structure, and
how does that affect its external designating)?  What if the
number comes from a location that
opaque to the language engine, like a web \API{}: should the compiler assume
that the \API{} is curated to the same specifications as the present code or
should it report that there is no way to verify that the declared \IntZToOH{} type
is being used correctly?  Moreover, would such choices lead to
behind-the-scenes, perhaps auto-generated code which is hard to
wrangle for reflection and development tools?
Are the benefits of automated gatekeeping worth the risk of
codewriters' mental disconnect with language internals?
Given these questions, it is reasonable for a language
designer to \i{allow} certain sorts of types to be described, but programmers
must explicitly implement them for the types to be actualized and available
for use in programs.  Therefore there is a difference between a
\i{described} type and an \i{actual} type, and the key criterion of actual
types is an addressable value constructor.  So the crucial
step for type-theoretic design promoting desired software qualities, like
safety and reliability, is to successfully pair the \i{description} of types
which have desirable levels of specification and granularity, with the
\i{implementation} of types that realize the design patterns promised by
their description.  In some cases the description must become
more complex and nuanced to make implementation feasible.
}
\p{Range-bounded types are a good example of types which can be succinctly
\i{described} but face complications when being concretely
implemented.  They are therefore a good example of
the potential contrast between \i{possible} and \i{actual} types.
I will examine this distinction in more detail and
then return to range-bounded types as a demonstrative example.
}
