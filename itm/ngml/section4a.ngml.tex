\spsubsectiontwoline{Phenomenology as Experiment}
\p{Phenomenology has not usually attended to formal analysis of
cognitive models, or models of mental phenomena overall.
This is entirely understandable, since exploration of formal
systems surely does not belong to phenomenological
analysis itself, at least outside the exotic sense of studying
what it is like to learn or think about mathematics, say,
from a first-person perspective.  Nevertheless, even a
writer committed to phenomenology does not solely do
phenomenological analysis, any more than a Structuralist
poeticist devotes all paragraphs to close-readings.
Equally intrinsic to the philosophical process is
how intra-methodological analyses (\q{readings}
of consciousness or literary works, say, respectively) are
placed in a larger context, which can have multiple
dimensions: relations to other scools of philosophy, but
also other academic disciples and other regimes of knowledge
and practice.  Pursuit of formal models can help ground and
orient phenomenological investigations, and vice-versa.
}
\p{As a case in point, consider how phenomeneology was among the
inspirations for what computer scientists call the Semantic
Web and Formal Ontologies \mdash{} protocols for sharing information
and aggregating \q{knowledge} across computer networks,
paricularly the World Wide Web.  Although the distinction 
is not sufficiently discussed, I believe, the Semantic Web idea really
has two dimesnions: we can call them \i{static} and \i{dynamic}.  
The \i{dynamic} aspect of the Semantic Web (and any data-sharing 
platform) reflects how the technology needs to enable, 
and verify, correct and useful behaviors.  In particular,
the Semantic Web needs to treat \i{information} as an 
asset that appreciates in value as it is shared and duplicated.  
Semantic Web technology needs to identify situations where 
it would be valuable for some aggregate of data 
present at locality $L1$ to be shared with locality 
$L2$; and should provide the technological 
capabilities to ensure that $L1$ and $L2$ can interoperate
properly to effectuate this sharing.  
}
\p{These goals and requirements are \i{dynamic}: they model and 
implement scenarios where some piece of software concludes that 
some remote information would be valuable, and 
initiates a process of acquiring that data, by interoperating 
with other software which takes steps to respond 
cooperatively (within the limits 
of proper data sharing protocols).  I will discuss this \i{dynamic} 
aspect of the Semantic Web below.
}
\p{By contrast, the \i{static} dimension of the Semantic Web 
reflects the goals of data representation itself: the 
essential information and structure manifest in data 
needs to be preserved arocss locations as data is routed and 
shared.  Accordingly, one goal of Semantic Web technology is 
to design data representations that retain a static 
meaning across contexts and locations.  These representations have 
to be sufficiently precise and unambiguous that 
heterogeneous software platforms can interoperate, by 
sharing the data represented, without 
distorting or misinterpreted the information 
encoded in the relevant shared data.
}
\p{Such information is not only
abstract mathematical or technical data: our world is founded
on communicating structured knowledge from many human and scintific
domains, like medicine and government.  We cannot in this
context construe data as just bytes and numbers, but rather
encodings of human concepts and judgments.  Trying to
map essentially informal human concepts, like \i{illness},
to a precise scientific formulation, is of course a
foundational concern in philosophy; but computer networks and
technology-enhanced knowledge sharing reveal practical applications
of these perhaps once purely abstract problems.  Scientists
use Formal Ontologies to codify conceptual systems
underlying human knowledge and beliefs.  This, moreover,
spans both fairly narrow and concrete propositions (e.g.,
that the pericardium surrounds the heart) and deeper, more
abstract, more cognitively primeval concepts
(like what it means for one thing to surround another thing).
}
\p{In short, part of the requiements of building knowledge
\q{engineering} or \q{sharing} systems was to give a technical
specification for apparently innate concepts or gestalts
like \i{part of}, \i{inside of}, \i{surrounding}, and
so on.  At least one method for approaching
this problem was via phenomenology, meaning that
phenomenology serves as one tributary among others that can
be followed into the technical context of data modeling and
data sharing protocols.  At the same time, the Semantic Web
community has converged on several preferred models and
formats \mdash{} bearing acronyms like OWL and RDF \mdash{} which
in turn have proved controversial.  Some
researchers, notably \Gardenfors{},
have critiqued the Semantic Web for a conceptual simplification
that does not really capture the semantics of technical domains
(like science and medicine), still less of Natural Language.
Others from a more implementation-minded perspetive
can highlight technical limitations of Semantic Web models,
such as how data sharing \i{between} computing environments can
best integrate with data managment \i{within} particular
databases and applications.  The Semantic Web \mdash{} whose
underlying representations are based on labeled, directed graphs
\mdash{} has been critiqued by advocates for modestly different
graph-like structurres, like Conceptual Graph Semantics and
Directed Hypergraphs.  These alternative models are arguably
more conceptually accurate and/or more practically efficacious,
from an engineering perspective, than the paradigms for
representing Formal Ontologies and Informations Spaces
or \q{linked data} which emerged as predominant in the
community of Semantic Web developers and researchers.
}
\p{These unfolding perspectives are relevant to phenomenology
partly because competing representational paradigms
can seem more or less phenomenologically faithful;
can intersect with phenomenological accounts in different
ways.  For example, suppose we judge that an
alternative representational model, like the Hypergraph
framework associated with the OpenCog project \mdash{} itself
oriented to Artificial General Intelligence \mdash{} is a
more realistic model of conceptual sructures insofar as
these intrinsically emerge from and regulate
cognitive/perceptual processes as articulated
in phenomenology.  That is just a claim, of course, but
assuming for sake of argument it is plausible, then we
have a case of two competing formal models \mdash{} both
reflecting some measure of at least informal influence
frpm phenomenological ideas, as far as seeking
philosophically well-grounded accounts of ontology and
intelligence.  These models can be contrasted on
philosophical grounds, but also technologically.
Neither OpenCog nor the overall Semantic Web are academic
theories per se, but technology projects with their
own software ecosystems and engineering norms,
alongside academic literature and philosophical
attitudes or intuitions that can be evaluated
theoretically.
}
\p{Even though phenomenology is \i{philosophy}, I believe,
that doesn't mean considerations from other disciplines
as they bear on \mdash{} to take this one specific case \mdash{}
OpenCog vs. the canonical Semantic Web are not potentially
relevant and interesting to the phenomenological project.
The relevant contrast here presents two competing
computational frameworks, and the ecosystems
can be scrutinized side-by-side with an eye to
the merits of their technology, to how they
are used, extended, and integrated into practical software
and respond to technical requirements.  These engineering
comparisons can co-exist with more philosophical ones:
if one paradigm seems superior \i{both} practicallly
and philosophically, this deserves consideration \mdash{} do
the two horns reinforce each other?  Conversely,
if the more phenomenologically faithful theory
proves less implementationally useful, does this
shed light on philosophically interesting issues
\mdash{} the weakness of \q{mind as compuer} analogies,
for instance?  We may not be able to specify \i{a priori}
what kind of significance to attach to comparisons between
formal models on phenomenological vs. engineering
grounds.  But we should recognize
that technical comparsions are at the least nontrivial data
points that should at the least be acknowledged in the background
while investigating formalizations that incoporate,
and insofar as they incorporate, phenomenological
intuitions.
}
\p{If this is plausible, then the disciplinary frontier of
phenomenology significantly expands.  Phenomenologists
can in any cases engage with the \i{academic} face
of, say, OpenCog and Semantic Web projects
\mdash{} texts by Ben Goertzle or \Gardenfors{},
for instance \mdash{} read as at least tangentially philosophical
ouevres.  But any discipline which finds relevance in
these \i{writings} should also find relevance in
the technology they are writing \i{about},
in their concrete form as technical artifacts and
(products of) engineering processes.  We can therefore
approach technologies like AtomSpace (a database associate
with OpenCog) and Fact++ (a Semantic Web engine)
from an engineering as well as theoretical perspective
\mdash{} how they are implemented, compiled, and interoperate with
other applications.  Or, as this one example illustrates,
we can incorporate within the phenomenological discipline an
interest in technical comparison between formal systems
which also manifest phenomenological intuitions, so the
phenomenological and engineering dimensions of their
juxtaposition can be juxtaposed in turn.  This represents
a new avenue for engagement with formalizations following
the example of, let's say, Husserl's \i{Formal and
Transcendental Logic}, which approached from a
phenomenological perspective then-contemporary issues in
logic and mathematics.  But a key difference is that the
formalizations Husserl entertained were fully abstract, while
the formal systems that can be approached phenomenologically
in our century are more concrete and enmeshed in
social-scientific practice (health care, environmental
policy, etc.).  Engaging with these \q{concretized}
formal systems is not a matter of proving theorems,
or understanding proofs of prior theorems; it
more involves compiling computer code,
or writing new code, and understanding the technical
structure of programming languages and data representations.
}
\p{In effect, computer code \mdash{} software and formal
languages (including markup and database query as well as
general-purpose programming languages) \mdash{} has in many
contexts supplanted abstract logic as formal foundations
of well-structured thought.  This even applies
to mathematics, where type-theoretic and proof-assistant
methods take the place of set theory and predicate logic
as foundations.  This new reality should be confronted
by philosophers as well \mdash{} what is the philosophical
analog of the Univalent Foundations project in mathematics?
How should Analytic Philosophers \mdash{} or indeed
phenomenologists \mdash{} re-evaluate the last century
with computers replacing logic as the instuitional
mechanization of thought?  How should Philosophy
be reconsidered if some founding books were reimagined
as, let's say, the \i{Formal and Transcendental
Computer Science} or the
\i{Tractatus Computational-Philosophical}?  To speak more
precisely, what are the consequences propagating across
Philosophy if quantification is foundationally
conceived as over type-instances rather than sets?  What
changes when the domain of a quantification has to have a
conceptual unity at least to the level of what can be modeled via
a formal type theory rather than being open-ended sets?
What changes when our reigning paradigm of perfect
reasoning is not proofs as a mental exercise, but the
engineering of computer systems and then the engineering
of (formal representations of) theories and then of apparent
theorems so as feed theories and theorems  into the aforementioned
computer systems for verification?  What changes when even mathematics
becomes a rather empirical domain that can be experimented
with on a computer?  What kind of Philosophy can
be done by experimenting on a computer?
}
\p{I will pick up this thread of discussion in a later section,
but beforehand will make explicit the potential
theoretical integration between linguistics and
\i{procedural networks}, which I have assumed but
not directly analyzed to this point.
}
\p{}
\p{}
