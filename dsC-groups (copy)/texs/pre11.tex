\atsp
\begin{frame}{\ft{Features of Dataset Applications for Books}}
\vspace{-13em}
\hspace*{6pt}\OneQuad
{
\begin{quadblock}{Datasets Ccompiled From Book Examples}
\hspace{1cm}{\parbox{19cm}{\LARGE
\fontseries{b}\selectfont
{
The remaining screenshots demonstrate how data sets can be 
used even outside of a lab contxt generating 
experiment data.  The pictured data set represents a corpus 
of linguistic examples mined from Wiley's \textit{Blackwell Handbook 
of Pragmatics}.  Creating data sets from book-length publications 
can encompass several steps:
\vspace{1em} 
\begin{description}
\item[Text Mining] In the case of linguistics, this involves 
locating example sentences within linguistics texts and 
storing them as an independent corpus.
\item[Canonical Formatting] If possible, linguistics 
texts should be annotated so that extracting exmples can be automated.
\item[Annotation] Linguistic corpuses are often annotated 
to identify structural details, beyond raw text, in 
each sample.
\end{description}
}
}}
\end{quadblock}
}
\end{frame}
