\atsp
\begin{frame}{\ft{Group 1: Features of Dataset Applications}}
	
%{\begin{minipage}{35cm}\ft{Group 1: Features of Dataset Applications}\hspace{7.4cm}\colorbox{purple!40!black}{\hspace{10cm}}\end{minipage}}
\vspace{-13em}
\OneQuad
{
\begin{quadblock}{User Interface Features Typical of Dataset Applications}
\hspace{1cm}{\parbox{19cm}{\LARGE
\fontseries{b}\selectfont
{
The code for each dsC data set includes a 
customized \curlyquote{Dataset Application} which displays 
individual samples and groups of samples via 
2D, 3D, and native-compiled GUI controls.  
Each Dataset Application can thereby make use of 
advanced visual and interactive features that 
are uniquely possible when using 
customized, native-compiled GUI classes.  
The following screenshots will 
show several examples of these features, including:
\vspace{1em} 


\begin{description}
\item[Specialized Top-Level Controls] Tree Widgets, Stacked Widgets, and Graphics Scenes.
\item[Context Menus] Systematically organize functionality around UI layouts.
\item[Multi-Window Displays] Divide application functionality in multiple 
specialized top-level windows and/or dialog boxes.
\end{description}
}
}}
\end{quadblock}
}
\end{frame}
