\documentclass{article}
%\documentclass{standalone}

% LinkAndDependencyGrammarAnnotations

\usepackage[active,pdftex,tightpage]{preview}
\usepackage{tikz}

\definecolor{logoYellow}{RGB}{255, 249, 232}
\definecolor{blGreen}{rgb}{.2,.7,.3}
\definecolor{darkRed}{rgb}{.3,0,0}
\newcommand{\sovn}[1]{\color{yellow!20!logoYellow}{{\textbf{#1}}}}

\newcommand{\mybox}[1]{\colorbox{grred!10!darkRed}{\parbox{2mm}{\sovn{#1}}}}

\newcommand{\circled}[1]{{\mybox{#1}}}


\PreviewEnvironment{tikzpicture}

% remove "[demo]" if you want include actual image!!!
\usepackage{graphicx}

\usepackage{tikz}

% LaTeX Overlay Generator - Annotated Figures v0.0.1
% Created with http://ff.cx/latex-overlay-generator/

\definecolor{postBkgColor}{rgb}{.95,.85,.95}
\definecolor{postCommentBkgColor}{rgb}{.85,.85,.95}

\definecolor{grammarArrowColor}{rgb}{.85,.85,.45}

\colorlet{brred}{brown!53!red}
\colorlet{grred}{grammarArrowColor!40!red!60}

\definecolor{logoRed}{rgb}{.3,0,0}
\definecolor{logoPeach}{RGB}{255, 159, 102}
\definecolor{logoCyan}{RGB}{66, 206, 244}
\definecolor{logoBlue}{RGB}{4, 2, 25}

\colorlet{lplr}{logoPeach!40!logoRed}

%%%%%%%%%%%%%%%%%%%%%%%%%%%%%%%%%%%%%%%%%%%%%%%%%%%%%%%%%%%%%%%%%%%%%%
%\annotatedFigureBoxCustom{bottom-left}{top-right}{label}{label-position}{box-color}{label-color}{border-color}{text-color}
\newcommand*\annotatedFigureBoxCustom[8]{\draw[#5,thick,rounded corners] (#1) rectangle (#2);\node at (#4) [fill=#6,thick,shape=circle,draw=#7,inner sep=2pt,font=\sffamily,text=#8] {\textbf{#3}};}
%\annotatedFigureBox{bottom-left}{top-right}{label}{label-position}
\newcommand*\annotatedFigureBox[4]{\annotatedFigureBoxCustom{#1}{#2}{#3}{#4}{lplr}{grred!30!black}{brred}{brred!20}}
\newcommand*\annotatedFigureText[4]{\node[draw=none, anchor=south west, text=#2, inner sep=0, text width=#3\linewidth,font=\sffamily] at (#1){#4};}
\newenvironment {annotatedFigure}[1]{\centering\begin{tikzpicture}
    \node[anchor=south west,inner sep=0] (image) at (0,0) { #1};\begin{scope}[x={(image.south east)},y={(image.north west)}]}{\end{scope}\end{tikzpicture}}
%%%%%%%%%%%%%%%%%%%%%%%%%%%%%%%%%%%%%%%%%%%%%%%%%%%%%%%%%%%%%%%%%%%%%%

\begin{document}

    \begin{figure}[h!t]

        \begin{annotatedFigure}
            {\includegraphics[scale=1]{x}}
            
  \node [text width=15cm,align=justify,fill=logoCyan!20, draw=logoBlue, 
  draw opacity=0.5,line width=1mm, fill opacity=0.9]
   at (0.51,0.73){\textbf{This screenshot show the current 
   data set as originally presented in a spreadsheet.  
   The point of including this example is to show that 
   spreadsheet applications do not have an obvious 
   mechanism for researching individual modeling 
   elements, such as column labels.  For instance, there 
   is no explanation near the labels ``WithFlow" 
   or ``Against" which explain what these parameters mean and how 
   they are used.}};


%\annotatedFigureBox{0.93,0.02}{0.985,0.945}{1}{0.985,0.945}%                
%\annotatedFigureBox{0.005,0.82}{0.43,0.98}{2}{0.43,0.82}%
%\annotatedFigureBox{0.01,0.1}{0.55,0.334}{3}{0.55,0.334}            
            
      %      \annotatedFigureBox{0.222,0.284}{0.3743,0.4934}{B}{0.3743,0.4934}%tr
      %      \annotatedFigureBox{0.555,0.784}{0.6815,0.874}{C}{0.555,0.784}%bl
      %      \annotatedFigureBox{0.557,0.322}{0.8985,0.5269}{D}{0.8985,0.5269}%tr
  

  
        \end{annotatedFigure}

   %     \caption{Expanded Sample (A)}
    %    \label{fig:teaser}

    \end{figure}

\end{document}
