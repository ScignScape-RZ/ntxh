\section{Hub Applications and Gatekeeper Code} 
\phantomsection\label{sOne}
\p{To begin, I will speak in general terms about hub applications and 
about the unique coding challenges which derive from Cyber-Physical
technologies' unique networking and safety requirements.
Implementing software hubs 
introduces technical difficulties which are distinct from 
manufacturing Cyber-Physical instruments themselves \mdash{} in particular,
devices are usually narrowly focused on a particular 
kind of data and measurement, while software hubs are 
multi-purpose applications that need to understand and 
integrate data from many different kinds of 
devices.  Hub applications also 
present technical challenges that are different from 
other kinds of software, even if 
these hubs are one specialized domain in the larger 
class of user-focused software.
}
\p{Any software application provides human users with tools to 
interactively and visually access data and computer 
files, either locally (data encoded on the \q{host} computer running 
the software) or remotely (data accessed over a network).  
Computer programs can be generally classified as 
\i{applications} (which are designed with a priority 
to User Experience) and \i{background processes} 
(which often start and maintain their state automatically 
and have little input or visibility to human users, 
except for special troubleshooting circumstances).  
Applications, in turn, can be generally classified as 
\q{web applications} (where users typically see one 
resource at a time, such as a web page, 
and where data is usually stored 
on remote servers) and \q{native applications} 
(characterized by more complex \GUI{} components, and 
by the ability to work with \q{local} data 
\mdash{} data stored on users' computers or accessible 
on a local network \mdash{} instead of or in addition to 
data acquired from a web service).  
}
\p{From a software engineering point of view, I 
believe we should conceptualize hub software as
native, desktop-style applications which 
leverage native \GUI{} features.  
Hub applications therefore embody a 
fundamentally different User Experience 
than other kinds of Cyber-Physical access
points, like touch screens or phone apps.
}
\p{To cite a concrete example, Teixeira \i{et. al.} 
describe a refrigerator \q{that notifies the user 
if the door stays accidentally open} and moreover 
\q{knows what is inside the refrigerator and [its] 
expiration dates}.  Consider then how we would design a 
User Interface networking with \q{smart} refrigerators.  
A simple indicator showing whether 
doors are open is straightforward, but an interface listing 
the items inside is much more complicated.  
Once a refrigerator can detect signals 
emanating from food items/containers, we can envision a list of 
items presented as a \GUI{} component, perhaps 
one food item per line 
(each line, say, showing a picture, price, text description, 
and expiration date).  This would require cross-referencing 
numeric codes, which might be broadcast by the 
items \i{inside} the refrigerator, against a 
database that would load the images, descriptions, dates, 
and prices.
}
\p{One question 
is then where this database would be hosted, 
and how it would be 
updated (insofar as food companies develop new products 
fairly often; any static database could quickly get 
outdated).  Food companies (or some middleware agent) 
would have to agree on a common format so that the 
refrigerator's access software can integrate data from 
many brands.  Since a refrigerator can hold many items, 
the \GUI{} would also need enough screen space 
(and maybe a multi-level design) for users to comfortably 
browse many artifacts; perhaps a line-by-line window 
supplemented with separate dialog windows for each item.  
}
\p{On the other hand, a phone app interfacing with that 
same data would be constrained by its lesser 
screen real estate and limited interactive modalities. 
With no room on-screen to show a 
complete list of items \mdash{} and no obvious 
gesture to navigate between individual and multi-item 
views \mdash{} such an app might choose to list only those 
items nearing their expiry date.  In general, when 
adapting to phone-like usage patterns (smaller screen, 
brief but frequent user engagement), designers 
have to offer compact but curated 
snippets of information.  That is, software is forced to anticipate 
what info carries the most user interest \mdash{} expiry 
dates are probably most important to users when items are 
near perishing.  This means 
that data mining, Artificial Intelligence, and other 
techniques for anticipating users' needs becomes 
proportionately more consequential: if the whole 
\GUI{} design is premised on \AI{}, then \AI{} 
ceases to be just a useful tool, augmenting 
software's analytic reach \mdash{} it 
becomes instead a make-or-break User Experience necessity.  
}
\p{Conversely, if we assume that hub applications will 
adopt the \q{look and feel} of native desktop applications, 
then they can present more holistic information \mdash{}  
taking advantage of larger screens, with secondary 
application windows and other interactive features that 
we associate with native \GUI{} components.  It is a 
reasonable hypothesis that this renders \AI{} less important: 
the more data that can be shown, the less need for 
software to filter information on users' behalf.     
As Teixera \i{et. al.} put it, \q{some authors argue that 
the number of interactions between users and the smart home 
must be kept to a minimum}, but 
\q{to remove obstacles in the adoption 
of smart home systems ... preserving the autonomy of the 
user may seem like the most sensible course of action}.  
In that spirit, investments in \AI{} solutions might be 
redirected to \HCI{}, securitization, data transparency, 
and other software virtues which customers may value 
more than \q{smart} software that actually strips them of 
control.
}
\p{Hub applications could therefore exemplify what 
Teixera \i{et. al.} call \q{user-centric} design.  
In this guise, hubs 
have at least three key responsibilities: 
\begin{enumerate}\item{} To present device and system data for human 
users, in graphical, interactive formats suitable 
for people to oversee the system and intervene 
as needed.
\item{} To validate device and system data, ensuring 
that the system is behaving correctly and predictably.
\item{} To log data (in whole or in part) for subsequent 
analysis and maintenance.
\end{enumerate}
Once software receives device data, it needs to 
marshal the information between different formats, 
exposing data in the different contexts of 
\GUI{} components, database storage, and 
analytic review, to confirm proper operation of  
devices and their handler code.
}
\p{The more rigorously that engineers understand and document 
the morphology of information across these different  
software roles, the more clearly we can 
define protocols for software design and user expectations.  
Careful design requires answering many technical questions: 
how should the application respond if it encounters 
unexpected data?  How, in the presence of erroneous data, 
can we distinguish device malfunction from coding error?  
How should users and/or support staff 
be notified of errors?  What is the optimal Interface Design 
for users to identify anomalies, or identify situations 
needing human intervention, and then be able to 
perform the necessary actions via software?  
What kind of database should hold system data retroactively, 
and what kind of queries or analyses should engineers 
be able to perform so as to study system data, to access the 
system's past states and performance? 
}
\p{Because Cyber-Physical devices are intrinsically \i{networked}
\mdash{} whether over special wireless networks or the World Wide Web \mdash{} 
there is an enlarged \q{surface area} for vulnerability.  Moreover, 
because they are often worn by people
or used in a domestic setting, they tend carry personal (e.g., location)
information, making network security protocols especially important
(\cite{RonaldAshri}, \cite{AbhishekDwivedi}, \cite{LalanaKagal},
\cite{TakeshiTakahashi}, \cite{MozhganTavakolifard},
\cite{BhavaniThuraisingham}).  In brief, the dangers
of coding errors and software vulnerabilities, in Cyber-Physical
Systems like the Internet of Things (\IoT{}), are even more pronounced
than in other application domains.  While it is
unfortunate if a software crash causes someone to lose data,
it is even more serious if a Cyber-Physical \q{dashboard}
application were to malfunction and leave physical, networked
devices in a dangerous state.
}
\p{It is helpful at this point to distinguish cyber 
\i{security} from \i{safety}.  When these concepts are 
separated, \i{security} generally refers to
preventing \i{deliberate}, \i{malicious} intrusion into 
Cyber-Physical networks.  Cyber \i{safety} refers to preventing
unintended or dangerous system behavior due to innocent human 
error, physical malfunction, or incorrect programming.  
Malicious attacks \mdash{} in particular the risks of 
\q{cyber warfare} \mdash{} are prominent in the 
public imagination, but innocent coding errors or design 
flaws are equally dangerous.  Incorrect data readings, 
for example, led to recent Boeing 737 MAX jet accidents 
causing over 300 fatalities (plus the worldwide grounding
of that airplane model and billions of dollars in losses 
for the company).  Software failures either 
in runtime maintenance or anticipatory risk-assessment 
have been identified as contributing factors to 
high-profile accidents like Chernobyl \cite{MikhailMalko} 
and the Fukushima nuclear reactor 
meltdown \cite{JoonEonYang}.
A less tragic but noteworthy 
case was the 1999 crash of NASA's US \$125 million 
Mars Climate Orbiter.  This crash was caused by 
software malfunctions which in turn were due 
to intercommunicating software components producing 
incompatible data \mdash{} in particular, employing 
incompatible scales of measurement 
(resulting in an unanticipated mixture of 
imperial and metric units).  In general, it 
is reasonable to assume that coding errors 
are among the deadliest and costliest sources 
of man-made injury and property damage. 
}
\p{Given the risks of undetected data corruption, seemingly 
mundane questions about how Cyber-Physical applications verify
data \mdash{} and respond to apparent anomalies \mdash{} 
become essential aspects of planning and development.  
Consider even a simple data aggregate like 
blood pressure (combining systolic and 
diastolic measurements).  Empirically, systolic pressure is 
always greater than diastolic.  Software systems 
need commensurately to agree on a protocol for encoding the 
numbers to ensure that they are in the correct order, and that they 
represent biologically plausible measurements.  
How should a particular software component test that 
received blood pressure data is accurate?  Should it 
always test that the systolic quantity is indeed 
greater than the diastolic, and that both numbers 
fall in medically possible ranges?  How should the 
component report data which fails this test?  If 
such data checking is not performed \mdash{} on the 
premise that the data will be proofed elsewhere 
\mdash{} then how can this assumption be 
justified?
}
\p{In general, how can engineers identify, in a 
large and complex software system, all the points 
where data is subject to validation tests; and 
then by modeling the overall system in terms  
of these check-points ensure that all needed 
verifications are performed at least one time?  
Continuing the blood-pressure example, 
how would a software procedure that \i{does} 
check the integrity of the systolic/diastolic 
pair indicate for the overall system model 
that it performs that particular verification?  
Conversely, how would a procedure which does 
\i{not} perform that verification indicate 
that this verification must be performed 
elsewhere in the system, to guarantee that 
the procedure's assumptions are satisfied?    
} 
\p{These questions are important not only for objective, 
measurable assessments of software quality, but 
also for people's more subjective trust in the reliability 
of software systems.  In the modern world we 
allow software to be a determining factor in 
systems' behavior, in places
where malfunction can be fatal \mdash{} airplanes, hospitals, 
electricity grids, trains carrying toxic chemicals, 
highways and city streets, etc.  
Consider the model of \q{Ubiquitous Computing} pertinent to the
book series to which this volume (and hence
this chapter) belongs.  As explained in the
series introduction:\urlfootnote{https://sites.google.com/view/series-title-ausah/home?authuser=0} 
\begin{dquote}U-healthcare systems ... will allow physicians to remotely diagnose, access, and monitor critical patient's symptoms and will enable real time communication with patients.  [This] 
series will contain systems based on the four future ubiquitous sensing for healthcare (USH) principles, namely i) proactiveness, where healthcare data transmission to healthcare providers has to be done proactively to enable necessary interventions, ii) transparency, where the healthcare monitoring system design should transparent, iii) awareness, where monitors and devices should be tuned to the context of the wearer, and iv) trustworthiness, where the personal health data transmission over a wireless medium requires security, control and authorize access.
\end{dquote}
Observe that in this scenario, patients will have to 
place a level of trust in Ubiquitous Health technology comparable 
to the trust that they place in human doctors and other 
health professionals.   
}
\p{All of this should cause software engineers and developers to 
take notice.  Modern society places trust in doctors 
for well-rehearsed and legally scrutinized reasons: 
physicians have to prove their competence 
before being allowed to practice medicine, and 
this right can be revoked due to malpractice.  Treatment 
and diagnostic clinics need to be licensed,
and pharmaceuticals (as well as medical equipment) are subject 
to rigorous testing and scientific investigation before being 
marketable.  Notwithstanding \q{free market} ideologies, 
governments are aggressively involved in caretaking
their healthcare systems; commercial activities (like marketing) are
regulated, and operational transparency 
(like reporting adverse outcomes) is mandated, more so 
than in most other sectors of the economy.  This level of 
oversight \i{causes} the public to trust that clinicians' 
recommendations are usually correct, or that medicines are 
usually beneficial more than harmful.  
}
\p{The problem, as software becomes an increasingly central feature 
of the biomedical ecosystem, is that no commensurate oversight 
framework exists in the software world.
Biomedical \IT{} regulations tend to be ad-hoc and narrowly domain-focused. 
For example, code bases in the United States which manage \HLSeven{}
data (a standardized \EMR{} \mdash{} Electronic Medical Record \mdash{} format) must 
meet certain requirements, but there is no comparable framework 
for software targeting other kinds of health-care information.  
This is not only \mdash{} or not primarily \mdash{} an issue of 
lax government oversight.  The deeper problem is that 
we do not have a clear picture, in the framework of 
computer programming and software development, of 
what a robust regulatory framework would look like: what 
kinds of questions it would ask; what steps a company could 
follow to demonstrate regulatory compliance; what indicators 
the public should consult to verify that any software 
that could affect their medical outcome has been properly vetted.
}
\p{In the United States, \HLSeven{} (which encompasses multiple 
clinical, continuity-of-care, and decision-support 
formats) is a good example of how technology can be backed 
with robust certification and Quality Assurance layers 
(the \q{\lclc{7}} refers to the seventh, or 
application-specific, layer of a technology stack which 
collectively enables network-based digital sharing of 
health records and other bioinformatic content). 
Government-supported programs such as  
ONC Health IT Certification (ONC stands for \q{Office of the National 
Coordinator for Health Information Technology}), which maintains 
a network of software-evaluation 
labs and testing infrastructure to vet Health IT products 
(called \q{modules}), confirm in particular that software 
claiming \HLSeven{} compliance can properly \i{consume} and 
\i{query} information presented to the system via document formats 
or data structures included in the \HLSeven{} family of standards.  
This gives developers clear guidelines of what behaviors their 
applications must emulate in order to legitimately 
participate in biomedical data sharing; and also 
allows doctors and patients to garner more information about 
Health IT vendors' solutions. 
}
\p{The limitation of \HLSeven{} is, however, that its robust 
evaluation methodology only applies to the collection of 
data formats standardized in the \HLSeven{} family.  
Consequently, any data generated by new treatments or 
technologies \mdash{} whether innovations in clinical methods, 
Cyber-Physical medical devices, pharmaceuticals, 
computational methodology, or any other research 
\mdash{} will remain outside the certification pipeline unless 
the new kind of generated data can be translated into 
one of the \HLSeven{} formats.  Some new research may be 
amenable to conventional \EMR{} standards: clinical 
trials of a new medication, for example, may fit 
comfortably in existing reporting formats because 
the desired outcomes \mdash{} lowering blood pressure, lowering 
viral load, shrinking tumors, and so forth \mdash{}  
can be measured and expressed via standard records. 
However, in many cases the whole premise of 
clinical innovations is to rethink 
analytic methodologies; or clinical technology 
is introduced which generates new kind of 
data (reflecting the novel physical design 
of diagnostic equipment, for example) 
that needs to be included in medical records 
according to its own, internal data model.
}
\p{As a concrete example, consider Eran Bellin's \q{Clinical Looking Glass} 
(\CLG{}) model, developed at Montefiore Medical Center in New York City 
\cite{EranBellin}.  
Inspired by \q{patient-centered} care trends, \CLG{} seeks to 
organize clinical data around \q{patient-centric time frames} 
and the structural representation of patients' trajectory through 
a medical system.  Diagnoses, for example, are then 
recorded as diagnostic \i{events} (to use an 
example from \CLG{}'s documentation, diabetes may be recorded 
as the event of a lab test 
reporting HemoglobinA1C above \lclc{9.5}), and 
subsequent time-points and time-spans are defined as offsets 
from such \q{index events}.  The clinical rationale for 
\CLG{} data models is to facilitate \q{patient-centered} 
studies of medical data \mdash{} analyzing for instance the 
length of time between patients' starting a treatment course 
and the point at which a desired target outcome is achieved; Bellin argues 
that conventional clinical software is designed around 
\q{business-volume-centric} queries which originate from 
hospitals' administrative requirements rather than 
patient-centered concerns, which makes it more difficult 
to query normal clinical databases in a patient-centered manner 
\cite[pages 5-6]{BellinExplore}.  The \CLG{} software counters
that trend by developing a \q{patient-centered} 
query methodology (and a \GUI{} for doctors to graphically 
formulate queries), along with a higher-scale 
\q{cohort model} where patients are aggregated 
(while the patient-centered data models 
are preserved at the cohort level).  
Internally, this query system depends upon an \q{object model} 
whose diagnostic, temporal, and demographic architecture 
differs from conventional Health IT representations.  
Moreover, the \CLG{} application can pool databases 
from multiple sources, so that data sharing and 
integration is an intrinsic \CLG{} feature.  
}
\p{Currently \CLG{} is embodied in a single proprietary 
application, but if the \CLG{} query framework 
is appreciated by doctors as noticeably more effective, 
it is easy to imagine that the \CLG{} object model 
will be generalized from its specific application context 
and repositioned as a object-system which diverse 
software components could adopt.  In this scenario, 
given that \CLG{} is expressly formulated as an alternative 
to traditional \EMR{}s, it is reasonable to assume that the 
\CLG{} object model would engender its own \q{consume and query} 
protocols \mdash{} for data sharing, validation, and analysis \mdash{} 
which, consequently, would be too divergent from \HLSeven{} formats 
for the \HLSeven{} certification process to be 
applicable.  In that instance, comparable rigor would thus 
demand either a separate certification initiative 
specific to \CLG{} or an informal (and therefore 
unregulated) commitment on the 
part of developers to comply with behavioral standards 
appropriate for the \CLG{} object model.  
}
\p{The case of \CLG{} is just one example of clinical innovation; 
given the (thankfully) fertile ground for research on new treatments and 
technologies, we can see our healthcare system as 
targets for many analogous innovations from the 
biochemical, bioinformatics, and Cyber-Physical domains.  
The sheer volume of new data models makes it likely 
that \q{data-centric} certification 
as in \HLSeven{} \mdash{} standardizing a predefined set of 
data models \mdash{} will continually become outdated.  
For this reason Requirements Engineering should 
look more toward a \q{software-centric} paradigm, 
where developers consciously adopt design patterns 
abstracted from particular data models.\footnote{As a case-study, the dataset for this chapter 
includes an object model with some similarities to 
\CLG{}; the documentation and protocols associated 
with query evaluators for that model demonstrate 
procedural-level Requirements Engineering using 
techniques that can be applied across 
many data profiles \mdash{} shifting the 
weight of certification, or code evaluation, away from 
the specific data model and toward procedural-level 
implementational maxims.   
} 
}
\p{Outside the medical arena, similar analyses \mdash{} of 
the need for robust but dynamic evolving 
vetting of new technologies \mdash{} could be 
made regarding software in Cyber-Physical settings like
transportation, energy (power generation and electrical 
grids), physical infrastructure, environmental protections, 
government and civic data, and so forth 
\mdash{} settings where software errors threaten personal
and/or property damages.  The public has a relatively inchoate 
idea of issues related to cyber safety, security, and 
privacy: we (collectively) have an informal impression that 
current technology is failing to meet the public's desired 
standards, but there is no clear picture of what 
\IT{} engineers can or should do to improve the technology 
going forward.
}
\p{Regulatory oversight is only effective in proportion to 
scientific clarity \visavis{} desired outcomes.  
Drugs and treatment protocols, 
for instance, can be evaluated through \q{gold standard} 
double-blind clinical trials \mdash{} alongside statistical 
models, like \q{five-sigma} criteria, which measure 
scientists' confidence that trial results are truly 
predictive, rather than results of random chance.  This package 
of scientific methodology provides a framework which can 
then be adopted in legal or legislative contexts.  
With respect to medications, policy makers can stipulate that 
pharmaceuticals should be tested in double-blind trials, 
with statistically verifiable positive results, before 
being approved for general-purpose clinical use.  Such a  
well-defined policy approach \i{is only possible} because 
there are biomedical paradigms which define how treatments 
can be tested to maximize the chance that positive test 
results predict similar results for the 
general patient population.
}
\p{Analogously, a general theory of cyber safety should 
be a software-design issue before it becomes a 
policy or contractual issue.  Software engineering and 
programming language design needs its own evaluative 
guidelines; its own analogs to double-blind trials 
and five-sigma confidence.  It is at the region of
low-level software design \mdash{} of actual source code 
in its local implementation and holistic integration 
\mdash{} that engineers can develop technical \q{best practices} 
which then provide substance to regulative oversight.  
Stakeholders or governments can recommend (or require) that 
certain practices adopted, but only if engineers 
identify coding standards which are believed, 
on firm theoretical grounds, to effectuate safer, 
more robust software.  
}
\vspace{-.1em}
\subsection{Gatekeeper Code}
\p{There are several design principles which can help ensure 
safety in large-scale, native/desktop-style \GUI{}-based 
applications.  These include:
\begin{enumerate}\item{}  Identify operational relationships between types.  
Suppose \calS{} is a data structure modeled via type \caltypeT{}.  
This type can then be associated with a type (say, 
\typeTp{}) of \GUI{} components which visually display 
values of type \caltypeT{}.  A simple data structure 
may have \GUI{} representation via small \q{widgets} 
embedded in other components (consider a thermometer icon 
to display temperature).  Conversely, if \calS{} has many component 
parts, its corresponding \GUI{} type may need to span its 
own application window, with a collection of nested textual 
or graphical elements.  There may also be a type 
(say, \typeTpp{}) representing \caltypeT{}-values in a format 
suitable for database persistence.  Application code should 
explicitly indicate these sorts of inter-type relationships.
\item{}  Identify coding assumptions which determine the validity 
of typed values and/or function calls.  For each 
application-specific data type, consider whether every 
computationally possible instance of that type is actually 
meaningful for the real-world domain which the type represents.  
For instance, a type representing blood pressure has a subset 
of values which are biologically meaningful \mdash{} where systolic 
pressure is greater than diastolic and where both numbers are 
in a sensible range.  Likewise, for every procedure defined 
on application-specific data types, consider whether the procedure 
might receive arguments that are computationally feasible but 
empirically nonsensical.  Then, establish a protocol for 
acting upon erroneous data values or procedure parameters.  
How should the error be handled, without disrupting the 
overall application?
\item{}  Identify points in the code base which represent new data 
being introduced into the application; and code which can materially
affect the \q{outside world}.  Most of the code behind \GUI{} 
software will manage data being transferred between different 
parts of the system, internally.  However, there will be 
specific implementations for 
receiving new data from external sources or signals.  
These are places where data 
\q{enters the system}.\footnote{A simple example is, for desktop applications, 
the preliminary code which runs when users click a mouse button.  
In the Cyber-Physical context, an example might be code which
is activated when motion-detector sensors signal something moving 
in their vicinity.
}  Conversely, other code points localize 
the software's capabilities to initiate external effects.\footnote{For 
instance, one consequence of users clicking a mouse button might 
be that the on-screen cursor changes shape.  Or, motion detection 
might trigger lights to be turned on.  In these cases the software 
is hooked up to external devices which have tangible capabilities, 
such as activating a light-source or modifying the on-screen cursor.
}  The functions which leverage such capabilities 
reveal data \q{leaving the system}.  
Distinguishing where data 
\q{enters} and \q{leaves} the system from where data is transferred 
\i{inside} the application helps ensure that 
incoming data and external effects are properly vetted.\footnote{Several mathematical frameworks have been developed 
to codify the intuition of software components as 
\q{systems} with external data sources and effects, 
extending the model of software as self-contained 
information spaces: notably, Functional-Reactive Programming 
(see e.g. \cite{WolfgangJeltsch}, \cite{JenniferPaykin},
\cite{PaykinKrishnaswami}) and (a little more indirectly)
the theory of Hypergraph Categories
(\cite{InteractingConceptualSpaces}, \cite{BrendanFong}, 
\cite{BrendanFongThesis}, \cite{AleksKissinger}). 
}
\end{enumerate}
Methods I propose in this chapter are applicable to each 
of these concerns, but for purposes of exposition I 
will focus on the second issue: testing 
type instances and procedure parameters for fine-grained 
specifications (more precise than strong typing alone). 
}
\p{Strongly-typed programming language offer some guarantees on 
types and procedures: a function which takes an \int{} will
never be called on a value that is \i{not} an integer 
(e.g., the character-string \q{\codeText{46}} instead of the \i{number} 
\lclc{46}).  Likewise, a type where one field is an \int{}
(representing someone's age, say), will never be instantiated 
with something \i{other than} an \int{} in that field.
Such minimal guarantees, however, are too coarse for 
safety-conscious programming.  Even the smallest 
\unsignedint{} type (8-bit) would permit someone's age to
be \lclc{255} years, which is surely an error.  So any 
safety-conscious code dealing with ages needs to check that 
the numbers fall in a range narrower than built-in 
types allow on their own, or to ensure that such checks are 
performed ahead of time.   
}
\p{The central technical challenge of safety-conscious coding 
is therefore to \i{extend} or \i{complement} each programming 
languages' built-in type system so as to represent 
more fine-grained assumptions and specifications.  
While individual tests may seem straightforward on a 
local level, a consistent 
data-verification architecture \mdash{} how this coding dimension 
integrates with the totality of software features and 
responsibility \mdash{} can be much more complicated.  
Developers need to consider several overarching questions, 
such as: 
\begin{itemize}\item{} Should data validation be included in the same 
procedures which operate on (validated) data, or 
should validation be factored into separate procedures?
\item{} Should data validation be implemented at the type 
level or the procedural level?  That is, should specialized 
data types be implemented that are guaranteed only to 
hold valid data?  Or should procedures work with more 
generic data types, and perform validations on a case-by-case 
basis?
\item{} How should incorrect data be handled?  In Cyber-Physical software,
there may be no obvious way to abort an operation in the 
presence of corrupt data.  Terminating the application may not be 
an option; silently canceling the desired operation or trying to substitute 
\q{correct} or \q{default} data may be unwise; and 
presenting technical error messages to human users may be confusing.  
\end{itemize}
These questions do not have simple answers.  As such, we 
should develop a rigorous theoretical framework so as to 
codify the various options involved \mdash{} what architectural 
decisions can be made, and what are the strengths and weaknesses 
of different solutions.
}
\p{I will use the term \i{gatekeeper code} for any code which checks 
programming assumptions more fine-grained than strong typing 
alone allows \mdash{} for example, that someone's age is not reported 
as \lclc{255} years, or that systolic pressure is not recorded as 
less than diastolic.  I will use the term \i{fragile code} for
code which \i{makes} programming assumptions \i{without itself} 
verifying that such assumptions are obeyed.  Fragile code is 
especially consequential when incorrect data would cause the 
code to fail significantly \mdash{} to crash the application, 
enter an infinite loop, or any other nonrecoverable scenario.
}
\p{Note that \q{fragile} is not a term of criticism \mdash{} some algorithms 
simply work on a restricted space of values, and it is inevitable 
that code implementing such algorithms will only behave properly 
when provided values with the requisite properties.  It is necessary 
to ensure that such algorithms are \i{only} called with 
correct data.  But insofar as testing of the data lies outside 
the algorithms themselves, the proper validation has to occur 
\i{before} the algorithms commence.  In short, \i{fragile} and
\i{gatekeeper} code often has to be paired off: for each 
segment of fragile code which \i{makes} assumptions, there should 
be a corresponding segment of gatekeeper code which
\i{checks} those assumptions.  
}
\p{In that general outline, however, there is room for a variety 
of coding styles and paradigms.  Perhaps these can be broadly 
classified into three groups: 
\begin{enumerate}\item{} Combine gatekeeper and fragile code in one procedure.
\item{} Separate gatekeeper and fragile code into different procedures.
\item{} Implement narrower types so that gatekeeper code is 
called when types are first instantiated.
\end{enumerate}
Consider a function which calculates the difference between 
systolic and diastolic blood pressure, returning an unsigned
integer.  If this code were called with malformed data wherein 
systolic and diastolic were inverted, the difference would 
be a negative number, which (under binary conversion to an 
unsigned integer) would come out as a potentially 
extremely large positive number (as if the patient had 
blood pressure in, say, the tens-of-thousands).  This nonsensical 
outcome indicates that the basic calculation is fragile.  
We then have three options: test \q{systolic-greater-than 
diastolic} \i{within the procedure}; require that this test 
be performed prior to the procedure being called; 
or use a special data structure configured such that 
systolic-over-diastolic can be confirmed as soon as 
any blood-pressure value is constructed in the system.
}
\p{There are strengths and weaknesses of each option.  
Checking parameters at the start of a procedure makes 
code more complex and harder to maintain, and also 
makes updating the code more difficult.  The 
blood-pressure case is a simple example, but in real 
situations there may be more complex data-validation 
requirements, and separating code which \i{checks} 
data from code which \i{uses} data, into different 
procedures, may simplify subsequent code maintenance.
If the \i{validation} code needs to be modified 
\mdash{} and if it is factored into its its own procedure \mdash{}  
this can be done without modifying the 
code which actually works on the data (reducing the 
risk of new coding errors).  
}
\p{Also, engineers may appreciate the flexibility to upgrade how 
improper data is handled \i{throughout the application}.  
Suppose it is decided to log, and periodically 
review, all instances of malformed 
parameters \q{rejected} by gatekeeper code.  
Every \ifthenelse{} block could potentially then 
need to be paired with logging code, and it would be 
time-consuming and error-prone to verify that such protocol 
is obeyed everywhere, throughout a large, complex code base.  
Isolating gatekeeper procedures, and labeling them as such, 
would make it 
both easier to find all gatekeeping logic and 
to modify the protocol for 
handling failed validations.
}
\p{In essence, factoring 
\i{gatekeeper} and \i{fragile} code into separate 
procedures exemplifies the programming maxim of 
\q{separation of concerns}: it makes the overall 
system more flexible and easier to maintain.  
However, such separation creates a new problem \mdash{} ensuring that
the gatekeeping procedure is always called.  
Meanwhile, using special-purpose, narrowed data types 
adds complexity to the overall project if these data types
are unique to that one code base, and therefore 
incommensurate with data provided by external sources.  
In these situations the software must transform data between 
more generic and more specific representations before 
sharing it (as sender or receiver), which makes 
the code more complicated.  
} 
\p{Because there is no one best \q{gatekeeping protocol}, these 
issues should be studied holistically, defining a range of 
options that can be weighed at the planning and prototyping 
stages \mdash{} well before 
most of the serious production code is implemented. 
} 
\p{In the specific Cyber-Physical context, gatekeeping is especially
important when working with device data.  Such data is 
almost always constrained by the physical construction of 
devices and the kinds of physical quantities they measure 
(if they are sensors) or their physical capabilities 
(if they are \q{actuators}, devices that cause changes in their 
environments).  For sensors, it is an empirical question what 
range of values can be expected assuming proper functioning 
(and therefore what validations can check that the 
instrument is working as intended).  For actuators, it should 
be similarly understood what range of values guarantee 
safe, correct behavior.  For any device then we can 
construct a \i{profile} \mdash{} an abstract, mathematical 
picture of the space of \q{normal} values associated with 
proper device performance.  Gatekeeping code can 
thereby ensure that data received from or sent to devices
fits within the profile.  Defining device profiles, and 
explicitly notating the corresponding gatekeeping code, 
should accordingly be an essential pre-implementation planning
step for Cyber-Physical software hubs.
}
\subsection{Fragile Code}
\p{Fragile code is code that makes assumptions stronger than the 
programming language on its own can guarantee.  Where 
safety and quality is a priority, fragile code needs gatekeeping 
code to ensure that these assumptions are warranted.
}
\p{Fragile code is not necessarily a harbinger of poor design.  Sometimes
implementations can be optimized for special circumstances, and
optimizations are valuable and should be used wherever possible.  Consider an
optimized algorithm that works with two lists that must be the same size.
Such an algorithm should be preferred over a less efficient
one whenever possible \mdash{} which is to say, whenever dealing with two
lists which are indeed the same size.  Suppose this algorithm is
included in an open-source library intended to be shared among many different
projects.  The library's engineer might, quite reasonably, deliberately
choose not to check that the algorithm is invoked on same-sized lists
\mdash{} checks that would complicate the code, and sometimes slow the
algorithm unnecessarily.  It is then the responsibility of code that
\i{calls} whatever procedure implements the algorithm to ensure that it
is being employed correctly \mdash{} specifically, that this
\q{client} code does \i{not} try
to use the algorithm with \i{different-sized} lists.  Here \q{fragility} is
probably well-motivated: accepting that algorithms are sometimes
implemented in fragile code can make the code cleaner, its intentions
clearer, and permits their being optimized for speed.
}
\p{The opposite of fragile code is sometimes called \q{robust} code.
While robustness is desirable in principle, code which simplistically
avoids fragility may be harder to maintain than deliberately fragile but
carefully documented code.  Robust code often has to check for many
conditions to ensure that it is being used properly, which can make
the code harder to maintain and understand.  The hypothetical
algorithm that I contemplated last paragraph
could be made robust by \i{checking}
(rather than just \i{assuming}) that it is invoked with same-sized lists.
But if it has other requirements \mdash{} that the lists are non-empty,
and so forth \mdash{} the implementation can get padded with a chain of
preliminary gatekeeper code.  In such cases the gatekeeper
code may be better factored into a different procedure, or expressed
as a specification which engineers must study before attempting to
use the implementation itself.
}
\p{Such transparent declaration of coding assumptions and specifications can
inspire developers using the code to proceed attentively,
which can be safer in the long run than trying to avoid fragile code
through engineering alone.  The takeaway is that while \q{robust} is
contrasted with \q{fragile} at the smallest scales (such as
a single procedure), the overall goal is systems and components that are robust at the
largest scale \mdash{} which often means accepting \i{locally} fragile
code.  Architecturally, the ideal design may combine
individual, \i{locally fragile} units with rigorous documentation and 
gatekeeping so that the \i{totality} is robust.  
Defining and declaring specifications is then 
an intrinsic part of implementing code bases which are both robust
and maintainable.
}
\p{Unfortunately, specifications are often created
only as human-readable documents, which might have a semi-formal
structure but are not actually machine-readable.
There is then a disconnect between features \i{in the code itself} that
promote robustness, and specifications intended for \i{human} readers
\mdash{} developers and engineers.  The code-level and
human-level features promoting robustness will tend to overlap partially
but not completely, demanding a complex evaluation of where gatekeeping
code is needed and how to double-check via
unit tests and other post-implementation examinations.  This is the
kind of situation \mdash{} an impasse, or partial but incomplete overlap,
between formal and semi-formal specifications \mdash{} which many programmers
hope to avoid via strong/expressive type systems.
}
\p{Advanced type-theoretic constructs \mdash{} including Dependent Types,
typestate, and effect-systems \mdash{} model requirements with more precision
than can be achieved via conventional type systems alone.  Integrating these
paradigms into core-language type systems permits data validation 
to be integrated with general-purpose type checking, without the need for
static analyzers or other \q{third party} tools (that is, projects maintained
orthogonally to the actual language \mdash{} i.e., to
compilers and runtimes).  This means that Requirements Specifications 
that would otherwise be \i{human} documents \mdash{} \communiques{} between 
developers, but only at best partially enforced by programming languages 
internally \mdash{} get formalized to the point where compilers (for example) 
evaluate requirements without human intervention.  Unfortunately, 
these advanced type systems are also more complex to implement.  
If Software Language Engineers aspire to make Dependent Types and
similar advanced constructs part of their core language, 
creating compilers and runtime engines for these languages 
becomes proportionately more difficult.
} 
\p{Programming languages are, at one level, artificial 
\i{languages} \mdash{} they allow humans to communicate 
algorithms and procedures to computer processors, and 
to one another.  But programming languages are also 
themselves engineering artifacts.  It is a complex
project to transform textual source-code \mdash{} which is 
human-readable and looks a little bit like natural 
language \mdash{} into binary instructions that computers 
can execute.  For each language, there is a stack 
of tools \mdash{} parsers, compilers, and/or runtime libraries 
\mdash{} which enable source code to be executed 
according to the language specifications.  
Language design is therefore constrained by 
what is technically feasible for these supporting tools.  
Practical language design, then, is an interdisciplinary 
process which needs to consider both the dimension of 
programming languages as communicative media and 
as digital artifacts with their own engineering challenges 
and limitations.
}
\p{These limitations then produce a split between 
tools \i{in the language itself} and those 
maintained as separate projects \i{analyzing} code 
in a given language.  They raise the question, 
which has no simple answer, of what 
should be guaranteed \i{by the language} 
and what should be tested externally.  I will now 
examine this question in greater detail.
}
\vspace{-.1em}
\subsection{Core Language vs. External Tools}
\p{Because of programming languages' engineering limitations, 
such as I just outlined, software projects should not 
necessarily rely on core-language features for 
responsible, safety-conscious programming.
In short, methodologies for safety-conscious coding can be 
split between those which depend on core-language features, 
and those which rely on external, retroactive analysis 
of sensitive code.  On the one hand, some languages and projects
prioritize specifications that are intrinsic to the language and integrate
seamlessly and operationally into the language's foundational
compile-and-run sequence.  Improper code (relative to specifications)
should not compile, or, as a last resort, should fail gracefully at run-time.
Moreover, in terms of programmers' thought processes, the
description of specifications should be intellectually continuous
with other cognitive processes involved in composing code, such
as designing types or implementing algorithms.  For sake of 
discussion, I will call this paradigm \q{internalism}.  
}
\p{The \q{internalist} mindset seeks to integrate data 
validation seamlessly with other language features.  
Malformed data should be flagged via similar mechanisms 
as code which fails to type-check; and errors should 
be detected as early in the development process as possible.   
Such a mindset is evident in passages like this (describing
the Ivory programming language):
\begin{dquote}Ivory's type system is shallowly embedded within Haskell's
type system, taking advantage of the extensions provided by [the
Glasgow Haskell Compiler].  Thus, well-typed Ivory programs
are guaranteed to produce memory safe executables, \i{all without
writing a stand-alone type-checker} [my emphasis].  In contrast, 
the Ivory syntax is \i{deeply} embedded within Haskell.
This novel combination of shallowly-embedded types and 
deeply-embedded syntax permits ease of development without sacrificing
the ability to develop various back-ends and verification tools [such as]  
a theorem-prover back-end.  All these back-ends share the
same AST [Abstract Syntax Tree]: Ivory verifies what it compiles.
\cite[page 1]{ivory}.
\end{dquote}   In other words, the creators of Ivory are promoting the
fact that their language buttresses via its type system 
\mdash{} and via a mathematical precision suitable for 
proof engines \mdash{} 
code guarantees that for most languages require external
analysis tools.
}
\p{Contrary to this \q{internalist} philosophy, other approaches
(perhaps I can call them \q{externalist}) favor a neater separation
of specification, declaration and testing from the core language.  
In particular \mdash{} according to 
the \q{externalist} mind-set \mdash{} most of the more important or complex
safety-checking does not natively integrate with the
underlying language, but instead requires
either an external source code analyzer, or 
regulatory runtime libraries, or some combination of the two.  
Moreover, it is unrealistic
to expect all programming errors to be avoided with enough proactive planning,
expressive typing, and safety-focused paradigms: any complex
code base requires some retroactive design, some combination
of unit-testing and mechanisms (including those
third-party to both the language and the projects whose code is
implemented in the language) for externally
analyzing, observing, and higher-scale testing for the code,
plus post-deployment monitoring.
}
\p{As a counterpoint to the features cited as benefits to the
Ivory language, which I identified as representing the 
\q{internalist} paradigm, consider Santanu Paul's Source Code Algebra (\SCA{})
system described in \cite{SantanuPaul} and
\cite{GiladMishne}, \cite{TillyEtAl}:
\begin{dquote}Source code files are processed using
tools such as parsers, static analyzers, etc. and the necessary information
(according to the SCA data model) is stored in a repository.  A user interacts
with the system, in principle, through a variety of high-level languages, or
by specifying SCA expressions directly.  Queries are mapped to SCA expressions,
the SCA optimizer tries to simplify the expressions, and finally, the SCA
evaluator evaluates the expression and returns the results to the user.\nl{}
\hspace{2em} We expect that many source code queries will be expressed using high-level
query languages or invoked through graphical user interfaces.  High-level queries
in the appropriate form (e.g., graphical, command-line, relational, or
pattern-based) will be translated into equivalent SCA expressions.  An SCA
expression can then be evaluated using a standard SCA evaluator, which
will serve as a common query processing engine.  The analogy from
relational database systems is the translation of SQL to expressions based on
relational algebra. \cite[page 15]{SantanuPaul}
\end{dquote}
So the \i{algebraic} representation of source code is favored
here because it makes computer code available
as a data structure that can be processed via \i{external}
technologies, like \q{high-level languages}, query languages, and
graphical tools.  The vision of an optimal development environment
guiding this kind of project is opposite, or at least
complementary, to a project like Ivory: the whole point
of Source Code Algebra is to pull code verification \mdash{} the
analysis of code to build trust in its safety and robustness
\mdash{} \i{outside} the language itself and into the surrounding
Development Environment ecosystem.
}
\p{These philosophical differences (what I dub \q{internalist} vs. \q{externalist}) 
are normative as well as descriptive:
they influence programming language design, and how languages in turn influence
coding practices.  One goal of language design is to produce languages 
which offer rigorous guarantees \mdash{} fine-tuning their 
type system and compilation model to maximize the level of detail 
guaranteed for any code which type-checks and compiles.  
Another goal of language design is to define syntax and 
semantics permitting valid source code to be analyzed 
as a data structure in its own right.  Ideally, 
languages can aspire to both goals.  In practice, however, 
achieving both equally can be technically difficult.  
The internal representations conducive to strong type and 
compiler guarantees are not necessarily amenable to 
convenient source-level analysis, and vice-versa.    
}
\p{Language engineers, then, have to work with
two rather different constituencies.  One community of
programmers tends to prefer that specification and validation be
integral to/integrated with the language's type system and
compile-run cycle (and standard runtime environment); whereas
a different community prefers to treat code evaluation
as a distinct part of the development process, something logically, operationally,
and cognitively separate from hand-to-screen codewriting
(and may chafe at languages restricting certain code constructs
because they can theoretically produce coding errors, even when
the anomalies involved are trivial enough to be tractable for
even barely adequate code review).  One challenge for language engineers is
accordingly to serve both communities.  We can, for example, aspire to
implement type systems which are sufficiently
expressive to model many specification, validation, and
gatekeeping scenarios, while also anticipating that language code
should be syntactically and semantic designed to be
useful in the context of external tools (like
static analyzers) and models (like Source Code
Algebras and Source Code Ontologies).
}
\p{The techniques I discuss here work toward these goals on two levels.  First, I
propose a general-purpose representation of computer code in terms
of Directed Hypergraphs, sufficiently rigorous to codify a
theory of functional types as types whose values are (potentially) initialized from
formal representations of source code \mdash{} which is to say, in the present
context, code graphs.  Next, I
analyze different kinds of \q{lambda abstraction} \mdash{} the idea of
converting closed expressions to open-ended formulae by asserting that
some symbols are \q{input parameters} rather than fixed values, as in
Lambda Calculus \mdash{} from the perspective of
axioms regulating
how inputs and outputs may be passed to and obtained from
computational procedures.  I bridge these topics \mdash{} Hypergraphs
and Generalized Lambda Calculi \mdash{} by taking abstraction as a
feature of code graphs wherein some hypernodes are singled out
as procedural
\q{inputs} or \q{outputs}.  The basic form of this model
\mdash{} combining what are essentially two otherwise unrelated
mathematical formations, Directed Hypergraphs and
(typed) Lambda Calculus \mdash{} is laid out in
Sections \sectsym{}\hyperref[sThree]{\ref{sThree}}
and \sectsym{}\hyperref[sFour]{\ref{sFour}}.
I then engage a more rigorous study of
code-graph hypernodes as \q{carriers} of runtime values, some of
which collectively form \q{channels} concerning values which
vary at runtime between different executions of a function body.
Carriers and channels piece together to form
\q{Channel Groups} that describe structures with meaning both
within source code as an organized system (at \q{compile time}
and during static code analysis) and at runtime.  Channel Groups
have four different semantic interpretations, varying via the
distinctions between runtime and compile-time and between
\i{expressions} and (function) \i{signatures}.
I use the framework of Channel Groups to identify
design patterns that achieve many goals of
\q{expressive} type systems while being implementationally
feasible given the constraints of mainstream programming
languages and compilers.
}
\p{Prior to launching into that mostly theoretical discussion, however, 
I will examine real-world Cyber-Physical data with a little more specificity.
My goal in the following pages is to describe the sorts of 
details that need to be expressed in Cyber-Physical data models
and therefore verified in Cyber-Physical code.  Whereas my subsequent
analyses of code representation will address \i{how} code can demonstrate
its underlying data model, my preliminary discussion will 
motivate \i{why} code needs to do so in the first place.  
}
