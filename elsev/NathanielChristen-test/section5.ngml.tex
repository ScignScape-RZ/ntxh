\section{Channels and Carriers}
\phantomsection\label{sFive}
\p{Suppose one procedure calls a second.  From a high level perspective,
this has several consequences that can be semantic-graph represented
\mdash{} among others, that the calling procedure depends on an
implementation of the callee being available \mdash{} but at the
source code level the key consequence is that a node representing
source tokens which designate functional values enters into
different semantic relations (modeled by different kinds
of edge-annotations) than nodes marking other types
of values and literals.  Suppose we have an edge-annotation
that \nodex{} is a value passed to \nodef{}; this graph is only
semantically well-formed if \nodef{}'s representatum has
functional type (by analogy to the well-formedness 
criteria of \lambdaxfx{}).
}
\p{This motivates the following: suppose we have a Directed
Hypergraph, where the
nodes for each hyper-edge represent source-code tokens (specifically,
symbols and literals).  Via the relevant Source Code Ontology, we can
assert that certain
edge-annotations are only possible if a token (in subject or object position)
designates a value passed to a function.  From the various edge-annotation
kinds which meet this criteria, we can define a set of \q{channel kinds}.
}
\p{This implicitly
assumes that symbols \q{hold} values; to make the
notion explicit, I will say that symbols are
\i{carriers} for values.  Carriers do not necessarily hold a
value at every point in the execution of a program; they
may be \q{preinitialized}, and also \q{retired} (the
latter meaning they no longer hold a meaningful value;
consider deleted pointers or references to out-of-scope
symbols).  A carrier may pass through a
\q{career} from preinitialized to initialized, maybe then
changing to hold \q{different} values, and maybe then
retired.\footnote{Because \q{uninitialized} carriers and \q{dangling pointers}
are coding errors, within \q{correct} code, carriers
and values are bound tightly enough that the whole carrier/value
distinction might be considered an artifact of programming practice,
out of place in a rigorous discussion of programming languages
(as logicomathematical systems, in some sense).  But even
if the \q{career} of symbols is unremarkable, we cannot avoid
in some contexts \mdash{} within a debugger and/or an \IDE{}
(Integrated Development Environment, software for writing programs),
for example \mdash{} needing to formally distinguish the carrier
from the value which it holds, or recognize that carriers
can potentially be in a \q{state} where, at some point in which
they are relevant for code analysis or evaluation, they
do not yet (or do not any longer) hold meaningful values.
Consequently, the \q{trajectory} of 
carrier \q{lifetime} \mdash{} from being declared,
to being initialized, to falling \q{out of scope} or otherwise \q{retired}
\mdash{} should be integrated into our formal inventory of programming
constructs, not relegated to an informal \q{metalanguage} suitable
for discussing computer code as practical documents but not as formal systems.
}  I assume each carrier is associated with a single type
throughout its career, and can only hold values appropriate for its
type.\footnote{The variety of possible
careers for carriers is not directly tied to its type: a
carrier which cannot change values (be reinitialized)
is not necessarily holding a \const{}-typed value.
}
}
\p{Via carrier \i{careers} we then have a ready account of
the transformations effectuated by a procedure:
in particular, \returnch{} or \exceptionch{}
(or other output channels') carriers transition from,
before the procedure begins, being preinitialized, to,
after it ends, being initialized (depending on
how the procedure exits).  Input carriers may also
be modified (i.e., undergo a state-change).  As such,
procedures result in \i{(potential) changes to the state} of
those carriers within their signature channel-complex (and
other carriers sharing values with them).  \i{Morphisms}
in carrier-states thereby play a logical role akin to
\i{reduction} in \mOldLambda{}-Calculi.
}
\p{In short, \i{carriers} embody the contrast between abstract or
mathematical type theory and practical languages' type systems.  
Instead of the (still rather abstract and circular)
notion of a \i{typed value} \mdash{} an instance of a
type \mdash{} we can focus on carriers which are tangible elements of 
source code and also (during runtimes) binary resources.  
Carriers evince different states; in those states where they 
hold a concrete value, carriers play a conceptual role analogous 
to type-instances in formal type theory.  On the other hand, 
carriers can have other states which are orthogonal to 
type systems: carriers holding \i{no} value, for example, 
are different than carriers holding values of a 
\q{\nulltt{}} type.    
}
\p{With this the basic idea, I will now consider
carrier operations in more detail, before then expanding on the
theory of carriers by considering how carriers group into channels.
}
\vspace{-.1em}
%\spsubsection{Carrier Transfers}
\subsection{Carrier Transfers}
\p{In this theory, carriers are the basic means by which values are represented
within computer code, including during communications between different parts of 
code source (such as calling a procedure).
The \q{information flow} modeled by a function-call includes values
held by carriers at the function-call site being transferred to
carriers at the function-implementation site.  This motivates
the idea of a \q{transfer} of values between carriers, a kind of primitive
operation on carriers, linking disparate pieces of code.  It also
illustrates that the symbols used to name function parameters, as
part of function signatures, should be considered \q{carriers} analogous
to lexically-scoped symbols.
}
\p{Taking this further, we can define a \i{channel} as a list of carriers which, by
inter-carrier transfers, signify (or orchestrate) the passage of data into and out
of function bodies.\footnote{Note that this usage varies somewhat from process calculi,
where a channel would correspond roughly to what is here called a single carrier;
I assume channels in the general case are composed of multiple carriers 
(see e.g. \cite{GregRestall} and \cite{ZhaohuiLuoSignatures}, or \cite{ToninhoEtAl}).
}  I'll use the notation \opTransfer{} to represent inter-carrier transfer: let
\carrOne{} and \carrTwo{} be carriers, then \carrOneOpTransferTwo{} is a
transfer \q{operator} (note that \opTransfer{} is non-commutative; the
\q{transfer} happens in a fixed direction), marking the
logical moment when a value is moved from code-point to code-point.
The \opTransfer{} is intended
to model several scenarios, including \q{forced coercions} where the associated
value is modified. 
Meanwhile, without further details a \q{transfer} can be
generalized to \i{channels} in multiple ways.
If \carrOne{} and \carrTwo{} are carriers which belong to two channels
(\chanOne{}, \chanTwo{}), then \carrOneOpTransferTwo{} elevates to
a transfer between the channels
\mdash{} but this needs two indices to be concrete: the notation has to
specify which carrier in \chanOne{} transfers to which carrier in \chanTwo{}.
For example, consider the basic function-composition \fofg{}:
\fDotOfGX{} = \fOfGx{}.  The analogous \q{transfer} notation
would be, say, \gOpTransferOneOneF{}:
here the first carrier in the \returnch{} channel
of \funG{} transfers to the first carrier in the \lambdach{} channel of \funF{}
(the subscripts indicate the respective positions).
}
\p{Most symbols in procedure code (so corresponding nodes in a
code graph) accordingly represent carriers, which
are either passed in to a function or lexically declared
in a function body.  Assume each function body corresponds
with one lexical scope which can have subscopes
(the nature of these scopes and how they fit in
graph representation will be addressed later in this section).
The \i{declared} carriers are initialized with
values returned from other functions (perhaps the
current function called recursively), which can
include constructors that work on literals (so, the
carrier-binding in source code can look like
a simple assignment to a literal, as in \intieqzero{}).
In sum, whether they are passed \i{to} a  function or
declared \i{in} a function, carriers are only initialized
\mdash{} and only participate in the overall semantics
of a program \mdash{} insofar as they are passed to other functions
or bound to their return values.
}
\p{Furthermore, both of these cases introduce associations between
different carriers in different areas of source code.  When a carrier
is passed \i{to} a function, there is a corresponding carrier
(declared in the callee's signature) that receives the
former's value: \q{calling a function} means
transferring values between carriers present at the site of
the function call to those present in the function's
implementation.  Sometimes this works in reverse: a function's
return may cause the value of one of its carriers to be
transferred to a carrier in the caller (whatever
carrier is bound to the caller's return value).
}
\p{Let \carOne{} and \carTwo{} be two carriers.  The \opTransfer{} operator
(representing a value passed from \carOne{} to \carTwo{}) encompasses
several cases.  These include:
\begin{enumerate}\eli{}  Values transfer directly between two carriers in one scope, like \aeqb{} or \aceqb{}.
\eli{}  A value transferred between one carrier in one function body when the return
value of that function is assigned to a carrier at the call site, as in \yeqfx{}
when \fSym{} exits with \retFive{}, so the value \five{} is transferred to \ySym{}.
\eli{}  \label{transf}A value transferred between a
carrier at a call-site and a carrier in the
called function's body.  Given \yeqfx{} and \fFun{} declared as, say,
\fIntI{}, then the value in carrier \xSym{} at the call-site is transferred to
the carrier \iSym{} in the function body.  In particular, every node
in the called function's code-graph whose vertex represents a source-code
token representing symbol \iSym{} then becomes a carrier whose value is
that transferred from \xSym{}.
\eli{}  A value transferred between a \returnch{} channel and either a
\lambdach{} or \sigmach{} channel, as part of a nested expression or a
\q{chain of method calls}.  So in \hfx{}, the value held by the carrier
in \fFuns{}'s \returnch{} channel is transferred to the first carrier in
\hFun{}'s \lambdach{}.  An analogous \returnch{}\opTransfer{}\sigmach{}
transfer is seen in code like \fxdoth{}: the value
in \fFuns{}'s \returnch{} channel becomes the value in \hFun{}'s \sigmach{},
i.e., its \q{\this{}} (we can use \opTransfer{}
as a notation between \i{channels} in this case because we understand
the Channel Algebra in force to restrict the size of both
\returnch{} and \sigmach{} to be at most one carrier).
\end{enumerate}
}
\p{Let \carOnetoTwof{} be the special case of \opTransfer{}
corresponding to item \hyperref[transf]{(3)}: a transfer
effectuated by a function call, where
\carOne{} is at the call site and \carTwo{} is part of a
function's signature.  If \fOne{} calls \fTwo{} then
\carOne{} is in \fOne{}'s context, \carTwo{} is in \fTwo{}'s
context, and \carTwo{} is initialized with a copy of
\carOne{}'s value prior to \fTwo{} executing.
A \i{channel} then becomes a
collection of carriers which are found in the
scope of one function and can be on the
right hand side of an \carOnetoTwof{} operator.
}
\p{To flesh out Channels' \q{transfer semantics} further, I will refer back
to the model of function-implementations as represented in code
graphs.  If we assume that all code in a computer program is found
in some function-body, then we can assume that any function-call
operates in the context of some other function-body.  In particular,
any carrier-transfer caused by a function call involves a link between
nodes in two different code graphs (I set aside the case of recursive
functions \mdash{} those which call themselves \mdash{} for this discussion).
}
\p{Analysis of value-transfers is particularly significant
in the context of Source Code Ontologies and \RDF{} or
Directed Hypergraph representations
of computer code.  This is because code-graphs give us a rigorous foundation
for modeling computer programs as sets of function-implementations
which call one another.  Instead of abstractly talking about
\q{procedures} as conceptual primitives, we can see procedures as
embodied in code-graphs (and function-values as
constructed from them).  
\begin{figure}
\centering
\begin{tikzgrid}

 \node[]  (xy) [ellipse,minimum width=13mm,minimum height=9mm,
 inner sep=2pt, outer sep=0pt,draw=blue!40!cyan,fill=blue!20]                  {}; 
    \node[]  (xl) [below left=-0.5mm and 3mm of xy, circle,minimum size=2mm,
    inner sep=2pt, outer sep=0pt,draw=darkRed,fill=white]                  {$x$}; 
    \node[]  (yl) [right=6mm of xl,circle,minimum size=2mm,
    inner sep=1.5pt, outer sep=0pt,draw=darkRed,fill=white]                  {$y$}; 
    

    
  
    \node[]  (ghostll)   [below right=6mm and 2mm of xy]  {};    
    
    \node[]  (fng)   [right=of ghostll]  {\gFunB{$x$}{$y$}}; 

    \node[]  (ghostl)   [right=8mm of fng]  {};    
    \node[]  (ghostt)   [above right=3.5mm of fng]  {};    
    \node[]  (ghostb)   [below right=3.5mm of fng]  {};    

    \node[]  (comment)   [above right=9.5mm and 8mm of fng,
    fill=white,draw=yellow!45!gray,text width=43mm,inner sep=3pt]  
    {{\scriptsize \textit{$x$ and $y$ (being lexically scoped)\\[-6pt] 
      remain initialized between {\gFunB} and {\fFunB}}}};

\node[draw=darkRed!50,
fill=red!20,thin,minimum width=11mm,minimum height=11mm]  (channell)  
[above right=1mm and 12mm of fng]  {}; 

\node[draw=darkRed!50,
fill=red!20,thin,minimum width=11mm,minimum height=21mm]  (channelr)  
[above right=-3.5mm and 34mm of fng]  {}; 

    \node[circle,minimum size=2mm,
    inner sep=2pt, outer sep=0pt,draw=darkRed,fill=white]  (xx)    [right= of ghostt]              {$x$}; 
    \node[circle,minimum size=2mm,
    inner sep=1.5pt, outer sep=1pt,draw=darkRed,fill=white]  (yy)    [right= of ghostb]              {$y$}; 



    \node[]  (zz)    [circle,minimum size=2mm,
    inner sep=2pt, outer sep=2pt,draw=darkRed,fill=white,below right= of fng]              {$z$}; 
 
    \node[]  (ghostr)   [right= of zz]  {}; 
     
    \node[]  (fnf)   [above right= of ghostr]  {\fFunB{$x$}{$y$}{$z$}}; 
        \node[]  (ghosttf)   [above left=3.5mm of fnf]  {};    
        \node[]  (ghostbf)   [below left=3.5mm of fnf]  {};   
        \node[]  (ghostbb)   [below =3mm of fnf]  {};   




\node[draw,line width=1pt,minimum width=1mm,minimum height=1mm ]  (carrt)   
  [right= of xx, xshift = -5mm]  {}; 
\node[draw,line width=1pt,minimum width=1mm,minimum height=1mm ]  (carrb)   
  [right= of yy, xshift = -5mm]  {};   

\draw[draw=blGreen!30!black,thick,->>,>=stealth,bend left=30]  (xx) 
to (carrt);

\draw[draw=blGreen!30!black,thick,->>,>=stealth,bend left=30]  (yy) 
to (carrb);


 %   \path[->]
    \draw   (xy) edge [below,  bend right=10, ->,-to]  node {\textit{input}} (fng);
    \draw   (xy) edge [bend left=10, ->,-to]  node {} (fng);
   
    \draw   (fng) edge [below, ->]  node [xshift = -10mm] {\textit{output}} (zz);
    
\node[draw,line width=2pt,minimum width=1mm,minimum height=1mm ]  (carz)   
  [below right= 12mm of fng]  {};    
    
    \draw   (zz) edge [below, ->,-to]  node [xshift=2mm]{\textit{input}} (ghostbb);

\node[draw,line width=1pt,minimum width=1mm,minimum height=1mm ]  (carzi)   
[above right= 7mm and 20.5mm of zz]  {}; 

    \draw   (zz) edge [->>,draw=blGreen!30!black,thick,
      >=stealth,bend left=19]  (carzi);
      
%    \draw   (carz) edge [->>,draw=blGreen!30!black,thick,
%    >=stealth,bend right=180,out=180,in=180]  (zz);


\node (gg) [below left=-1mm of carz.south] {};
\node (ggg) [below=3.1mm of carz.south] {};
\draw[draw=blGreen!30!black,thick]  (gg) 
  arc[radius=2mm,start angle=90,delta angle=200];
  
\draw[draw=blGreen!30!black,thick,->>,>=stealth]  (ggg) 
 to (zz.west);

    \draw   (ghostt) edge [->]  node {} (xx);
    \draw   (ghostb) edge [->]  node {} (yy);

    \draw   (xx) edge [->,-to]  node {} (ghosttf);
    \draw   (yy) edge [->,-to]  node {} (ghostbf);


    \node[]  (code)  [below right = 28mm and 38mm of xy]   
     {\lstinlinebstyle{z = g(x, y); f(x, y, z)}}; 

    \node[]  (keyof)   [below left = 35mm and 3mm of xy]  {}; 
    \node[]  (keyff)   [below right = 20mm and 79.5mm of keyof]  {}; 

    \node[draw=yellow!25,line width=1mm,fill=cyan!8,
      fit={(keyof) (keyff)}] {};  

    \node[draw=yellow!25!black,line width=.5mm,
    fit={(keyof) (keyff)}] {};  

%    \draw[draw=yellow!25!black,line width=1mm] (keyof) rectangle (keyff);  


    \node[draw,line width=2pt,minimum width=1mm,minimum height=1mm ]  (keyo)  
       [below right=1mm and 3mm of keyof]  {}; 
    \node[right = 2mm of keyo.west]  {: output carriers}; 
    
    \node[draw,line width=1pt,minimum width=1mm,minimum height=1mm ]  (keyi)   [right = 45mm  of keyo]  {}; 
    \node[right = 2mm of keyi.west]  {: input carriers}; 

    \node (kx)   [below = 6mm of keyo,circle,minimum size=2mm,
    inner sep=2pt, outer sep=0pt,draw=darkRed,fill=white]  {$x$}; 
    \node (ky)   [right = 5mm of kx,circle,minimum size=2mm,
    inner sep=1.5pt, outer sep=0pt,draw=darkRed,fill=white]  {$y$};
    \node[below right = -3.5mm and 4mm of ky.west,text width=70mm]  {: %
    	\parbox[t][][t]{60mm}{carriers embedded in lexically-\nl scoped symbols}};     

 
    \node (keyaa)   [below = 9mm of kx]  {}; 
    \node (keyab)   [right = 6.2mm of keyaa]  {}; 
    \draw[draw=blGreen!30!black,thick,->>,>=stealth]  (keyaa.west) 
    to (keyab.east);
    
    \node[below right = -3.5mm and 2mm of keyab.west] (txt)  {: %
    	\parbox[t][][t]{60mm}{handoffs (value transferred between \nl carriers)}};   

%    \draw[fill=yellow,opacity=.1] (keyof) rectangle (keyff);  

    
%   \draw (fnf.north)+(-19mm,2mm) edge [yshift=5mm,
%     shorten <=-19mm, shorten >= 2mm, bend right = 41, <-]  
%     node [yshift=5mm,xshift=-12mm] {} (fng);

%   \draw (fnf.north)+(-14mm,2mm) edge [yshift=1mm,
%     shorten <=-11mm, shorten >= 2mm, bend right = 24, <-]  
%     node [yshift=5mm,xshift=-12mm] {} (fng);



 
\end{tikzgrid}
\tmphscol\caption{\emblink{\#fig--GFcomplexc--\thefigure.pgvm-pdf}{Visualizing Carrier Transfers (\q{Handoffs})}} \label{fig:figGFcomplexc}
\end{figure}
Figure~\ref{fig:figGFcomplexc}, for example, 
shows how graph constructions can track the flow of 
carriers and values between procedures: this graph-modeling 
use-case (with additional illustrations) is discussed in greater 
detail within the demo's documentation.
\q{Passing values between} procedures is
then explicitly a family of relationships between nodes
(or hypernodes) in disparate code-graphs, and the various semantic
nuances associated with some such transfers (type casts, for example)
can be directly modeled by edge-annotations.  Given these
possibilities, I will now explore further how the framework of
\i{carriers} and \i{channels} fits into a code-graph context.
}
\vspace{-.1em}
%\spsubsection{Channel Groups and Code Graphs}
\subsubsection{Channel Groups and Code Graphs}
\p{For this discussion, assume that \fOne{} and \fTwo{} are implemented functions with
code graphs \gammaOne{} and \gammaTwo{}, respectively.  Assume
furthermore that some statement
or expression in \fOne{} involves a call to \fTwo{}.  There are several specific
cases that can obtain: the expression calling \fTwo{} may be nested in a
larger expression; \fTwo{} may be called for its side effects alone,
with no concern to its return value (if any); or the result of \fTwo{}
may be bound to a symbol in \fOne{}'s scope, as in \yeqfx{}.  I'll take
this third case as canonical; my discussion here extends to the other
cases in a relatively straightforward manner.
}
\p{A statement like \yeqfx{} has two parts: the expression \fx{} and the symbol
\ySym{} to which the results of \fFun{} are assigned.  Assume that this
statement occurs in the body of function \fOne{}; \xSym{} and
\ySym{} are then symbols in \fOne{}'s scope and the symbol \fFun{} designates
(or resolves to) a function which corresponds to what I refer to here as
\fTwo{}.  Assume \fTwo{} has a signature like \fIntI{}  As such, the
expression \fx{}, where \xSym{} is a carrier in the context of
\fOne{}, describes a \i{carrier transfer} according to which the value
of \xSym{} gets transferred to the carrier \iSym{} in \fTwo{}'s context.
}
\p{I will say that \fTwo{}'s signature represents a
channel \q{complex} \mdash{} which, in the current example, has
a \lambdach{} channel of size one (with one carrier of
type \int{}) and a \returnch{} channel of size one (\fTwo{}
returns one \int{}).  Considered in the context of carrier-transfers
between code graphs, a channel complex may be regarded as a description
of how two distinct code-graphs are to be connected via carrier
transfers.  When a function is called, there is a channel
group which I'll call a \i{package} that supplies values to
the channel \i{complex}.  In the concrete example, the
statement \yeqfx{} is a \i{call site} describing a channel
\i{package}, which becomes connected to a function
implementation whose signature represents a channel
\i{complex}: a collection of transfers \carrOneOpTransferTwo{}
together describe an overall transfer between a
\i{package} and a \i{complex}.
}
\p{More precisely, the \fx{} example represents a carrier transfer whose
target is part of \fTwo{}'s \lambdach{} channel, which we can notate
\carrOneOpTransferTwolambda{}.  Furthermore, the full statement \yeqfx{}
shows a transfer in the opposite direction: the value in
\fTwo{}'s \i{\returnch{}} channel is transferred to the carrier \ySym{} in
the \i{package}.  This relation, involving a return
channel, can be expressed with notation like \carrTwoOpTransferOnereturn{}.
The syntax of a programming language governs how code at a call site
supplies values for carrier transfers to and from a function body: in
the current example, binding a call-result to a symbol always
involves a transfer from a \i{\returnch{}} channel, whereas handling an
exception via code like \catchexce{} transfers a value from a called
function's \i{\exceptionch{}} channel.  The syntactic difference between
code which takes values from \returnch{} and \exceptionch{} channels,
respectively, helps reinforce the \i{semantic} difference between
exceptions and \q{ordinary} returns.  Similarly, method-call syntax
like \objfx{} visually separates the values that get transferred to a
\q{\sigmach{}} channel (\obj{} in this case)
from the \q{ordinary} (\lambdach{}) inputs, reinforcing Object semantics.
}
\p{To consolidate the terms I am using: we can interpret both function
\i{signatures} and \i{calls} in terms of channels.  Both involve
\q{carrier transfers} in which values are transferred \i{to} or \i{from}
the channels described by a function signature.  I will say that 
channels are \i{convoluted} if there is a potential carrier-transfer 
between them.  The distinction between
procedures' \q{inputs} and \q{outputs} can be more rigorously stated,
with this further background, as the distinction between channels
in function signatures which receive values \i{from} carriers at
a call site (inputs), and those \i{from which} values are obtained
as a procedure has completed (outputs).
}
\p{A Channel Expression Language (\CXL{}) can describe channels both in
signatures and at call-sites.  The aggregation of channels generically
described by \CXL{} expressions I am calling a \i{Channel Group}.
A Channel Group representing a function \i{signature} I am calling
a \i{channel complex}, whereas groups representing a function
\i{call} I am calling a \i{channel package}.  Input channels
are then those whose carrier transfers occur in the package-to-complex
direction, whereas output channels are the converse.
}
\p{Alongside the package/complex distinction, we can also
understand Channel Groups at two further levels.  On the one hand,
we can treat Channel Groups as found \i{in source code},
where they describe the general pattern of package/complex
transfers.  On the other hand, we can represent Channel Groups
\i{at runtime} in terms of the actual values and types held by
carriers as transfers are effectuated prior to, and then
after, execution of the called function.  Accordingly,
each Channel Group may be classified as a \i{compile-time}
package or complex, or a \i{runtime} package or complex, respectively.
The code accompanying this chapter includes a \q{Channel Group library}
\mdash{} for creating and analyzing Channel Groups via a special
Intermediate Representation \mdash{}
that represents groups of each variety, so it can be used
both for static analysis and for enhanced runtimes and scripting.
}
\p{The channel/group/complex/package/carrier vocabulary,  
I believe, codifies a descriptive framework integrating the
\i{semantic} and \i{syntactic} dimensions of source code and program
execution.  Specifically, on the \i{semantic} side, 
computer programs can be understood in
terms of \mOldLambda{}-Calculi combined with
models of computation (call-by-value or
by-reference, eager and lazy evaluation, and so forth).  These
semantic analyses focus on how values change and are passed between
functions during the course of a running program.  From this
perspective, source code is analyzed in terms of the semantics
of the program it describes: where and when are values moved around?
}
\p{Conversely, source code
can also be approached \i{syntactically}, as well-formed
expressions of a computer language.  From this perspective,
correct source code is matched against language grammars, and 
individual code elements
(like tokens, code blocks, expressions, and statements) \mdash{} plus
their inter-relationships \mdash{} are established against this background.
}
\p{The theory of Channel Groups straddles both the semantic and syntactic
dimensions of computer code.  Semantically, carrier-transfers capture the
fundamental building blocks of program semantics: the overall
evolving runtime state of a program can be modeled as a succession
of carrier-transfers, marking
code-points bridged via a transfer.  Meanwhile,
syntactically, how carriers belong to channels \mdash{}
the carrier-to-channel map fixing carriers' semantics \mdash{} structures and
motivates languages' grammars and rules.  In particular,
carrier-transfers induce relationships between code-graph nodes.
As a result, language
grammars can be studied through code-graphs' profiles
insofar as they satisfy \RDF{}
and/or \DH{} Ontologies.
}
\p{In sum, a \DH{} and/or Semantic Web representation of computer code
can be a foundation for both semantic and syntactic analyses, and this
may be considered a benefit of Channel Group representations
even if they only restate what are established semantic patterns in 
mainstream programming language \mdash{} for example, even if they
are restricted to a \sigmach{}-\lambdach{}-\returnch{}-\exceptionch{}
Channel Algebra modeled around, say, \Cpp{} semantics
prior to \CppEleven{} (more recent \Cpp{} standards also call for
a \q{\capturech{}} channel for inline anonymous functions).
}
\p{At the same time, one of my claims in this chapter is that
more complex Channel Algebras can lead to new tactics for
introducing more expressive type-theoretic semantics in
mainstream programming environments.  As such, most of the
rest of this section will explore additional Channel Kinds
and associated Channel Groups which extend, in addition
to merely codifying, mainstream languages' syntax and semantics.
}
