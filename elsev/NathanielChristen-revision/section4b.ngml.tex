\vspace{-.1em}
\subsubsection{Dependent Types and Typestate}
\p{Typestates are finer-grained classifications than types.  
A canonical example of typestate is restricting how functions are
called which operate on files.  A single \q{file} type actually covers several
cases, including files that are open or closed, and even files that
are nonexistent \mdash{} they may be described by a path on a filesystem
which does not actually point to a file (perhaps in preparation for
creating such a file).  Instead of \i{one} type covering
each of these cases, we can envision \i{different} types for nonexistent,
closed, or open files.  With these more detailed types, constraints
like \q{don't try to create an already-existing file}
or \q{don't try to modify a closed or nonexistent file} are enforced by
type-checking.
}
\p{While this kind of gatekeeping is valuable in theory, it raises
questions in practice.  Reifying \q{cases} \mdash{} i.e., \i{typestates}
like open, closed, or nonexistent \mdash{} to distinct \i{types}
implies that a \q{file} value can go through different
types between construction and destruction.  If this is literally true, it
violates the convention that types are an intrinsic and fixed aspect of
typed values.  It is true that, as part of a type cast, values can be
reinterpreted (like treating an \int{} as a \float{}), but this
typically assumes a mathematical
overlap where one type can be considered as subsumed by a different type
for some calculation, \i{without this changing anything}:
any integer is equally a ratio with unit denominator, say.
\q{Casting} a closed file to an open one is the opposite effect,
using disjunctures between types to capture the fact that state
\i{has} changed; to capture a trajectory of states for one
value \mdash{} which must then have different types at
different times, since this is the whole point of modeling successive
states via alternations in type-attribution.
}
\p{An alternative interpretation is that the \q{trajectory} is not a
single mutated value but a chain of interrelated
values, wherein each successive value is obtained via a state-change
from its predecessor.  But a weakness of this chain-of-values
model is that it assumes only one value in the chain is currently
correct: a file can't be both open and closed, so if one value
with type \q{closed file} is succeeded by a different value with
type  \q{opened file}, the latter value will be correct only
if the file was in fact opened, and the former otherwise \mdash{} but
a compiler can't know which is which, \i{a priori}.  Or,
instead of a \q{chain} of differently-typed values we can employ a
single general \q{file} type and then \q{cast} the value to
an \q{open file} type when a function needs specifically
an \i{open} file, and so forth.   The effect in that case is to
insert the cast operator as a \q{gatekeeper} function preventing
the function receiving the casted value from getting nonconformant
input.  Again, though, the compiler cannot make any assumptions about
whether the \q{casts} will work (e.g., whether the attempt to open
a file will succeed).
}
\p{In short, typestate forces us to modify some basic assumptions about 
the relationship between types and values: either values can 
change types mid-stream, or a lexical scope can subsume a sequence 
of value \q{holders} which share the same symbol-name (and maybe the same type) 
but differ in state (some holding values unrelated to actual program 
state).  Both options upend normal programming expectations. 
This situation can be juxtaposed with the \q{metaconstructor problem}, 
i.e., how Dependent Types force a rethink on basic value-constructor 
theory.
}
\p{A good real-world example of the overlap between Dependent Types and
typestate (also grounded on file input/output) comes
from the \q{Dependent Effects} tutorial from the Idris
(programming language) documentation \cite{IdrisEffects}:
\begin{dquote}A practical use for dependent effects is in specifying resource usage
protocols and verifying that they are executed correctly.  For example,
file management follows a resource usage protocol with ... requirements
[that] can be expressed formally in [Idris] by creating a
\idrisText{FILE\_IO} effect parameterised over a file handle state,
which is either empty, open for reading, or open for writing.
In particular, consider the type of [a function
to open files]: This returns a \idrisText{Bool} which indicates
whether opening the file was successful.  The resulting state
depends on whether the
operation was successful; if so, we have a file handle open for the
stated purpose, and if not, we have no file handle.  By case analysis
on the result, we continue the protocol accordingly. ...
If we fail to follow the protocol correctly (perhaps by forgetting to
close the file, failing to check that open succeeded, or opening the
file for writing [when given a read-only file handle]) then we will
get a compile-time error.
\end{dquote}
So how does Idris mitigate the type-vs.-typestate conundrum?  Apparently
the key notion is that there is one single \tyFile{} \i{type}, but
a more fine-grained type-\i{state}; and, moreover, an
\i{effect system \q{parametrized over} these typestates}.
In other words, the \i{effect} of \tyFile{} operations is to
modify \i{typestates} (not types) of a \tyFile{} value.
Moreover, Dependent Typing ensures that functions cannot be called
sequentially in ways which \q{violate the protocol}, because
functions are prohibited from having effects that are incompatible
with the potentially affected values' current states.
This elegant syntheses of Dependent Types, typestate, and
Effectual Typing brings together three of the key
features of \q{fine-grained} or \q{very expressive} type systems.
}
\p{But the synthesis achieved by Idris relies on Dependent Typing:
typestate can be enforced because Idris functions can support
restrictions which \i{depend} on values' current typestate to
satisfy effect-requirements in a type-checking way.  In effect,
Idris requires that all possible variations in values' unfolding
typestate are handled by calling code, because otherwise the
handlers will not type-check.  An analogous tactic in 
\Cpp{} would be to provide an \q{open file} function only with a
signature that takes two callbacks, one for when the \openFn{}
succeeds and a second for when it fails (to mimic the Idris tutorial's
\q{case analysis}).  But that \Cpp{} version still requires convention
to enforce that the two callbacks behave differently: via Dependent Types
Idris can confirm that the \q{open file} callback, for example, is only
actually supplied as a callback for files that have indeed been
opened.  A better \Cpp{} approximation to this design would be
to cast files to separate types \mdash{} not only
typestates \mdash{} after all, but only when passing these values to the callback
functions (or, as I will discuss later, using a \q{passkey} 
to vouch that a callback's file argument \i{can} be thus cast). 
}
\p{In the case of Idris, Dependent Types are feasible because the final
\q{reduction} of expressions to evaluable representations occurs at
runtime.  In the language of the Idris tutorial:
\begin{dquote}In Idris, types are first class, meaning that they can be computed and
manipulated (and passed to functions) just like any other language construct.
For example, we could write a function which computes a type [and]
use this function to calculate a type anywhere that a type can be used.
For example, it can be used to calculate a return type [or]
to have varying input types.
\end{dquote}
More technically, Edwin Brady (and, here, Mat\'u\v{s} Teji\v{s}\v{c}\'ak)
elaborate that
\begin{dquote}\i{Full-spectrum} dependent types ... treat types as first-class language
constructs.  This means that types can be built by computation just like
any other value, leading to powerful techniques for generic programming.
Furthermore, it means that types can be parameterised on values,
meaning that strong, explicit, checkable relationships can be stated
between values and used to verify properties of programs at compile-time.
This expressive power brings new challenges, however, when
compiling programs. ... The challenge, in short, is to identify a
phase distinction between compile-time and run-time objects.
Traditionally, this is simple: types are compile-time only, values are
run-time, and erasure consists simply of erasing types. In a dependently
typed language, however, erasing types alone is not
enough \cite[page 1]{BradyMatus}.
\end{dquote}
To summarize, Idris works by \q{erasing} some, but not all, of the
extra contextual detail needed to ensure that dependent-typed
functions are used (i.e., called) correctly (see also \cite{Christiansen},
and \cite[page 195]{RichardEisenberg}).  This means that a lot of 
contextual detail is \i{not} erased; Idris provides machinery to
join executable code and user specifications onto
\i{types} so that they take effect whenever affected types' values are
constructed or passed to functions.
}
\p{Despite a divergent technical 
background, the net result is arguably not vastly
different from an Aspect-Oriented approach wherein constructors
and function calls are \q{pointcuts} setting anchors upon 
source locations or logical run-points, where extra code can be
added to program flow (see e.g. \cite{DineshGopalani},
\cite{KatharinaMehner}, \cite{CharlesZhang}).  Recall my contrast of
\q{internalist} and \q{externalist} paradigms, sketched at the
top of this chapter: Aspect-Oriented Programming involves
extra code added by external tools (that \q{modify} code by
\q{weaving} extra code providing extra features or gatekeeping).
Implementations like Idris pursue what often are in effect similar
ends from a more \q{internalist} angle, using the type system
to host added code and specifications without resorting to some
external tool that introduces this code in a manner orthogonal
to the language proper.  But Idris relies on Haskell 
to provide its operational environment; it is not clear
how Idris's strategies (or those of other Haskell and \ML{}-style
Dependent Type languages) for attaching runtime expressions to
type constructs would work in an imperative or
Object-Oriented environment, like \Cpp{} as a host language.
}
\subsubsection{Simulating Dependent Types with Preconstructors}
\p{Because Dependent Types and typestate recognize fine-grained 
requirements on which values may be passed to which functions, 
it might seem as if they are a logical continuation of the 
telos toward granular data modeling.  If our goal is to 
provide the most expressive data models possible within 
the bounds of computational tractability (I will return to 
this in the conclusion), we should certainly allow for 
Dependent Types and any other constructions which 
logically imply them (essentially, any formulation 
wherein types or their constructors are ad-hoc temporary 
values).    
}
\p{However, Dependent Types have the technical consequence of 
leaving pre-runtime values (or whatever construct we 
recognize as \q{holding} or \q{carrying} values) 
either \i{without} types, or with \i{different} types 
than they have at runtime.
In this case it becomes difficult to query
code \i{outside} runtime, which arguably \i{subtracts} 
expressiveness from the framework.  In short, we 
are free to explore some foundation for emulating 
Dependent Types \i{without} giving up on 
static reflection; the resulting system would 
not necessarily be expressively lesser than 
a \TyS{} with full-fledged Dependent Type support. 
}
\p{An elegant compromise might seem to be allowing each value to
have two potential type attributions \mdash{} one \q{static}
and one at runtime, potentially changing with each
call to the surrouding function-body.  After all, it is often
consequential that a value of type \tOne{} may be
coerced to, or reinterpreted as an instance of, \tTwo{}.  This
means that a \tOne{}'s specific value is consistent with also
being a \tTwo{}: it falls into the \q{space} where \tOne{} and
\tTwo{} overlap; or there is an available conversion that
creates a \tTwo{} from the \tOne{}.  However, this convertibility
is usually a factor when (using this chapter's terms)
the \tOne{} is transferred to a channel in a place for a
\tTwo{}.  Bear in mind that types are not \q{sets};
it's not as if we can regard two types as indistinguishable
within the collection of values where they can be interconverted.
In my treatment, types are manifest first and foremost as
recipes for values to be \q{handed off} between channels.
}
\p{So, a \q{second} or \q{dynamic} type for some value would only
be operational if it corresponds to the \q{static} type in
some receiving channel (this is how subtyping works).  But
then we are no closer to dealing with \q{temporary} types,
because the \q{metaconstructor} problem simply reappears
within the new channel.  This is not to rule out
the \i{receiving} context having its own duality
of static and dynamic types, where the hand-off
has to match requirements on both static-to-static
and dynamic-to-dynamic type-pairs.  In that case,
however, the dynamic types are not really
\q{second} types to which the initial values are
cast; they are more like logical preconditions which
must be satisfied at both ends of a channel-transfer.
}
\p{Indeed, type-attributions do two different things:
first they establish that some value is suitable
for transfer between procedures, but, second, they
affirm certain predicates \visavis{} that value.
With dynamically-recognized, temporarily \q{constructed}
types we are actually dealing with the second
salience: proof that values \i{can} be attributed
to some (maybe temporary and context-specific) type
establishes facts about that value.  But in this
case we are not interested in using that
second type as the infrastructure for a carrier-transfer;
we are instead trying to employ the type-attribution
\i{logically} as a transfer precondition.  Perhaps
a credible analogy is the post office only accepting boxes
within a certain size and weight (manifesting logistical
constraints in how the boxes can be handled) vs.
only accepting boxes which their sender certifies
not to contain dangerous contents (establishing contractual
rules that transcend the raw logistics of transporting packages).
Channel packages-to-complexes is an analogous \q{binary}
transporting which can be factored into an underlying
digital logistics and a more nuanced accounting of
package/complex compatibility, wherein we desire to
reject certain package/complex pairs which ordinary
type systems would allow (e.g. an index parameter on
the package side incompatible with the size of
a list parameter).  The problem is how to
achieve semantics modeling Dependent Types within
a framework that situates type theory itself
in a channel-transfer (and graph-oriented)
context: types only \q{exist} insofar as they
regulate inter-channel handoffs.  A given type
therefore only exists if there are capabilities
in the system to test package/complex
matches against the proposed type's logical posits.
}
\p{The solution I suggest to effectuate this compromise
involves using preconstructors as witnesses that a given 
value construction \i{could} be performed 
\mdash{} so that a given value (or values) is/are \i{logically consistent} 
with being construed as instance of some (perhaps ad-hoc) type, 
but are not \i{literally} assigned to that type.  The preconstructor 
can then be passed in to functions as an extra parameter, 
which when valid (e.g., when not a null function pointer) vouches 
for the co-construction it references being permissible.
That is, the preconstructor become a \q{passkey} parameter 
returned from a gatekeeper procedure and then passed 
on as evidence of the gatekeeping validation.  
}
\p{As a concrete example, suppose a procedure requires two numbers 
where the second is greater than the first (the inverse of 
the systolic-over-diastolic mandate): 
\fxy{} where \xlty{}.  How can we express the \xlty{} condition
within \fFun{}'s signature, assuming the signature can only
express semantics pertinent to \fFun{}'s type attribution?
On the face of it, we know that the desired \q{increasing}
condition is equivalent to \yVal{} having a type like
\RangeGTValx{} \mdash{} a range bounded (only) from below \mdash{} 
where this \q{\xVal{}} is the \xVal{} preceding
\yVal{} in \fFun{}'s signature.  But using such 
directly as \yVal{}'s type-attribution means that
from the perspective of \fFun{}'s own type-attribution,
\yVal{} does not have a single, fixed type; its type
varies according to the value of \xVal{}.  Here again
we encounter a \q{metaconstructor} problem: in order
for the \xlty{} condition to be modeled \i{directly} by
\yVal{}'s type-attribution we would need the
constructor for \yVal{}'s value-constructor to
be some operation that produces a temporary function-value
\mdash{} not simply the compilation of a code-graph to an 
addressable, non-temporary implementation.
}
\p{These issues go away if, instead of working with a function
taking \i{two} integers, we instead consider a function taking \i{one}
value which is a monotone-increasing pair (something like
\fmipair{}).  A type like
\MIpair{}, based on ordered pairs \xCommaY{} of \int{}s, solves the
metaconstructor problem for \yVal{} because \xVal{} and \yVal{}
are no longer distinct \fFun{} parameters with distinct value-constructors;
they are subsumed into one pair, whose own value-constructor
can check the \xlty{} condition.  The \i{requirements} for the
original (two-valued) \fFun{} may then be \i{described} as
\xVal{} and \yVal{} being convertible into a pair \prVal{}
which is an instance of \MIpair{} (so that \xlty{}).  This
\i{description} is not a \i{type}, but elevating the
description to type level can be at least approximated with
a wrapper like \fxypreqMIpairxy{}, so \fFun{} when used as
\fxy{} will silently call the \MIpair{} constructor.  This is only
approximate because it allows anomalies like \fxypreqMIpairZeroOne{},
 \mdash{} taking \MIpair{}s on anything \i{but} \xVal{} and \yVal{}
defeats the purpose \mdash{} but at least we can
approximate a Dependent Type signature with a passkey protocol 
that is not difficult to enforce via calling conventions
(client code should never call the three-argument form directly,
which could be sequestered to a file-scoped or private member function).
}
\p{Now, notice that we do not actually need the third argument; we 
just want to know that the \MIpair{} constructor \i{would} accept 
the \xCommaY{} pair.  So we can replace the \i{actual} \MIpair{} constructor 
with a \i{preconstructor} that \i{could} be used as a factory 
for \MIpair{} instances if needed, but can \i{also} serve to 
certify that a certain set of arguments (here a pair of numbers) 
meets the logical preconditions which an actual constructor 
would check off.
}
\p{For this to work, the 
\MIpair{} type would need to be implemented with a static 
procedure that returns a valid preconstructor for valid inputs
\mdash{} plus, assuming the preconstructor/co-constructor pattern I
am advocating, a co-constructor whose address would be 
the basis of the preconstructor value.  These are obviously 
not features of \Cpp{} (or any other language I know of) 
but could readily be an implementational norm for 
data types used in a safety-conscious project.
In effect, consistent use of preconstructors for 
fine-grained types is one strategy for siting 
gatekeeping code broadly throughout a code base.
}
\p{Further discussion of preconstructors is outside the scope 
of this paper, but concrete examples of range-checking via 
preconstructor passkeys can be found in the demo.  
Here I'll make the further point that \mdash{} if we accept a 
Channel Algebra which expands beyond present programming 
languages \mdash{} we can move preconstructor passkeys to a 
separate channel, thereby approximating Dependent Types 
more eloquently.  Co-constructors may be identified via
a dedicated co-constructing channel \mdash{} \coconch{} 
\mdash{} which signals that a return value is not 
\i{any} procedure returning the associated type, but 
a constructing procedure which is part of the 
type's interface and helps to demarcate its space 
of values.  A \coconch{} channel paired with a 
special \preconch{} channel, for preconstructor passkeys, 
provides a metamodel wherein Dependent Types, typestate, 
and many effect-systems can be reasonably encoded. 
}
\p{This last case also points to how a theory of channels 
adds semantic expressiveness to code models: 
we can achieve via descriptions of inter-procedure 
information flows \mdash{} including distinguishing distinct 
roles such as passkeys vs. ordinary parameters, 
and constructing returns vs. ordinary procedures 
happening to return a given type \mdash{} a semantic exactitude 
that is implementationally harder (from a language engineering
perspective) to achieve directly within a type system.  
Channel Algebras are not limited to channels 
actually recognized by existing languages \mdash{} they could 
be the basis for new languages, and/or new analytic tools 
isolating patterns in existing code.  With this flexibly 
channels can be lifted into a construct recognized within 
data and code modeling paradigms \mdash{} as well as an 
added structural layer within hypergraphs \mdash{} in general.  
These possibilities may become clearer as I present a 
theory of channels in more detail next section.    
}
