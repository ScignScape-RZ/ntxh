\section{Directed Hypergraphs and Generalized Lambda Calculus}
\phantomsection\label{sThree}
\p{Thus far in this chapter, I have written in general terms about 
architectural features related to Cyber-Physical software;
especially, verifying coding assumptions concerning 
individual data types and/or procedures.  My comments 
were intended to summarize the relevant territory, 
so that I can add some theoretical details or suggestions 
from this point forward.  In particular, I will explore 
how to model software components at different scales 
so as to facilitate robust, safety-conscious coding practices.   
}
\p{Note that almost all non-trivial software is in some 
sense \q{procedural}: the total package of 
functionality provided by each software component 
is distributed among many individual, interconnected 
procedures.  Each procedure, in general, 
implements its functionality by calling \i{other} procedures 
in some strategic order.  Of course, often 
inter-procedure calls are \i{conditional} \mdash{} a calling 
procedure will call one (or some sequence of) procedures 
when some condition holds, but call alternate procedures
when some other conditions hold.  In any case, computer 
code can be analyzed as a graph, where connections exist 
between procedures insofar as one procedure calls, 
or sometimes calls, the other.
}
\p{This general picture is only of only limited applicability 
to actual applications, however, because the basic concept 
of \q{procedure} varies somewhat between different 
programming languages.  As a result, it takes some effort 
to develop a comprehensive model of computer code which 
accommodates a representative spectrum of coding styles 
and paradigms. 
}
\p{There are perhaps three different perspectives for 
such a comprehensive theory.  One perspective is 
to consider source code as a data structure in its 
own right, employing a Source Code Algebra or 
Source Code Ontology to assert properties of source code 
and enable queries against source code, qua information 
space.  A second option derives from type theory: to 
consider procedures as instances of functional types,
specified by tuples of input and output types.  
A procedure is then a transform which, in the presence 
of (zero or more) inputs having the proper types, produces 
(one or more) outputs with their respective types.   
(In practice, some procedures do not return values, but they 
\i{do} have some kind of side-effect, which can be analyzed as a 
variety of \q{output}.)  Finally, third, procedures can be 
studied via mathematical frameworks such as Lambda Calculus, 
allowing notions of functions on typed parameters, and 
of functional application \mdash{} applying functions to concrete 
values, which is analogous to calling procedures with 
concrete input arguments \mdash{} to be made formally rigorous.
}
\p{I will briefly consider all three of these perspectives \mdash{} Source Code 
Ontology, type-theoretic models, and Lambda Calculus \mdash{} in 
this section.  I will also propose a new 
model, based on the idea of \q{channels}, which combines 
elements of all three. 
}
\vspace{-.1em}
\subsection{Generalized Lambda Calculus}
\p{Lambda (or \mOldLambda{}-) Calculus emerged in the early 
20th Century as a formal model of mathematical 
functions and function-application.  There are 
many mathematical constructions which can be 
subsumed under the notion of \q{function-application}, 
but these have myriad notations and conventions 
(compare the visual differences between mathematical 
notations \mdash{} integrals, square roots, super- and 
sub-scripted indices, and so forth \mdash{} to the much 
simpler alphabets of mainstream programming languages).  
But the early 20th century was a time of 
great interest in \q{mathematical foundations}, seeking 
to provide philosophical underpinnings for mathematical 
reasoning in general, unifying disparate mathematical 
methods and subdisciplines.  One consequence of 
this foundational program was an attempt to capture 
the formal essence of the concept of \q{function} and 
of functions being applied to concrete values.       
}
\p{A related foundational concern is how mathematical 
formulae can be nested, yielding new formulae.  
For example, the volume of a sphere 
(expressed in terms of its radius \rRad{}) is \VolSphere{}.  
The symbol \rRad{} is just a mnemonic which could be replaced with a different symbol, 
without the formula being different.  But it can also be replaced by a more complex 
expression, to yield a new formula.  In this case, 
substituting the formula for a cube's half-diagonal \mdash{} \crVOverRTwo{}
where \vVol{} is its volume \mdash{} for \rRad{}, in the first formula, 
yields \volSphCube{}: a formula for the sphere's volume in terms 
of the volume of the largest cube that can fit inside it     
(\cite{KennethAnderson} has similar interesting examples in the 
context of code optimization).  This kind of tinkering with equations is 
of course a bread-and-butter of mathematical discovery.  In terms 
of foundations research, though, observe that the derivation depended on 
two givens: that the \rRad{} symbol is \q{free} in the first formula 
\mdash{} it is a place-holder rather than the designation of a concrete 
value, like \piSym{} \mdash{} and that free symbols (like \rRad{}) can be 
bound to other formulae, yielding new equations. 
}
\p{From cases like these \mdash{} relatively simple geometric 
expressions \mdash{} mathematicians began to ask foundational 
questions about mathematical formulae: what are all formulae 
that can be built up from a set of core equations via 
repeatedly substituting nested expressions for
free symbols?  This question turns out to be related to 
the issue of finite calculations: in lieu of 
building complex formulae out of simpler parts, we can 
proceed in the opposite direction, replacing nested 
expressions with values.  Formulae are constructed 
in terms of unknown values; when we have concrete measurements 
to plug in to those formulae, the set of unknowns decreases.
If \i{all} values are known, then a well-constructed formula 
will converge to a (possibly empty) set of outcomes.  
This is roughly analogous to a computation which 
terminates in real time.  On the other hand, 
a \i{recursive} formula \mdash{} an expression nested inside 
itself, such as a continued fraction \mdash{} is analogous to 
a computation which loops indefinitely.\nobrfootnote{Although there are sometimes techniques for converting formulae 
like Continued Fractions into \q{closed form} equations 
which do \q{terminate}.
% It may be desirable to write this as "nobrfootnote" ...
}
}
\p{In the early days of computer programming, it was natural to 
turn to \mOldLambda{}-Calculus as a formal model of 
computer procedures, which are in some ways analogous to 
mathematical formulae.  As a mathematical subject, 
\mOldLambda{}-Calculus predates digital computers as 
we know them.  While there were no digital computers at the time,
there \i{was} a growing interest in mechanical computing 
devices, which led to the evolution of 
cryptographic machines used during the Second World War.  
So there was indeed a practical interest in 
\q{computing machines}, which eventually led to 
John von Neumann's formal prototypes for digital 
computers.  
}
\p{Early on, though, \mOldLambda{}-Calculus was less about blueprints
for calculating machines and more about \i{abstract} formulation of
calculational processes.  Historically, the original 
purpose of \mOldLambda{}-Calculus was largely a mathematical \i{simulation} of
computations, which is not the same as a mathematical \i{prototype}
for computing machines.  Mathematicians in the decades before WWII
investigated logical properties of computations, with particular
emphasis on what sort of problems could always be solved in
finite time, or what kinds of procedures can be guaranteed to
terminate \mdash{} a \q{Computable Number}, for example, is a number
which can be approximated to any degree of precision by a terminating
function.  Similarly, a Computable Function is a function from
input values to output values that can be associated with an
always-terminating procedure which necessarily calculates the desired
outputs from a set of inputs.  The space of Computable Functions
and Computable Numbers are mathematical objects whose properties
can be studied through mathematical techniques \mdash{} for instance,
Computable Numbers are known to be a
countable field within the real numbers.
These mathematical properties are proven using a formal
description of \q{any computer whatsoever}, which has no
concern for the size and physical design of the \q{computers}
or the time required for its \q{programs}, so long
as they are finite.  Computational procedures in
this context are not actual implementations but rather
mathematical distillations that can stand in for
calculations for the purpose of mathematical analysis
(interesting and representative contemporary articles
continuing these perspectives include, e.g., \cite{MartinEscardo},
\cite{MasahitoHasegawa}, \cite{TuckerZucker}).
}
\p{It was only after the emergence of modern digital computers 
that \mOldLambda{}-Calculus become reinterpreted as a 
model of \i{concrete} computing machines.  In its 
guise as a Computer Science (and not just Mathematical 
Foundations) discipline, \mOldLambda{}-Calculus has been 
most influential not in its original form but in 
a plethora of more complex models which track the 
evolution of programming languages.  Many programming 
languages have important differences which are not 
describable on a purely mathematical basis: two 
languages which are both \q{Turing complete} are 
abstractly interchangeable, but it is important to
represent the contrast between, say,
Object-Oriented and Functional programming.  
In lieu of a straightforward, mathematical model 
of formulae as procedures which map inputs to 
outputs, modern programming languages add 
may new constructs which determine different
mechanisms whereby procedures can read and 
modify values: objects, exceptions, closures, 
mutable references, side-effects, signal/slot 
connections, and so forth.  Accordingly, 
new programming constructions have inspired 
new variants of \mOldLambda{}-Calculus, analyzing
different features of modern programming languages \mdash{} Object
Orientation, Exceptions, call-by-name, call-by-reference,
side effects, polymorphic type systems, lazy evaluation 
\mdash{} in the hopes of deriving formal proofs of 
program behavior insofar as computer code uses the 
relevant constructions.  In short, a reasonable
history can say that \mOldLambda{}-Calculus mutated from being an
abstract model for studying Computability as a mathematical concept,
to being a paradigm for prototype-specifications of concretely
realized computing environments.
}
\p{Modern programming languages have many different ways of handing-off
values between procedures.  The \q{inputs} to a function can be \q{message receivers}
as in Object-Oriented programming, or lexically scoped
values \q{captured} in an anonymous function that inherits
values from the lexical scope (loosely, the area of source code)
where its body is composed.  Procedures can also \q{receive} data indirectly
from pipes, streams, sockets, network connections, database connections, or files.
All of these are potential \q{input channels} whereby a function implementation
may access a value that it needs.  In addition, procedures can \q{return} values
not just by providing a final result but by throwing exceptions, writing
to files or pipes, and so forth.  To represent these myriad
\q{channels of communication} computer scientists have invented a menagerie
of extensions to \mOldLambda{}-Calculus \mdash{} a noteworthy 
example is the \q{Sigma} calculus to model Object-Oriented
Programming; but parallel extensions represent call-by-need evaluation, 
exceptions, by-value and by-reference capture, etc.
}
\p{Rather than study each system in isolation, in this 
chapter I propose an integrated strategy for 
unifying disparate \mOldLambda{}-Calculus extensions 
into an overarching framework.  The 
\q{channel-based} tactic I endorse here may not be 
optimal for a \i{mathematical} calculus which has 
formal axioms and provable theorems, but I believe 
it can be useful for the more practical goal 
of modeling computer code and software 
components, to establish recommended design patterns 
and to document coding assumptions.  
}
\p{In this perspective, different extensions
or variations to \mOldLambda{}-Calculus model different 
\i{channels}, or data-sources through which procedures 
receive and/or modify values.  Different channels 
have their own protocols and semantics
for passing values to functions.
We can generically discuss \q{input} and
\q{output} channels, but programming languages have different specifications
for different genres of input/output, which we can model via
different channels.  For a particular channel, we can recognize
language-specific limitations on how values passed in to or
received from those channels are used, and how the symbols carrying those
values interact with other symbols both in function call-sites and in the 
body of procedure implementations.  
For example, procedures can output values by throwing exceptions, but
exceptions are unusual values which have to be handled
in specific ways \mdash{} languages employ exceptions to signal possible
programming errors, and they are engineered to interrupt
normal program flow until or unless exceptions are \q{caught}.
}
\p{Computer scientists have explored these more complex programming paradigms
in part by inventing new variations on \mOldLambda{}-calculi.  
Here I will develop one theory representing
code in terms of Directed Hypergraphs, which are subject to multiple kinds
of lambda abstraction \mdash{} in principle, unifying multiple 
\mOldLambda{}-Calculus extensions.
The following subsection will lay out the details of this form of Directed Hypergraph
and how \mOldLambda{}-calculi
can be defined on its foundation, while the last subsection 
summarizes an expanded type theory which follows organically
from this approach.
}
\p{Many concepts outlined here are reflected in the accompanying code set
(which includes a \Cpp{} Directed Hypergraph library).
My strategy for unifying multiple \mOldLambda{}-calculi depends 
in turn on hypergraph code representations, which is a theme 
in the umbrella of graph-based data modeling, 
to which I now turn. 
}
\vspace{-.1em}
\subsectiontwolinerepl{Directed Hypergraphs and \q{Channel Abstractions}}%
{Directed Hypergraphs and 'Channel Abstractions'}
\p{A \i{hypergraph} is a graph whose edges (a.k.a. \q{hyperedges}) can span
more than two nodes (\cite[e.g. volume 2, page 24]{BenGoetzel},
\cite{HaishanLiu}, \cite{MarkMinas} and \cite{MinasSchneider}, 
\cite{BalintMolnar},
\cite{AlexandraPoulovassilis}, \cite{JohnStell},
\cite{JohnStellFCA}).
A \i{directed} hypergraph (\q{\DH{}}) is a hypergraph
where each edge has a \i{head set} and
\i{tail set} (both possibly empty).  Both of these are sets of nodes
which (when non-empty) are called \i{hypernodes}.  A hypernode can
also be thought of as a hyperedge whose tail-set
(or head-set) is empty.  Note that a typical hyperedge
connects two hypernodes (its head- and tail-sets), so if
we consider just hypernodes, a hypergraph potentially
reduces to a directed ordinary graph.\footnote{Here when distinguishing \q{head} and \q{tail} I will 
invert the orientation which most mathematical treatments 
of hypergraphs use: that is, I define hyperedges such 
that the edge \i{starts at} the head and \i{ends at} the 
tail.  My rationale is that hyperedges induce an 
orientation not only on the head/tail pair, but 
\i{within} the head and tail, which become ordered 
tuples rather than sets.  Hyperedges can therefore be 
seen as paths that \q{visit} a chain of hyponodes, first 
those in the head, then those in the tail.  My terminology 
is consistent with software libraries wherein the 
\i{beginning} of an ordered list is called its \q{head}. 
}  While
\q{edge} and \q{hyperedge} are formally equivalent,
I will use the former term when attending more to the
edge's representational role as linking two hypernodes,
and use the latter term when focusing more on its tuple
of spanned nodes irrespective of their partition into
\i{head} and \i{tail}.
}
\p{I assume that hyperedges always span an \i{ordered} node-tuple
which induces an ordering in the head- and tail-sets: so a
hypernode is an \i{ordered list} of nodes, not just a
\i{set} of nodes.  I will say that
two hypernodes \i{overlap} if they
share at least one node; they are \i{identical}
if they share exactly the same nodes in the same order; and
\i{disjoint} if they do not overlap at all.  I call a
Directed Hypergraph
\q{reducible} if all hypernodes are either disjoint or
identical.  The information in reducible \DH{}s can be factored
into two \q{scales}, one a directed graph whose nodes are the
original hypernodes, and then a table of all nodes
contained in each hypernode.  Reducible \DH{}s allow
ordinary graph traversal algorithms when hypernodes
are treated as ordinary nodes on the coarser scale
(so that their internal information \mdash{} their list
of contained nodes \mdash{} is ignored).\footnote{A weaker restriction on \DH{} nodes is that two
non-identical hypernodes \i{can} overlap, but
must preserve node-order: i.e., if the first
hypernode includes nodes \nodeNOne{}, and \nodeNTwo{}
immediately after, and the second hypernode
also includes \nodeNOne{}, then the second
hypernode must also include \nodeNTwo{} immediately
thereafter.  Overlapping hypernodes
can not \q{permute} nodes
\mdash{} cannot include them in different orders or in a way
that \q{skips} nodes.
Trivially, all reducible \DH{}s meet this condition.  
Any graphs discussed here are assumed to meet 
this condition.    
}
}
\p{To avoid confusion, I will hereafter use the word \q{hyponode} in place
of \q{node}, to emphasize the container/contained relation between
hypernodes and hyponodes.  I will use \q{node} as an informal word
for comments applicable to both hyper- and hypo-nodes.  Some
Hypergraph theories and/or implementations
allow hypernodes to be nested: i.e., a hypernode can contain
another hypernode.  In these theories, in the general case any node
is potentially both a hypernode and a hyponode.  For this chapter,
I assume the converse: any \q{node} (as I am hereafter using the term) is
\i{either} hypo- or hyper-.  However, multi-scale Hypergraphs can be approximated
by using hyponodes whose values are proxies to hypernodes.
}
\p{Here I will focus on a class of \DH{}s which (for reasons to emerge)
I will call \q{Channelizable}. Channelizable Hypergraphs
(\CH{}s) have these properties:
\begin{enumerate}\item{}  They have a Type System \TyS{} and all hyponodes and hypernodes are assigned
exactly one canonical type (they may also be considered instances of super- or subtypes
of that type).
\item{}  All hyponodes can have (or \q{express}) at most one value, an instance of its
canonical type, which I will call a \i{hypovertex}.  Hypernodes, similarly,
can have at most one \i{hypervertex}.  Like \q{node} being an informal
designation for hypo- and hyper-nodes, \q{vertex} will be a general term
for both hypo- and hyper-vertices.  Nodes which do have a vertex
are called \i{initialized}.  The hypovertices \q{of} a hypernode are those
of its hyponodes.
\item{}  Two hyponodes are \q{equatable} if they express the same value of the same
type.  Two (possibly non-identical) hypernodes are \q{equatable} if all of their
hyponodes, compared one-by-one in order, are equatable.  I will also say that values
are \q{equatable} (rather than just saying \q{equal}) to emphasize that
they are the respective values of equatable nodes.
\item{}  There may be a stronger relation, defined on equatable non-equivalent hypernodes,
whereby two hypernodes are \i{inferentially equivalent} if any inference justified via
edges incident to the first hypernode can be freely combined with inferences
justified via edges incident to the second hypernode.  Equatable nodes are not
necessarily inferentially equivalent.
\item{}  Hypernodes can be assumed to be unique in each graph, but it is
unwarranted to assume (without type-level semantics) that two equatable
hypernodes in different graphs are or are not inferentially equivalent.
Conversely, even if graphs are uniquely labeled \mdash{} which would
appear to enable a formal distinction between hypernodes in one
graph from those in another, \CH{}
semantics does not permit the assumption that this separation alone
justifies inferences presupposing that their hypernodes
\i{are not} inferentially equivalent.
\item{}  All hypo- and hypernodes have a \q{proxy}, meaning there is a type in
\TyS{} including, for each node, a unique identifier designating
that node, that can be expressed in other hyponodes.
\item{}  There are some types (including these proxies) which may only be expressed
in hyponodes.  There may be other types which may only be expressed
in hypernodes.  Types can then be classified as \q{hypotypes} and \q{hypertypes}.
The \TyS{} may stipulate that all types are \i{either} hypo or hyper.  In
this case, it is reasonable to assume that each hypotype maps to a unique
hypertype, similar to \q{boxing} in a language which recognizes \q{primitive}
types (in Object-Oriented languages, boxing allows non-class-type
values to be used as if they were objects).
\item{}  Types may be subject to the restriction that any hypernode which has that
type can only be a tail-set, not a head-set; call these \i{tail-only} types.
\item{}  Hyponodes may not appear in the graph outside of hypernodes.  However, a
hypernode is permitted to contain only one hyponode.
\item{}  Each edge, separate and apart from the \CH{}'s actual graph structure,
is associated with a distinct hypernode, called its \i{annotation}.  This
annotation cannot (except via a proxy) be associated with any other hypernode
(it cannot be a head- or tail-set in any hypernode).
The first hyponode in its annotation I will 
dub a hyperedge's \i{classifier}.  The outgoing edge-set of a hypernode can
always be represented as an associative array indexed by the classifier's vertex.
\item{}  A hypernode's type may be subject to restrictions such that there is a
single number of hyponodes shared by all instances.  However, other types may be
expressed in hypernodes whose size may vary.  In this case the
hyponode types cannot be random; there must be some pattern linking
the distribution of hyponode types evident in hypernodes (with the same
hypernode types) of different sizes.  For example, the hypernodes
may be dividable into a fixed-size, possibly empty sequence of hyponodes,
followed by a chain of hyponode-sequences repeating the same type pattern.
The simplest manifestation of this structure is a hypernode all of whose
hyponodes are the same type.
\item{}  Call a \i{product-type transform} of a hypernode to be a different
hypernode whose hypovertices are tuples of values equatable to those from the first hypernode,
typed in terms of product types (i.e., tuples).  For example, consider two
different representations of semi-transparent colors: as a 4-vector
\vecrgbt{}, or as an \vecrgb{} three-vector paired with a transparency magnitude.
The second representation is a product-type transform of the first, because the first
three values are grouped into a three-valued tuple.  We can
assert the requirement in most contexts that \CH{}s whose hypernodes are
product-type transforms of each other contain \q{the same information}
and as sources of information are interchangeable.
\item{}  The Type System \TyS{} is \i{channelized}, i.e., closed under a
Channel Algebra, as will be discussed below.
\end{enumerate}
}
\p{These definitions allude to two strategies for computationally representing
\CH{}s.  One, already mentioned, is to reduce them to directed graphs
by treating hypernodes as integral units (ignoring their internal structure).
A second is to model hypernodes as a \q{table of associations} whose
keys are the values of the classifier hyponodes on each of their edges.
A \CH{} can also be transformed into an \i{undirected} hypergraph by
collapsing head- and tail- sets into an overarching tuple.  All of these
transformations may be useful in some analytic/representational contexts,
and \CH{}s are flexible in part by morphing naturally into these various
forms.\phantomsection\label{unplug}
}
\spinctc{unplug}{Unplugging a Node.}{fig:unplug}
\p{Notice that information present \i{within} a hypernode can also be expressed as
relations \i{between} hypernodes.  For example, consider the information that
I (Nathaniel), age \FourtySix{}, live in Brooklyn as a registered Democrat.  This may be
represented as a hypernode with hyponodes \NathFF{}, connected to a hypernode
with hyponodes \BrookDem{}, via a hyperedge whose classifier encodes the
concept \q{lives in} or \q{is a resident of}.  However, it may also be
encoded by \q{unplugging} the \q{age} attribute so the first hypernode becomes
just \Nath{} and it acquires a new edge, whose tail has a single
hyponode \ageFF{} and a classifier (encoding the concept) \q{age}
(see the comparison in Diagram \hyperref[fig:unplug]{\ref{fig:unplug}}).
This construction can work in reverse:
information present in a hyperedge can be refactored so that it \q{plugs in}
to a single hypernode.
}
\p{These alternatives are not redundant.  Generally, representing information
via hyperedges connecting two hypernodes implies that this information is
somehow conceptually apart from the hypernodes themselves, whereas representing
information via hyponodes \i{inside} hypernodes implies that this information
is central and recurring (enforced by types), and that the data
thereby aggregated forms a recurring logical unit.  In a political survey,
people's names may \i{always} be joined to their age, and
likewise their district of
residence \i{always} joined to their political affiliation.  The left-hand side
representation of the info (seen as an undirected hyperedge) \NathFFBD{}
in Diagram \hyperref[fig:unplug]{\ref{fig:unplug}} 
captures this semantics better because it describes
the name/age and \mbox{place/party} pairings as
\i{types} which require analogous
node-tuples when expressed by other hypernodes.  For example, any two
hypernodes with the same type as \NathFF{} will necessarily have
an \q{age} hypovertex
and so can predictably be compared along this one axis.  By contrast, the
right-hand (\q{unplugged}) version in Diagram \hyperref[fig:unplug]{\ref{fig:unplug}}
implies no guarantees that the \q{age}
data point is present as part of a recurring pattern.
}
\itclfig{initializing-hypernodes}{fig:initializinghypernodes}
\p{The two-tiered \DH{} structure is also a factor when integrating 
serialized or shared data structures with runtime data values.  
In the demo \DH{} library, for example, it is assumed that 
each node can be associated with a runtime, binary data allocation 
(practically speaking, a pointer to user data).  Hypernodes' internal 
structure can therefore be represented \i{either} via hyponodes explicit 
in the graph content \i{or} by internal structure in the user 
data (or some combination).  Graph deserialization can then be 
a matter of mapping hyponodes to fields in the \q{internal} data 
allocations, before then mapping inter-hypernode relations to 
the proper hypervertex-relations.  Code sample 
\ref{lst:initializing-hypernodes} demonstrates the pattern 
of hypervertex construction as \Cpp{} objects that get wrapped 
in new nodes ({\OneOverlay}-{\TwoOverlay}), 
along with obtaining nodes already registered in a runtime graph 
({\ThreeOverlay}) and then inserting the new nodes (with stated 
relationships) alongside prior ones into the runtime graph 
({\FourOverlay}).\footnote{The code samples in this text are drawn from a working demo 
at the time of writing; the actual code belonging to a downloadable 
data set at the time of publication may be slightly revised.  
The data set will include components to help readers cross-reference 
between the chapter's samples and working demo code.  
}       
}
\p{In general, graph representations like \CH{} and \RDF{} serve two goals: first,
they are used to \i{serialize} data structures (so that
they may be shared between
different locations; such as, via the internet);
and, second, they provide
formal, machine-readable descriptions of information content, allowing for
analyses and transformations, to infer new information or produce new data
structures.  The design and rationale of representational paradigms is
influenced differently by these two goals, as I will review now with an eye
in part on drawing comparisons between \CH{} and \RDF{}.
}
\vspace{-.1em}
\subsection{Channelized Hypergraphs and \largeRDF{}}
\phantomsection\label{RDF}
\p{The Resource Description Framework (\RDF{}) models information
via directed graphs (\cite{MadalinaCroitoru}, \cite{ErnestoDamiani},
\cite{AnglesGuttierez}, and \cite{RodriguezWatkins} are good discussions of
Semantic Web technologies from a graph-theoretic perspective),
whose edges are labeled with concepts that,
in well-structured contexts, are drawn from published Ontologies
(these labels play a similar role to \q{classifiers} in \CH{}s).
In principle, all data expressed via \RDF{} graphs is defined
by unordered sets of labeled edges, also called \q{triples}
(\q{\SPO{}}, where the \q{Predicate} is the label).  In practice,
however, higher-level \RDF{} notation such as \TTL{} (\Turtle{} or
\q{Terse \RDF{} Triple Language}) and Notation3 (\NThree{})
deal with aggregate groups of data, such as \RDF{} containers and
collections.\phantomsection\label{lived}
}
\spinctc{lived}{CH vs. RDF Collections.}{fig:lived}
\p{For example, imagine a representation of the
fact \q{(A/The person named) Nathaniel, \FourtySix{}, has lived in Brooklyn,
Buffalo, and Montreal} (shown in Diagram \hyperref[fig:lived]{\ref{fig:lived}} as both
a \CH{} and in \RDF{}).  If we consider \Turtle{} or \NThree{} as \i{languages} and
not just \i{notations}, it would appear as if their semantics is built
around hyperedges rather than triples.  It would seem that these
languages encode many-to-many or one-to-many assertions, graphed as
edges having more than one subject and/or predicate.  Indeed,
Tim Berners-Lee himself suggests that
\q{Implementations may treat list as a data type rather than just
a ladder of rdf:first and rdf:rest properties} \cite[page 6]{TimBernersLee}.
That is, the specification for
\RDF{} list-type data structures invites us to consider that
they \i{may} be regarded integral units rather than
just aggregates that get pulled apart in semantic interpretation.
}
\p{Technically, perhaps, this is an illusion.  Despite their higher-level
expressiveness, \RDF{} expression languages are, perhaps,
supposed to be deemed \q{syntactic sugar}
for a more primitive listing of triples: the \i{semantics} of
\Turtle{} and \NThree{} are conceived to be defined by translating
expressions down to the triple-sets that they logically imply
(see also \cite{YurickWilks}).
This intention accepts the paradigm that providing semantics
for a formal language is closely related to defining which
propositions are logically entailed by its statements.
}
\p{There is, however, a divergent tradition in formal semantics that is oriented to
type theory more than logic.  It is consistent with this alternative
approach to see a different semantics for a language like \Turtle{},
where larger-scale aggregates become \q{first class} values.
So, \NathFF{} can be seen as a (single, integral)
\i{value} whose \i{type} is a \nameAge{} pair.  Such a value has an
\q{internal structure} which subsumes multiple data-points.  The
\RDF{} version is organized, instead, around a \i{blank node} which
ties together disparate data points, such as my name and my age.
This blank node is also connected to another blank node which
ties together place and party.  The blank nodes
play an organizational role, since nodes are grouped together
insofar as they connect to the same blank node.  But the
implied organization is less strictly entailed; one might
assume that the \BrookDem{} nodes could just as readily
be attached individually to the \q{name/age} blank
(i.e., I live in Brooklyn, \i{and} I vote Democratic).
}
\p{Why, that is, are Brooklyn and Democratic grouped together?
What concept does this fusion model?
There is a presumptive rationale for the name/age blank
(i.e., the fusing name/age by joining them to a blank
node rather than allowing them to take edges independently):
conceivably there are multiple \FourtySix{}-year-olds named Nathaniel,
so \i{that} blank node plays a key semantic role
(analogous to the quantifier in \q{\i{There is} a Nathaniel,
age \FourtySix{}...}); it provides an unambiguous nexus so that
further predicates can be attached to \i{one specific}
\FourtySix{}-year-old Nathaniel rather than any old \NathFF{}.  But there is no
similarly suggested semantic role for the \q{place/party} grouping.  The name
cannot logically be teased apart from the name/age blank (because there
are multiple Nathaniels); but there seems to be no \i{logical}
significance to the \mbox{place/party} grouping.  Yet pairing
these values \i{can} be motivated by a modeling convention
\mdash{} reflecting that geographic and party affiliation data
are grouped together in a data set or data model.  The logical
semantics of \RDF{} make it harder to express these kinds of modeling
assumptions that are driven by convention more than logic
\mdash{} an abstracting from data's modeling environment that can be desirable
in some contexts but not in others.
}
\p{So, why does the Semantic Web community effectively insist on
a semantic interpretation of \Turtle{} and \NThree{} as \i{just} a
notational convenience for \NTrips{} rather than as higher-level
languages with a different higher-level semantics \mdash{} and despite
statements like the above Tim Berners-Lee quote insinuating that
an alternative interpretation has been contemplated even by
those at the heart of Semantic Web specifications?  
Moreover, defining hierarchies of material composition or 
structural organization \mdash{} and so by extension, 
potentially, distinct scales of modeling resolution \mdash{} 
has been identified as an intrinsic part of domain-specific 
Ontology design (see \cite{Aranda}, \cite{BittnerSmithDonnelly},
\cite{BittnerSmith}, \cite{MaureenDonnelly},
\cite{Fabrikant}, \cite{PetitotSmith}, \cite{SegevGal},
\cite{BarrySmithBlood},
or \cite{PietroRamellini}).
Semantic Web advocates have not however promoted multitier 
structure as a feature \i{of} Semantic models fundamentally, 
as opposed to criteriology \i{within} specific Ontologies.        
To the degree that this has an explanation, it probably has something to
do with reasoning engines: the tools that evaluate \SPARQL{} queries
operate on a triplestore basis.  So the \q{reductive} semantic
interpretation is arguably justified via the warrant that the
definitive criteria for Semantic Web representations are not their
conceptual elegance \visavis{} human judgments but their utility in
cross-Ontology and cross-context inferences.
}
\p{As a counter-argument,
however, note that many inference engines in Constraint Solving,
Computer Vision, and so forth, rely on specialized algorithms
and cannot be reduced to a canonical query format.  Libraries such
as \GeCODE{} and \ITK{} are important because problem-solving
in many domains demands fine-tuned application-level engineering.
We can think of these libraries as supporting \i{special} or
domain-specific reasoning engines, often built for specific
projects, whereas \OWL{}-based reasoners like \FactPP{} are
\i{general} engines that work on general-purpose \RDF{} data
without further qualification.  In order to
apply \q{special} reasoners to \RDF{}, a contingent of nodes must
be selected which are consistent with reasoners' runtime requirements.
}
\p{Of course, special reasoners cannot be expected to run on the domain of
the entire Semantic Web, or even on \q{very large} data sets in general.
A typical analysis will subdivide its problem into smaller parts
that are each tractable to custom reasoners \mdash{} in radiology, say,
a diagnosis may proceed by first selecting a medical
image series and then performing
image-by-image segmentation.  Applied to \RDF{}, this
two-step process can be considered a combination of general and special
reasoners: a general language like \SPARQL{} filters many nodes down to a smaller
subset, which are then mapped/deserialized to domain-specific representations
(including runtime memory).  For example, \RDF{} can link a patient to a
diagnostic test, ordered on a particular date by a particular doctor, whose
results can be obtained as a suite of images \mdash{} thereby selecting the
particular series relevant for a diagnostic task.  General reasoners
can \i{find} the images of interest and then pass them to
special reasoners (such as segmentation algorithms)
to analyze.  Insofar as this architecture is in effect, Semantic Web
data is a site for many kinds of reasoning engines.  Some of these engines
need to operate by transforming \RDF{} data and resources to an optimized,
internal representation.  Moreover, the semantics of these representations
will typically be closer to a high-level \NThree{} semantics taken as
\suigeneris{}, rather than as interpreted reductively as a notational
convenience for lower-level formats like \NTrip{}.  This appears
to undermine the justification for reductive semantics in terms of
\OWL{} reasoners.
}
\p{Perhaps the most accurate paradigm is that Semantic Web data has two
different interpretations, differing in being consistent with
special and general semantics, respectively.  It makes sense to
label these the \q{special semantic interpretation} or
\q{semantic interpretation for special-purpose reasoners}
(\SSI{}, maybe) and the \q{general semantic interpretation}
(\GSI{}), respectively.  Both these interpretations should be deemed
to have a role in the \q{semantics} of the Semantic Web.
}
\p{Another order of considerations involve the
semantics of \RDF{} nodes and \CH{} hypernodes
particularly with respect to uniqueness.  Nodes in \RDF{} fall into three classes:
blank nodes; nodes with values from a small set of basic types like strings and
integers; and nodes with \URL{}s which are understood to be unique across the
entire World Wide Web.  There are no blank nodes in \CH{}; and intrinsically
no \URL{}s either, although one can certainly define a \URL{} \i{type}.
There is nothing in the semantics of \URL{}s
which guarantees that each \URL{} designates a distinct internet resource;
this is just a convention which essentially, \i{de facto},
fulfills itself because
it structures a web of commercial and legal practices, not just digital
ones; e.g. ownership is uniquely granted for each internet domain name.
In \CH{}, a data type may be structured to reflect institutional
practices which guarantee the uniqueness of values in some context:
books have unique \ISBN{} codes; places have distinct \GIS{} locations,
etc.  These uniqueness requirements, however, are not intrinsically
part of \CH{}, and need to be expressed with additional axioms.  In
general, a \CH{} hypernode is a tuple of relatively simple values
and any additional semantics are determined by type
definitions (it may be useful to see \CH{} hypernodes as roughly analogous to
\CStruct{}s \mdash{} which have no \i{a priori} uniqueness mechanism).
}
\p{Also, \RDF{} types are less intrinsic to \RDF{} semantics than in \CH{}
(see \cite{HeikoPaulheim}).  The
foundational elements of \CH{} are value-tuples (via nodes expressing values,
whose tuples in turn are hypernodes).  Tuples are indexed by position, not by
labels: the tuple \NathFF{} does not in itself draw in the labels \q{name} or
\q{age}, which instead are defined at the type-level (insofar as type-definitions
may stipulate that the label \q{age} is an alias for the node in its
second position, etc.).  So there is no way to ascertain the semantic/conceptual
intent of hypernodes without considering both hyponode and hypernode types.  Conversely,
\RDF{} does not have actual tuples (though these can be represented as collections,
if desired); and nodes are always joined to other nodes via labeled connectors
\mdash{} there is no direct equivalent to the \CH{}
modeling unit of a hyponode being included in a hypernode
by position.
}
\p{At its core, then, \RDF{} semantics are built on the proposition that many
nodes can be declared globally unique by fiat.  This does not need to be
true of all nodes \mdash{} \RDF{} types like integers and floats are more
ethereal; the number \FourtySix{} in one graph is indistinguishable from \FourtySix{} in
another graph.  This can be formalized by saying that some nodes can be
\i{objects} but never \i{subjects}.  If such restrictions were not enforced,
then \RDF{} graphs could become in some sense overdetermined, implying
relationships by virtue of quantitative magnitudes devoid of semantic
content.  This would open the door to bizarre judgments like
\q{my age is non-prime} or \q{I am older than Mohamed Salah's
2018 goal totals}.
One way to block these inferences is to prevent nodes like
\q{the number \FourtySix{}} from being subjects as well as objects.
But nodes which are not primitive values \mdash{} ones, say, designating
Mohamed Salah himself rather than his goal totals \mdash{} are justifiably
globally unique, since we have compelling reasons to adopt a model
where there is exactly one thing which is \i{that} Mohamed Salah.
So \RDF{} semantics basically marries some primitive types which are
objects but never subjects with a web of globally unique but internally
unstructured values which can be either subject or object.
}
\p{In \CH{} the \q{primitive} types are effectively hypotypes; hyponodes
are (at least indirectly)
analogous to object-only \RDF{} nodes insofar as
they can only be represented via inclusion inside
hypernodes.  But \CH{} hypernodes are neither (in themselves) globally
unique nor lacking in internal structure.  In essence, an \RDF{}
semantics based on guaranteed uniqueness for atom-like
primitives is replaced by a semantics based on structured building-blocks
without guaranteed uniqueness.  This alternative may be considered in the
context of general versus special reasoners: since general reasoners
potentially take the entire Semantic Web as their domain, global
uniqueness is a more desired property than internal structure.
However, since special reasoners only run on specially selected data,
global uniqueness is less important than efficient mapping
to domain-specific representations.  It is not computationally
optimal to deserialize data by running \SPARQL{} queries.
}
\p{Finally, as a last point in the comparison between
\RDF{} and \CH{} semantics,
it is worth considering the distinction 
between \q{declarative knowledge} and \q{procedural knowledge} 
(see e.g. 
\cite[volume 2, pages 182-197]{BenGoetzel}).  According
to this distinction, canonical \RDF{} data exemplifies \i{declarative} knowledge
because it asserts apparent facts without explicitly trying to interpret or
process them.  Declarative knowledge circulates among software in canonical,
reusable data formats, allowing individual components to use or make inferences from
data according to their own purposes.
}
\p{Counter to this paradigm, return to
hypothetical {\sgapped}\USH{}{\egapped}{\sadded}Cyber-Physical{\eadded} examples, such as 
the conversion of Voltage data to acceleration data, which is a
prerequisite to accelerometers' readings being useful in most contexts.  Software
possessing capabilities to process accelerometers therefore reveals
what can be called \i{procedural} knowledge, because software
so characterized not only receives data
but also processes such data in standardized ways.
}
\p{The declarative/procedural distinction perhaps fails to capture how
procedural transformations may be understood as intrinsic to some semantic
domains \mdash{} so that even the information we perceive as
\q{declarative} has a procedural element.  For example, the
very fact that \q{accelerometers} are not
called \q{Voltmeters} (which are something else) suggests how the
Ubiquitous Computing community perceives voltage-to-acceleration
calculations as intrinsic to accelerometers' data.  But strictly speaking
the components which participate in \USH{} networks are not just
engaged in data sharing; they are functioning parts of the network because
they can perform several widely-recognized
computations which are understood to be central to the relevant
domain \mdash{} in other words, they have (and share with their
peers) a certain \q{procedural knowledge}.
}
\p{\RDF{} is structured as if static data sharing were the sole arbiter of
semantically informed interactions between different components,
which may have a variety of designs and rationales \mdash{} which
is to say, a
Semantic Web.  But a thorough account of formal communication semantics
has to reckon with how semantic models are informed by the implicit, sometimes
unconscious assumption that producers and/or consumers of data will
have certain operational capacities: the dynamic processes anticipated as
part of sharing data are hard to conceptually separate from the static
data which is literally transferred.  To continue the accelerometer
example, designers can
think of such instruments as \q{measuring acceleration} even though
\i{physically} this is not strictly true; their
output must be mathematically transformed for it to be interpreted in
these terms.  Whether represented via \RDF{} graphs or Directed Hypergraphs,
the semantics of shared data is incomplete unless the operations
which may accompany sending and receiving data are recognized as
preconditions for legitimate semantic alignment.
}
\p{While Ontologies are valuable for coordinating and integrating
disparate semantic models, the Semantic Web has perhaps influenced
engineers to conceive of semantically informed data sharing
as mostly a matter of presenting static data conformant to published
Ontologies (i.e., alignment of \q{declarative knowledge}).  In reality,
robust data sharing also needs an \q{alignment of \i{procedural}
knowledge}: in an ideal Semantic Network, procedural capabilities
are circled among components, promoting an emergent \q{collective
procedural knowledge} driven by transparency about code and libraries as
well as about data and formats.  The \CH{} model arguably supports this
possibility because it makes type assertions fundamental to semantics.
Rigorous typing both lays a foundation for procedural alignment
and mandates that procedural capabilities be factored in to assessments
of network components, because a type attribution has no meaning
without adequate libraries and code to construct and interpret
type-specific values.
}
\thindecoline{}
\p{Despite their differences, the Semantic Web, on the one hand, 
and Hypergraph-based frameworks,
on the other, both belong to the overall
space of graph-oriented semantic models.
Hypergraphs can be emulated in \RDF{}, 
and \RDF{} graphs can be organically mapped to 
a Hypegraph representation (insofar as 
Directed Hypegraphs with annotations are a 
proper superspace of Directed Labeled Graphs). 
Semantic Web Ontologies for computer source code can 
thus be modeled by suitably typed \DH{}s as well, 
even while we can also formulate Hypergraph-based 
Source Code Ontologies as well.  So, we are justified 
in assuming that a sufficient Ontology exists 
for most or all programming languages. 
This means that, for any given
procedure, we can assume that there is a 
corresponding \DH{} representation
which embodies that procedure's implementation.
}
\p{\phantomsection\label{detachedeval}
Procedures, of course, depend on \i{inputs} which are fixed for 
each call, and produce \q{outputs} once they terminate.  
In the context of a graph-representation, this implies that some
hypernodes represent and/or express values that are \i{inputs}, while 
others represent and/or express its \i{outputs}.  These
hypernodes are \i{abstract} in the sense (as in Lambda Calculus) that they
do not have a specific assigned value within the body, \i{qua} formal
structure.  Instead, a \i{runtime manifestation} of a \DH{}
(or equivalently a \CH{}, once channelized types are introduced) populates
the abstract hypernodes with concrete values, which in turn allows
expressions described by the \CH{} to be evaluated.
}
\p{These points suggest a strategy for unifying 
Lambda Calculi with Source Code Ontologies.
The essential construct in \mOldLambda{}-calculi is
that mathematical formulae include 
\q{free symbols} which are \i{abstracted}: sites 
where a formula can give rise to a concrete 
value, by supplying values to unknowns; or 
give rise to new formulae, via nested expressions.  
Analogously, nodes in a graph-based source-code 
representation are effectively \mOldLambda{}-abstracted 
if they model input parameters, which are 
given concrete values when the procedure runs.  
Connecting the output of one procedure to the 
input of another \mdash{} which can be modeled as a 
graph operation, linking two nodes \mdash{} is then 
a graph-based analog to embedding a complex expression 
into a formula (via a free symbol in {\sadded}the{\eadded} latter).    
}
\p{Carrying this analogy further, I earlier mentioned different 
\mOldLambda{}-Calculus extensions inspired by programming-language 
features such as Object-Orientation, exceptions, and 
by-reference or by-value captures.  
These, too, can be incorporated into a Source Code Ontology: 
e.g., the connection between a node holding a value passed to an 
input parameter node, in a procedure signature, is semantically 
distinct from the nodes holding \q{Objects} which are 
senders and receivers for \q{messages}, in Object-Oriented 
Parlance.  Variant input/output protocols, including 
Objects, captures, and exceptions, are certainly semantic 
constructs (in the computer-code domain) which 
Source Code Ontologies should recognize.  So we can see 
a convergence in the modeling of multifarious input/output protocols 
via \mOldLambda{}-Calculus and via Source Code Ontologies.  
I will now discuss a corresponding expansion in the 
realm of applied Type Theory, with the goal of 
ultimately folding type theory into this convergence as well.   
}
\vspace{-.1em}
\subsectiontwoline{Procedural Input/Output Protocols via Type Theory}
\p{\label{types}Parallel to the historical evolution where \mOldLambda{}-Calculus
progressively diversified and re-oriented toward concrete
programming languages, there has been an analogous (and
to some extent overlapping) history in Type Theory.
When there are multiple ways of passing input to a
function, there are {\sgapped}at{\egapped} potentially multiple kinds
of function types.  For instance, Object-Orientation inspired
expanded \mOldLambda{}-calculi that distinguish function
inputs which are \q{method receivers} or \q{\this{} objects} from
ordinary (\q{lambda}) inputs.  Simultaneously, Object-Orientation also
distinguishes \q{class} from \q{value} types
and between function-types which are \q{methods} versus ordinary
functions.  So, to take one example, a function telling
us the size of a list can exhibit two different types, depending
on whether the list itself is passed in as a method-call target
(\listsize{} vs. \sizelist{}).
}
\p{One way to systematize the diversity of type systems
is to assume that, for any particular type system, there
is a category \tCat{} of types conformant to that system.  This requires
modeling important type-related concepts as \q{morphisms} or maps
between types.  Another useful concept is an \q{endofunctor}:
an \q{operator} which maps elements in a category
to other (or sometimes the same) elements.  In a \tCat{} an endofunctor
selects (or constructs) a type \tyTwo{} from a type \tyOne{} \mdash{} note how this is
different from a morphism which maps \i{values of} \tyOne{} to \tyTwo{}.
Type systems are then built up from a smaller set of \q{core} types via
operations like products, sums, enumerations, and forming \q{function-like} types.
}
\p{We may think of the
\q{core} types for practical programming as number-based
(booleans, bytes, and larger integer types), with everything else built up by aggregation
or encodings (like \ascii{} and \unicode{}, allowing types to include text and 
alphabets; or pixel-coordinates and colors, 
allowing for graphical/visual components).\footnote{In other contexts, however, non-mathematical core types may be appropriate: for example,
the grammar of natural languages can be modeled in terms of a type system whose core are
the two types \tyNoun{} and \tyProposition{} and which also includes
function types (maps) between pairs or tuples of types (verbs,
say, map \tyNoun{}s \mdash{} maybe multiple nouns, e.g. direct objects
\mdash{} to \tyProposition{}s).
}  Ultimately, a type system \tCat{} is characterized
(1) by which are its core types and
(2) by how aggregate types are built from simpler ones
(which essentially involves endofunctors and/or products).
}
\p{In Category Theory, a Category \cCat{} is called \q{Cartesian Closed} if
for every pair of elements \eOne{} and \eTwo{} in \cCat{} there is an
element \eOneToeTwo{} representing (for some relevant notion of
\q{function}) all functions from \eOne{} to \eTwo{} \cite{RBrown}.  The stipulation that
a type system \TyS{} include function-like types is roughly equivalent, then,
to the requirement that \TyS{}, seen as a Category, is Cartesian-Closed.
The historical basis for this concept (suggested by the terminology)
is that the construction to form function-types is an \q{operator}, something that
creates new types out of old.  A type system
\TyS{} may then be \q{closed} under products:
if \tOne{} and \tTwo{} are in \TyS{} then \tOneTimesTTwo{} must be as well.
Analogously, \TyS{} supports function-like types if it 
is closed under a kind of \q{functionalization} operator \mdash{} if the
\tOneTimesTTwo{} product can be mapped onto a function-like type
\tyOneTotyTwo{}.
}
\p{In general, more sophisticated type systems \TyS{} are described by
identifying new kinds of inter-type operators and studying those
type systems which are closed under these operators: if \tyOne{} and
\tyTwo{} are in \TyS{} then so is the combination of \tyOne{} and
\tyTwo{}, where the meaning of \q{combination} depends on the
operator being introduced.  Expanded \mOldLambda{}-calculi \mdash{} which
define new ways of creating functions \mdash{} are correlated with new
type systems, insofar as \q{new ways of creating functions}
also means \q{new ways of combining types into function-like types}.
}
\p{Furthermore, \q{expanded} \mOldLambda{}-calculi generally involve
\q{new kinds of abstraction}: new ways that the building-blocks
of functional expressions, whether these be mathematical formulae
or bodies of computer code, can be \q{abstracted}, treated as
inputs or outputs rather than as fixed values.  In this chapter, I attempt to
make the notion of \q{abstraction} rigorous by analyzing it against
the background of \DH{}s that formally model computer code.
So, given the correlations I have just described between
\mOldLambda{}-calculi and type systems \mdash{} specifically, on
\TyS{}-closure stipulations \mdash{} there are parallel correlations
between type systems and \i{kinds of abstraction defined on
Channelized Hypergraphs}.  I will now discuss this further.
}
\subsubsection{Kinds of Abstraction}
\p{The \q{abstracted} nodes in a \CH{} are loosely classifiable as
\q{input} and \q{output}, but in practice there are various paradigms
for passing values into and out of functions, each with their own semantics.
For example, a \q{\this{}} symbol in \Cpp{} is an abstracted, \q{input}
hypernode with special treatment in terms of overload resolution and access
controls.  Similarly, exiting a function via \returnct{} presents
different semantics than exiting via \throw{}.  As mentioned earlier,
some of this variation in semantics has been formally modeled
by different extensions to \mOldLambda{}-Calculus.
}
\p{So, different hypernodes in a \CH{} are subject to different kinds of abstraction.
Speaking rather informally, hypernodes can be grouped into \i{channels} based on
the semantics of their kind of abstraction.  More precisely,
channels are defined initially on \i{symbols}, which are associated with hypernodes:
in any \q{body} (i.e., an \q{implementation graph}) hypernodes can be grouped
together by sharing the same symbol, and correlatively sharing the
same value during a \q{runtime manifestation} of the \CH{}.  Therefore,
the \q{channels of abstraction} at work in a procedure can be identified
by providing a name representing the \i{kind} of channel and a list of
symbols affected by that kind of abstraction.{\sgapped}In the notation I adopt here,
conventional lambda-abstraction like \lXY{} would be written as \CHlXY{}.
{\egapped}
}
\p{I propose \q{Channel Algebra} as a tactic for capturing the 
semantics of channels, so as to model programming languages' 
conventions and protocols with respect to calls between 
procedures.  Once we get beyond the basic contrast between 
\q{input} and \q{output} parameters, it becomes necessary to 
define conditions on channels' size, and on how 
channels are associated with different procedures that may 
share values.  Here are several examples: 
\begin{itemize}\item{} In most Object-Oriented languages, any procedure can 
have at most one \this{} (\q{message receiver}) object.  
Let \sCh{} model a \q{Sigma} channel, as in \q{Sigma Calculus}
(written as \sigmaCalculus{}: see e.g. \cite{MartinAbadi},
\cite{CamposVasconcelos},
\cite{KathleenFisher}, \cite{EdwardZalta}, etc.).
We then have the requirement than any procedure's 
\sCh{} channel can carry at most one value.
\item{} \label{retexc} In all common languages which have exceptions, 
procedures can \i{either} throw an exception \i{or} 
return a value.  If \return{} and \exception{} 
model the channels carrying standard returns and 
thrown exceptions, respectively, this convention 
translates to a requirement that the two channels 
cannot both be non-empty.
\item{} A thrown exception cannot be handled as an ordinary 
value.  The whole point of throwing exceptions is to 
disrupt ordinary program flow, which means the exception 
value is only accessible in special constructs, like a 
\catch{} block.  One way to model this restriction is 
to forbid \exception{} channels from transferring values to 
other channels.  Instead, 
exception values are bound (in \catch{} blocks) to 
lexically-scoped symbols (I will discuss channel-to-symbol 
transfers below).
\item{} Suppose a procedure is an Object-Oriented method 
(it has a non-empty \q{\sCh{}} channel).  Any other methods 
called from that procedure will \mdash{} at least in the 
conventional Object-Oriented protocol \mdash{} automatically 
receive the enclosing method's \sigmach{} channel unless 
a different object for the called method is supplied expressly. 
\item{} \phantomsection\label{chaining} In the object-oriented 
technique known as \q{method chaining},
one procedures' \return{} channel is transferred to a 
subsequent procedures' \sCh{} channel.  The pairing of 
\return{} and \sCh{} thereupon gives rise to one 
function-composition operator.  With suitable restrictions 
(on channel size), \return{} and \lambda{} channels engender a 
different function-composition operator.  So channels can be 
used to define operators between procedures which yield new 
function-like values (i.e., instances of function-like 
types).  In some cases, function-like values defined via 
inter-function operators can be used in lieu of those 
instantiated from implemented procedures (although the 
specifics of this substitutability \mdash{} an example of 
so-called \q{eta ($\eta{}$) equivalence} \mdash{} varies by language). 
\end{itemize}
}
\p{The above examples represent possible combinations or 
interconnections (sharing values) between channels, 
together with semantic restrictions on when 
such connections are possible.  In this chapter, I 
assume that notations describing these connections and 
restrictions can be systematized into a \q{Channel Algebra}, 
and then used to model programming language conventions and 
computer code.  A basic example of inter-channel 
aggregation would be how a \lambda{} channel, combined 
with a \return{} channel, associated with one procedure, 
yields a conventional input/output pairing.  
One particular channel formation \mdash{} \lambdaPLUSreturn{}, 
say \mdash{} therefore models 
the basic \mOldLambda{}-Calculus and, simultaneously, 
a minimal definition of function-like types.
Notionally, a procedure is, in the simplest
conceptualization, the unification of an
input channel and an output channel 
\mdash{} written, say, \chaOnePluschaTwo{} 
(with the \chplus{} possibly holding extra 
stipulations, like \cChaOne{} and \cChaTwo{}
cannot both be non-empty).
So a \q{channel sum} creates the basic
foundation for a procedure, analogous to
how input and output graph elements
yield the foundations for morphisms
in Hypergraph Categories.
More complex channel combinations and protocols 
can then model more complex 
variations on \mOldLambda{}-Calculi and on 
programming language type systems. 
}
\subsubsection{Channelized Type Systems}
\p{Collectively, to summarize my discussion to this point,
I will say that formulations 
describing channel kinds, their restrictions, and
their interrelationships
{\sgapped}describe{\egapped}{\sadded}outline{\eadded} a \i{Channel Algebra},
which express how channels combine to
describe possible function signatures \mdash{} and 
accordingly to describe functional \i{types}.  The
purpose of a Channel Algebra is, among other things, to elucidate 
how formal languages (like programming languages) formulate functions 
and procedures, and
the rules they put in place for inputs and outputs.  If \Chi{} is a
Channel Algebra, a language adequately described by its formulations 
(channel kinds, restrictions, and interrelationships) can be called a
\Chi{}-language.  The basic \mOldLambda{}-Calculus can be described as a
\Chi{}-language for the algebra defined by a minimal 
\lambdaPLUSreturn{} combination (with \return{} channels 
restricted to at most one element).
Analogously, a type system \TyS{} is 
a \q{\Chi{}-type-system}, and is \q{closed} with respect to \Chi{}, 
if valid signatures
{\sadded}characterized{\eadded}{\sgapped}described{\egapped} using
channel kinds in
\Chi{} correspond to types found in \TyS{}.  Types may be less granular than
signatures: as a case in point,
functions differing in signature only by whether they
throw exceptions may or may not be deemed the same type.  But a channel
construction on types in \TyS{} must also yield a type in \TyS{}.
}
\p{I say that a type system is \i{channelized} if it is
closed with respect to some Channel Algebra.
Channelized Hypergraphs are then \DH{}s whose type system is Channelized.
We can think of channel constructions as operators which combine
groups of types into new types.
Once we assert that a \CH{} is Channelized, we know that there is a mechanism
for describing some Hypergraphs or subgraphs as \q{procedure 
implementations} some of whose hypernodes are subject to
kinds of abstraction present in the relevant Channel Algebra.  
Channel formulae and signatures describe 
source-code norms which could also be expressed via more conventional
Ontologies.  So Channel Algebra can be seen as a generalization of
(\RDF{}-environment) Source Code Ontology
(of the kinds studied for example by
\cite{ImanKeivanloo}, \cite{WernerKlieber},
\cite{JohnathanLee}, \cite{TurnerEden},
\cite{ReneWitte}, \cite{PornpitWongthongtham}).  Given the relations between
\RDF{} and Directed Hypergraphs (despite differences I have discussed here),
Channel Algebras can also be seen as adding to Ontologies governing
Directed Hypergraphs.  Such is the perspective I will take
for the remainder of this chapter.
}
\p{For a Channel Algebra \Chi{} and a \Chi{}-closed type system
(written, say) \TySChi{}, \Chi{} extends \TyS{} because function-signatures
conforming to \Chi{} become types in \TyS{}.  At the same time,
\TyS{} also extends \Chi{}, because the elements that
populate channels in \Chi{} have types within \TyS{}.  Assume that for
any type system, there is a
partner \q{Type Expression Language} (\TXL{}) which governs how type
descriptions (especially for aggregate types that do not have a
single symbol name) can be composed consistent with the logic of
the system.  The \TXL{} for a type-system \TyS{} can be
notated as \TXLTyS{}.  If \TyS{} is channelized then its
\TXL{} is also channelized \mdash{} say, \TXLTySChi{} for some \Chi{}.
}
\p{Similarly, we can then develop for Channel Algebras a \i{Channel
Expression Language}, or \CXL{}, which can indeed be integrated with
appropriate \TXL{}s.  Formal declarations of channel axioms 
\mdash{} e.g., restrictions on channel sizes, alone or in combination \mdash{} 
are examples of terms that should be representable in a \CXL{}.   
However, whereas the \CXL{} expressions I have described so far
{\sgapped}describe{\egapped}{\sadded}elucidate{\eadded} the overall shape of 
channels \mdash{} which channels exist in a given context and their sizes
\mdash{} \CXL{} expressions can also add details concerning the \i{types} of
values that can or do populate channels.
\CXL{} expressions with these extra specifications then become
function signatures, and as such type-expressions in the
relevant \TXL{}.  A channelized \TXL{} is then a
superset of a \CXL{}, because it adds \mdash{} to \CXL{} expressions
for function-signatures \mdash{} the stipulation that a particular
signature does describe a \i{type}; so \CXL{} expressions
become \TXL{} expressions when supplemented with a proviso
that the stated \CXL{} construction describes a
function-like type's signature.  With such a proviso, descriptions of
channels used by a function qualifies as a type attribution,
connecting function symbol-names
to expressions recognized in the \TXL{} as describing a type.
}
\p{Some \TXL{} expressions
designate function-like types, but not all, since there are many types
(\int{}, etc.) which do not have channels at all.
While a \TXL{} lies \q{above} a \CXL{} by adding provisos that
yield type-definition semantics from \CXL{} expressions,
the \TXL{} simultaneously in a sense lies \q{beneath} the
\CXL{} in that it provides expressions for the non-functional
types which in the general case are the basis for \CXL{}
expressions of functional types,
since most function parameters \mdash{} the input/output values
that populate channels \mdash{} have non-functional types.
Section \sectsym{}\hyperref[sFive]{\ref{sFive}} will discuss the elements that \q{populate}
channels (which I will call \q{carriers}) in more detail.
}
\p{In the following sections I will sketch a 
Channel Algebra that codifies the graph-based representation
of functions as procedures whose inputs and
outputs are related to other functions by variegated semantics
(semantics that can be catalogued in a Source Code
Ontology).  With this foundation, I will argue that Channel-Algebraic
type representations can usefully model higher-scale
code segments (like statements and code blocks)
within a type system, and also how type interpretations
can give a rigorous interpretation to modeling
constructs such as code specifications and
\q{gatekeeping} code.  I will start this
discussion, however, by expanding on the idea of 
employing code-graphs \mdash{} hypergraphs annotated according 
to a Source Code Ontology \mdash{} to represent 
procedure implementations, and therefore to model 
procedures as instances of function-like types.
}
