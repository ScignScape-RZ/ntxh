\p{The question of whether computers can be programmed to understand 
language may be philosophical, but it overlaps with 
broad methodological bifurcations: after all, linguists 
\i{are} programming computers to \q{understand} language, at 
least to some approximation.  Given that computational 
linguistics is now a well-established practice, we can 
consider how this program for investigating the nature of 
language orients into linguistics as a whole: to what degree 
are the computers really \q{understanding} their linguistic 
input?  How much does \i{behavior} consistent with language-understanding 
suggest actual understanding?  Is linguistic competence mostly a 
behavioral phenomenon, or something more holistic and (inter-) subjective?  
Are the imperfections of automated Natural Language Processing 
inevitable, and if so, does that foreclose the possibility of 
\NLP{} engines being considered truly linguistic?  That is, 
should we treat flawed and oversimplistic (but practically useful) 
\NLP{} software \mdash{} or \q{personas} driven by this software, like 
\q{digital assistants} \mdash{} as bonafide (if rather primitive) 
participants in the world of human language?  Or are they merely 
machines that simulate linguistic behavior without manifesting 
real linguistic behavior, as a computer simulation of a 
celestial galaxy is not a real galaxy?  
}
\p{These are methodological as well as thematic questions.  There is a 
wide swath of formal and computational linguistics, for instance, for 
which the measure of a theory is its chance of being operationalized 
on \NLP{} terms and within \NLP{} tools, yielding automated systems whose accuracy and/or 
computational efficiency competes favorably with other systems.  
Faithfulness to how \i{humans} process language is at most a secondary 
concern.  Conversely, there is a broad literature in Cognitive Linguistics 
and the Philosophy of Language for which uncovering the cognitive and 
interpretive registers through which \i{we} understand, produce, and 
are affected by language is the main goal.  For scholars chasing 
that telos, failure to encode theoretical models in mathematical 
or software systems is not \i{prima facie} an explanatory limitation 
\mdash{} conversely, we might take this as evidence that cognitive models 
are addressing the deep, subtle realities of language that are 
opaque to computer simulation.   
}
\p{Then there is hybrid work, like attempts to formalize 
Cognitive Grammar (Matt Selway \cite{MattSelway}, 
Kenneth Holmqvist \cite{KennethHolmqvist}), or 
other branches of Cognitive Linguistics 
(cf. Terry Regier's influential \cite{TerryRegier}), 
or Conceptual Space Theory as initiated by 
Peter \Gardenfors{} (which has seen several attempts at 
mathematical-computational formalization, such as 
Frank Zenker, Martin Raubal, and Benjamin Adams's 
metascientific perspectives, and more recent 
Category-Theoretic structures linked to mathematicians 
such as Bob Coecke and David Spivak \cite{InteractingConceptualSpaces}).  
To this list we could add research that extends beyond 
language alone to broader cognitive-perceptual and 
conceptual themes, like formal descriptions rooted in 
Husserlian analyses by phenomenologists whose methods 
encompass some computer-scientific techniques, like 
Barry Smith (as in \cite{BittnerSmithDonnelly}) and 
Jean Petitot (see \cite{JeanPetitot}); we can see 
these accounts as generalizing cognitive-linguistic 
theories by noting the phenomenological basis 
of linguistic phenomena, as articulated by (say) Olav 
K. Wiegand (\cite{OlavKWiegand}, \cite{WiegandGestalts}) 
and Jordan Zlatev (\cite{JordanZlatev}).  
In each of the works just cited (prior anyhow to the 
last two)  
we can find formal/computational models whose 
rationale is, in large part, to 
shed light on human cognitive processes 
(albeit not necessarily translating to practical 
\NLP{} components in any straightforward way). 
}
\p{This kind of \q{intermediate} research is perhaps 
under-appreciated, because it neither accepts the dismissive 
attitude that formal models are a distraction from 
the analysis of \q{real} language, nor the reductionistic 
faith that language is \i{intrinsically} computational, 
so that progress toward ideal \NLP{} avatars is just a matter 
of time.  To be sure, layering formal systems on a 
cognitive/phenomenological foundation adds a complexity of 
theoretical structure, which could prompt questions about the 
efficacy of the theoretical dilation: is extra complexity 
desirable as an end in itself, if the new formalizations 
have only limited explanatory or practical pay-offs?  
If there is a human kernel in language that is intrinsically 
non-computable and non-formalizable, does analysis of 
language truly benefit from complex but only partly 
applicable structural overlays?  On the other hand, if 
language \i{is} computationally tractable, shouldn't 
\NLP{} implementation be a factor in assessing which 
formal models are worthy of attention? 
}
\p{Perhaps for these kinds of considerations, linguistics seems to 
bifurcate between a camp that essentially ignores 
computational methodology and resources and a camp 
that centers its whole attention on building better automated 
\NLP{} tools.  Left out of this division is research that 
invokes formal models as explanatory vehicles while not 
enmeshing them in an ecosystem oriented toward automation \mdash{} 
the difference between deploying formal repesentations to 
model (some aspects) of language processing, syntax, and 
semantics, and trying to program software to \i{automate} the 
construction, translation, and pipeline between and among
formal models.  When situating research relative to 
computational linguistics, we should bear in mind the 
metatheoretical point that \i{incorporating} formal 
schema into linguictis models does not \i{necessarily} 
mean committing ourselves to a task of programming 
computers to discover the target representations 
on their own, given raw linguistic input.
}
\p{But this perspective is not only metatheoretical: I believe  
that the nature of language is \i{intrinsically} \q{hybrid} 
in a manner that warrants neither blind faith in 
automation nor \i{a priori} disengagement with formalizing 
projects.  This is first of all because language is neither 
wholly isolated from other cognitive phenomena nor 
without some structural autonomy: it is reasonable to
suppose that there are distinct intellectual faculties 
internal to our understanding of language, while other 
reasonings intrinsic to parsing the form and intentions 
of a linguistic unit are drawn from the wider 
inventory of situational, conceptual, social, and 
practical/enactive cognition.  Sentences can vary 
in terms of their context-sensitivity and the degree to 
which extralinguistic rationality is implicit in 
grasping intended meanings.  So neither a theory which 
ignores extralinguistic cognition, nor one which 
treats \i{all} linguistic processing as inseperable 
from the totality of our cognitive processes from moment to moment, 
are complete; the larger account needs to place 
sentences on a spectrum, where extralinguistic and (I'll say)
\q{intra-linguistic} theoretical machinery is 
available to analyze different sentences as their 
form and context demands. 
}
\newsavebox{\mxbox}
\begin{lrbox}{\mxbox}
\begin{tikzpicture}[baseline=(current bounding box.center)]
\matrix (m) [matrix of math nodes,nodes in empty cells,right delimiter={.},left delimiter={.}]{
	\text{Logicomorphic}  & \text{\rotatebox{90}{$\parallel$}} & \text{Extra-Linguistic} \\
	\| &  &   \| \\
	\text{Intra-Linguistic}  & \text{\rotatebox{90}{$\parallel$}}  &  \text{Interpretive} \\
};
\draw (m-1-1)-- (m-3-3);
\draw (m-3-1)-- (m-1-3);
\end{tikzpicture}
\end{lrbox}
\p{Extending this point, I believe a comparable spectrum 
matches the duality of language seen as intrinsically 
formalizable and computable or as too subtle, social/cultural, 
embodied, and context-dependent to be tractable to any 
computer or any idealized logicomathematical abstraction.  
Some sentence are more logically straightforward; others 
are more elusive, requiring holistic and context-sensitive 
interpretation on conversants' parts to be understood.  
Combined with my claims last paragraph, I argue accordingly 
that we can (at least as a suggestive picture) view 
linguistic artifacts (canonically, sentences) 
along a two-axis spectrum defined both by extralinguistic 
integration (or lack thereof) and by formal tractability 
(or lack thereof), like so: (this is intended as an intuitive 
sketch, not a formal model). 
\begin{center}\usebox{\mxbox}\end{center}
I will elucidate the terms on that picture later.  Summarily, though, 
I claim that \i{some} sentences evince logically straightforward 
compositional patterns that can be analyzed \i{either} at the
language-specific (syntactic or semantic) level \i{or} within 
cognitive registers outside of language proper (e.g., situational 
shemata); conversely, some sentences have nuances that 
call for interpretive judgments which appear to transcend 
formal simulacra outside the full range of human 
intelligence, emotion, and embodiment, \i{either} in 
terms of parsing complex syntactic or semantic structures 
\i{or} in grounding linguistic phenomena in ambient contexts.  
My overall point is then that sentences take a spectrum of 
models spanned by these axes; no one paradigm is 
self-contained as a metalinguistic commitment. 
} 
\p{In effect, the choice between paradigms wherein language is or 
is not formally/computationally tractable, and between paradigms 
wherein language is or is not intellectually autonomous \visavis{} 
our total cognitive faculties, should not be seen as a 
metaphysical alternative anterior to language as a totality.  
Instead, these spectra are threaded into language internally, 
competing polarities which rise or fall from sentence to 
sentence.  Language is not \i{intrinsically} either formal 
or non-formal; autonomous or non-autonomous.    
}
\p{But at the same time, sentences are clearly phenomena of the 
same ilk; the distinctions I have made are not so sharp as to 
disrupt the ontological similitude among sentences, so that 
two sentences (however much they differ on the spectra of 
my diagram) are still manifestations of the same ontological 
place; they are still roughly the same \i{sort} of existents.  
Accordingly, we should conclude philosophically that there are 
certain aspects of sentences that lie beneath the formal/interpretive  
and intralinguistic/extealinguistic dualities.  There are, in short, 
paleostructures in language that manifest \i{either} with formal 
specificity or with contextual nuance; \i{either} 
internal to syntax or semantics or external to intrinsically 
linguistic cognition; varying from sentence to sentence.  
This paper will present a theory of one such paleostructure, 
drawing inspiration both from formal theories (Dependency Grammar, 
Type-Theoretic Semantics, Generative Lexicon) and from 
more philosophical approaches (Cognitive Grammar, Semiotics, Phenomenology).    
}
\p{The central element in my analysis is the \i{conceptual modification} 
implied or effected by one word in the presence of another word.  
More precisely, some words' linguistic roles can be analyzed as 
adding cognitive detail (conceptual and/or perceptual and/or pragmatic) 
to the ideas or referents signified via other words.  The underlying scheme, 
at this level where the model is undeveloped and thus fairly simple, 
is close to Dependency Grammar: in lieu of a head/dependent relation 
we can treat one word (or similar lexical unit) as a \i{modifier} and the  
second as a \q{ground}.  I believe this two-pronged picture is not 
completely vacuous, but is general and underspecified enough to
spread over both syntax and semantics, and over competing 
paradigms.  I will call the modifier's effect on its ground 
a \i{transform}, and the two words together (taking \q{word} as a 
convenient designation for lexemes in general) as a 
\i{transform pair}.
}
\p{I will argue that the underlying transform-pair concept is 
amenable to both more formal and more philophical/interpretive 
development.  One the one hand, transform-pairs can sometimes 
be mapped explicitly to dependency or link-grammar pairs, 
so the theory can be treated as a philosphical preliminary 
or motivation for Dependecy or Link Grammar.  Relatedly, 
transforms themselves can be integrated into type systems 
(e.g., the kinds of transforms associated with adjectives 
depend on their ground being typed as a noun), so 
a theory of transform-pairs can motivate elaboration in a 
Type-Theoretic context.  On the other hand, we can focus 
on the interpretive and situational nuanced often evident in 
cognitive transformations to find evidence for linguistic 
phenomena which do not fit neatly into Dependecy Grammar 
or Type-Theoretic (or any other) formalization.  
These various continuations, which I will present further 
over the next several sections, try not to foreclose either 
formal or philosophical/interpretive paradigms.  The goal is 
to trace both formal models of language and alterntive 
models \mdash{} for which excessive formalization is reductionistic 
and depends on ad-hoc avoidance of many real-world 
significations \mdash{} to a common structural kernel from which 
both perspectives can be deployed on a case-by-case 
(e.g., sentence-by-sentence) basis.
}
\p{In keeping with the perspective that formal models have \i{some} 
merit, I have tried to orient the presentation around certain 
computational-linguistic techniques.  Some of my examples are 
drawn from popular annotated corpora, and I provide models 
of other examples as processed by representative \NLP{} technologies 
(e.g., Malt Parser trained against the most recent Universal 
Dependency training data at the time of writing).   
I have packaged the examples and supporting code into an 
open-source data set for purposes of demonstration.  
I do not dwell on the accuracy of the \NLP{} components, 
in part because I am motivated in this paper to examine 
how computational methods can be employed as explanatory 
tools separate and apart from their feasibility in 
automated pipelines.  That is, the computational 
resources I present here are designed more as technological 
supplements to philosophical, interpretive, and 
speculative examinations of interesting linguistic 
examples, rather than as algorithms ultimately targeted 
at fully automated \NLP{} frameworks.  I incorporate 
code and data as a supplement to my argumentation in the 
hope that this can serve as an example of computational 
linguistics adopted outside the priorities of \NLP{} 
automation and Artificial Intelligence in its more reductive, 
science-fictional sense.  I am not aspiring to 
develop code or theoretical claims that could advance the 
hypothetical project of implementing artificial agents that 
can mimic and understand human language and behavior in 
all its subjectivity and complexity.  However, I \i{do} 
want to leverage certain computational techniques as  
offering their own explanatory perspectives on structures 
in human language.  
}
\p{The remainder of this paper will draw and expand on the outline 
of terms and structures sketched thus far, specifically the 
contrast of intra/extra-linguistic and \q{logicomorphic}/interpretive 
aspects from my \q{diagram} of sentences' paradigmatic spectrum, 
and the basic modifier/ground transform-pair account.  I 
use the phrase Cognitive Transform Grammar for the core notion of 
\q{transform-pairs} as, at core, cognitive phenomena, which nonetheless 
allow for further exposition via different methods.  I will explain 
some of this variation in the first section.  
}
