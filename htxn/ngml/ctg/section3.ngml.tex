\section{Gaps in Truth-Theoretic Semantics}
\label{s3}
\label{sec:Gaps}
\p{I take as a given that typical sentences have a propositional core, 
against which they take a performative stance 
(which can be outright assertion, or else asserting speakers' 
more complex propositional attitudes).  I would further 
say that \i{truth-theoretic} semantics, in particular, 
is organized around this propositional content as the 
core target of linguistic analysis.  I am thinking of 
truth-theoretic semantics in a broad sense, perhaps the 
most influential paradigm in the Philosophy of Language 
and by extension philosophy and linguistics in general 
(not to mention Computer Science and Artificial Intelligence 
research).  The most notable counter-paradigm is Cognitive Linguistics; 
consider George Lakoff and Mark Johnson's extended critique 
of truth-theoretic paradigms in \i{Philosophy in the Flesh}.  
Adherents of the latter perspective need not 
dispute the logical substance of language artifacts' propositional 
content, but tend to direct theoretical attention not to 
the nature of propositional content itself, but to the cognitive 
processes through which this content is understood.}

\p{As I argued earlier, many sentences do not simplistically 
reproduce the logical structure of their propositional 
content, so models of that structure are only tangentially 
relevant to analysis on the language side.  This is why we 
need distinct analyses, beyond a mere logical gloss, 
covering the interpretive steps leading to holistic 
sentence-understanding.   This section will consider several 
cognitive and pragmatic themes moving toward a general theory 
of this phenomenon.}

\subsection{Enaction and Illocutionary Force}

\p{I will start by reviewing illocutionary pragmatics, to 
identify some of the contextual and interpretive 
transformations that pertain to mapping surface language 
to propositional contents.  My point is to establish 
what should be a common theory of logicality that can be
shared by both critics and defenders of \q{truth-theoretic}
paradigms, on which basis their legitimate disputes 
can be investigated.}

\p{Many linguists (on both sides, I would say, of my 
central truth-theoretic pro/con), 
seem to analyze hedges like \q{could you please}
as merely dressing over crude commands: we don't
want to come across as giving people orders, but
sometimes we do intend to ask people to do specific
things.  As a result, we feel obliged to couch the
request in conversational gestures that signal
our awareness of how bald commands may lie outside
the conversational norms.  These ritualistic
\q{could you please}-like gestures may have
metalinguistic content, but {\mdash} so the theory
goes {\mdash} they do not \i{semantically} alter
the speech-act's directive nature.}

\p{The problem with this analysis is that sometimes
directive and \q{inquisitive} dimensions can
overlap:

\begin{sentenceList}

\sentenceItem{} \swl{itm:almond}{Do you have almond milk?}{pra}
\sentenceItem{} \swl{}{Can you get MsNBC on your TV?}{pra}
\sentenceItem{} \swl{itm:needcorkscrew}{This isn't a screw-cap bottle: I need a corkscrew.}{pra}
\end{sentenceList}

These \i{can} be read as bare directives, and would
be interpreted as such if the hearer believed the
speaker already knew that yes, he has almond milk, and yes,
he gets MsNBC.  In (\ref{itm:needcorkscrew}), if both parties
know there's one corkscrew in the house,
the statement implies a directive to fetch \i{that} corkscrew.
But, equally, (\ref{itm:almond})-(\ref{itm:needcorkscrew}) can \i{also} be read as bare
questions with no implicature: say, as fans of
almond milk and MsNBC endorsing those selections,
or pointing out that opening the bottle
will need \i{some} corkscrew.
And, meanwhile, (\ref{itm:almond})-(\ref{itm:needcorkscrew})
can \i{also} be read as a mixture of the
two, as if people expressed themselves like this:

\begin{sentenceList}

\sentenceItem{} \swl{}{I think the window is open, can you close it?}{pra}
\sentenceItem{} \swl{}{I see you have almond milk, can I have some?}{pra}
\sentenceItem{} \swl{}{If you get MsNBC, can you turn on Rachel Maddow?}{pra}
\sentenceItem{} \swl{}{If there is a corkscrew in the house, can you get it?}{pra}
\end{sentenceList}
}

\p{I think the mixed case is the most prototypical, and pure
directives or inquiries should be treated as degenerate
structures where either directive or inquisitive content
has dropped out.  After all, even a dictatorial
command includes the implicit assumption that the order
both makes sense and is not impossible.  On
the other hand, we don't ask questions for no
reason: \q{do you have almond milk} may be a
suggestion rather than a request, but it still
carries an implicature (e.g., that the addressee
\i{should} get almond milk).}

\p{Ordinary requests carry the assumption that addressees
can follow through without undue inconvenience,
which includes a package of assumptions about both
what is currently the case and what is possible.
\q{Close the window} only has literal force if the
window is open.  So, when making a request, speakers
have to signal that they recognize the request involves
certain assumptions and are rational enough to
accept modifications of these assumptions in
lieu of literal compliance.  This is why
interrogative forms like \q{can you} or
\q{could you} are both semantically nontrivial
and metadiscursively polite: they leave open the
possibility of subsequent discourse framing the original
request just as a belief-assertion.  Developments
like \i{can you open the window} {\mdash} \i{no, it's closed}
are not ruled out.  At the same time, interrogative forms
connote that the speaker assumes the addressees can
fulfill the request without great effort: an implicit
assumption is that they \i{can} and also \i{are
willing to} satisfy the directive.  This is an
assumption, not a presumption: the speaker
would seem like a bully if he acted as if he
gave no thought to his demands being too much
of an imposition {\mdash} as if he were taking
the answer to \q{can you} questions for granted.
This is another reason why requests
should be framed as questions.  So, in short,
\q{commands} are framed as questions because the speaker
literally does not know for sure whether the command is
possible; given this uncertainty a command \i{is} a question,
and the interrogative form is not just a non-semantic
exercise in politesse.}

\p{Sometimes the link between directives and
belief assertions is made explicit.  A common
pattern is to use \i{I believe} or \i{I believe that} as an
implicature analogous to interrogatives:

\begin{sentenceList}

\sentenceItem{} \swl{}{I believe you have a reservation for Jones?}{pra}
\sentenceItem{} \swl{}{I believe this is the customer service desk?}{pra}
\sentenceItem{} \swl{}{I believe we ordered a second basket of garlic bread?}{pra}
\sentenceItem{} \swl{}{I believe you can help me find computer
accessories in this section?}{pra}
\end{sentenceList}

These speakers are indirectly signaling what they want
someone to do by openly stating the requisite
assumptions {\mdash} \i{I believe you can} in place
of \i{can you?}.  The implication is that
such assumptions translate clearly to a
subsequent course of action {\mdash} the guest who
\i{does} have that reservation should be checked in;
the cashier who \i{can} help a customer find
accessories should do so.  But underlying these
performances is recognition that
illocutionary force is tied to background
assumptions, and conversants are reacting to
the propositional content of those assumptions
as well as the force itself.  If I \i{do} close the
window I am not only fulfilling
the request but also confirming that the window
\i{could} be closed (a piece of information
that may become relevant in the future).}

\p{In sum, when we engage pragmatically with other
language-users, we tend to do so cooperatively,
sensitive to what they wish to achieve with
language as well as to the propositional
details of their discourse.  But this often means
that I have to interpret propositional
content in light of contexts and implicatures.
Note that both of these are possible:

\begin{sentenceList}

\sentenceItem{} \swl{}{Do you have any milk?}{pra}
\sentenceItem{} \swl{}{Yes, we have almond milk.}{pra}
\sentenceItem{} \swl{}{No, we have almond milk.}{pra}
\end{sentenceList}

A request for milk in a vegan restaurant could plausibly be
interpreted as a request for a vegan milk-substitute.
So the concept \i{milk} in that context may actually be
interpreted as the concept \i{vegan milk}.  
Responding to the force of speech-acts
compels me to treat them as not \i{wholly}
illocutionary {\mdash} they are in part statements of
belief (like ordinary assertions).  One reason I need
to adopt an epistemic (and not just obligatory)
attitude to illocutionary acts is that I need to
clarify what meanings the speaker intends, which
depends on what roles she is assigning to
constituent concepts.}

\p{Suppose my friend says this, before and after:

\begin{sentenceList}

\sentenceItem{} \swl{itm:put}{Can you put some almond milk in my coffee?}{pra}
\sentenceItem{} \swl{itm:after}{Is there milk in this coffee?}{pra}
\end{sentenceList}

Between (\ref{itm:put}) and (\ref{itm:after}) I do put almond milk
in his coffee and affirm \q{yes} to (\ref{itm:after}).  I feel it
proper to read (\ref{itm:after})'s \q{milk} as really meaning
\q{almond milk}, in light of (\ref{itm:put}).  Actually
I should be \i{less} inclined to say \q{yes}
if (maybe as a prank) someone had instead
put real (cow) milk in the coffee.  In responding
to his question I mentally substitute what
he almost certainly \i{meant} for how
(taken out of context) (\ref{itm:after}) would usually
be interpreted.  In this current
dialog, the \i{milk} concept not only
includes vegan milks, apparently, but
\i{excludes} actual milk.}

\p{It seems {\mdash} on the evidence of cases like this one {\mdash}
as if when we are dealing with
illocutionary force we are obliged to subject
what we hear to extra interpretation, rather
than resting only within \q{literal} meanings
of sentences, conventionally understood.
This point is worth emphasizing because it complicates
our attempts to link illocution with propositional
content.  Suppose grandma asks us to close the
kitchen window.  Each of these are plausible and
basically polite responses:

\begin{sentenceList}

\sentenceItem{} \swl{}{It's not open, but there's still some
cold air coming through the cracks.}{pra}
\sentenceItem{} \swl{}{It's not open, but I closed the window in
the bedroom.}{pra}
\sentenceItem{} \swl{}{I can't {\mdash} it's stuck.}{pra}
\end{sentenceList}

In each case I have not fulfilled her request \visavis{}
its literal meaning, but I \i{have} acted benevolently
in terms of conversational maxims.  Similarly, 
the \i{Handbook} has this case (example 12, p. 203, chapter 8):

\begin{sentenceList}

\sentenceItem{} \swl{itm:TheWindow}{The window, it's still open.}{pra}
\sentenceItem{} \swl{itm:AWindow}{A window, it's still open.}{pra}
\end{sentenceList}

Chapter authors Jeanette K. Gundel and Thorstein Fretheim 
suggest that (\ref{itm:AWindow}) is dubious, but in context 
it may make perfect sense {\mdash} particularly if it follows 
a discourse where a requester wanted \i{whatever} window 
closed (whichever window was causing a draft); even if that 
wish was \i{expressed} via a \i{close the window}.\footnote{Consider also the case of (\i{Handbook}, example 9, 
page 132, chapter 6):
\begin{sentenceList}

\sentenceItem{} \swl{itm:scribbled}{He scribbled on a living-room wall.}{sem}
\end{sentenceList}

Barbara Abbott (chapter author) finds the indefinite article 
in (\ref{itm:scribbled}) awkward.  Here, like in (\ref{itm:AWindow}), 
I think the acceptability of \i{both} definite and indefinite 
articles points to the flexibility of articles in English: 
we can use \i{close \underline{the} window} even if we are referentially 
ambiguous about which window is open {\mdash} both (\ref{itm:AWindow}) 
and (\ref{itm:TheWindow}) are almost interchangeable.  The reason
is apparently that the very act of talking about an open or 
closed window foregrounds it such that it can be approached 
definitely.  In (\ref{itm:scribbled}) Abbott's intuition is 
probably that we usually speak of \i{the} wall of a room {\mdash} even 
in the normal case of a room with four walls {\mdash} because there is 
no common reason to single out one wall, as with an indefinite 
article (saying \i{a} wall implies that the other walls are 
differentiated from the one referenced).  But I believe we can 
clearly cognize the walls of a room as a collection of discrete 
things, and that the formation \i{the} wall {\mdash} unifying 
them into a single {\mdash} is not so much a matter of discounting 
the perceptual multiplicity of four walls, but 
of conceiving \q{walls} themselves functionally as well as 
perceptually.  In my reading, \i{the wall} refers not only 
to something perceptually individuated but to an element 
of the architectural complex of a building: a wall is something 
that prohibits movement, muffles sounds, provides some privacy, 
protects people inside the room (part of a building's 
structural integrity), and so forth.  We can say \i{the} wall 
because we cognize walls as occupying that phenomenal niche, 
so all four walls of a room are collectively \q{the} wall
\visavis{} such a niche, even while on sensory grounds we 
can switch to treating them as separate (warranting the (\ref{itm:scribbled}) 
version).}
}

\p{Part of reading propositional content is
syncing our conceptual schemas with our fellow
conversants.  The illocutionary
dimension of a request like \i{can I have some milk?}
makes this interpretation especially important,
because the addressee wants to make a good-faith effort
to cooperate with the pragmatic intent of the
spech-act.  But cooperation requires the
cooperating parties' conceptual schemas to
be properly aligned.  I therefore have to
suspend the illocutionary force of a directive
temporarily and treat it as locutionary
statement of belief, interpret its apparent
conceptual underpinnings in that mode, and
then add the illocutionary force back in: if I
brought \i{real} milk to a vegan customer who
asked for \q{milk} I would be \i{un}-cooperative.}

\p{The upshot is that conversational implicatures
help us contextualize the conceptual negotiations
that guarantee our grasping the correct
propositional contents, and vice-versa.  This means
that propositionality is woven throughout both
assertive and all other modes of language, but it
also means that propositional content can be
indecipherable without a detailed picture
of the current context (including illocutionary
content).  The propositional content of,
say, \i{there is milk in this coffee} has to be
judged sensitive to contexts like \i{milk}
meaning \i{vegan milk} {\mdash} and this
propagates from a direct propositional
to any propositional attitudes which may
be directed towards it, including requests like
\i{please put milk in this coffee}.}

\p{Suppose the grandkids close grandma's bedroom window
when she asks them to close the kitchen window.
The propositional content at the core of grandma's
request is that the kitchen window be closed; the
content attached to it is an unstated belief that
this window is open.  Thus, the truth-conditions
satisfying her implicit understanding would be
that the kitchen window went from being open to being
closed.  Suppose, as it happens, that window is already
closed.  So the truth-conditions that would satisfy
grandma's initial belief-state do not obtain {\mdash} her
beliefs are false {\mdash} but the truth conditions satisfying
her desired result \i{do} obtain.  The window
\i{is} closed.  Yet the grandkids should not thereby
assume that her request has been properly responded to;
it is more polite to guess at the motivation behind
the request, e.g., that she felt a draft
of cold air.  In short, they should look outside
the truth conditions of her original
request taken literally, and \i{interpret}
her request, finding different content
with different truth-conditions that are both consistent
with fact and address whatever pragmatic goals
grandma had when making her request.  They might
infer her goal is to prevent an uncomfortable
draft, and so a reasonable \q{substitute content} is
the proposition that \i{some} window is open,
and they should close \i{that} one.}

\p{So the grandkids should reason as if translating
between these two implied meanings:

\begin{sentenceList}

\sentenceItem{} \swl{}{I believe the kitchen window
is open {\mdash} please close it!}{pra}
\sentenceItem{} \swl{}{I believe some window
is open {\mdash} please close it!}{pra}
\end{sentenceList}

They have to revise the simplest reading of
the implicit propositional content of grandma's
\i{actual} request, because the actual request is
inconsistent with pertinent facts.  In short, they
feel obliged to explore propositional alternatives
so as to find an replacement, implicit request whose
propositional content \i{is} consistent with
fact and also meets the original request's illocutionary
force cooperatively.}

\p{In essence, we need to express a requester's desire as
itself, in its totality, a specific propositional content,
thinking to ourselves (or even saying to others) things
like

\begin{sentenceList}

\sentenceItem{} \swl{}{Grandma wants us to close the window.}{pra}
\sentenceItem{} \swl{}{He wants a bottle opener.}{pra}
\end{sentenceList}

But to respond politely we need to modify
the parse of their requests to capture the
\q{essential} content:

\begin{sentenceList}

\sentenceItem{} \swl{}{Grandma wants us to eliminate the cold draft.}{pra}
\sentenceItem{} \swl{}{He wants something to open that bottle.}{pra}
\end{sentenceList}

We have to read outside the literal interpretation
of what they are saying.  This re-reading is something
that may be appropriate to do with respect to
other forms of speech also; 
but our conversational responsibility to infer
some unstated content is especially pronounced
when we are responding to an explicit
request for something.}

\p{Certainly, in many cases, meanings are not literal.
But how then do we understand what people are saying?
Trying to formulate a not-entirely-haphazard
account of this process, we can speculate
that interpreting what someone is \q{really} saying
involves systematically mapping their apparent
concepts and references to some superimposed
inventory designed to mitigate false beliefs or
conceptual misalignments among language users in some
context.  That means, we are looking for mappings
like \i{milk} to \i{almond milk} in (\ref{itm:can}) from a
vegan restaurant, or \i{kitchen window} to
\i{bedroom window} in (\ref{itm:close}) if it is the latter
that is open:

\begin{sentenceList}

\sentenceItem{} \swl{itm:can}{Can I have some milk?}{pra}
\sentenceItem{} \swl{itm:close}{Can you close the kitchen window?}{pra}
\end{sentenceList}

The point of these \q{mappings} is that they
preserve the possibility of
modeling the \i{original} propositional content
by identifying truth conditions
for that content to be satisfied.}

\p{A \i{literal} truth-condition model doesn't work in
cases like (\ref{itm:can}) and (\ref{itm:close}): the diner's request
is \i{not} satisfied if it is the case
that there is now (real) milk in her coffee; and
grandma's request is not necessarily satisfied if it is
the case that the kitchen window is closed.  The
proposition \q{the kitchen window is closed} only bears on
grandma's utterance insofar as she believes that
this window is open and causing a draft.  So if we want
to interpret the underlying locutionary content
of (\ref{itm:can}) and (\ref{itm:close}) truth-theoreticaly, we need to
map the literal concepts appearing
in these sentences to an appropriate translation,
a kind of \q{coordinate transformation} that
can map concepts onto others, like milk/almond milk
and kitchen window/bedroom window.}

\p{In sum, a theory of sentences' logical nexus can only 
be complete with some model of discursive context
\i{structured in such a way} that we can repesent the 
interpretations and concept-transforms internal to 
parsing sentences to their propositional core.  
I will now consider what such a \q{theory of context}
might look like.}

\spsubsectiontwoline{The co-framing system and the
doxa system}

\p{Illucutionary acts expressly signify our desire for 
something to change in our environment (with the 
help of our addressees), but similar implications 
of pragmatic desire are evident even when sentences 
are more directly assertorial, or less directly 
illocutionary.  Compare between:

\begin{sentenceList}

\sentenceItem{} \swl{itm:store}{Remember that wine we tasted on the Niagara 
Peninsula last summer?  Can you find it in our 
local liquor store?}{pra}
\sentenceItem{} \swl{itm:varietal}{Remember that wine we tasted on the Niagara 
Peninsula last summer?  What varietal was that again?}{pra}
\end{sentenceList}

The first sentence in each pair attempts to 
establish a common frame of reference between 
addresser and addressee {\mdash} it does not, in and 
of itself, request any practical (extramental) action.  
The second sentence in (\ref{itm:store}) \i{can} be read as 
requesting that the addressee buy a bottle, though an alternate
interpretation is to learn for \i{future reference}
whether someone \i{could} buy that bottle.  The 
second sentence in (\ref{itm:varietal}) carries no directive 
implicature at all, at least with any directness; 
it asks for more information.}

\p{Despite these variations, it seems reasonable to say that 
language is always performed in an overarching setting 
where concrete (extralinguistic) activity 
will \i{eventually} take place.  If in (\ref{itm:varietal}) I intend
to recommend that grape variety to a friend, I may not be  
making a direct request of him, but I \i{am}
proposing an eventual action that he 
might take.  If in (\ref{itm:store}) I am not issuing a directive, I 
am however establishing (and reserving the future possibility) 
that such a directive would be reasonable.  As a result, some 
extralinguistic state change seems to be lurking 
behind the linguistic content: I want my friend 
to go from having never tasted that varietal 
to having tasted it. 
Or I want to go from not having a bottle of that 
wine to having one.  Or, if I do not 
want these things at the moment, I want 
to confirm intellectually that these wishes are 
plausible.  We seem to use language to 
set up the interpersonal understandings needed 
to \i{eventually} engage in (usually collective) 
practical activity, which means effectuating some 
(extralinguistic) change.}

\p{That is, most expressions are not direct 
requests or suggestions of the \q{close the window}
or \q{let's get some wine} variety, but they are 
stitches in the thread of coordinated human actions.
Often however we use language to \i{prepare},
\i{negotiate}, and \i{decide upon} joint actions.  
We may have a
\i{holistic} sense that meanings orbit around 
extralinguistic and extramental state-change, 
but at the level of particular sentences most 
changes that occur, or are proposed, tend to be 
changes in our conceptualization of situations.  
Accordingly, we can pursue a semantic theory 
based on \i{change of state} if we accept that 
such changes run the gambit from changes
\i{internal} to language {\mdash} to conversants' 
appraisal of dialogic context {\mdash} to 
changes effectuated by human activity inspired by language. 
Dialogs 
themselves change: the first sentences in 
(\ref{itm:store}) and (\ref{itm:varietal}) 
modify the discursive frame so that, for example, 
a particular wine becomes available as the anaphoric
target for \i{that} and \i{that wine} {\mdash} and also, 
metonymically, \i{that varietal}, \i{that grape},
\i{that winery}.  Conceptual frames can change: 
if we are discussing a visit to Ontario and 
I mention a particular winery, one effect is to 
(insofar as the conversation follows my lead) 
refigure our joint framing to something 
narrower and more granular that the prior frame (but 
still contained in it; I am not changing the subject 
entirely).  We can pull a frame out as well as in
{\mdash} e.g., switch from talking about one winery visit to 
the whole trip, or one Leafs game to the entire season.
Moreover, our beliefs can change/evolve: if you tell me 
the wine was Cabernet Franc, I have that piece of 
info in my arsenal that I did not have before.}

\p{So I assume in this paper that
linguistic meanings are grounded in state-changes, 
with the stipulation that the \q{register} where the changes occur can 
vary over several cognitive and extramental options: 
actual change in our environment (the window closed, 
milk in the coffee, the bottle opened); changes to the 
dialog structure (for anaphoric references, pronoun 
resolution, metalinguistic cues like \i{can you say that 
again}, etc.); changes to conceptual framings
(zoom in, zoom out, add detail); changes to beliefs.
Each of these kinds of changes deserve their own analysis, 
but we can imagine the totality of such analyses 
forming an umbrella theory of meanings.}

\p{During the course of a conversation {\mdash} and indeed 
of any structured cognitive activity {\mdash} we 
maintain conceptual frames representing relevant 
information; what other people know or believe;
what are our goals and plans (individually and 
collectively); and so forth.  We update
these frames periodically, and use language to 
compel others to modify their frames in
ways that we can (to some approximation) anticipate 
and encode in linguistic structure.}

\p{In the simplest case, we can effectuate changes in 
others' frames by making assertions they are likely 
to believe to be true (assuming they deem us 
reliable).  In general, it is impossible 
to extricate the explicit content of the relevant 
speech-acts from the relevant cognitive, linguistic, and 
real-world situational contexts:

\begin{sentenceList}

\sentenceItem{} \swl{}{That wine was a Cabernet Franc.}{ref}
\sentenceItem{} \swl{}{Those dogs are my neighbor's.  
They are very sweet.}{ref}
\end{sentenceList}

Although there is a determinate propositional content being 
asserted and although there is no propositional attitude 
other than bald assertion to complicate the pragmatics, 
still the actual words depend on addressees drawing from 
the dialogic context in accord with how I expect them to
(as manifest in open-ended expressions like \i{that wine},
\i{those dogs}, \i{they}).  Moreover, the  
open-ended components can refer outward in different
\q{registers}: in \i{that wine} I may be 
referencing a concept previously established in 
the conversation, while \i{those dogs} may refer to 
pets we saw or heard but had not previously 
talked about.  Of course, the scenarios 
could be reversed: I could introduce \i{that wine}
into the conversation by gesturing to a bottle 
you had not noticed before, and refer via
\i{those dogs} to animals you have never seen or heard but 
had talked about, or heard talk about, in the recent past.
These dialog steps need to be resolved via a mixture of 
linguistic and extra-linguistic cues: 
surface-level language is not always clear as to whether 
referring expressions are to work \q{deictically}
(drawing content from the ambient context, 
signified by gestures, rather than from any 
linguistic meaning proper), \q{discursively}
(referring within chains of dialog, e.g. 
anaphora), or \q{descriptively} (using 
purely semantic means to establish a designation, 
like \q{my next-door neighbor's dogs} or
\q{Inniskillin Cabernet Franc Icewine 2015}).}

\p{Let's agree to call the set of entities sufficiently 
relevant to a discourse or conversation context the \i{ledger}.
By \q{sufficiently relevant} I mean whatever is already 
established in a discourse so it can be referenced with something 
less that full definite description (and without the aid 
of extralinguistic gestures).  I assume that gestures
and/or descriptions are communicative acts which \q{add}
to the ledger.  The purely linguistic case {\mdash} let's say,
\i{descriptive additions} {\mdash} can themselves be distinguished
by their level of grounding in the current context.  
A description can be \q{definitive} in a specific situation without 
being a \i{definite description} in Bertrand Russell's sense
(see \q{that wine we tasted last summer}).}

\p{So, descriptive additions to the ledger are one kind of 
semantic side-effect: we can change the ledger via 
language acts.  I will similarly dub another facet of 
cognitive-linguistic frames as a \i{lens}: the idea
that in conversation we can \q{zoom} attention 
in and out and move it around in time. \q{That wine we 
tasted last summer in Ontario} both modifies 
the \i{ledger} (adding a new referent for convenient
designation) and might alter the \i{lens}:
potentially compelling subsequent 
conversation to focus on that time and/or place.  
Finally, I will identify a class of frame-modifications 
which do directly involve propositional content: 
the capacity for language to promote shared beliefs 
between people whose cognitive frames are in 
the proper resonance, by adding details to conceptual
pictures already established: \i{those dogs are Staffordshires},
\i{that wine is Cabernet Franc}, \i{we have almond milk}, etc.}

\p{For sake of discussion, I will call this latter part of the \q{active}
cognitive frame, for some discussion 
{\mdash} the part concerning shared beliefs or asserted facts {\mdash}
the \i{doxa inventory}.
This \q{database}-like repository stands alongside the
\q{ledger} and \q{lens} to track propositional content 
asserted, collectively established, or already 
considered as background knowledge, \visavis{} some 
discourse.  Manipulations of the lens and ledger allow 
speakers to designate (using referential cues
that could be ambiguous out-of-context) 
propositional contents which they 
wish to add to the \q{doxa inventory}.  I'll also say 
that modifying this inventory \i{can} be done through 
language, but participants in a discourse are entitled 
to assume that everyone formulates certain beliefs 
which are observationally obvious, and can therefore 
be linguistically presupposed rather than reported 
(the likes of that a traffic light 
is red, or a train has pulled into a station, or 
that it's raining).}

\p{So, I will assume that the machinery of frames is cognitive, 
not just linguistic.  We have analogous faculties for
\q{refocusing} attention and adding conceptual details
via interaction with our environment, both alone and with others, 
and both via language and via other means.  Some 
aspects of \i{linguistic} cognitive framing {\mdash} like 
the \q{ledger} of referents previously established in a 
conversation {\mdash} may be of a purely linguistic character, 
but these are the exception rather than the rule.  
In the typical case we have a latent ability to 
direct attention and form beliefs by 
direct observation \i{or} by accepting others' reports as
proxies for direct observation.}

\p{When we are told that two dogs are male, for instance, 
we may not perceptually encounter the dogs but we understand what 
sorts of perceptual disclosures could
serve as motivation for someone believing that idea.  We therefore 
assume that such belief was initially warranted by 
observation and subsequently got passed through a chain 
of language-acts whose warrants are rooted in the perceived 
credibility of the speaker.  Internal to this 
process is our prior knowledge of the parameters for judging 
statements like \i{this dog is male} observationally.}

\p{True, sometimes such observational warrants are less on 
display.  If I had never heard of Staffies (Staffordshire
pit bulls), I would be fuzzier about observational warrants
and could end up in a conversation like:

\begin{sentenceList}

\sentenceItem{} \swl{}{Those dogs are Staffordshires.}{pra}
\sentenceItem{} \swl{}{What's a Staffordshire?}{pra}
\sentenceItem{} \swl{}{It's a breed of dog.}{pra}
\end{sentenceList}

\noindent{}Here I still don't really have a picture of what it is 
like to tell observationally that a dog is a Staffordshire.  
There may not be any visual cues {\mdash} at least none I 
know of {\mdash} which announce to the world that some dog is
a Staffy (compared to those announcing that it is male,
say).  But insofar as I am acquainted 
with the concept \i{dog breed}, I also understand 
the general pattern of these observations.  For instance 
I may know breeds like poodles or huskies and be able to 
identify \i{these} by distinctive visual cues.  I also 
understand that dogs' parentage is often documented, allowing 
informed parties to know their breeds via those of their 
forebearers.  That is, I am familiar with 
how beliefs about breeds are formed based on 
observation rather than just accepting others' 
reports, so I know the extralinguistic epistemology 
anchoring chains of linguistic reports in this area 
to originating observations {\mdash} even if 
I cannot in this case initiate such a chain myself.}

\p{My overall point is that language enables us to formulate beliefs 
based on the beliefs of others, but this is possible because 
we also realize what it is like to formulate \i{our own}
beliefs, and envision that sort of practice at the 
origin of reports that later get circulated via language.  
If we can't sufficiently picture the originating 
observations, we don't feel like we are grasping 
the linguistic simulacrum of those reports with enough 
substance.  If I never learn what Stafforshire is, 
an assertion that some dogs are Staffordshires 
has no real meaning for me {\mdash} even if I trust the asserter 
and do indeed thereby believe that the dogs are Staffordshires.  
Notice that merely knowing Staffordshire is a breed of 
dog does not expand my conceptual repertoire very much
{\mdash} it does not tell me how to recognize a Staffordshire 
or what I can do with the knowledge 
that a dog is one (it cannot, for instance, help 
me anticipate his behavior).  Nevertheless even (only) knowing 
that Staffordshire is a breed of 
dog seems to fundamentally change the status 
of sentences like \i{those dogs are Staffies}
for me: I do not \i{have} the conceptual machinery 
to exploit that knowledge, but I understand
what \i{sort} of machinery is involved.}

\p{In short, the \i{linguistic} meaning of concepts is tightly bound 
to how concepts factor in perceptual observations anterior 
to linguistic articulation.  As a result, 
during any episode wherein conversants use language to 
compel others' beliefs, an intrinsic dimension of the 
unfolding conversation is that people will form 
their own (extralinguistic) beliefs {\mdash} and can also 
imagine themselves in the role of originating the 
reports they hear via language, whether or not they 
can actually test out the reports by their own 
observations.}

\p{This extralinguistic epistemic 
capacity is clearly exploited by the form of language itself.  
If a tasting organizer hands me a glass and says
\q{This is Syrah}, she clearly expects me to infer that I 
should take the glass from her and taste the wine (and 
know that the glass contains wine, etc.).  These conventions
may be \i{mediated} by language {\mdash} we are more likely to 
understand \q{unspoken} norms by asking questions, until we gain 
enough literacy in the relevant practical domain to 
understand unspoken cues and assumptions.  But many situational 
assumptions are extralinguistic because 
they are (by convention) not explicitly stated, 
even if they accompany content that \i{is} explicitly stated.
\i{This is Syrah} accompanied by the gesture of handing
me a glass is an indirect invitation for me to drink it 
(compare to \i{Please hold this for a second?} or
\i{Please hand this to the man behind you?}).}

\p{I bring to every linguistic situation a 
capacity to make extralinguistc observations, and 
to understand every utterance in the context of hypothetical 
extralinguistc observations from which is originates.  
My conversation peers can use language to trigger 
these extralinguistic observations.  Sometimes the
\q{gap} {\mdash} the conceptual slot which 
extralinguistic reasoning is expected to fill 
{\mdash} is directly expressed, as in \i{See the
dog over there?}.  But elsewhere the
\q{extralinguistic implicature} is more indirect, 
as in \i{This is Syrah} and my expected belief that 
I should take and taste from the glass.  But in any 
case the phenomenon of triggering these 
extralinguistic observation is 
one form of linguistic \q{side effect}, initiating a 
change in my overall conceptualization of a situation by 
compelling me to augment beliefs with new observations.}

\p{All told, then, the language which is presented to me has the effect 
of initiating changes in what I believe 
{\mdash} partly via signifying propositional
content that I could take on faith, but partly 
also via directing my attention and my interpretive 
dispositions to guide me towards extralinguistic 
observations.  Here I will argue that side-effects like 
these are not side-effects \i{of} linguistic meaning, 
but are in some sense \i{constitutive} of 
meaning.}

\spsubsectiontwoline{Side-Effects and Logical Incompleteness}
\p{In the most common analysis, I would argue, side-effects and propositional 
content overlap.  The effect of an utterance, all else being equal, is 
that addressees understand the claimed report as believed by the speaker and 
{\mdash} depending on their deeming the speaker credible {\mdash} accept it as 
a provisional doxic given themselves.  There are many other effects, 
such as how one speaker's assertion changes dialogic context in a way 
that alters the current topical focus and the interpretation of 
context-specific expressions.  But these generally build off of 
a statements' coherence and credibility {\mdash} a non-sensical 
comment is less likely to veer a discourse in new directions 
or reinscribe the dialogic \q{ledger}.  For these reasons, we 
may be tempted to focus analysis on the architecture of 
propositional content rather than the itemization of side-effects.  
I would argue, however, that linguistic side-effects are more 
universal than logically pristine propositional signification, and 
that communicating doxic content is a special case of side effects 
more than side-effects are the passive consequence of rational 
conversation.}

\p{I will defend this perspective via cases where 
language users seem to traffic in 
a relative \i{absence} of semantic determinism, with no 
detrimental effects to the \i{telos} of language in context.  
These cases buttress an idea that language is not targeted at 
doxic specificity as a precondition for meaning in general, 
but rather packages doxa along with other contextualizing 
constituents in the service of pragmatic ends.  
Consider:

\begin{sentenceList}

\sentenceItem{} \swl{}{My colleague Ms. O'Shea would like to interview
Mr. Jones, who's an old friend of mine.  Can he take this call?}{pra}
\sentenceItem{} \swl{itm:Jones}{I'm sorry, this is his secretary.  Mr. Jones
is not available at the moment.}{pra}
\end{sentenceList}

It sounds like Ms. O'Shea is trying to use personal
connections to score an interview with Mr. Jones.  Hence
her colleague initiates a process intended to
culminate in Ms. O'Shea getting on the telephone
with Mr. Jones.  But his secretary demurs with a
familiar phrase, deliberately formulated to
foment ambiguity: (\ref{itm:Jones}) could mean that Mr. Jones
is not in the office, or that he is in a meeting, or he is
unwilling to talk, or even missing (like
the ex-governor consummating an affair in Argentina
while his aides thought he was hiking in Virginia).
Or:

\begin{sentenceList}

\sentenceItem{} \swl{}{Mr. Jones, were you present at a meeting where
the governor promised your employer
a contract in exchange for campaign contributions?}{pra}
\sentenceItem{} \swl{}{After consulting with my lawyers, I decline
to answer that question on the grounds that it
may incriminate me.}{pra}
\end{sentenceList}

Here Mr. Jones neither confirms nor denies his
presence at a corrupt meeting.}

\p{As these examples intimate, the processes language
initiates do not always result in a meaningful
logical structure.  But this is not necessarily
a complete breakdown of language:

\begin{sentenceList}

\sentenceItem{} \swl{itm:isJones}{Is Jones there?}{pra}
\sentenceItem{} \swl{itm:not}{He is not available.}{pra}
\end{sentenceList}

The speaker of (\ref{itm:not}) does not
provide any prima facie logical content: it neither affirms
nor denies Jones's presence.  Nonetheless that speaker
is a cooperative conversational partner
(even if they are not being very cooperative in real life):
(\ref{itm:not}) responds to the implicature in
(\ref{itm:isJones}) that what the
first speaker really wants is
to interview Jones.
So the second speaker conducts what I called
a \q{transform} and maps \i{Jones is here} to
\i{Jones is willing to be interviewed}.
Responding to this \q{transformed} question allows
(\ref{itm:not}) to be (at least) linguistically cooperative
while nonetheless avoiding a response at the
\i{logical} level to (\ref{itm:isJones}).  (\ref{itm:not}) obeys
conversational maxims but is still rather obtuse.}

\p{So one problem for theories that read meanings in terms
of logically structured content {\mdash} something like, the
meaning of an (assertorial) sentence is what the world would be
like if the sentence were true {\mdash} is that the actual
logical content supplied by some constructions
(like \i{Jones is not available}) can be pretty
minimal {\mdash} but these are still valid and
conversationally cooperative segments of discourse.  To be sure, this
content does not appear to be \i{completely}
empty: \q{Jones is not available} means the
conjunction of several possibilities (he cannot be found
or does not want to talk or etc.).
So (\ref{itm:not}) does seem to evoke some
disjunctive predicate.  But such does not mean
that this disjunctive predicate is the \i{meaning} of
(\ref{itm:not}).  It does not seem as if (\ref{itm:not})
when uttered by a bodyguard is intended first and foremost to
convey the disjunctive predicate.  Instead, the
bodyguard is responding to the implicature
in the original \i{Is Jones there?} query {\mdash} the
speaker presumably does not merely want
to know Jones's location, but to see Jones.
Here people are acting out social roles, and just happen
to be using linguistic expressions to negotiate
what they are able and allowed to do.}

\p{Performing social roles {\mdash} including through language {\mdash}
often involves incomplete information: possibly
the secretary or bodyguard themselves do not know
where Jones is or why he's not available.
We could argue that there is \i{enough} information to
still ground \i{some} propositional content.  But this
is merely saying that we can extract some propositional content from
what speakers are supposed to say as social acts, which seems
to make the content (in these kinds of cases)
logically derivative on the enactive/performative
meaning of the speech-acts, whereas a truth-theoretic
paradigm would need the derivational dependence
to run the other way.  By saying \i{Jones is unavailable}
the speaker is informing us that our own prior speech
act (asking to see or talk to him) cannot have
our desired effect {\mdash} the process we initiated cannot be
completed, and we are being informed of that.  The
person saying \i{Jones is unavailable} is likewise
initiating a \i{new} process, one that counters our process
and, if we are polite and cooperative, will have its
own effect {\mdash} the effect being that we do not insist on
seeing Jones.  The goal of \q{Jones is unavailable} is to create
that effect, nudging our behavior in that direction.
Any \i{logic} here seems derivative on the practical initiatives.}

\p{And moreover this practicality is explicitly marked by how
the chosen verbiage is deliberately vague.  The declaration
\q{Jones is unavailable} does not \i{need} logical precision to
achieve its effect.  It needs \i{some} logical content, but it exploits a
kind of disconnect between logical and practical/enactive
structure, a disconnect which allows \q{Jones is unavailable} to
be at once logically ambiguous and practically clear {\mdash}
in the implication that we should not try to see Jones.
I think this example has some structural
analogs to the grandma's window case: \i{there} we
play at logical substitutions to respond practically
to grandma's request in spirit rather than \i{de dicto}.
\i{Here} a secretary or bodyguard can engage in logical
substitution to formulate a linguistic performance
designed to be conversationally decisive
while conveying as little information as possible.  The logical
substitution in grandma's context \i{added} logical content by
trying alternatives for the window being closed; here,
the context allows a \i{diminution} in
logical content.  We can strip away logical detail from
our speech without diminishing the potency of
that speech to achieve affects.  And while the remaining residue
of logical content suggests that some basic logicality is still
essential to meaning, the fact that logical content can
be freely subtracted without altering practical effects
suggests that logic's relation to meaning is something
other than fully determinate: effect is partially autonomous from
logic, so a theory of effect would seem to be
partially autonomous from a theory of logic.
I can be logically vague without being
conversationally vague.   This evidently means that conversational
clarity is not identical to logical clarity.}

\p{Let us agree that {\mdash} beneath surface-level
co-framing complexity {\mdash} many language acts have a
transparent content as \q{doxa} that gets conveyed between
people with sufficiently resonant
cognitive frames.  So \i{in the overall course of communication}
we have propositional contents that converge among discourse 
partners, suspended between the various cognitive and pragmatic 
units which contextualize a given, unfolding dialog.  
There is in short a \i{holistic} mapping between units of 
discourse and \q{units} of propositionality, or \q{doxa}.  
This general observation leaves unstated, however,
\i{how} language elements map to corresponding doxic 
particulars.  I have argued that focusing on the \i{logical 
structures} of propositions can lead us astray if we 
seek to find concordant formations on the language side.}

\p{Consider our attempts to close grandma's kitchen window.  
My analysis related to conceptual \q{transforms}
assumed that we can find, substituting for \i{literal}
propositional content, some \i{other}
(representation of a) proposition that fulfills a
speaker's unstated \q{real} meaning.  Sometimes
this makes sense: the proposition \q{that the
\i{bedroom} window is closed} can neatly,
if the facts warrant, play the role of the
proposition that \i{the kitchen window is closed}.
But we can run the example differently: there
may be \i{no} window open, but instead a draft
caused by non-airtight windows (grandma might ask
us to put towels by the cracks).  Maybe there is
no draft at all (if grandma is cold, we can
fetch her a sweater).  Instead of a single
transform, we need a a system of potential transforms
that can adapt to the facts as we discover them.
Pragmatically, the underlying problem is
that \i{grandma is cold}.  We can address this
{\mdash} if we want to faithfully respond to her request,
playing the role of cooperative conversation
partners (and grandkids) {\mdash} via a matrix
of logical possibilities:

\begin{sentenceList}

\sentenceItem{} \swl{}{If the kitchen window is closed,
we can see if other windows are open.}{pra}
\sentenceItem{} \swl{}{If no windows are open,
we can see if there is a draft through the window-cracks.}{pra}
\sentenceItem{} \swl{}{If there is no draft, we can
ask if she wants a sweater.}{pra}
\end{sentenceList}

This is still a logical process: starting from an
acknowledged proposition (grandma is cold) we
entertain various other propositional possibilities,
trying to rationally determine what pragmas we
should enact to alter that case
(viz., to instead make true the proposition that
\i{grandma is warm}).  Here we are not just testing
possibilities against fact, but strategically
acting to modify some facts in our environment.}

\p{But the kind of reasoning involved here is not logical
reasoning per se: abstract logic does not tell us
to check the bedroom window if the kitchen window is closed,
or to check for gaps and cracks if all windows
are closed.  This all solicits practical, domain-specific knowledge
(about windows, air, weather, and houses).
Yet we are still deploying our practical
knowledge in logical ways {\mdash} there is a logical
structure underpinning grandma's request and
our response to it.  In sum: we (the grandkids)
are equipped with some practical knowledge
about houses and a faculty to logically utilize this
knowledge to solve the stated problem, reading
beyond the \i{explicit} form of grandma's discourse.
We use a combination of logic and background
knowledge to reinterpret the discourse as needed.
By making a request, grandma is not expressing
one attitude to one proposition, so much as
\i{initiating a process}.  This is why it would
be impolite to simply do no more
if the kitchen window is closed: our conversational
responsibility is to enact a process trying to
redress grandma's discomfort, not to entertain the
truth of any one proposition.}

\p{For all that, there is still an overarching logical
structure here that language clearly marshals.  We read
past grandma's explicit request to infer what she is
\q{really saying} {\mdash} e.g., \i{that she is cold} {\mdash} but we
still regard her speech act in terms of its (now indirect)
propositional content.  However, notice 
how our ascertaining this content only one step toward 
legitimate understanding of the original speech-act 
(even accounting for its illocutionary dimensions).  
The doxa are \i{factors} in understanding but, 
given these cases, are not straightforward \i{designata}
of linguistic compounds.  This implies a critique of 
truth-theoretic paradigms from a semiotic and 
compositional perspective: language is not \i{composed}
to convey doxa through semantic reference and 
grammatic form internally (without the mediation of 
extralinguistic cognition); propositional content does 
not \i{fall out} of syntax and semantics.  I will expand 
on this critique in the next section.}

\p{To summarize my current arguments, then, I believe that most sentences 
have an accompanying propositional content, and that during conversations 
we interpret this content as a factor in sentence meanings, becoming aware of 
what our partners believe, desire, or inquire to be the case.  We retain this 
awareness in a cumulative model of conversational context {\mdash} a \q{doxa inventory}
{\mdash} alongside other referential and deictic axes establishing each dialogic 
setting.  Essentially, 
I grant that this doxic layer is central to 
linguistic performance in general {\mdash} but given this very centrality I 
will argue that the logical substratum of language cannot be \i{separated}
from the totality of syntactic, semantic, and pragmatic processing such 
that models based on formal logic could be curated in isolation from 
the overarching interconnectedness of language as a cognitive system.}

\subsection{The Illogic of Syntax}

\p{As I understand it, a non-trivial truth-theoretic semantics 
requires more than a holistic association between 
sentences and propositional content: it requires that this 
association be established \i{by linguistic means} and
\i{on linguistic grounds} (syntax, semantics, pragmatics).  
I will present several arguments against this possibility, in 
the general cases {\mdash} that is, against the possibility that 
for \i{typical} sentences we can analyze syntactic form through 
the lens of the logical structure of propositions signified 
via a sentence; or analyze natural-language semantics through a 
logically well-structured semantics of propositions.  
I will emphasize two issues: first, that the architecture of 
linguistic performances \i{does not}, in the general case,
\i{recapitulate propositional structure}; and, second, 
that language-acts work through gaps in logical specificity that 
complicate how we should theorize the triangular relation between 
surface language, propositional content, and side-effect meanings.}

\p{Since it is widely understood that the essence of language
is compositionality, the clearest path to
a truth-theoretic semantics would be via the
\q{syntax of semantics}: a theory of how
language designates propositional content by
emulating or iconifying propositional structure
in its own structure (i.e., in grammar).
This would be a theory of how linguistic
connectives reciprocate logical connectives,
phrase hierarchies reconstruct propositional
compounds, etc.  It would be the kind of theory motivated
by cases like

\begin{sentenceList}

\sentenceItem{} \swl{}{This wine is a young Syrah.}{log}
\sentenceItem{} \swl{}{My cousin adopted one of my neighbor's dog's puppies.}{log}
\end{sentenceList}

where morphosyntactic form {\mdash} possessives, adjective/noun
links {\mdash} seems to transparently recapitulate predicate
relations.  Thus the wine is young \i{and}
Syrah, and the puppy is the offspring of a dog who
is the pet of someone who is the neighbor of the speaker.  These
are well-established logical forms: predicate conjunction, here;
the chaining of predicate
operators to form new operators, there.  Such are embedded in
language lexically as well as grammatically: the conjunction
of husband and \q{former, of a prior time} yields ex-husband;
a parent's sibling's daughter is a cousin.}

\p{The interesting question is to what extent \q{morphosyntax
recapitulates predicate structure} holds in general
cases.  This can be considered by examining the logical
structure of reported assertions and then the structures
via which they are expressed in language.  I'll
carry out this exercise \visavis{} several sentences,
such as these (supplementing my earlier, more preliminary discussion of 
(\ref{itm:ants})-(\ref{itm:princess})):

\begin{sentenceList}

\sentenceItem{} \swl{itm:maj}{The majority of students polled were
opposed to tuition increases.}{log}
\sentenceItem{} \swl{itm:most}{Most of the students expressed disappointment
about tuition increases.}{log}
\sentenceItem{} \swl{itm:many}{Many students have protested the tuition increases.}{log}
\end{sentenceList}
}

\p{There are several logically significant elements here that
seem correspondingly expressed in linguistic
elements {\mdash} that is, to have some model
in both prelinguistic predicate structure
and in, in consort, semantic or syntactic principles.
All three of (\ref{itm:maj})-(\ref{itm:many}) have similar but not
identical meanings, and the differences are
manifest both propositionally and
linguistically (aside from the specific superficial
fact that they are not the same sentence).
I will review the propositional differences first,
then the linguistic ones.}

\p{One obvious predicative contrast is that
(\ref{itm:maj}) and (\ref{itm:most})
ascribes a certain \i{quality} to students (e.g., disappointment),
whereas (\ref{itm:most}) and (\ref{itm:many})
indicate \i{events}.  As such
the different forms capture the contrast between \q{bearing
quality $Q$} and \q{doing or having done action
$A$}: the former a predication and the latter an event-report.
In the case of (\ref{itm:most}), both forms are available because we can
infer from \i{expressing} disappointment
to \i{having} disappointment.
There may be logics that would map one
to the other, but let's assume we can
analyze language with a logic expressive enough
to distinguish events from quality-instantiations.}

\p{Other logical forms evident here involve how the
subject noun-phrases are constructed.
\q{A majority} and \q{many} imply a multiplicity
which is within some second multiplicity, and
numerically significant there.  The sentences differ
in terms of how the multiplicities are circumscribed.
In the case of \i{students polled}, an extra determinant is
provided, to construct the set of students forming the
predicate base: we are not talking about students in
general or (necessarily) students at one school,
but specifically students who participated in a poll.}

\p{Interrelated with these effects are how the
\i{tuition increases} are figured.  Using
the explicit definite article suggests that there is
\i{some specific} tuition hike policy
raising students' ire.  This would also favor a
reading where \q{students} refers collectively to those
at a particular school, who would be directly affected by the
hikes.  The \i{absence} of an article on
\q{tuition increases} in (\ref{itm:most}) leaves open an interpretation
that the students are not opining on some specific policy, but on
the idea of hikes in general.}

\p{Such full details are not explicitly laid out in the sentences,
but it is entirely possible that they are clear in context.
Let's take as given that, in at least some cases where they would
occur, the sentences have a basically pristine
logical structure given the proper contextual framing {\mdash}
context-dependency, in and of itself, does not weaken
our sense of language's logicality.  In particular,
the kind of structures constituting the sentences'
precise content {\mdash} the details that seem context-dependent
{\mdash} have bona fide logical interpretations.  For
example, we can consider whether students are
responding to \i{specific} tuition hikes
or to hikes in general.  We can consider
whether the objectionable hikes have already
happened or instead are proposed for the future.  Context
presumably identifies whether \q{students} are
drawn from one school, one governmental
jurisdiction, or some other aggregating criteria
(like, all those who took a poll).
Context can also determine whether aggregation is
more set- or type-based, more extensional
or intensional.  In (\ref{itm:maj})-(\ref{itm:many}) the implication
is that we should read \q{students} more as a set or
collection, but variants like \i{students
hate tuition hikes} operates more at the level
of students as a \i{type}.  In \q{students polled}
there is a familiar pattern of referencing a set by
marrying a type (students in general) with a descriptive
designation (e.g., those taking a specific
poll).  The wording of (\ref{itm:maj}) does not
mandate that \i{only} students took the poll; it
does however employ a type as a kind of operator on a set:
of those who took the poll, focus on students
in particular.}

\p{These are all essentially logical structures and can
be used to model the propositional
content carried by the sentences {\mdash} their \q{doxa}.
We have operators and distinctions like past/future,
set/type, single/multiple, subset/superset, and
abstract/concrete comparisons like tuition hikes \i{qua} idea
vs. \i{fait accompli}.  A logical system could
certainly model these distinctions and accordingly capture
the semantic differences between (\ref{itm:maj})-(\ref{itm:many}).
So such details are all still consistent
with a truth-theoretic paradigm, although
we have to consider how linguistic form actually
conveys the propositional forms carved out via
these distinctions.}

\p{Ok, then, to the linguistic side.  My first observation is that some
logically salient structures have fairly clear analogs
in the linguistic structure.  For instance, the logical operator for
deriving a set from criteria of \q{student} merged with \q{taking a
poll} is brought forth by the verb-as-adjective
formulation \i{students polled}.  Subset/superset arrangements
are latent as lexical norms in senses like \i{many} and
\i{majority}.  Concrete/abstract and past/future distinctions
are alluded to by the presence or absence of a definite
article.  So \q{ \i{the} tuition increases} connotes that
the hikes have already occurred, or at least been
approved or proposed, in the past relative to the
\q{enunciatory present} (as well as that they are a
concrete policy, not just the idea),
whereas articleless \q{tuition increases} can be read as
referring to future hikes and the idea of hikes
in general: past and concrete tends to
contrast with future and abstract.}

\p{A wider range of
logical structures can be considered by subtly varying
the discourse, like:

\begin{sentenceList}

\sentenceItem{} \swl{}{Most students oppose the tuition increase.}{sem}
\sentenceItem{} \swl{itm:indef}{Most students oppose a tuition increase.}{sem}
\end{sentenceList}

These show the possibility of \i{increase} being singular
(which would tend to imply it refers to a concrete
policy, some \i{specific} increase), although in
(\ref{itm:indef}) the \i{in}definite article \i{may} connote
a discussion about hikes in general.}

\p{But maybe not; cases like these are perfectly plausible:

\begin{sentenceList}

\sentenceItem{} \swl{itm:today}{Today the state university system
announced plans to raise tuition by
at least 10%.  Most students oppose a tuition increase.}{sem}
\sentenceItem{} \swl{itm:colleges}{Colleges all over the country, facing
rising costs, have had to raise tuition, but
most students oppose a tuition increase.}{sem}
\end{sentenceList}

In (\ref{itm:today}) the definite article could also be used, but saying
\q{ \i{a} tuition increase} seems to reinforce the
idea that while plans were announced, the details
are not finalized.  And in (\ref{itm:colleges}) the plural \q{increases}
could be used, but the indefinite singular connotes the
status of tuition hikes as a general phenomenon
apart from individual examples {\mdash} even though the
sentence also makes reference to concrete examples.  In other
words, these morphosyntactic cues are like
levers that can fine-tune the logical designation
more to abstract or concrete, past or future, as
the situation warrants.  Again, context should
clarify the details.  But morphosyntactic forms
{\mdash} e.g., presence or absence of articles (definite
or indefinite), and singular/plural {\mdash} are
vehicles for language, through its own
forms and rules, to denote propositional-content structures
like abstract/concrete and past/future.}

\p{So these are my \q{concession} examples: cases where 
language structures \i{do}, in their compound architectonics, 
signifying propositional contents {\mdash} and moreover the 
lexical and morphosyntactic cues (like singular/plural or the 
choice of articles) drive this language-to-logic mapping in 
an apparently rule-bound and replicable fashion.  These are 
potential case-studies of how a truth theory of language, 
without neglecting contextual and semantic subtelties,
\i{could} work: capturing granular semantic constructs via sufficiently
nuanced logics, and theorizing word-senses and morphology through 
the aegis of a structural reduction between surface language and 
predicate structure.  My tactic for critiquing truth-theoretic 
paradigms is to argue that many sentences \i{fail} to 
display a mapping between lexico/morphosyntactic details and 
predicate structure \i{in this relatively mechanical fashion}.
By pointing out examples where morphosytax \i{does} rather seamlessly 
recapitulate propositional content (e.g. \i{the tuition hike} plural/definite), 
we can appreciate the more circuitous hermeneutics for examples I 
will present wherein the morphosytax-to-logic translation, while 
present, is not \i{sui generis}.}

\p{Varying the current examples yields cases 
{\mdash} in a sense, intermediate between 
logical transparency and oblique constructions I will 
discuss next {\mdash} 
where logical implications are more circuitous.  For instance,
describing students as \i{disappointed} (\ref{itm:most})
implies that the disliked hikes have already occurred,
whereas phraseology like \q{students are gearing for a fight}
would imply, conversely, that they are sill only planned or proposed.  
The mapping from
propositional-content structure to surface language here
is less mechanical than, for instance, merely
using the definite article on \i{the tuition increases}.
Arguably \q{disappointment} {\mdash} rather than just, say,
\q{opposition} {\mdash} implies a specific timeline
and concreteness, an effect analogous to the definite
article.  The semantic register
of \q{disappointment} bearing this implication is a
more speculative path of conceptual resonances, compared
to the brute morphosyntactic \q{the}.  There is subtle
conceptual calculation behind the scenes in the former
case.  Nonetheless, it does seem as if via this
subtlety linguistic resources are expressing
the constituent units of logical forms, like
past/future and abstract/concrete.}

\p{So, I am arguing (and conceding) that there are units of
logical structure that are conveyed by units of
linguistic structure, and this is partly how
language-expressions can indicate propositional
content.  The next question is to explore this
correspondence compositionally {\mdash} is there a
kind of aggregative, hierarchical order in terms
of how \q{logical modeling elements} fit together,
on one side, and linguistic elements fit
together, on the other?
There is evidence of compositional concordance
to a degree, examples of which I have cited.  In
\i{students polled}, the compositional structure of
the phrase mimics the logical construct {\mdash} deriving
a set (as a predicate base) from a type crossed with
some other predicate.  Another example is the
phraseology \i{a/the majority of}, which directly
nominates a subset/superset relation and so
reciprocates a logical quantification (together
with a summary of relative
size; the same logical structure, but with
different ordinal implications, is seen in cases
like \i{a minority of} or \i{only a few}).
Here there is a relatively mechanical translation
between propositional structuring elements
and linguistic structuring elements.}

\p{However, varying the examples {\mdash} for instance,
varying how the subject noun-phrases are conceptualized
{\mdash} points to how the synchrony between
propositional and linguistic composition can break down:

\begin{sentenceList}

\sentenceItem{} \swl{itm:sas}{Student after student came out against the tuition hikes.}{sem}
\sentenceItem{} \swl{itm:substantial}{A substantial number of students
have come out against the tuition hikes.}{sem}
\sentenceItem{} \swl{itm:mass}{The number of students protesting the tuition hikes
may soon reach a critical mass.}{sem}
\sentenceItem{} \swl{itm:tipping}{Protests against the tuition hikes
may have reached a tipping point.}{sem}
\end{sentenceList}

Each of these sentences says something about a large number
of students opposing the hikes.  But in each
case they bring new conceptual details to the fore, and
I will also argue that they do so in a way that
deviates from how propositional structures are composed.}

\p{First, consider \i{student after student} as a way of
designating \i{many students} (I analyzed the very similar 
case (\ref{after}) in terms of \q{space building}).  
There is a little more
rhetorical flourish here than in, say, \i{a majority
of students}, but this is not just a matter of
eloquence (as if the difference were stylistic,
not semantic). \q{Student after student} creates a
certain rhetorical effect, suggesting via how it
invokes its multiplicity a certain recurring or
unfolding phenomenon.  One imagines the speaker, time
and again, hearing or encountering an angry student.
To be sure, there are different kinds of contexts
that are consistent with (\ref{itm:sas}): the events could
unfold over the course of a single hearing
or an entire semester.  Context would foreclose
some interpretations {\mdash} but it would do so in any
case, even with simpler designations like \i{majority
of students}.  What we \i{can} say is that the speaker's
chosen phraseology cognitively highlight a
dimension in the events that carries a certain
subjective content, invoking their temporality and
repetition.  The phrasing carries an effect of
cognitive \q{zooming in}, each distinct event
figured as if temporally inside it; the sense
of being tangibly present in the midst of the event is
stronger here than in less temporalized language,
like \q{many students}.  And then at the same time
the temporalized event is situated in the context of many
such events, collectively suggesting a recurring presence.
The phraseology zooms in and back out again,
in the virtual \q{lens} our our cognitively figuring
the discourse presented to us {\mdash} all in just
three or four words.  Even if \q{student after student}
is said just for rhetorical effect {\mdash} which
is contextually possible {\mdash} \i{how} it
stages this effect still introduces a subjective
coloring to the report.}

\p{Another factor in (\ref{itm:sas}) and (\ref{itm:substantial})
is the various possible meanings of
\q{come out against}.  This could be
read as merely expressing a negative opinion, or as a
more public and visible posturing.  In fact, a similar
dual meaning holds also for \q{protesting}.  Context,
again, would dictate whether \i{protesting} means
actual activism or merely voicing displeasure.  Nonetheless,
the choice of words can shade how we frame situations.
To \i{come out against} connotes expressing disapproval
in a public, performative forum, inviting
the contrast of inside/outside (the famous example
being \i{come out of the closet} to mean publicly
identifying as \LGBTQ{}).  Students may not literally
be standing outside with a microphone, but {\mdash} even
if the actual situation is just students
complaining rather passively {\mdash} using \i{come out against}
paints the situation in an extra rhetorical hue.  The students
are expressing the \i{kind} of anger that can goad
someone to make their sentiments known theatrically and
confrontationally.  Similarly, using \q{protest} in lieu
of, say, \q{criticize} {\mdash} whether or not students are actually
marching on the quad {\mdash} impugns to the students a level
of anger commensurate with politicized confrontation.}

\p{All these sentences are of course \i{also} compatible with
literal rioting in the streets; but for sake of argument
let's imagine (\ref{itm:sas})-(\ref{itm:tipping}) spoken in contexts
where the protesting is more like a few comments to a school
newspaper and hallway small-talk.  The speakers have still
chosen to use words whose span of meanings
includes the more theatrical readings: \q{come out against}
and \q{protest} overlap with \q{complain about} or \q{oppose},
but they imply greater agency, greater intensity.  These lexical
choices establish subtle conceptual variations; for instance,
to \i{protest} connotes a greater shade of anger than to \i{oppose}.}

\p{Such conceptual shading is not itself unlogical; one
can use more facilely propositional terms to evoke
similar shading, like \q{very angry} or \q{extremely angry}.
However, consider \i{how} language like
(\ref{itm:sas})-(\ref{itm:tipping})
conveys the relevant facts of the mater: there is
an observational, in-the-midst-of-things staging at
work in these latter sentences that I find missing
in the earlier examples. \q{The majority of} sounds
statistical, or clinical; it suggests journalistic reportage,
the speaker making an atmospheric effort to sound like someone
reporting facts as established knowledge rather than
observing them close-at-hand.
By contrast, I find (\ref{itm:sas})-(\ref{itm:tipping})
to be more \q{novelistic} than
\q{journalistic}.  The speaker in these cases is reporting
the facts by, in effect, \i{narrating} them.  She is building
linguistic constructions that describe propositional
content through narrative structure {\mdash} or, at least, cognitive structures
that exemplify and come to the fore in narrative understanding.  Saying
\q{a substantial number of students}, for example, rather than
just (e.g.) \i{many} students, employs semantics
redolant of \q{force-dynamics}: the weight of student
anger is described as if a \q{substance}, something with the
potency and efficacy of matter.}

\p{This theme is also explicit in
\q{critical mass}, and even \i{tipping point} has material
connotations.  We can imagine different versions of what
lies one the other side of the tipping point
{\mdash} protests go from complaining to activism?  The school
forced to reverse course?  Or, contrariwise, the
school \q{cracking down} on the students
(another partly imagistic, partly force-dynamic metaphor)?
Whatever the case, language
like \q{critical mass} or \q{tipping point} is language
that carries a structure of story-telling;
it tries tie facts together with a narrative coherence.  The
students' protests grew more and more strident until ...
the protests turned aggressive; or the school dropped its plans;
or they won public sympathy; or attracted media attention, etc.
Whatever the situation's details, describing the facts in
force-dynamic, storylike, spatialized language (e.g. \q{come \i{out}
against}) represents an implicit attempt to report
observations or beliefs with the extra fabric and completeness
of narrative.  It ascribes causal order to how the
situation changes (a critical mass of anger could \i{cause} the
school to change its mind).  It brings a photographic or
cinematic immersion to accounts of events and descriptions:
\i{student after student} and \i{come out} invite us
to grasp the asserted facts by \i{imagining} situations.}

\p{The denoument of my argument is now that these narrative, cinematic,
photographic structures of linguistic reportage {\mdash} signaled
by spatialized, storylike, force-dynamic turns of phrase {\mdash}
represent a fundamentally different way of signifying
propositional content, even while they
\i{do} (with sufficient contextual grounding) carry
propositional content through the folds of the narrative.
I don't dispute that hearers understand logical forms
via (\ref{itm:sas})-(\ref{itm:tipping})
similar to those more \q{journaslistically}
captured in (\ref{itm:maj})-(\ref{itm:many}).  Nor do I
deny that the richer rhetoric of
(\ref{itm:sas})-(\ref{itm:tipping}) play a logical
role, capturing granular shades of meaning.  My
point is rather that the logical picture painted by the latter
sentences is drawn via (I'll say as a kind of suggestive
analogy) \i{narrative structure}.}

\p{I argued earlier that elements of propositional
structure {\mdash} for example, the
set/type selective operator efficacious in \i{students polled} {\mdash} can
have relatively clean morphosyntactic manifestation in
structural elements in language, like the verb-to-adjective
mapping on \i{polled} (here denoted, in English, by unusual
word position rather than morphology, although the
rules would be different in other languages).  Given my
subsequent analysis, however, I now want to claim that
the map between propositional structure and linguistic structure
is often much less direct.  I'm not arguing that
\q{narrative} constructions lack logical structure, or
even that their rhetorical dimension lies outside
of logic writ large: on the contrary, I believe
that they use narrative effects to communicate granular
details which have reasonable logical bases, like
degrees of students' anger, or the causative
interpretation implied in such phrases as \i{critical mass}.
The rhetorical dimension does not prohibit a reading
of (\ref{itm:sas})-(\ref{itm:tipping}) and 
(\ref{itm:students})-(\ref{itm:parents}) 
as expressing propositional content {\mdash} and using
rhetorical flourishes to do so.}

\p{I believe, however, that \i{how} they do so unzips any neat
alignment between linguistic and propositional structure.
Saying that students' protests \q{may have reached a critical mass}
certainly expresses propositional content (e.g., that enough
students may now be protesting to effectuate change),
but it does so not by mechanically asserting its propositional
idea; instead, via a kind of mental imagery which portrays
its idea, in some imaginative sense, iconographically.
\q{Critical mass} compels us to read its meaning imagistically;
in the present context we are led to actually visualize
students protesting \i{en masse}.  Whatever the actual, empirical
nature of their protestation, this language paints a picture
that serves to the actual situation as an interpretive
prototype.  This is not only a conceptual image, but
a visual one.}

\p{Figurative language {\mdash} even if
it is actually metaphorical, like \q{anger boiling over}
{\mdash} has similar effect.  Alongside the analysis
of metaphor as \q{concept blending}, persuasively
articulated by writers like Gilles Fauconnier and
Per Aage Brandt, we should also recognize how metaphor
(and other rhetorical effects) introduces into
discourse language that invites visual imagery.  Sometimes
this works by evoking an ambient spatiality (like
\q{come out against}) and sometimes by figuring
phenomena that fill or occupy space (like \q{students protesting}
{\mdash} one salience of this language is that we imagine
protest as a demonstrative gesture expanding outward, as if
space itself were a theater of conflict:
protesters arrayed to form long lines, fists splayed
upward or forward).  There is a kind of visual patterning
to these evocations, a kind of semiotic grammar:
we can analyze which figurative senses work via connoting
\q{ambient} space or via \q{filling} space,
taking the terms I just used.  But the details of such a
semiotic are tangential to my point here, which is that
the linguistic structures evoking
these visual, imagistic, narrative frames are not
simply reciprocating propositionl structure
{\mdash}- even if the narrative frames, via an \q{iconic}
or prototype-like modeling of the actual situation,
\i{are} effective vehicles for \i{communicating}
propositional structure.}

\p{What breaks down here is not propositionality but \i{compositionality}:
the idea that language signifies propositional content \i{but
also} does so compositionally, where we can
break down larger-scale linguistic elements to smaller parts
\i{and} see logical structures mirrored in the parts'
combinatory maxims.  In the later examples, I have argued that
the language signifies propositional content by creating narrative
mock-ups.  The point of these imagistic frames
is not to recapitulate logical structure, but to have a kind
of theatrical coherence {\mdash} to evoke visual and narrative order,
an evolving storyline {\mdash} from which we then understand
propositional claims by interpreting the imagined scene.
Any propositional signifying in these kinds of
cases works through an intermediate stage of
narrative visualization, whose structure is
holistic more than logically compositional.
It relies on our faculties for imaginative
reconstruction, which are hereby drafted into
our language-processing franchise.}

\p{This kind of language, in short, leverages its
ability to trigger narrative/visual framing as a
cognitive exercise, intermediary to the eventual
extraction of propositional content.  As such
it depends on a cognitive layer of narrative/visual
understanding {\mdash} which, I claim, belongs
to a different cognitive register than building
logical models of propositional content directly.}

\subsection{Triggers vs. Compositionality}
\p{In the absence of a compelling analysis of \i{compositionality}
in the structural correspondence between narrative-framed language
and logically-ordered propositional content, I consequently think we
need a new theory of how the former signifies the latter.  My own
intuition is that language works by trigering \i{several
different} cognitive
subsystems.  Some of these hew closely to predicate
logic; some are more holistic and narrative/visual.
Cognitive processes in the second sense may be informed
and refined by language, but they have an extralinguistic
and prelinguistic core: we can exercise faculties of narrative
imagination without explicit use of
language (however much language orders our imaginations
by entrenching some concepts more than others, via lexical
reinforcement).}

\p{I'm not just talking here about \q{imagination} in the sense
of fairy tales: we use imaginative cognition to make sense
of any scenario described to us from afar.  When presented with
linguistic reports of not-directly-observable situations,
we need to build cognitive frames modeling the context
as it is discussed.  In the terms I suggested earlier, we
build a \q{doxa inventory} tracking beliefs and
assertions.  Sometimes this means internalizing
relatively transparent logical forms.  But sometimes
it means building a narrative/visual account, playing
an imaginary version of the situation in our minds.
Language could not signify in its depth and
nuance without triggering this \i{interpretive-imaginative}
faculty.  Cognitively, then, language is an
\i{intermediary} to this cognitive system.  
Using terms from Olin Vakarelov's \q{Interface Theory 
of Meaning} \cite{OrlinVakarelov}, \cite{VakarelovAgent}
we might say that 
language is an \i{interface to} interpretive-imaginative
cognitive capabilities.}

\p{My argument is then that extra-linguistic cognition {\mdash} e.g., narrative 
construals, or situational understanding {\mdash} can supply crucial 
steps in the emergence of sentences' intended propositional content; 
so relying on logical assessment of propositional content and 
correlated linguistic patterns is, at best, incomplete.  
I have presented numerous sentences which, in my opinion, evince 
patterns whose rational contributions lie outside linguistic 
cognition proper, but which become attached as interpretations 
or refinements of given linguistic content; for example 
the semantic model of \i{critical mass} or \i{tipping point}
encapsulating a complex situational presenting.}

\p{The extra-linguistic source of many signifying contributors can 
be observed directly in reference to more formalized 
(say, Dependency-Grammatic) analyses (or indeed what they 
leave out).  In the (\ref{itm:actually}) case, the modifier
\i{actually} implies that the asserted fact is somehow surprising 
and counterintuitive; it's easy enough to conclude that the 
speaker senses a certain lexical frisson, particularly in 
how \i{camping} is described as \i{lodging}.  So the work 
of (\ref{itm:actually}) is not just to point out that 
people camp on the beach, but also that this arrangement is the 
most popular \q{lodging} even though in its usual sense
\i{lodging} applies more to hotels and inns.  This latter 
pattern circuits especially through the words \i{actually},
\i{lodging}, and \i{camping}, none of which are word-relations 
identified by the Universal Dependency parse (refer back to 
Figure~\ref{fig:actuallydbl}) but which can be seen as a second-order 
network superimposed on the sentence's syntactic core.}

\p{For other examples, in (\ref{itm:ambience}) the \i{colonial ambience} and
\i{tropical climate} disjunction partly shifts the sense of
\i{the city} from its status as something constructed and
historical (the \i{architecture} is colonial) to a geographic location.  We 
of course read \i{the climate} as \i{its} (the city's) climate, which 
establish a link between the two \i{the}.  This is almost anaphoric 
{\mdash} if the phrase were \i{its climate} we would resolve \i{it} backward
to \i{the city} {\mdash} but the pattern is varied, by replacing
\i{its climate} with \i{the climate}, which conceptually foregrounds 
the aspect of climate as an ambient phenomenon (amenable to a 
definite article) rather than just a possession (viz., a property 
of geographic places).  So there are several intersecting patterns 
{\mdash} both referential commonality and interpretive changes (see also 
the \q{Liverpool} examples for city qua architecture and qua locale) 
{\mdash} between the two sentence-halves, which are not described in the 
parse-graph.  And in, say,

\begin{sentenceList}

\sentenceItem{} \swl{itm:pipe}{We understand that the pipe fitters are also planning to picket the Lake Worth, Florida project as well.}{sco}
\udref{en_gum-ud-train}{email-enronsent03_01-0143}
\end{sentenceList}

there is a two-layer scope implication: given \i{also} we read the hearer as knowing that 
they are picketing some \i{other} project.  But we also hear \i{the pipe fitters}
as referring to some specific group of pipefitters, with which the speaker has 
some contractual relation; we do not interpret this as a comment about pipe-fitters in 
general.}

\p{The details of (\ref{itm:pipe}), in short, bound the scope of
\i{the pipe fitters} from both above and below.  The importance of
\i{also} for this interpretive understanding is elided from the 
corpora parse-graph, which just sees \i{also} as a common auxiliary.  
In general, the extra \q{layers} of meaning I have identified for 
(\ref{itm:actually}), (\ref{itm:ambience}), and (\ref{itm:pipe}) 
require some interpretive (and seemingly extralinguistic) 
reasoning, so they are not explicitly traced in purely 
formulaic Intermediate Representations involving core 
linguistic aspects, such as parse-graphs \visavis{}
syntax.}

\p{The extralinguistic dimensions of language are not, of 
course, completely apart from language: extralinguistic 
reasoning is appropriate to fill in the gaps left 
by straightforward syntactic or logico-semantic 
intellection alone, but those gaps exist \i{because}
language leads (via intra-linguistic structures) to 
signifying complexes.  The logic and conceptual detail 
achieved through intra-linguistic processes set forth 
the parameters on extra-linguic cognition that 
supplements them; so analytic coverage of the 
intra-linguistic aspect is still requisite for 
good analysis of the extra-linguistic.  This 
formal necessity, however, should not be mistaken 
for a belief that logically-inflected models 
of syntax and semantics are complete or 
self-sufficient.  Full implementation of a 
rigorous, logical syntactic-semantic system still 
does not get us to \i{language}.}

\thindecoline{}

\p{This then raises the question of what kinds of 
formal systems \i{are} appropriate models for language.  
In a truth-theoretic semantics, the founding analogy is 
that meanings are propositional, so that language builds 
up aggregate content much as logical expressions build 
up complex propositions.  This analogy allows the 
resources of mathematical logic to transfer, with 
some degree of rigor, over to linguistic 
representation {\mdash} not only predicate logic but 
also modal logic, typed lambda calculus, and 
type theory in general.  On the other hand, if 
we accept the picture that linguistic structures 
are often triggers to extralinguistic cognitive 
activity {\mdash} that this is their contribution 
to language comprehension {\mdash} then we may need 
to draw inspiration from a different suite of 
formal theories.}

\p{Probably influenced by early Analytic Philosophy, 
linguistics' formalizing paradigms have gravitated 
to mathematical logic, reflecting the influence of  
early 20th-century symbolic logic on a broad swath of 
Anglo-American science and intellectual life.  
Mathematics {\mdash} and by extension formal logic as a
theory of mathematical foundations {\mdash} came to be 
seen as a template for rigorous analysis to which 
philosophy should aspire; on that hypothesis 
the speculative styles of phenomenology and 
Continental Philosophy were widely seen as
methodologically flawed, compared to the 
systematized argumentation presented by the likes
of Gottlob Frege, Bertrand Russel, Rudolf Carnap, or 
Willard Van Orman Quine.  It is not a stretch to 
see the contemporary split between Cognitive and 
Computational linguistics as retracing, reiterating, 
or in a sense derived from last century's 
divorce between Analytic and Continental Philosophy.}

\p{On the other hand, in the digital age, a 
host of new formalizing projects offer an 
alternative to this mathematics-centric 
ideology.  These probably include \AI{} research 
to some degree, but also many other 
technical disciplines associated with the 
inner workings of computers and software 
{\mdash} compilers and compiler design; 
stack machines and other low-level computing 
frameworks; applied type theory; database implementation; 
process algebras and other theories of 
communicating, interdependent, or concurrent 
processes.  This technical ecosystem is 
arguably a better mine for linguistic 
inspiration precisely because all of these 
fields grapple with the problem of how 
to wrangle rationalistic behavior out of 
stubbornly unintelligent machines.  
When trying to coax a compiler to identify 
the correct functions for function-calls 
notated at each source-code location, for 
example, it does not directly matter whether 
the function-resolution is theoretically 
transparent {\mdash} i.e., that a correct 
and unique resolution can be demonstrated
by formal type analysis.  A compiler 
(or other software) ecosystem does not 
internally manifest type theory; at best
\i{humans} engineer software to abide by 
regulations which guarantee program  
behaviors through type-theoretic 
(and similar) analysis.  But engineering 
computational systems in conformance with 
mathematical, or otherwise formalizable, 
abstractions is itself a significant 
technical project.  The mere presence of 
abstractly provable structural properties 
is of only indirect relevance to 
computer implementations.}

\p{Likewise, it seems, the mere presence of system 
invariants which can be formalized from mental 
activity and given their own eidetic analysis is 
only indirectly relevant to cognitive science
\i{per se}.  Presumably brains have evolved under 
evolutionary pressures to undergird reasonable behavior; 
and perhaps structuralistic laws alongside physical ones 
are inevitable for any intelligent systems 
manifest in physical substance.  So it is likely that 
logical and other formal theories have some explanatory 
value in identifying regulative parameters influencing 
the gestation and physical realization of intelligence.  
But unlike mathematics, where the mere demonstration of 
the necessary truth of a theorem guarantees its existence 
as mathematical fact, in the realms of both cognition 
and computation abstract formal truths are not 
manifested merely by being true.  Organized systems 
must provide a dynamic space whose states are constrained 
by the relevant formal properties, in order for 
these formal certainties to be empirically 
influential.}

\p{Ultimately, computers have a limited collection of kernel 
functions, and software stacks are architectures that 
strategically call these functions in turn {\mdash} 
through several layers of translation and interpretation 
{\mdash} to achieve holistic effects (like displaying a 
textual document on a computer screen, or creating a 
three-dimensional animation from a set of 
geometric instructions).  While some logical 
considerations structure the process by which high-level 
computer code is translated to low-level system functionality, 
the manifest ground of this system is not any logical 
structure but the low-level functions themselves, along with 
their physical side-effects (like rendering colors on-screen).  
This is a useful analogy for natural language: the 
material unfolding of language as an empirical 
phenomenon is not a direct concretization of any 
logical structure, but rather the inventory of 
cognitive processes which language may trigger.  
Extralinguistic cognition is then analogous, in the 
human mind, to a computer's \q{system kernel}.  
While \q{mind as computer} analogies are 
usually reductionistic {\mdash} software has no human 
subtlety to (inter)subjectivity {\mdash} this architectural 
metaphor perhaps has some merit.  At least it can 
point to topics in computer science that could be
placed alongside predicate logic as formalizing 
intuitions for the study of natural language.}
