\section{Cognitive and Computational Processes}
\label{s4}
\p{Any attempt to bridge Computational Linguistics and
Cognitive Grammar or Phenomenology must solicit one or several
\q{founding analogies}, linking phenomena on the
formal/computational side with those on the
cognitive/computational side.  Here, I will start from
the analogy of \i{cognitive} and \i{computational} \i{process},
or generically \q{process} (of either variety).
Processes, per se, I will
leave undefined, although a \q{computational} process
can be considered roughly analogous to a single
procedure implemented in a computer programming language.
The story I want to tell goes something like this: understanding
language involves many cognitive processes, many of
which are subtly determined by each exact language artifact
and the context where it is created.  Properly understanding a
piece of language depends on correctly weaving together
the various processes involved in understanding its
component parts, and the structure of the
multi-process integration is suggested by the grammar of
the artifact.  Grammar, in a nutshell, uses relationships
between words to evoke relationships between
cognitive processes.}

\p{My formal elaboration of this model will be inspired at an
elementary level by process \i{algebra} in the computational
setting, as well as applied \i{type theory}.
Inter-process relations are the core topic of Process
Algebra, including sequentiality (one process followed by
another) and concurrency (one process executing alongside
another).  In practice, detailed research around Process Algebra
seems to focus especially on concurrency, perhaps because
this is the more complex area of application
(designing computer systems which can run multiple threads in
parallel).  It is likewise tempting to
imagine that cognitive-linguistic processes exhibit some
degree of parallelism, so that the various pieces of
understanding \q{fall into place} together as we grasp
the meaning of a sentence (henceforth using \i{sentence} as a
representative example of a mid-size language artifact in general).
Nevertheless, I will focus more on \i{sequential} relations between
processes, suggesting a language model (even if rather idealized)
where cognitive processes unfold in a temporal order.}

\p{On both the cognitive and computational side, temporality is relative
rather than quantitative: the significant detail is not
\q{before} and \q{after} in the sense of measuring time but rather how
one process logically precedes another in effects and prerequisites.
No theoretical importance is attached to \i{how long} it takes
before processes finish, or how much time elapses between
antecedent and subsequent processes (in contrast to subjects like
optimization theory, where such details are often significant).
We can set aside notions of a temporal continuum
where subsequent processes occupy disjoint, extended time-regions;
instead, one process follows another if anything affected by the first
process reflects this effect at the onset of the second process.
Time, in this sense, only exists as manifest in the variations
of any state relevant to processes {\mdash} in the computational
context, in the overall state of the computer (and potentially
other computers on a network) where a computation is
carried out.  Two times are different only insofar as the
overall state at one time differs from the state at the second time.
Time is \i{discrete} because the relevant states are discrete, and
because beneath a certain scale of time delta there is no
possibility of state change.}

\p{Analogously, in language, I suggest that we set aside notions of
an unfolding process reflecting the temporality of expression.
Of course, the fact that parts of a sentence are heard first
biases understanding somewhat; and speakers often exploit
temporality for rhetorical effect, elonging the pronunciation
of words for emphasis, or pausing before words to
signal an especially calculated word choice, for example.
These data are not irrelevant, but, for core semantic and
syntactic analysis, I will nonetheless treat a sentence as
an integrated temporal unit, with no value attributed to
temporal ordering amongst words except insofar as temporal
order establishes word order and word order has grammatical
significance in the relevant natural language/dialect.}

\p{While antecedent/subsequent inter-process relations are among those
formally recognized in Process Algebra, this specific genre
of relation is implicit to other models important
to computer science, such as Type Theory and Lambda Calculus.
If \typeT{} is a type, then any computational process
which produces a value of type \typeT{} has a corresponding
(\q{functional}) type (for sake of discussion, assume a \q{value}
is anything that can be encoded in a finite sequence of numbers
and that \q{types} are classifications for values that introduce
distinctions between functions {\mdash} e.g., the function to add two
integers is different than the function to add two decimals; more
rigorous definitions of primordial notions like \q{type} and
\q{value} are possible but not needed for this paper).
Similarly a process which takes as \i{input} a value of
\typeT{} is its own type.  If two processes have these two
types respectively {\mdash} one outputs \typeT{} and the other
inputs \typeT{} {\mdash} then the two can be put in sequence, where
the output from the antecedent becomes the input to the subsequent.
In this manner inter-process sequential relations become
subsumed into \q{type systems} and can be studied using
type-theoretic machinery rather than Process Algebras or
Process Calculi as such.}

\p{The above comments apply to type theory applied in 
constructed environments like computer programming 
languages; but as I have discussed here 
there exists a robust type-theoretic tradition
in (Natural Language) semantics, which is disjoint from
but not entirely irrelevant to the type systems of
formal and programming languages.  Semantic types are
recognized at several different levels of classification 
{\mdash} I proposed the terms \i{macrotype}, \i{mesotype}, 
and \i{microtype}.  Some of the most 
interesting type-theoretic effects
involve medium-grained semantic criteria that are
more general than lexical entries but more specific than
Parts of Speech; this is the level where linguists have 
seemed to find the most fertile applications of 
sophisticated aspects to formal type theory, like 
dependent types and type-coercion.  This perhaps 
reflects the situation of \q{mesotypes} as 
intermediary between and therefore interconnecting 
types in a mostly grammatic (Part of Speech) and 
a mostly semantic (lexical) sense; and therefore 
a topic or tool for both syntax and semantics.}

\p{Here my argumentation is informed 
particularly by the merged notion of \i{typed}
processes.  If we say that something (sticking with the 
mesotype level) has the \i{type}
of a physical-body noun {\mdash} that \q{Physical Body} is
a type in the overall semantics of language {\mdash} then
I propose a corresponding type of cognitive
(perceptual and conceptual) processes characteristic
of perceiving and reasoning about physical things.  A particular
designatum {\mdash} a bag of rice, say {\mdash} is subsumed under
the semantic type insofar as our perceptual encounters with that
thing {\mdash} and/or our conceptual exercises pertaining to
its properties and proclivities (like being difficult to carry)
{\mdash} are roughly prototyped by a certain generic kind of
cognitive process.  This assumes that there is a similitude among
processes of perceiving and thinking about physical bodies
(at least the mid-sized, quotidian physical things that tend
to enter nonspecialist language) sufficient to subsume them under
a common prototype, which I then argue forms the cognitive
substratum for the semantic type \q{Physical Object}.
Moreover, I contend a similar cognitive substratum
can be found for other mid-scale semantic types that underlie analyses
of semantic acceptability and metaphoricality, like
\q{Living Thing}, \q{Sentient Living Thing} (\i{flowers want
sunshine} is metaphorical because it ascribes propositional
attitudes to something whose type does not literally support them),
and \q{Social Institutions} (\i{The newspaper you're reading
fired its editor} exhibits a \q{type coercion} where \i{newspaper} is read
first as an object and then as a company).  One feature of semantic
types is the lexical superposition of different types to produce what
(in a slightly different context) Fauconnier calls a \q{blend}:
in \i{Liverpool, which is near the ocean, built new docks}
(see \ref{itm:Liverpool}), the
city is treated as both a geographic region and a body politic.}

\p{

\q{Weighs}, too, as a verb, can be given a typed-process
interpretation.  In its least metaphoric sense, \q{to weigh}
connotes a practical action of measuring some object's weight by
using something like a scale; as \i{cognitive} process the
verb embodies are ability to plan, reflect upon, or contemplate
this practice.  So an \q{idea weighing 5 pounds} is anomalous because
it is hard to play out in our minds a form of this practical act
where the thing being weighed is mental.  However, there are plenty
of more figurative uses related to \q{weighing ideas}, \q{heavy ideas},
and so forth, so we are able to isolate the dimension of
\q{judging} and \q{measuring} which is explicit in literal
\q{weighing}, and abstracting from the physical details use
\q{weigh} to mean \q{measure} or \q{assess} in general.
The phrase \i{weigh an idea} therefore connotes its own cognitive
process {\mdash} imagining someone thinking about the idea in an
evaluative way {\mdash} but this figurative \q{script} is closed off by
\q{5 pounds} which forces us to conceive the weighing literally
with a scale, not figuratively as a kind of mental assessment.
Once again, the type anomaly can be seen as a failure to
map the linguistic senses evident in a sentence to an internally
consistent set of cognitive procedures for dilating the semantic
content seeded within each word.}

\p{Notice that I am treating cognitive processes, in themselves,
as semantic more than grammatical phenomena.  Literally,
weighing something is a multi-step act
(lifting it on the sale, reading the measurement), and even in
our mental replay of hypothetical weighing-acts it seems impossible
not to imagine distinct phases (just as it is impossible not
to picture left and right sides of an imaginary cow).
However, I assume that the cognitive script is figured by the
lexeme \q{weighs} as a connotative unit: whatever internal structure
our mental script of \q{weighing something} has,
this structure is not a \i{linguistic} structure that must be
encoded grammatically.  Similarly, the concept
\i{buttered toast} suggests a confluence of
perceptual, physical-operational, and conceptual aspects
{\mdash} we are inclined to regard toast as \i{buttered} if it
looks a certain way and also if we have seen someone apply
butter to it (or have done so ourselves) and also if
we are in a context where we expect to find toast that may be
buttered (we are not disposed to call a piece of bread in a
grocery store \q{buttered toast} even if it has that appearance).
So the adjective \i{buttered} introduces multiple cross-modal
parameters in addition to the underlying concept \i{toast}; but I feel
that the lexeme aggregates these parameters into a
single \i{linguistic} unit.  In Langacker's terms, the various
elements of \q{buttered} do not suggest \i{constructive effort},
as if deliberate \i{linguistic} processing were needed to unpack the
linguistic entity to its constituent parts.  Instead, \q{buttered}
functions \i{semantically} as a unit (and likewise syntactically
as the unit entering relations with other words {\mdash} e.g. buttered-toast
is an adjective/noun pair, not the noun \i{butter} at the root of the
adjective) {\mdash} even if its cognitive process
is multi-faceted.  Indeed, this is precisely the signifying economy
of language: it captures complex cognitive procedures by
iconic, repeatable lexical units.}

\p{On that theory, tieing specific word-senses to stereotyped
cognitive processes is a matter of semantics, not grammar per se.
Grammar, I contend, comes into play when multiple processes need to
be integrated.  The concept \q{buttered toast}, for example, seems
to start from a more generic concept (toast) and then add
extra detail (the buttering, with all that implies conceptually,
pragmatically, and sensorially).  This is suggested by the
substitutability of just \i{toast} for \i{buttered toast}:

\begin{sentenceList}

\sentenceItem{} \swl{}{I snacked on toast and coffee.}{sem}
\sentenceItem{} \swl{}{I snacked on buttered toast and iced coffee.}{sem}
\end{sentenceList}

Because the first sentence is perfectly clear, it seems
that the ideas expressed (at least in this context) by
\i{toast} and \i{coffee} are reasonably complete
in themselves, so the adjectives have the effect of starting
with a complete idea and adding on extra detail.  Procedurally,
then, it seems like we have some process which takes us to
\q{toast} and \q{coffee} and then, subsequent to that
(logically if not temporally) we add the wrinkle of
re-conceiving the toast as buttered and the coffee as iced.
In short, the adjective-noun pairing is compelling us to
run a pair of cognitive processes in sequence, one
establishing the noun-concept as a baseline and one adding
descriptive detail by an \q{adjectival}, a specificational
process.}

\p{Counter to that analysis, someone might judge that
phrases like \q{buttered toast} and \q{iced coffee} are
conventional enough that we don't interpret them through
two meaningfully disjoint processes.  This is entirely possible,
given how erstwhile aggregate expressions become established units.  
Different phrases exhibit different levels of entrenchment:

\begin{sentenceList}

\sentenceItem{} \swl{}{I snacked on toast and instant coffee.}{sem}
\sentenceItem{} \swl{}{I snacked on toast and Eritrean coffee.}{sem}
\end{sentenceList}

Arguably \q{instant coffee} is a de facto lexical unit, partly
because reading it in terms of constituent parts is rather
nonsensical (there's no non-oblique way to understand
\q{coffee} being qualified as \q{instant}).  Surely, however,
\i{Eritrean coffee} is heard as a compound phrase
(at least in 2019 {\mdash} it is unlikely, but not
impossible, that future Eritrean coffee growers will be so
successful that we hear the phrase as a brand name
or culinary term of art, like \q{Hershey's kisses}
or \q{French toast}).  The status of \i{iced coffee} is probably
somewhere between these two.  But to the degree that
a language element (whether word or phrase) is entrenched and
generally processed linguistically as a unit, I maintain,
it tends to be governed by an integrally complete cognitive
process {\mdash} not necessarily one without inner structure, but where the
elements of this structure piece together perceptually and
situationally, rather than seeming to be
\i{linguistically} disjoint conceptualizations that are brought together
by the shape of linguistic phrases.  Conversely, where a cognitive
process has this integral character, discursive pressures nudge
the language toward entrenching some descriptive phrase as a
quasi-lexeme; what starts being heard as a compound designation
evolves to the point where language users don't attend to
constituent parts.}

\p{Obviously, this theory presupposes that there is an available
distinction to be drawn between a \q{procedural} synthesis of
disparate cognitive processes and a perceptual and/or conceptual
synthesis constitutive of individual cognitive episodes.
Phenomonology seems to back this up {\mdash} there are some
conceptual compounds that come across as more episodically fused than
others.  Buttered toast may evoke a temporally
not-quite-instantaneous conceptualization {\mdash} at the core of
the concept is a practical activity that takes a few seconds to
complete {\mdash} but we also can imagine the buttering-act
apprehended in one sole episode.  On the other hand,
\q{Eritrean coffee} ties together concepts of much more scattered
provenance; the perceptual unity of \i{coffee} (in the sense
of a specific liquid in a specific container) along with
the geopolitical \q{background knowledge}
implicit in the adjective \i{Eritrean}.  As a cognitive
synthesis \i{Eritrean coffee} is conceptual rather than
perceptual.  Provisionally we can treat this in the context of
\i{buttered toast} being a partially-entrenched phraseology
while \i{Eritrean coffee} is undeniably a phrasal compound,
something whose constructive form must be parsed linguistically
rather than figuratively.}

\p{This analysis, though, needs many caveats.  After all, many
bonafide \i{phrases} (not \q{quasi-lexemes}) nevertheless
exhibit significant phenomenological unity {\mdash} i.e., they
evoke integral perceptual complexes: \i{big dog};
\i{hot coffee}; \i{speeding car}; \i{red foliage}.
Linguistically these seem like an underlying concept
acquiring perceptual specificity via adjectival
modification: \q{hot} was how the coffee came to my
experience because I experienced it as hot (it wasn't like
I experienced the coffee and then had to contemplate
whether it is hot or cold).  Coffee can't \i{not} be
experienced as hot, cold, or lukewarm; it cannot be experienced
without temperature (assuming I am coming into contact
with it and not just looking at it).  Similarly a car must be
seen as at rest, moving slowly, or speeding along; 
foliage must be
seen as having some color(s).}

\p{I have argued, however, that
unless entrenched as idiomatic phrases adjective-noun pairs
like \i{hot coffee} or \i{buttered toast} should be read
as grammatical complexes and accordingly (in my theory) as junctures
between distinct cognitive processes.  On the other hand, I argued
that \q{instant coffee} was effectively entrenched \i{because}
there is no simplistic conceptual unity between \i{instant} and
\i{coffee}, which makes it harder to hear the phrase as descriptive.
Instead, the semantics of that particular adjective-noun
connection are circuitous and a little hyperbolic: \q{instant}
coffee is coffee as a substance (not a drink, in that state)
from which coffee the drink can be quickly (but not
instantaneously) prepared.  There is a lot going on the seemingly
simple \q{instant coffee}: the shift from coffee-as-substance
to coffee-as-drink; the \q{instant} exaggeration.  In this
case, the adjectival modification has \i{so many} moving
parts that, I'm inclined to argue, it is hard to cover the
whole scenario with a descriptive phrase; which in turn
creates selective pressures for some pseudo-lexical unit
to emerge (which turned out to be \i{instant coffee})
as a mnemonic for the whole conceptual multiplex.  Indeed
conceptually intricate wholes tend to quickly acquire
pithy conventional nominalizations simply
for rhetorical convenience (\q{Brexit};
\q{Quantum Gravity}; \q{International Transfer
Window}; \q{\#metoo}).}

\p{Notwithstanding these variations, I still find a certain logic
in the relation between phenomenological unity and semantic
entrenchment.  Perceptually integrated wholes may correlate
with linguistically aggregate constructions insofar as there
is a transparent logical destructuring in the perceptual
unity: in the case of substance-attribute pairs (like
\i{hot coffee}) {\mdash} deferring in the phenomenological
context to Husserl's account of \q{dependent moments} {\mdash} there
is a basically unsubtle distinction between an underlying
concept (like coffee) and the qualities which are its
mode of appearance as well as conceptual predicates (like
hot, cold,  black, or light, describing sensory properties
innate to the experience of a coffee-token as well as
state-reports that can be propositionally attributed to it).
Although the minimal sensate intention of the coffee and
the predicative disposition toward ascriptions like \i{black}
and \i{hot} are consciously intertwined, surely I am aware of
a logicality in experience that gives the sensate and predicative
dimensions different epistemic status.  I don't think of
my experience of the coffee's being hot as just a hot sensation
qua medium of my sensorily apprehending the coffee, but rather as
the sensate mechanism by which I observe the apparent fact that
the coffee is hot, as a state of affairs and not just as
subjective impression.  We are constantly extrapolating our perceptual
encounters to propositional content along these lines.}

\p{As such, I contend that such an (in some sense) innate
perception-to-predication instinct grounds the procedural
slicing of linguistic processes: \i{hot coffee} retraces in a 
linguistic construction the logical order of a coffee perception
which on one level is a raw perceptual encounter but is simultaneously
a predicative attribution. \q{Hot coffee} denotes a substance that
can be experienced in the mode of a base concept (coffee)
which is given predicative qualification (the coffee \i{is} hot).
The fact that there may be no noticed temporal gap
in \i{experience} between the sensate perception and the
epistemic posture does not preclude a certain logical
antecedent-subsequent ordering: the concept \i{coffee} is the
predicative base for my propositional attitude that what I
am dealing with here is hot coffee, not
hot-sensations-disclosing-coffee or coffee-I-experience-as-hot
(but who knows, maybe I'm hallucinating) or any other artificial
skeptifying of my actual experience, which is of raw perception
pregnant with propositional content.}

\p{So I wish to justify claims that (non-entrenched) phrase 
complexes like \q{hot coffee} are unions of disjoint cognitive
processes by noting that while such phrases evoke a certain perceptual
unity, they evoke a \i{kind} of unity
which we habitually regard \i{conceptually} as divided into
sensate givenness founding epistemic attitudes.  Cognitive
processes are not exclusively perceptual; they are
some mixture of perceptual and conceptual (and
enactive/kinaesthetic/operational).  A perceptual unity can
cover two conceptual aspects, like a sheet covering two
mattresses.  So the perceptual unity of hot coffee can
become the conceptual two-step of coffee as substance
and hot as attribute; committing this unity to
cognition as an overarching lifelong faculty involves registering
a thought-process of coffee as a substance which can, in acts of
logical predication, be believed to be hot or cold, black or
light, etc.  The apprehension of the substance is a different
cognitive process than the predication of the attribute, in terms of
how these mental acts fit within our epistemic models, even if
these two processes are experientially fused.  Typically we see the
coffee before we touch or taste it, so already the coffee has a logical
status apart from the heat we predicate in it.}

\p{Likewise, even if the black color is inextricable
from our perceiving (apart from
odd situations where we drink the coffee without looking at it),
we know the color will change if we add milk (even if just in
principle because, preferring black coffee, I don't actually do so);
so we know the coffee has a logical substrate apart from its color
too.  All of this ideation is latent in the coffee-perceptions
notwithstanding whatever perceptual unities we experience that
cloak logical forms like substance/attribute under the inexorable
togetherness of disclosure (the phenomenological impossibility of
spatial expanse without color, say).  In short, disjoint
cognitive processes can be required to reconstruct a perceptually
unified situation or episode, insofar as we are not just living
through the episode but prototyping, logically reconstructing,
signifying it {\mdash} the perceptual unity in the moment does not
propagate to procedural atomicity in absorbing the episode into
rational exercises.}

\p{Experience, then, presents \i{both} perceptual unities and
cognitive-propositional multiplicity; language can inherit
both holism on the perceptual side and compositionality
on the rational side, even in a single enactive/perceptual
episode.  Depending on how we via language want to
figure and express experience, we can bring either unity
or compositionality to the fore.  Our linguistic choices
will evoke perceptual unity if they select entrenched word-senses
or quasi-lexical forms; they will evoke compositionality
if they gravitate toward compound phrases and complex,
relatively rare lexicalizations and modes of expression.
To the degree that we are interested in a
cognitive-phenomenological \i{semantics}, we can attend
to the first part of this equation, to how the understood
atomicity of a word sense or a conventionalized phrase often
suggests an object or phenomenon consciously apprehended as
an integral whole; we can trace phenomenologically the
apperceptive unity that seems to drive the linguistic
community's accepting lexical atoms in this sense.
Conversely, to the degree that we are interested in a
cognitive-phenomenological \i{grammar}, we can attend to how
logically composite predication emerges even within
perceptual unity, because our encounter with phenomena is
not (save for exotic artistic or meditative pursuits) the
\q{ \i{dasein}} of irreflective sensory beings immersed in a
world of pure experience but the deliberate action of
epistemic beings carrying (modifiable but not random)
propositional attitudes to perceptual encounters.}

\p{Modeling grammar as a coordination between cognitive processes may
be an idealization, precisely because the compositive and integrative
faces of consciousness are two sides of the same
coin: it's not as if we work through a thought of
\i{coffee} or \i{toast}, abstract and without sensory
specificity, noticeably prelude to conceived/perceived
attributes like \i{hot}, \i{cold}, or \i{buttered}.
But we can still ascribe to linguistic-understanding
processes an idealized, \q{as if} temporality, treating the
elucidating of a sentence as a sequence of procedures leading from
bare concepts to well-rendered logical tableau,
suffused with some level of descriptive and situational
particularity.  So we go from \i{coffee} to \i{iced coffee} to
\i{buttered toast and iced coffee} to \i{snacking on
buttered toast and iced coffee}; each link in the chain
stepping up toward propositional totality.}

\p{My point is not that the logical form of the 
sentence is composed from
logically primitive and abstract parts, which is fairly trite;
my point is that such logical composition is only apparent
after a pattern of cognitive integration
that is more subtle and exceptional.  Extra-mentally, buttered
toast is just toast with butter on it, a fairly simplistic
logical conjunction.  Read as a baton passed between
two acts of mind, however {\mdash} conceiving toast and then conceiving
it buttered {\mdash} the conjunction is more elaborate;
the cognitive resources of \i{buttered} are not just
\q{something with butter on it} but the implication of a sensory
summation (the flavor, color, scent) and operational narrative
(we have seen or performed the deliberate act of applying the
butter).  Similarly a person dressed up is not just
someone whose torso is encircled by articles of clothing; a
barking dog is not just an animal making random noises;
a stray cat is different from a lost cat.  In their
interpenetration, cognitive processes develop (in the
photographer's sense) narrative
and causative threads that are latent in worldly situations
but reduced out of logical glosses; that is why it
seems incomplete, lacking nuance, or
beside the point to explicate semantic meanings in logical
terms, like \q{bachelor} as \q{unmarried man} (we can
certainly imagine a sentence like \i{My best friend
has been married for years but he's still a bachelor},
to imply he still has the habits and
attitudes of his single days).}

\p{A theory of sentences building from conceptual
underspecification to logical concreteness does
not preclude there being different scales
of specificity. \i{I snacked on toast and coffee}
is just as acceptable as \i{I snacked on buttered toast
and iced coffee}.  The communication conveys as much
situational detail as warranted in the conversational,
pragmatic context.  Language always has the \i{capability} to
push further and further into specificity; how
exhaustively the language user avails of this
capability is a matter of choice.  As theorists of language we
must then analyze how language possesses the \i{latent} capacity
to draw ever finer pictures; the adjectival \i{buttered} toast and
\i{iced} coffee takes the granularity of signifying at one
level (the level of the \i{I snacked on toast and coffee} sentence)
and layers on (or really layers \i{within}) a yet more
specific level.  The architecture of how this happens is
well addressed by type-theoretic methods (both coarse and mid-grained).}

\p{The remainder of this section will try to expand on this type-theoretic intuition.
My central thesis is that language understanding
involves integrating diverse \q{cognitive procedures},
each associated with specific words, word morphologies (plural
forms, verb tense, etc) and sometimes phrases.
The form of type theory appropriate in this context
is therefore one closely associated with procedural typing
and resolution: that is, assigning types to procedures,
and differentiating procedures based on the types of
their \q{arguments} or \q{parameters} (input and output data).}

\subsection{Interpretive Processes and Triggers}

\p{The type-theory/procedural perspective I will mostly work from
here contrasts with and adds nuance to
a more \q{logical} or \q{truth-theoretic} paradigm which
tends to interpret semantic phenomena via formal logic {\mdash}
for example, singular/plural in Natural Language
as a basically straightforward translation of the
individual/set distinction in formal logic.  Such
formal intuitions are limited in the sense that (to
continue this example) the conceptual mapping from
single to plural can reflect a wide range of
residual details beyond just quantity and multitudes.
Case in point are plurals: for each plural usage we have a conceptual
transformation of an underlying singular to a collective, but
how that collective is pictured varies in context.  One
dimension of this variation lies with mass/count: the
mass-plural \i{coffee} (as in \q{some coffee}) figures the
plurality of coffee (as liquid, or maybe coffee grounds/beans)
in spatial and/or physical/dynamic terms.  So we have:

\begin{sentenceList}

\sentenceItem{} \swl{itm:coffeeshirt}{There's some coffee on your shirt.}{ref}
\sentenceItem{} \swl{}{There's coffee all over the table.}{ref}
\sentenceItem{} \swl{}{She poured coffee from an ornate beaker.}{ref}
\sentenceItem{} \swl{}{There's too much coffee in the grinder.}{ref}
\sentenceItem{} \swl{}{There's a lot of coffee left in the pot.}{ref}
\sentenceItem{} \swl{}{There's a lot of coffee left in the pot
{\mdash} should I pour it out?}{ref}
\end{sentenceList}

These sentences use phrases associated with plurality (\i{all over},
\i{a lot}, \i{too much}) but with referents that on perceptual and
operational grounds can be treated as singular {\mdash} as in the
appropriate pairing of \i{a lot} and \i{it} in the last sentence.
With count-plurals the collective is figured more as an
aggregate of discrete individuals:

\begin{sentenceList}

\sentenceItem{} \swl{}{There are coffees all over the far wall at the espresso bar.}{ref}
\sentenceItem{} \swl{}{She poured coffees from an ornate beaker.}{ref}
\sentenceItem{} \swl{itm:coffeetable}{There are a lot of coffees left on the table
 {\mdash} shall I pour them out?}{ref}
\end{sentenceList}
}

\p{So mass versus count {\mdash} the choice of which plural form to use {\mdash}
triggers an interpretation shaping how the plurality is pictured and
conceived (which is itself triggered by the use of a plural to
begin with).  Compare \i{I sampled some chocolates} (where the count-plural
suggests \i{pieces} of chocolate) and
\i{I sampled some coffees} (where the count-plural implies
distinguishing coffees by virtue of grind, roast, and other
differences in preparation) (note that both are contrasted
to mass-plural forms like \i{I sampled some coffee} where
plural agreement points toward material continuity; there
is no discrete unit of coffee qua liquid).  Or compare
\i{People love rescued dogs} with \i{People fed the rescued dogs}
{\mdash} the second, but not the first, points toward
an interpretation that certain \i{specific} people
fed the dogs (and they did so \i{before} the dogs
were rescued).}

\p{But if we restrict attention to just, say,
count-plurals, there are still different schemata for intending
collections.  This is one dimension I left unexamined 
in the earlier \q{New York} examples:

\begin{sentenceList}

\sentenceItem{} \swl{itm:NYboroughs}{New Yorkers live in one of five boroughs.}{ref}
\sentenceItem{} \swl{itm:NYDem}{New Yorkers reliably vote for Democratic presidential candidates.}{ref}
\sentenceItem{} \swl{itm:NYcommute}{New Yorkers constantly complain about how long it takes to commute
to New York City.}{ref}
\end{sentenceList}

The first sentence is consistent with a reading applied to
\i{all} New Yorkers {\mdash} the five boroughs encompass the whole
extent of New York City.  The second sentence is only reasonable
when applied exclusively to the city's registered voters {\mdash} not all
residents {\mdash} and moreover there is no implication that the
claim applies to all such voters, only a proportion north of one-half.
And the final sentence, while perfectly reasonable, uses \q{New Yorkers}
to name a population completely distinct from the
first sentence {\mdash} only residents from the metro
area, but not the city itself, commute \i{to} the city.}

\p{The assumption that logical modeling can capture all the
pertinent facets of Natural-Language meaning can
lead us to miss the amount of situational reasoning
requisite for commonplace understanding.  In
\i{People fed the rescued dogs} there is an exception to the
usual pattern of how tense and adjectival modification
interact: we read \q{people fed} in
\i{People fed the rescued dogs} as occurring \i{before} the rescue;
because we assume that \i{after} being rescued the dogs would be
fed by veterinarians and other professionals (who would
probably not be designated with the generic \q{people}), and
also we assume the feeding helped the dogs survive.  We also
hear the verb as describing a recurring event; compare
with \i{I fed the dog a cheeseburger}.}

\p{To be sure, there are patterns and templates governing
scope/quantity/tense interactions that help us build logical models
of situations described in language.  Thus
\i{I fed the dogs a cheeseburger} can be read such that there
are multiple cheeseburgers {\mdash} each dog gets one {\mdash}
notwithstanding the singular form on \i{a cheeseburger}:
the plural \i{dogs} creates a scope that can elevate
the singular \i{cheeseburger} to an implied plural;
the discourse creates multiple reference frames each
with one cheeseburger.  Likewise the morphosyntax is
quite correct in: \i{All the rescued dogs are taken to an
experienced vet; in fact, they all came from the same
veterinary college} {\mdash} the singular on \i{vet} is properly
aligned with the plural \i{they} because of the scope-binding
(from a syntactic perspective) and space-building
(from a semantic perspective) effects of the \q{dogs} plural.
Or, in the case of \i{I fed the dog a cheeseburger every day}
there is an implicit plural because \q{every day} builds
multiple spaces: we can refer via the spaces collectively
using a plural (\i{I fed the dog a cheeseburger every day {\mdash}
I made them at home with vegan cheese}) or refer within
one space more narrowly, switching to the singular
(\i{Except Tuesday, when it was a turkey burger}).}

\p{Layers of scope, tense, and adjectives interact in complex ways that
leave room for common ambiguities: \i{All the rescued dogs are [/were]
taken to an experienced [/specialist] vet} is consistent with a reading
wherein there is exactly one vet, and she has or had treated every dog.
It is \i{also} consistent with a reading where there
are multiple vets and each dog is or was treated
by one or another.  Resolving such ambiguities
tends to call for situational reasoning and a \q{feel} for situations,
rather than brute-force logic.  If a large dog shelter describes
their operational procedures over many years, we might assume
there are multiple vets they work or worked with.  If instead the
conversation centers on one specific rescue we would be
inclined to imagine just one veterinarian.  Lexical and tense
variation also guides these impressions: the past-tense
form (\i{...the rescued dogs were taken...}) nudges us
toward assuming the discourse references one rescue (though it
could also be a past-tense retrospective of general operations).
Qualifying the vet as \i{specialist} rather than the vaguer
\i{experienced} also nudges us toward a singular interpretation.}

\p{What I am calling a \q{nudge}, however, is based on situational
models and arguably flows from a conceptual stratum outside
of both semantics and grammar proper; maybe it is even prelinguistic.
Consider

\begin{sentenceList}

\sentenceItem{} \swl{itm:pf}{People fed the rescued dogs.}{sem}
\sentenceItem{} \swl{itm:ve}{Vets examined the rescued dogs.}{sem}
\end{sentenceList}

There appears to be no explicit principle either in the semantics
of the lexeme \i{to feed}, or in the relevant tense agreements,
stipulating that the feeding in (\ref{itm:pf})
was prior to the rescue {\mdash} or conversely that
(\ref{itm:ve}) describes events
\i{after} the rescue.  Instead, we interpret the discourse through
a narrative framework that fills in details not provided by
the language artifacts explicitly (that abandoned dogs are
likely to be hungry; that veterinarians treat dogs in clinics, which
dogs have to be physically brought to).  For a similar case-study,
consider the sentences:

\begin{sentenceList}

\sentenceItem{} \swl{}{Every singer performed two songs.}{sem}
\sentenceItem{} \swl{}{Everyone performed two songs.}{sem}
\sentenceItem{} \swl{}{Everyone sang along to two songs.}{sem}
\sentenceItem{} \swl{}{Everyone in the audience sang along to two songs.}{sem}
\end{sentenceList}

The last of these examples strongly suggests that of potentially
many songs in a concert, exactly two of them were popular and singable
for the audience.  The first sentence, contrariwise, fairly strongly
implies that there were multiple pairs of songs, each pair performed
by a different singer.  The middle two sentences imply either
the first or last reading, respectively (depending on how we
interpret \q{everyone}).  Technically, the
first two sentences imply a multi-space reading and the latter two
a single-space reading.  But the driving force
behind these implications are the pragmatics of \i{perform} versus
\i{sing along}: the latter verb is bound more tightly to
its subject, so we hear it less likely that
\i{many} singers are performing \i{one} song pair, or conversely that
every audience member \i{sings along} to one song pair, but
each chooses a \i{different} song pair.}

\p{The competing interpretations for \i{perform} compared to
\i{sing along}, and \i{feed} compared to \i{treat}, are grounded
in lexical differences between the verbs.  I contend, though, that
the contrasts are not laid out in lexical specifications
for any of the words, at least so that the implied readings
follow just mechanically, or on logical considerations
alone.  After all, in more exotic but not implausible
scenarios the readings would be reversed:

\begin{sentenceList}

\sentenceItem{} \swl{}{The rescued dogs had been treated by vets in the past
(but were subsequently abandoned by their owners).}[-> sem][The rescued dogs had been treated by vets in the past.]
\sentenceItem{} \swl{}{Every singer performed (the last) two songs
(for the grand finale).}[-> sem][Every singer performed the last two songs for the grand finale.]
\sentenceItem{} \swl{}{Everyone in the audience sang along to two songs
(they were randomly handed lyrics to different songs when
they came in, and we asked them to join in when the song being
performed onstage matched the lyrics they had in hand).}{sem}
\end{sentenceList}

In short, it's not as if dictionary entries would specify that
\i{to feed} applies to rescued dogs before they are rescued,
and \i{to treat} applies after they are rescued.  Or
that \i{sing along} nudges scope interpretation in one direction
and \i{perform} nudges in a different direction.
These interpretations are driven by narrative
construals narrowly specific to given expressions.  The
appraisals would be very different for other uses of the verbs in
(lexically) similar (but situationally different) cases:
to \q{treat} a wound or a sickness, to \q{perform} a gesture or a
play.  We construct an interpretive scaffolding
for resolving issues like scope-binding and space-building based
on fine-tuned narrative construals that can vary alot
even across small word-sense variance:
As we follow along with these sentences, we have to build a narrative
and situational picture which matches the speaker's intent,
sufficiently well.}

\p{And that requires prelinguistic background
knowledge which is leveraged and activated (but not mechanically
or logically constructed) by lexical, semantic, or grammatical
rules and forms: \i{rescued dogs} all alone constructs a fairly
detailed mental picture where we can fill in many details by
default, unless something in the discourse tells otherwise
(we can assume such dogs are in need of food, medical care,
shelter, etc., or they would not be described as
\q{rescued}).  Likewise \i{sing along} carries a rich mental
picture of a performer and an audience and how they interact, one
which we understand based on having attended concerts rather than
by any rule governing \i{along} as a modifier to \q{sing}
{\mdash} compare the effects of \i{along} in \i{walk along},
\i{ride along}, \i{play along}, \i{go along}.  Merely
by understanding how \i{along} modifies \i{walk}, say
(which is basically straightforward; to
\q{walk along} is basically to \q{walk alongside}) we
would not automatically generalize to more idiomatic
and metaphorical uses like \q{sing along} or \q{play along}
(as in \i{I was skeptical but I played along (so as not to
start an argument)}).}

\p{We have access to a robust collection of \q{mental scripts} which
represent hypothetical scenarios and social milieus where
language plays out.  Language can activate various such
\q{scripts} (and semantic as well as grammatical formations
try to ensure that the \q{right} scripts are selected).
Nonetheless, we can argue that the conceptual and cognitive
substance of the scripts comes not from language per se
but from our overall social and cultural lives.
We are disposed to make linguistic inferences {\mdash} like
the timeframes implied by \i{fed the rescued dogs} or the scopes
implied by \i{sang along to two songs} {\mdash} because of
our enculturated familiarity with events like dog rescues
(and dog rescue organizations) and concerts
(plus places like concert halls).  These concepts are not
produced by the English language, or even by any dialect
thereof (a fluent English speaker from a different
cultural background would not necessarily make the
same inferences {\mdash} and even if we restrict attention to,
say, American speakers, the commonality of disposition
reflects a commonality of the relevant cultural
anchors {\mdash} like dog rescues, and concerts {\mdash} rather than
any homogenizing effects of an \q{American} dialect).
For these reasons, I believe that trying to account for
situational particulars via formal language models alone
is a dead end.  This does not mean that formal language
models are unimportant, only that we need to picture them
resting on a fairly detailed prelinguistic
world-disclosure.}

\p{There are interesting parallels in this thesis to the role
of phenomenological analysis, and the direct thematization
of issues like attention and intentionality: analyses
which are truly \q{to the things themselves} should take for
granted the extensive subconscious reasoning that undergirds
what we consciously thematize and would be aware of, in terms of
what we deliberately focus on and are conscious of
believing (or not knowing), for a first-personal \expose{}.
Phenomenological analysis should not consider itself as
thematizing every small quale, every little patch of
color or haptic/kinasthetic sensation which by some subconscious
process feeds into the logical picture of our surroundings that
props up our conscious perception.  Analogously, linguistic
analysis should not thematize every conceptual and inferential
judgment that guides us when forming the mental, situational
pictures we then consult to set the groundwork for linguistic
understanding proper.}

\p{These comments apply to both conceptual \q{background knowledge} and
to situational particulars of which we are cognizant in
reference to our immediate surroundings and actions.  This
is the perceptual and operational surrounding that gets
linguistically embodied in deictic reference and other
contextual \q{groundings}.  Our situational awareness therefore
has both a conceptual aspect {\mdash} while attending a concert,
or dining at a restaurant, say, we exercise cultural background
knowledge to interpret and participate in social events
 {\mdash} and also our phenomenological construal of our locales,
our immediate spatial and physical surroundings.
Phenomenological philosophers have explored in detail how these
two facets of situationality interconnect (David Woodruff Smith and
Ronald McIntyre in \i{Husserl and Intentionality:
A Study of Mind, Meaning, and Language}, for instance).
Cognitive Linguistics covers similar territory; the \q{cognitive}
in Cognitive Semantics and Cognitive Grammar generally tends
to thematize the conception/perception interface and
how both aspects are merged in situational understanding
and situationally grounded linguistic activity (certainly
more than anything involving Artificial Intelligence or
Computational Models of Mind as are connoted by terms like
\q{Cognitive Computing}).  Phenomenological and Cognitive
Linguistic analyses of situationality and perceptual/conceptual
cognition (cognition as the mental synthesis of
perception and conceptualization) can certainly enhance and
reinforce each other.}

\p{But in addition, both point to a cognitive and situational
substratum underpinning both first-person
awareness and linguistic formalization proper {\mdash} in other words,
they point to the thematic limits of
phenomenology and Cognitive Grammar and
the analytic boundary where they give way to
an overarching Cognitive Science.  In the case of
phenomenology, there are cognitive structures that suffuse
consciousness without being directly objects of attention
or intention(ality), just as sensate hyletic experience is
part our consciousness but not, as explicit content,
something we in the general case are conscious \i{of}.
Analogously, conceptual and situational models
permeate our interpretations of linguistic forms, but
are not presented explicitly \i{through} these
forms: instead, they are solicited obliquely and
particularly.}

\p{What phenomenology \i{should} explicate is not background situational
cognition but how attention, sensate awareness, and intentionality
structure our orientation \i{ \visavis{}} this background: how variations
in focus and affective intensity play strategic roles in our engaged
interactions with the world around us.  Awareness is a scale, and
the more conscious we are of a sense-quality, an attentional focus,
or an epistemic attitude, reflects our estimation of the
importance of that explicit content compared to a muted experiential
background.  Hence when we describe consciousness as a stream
of \i{intentional} relations we mean not that the intended
noemata (whether perceived objects or abstract thoughts)
are sole objects of consciousness (even in the moment)
but are that within conscious totality which we are most aware
of, and our choice to direct attention here and there reflects our
intelligent, proactive interacting with the life-world.
Situational cognition forms the background,
and phenomenology addresses the structure of intentional
and attentional modulations constituting the conscious
foreground.}

\p{Analogously, the proper role for linguistic
analysis is to represent how multiple layers or strands
of prelinguistic understanding, or \q{scripts}, or
\q{mental spaces}, are woven
together by the compositional structures of language.
For instance, \i{The rescued dogs were treated by an experienced vet}
integrates two significantly different narrative frames
(and space-constructions, and so forth): the frame implied
by \q{rescued dogs} is distinct from that implied
by \q{treated by a veterinarian}.  Note that both spaces are
available for follow-up conversation:

\begin{sentenceList}

\sentenceItem{} \swl{}{The rescued dogs were treated by an experienced vet.
One needed surgery and one got a blood transfusion.  We went there
yesterday and both looked much better.}{sem}
\sentenceItem{} \swl{}{The rescued dogs were treated by an experienced vet.
One had been struck by a car and needed surgery on his leg.  We
went there yesterday and saw debris from another car crash
{\mdash} it's a dangerous stretch of highway.}{sem}
\end{sentenceList}

In the first sentence \i{there} designates the veterinary clinic, while in
the second it designates the rescue site.  Both of these locales are
involved in the original sentence (as locations and also
\q{spaces} with their own environments and configurations:
consider these final three examples).

\begin{sentenceList}

\sentenceItem{} \swl{}{The rescued dogs were treated by an experienced vet.
We saw a lot of other dogs getting medical attention.}{sem}
\sentenceItem{} \swl{}{The rescued dogs were treated by an experienced vet.
It looked very modern, like a human hospital.}{sem}
\sentenceItem{} \swl{}{The rescued dogs were treated by an experienced vet.
We looked around and realized how dangerous that road is {\mdash}
for humans as well as dogs.}{sem}
\end{sentenceList}
}

\p{What these double space-constructions reveal is that accurate
language understanding does not only require
the proper activated \q{scripts} accompanying words and
phrases, like \q{rescued dogs} and \q{treated by a vet}.
It also requires the correct integration of each script,
or each mental space, tieing them together in accord with
speaker intent.  So in the current example we should read that
the dogs \i{could} be taken to the vet \i{because} they were
rescued, and \i{needed} to be taken to the vet \i{because} their
needing rescued was associated with being injured or in poor 
health.  Language structures guide us
toward how we should tie the mental spaces, and the
language segments where they are constructed, together: the
phrase \q{ \i{rescued} dogs} becomes the subject of the passive-voice
\i{were treated by a vet} causing the two narrative strands of the
sentence to encounter one another, creating a hybrid space
(or perhaps more accurately a patterning between
two spaces with a particular temporal and causal
sequencing; a hybrid narration bridging the spaces).
It is of course this hybrid space, this narrative
recount, which the speaker intends via the sentence.  This
idea is what the sentence is crafted to convey {\mdash} not just
that the dogs were rescued, or that they were taken to a vet, but
that a causal and narrative thread links the two events.}

\p{I maintain, therefore, that the analyses which are proper to linguistics
{\mdash} highlighting what linguistic reasoning contributes above and beyond
background knowledge and situational cognition {\mdash} should focus on
the \i{integration} of multiple mental \q{scripts},
each triggered by different parts and properties of the linguistic
artifact.  The \i{triggers} themselves can be individual words, but
also morphological details (like plurals or tense marking) and
morphological agreement.  On this theory, analysis has two distinct
areas of concerns: identification of grammatical, lexical, and
morphosyntactic features which trigger (assumedly prelinguistic)
interpretive scripts; and reconstructing how these scripts
interoperate (and how language structure determines such integration).}

\p{In the case of isolating triggers, a wide range of linguistic features
can trigger interpretive reasoning {\mdash} including base lexical choice;
word-senses carry prototypical narrative and situational templates that
guide interpretation of how the word is used in any given context.
\i{Rescued}, for example, brings on board a network of likely
externalities: that there are rescuers, typically understood to be
benevolent and intending to protect the rescuees from harm; that
the rescuees are in danger prior to the rescue but safe afterward;
that they need the rescuers and could not have reached safety themselves.
Anyone using the word \q{rescue} anticipates that their addressees will
reason through some such interpretive frame, so the speaker's role is
to fill in the details descriptively or deictically: who are the rescuees
and why they are in danger; who are the rescuers and why they are benevolent
and able to protect the rescuees.  The claim that
the word \i{rescue}, by virtue of its lexical properties, triggers an
interpretive \q{script}, is a proposal to the effect that when trying
to faithfully reconstruct speaker intentions
we will try to match the interpretive
frame to the current situation.}

\p{The \q{script} triggered by word-choice is not just an interpretive
frame in the abstract, but the interpretive \i{process} that matches
the frame to the situation.  This process can be exploited for
metaphorical and figurative effect, broadening the semantic scope
of the underlying lexeme.  In the case of \q{rescue} we have less
literal and more humorous or idiomatic examples like:

\begin{sentenceList}

\sentenceItem{} \hspace{-3pt}\swl{}{The trade rescued a star athlete from a losing team.}{idi}
\sentenceItem{} \swl{}{New mathematical models rescued her original research from obscurity.}{idi}
\sentenceItem{} \hspace{-3pt}\swl{}{Discovery of nearby earth-like planets rescued that
star from its reputation as ordinary and boring and revealed that its solar
system may actually be extraordinary.}{idi}
\end{sentenceList}

Each of these cases subverts the conventional \q{rescue} script by
varying some of the prototypical frame details:  maybe the
\q{danger} faced by the rescuee is actually trivial (as in the
first three), or the rescuee is not a living thing
whose state we'd normally qualify in terms of \q{danger} or \q{safety},
or by overturning the benevolence we typically attribute to
rescue events.  
But in these uses subverting the familiar script does not
weaken the lexical merit of the word choice; instead, the interpretive
act of matching the conventional \q{rescue} script to the matter at hand
reveals details and opinions that the speaker wishes to convey.  The
first sentence, for instance, uses \q{rescue} to connote
that being stuck on a losing team is an unpleasant (even if not life-threatening)
circumstance.  So one part of the frame (that the rescuee needs
outside intervention) holds while the other (that
the rescue is in danger) comes across as excessive but
(by this very hyperbole) communicating speaker sentiment.  By
both invoking the \q{rescue} script and exploiting mismatches between
its template case and the current context, the speaker
conveys both situational facts and personal opinions quite
economically.  Similarly, \i{rescue a paper from obscurity} is
an economical way of saying that research work has been rediscovered
in light of new science; and \q{rescued from a reputation} is a 
clever way of describing, with rhetorical force, 
how opinions have changed about someone or something.}

\p{All of these interpretive effects {\mdash} both conventional
and unconventional usages {\mdash} stem from the interpretive
scripts bound to words (and triggered by word-choice) at
the underlying lexical level {\mdash} we can assess these by reference
to lexical details alone, setting aside syntactic and morphological
qualities.  When morphosyntactic details \i{are} considered 
{\mdash} e.g. plurals, as in (\ref{itm:coffeeshirt})-(\ref{itm:coffeetable}) 
{\mdash} we then have a spectrum of other linguistic \q{triggers}, 
involving perceptual and enactive figurations (e.g. how 
plurality/multiplicities are conceived), alongside interpretive
\q{scripts}.  My essential point however is that language 
needs to trigger a collection of interpretive, perceptual, and 
enactive/operational processes, for complete understanding; 
and moreover that that language structures need to signal 
the \q{routing} of information between such processes, 
establishing patterns of priority/sequencing and 
information-supplementation among them.  These patterns, 
I believe, can be modeled to some approximation via 
computational type and process theories.}

\p{I contend, moreover, that these cognitive processes are not
\i{themselves} linguistic: while they may overlap with 
some language-relevant concerns (like conceptualization, and 
doxic specificity) they are not woven from the cloth of 
syntactic, semantic, or pragmatic elements internal to language.  
It is not within the purview of linguistics then to analyze 
interpretive scripts (except as a subsidiary case-study), or 
perceptual understanding, or situationally-mediated action.  
What \i{can} be left for linguistics proper is identifying the
\i{triggers} to these cognitive realities {\mdash} insofar as content or 
formations in language, within our goal-directed attempt to 
understand others' linguistic performances, compels us toward 
these extra-linguistic registers.}

\p{Linguistics on this perspective is necessarily and properly incomplete: 
we should not look to linguistic analysis to materially or structurally 
explain the cognitive processes triggered by language.  But we \i{can}
analyze the triggers themselves {\mdash} potentially via formal and 
even computational methods.  So the formal models I reviewed 
earlier {\mdash} such as the combination of Dependency graphs 
with typed S-Expressions encoded via double-indices {\mdash} can be adopted 
in a basically non-formalized, cognitive-linguistic paradigm, 
insofar as we ascribe to the global picture of language as an
\q{interface} or \q{trigger} to extralinguistic cognition.}

\thindecoline{}

\p{Suppose, for instance, that morphosyntax confirms the 
presence of a pluralizing operation, like
\i{New Yorkers}.  Against this minimal linguistic
\q{information}, we need
to invoke narrative frames, interpretive scripts, and prelinguistic
background knowledge to understand what \i{sort} of plurality
the speaker intends.  The phrase \i{New Yorkers} can refer 
to those who live in New York City (\ref{itm:NYboroughs}, \ref{itm:boroughs}); 
in the New York 
are but \i{outside} the city (\ref{itm:NYcommute}, \ref{itm:NYcommute}); 
or some combination.  
It can also mean \i{all} (\ref{itm:NYboroughs}, \ref{itm:boroughs}) or 
just \i{some} New Yorkers (\ref{itm:NYcommute}, \ref{itm:NYcommute}, \ref{itm:NYDem}).
Usually these alternatives are not explicitly 
spelled out, and have to be inferred from context 
and via background knowledge about which readings 
are plausible.    
As I argued from multiple angles earlier,
in the typical case {\mdash} i.e., stylistically
neutral, day-to-day language {\mdash} syntactic
composition does not neatly recapitulate logical form.}

\p{My prior analysis demonstrated warrants for this idea by
highlighting narrative and imagistic aspects
of language used to convey ideas, like \q{come out against}
providing the verb-phrases in reports of people
criticizing something.  The \q{New York} examples like
(\ref{itm:NYboroughs})-(\ref{itm:NYcommute}) point to a similar
conclusion, but from a more lexico-semantic orientation:
words like \i{borough} and \i{commute} carry a space
of logical details that tend to force logical
interpretations one way or another
(e.g., the detail that the territory of a city
is fully partitioned by boroughs, \i{so}, it
is \i{all} citizens who live in a borough).  This part of the
logic is however not reflected in sentence-structure; it is,
rather, latent
in lexical norms and assumed part of understanding
relevant sentences only because linguistic
competence is understood to include
familiarity with the logical
implications of the lexicalized concepts:
e.g. that the quantification in \q{New Yorkers
live in one of five boroughs} is \i{all},
but the quantification in \q{New Yorkers
vote democratic} is \i{most}.}

\p{Here I'll also add the
following: the current examples show
how if addressees \i{have} the requisite background knowledge,
linguistic structure does not have to replicate logical structure very
closely to be understood.  The content which addressees understand may
have a logical form, and language evokes this form {\mdash} guides
addressees toward considering specific propositional content {\mdash} but
this does not happen because linguistic structure in any precise way mimics,
replicates, reconstructs, or is otherwise organized propositionally.  Instead,
the relation of language to predicate structures is evidently
oblique and indirect: language triggers interpretive processes
which guide us toward propositional content, but the structure
of language is shaped around fine-tuning the activation of
this background cognitive dynamics more than around any need to model
predicate organization architecturally.   In the case of plurals,
the appearance of plural forms like \i{New Yorkers} or \i{coffees}
compels us to find a reasonable cognitive model for the
signified multitude, and this model will have a logical form
{\mdash} but the linguistic structures themselves do not in general
model this form for us, except to the limited degree needed to
activate prelinguistic interpretive thought-processes.}

\p{I make this point in terms of plural \i{forms}, and earlier made
similar claims in terms of lexical details (cf. \i{(live in a) borough}
vs. \i{vote (Democratic)}.  A third group of triggers I
outlined involved morphosyntactic \i{agreement}, which establishes
inter-word connections that themselves trigger interpretive processing.
Continuing the topic of plurals, how words agree with other words
in singular or plural forms evokes schema which guide situational
interpretations.  So for instance:

\begin{sentenceList}

\sentenceItem{} \swl{}{My favorite band gave a free concert last night.
They played some new songs.}{ref}
\sentenceItem{} \swl{}{There was some pizza earlier, but it's all gone.}{ref}
\sentenceItem{} \swl{}{There were some slices of pizza earlier, but it's all gone.}{ref}
\sentenceItem{} \swl{}{There were some slices of toast earlier, but there's none left.}{ref}
\sentenceItem{} \swl{}{There was some toast earlier, but they're all gone.}{ref}
\sentenceItem{} \swl{}{That franchise had a core of talented young players, but it
got eroded by trades and free agency.}{ref}
\sentenceItem{} \swl{}{That franchise had a cohort of talented players, but they
drifted away due to trades and free agency.}{ref}
\sentenceItem{} \swl{}{Many star players were drafted by that franchise, but it
has not won a title in decades.}{ref}
\sentenceItem{} \swl{}{Many star players were drafted by that franchise, but they
failed to surround them with enough depth.}{ref}
\sentenceItem{} \swl{}{Many star players were drafted by that franchise, but they
were not surrounded with enough depth.}{ref}
\sentenceItem{} \swl{}{Many star players were drafted by that franchise, but they
did not have enough depth (around them).}{ref}
\end{sentenceList}

Plurality here is introduced not only by isolated morphology (like \i{slices},
\i{players}, \i{songs}), but via agreements marked by
word-forms in syntactically significant pairings: was/were, it/they,
there is/there are.  Framing all of these cases is how we can usually schematize
collections both plurally and singularly: the same set can be cognized as a
collection of discrete individuals one moment and as an integral whole the next.
This allows language some flexibility when designating plurals
(cf. \i{Three times, students asked an interesting question}).  A sentence
discussing \i{slices of pizza} can schematically shift to treating
the pizza as a mass aggregate in \i{it's all gone}.  Here the antecedent
of \i{it} is \i{slices} (of pizza).  In the opposite direction, the mass-plural
\q{toast} can be refigured as a set of individual pieces in \i{they're all gone}.
The single \i{band} becomes the group of musicians in the band.
In short, how agreements are executed invites the addressee to reconstruct
the speaker's conceptualization of different referents discussed by
a sentence, at different parts of the sentence: linking
\i{it} to \i{slices} or \i{cohort}, or \i{they} to \i{the band} or
\i{the toast}, evokes a conceptual interpretation shaped in
part by how morphosyntactic agreement overlaps with \q{semantic}
agreement.  Matching \i{they} to \i{the band} presents agreement
in terms of how we conceive the aggregate (as a collection of
musicians); using \q{it} would also present an agreement, but
one schematizing other aspect of the band concept.}

\p{In the last five above cases, \i{it} similarly binds (being singular) to
\i{the franchise} seen as a single unit {\mdash} here basic grammar and
conceptual schema coincide {\mdash} but \i{it} also binds
to the \i{core of young players}.  The
players on a team can be figured as a unit or a multiple.  The franchise itself
can be treated as a multiple (the various team executives and decision-makers),
as in \i{they failed to surround the stars with enough depth}.  The last sentence
is ambiguous between both readings: \q{they} could designate either the players
or the franchise.  Which reading we hear alters the sense of \q{have}: asserting
that the star \i{players} lack enough depth implies that they cannot execute
plays during the game as effectively as with better supporting players;
asserting that the \i{franchise} lacks depth makes the subtly different point
that there is not enough talent over all.  The variant which would include
\q{around them} nudges toward the second reading; but it is
still permissible to, according to speaker intent, parse the last
\i{they} as designating the \i{franchise} and the last \i{them} as
the \i{players} {\mdash} i.e., that final \i{they}/\i{them} pair
having different antecedents.}

\p{The unifying theme across these cases is that when forming sentences we often
have a choice of how we figure plurality, and moreover these choices can be
expressed not only in individual word-forms but in patterns of agreement.
Choosing to pronominalize \i{slices of pizza} or \i{cohort of players}
as \i{it}, or alternatively \i{they}, draws attention to either the more
singular or more multitudinal aspects of the aggregate in question.
But this effect is not localized to the individual
\i{it}/\i{they} choice; it depends on
tracing the pronoun to its antecedent and construing
how the antecedent referent has both individuating and
multiplicity-like aspects.
Thus both individuation and plurality are latent in phrases like
\i{slices of} or \i{cohort of}, and this singular/plural co-conceiving
is antecedently figured by how subsequent morphosyntax agrees
with the singular or, alternatively, the plural.}

\p{Moreover, these patterns of agreement invoke new layers of
interpretation to identify the proper conceptual scope of plurals.
In \i{The band planned a tour, where they debuted new songs} we hear
the scope of \q{they} as narrower than its antecedent \q{the band},
because only the band's \i{musicians} (not stage crew, managers, etc.)
typically actually perform:

\begin{sentenceList}

\sentenceItem{} \swl{tour}{The band planned a tour, where
they debuted new songs.}{ref}
\sentenceItem{} \swl{itm:teamflew}{The team flew to
New York and they played the Yankees.}{ref}
\sentenceItem{} \swl{itm:theflies}{The city's largest theater company
will perform \q{The Flies}.}{ref}
\end{sentenceList}

Likewise in (\ref{itm:teamflew}), only the athletes are referenced via
\i{they played}; but presumably many other people (trainers, coaches, staff)
are encompassed by \q{the team flew}.  And in (\ref{itm:theflies})
we do not imply that the Board of Directors will
actually take the stage (the President
as Zeus, say).  Even in the course of one sentence, plurals
are reinterpreted and redirected:

\begin{sentenceList}

\sentenceItem{} The city's largest theater company
performed \q{The Flies} in French, but everyone's accent
sounded Quebecois.
\sentenceItem{} The city's largest theater company
performed \q{The Flies}; then they invited a professor
to discuss Sartre's philosophy when the play was over.
\end{sentenceList}

In the first sentence, the \q{space} built by the sentence is wider
initially but narrows to encompass only the actual actors on stage.
In the second, the \q{space} narrows in a different direction, since
we hear a programming decision like pairing a performance with a
lecture as made by a theater's administrators rather than its actors.
I discussed similar modulation in conceptual schemas related to plurality
and pluralization earlier; what is distinct in these last examples is how
the interpretive processes for cognizing plurality are shaped by
agreement-patterns (like \i{it} or \i{they} to a
composite antecedent) as
much as by lexical choice, or morphology, in isolation.}

\p{I have accordingly outlined a theory where lexical, morphological, and
morphosyntactic layers all introduce \q{triggers} for cognitive processes,
and it is these processes which (via substantially prelinguistic perception
and conceptualization) ultimately deliver linguistic meaning.  What is
\i{linguistic} about these phenomena is how specifically linguistic
formations {\mdash} word choice, word forms, inter-word agreements in form {\mdash}
trigger these (in no small measure pre- or extra-linguistic) interpretations.
But as I suggested this account is only preliminary to analysis of
how multiple interpretive processes are \i{integrated}.  Linguistic
\i{structure} contributes the arrangements through which the
crossing and intersecting between interpretive
\q{scripts} are orchestrated.  Hence at the higher linguistic scales and
levels of complexity, the substance of linguistic research, on this view,
should gravitate toward structural integration of interpretive
processes, even more than individual interpretive triggers themselves.}

\p{I have thus far emphasized types' structural role, on the 
premise that type theory offers technical models that can 
be applied to the study of \q{cognitive process integration}.  
Requirements on the transfer of information between 
processes {\mdash} how processes complement one another and 
take other's details as givens {\mdash} are suggestive analogies 
to latent interactions between mental \q{scripts} we are 
poised to apply to situations; and these interactions 
are the essential cognitive targets which linguistic structures 
have to trigger by semantic/lexical, and syntactic/morphosyntactic 
means.  The point of analyzing process preconditions 
type-theoretically is to leverage types as constructional  
media for structural theories of inter-process integration.}

\p{This approach is largely syntactic: while many lexical and semantic 
details may be analytically relevant, the style of analysis I 
just proposed is ultimately a form of grammar, identifying 
how organizational patterns among language-elements 
trigger (and are thereby explained by the intention of) 
corresponding patterns among cognitive processes.  
For sake of completeness, though, I will conclude by 
offering a few thoughts on the more \q{semantic} manifestation 
of types (and type theory) as premises of 
organizing concepts.}

\subsection{Types, Sets, and Concepts}
\p{In formal/computational contexts, types can be defined as sets of both values and
\q{expectations} \cite{MathieuBouchard} (meaning assumptions which may be made about
all values covered by the type); alternatively, we can (perhaps better) consider types as
\i{spaces} of values.  Types' extensions have internal structure; there
can be \q{null} or \q{invalid} values, default-constructed values, and
so forth, which are \q{regions} of the conceptual space spanned or 
encompassing types.  
There is definitional interdependence
between types and functions: a function is defined in terms of the types it accepts as parameters and
returns {\mdash} rather than its entire set of possible inputs and outputs, which can
vary across computing environments.\footnote{Moreover, expectations in a particular case
may be more precise than what is implied by the type itself {\mdash} it is erroneous
to assume that a proper type system will allow a correct \q{set of values} to
be stipulated for each point in a computation (the kind of contract enforced via
by documentation and unit testing).  So state-space in a given context may include many
\q{unreasonable} values, implying that within the overall space there is a \q{reasonable}
subspace, except that this subspace may not be crisply defined.} These are some reasons why in theoretical
Computer Science types are not \q{reduced} to underlying sets; instead, extensions
are sometimes complex spaces that model states of, or internal organization of comparisons
among, type instances.}

\p{Moreover, despite mathematical set theory's influence on 
formal logic (and philosophy, by extension), 
philosophers have long been troubled by 
the seeming arbitrariness and minimal structure of 
the essential notion of \i{sets} themselves,
and have explored other constructs to play 
analogous intellectually foundational roles, 
such as Michael Jubien's Property Theory
\cite{MichaelJubien}
or Jiri Benovsky \q{modal perdurants} \cite[p. 8]{JiriBenovsky}
(here citing two treatments I find particularly 
persuasive or thought-provoking).  
This is one \i{philosophical} interest in 
type theory as types have more formal structure 
and also, in a sense, more metaphysical coherence 
than \q{sets} (in the classical sense allowing 
set-composition from unrelated members).  
In a certain intellectual role, however, types are 
still somehow \q{set-like} in that we can 
consider the totality of values that may inhabit 
each type, and we need some theory or model 
of that totality {\mdash} which in turn engenders 
different versions of type theory.}

\p{An obvious paradigm is organizing type-extensions around prototype/borderline
cases {\mdash} there are instances which are clear examples of types and
ones whose classification is dubious.  I contend, however, 
that common resemblance is not always a good marker
for types being well-conceived {\mdash} many useful concepts are common
precisely because they cover many cases, which makes defining
\q{prototypes} or \q{common properties} misleading.  
Also, sometimes the clearest
\q{representative} example of a type or concept is actually not a
\i{typical} example: a sample latter or model home is actually not (in
many cases) a real letter or home.  So resemblance-to-prototype is at
best one kind of \q{inner organization} of concepts' and types' spaces
of extension.}

\p{Sets, concepts, and types represent three different primordial thought-vehicles for
grounding notions of logic and meaning.  To organize systems around \i{sets} is
to forefront notions of inclusion, exclusion, extension, and intersection,
which are also formally essential to mathematical logic and undergird the
classical interdependence of sets, logic, and mathematics.
To organize systems around \i{concepts} is to forefront practical engagement
and how we mold conceptual profiles, as collections of ideas and pragmas,
to empirical situations.  To organize systems around \i{types} is to forefront
\q{functions} or transformations which operate on typed values, the interrelationships
between different types (like subtypes and inclusion {\mdash} a type can itself
encompass multiple values of other types), and the conceptual abstraction
of types themselves from the actual sets of values they may exhibit
in different environments.  Sets and types are
formal, abstract phenomena; whereas concepts are characterized by
gradations of applicability, and play flexible roles in thought and language.
The cognitive role of concepts can be discussed with some rigor, but there is a
complex interplay of cognitive schema and practical engagements which
would have to be meticulously sketched in many real-world scenarios, if
our goal were to translate conceptual reasoning to formal structures
on a case-by-case basis.  We can, however, consider in general
terms how type-theoretic semantics can capture conceptual structures
as part of the overall transitioning of thoughts to language.}

\p{A concept does not merely package up a definition, like \q{restaurant} as
\q{a place to order food}; instead concepts link up with other concepts
as tools for describing and participating in situations.  Concepts are
associated with \q{scripts} of discourse and action, and find their
range of application through a variegated pragmatic scope.
We should be careful not to overlook these pragmatics, and
assume that conceptual structures can be simplistically
translated to formal models.
Cognitive Linguistics critiques
Set-Theoretic or Modal Logic reductionism (where a concept is just a set
of instances, or an extension across different possible worlds) {\mdash} George Lakoff and Mark Johnson,
prominently, argue for concepts' organization around
prototypes (\cite[p. 18]{LakoffJohnson} ; \cite[p. 171, or p. \textit{xi}]{Johnson})
and embodied/enactive patterns of interaction (\cite[p. 90]{LakoffJohnson} ;
\cite[p. 208]{Johnson}).

Types, by contrast, at least in linguistic applications of type theory, are abstractions
defined in large part by quasi-functional notions of phrase structure.
Nevertheless, the \i{patterns} of how types may inter-relate
(mass-noun or count-noun, sentient or non-sentient, and so forth)
provide an infrastructure for conceptual understandings to be
encoded in language {\mdash} specifically, to be signaled by which typed
articulations conversants choose to use.  A concept like
\i{restaurant} enters language with a collection of understood
qualities (social phenomena, with some notion of spatial location and
being a \q{place}, etc.) that in turn can be marshaled by sets of
allowed or disallowed phrasal combinations, whose parameters
can be given type-like descriptions.  Types, in this sense,
are not direct expressions of concepts but vehicles for
introducing concepts into language.}

\p{Concepts (and types also) are not cognitively the same as their
extension {\mdash} the concept \i{restaurant}, I believe, is distinct from
concepts like \i{all restaurants} or \i{the set of all restaurants}.
This is for several reasons.  First, concepts can be pairwise different
not only through their instances, but because they highlight different
sets of attributes or indicators.  The concepts \q{American President} and \q{Commander in Chief}
refer to the same person, but the latter foregrounds a military role.
Formal Concept Analysis considers \i{extensions} and \q{properties}
{\mdash} suggestive indicators that inhere in each
instance {\mdash} as jointly (and co-dependently) determinate: concepts
are formally a synthesis of instance-sets and property-sets \cite{YiyuYao},
\cite{Belohlavek}, \cite{Wille}.  Second,
in language, clear evidence for the contrast between \i{intension} and
\i{extension} comes from phrase structure: certain constructions specifically
refer to concept-extension, triggering a mental shift from thinking of the
concept as a schema or prototype to thinking of its extension (maybe in some context).
Compare:

\begin{sentenceList}

\sentenceItem{} \swl{itm:rhinor}{Rhinos in that park are threatened by poachers.}{sem}
\sentenceItem{} \swl{}{Young rhinos are threatened by poachers.}{sem}
\end{sentenceList}

Both sentences focus a conceptual lens in greater detail than \i{rhino} in general, but
the second does so more intensionally, by adding an extra indicative criterion; while
the former does so extensionally, using a phrase-structure designed to operate on
and narrow our mental construal of \q{the set of all rhinos}, in the sense of
\i{existing} rhinos, their physical place and habitat, as opposed to
the \q{abstract} (or \q{universal}) type.  So there is a familiar semantic
pattern which mentally transitions from a lexical type to its extension and
then extension-narrowing {\mdash} an interpretation that, if accepted, clearly
shows a different mental role for concepts of concepts' \i{extension} than the
concepts themselves.\footnote{There is a type-theoretic correspondence between intension and
extension {\mdash} for a type \Tnoindex{} there is a corresponding \q{higher-order} type
of \i{sets} whose members are \Tnoindex{}
(related constructions are the type of \i{ordered sequences} of \Tnoindex{};
unordered collections of \Tnoindex{} allowing repetition; and stacks, queues, and
deques {\mdash} double-ended queues {\mdash} as \Tnoindex{}-lists that can grow or shrink
at their beginning and/or end).  If we take this (higher-order)
type gloss seriously, the extension of a concept is not its \i{meaning}, but a
different, albeit interrelated concept.  Extension is not definition.
\i{Rhino} does not mean \i{all rhinos} (or \i{all possible rhinos}) {\mdash} though arguably
there are concepts \i{all rhinos} and \i{all restaurants} (etc.) along with the concepts
\i{rhino} and \i{restaurant}.}
}

\p{Concepts, in short, do not mentally signify sets, or
extensions, or sets-of-shared-properties.  Concepts, rather, are cognitive/dialogic tools.
Each concept-choice, as presentation device,
invites its own follow-up. \i{Restaurant} or \i{house} have meaning not via
idealized mental pictures, or proto-schema, but via kinds of things
we do (eat, live), of conversations we have, of qualities we deem relevant.  Concepts do not
have to paint a complete picture, because we use them as part of ongoing situations
{\mdash} in language, ongoing conversations.  Narrow concepts {\mdash} which may best exemplify
\q{logical} models of concepts as resemblance-spaces or as rigid designators to
natural kinds {\mdash} have, in practice, fewer use-cases \i{because} there
are fewer chances for elaboration.  Very broad concepts, on the other hand, can have,
in context, too \i{little} built-in \i{a priori} detail.
(We say \q{restaurant} more often than \i{eatery}, and
more often than \i{diner}, \i{steakhouse}, or \i{taqueria}).  Concepts dynamically play
against each other, making \q{spaces} where different niches of meaning, including
levels of precision, converge as site for one or another.  Speakers need freedom to choose
finer or coarser grain, so concepts are profligate, but the most oft-used trend toward middle
ground, neither too narrow nor too broad. \i{Restaurant} or \i{house} are useful because they are noncommittal, inviting more detail.
These dynamics govern the flow of inter-concept relations (disjointness, subtypes, partonymy, etc.).}

\p{Concepts are not rigid formulae (like instance-sets or even attributes fixing when
they apply); they are mental gadgets to initiate and
guide dialog.  Importantly, this
contradicts the idea that concepts are unified around instances' similarity (to each other or
to some hypothetical prototype): concepts have avenues for contrasting
different examples, invoking a \q{script} for further elaboration, or for building temporary filters.  
In, say,

\begin{sentenceList}

\sentenceItem{} \swl{}{Let's find a restaurant that's family-friendly.}{sem}
\end{sentenceList}

allowing such one-off narrowing is a feature
of the concept's flexibility.}

\p{In essence: no less important, than acknowledged similarities across all instances, are well-rehearsed ways
\visavis{} each concept to narrow scope by marshaling lines of \i{contrast}, of \i{dissimilarity}.
A \i{house} is obviously different from a \i{skyscraper}
or a \i{tent}, and better resembles other houses; but there are also more nontrivial \i{comparisons}
between houses, than between a house and a skyscraper
or a tent.  Concepts are not only spaces of similarity, but of \i{meaningful kinds of differences}.}

\p{To this account of conceptual breadth we can add the conceptual matrix spanned by
various (maybe overlapping) word-senses: to \i{fly}, for example, names
not a single concept, but a family of concepts all related to airborn
travel.  Variations highlight different features: the path of flight (\i{fly to Korea}, \i{fly over the mountain});
the means (\i{fly Korean air}, \i{that model flew during World War II});
the cause (\i{sent flying (by an explosion)}, \i{the bird flew away (after a loud noise)},
\i{leaves flying in the wind}).  Words allow different use-contexts
to the degree that their various \i{senses} offer an inventory of aspects for
highlighting by \i{morphosyntactic} convention.  Someone who says \i{I hate to fly} is not
heard to dislike hand-gliding or jumping off mountains.\footnote{People, unlike birds, do not fly {\mdash} so the verb, used intransitively
(not flying \i{to} somewhere in particular or \i{in} something in particular),
is understood to refer less to the physical motion and more to the socially
sanctioned phenomenon of buying a seat on a scheduled flight on an airplane. The construction
highlights the procedural and commercial dimension, not the physical mechanism and
spatial path.  But it does so \i{because} we know human flight is
unnatural: we can poetically describe how the sky is filled with flying leaves or birds,
but not \q{flying people}, even if we are nearby an airport.} Accordant variations
of cognitive construal (attending more to mode of action, or path, or motives, etc.),
which are elsewhere signaled by grammatic choices, are also spanned by a conceptual
space innate to a given word: senses are finer-grained meanings availing themselves to one construal or another.}

\p{So situational construals can be signaled by word- and/or
syntactic form choice (locative, benefactive, direct and indirect
object constructions, and so forth).  Whereas conceptual organization
often functions by establishing classifications, and/or invoking
\q{scripts} of dialogic elaboration, cognitive structure tends to apply more
to our attention focusing on particular objects, sets of objects, events, or
aspects of events or situations.  
So the contrast between singular, mass-multiples, and count-multiples,
among nouns, depends on cognitive
construal of the behavior of the referent in question (if singular, its
propensity to act or be conceived as an integral whole; if multiple, its
disposition to either be divisible into discrete units, or not).
Or, events can be construed in terms of their causes
(their conditions at the outset), or their goals (their conditions at
the conclusion), or their means (their conditions in the interim).
Compare \i{attaching} something to a wall (means-focused) to
\i{hanging} something on a wall (ends-focused); \i{baking} a cake
(cause-focus: putting a cake in the oven with deliberate intent to cook it)
to \i{burning} a cake (accidentally overcooking it).\footnote{We can express
an intent to bake someone a cake, but not (well, maybe comedically) to
\i{burn} someone a cake (\q{burn}, at least in this context, implies
something not intended); however, we \i{can} say
\q{I burnt your cake}, while it is a little jarring to say
\q{I baked your cake} {\mdash} the possessive implies that some
specific cake is being talked about, and there is less apparent reason
to focus on one particular stage of its preparation (the baking) once
it is done.  I \i{will} bake a cake, in the future, uses
\q{bake} to mean also other steps in preparation (like \q{make}), while,
in the present, \q{the cake \i{is} baking} emphasizes more its
actual time in the oven.  I \i{baked your cake} seems to focus
(rather unexpectedly) on this specific stage even after it is completed,
whereas \i{I baked you a cake}, which is worded as if the recipient
did not know about the cake ahead of time, apparently uses \q{bake} in
the broader sense of \q{made}, not just \q{cooked in an oven}.
Words' senses mutate in relation to the kinds of situations where they are used
{\mdash} why else would \i{bake} mean \q{make}/\q{prepare} in the past or future tense but
\q{cook}/\q{heat} in the present?}
These variations are not random assortments of polysemous words' senses:
they are, instead, rather predictably distributed according
to speakers' context-specific knowledge and motives.}

\p{I claim therefore that \i{concepts} enter language complexly, influenced by
conceptual \i{spaces} and multi-dimensional semantic and syntactic selection-spaces.
Concepts are not simplistically \q{encoded} by types, as if for
each concept there is a linguistic or lexical type that just
disquotationally references it {\mdash} that the type \q{rhino} means the concept
\i{rhino} (\q{type} in the sense that type-theoretic semantics would model lexical
data according to type-theoretic rules, such as \i{rhino} as subtype of \i{animal} or
\i{living thing}).
Cognitive schema, at least in the terms I just laid out, select particularly
important gestalt principles (force dynamics, spatial frames, action-intention)
and isolate these from a conceptual matrix.  On this basis, we can argue that
these schemata form a precondition for concept-to-type association; or,
in the opposite logical direction, that language users' choices to employ
particular type articulations follow forth from their prelinguistic
cognizing of practical scenarios as this emerges out of collections
of concepts used to form a basic understanding of and self-positioning within them.}

\p{In this sense I called types \q{vehicles} for concepts: not that types \i{denote}
concepts but that they (metaphorically) \q{carry} concepts into language.
\q{Carrying} is enabled by types' semi-formal rule-bound
interactions with other types, which are positioned to capture concepts' variations and
relations with other concepts.  
This is the relationship I would propose for integrating 
theories of concepts {\mdash} as cognitive and lexicosemantic 
phenomena {\mdash} with applications of type theory which, 
as I argued above, are fundamentally syntactic: 
types as structural invariants patterning the integration 
between processes operative in language understanding.}

\p{To express a noun in the benefactive case, for example, which can be seen as attributing to
it a linguistic type consistent with being the target of a benefactive,
is to capture the concept in a type-theoretic gloss.
It tells us, I'm thinking about this thing in such a way that it
\i{can} take a benefactive (the type formalism attempting to capture
that \q{such a way}).
A concept-to-type \q{map}, as I just
suggested, is mediated (in experience and practical reasoning) by
cognitive organizations; when (social, embodied) enactions take
linguistic form, these organizing principles can be encoded in how
speakers apply morphosyntactic rules.}

\p{So the linguistic structures,
which I propose can be formally modeled by a kind of type theory, work
communicatively as carriers and thereby signifiers of cognitive
attitudes. The type is a vehicle for the concept because it takes part in constructions
which express conceptual details {\mdash} the details
don't emerge merely by virtue of the type itself.
I am not arguing for a neat concept-to-type correspondence; instead, a type system provides a
\q{formal substrate} that models (with some abstraction and simplification) how
properties of individual concepts translate
(via cognitive-schematic intermediaries) to their
manifestation in both semantics and syntax.}

\p{Continuing with declension as a case study,
consider how an \q{ontology} of word senses 
can interrelate with the benefactive.
A noun as a benefactive target most often is a person or some other
sentient/animate being; an inanimate benefactive is most likely
something artificial and constructed (cf., \i{I got the car new tires}).
How readily hearers accept a sentence {\mdash} and the path they
take to construing its meaning so as to make it grammatically acceptable
{\mdash} involves interlocking morphological and type-related considerations;
in the current example, the mixture of benefactive case and which noun
\q{type} (assuming a basic division of nouns into e.g.
animate/constructed/natural) forces a broader or narrower
interpretation.  A benefactive with an \q{artifact} noun, for example,
almost forces the thing to be heard as somehow disrepaired:

\begin{sentenceList}

\sentenceItem{} \swl{}{I got glue for your daughter.}{sem}
\sentenceItem{} \swl{}{I got glue for your coffee mug.}{sem}
\end{sentenceList}

We gather (in the second case) that the mug is broken {\mdash} but this is never spelled out
by any lexical choice; it is implied indirectly by using benefactive case.  
It is easy to design similar examples with other cases:
a locative construction rarely targets \q{sentient} nouns, so in 
(reprising (\ref{itm:grandma})-(\ref{itm:press}))

\begin{sentenceList}

\sentenceItem{} \swl{}{We're going to Grandma!}{sem}
\sentenceItem{} \swl{}{Let's go to him right now.}{sem}
\sentenceItem{} \swl{}{Let's go to the lawyers.}{sem}
\sentenceItem{} \swl{}{Let's go to the press.}{sem}
\end{sentenceList}

we mentally substitute the person with the place where they live or work.}

\p{Morphosyntactic
considerations are also at play: \i{to the lawyers} makes \q{go} sound more like \q{consult with},
partly because of the definite article (\i{the} lawyers implies conversants have some prior involvement
with specific lawyers or else are using the phrase metonymically, as in \q{go to court} or
\q{to the courts}, for
legal institutions generally; either reading draws attention away from literal spatial implications of
\q{go}). \i{Go to him} implies that \q{he} needs
some kind of help, because if the speaker just meant going to wherever he's at, she probably would
have said that instead.}

\p{Similarly, the locative in \i{to the press} forces the mind to
reconfigure the landmark/trajector structure, where \i{going} is thought not as a literal
spatial path and \i{press} not a literal destination {\mdash} in other words, the phrase must be
read as a metaphor.  But the \q{metaphor} here is not \q{idiomatic} or removed from linguistic rules
(based on mental resemblance, not language structure); here it
clearly works off of formal language patterns: the landmark/trajector
relation is read abstracted from literal spatial movement because the locative is applied
to an expression (\i{the press}) which does not (simplistically) meet
the expected interpretation as \q{designation of place}.
In short, there are two different levels of \i{granularity} where we 
can look for agreement requirements: a more fine-grained level where e.g.
\i{locative} draws in a type-specification of a \i{place} or \i{location}; 
and a coarser level oriented toward Parts of Speech, and typologies 
of phrasal units.  The former analysis addresses the level I have called
\q{macrotypes}, while the latter scale is at the \q{macrotype} level.}

\p{I envision the unfolding that I have just sketched out as something Phenomenological
{\mdash} it arises from a unified and subjective consciousness, one marked by
embodied personal identity and social situation.  If there are structural stases
that can be found in this temporality of experience, these are not constitutive
of conscious reality but a mesh of rationality that supports it, like the veins in
a leaf.  Stuctural configurations can be lifted from language insofar as it is a
conscious, formally governed activity, and lifted from the ambient situations which
lend language context and meaning intents.  So any analytic emphasis on
structural fixpoints threaded through the lived temporality of consciousness is an
abstraction, but one that is deliberate and necessary if we want to make scientific
or in any other manner disputatable claims about how language and congition works.}

\p{To return to the example of \i{Student after student}, I 
commented that designating
one word to \q{represent} the phrase seemed arbitrary.  
If we consider functional-typing alone, \i{after} is the only non-noun,
the natural conclusion is that \q{after} should be typed \NNtoN{}
(which implies that \q{after} is analogous to the \q{functional} position, and
in a lambda-calculus style reconstruction would be considered the \q{head}
{\mdash} Figure ~\ref{fig:ESA} is an example of how
the sentence could be annotated, for sake of discussion).
This particular idiom depends however
on the two constituent nouns being the same word (a pattern I've also alluded to with
idioms like \i{time after time}).  
Technically, this appears to be an example of \i{dependent types}: 
specifically, a type-theoretic model of \i{after} in this would seem 
to require that the \NNtoN{} signature be constrained so that the second 
noun matches the first {\mdash} so the second \N{} type is actually constrained 
to be a singleton type dependent on the first \N{}'s value.  
By way
of illustration, Figure ~\ref{fig:ESA} shows a
destructuring with implicit type annotations. \begin{figure*}
\caption{Dependency-style graph with argument repetition}	
\label{fig:ESA}
\vspace{1em}
\hspace{0.15\textwidth}	
\begin{minipage}{0.7\textwidth}
\begin{tikzpicture}

%\tikzset{snake it/.style={decorate, decoration=snake, segment length=5mm, amplitude=15mm}}

%\draw

%\node [s1] at (0,0) {Student};

\node (s1) at (1,1) {\textbf{Student}};
\node (after) [right=5mm of s1] {\textbf{after}};
\node (s2) [right=5mm of after] {\textbf{student}};
\node (complained) [right=5mm of s2] {\textbf{complained}};

\node (s1Rep) [double,draw=black,shape=circle,thick,fill=gray!50,inner sep=.5em,below=3.5cm of s1] {};
\node (afterRep) [double,draw=black,shape=circle,thick,fill=gray!50,inner sep=.5em,below=2cm of after] {};
\node (s2Rep) [double,draw=black,shape=circle,thick,fill=gray!50,inner sep=.5em,below=3.5cm of s2] {};
\node (complainedRep) [double,draw=black,shape=circle,thick,fill=gray!50,inner sep=.5em,below=2.5cm of complained] {};

\draw [ |-,-|, <->, line width = .8mm, draw=gray!70, 
 dashed, double equal sign distance, >= stealth, shorten <= .25cm, shorten >= .25cm ]
 (s1) to (s1Rep);

\draw [ |-,-|, <->, line width = .8mm, draw=gray!70, 
 dashed, double equal sign distance, >= stealth, shorten <= .25cm, shorten >= .25cm ]
(after) to (afterRep);
 
\draw [ |-,-|, <->, line width = .8mm, draw=gray!70,  
 dashed, double equal sign distance, >= stealth, shorten <= .25cm, shorten >= .25cm ]
(s2) to (s2Rep);

\draw [ |-,-|, <->, line width = .8mm, draw=gray!70, 
 dashed, double equal sign distance, >= stealth, shorten <= .25cm, shorten >= .25cm ]
(complained) to (complainedRep);
 
\draw [shorten <= .25cm, shorten >= .25cm ] 
(afterRep) to node [draw=black,shape = star,star points=4,thick,inner sep = 0mm, above] {1} (s1Rep);

\draw [shorten <= .25cm, shorten >= .25cm ] 
(afterRep) to node [draw=black,shape = star,star points=4,thick,inner sep = 0mm,above] {1} (s2Rep);

\node (s1E) [left=-2mm of s1Rep.east]{};
\node (s2E) [left=1mm of s2Rep.west]{};

\draw [shorten <= 2cm, shorten >= .3cm, double, % bend left = 5, 
decorate, decoration={snake, segment length=5mm, amplitude=.5mm}] %, amplitude=15mm]  
(s1E) to node [draw=black,shape = star,star points=4,thick,inner sep = 0mm, below] {2} (s2E);

\draw [shorten <= .5cm, shorten >= .5cm ] 
(afterRep) edge [bend left=20,looseness=1] node [draw=black,shape = star,star points=4,thick,inner sep = 0mm, 
 above, near end ] {3} (complainedRep);

\node (frameTopLeft) [below left = 1.5cm and -.65 cm of s1] {};
\node (frameBottomLeft) [below = 2.5cm of frameTopLeft] {};
\node (frameBottomRight) [right = 3.95cm of frameBottomLeft] {};
\node (frameTopRight) [above = 2.5cm of frameBottomRight] {};

\draw [shorten <= 0.15cm, shorten >= 0.15cm ] 
(frameTopLeft) edge [bend right=30,looseness=1] (frameBottomLeft);

\draw [shorten <= 0.15cm, shorten >= 0.15cm ] 
(frameBottomLeft) edge [bend right=30,looseness=1] (frameBottomRight);

\draw [shorten <= 0.15cm, shorten >= 0.15cm ] 
(frameBottomRight) edge [bend right=35,looseness=1] (frameTopRight);

\draw [shorten <= 0.15cm, shorten >= 0.45cm ] 
(frameTopRight) edge [bend right=30,looseness=1]
node [draw=black,shape = regular polygon,regular polygon sides=3,thick,inner sep = .2mm, 
above, near start, shape border rotate = 180] {4} (frameTopLeft);

%node [draw=black,shape = star,star points=4,thick,inner sep = 0mm, above, 
%bend left=100,looseness=3] {3}

%;


\end{tikzpicture}
\end{minipage}

\hspace{0.1\textwidth}
\begin{minipage}{0.8\textwidth}
			\renewcommand{\labelitemi}{$\blacklozenge$}
	
\begin{itemize}\setlength\itemsep{-.3em}
\item 1 \hspace{12pt}  Head/dependent relation
\item 2 \hspace{12pt}  Argument repetition ({}\BlankAfterBlank{} idiom)
\item 3 \hspace{12pt}  Propositional completion ({}\VisNtoS{})	
			\renewcommand{\labelitemi}{$\blacktriangledown$}
\item 4 \hspace{12pt}  Phrase (modeled as applicative structure), typed as {}\NPl{} 
\end{itemize}
\end{minipage}
\end{figure*}
Earlier I opined that the repetition in \BlankAfterBlank{} is a 
rhetorical effect emphasizing that the recurrence of some
phenomenon is tedious or noteworthy {\mdash} in this one usage \i{after}
becomes a pluralizing adjective that takes two arguments, but
requires them to be the same.  This sense of \i{after} can be
given a function-like \POS{} notation as, say,
\AfterNSingAndNSingToNPl{} (using \NSing{} and \NPl{} to mean singular
and count-plural nouns, respectively), but with a further 
stipulation that the second argument duplicates the first.  
Formally, then, the parameters
for \i{after} are a dependent type pair
\cite{BernardyEtAl}, \cite{TanakaEtAl} satisfied by an identity comparison between
the two nouns.  This analysis captures a type-theoretic gloss on the 
structural contrast between \i{Student after student} and
\i{Many students}, phrases which are similar but not identical 
in meaning (so whose differences need explaining).}

\p{I have offered a more cognitive account focusing on 
the implicit temporality of \i{Student after student}; this later 
type-oriented model is more formal, or at least leaves open the 
possibility that language is organically taking in structures 
engineered into artificial (e.g., computer programming) languages. 
It is certainly possible to witness formalizable structures in 
language patterns {\mdash} Zhaohui Luo finds strong evidence for
dependent types being a good model for semantic norms in
\cite{LuoSoloviev}, for example.  Whether these kinds of 
formalisms have important causal influence on 
language acquiring its evident patterns, or are more like 
just convenient representational tools, is perhaps an 
open (and maybe case-by-case) question.}

\p{Consider alternatives for \q{many students}.  The phrase as
written suggests a type signature (with \q{many} as the \q{function-like} or
derivative type) \NpltoNpl{}, yielding a syntactic interpretation of the phrase; this
interpretation also suggests a semantic progression, an accretion of intended detail.
From \i{students} to \i{many students} is a conversion between two plural nouns
(at the level of concepts and semantic roles); but it also implies relative size,
so it implies some \i{other} plural, some still larger group of students from which
\q{many} are selected.  While rather abstract and formal, the \NpltoNpl{} representation
points toward a more cognitive grounding which considers this \q{function} as a form
of thought-operation; a refinement of a situational model, descriptive resolution,
and so forth.  If we are prepared to accept a cognitive underpinning to semantic
classification, we can make the intuition of part of speech signatures as \q{functions}
more concrete: in response to what \q{many} (for example) is a function \i{of},
we can say a function of propositional attitude, cognitive schema, or attentional
focus.}

\p{The schema which usefully captures the sense and picture of \i{students} is
distinct (but arguably a variation on) that for \i{many students}, and there is a
\q{mental operation} triggered by the \i{many students} construction which
\q{maps} the first to the second.  Similarly, \i{student after student} triggers a
\q{scheme evolution} which involves a more explicit temporal unfolding
(in contrast to how \i{many students} instead involves a more explicit
quantitative \i{many/all} relation).  What these examples show is that
associating parts of speech with type signatures is not just a formal
fiat, which \q{works} representationally but does not necessarily capture
deeper patterns of meaning.  Instead, I would argue, type signatures
and their resonance into linkage acceptability structures
(like singular/plural and mass/count agreement) \i{point toward} the
effects of cognitive schema on what we consider meaningful.}

\p{In \i{Student after student came out against the proposal},
to \i{come out}, for/against, lies in the semantic frame of attitude and expression
(it requires a mental agent, for example), but, as per my prior analysis, its reception
carries a trace of spatial form: to come out \i{to} a public place, to
go on record with an opinion.  Usually
\q{come out [for/against]}, in the context of a policy or idea, is similarly
metaphorical.  But the concrete spatial interpretation remains latent, as a kind
of residue on even this abstract rendition, and the spatial 
undercurrent is poised to emerge
as more literal, should the context warrant.  However literally or metaphorically
the \q{space} of the \q{coming out} is
understood, however explicit or latent its cogitative figuration,
is not something internal to the language; it is a potentiality which
will present in different ways in different circumstances.  This is not to say that
it is something apart from linguistic meaning, but it shows how linguistic meaning
lies neither in abstract structure alone, nor contextual pragmatics, but in their cross-reference.}

\p{}

\p{}

\p{}

\p{}

\p{}
