
\begin{frame}{\ft{The NCN (Native Cloud/Native) Protocol}}
	\section{NCN}
\vspace{-3em}	
{\Large\fontfamily{uhv}\selectfont
\vspace{1em}
\begin{center}
\begin{minipage}{.9\textwidth}
\vspace{1em}
\fcolorbox{lqboutercolor}{lqbinnercolor}{\begin{minipage}{\textwidth}%
\begin{lightquadblock}{Cloud/Native Components as back-ends 
for native software}
\begin{center}\begin{minipage}{.98\textwidth}
{\Large \begin{itemize}
\item Our \q{Native Cloud/Native} protocol refers to native application front-ends paired with 
Cloud/Native (back-end) container instances.\vspace{1em}
\item  Allows sharing of code libraries and data representation 
across both endpoints.\vspace{1em}
\item  Common representation on both server- and client-side 
streamlines network communications (no need to marshal data between 
different formats).\vspace{1em}
\item  The NA3 technology can be 
ported to other application frameworks apart from 
Qt (wxWidgets, XCode, MFC, etc.).
This presentation will focus on NA3's default 
Qt{} implementation. 
\\\vspace{1.5em}
\end{itemize}}\end{minipage}
\end{center}
\end{lightquadblock}
\end{minipage}}
\fcolorbox{lqboutercolor}{lqbinnercolor}{\begin{minipage}{\textwidth}%
\begin{lightquadblock}{How Cloud Back-Ends Enhance Native Front Ends}
{\Large\begin{itemize}
\item  Cloud Backup; Share Data between Users; Collaborative Editing \vspace{1em}
\item  Maintain users' application state across different 
computers (home/school/office)\vspace{1em}
\item  Upgrade running applications without a need to re-compile
\vspace{1em}
\end{itemize} }
\end{lightquadblock}
\end{minipage}}

\end{minipage}
\end{center}
}

\end{frame}
