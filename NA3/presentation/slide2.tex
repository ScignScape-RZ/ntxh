
\begin{frame}{\ft{Our NCN (Native Cloud/Native) Protocol for Cloud Backends}}
	\section{NCN}
\vspace{-3em}	
{\Large\fontfamily{uhv}\selectfont
\vspace{1em}
\begin{center}
\begin{minipage}{.9\textwidth}
\vspace{1em}
\fcolorbox{lqboutercolor}{lqbinnercolor}{\begin{minipage}{\textwidth}%
\begin{lightquadblock}{Cloud/Native Components as Back-Ends 
for Native Software}
\begin{center}\begin{minipage}{.99\textwidth}
{\Large \begin{itemize}
\sqitem {\lsep} \parbox[t]{17cm}{Our \q{Native Cloud/Native} service is a protocol, which refers to native application front-ends paired with 
Cloud/Native (back-end) container instances.}\vspace{1em}
\sqitem {\lsep}  Code libraries and data representation may be shared 
across both endpoints.\vspace{1em}
\sqitem {\lsep}  \parbox[t]{17cm}{Common representation on both server- and client-side 
streamlines network communications (no need to marshal data between 
different formats).}\vspace{1em}
\sqitem {\lsep}  \parbox[t]{17cm}{Our NCN technology 
can be ported to other (non-Qt) application frameworks 
(wxWidgets, XCode, MFC, etc.).\\
\hspace*{5pt}{\MyOct}\hspace{-4pt} Note: This presentation will focus on NCN's default 
Qt{} implementation.} 
\\\vspace{1em}
\end{itemize}}\end{minipage}
\end{center}
\end{lightquadblock}
\end{minipage}}
\fcolorbox{lqboutercolor}{lqbinnercolor}{\begin{minipage}{\textwidth}%
\begin{lightquadblock}{How Cloud Back-Ends Enhance Native Front Ends}
{\Large\begin{itemize}
\sqitem {\lsep}  Cloud Backup {\MyOct} Share Data between Users {\MyOct} Collaborative Editing \vspace{1em}
\sqitem {\lsep}  \parbox[t]{14cm}{Maintain users' application state across different 
computers (home/school/office)\vspace{1em}}
\sqitem {\lsep}  Upgrade running applications without needing to re-compile
\vspace{1em}
\end{itemize} }
\end{lightquadblock}
\end{minipage}}

\end{minipage}
\end{center}
}

\end{frame}
