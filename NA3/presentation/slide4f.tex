
\begin{frame}{\ft{ROI and Development Phases}}
\section{Development Phases}

\definecolor{dclr}{RGB}{103,84,17}
\setbeamercolor{description item}{fg=dclr}
\setbeamersize{description width=10pt}

\vspace{-1em}

{\hspace{1.5em}\begin{minipage}[l]{.9\textwidth}\Large\centering\color{slidePartHeadColor} 	
{\LARGE \textbf{This slide divides a 5-year NCN development 
strategy into 6 phases, identifying how the market for 
NCN may be projected to expand correlatively.}}
\vspace{.7em}	
\end{minipage}}	

%\begin{lightquadblockc}{0,0.1,0.1}{\parbox{21cm}{\centering \vspace{10pt}OUTLINE:\vspace{3pt}}}

\hspace{20pt}\begin{minipage}{.995\textwidth}
{\LARGE \fontfamily{bch}\fontseries{eb}\selectfont \setlength{\leftmargini}{3pt}\begin{description}
\item[9 mos] Establish a hosting platform 
(projected to take the form of a 
RedHat Enterprise Service or Kamatera Partner affiliation) within which 
LTS can license individual cloud back-ends 
on a per-client basis, to be paired with 
clients' desktop front-ends. 
\item[12 mos] Officially release dual-licensed 
A3R libraries, along with support materials 
(documentation, web tutorials, test suites), 
which developers can use to create applications 
that leverage 
NCN back-ends (including those hosted by 
LTS directly).
\item[1-2 yrs] During year two, 
LTS will prioritize marketing its development libraries 
and cloud service, 
with an emphasis on explaining to Qt-based 
companies that the LTS hosting option provides 
functionality similar to the discontinued 
Qt Cloud Services.   
\item[2-3 yrs] Generalize NA3 to standard 
C++ (eliminating Qt dependencies); and 
implement NA3 in an Apple-specific version 
targeting XCode.  
\item[3-4 yrs] Port NA3 to Java; and 
build a Windows-specific implementation 
via MFC.
\item[4-5 yrs] With NA3 now realized 
in Qt, Windows, Mac, and Java versions, 
consolidate each of these implementations 
into canonical container prototypes, 
such as RedHat \q{Cartridges}.  This 
collection of cartridges then becomes 
a comprehensive, multi- platform 
desktop/cloud integration technology 
which could potentially be sold as a 
product suite to a large cloud and/or 
desktop-software vendor, such as  
RedHat, Autodesk, Fuji, Amazon 
(via Amazon Web Services), Rackspace, Adobe, etc.  
\end{description}}\end{minipage}


\end{frame}
