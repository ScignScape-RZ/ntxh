
\begin{frame}{\ft{ROI and Development Phases}}
\section{Development Phases}

\definecolor{dclr}{RGB}{103,84,17}
\setbeamercolor{description item}{fg=dclr}
\setbeamersize{description width=10pt}

\vspace{-.8em}


%\begin{lightquadblockc}{0,0.1,0.1}{\parbox{21cm}{\centering \vspace{10pt}OUTLINE:\vspace{3pt}}}

\hspace{20pt}\begin{minipage}{.996\textwidth}
{\fontsize{17}{23}\selectfont\fontfamily{bch}\selectfont \setlength{\leftmargini}{3pt}\begin{description}
\item[I (9-12 mos)] Establish a hosting platform 
(projected to take the form of a 
RedHat Enterprise Service or Kamatera Partner affiliation) within which 
LTS can license individual cloud back-ends 
on a per-client basis, paired with 
clients' desktop front-ends.  We will make tools 
available to help developers create applications 
that leverage NCN back-ends, including those hosted by 
our company.
\vspace{12pt}
\item[II (1-2 yrs)]  
LTS will prioritize marketing its development libraries 
and cloud service, 
with an emphasis on explaining to Qt-based 
companies that the LTS hosting option provides 
functionality similar to the discontinued 
Qt Cloud Services.   
\vspace{12pt}
\item[III (2-4 yrs)] Generalize NA3 to standard 
C++ (eliminating Qt dependencies),   
implement NA3 in an Apple-specific version 
targeting XCode, port NA3 to Java, and 
build a Windows-specific implementation 
via MFC.
\vspace{12pt}
\item[IV (Fifth yr exit)] With NA3 now realized 
in Qt, Windows, Mac, and Java versions, 
consolidate each of these implementations 
into canonical container prototypes, 
such as RedHat \q{Cartridges}.  This 
collection then becomes 
a comprehensive, multi-platform 
desktop/cloud integration technology 
valued at \$100M.
%which could potentially be sold as a 
%product suite to a large cloud and/or 
%desktop-software vendor, such as  
%RedHat, Autodesk, Fuji, Amazon 
%(via Amazon Web Services), Rackspace, Adobe, etc.  
\end{description}}\end{minipage}


\end{frame}
