
\begin{frame}{\ft{NCN Revenue Sources}}
\section{NCN Revenue}
\vspace{-.5em}	

{\fontfamily{fvs}\selectfont
%\begin{center}
\hspace*{-20pt}\begin{minipage}{1.1\textwidth}
\vspace{4pt}

%\definecolor{blback}{RGB}{0,100,100}	
%\definecolor{blfront}{RGB}{0,100,50}

%\fcolorbox{lqboutercolor}{lqbinnercolor}{\begin{minipage}{\textwidth}%
		
\begin{lightquadblockc}{0,0.1,0.1}{}
\begin{center}\begin{minipage}{1.05\textwidth}
{\fontsize{16}{23}\selectfont \setlength{\leftmargini}{30pt}\begin{enumerate}
\dmitem \textbf{Customization} \hspace{.5em} Custom-implemented applications 
using project-specific versions of NCN and/or \AtR{} 
(see slide 11).\vspace{10pt}

\dmitem \textbf{Licensing}  \hspace{.5em} Commercial licenses required for 
any deployment of NCN outside LTS-controled 
servers and/or any commercial deployment of \AtR{} 
applications.\vspace{10pt}

\dmitem \textbf{Hosting}  \hspace{.5em} Running proprietary 
containers via a Cloud-Native service such as 
OpenShift, LTS can offer integrated hosting and consulting 
wherein LTS fully implements and maintains a back-end 
paired to any desktop/native client software.
(Because the expertise involved 
in building native desktop applications is very different 
from the techniques required to deploy a Cloud-Native container 
image, the option of delegating all 
backend responsibilities to LTS may 
appeal to Qt-oriented development teams.)\vspace{10pt}

\dmitem \textbf{Sponsorship}  \hspace{.5em} 
Running a data-sharing platform which would be a 
publicly-visible introduction to NCN.  
This \q{demo} container 
would host research data sets (and 
would therefore be a resource in the public 
interest) allowing LTS to receive compensation 
from companies financially supporting the 
portal because it is a technology which
benefits science and research.
\end{enumerate}
}\end{minipage}
\end{center}
\end{lightquadblockc}
\end{minipage}

%}
%\end{minipage}
%\end{center}
}

\end{frame}
