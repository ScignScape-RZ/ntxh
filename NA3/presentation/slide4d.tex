
\begin{frame}{\ft{NA3 Revenue Sources}}
\section{Qt}
\vspace{.5em}	

%\MyDiamond{}
{\Large\fontfamily{uhv}\selectfont
\vspace{1em}
\begin{center}
\begin{minipage}{.9\textwidth}
\vspace{1em}
\fcolorbox{lqboutercolor}{lqbinnercolor}{\begin{minipage}{\textwidth}%
\begin{lightquadblock}{However ... Limited Qt Cloud Integration Support}
\begin{center}\begin{minipage}{.98\textwidth}
{\LARGE \setlength{\leftmargini}{30pt}\begin{enumerate}
\item \textbf{Customization} \hspace{.5em} Custom-implemented applications 
using project-specific versions of NCN and/or A3R.

\item \textbf{Licensing}  \hspace{.5em} Commercial licenses required for 
any deployment of NCN outside LTS-controled 
servers and/or any deployment of A3R 
applications (or of software including 
A3R components for such 
development requirements as databases, data modeling, 
scripting, data serializing/deserialization, 
and text parsing) in a commercial context.

\item \textbf{Hosting}  \hspace{.5em} LTS anticipates running proprietary 
containers via a Cloud-Native service such as 
OpenShift, and then leasing access to this service 
to NA3 users.  LTS can offer integrated hosting and consulting 
wherein LTS fully implements and maintains a back-end 
paired to any desktop/native client software.
Because the expertise involved 
in building native desktop applications is very different 
from the techniques required to deploy a Cloud-Native container 
image, the option of delegating all 
backend responsibilities to LTS may 
appeal to Qt-oriented development teams.

\item \textbf{Sponsorship}  \hspace{.5em} As discussed below, LTS anticipates 
running a data-sharing platform which would be a 
publicly-visible introduction to LTS's in-house 
NCN service (whereas other sub- or para-containers 
would be leased to third parties and provide 
publicly-visible content only at their discretion).  
This \q{demo} container, while being a vehicle for 
the general public to learn about NA3, 
would also host research data sets and 
would therefore be a resource in the public 
interest, allowing LTS to receive compensation 
from companies financially supporting the 
portal because of its merits as a technology 
benefiting science or scholarship.
\end{enumerate}
}\end{minipage}
\end{center}
\end{lightquadblock}
\end{minipage}}


\end{minipage}
\end{center}
}

\end{frame}
