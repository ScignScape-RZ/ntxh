
\begin{frame}{\Huge{\textbf{The NCN (Native Cloud/Native) Protocol}}}
\vspace{-3em}	
{\Large\fontfamily{uhv}\selectfont
\vspace{1em}
\begin{center}
\begin{minipage}{.9\textwidth}
\vspace{1em}
\fcolorbox{lqboutercolor}{lqbinnercolor}{\begin{minipage}{\textwidth}%
\begin{lightquadblock}{Cloud/Native Components as back-ends 
for native software}
\begin{center}\begin{minipage}{.98\textwidth}
{\bf \begin{itemize}
\item \q{Native Cloud/Native} refers to native application 
front-ends paired with Cloud/Native container instances.\vspace{1em}
\item  Share code libraries and data representation 
across both endpoints.\\\vspace{1.5em}
\end{itemize}}\end{minipage}
\end{center}
\end{lightquadblock}
\end{minipage}}
\fcolorbox{lqboutercolor}{lqbinnercolor}{\begin{minipage}{\textwidth}%
\begin{lightquadblock}{How Cloud Back-Ends Enhance Native Front Ends}
\begin{itemize}
\item  Cloud Backup; Share data between users; Collaborative Editing; 
\item  Persist users' application state across different 
computers (home/school/office, etc.);
\item  Upgrade running applications without re-compile.;
\end{itemize} 
\end{lightquadblock}
\end{minipage}}

\end{minipage}
\end{center}
}

\end{frame}
