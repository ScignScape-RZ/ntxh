
\begin{frame}{\ft{NA3 In Different Software Ecosystems}}
\section{Qt}

\vspace{-1.5em}	

	
{\Large\fontfamily{uhv}\selectfont
\vspace{1em}
\begin{center}
\begin{minipage}{\textwidth}
\vspace{1em}
\fcolorbox{lqboutercolor}{lqbinnercolor}{\begin{minipage}{\textwidth}%
\begin{lightquadblock}{Potentiel NA3 Markets}
\begin{center}\begin{minipage}{\textwidth}
{\LARGE \setlength{\leftmargini}{3pt}\begin{description}
\item[Windows MFC] (~\$135b market size) A3R can be implemented 
in C++/CLI, building off of a generic-C++ version using 
the C++ Standard Library in place of Qt-specific data structures. 
\item[Apple XCode] (~\$25b market size) Apple Operating Systems 
are based on Linux, so a Linux-oriented A3R implementation can 
form the basis of an XCode version.  This XCode implementation 
would also be built around the C++ Standard Library. 
\item[JavaFX] (~\$12.5b market size)  The Java programming 
language provides the most widely used cross-platform 
application development framework outside of Qt.  
It is feasible to port C++ A3R implementations 
to Java.  The core of this re-implementation would 
involve designing a Java Hypergraph Library compatible 
with the A3R serialization and Interface Definition 
protocol.
\item[Workflow Management] (~\$10b market size) 
A3R plugins can be added to new or existing 
applications to support inter-application 
networking, allowing multiple applications to 
be unified into workflow-management systems.  
\end{description}}\end{minipage}
\end{center}
\end{lightquadblock}
\end{minipage}}


\end{minipage}
\end{center}
}

\end{frame}
