
\begin{frame}{\ft{NA3 In Different Software Ecosystems}}
\section{Potential NA3 Markets}

\vspace{-3.5em}	

\definecolor{windowsc}{rgb}{0,0.24,0.25}
\definecolor{applec}{rgb}{0.2,0,0.5}
\colorlet{javac}{red!40!green}
\definecolor{dclr}{RGB}{103,84,17}

%\setbeamercolor{description item}{fg=dclr}
\setbeamersize{description width=10pt}
\setbeamercolor{description item}{fg=windowsc}	
	
{\Large\fontfamily{uhv}\selectfont
\vspace{1em}
\begin{center}
\begin{minipage}{\textwidth}
\vspace{1em}
\fcolorbox{lqboutercolor!65!cyan}{lqbinnercolor}
{\begin{minipage}{\textwidth}%
\begin{lightquadblockc}{0.2,0.1,0.5}{\parbox{21cm}{\vspace*{10pt}\centering Potential NA3 Markets \\(see Slide 6 for overview)\vspace*{10pt}}}
\hspace{10pt}\begin{minipage}{1.1\textwidth}
{\LARGE \fontfamily{bch}\fontseries{eb}\selectfont \setlength{\leftmargini}{3pt}\begin{description}
\item[Windows MFC] ({\texttildelow}\$135b market size) A3R can be implemented 
in C++/CLI, \\building off of a generic-C++ version using 
the C++ Standard Library in place \\of Qt-specific data structures.\vspace{10pt} \setbeamercolor{description item}{fg=applec}	
\item[Apple XCode] ({\texttildelow}\$25b market size) Apple Operating Systems 
are based on Linux, so a Linux-oriented A3R implementation can 
form the basis of an XCode version.  This XCode implementation 
would also be built around the C++ Standard Library. \vspace{10pt}\setbeamercolor{description item}{fg=javac}
\item[JavaFX] ({\texttildelow}\$12.5b market size)  The Java programming 
language provides the most widely used cross-platform 
application development framework outside of Qt.  
It \\is feasible to port C++ A3R implementations 
to Java.  The core of this re-implementation would 
involve designing a Java Hypergraph Library \\compatible 
with the A3R serialization and Interface Definition 
protocol.\vspace{10pt}\setbeamercolor{description item}{fg=dclr}
\item[Workflow Management] 
({\texttildelow}\$10b market size -- 
source: 
{\fontfamily{Roboto-LF}\fontseries{i}\selectfont MarketsandMarkets}) 
A3R plugins can be added to new or existing 
applications to support inter-application 
networking, unifying multiple applications into workflow-management systems.
\vspace{.75em}  
\end{description}}\end{minipage}
\end{lightquadblockc}
\end{minipage}}


\end{minipage}
\end{center}
}

\end{frame}
