\documentclass[10pt,letterpaper]{article}

\usepackage{eso-pic}

\AddToShipoutPictureBG{%

\ifnum\value{page}>0{
\AtTextUpperLeft{
\makebox[18.5cm][r]{
\raisebox{-2.3cm}{%
{\transparent{0.3}{\includegraphics[width=0.29\textwidth]{e-logo.png}}	}} } }
}\fi
}

\AddToShipoutPicture{%
{
 {\color{blGreen!70!red}\transparent{0.9}{\put(0,0){\rule{.55cm}{\paperheight}}}}%
 {\color{darkRed!70!purple}\transparent{1}\put(6,0){{\rule{.3cm}{\paperheight}}}}
% {\color{logoPeach!80!cyan}\transparent{0.5}{\put(0,700){\rule{1cm}{.6cm}}}}%
% {\color{darkRed!60!cyan}\transparent{0.7}\put(0,706){{\rule{1cm}{.6cm}}}}
% \put(18,726){\thepage}
% \transparent{0.8}
}
}



\AddToShipoutPicture{%

\ifnum\value{page}>0


{\color{blGreen!70!red}\transparent{0.9}{\put(300,8){\rule{0.5\paperwidth}{.3cm}}}}%
{\color{inOne}\transparent{0.8}{\put(300,10){\rule{0.5\paperwidth}{.3cm}}}}%
{\color{inTwo}\transparent{0.3}\put(300,13){{\rule{0.5\paperwidth}{.3cm}}}}

\put(301,16){%
\transparent{0.7}{
\includegraphics[width=0.2\textwidth]{logo.png}} }

{\color{blGreen!70!red}\transparent{0.9}{\put(5.6,5){\rule{0.5\paperwidth}{.4cm}}}}%
{\color{inOne}\transparent{1}{\put(5.6,10){\rule{0.5\paperwidth}{.4cm}}}}%
{\color{inTwo}\transparent{0.3}\put(5.6,15){{\rule{0.5\paperwidth}{.4cm}}}}

\fi
}

%\pagestyle{empty} % no page number
%\parskip 7.2pt    % space between paragraphs
%\parindent 12pt   % indent for new paragraph
%\textwidth 4.5in  % width of text
%\columnsep 0.8in  % separation between columns

\setlength{\footskip}{23pt}

\usepackage[paperheight=14.5in,paperwidth=9.1in]{geometry}
\geometry{left=.9in,top=1.1in,right=.6in,bottom=1.4in} %margins

\newcommand{\sectsp}{\vspace{12pt}}

\usepackage{graphicx}
\usepackage{color,framed}

\usepackage{float}

\usepackage{mdframed}


\usepackage{setspace}
\newcommand{\rpdfNotice}[1]{\begin{onehalfspacing}{

\Large #1

}\end{onehalfspacing}}

\usepackage{xcolor}

\usepackage[hyphenbreaks]{breakurl}
\usepackage[hyphens]{url}

\usepackage{hyperref}
\newcommand{\rpdfLink}[1]{\href{#1}{\small{#1}}}
\newcommand{\dblHref}[1]{\href{#1}{\small{\burl{#1}}}}
\newcommand{\browseHref}[2]{\href{#1}{\Large #2}}

\hypersetup{
    colorlinks=true,
    linkcolor=cyan,
    filecolor=magenta,
    urlcolor=blue,
}

\urlstyle{same}

\definecolor{blGreen}{rgb}{.2,.7,.3}
\definecolor{darkRed}{rgb}{.2,.0,.1}

\definecolor{darkBlGreen}{rgb}{.1,.3,.2}

\definecolor{oldBlColor}{rgb}{.2,.7,.3}

\definecolor{blColor}{rgb}{.1,.3,.2}

\definecolor{elColor}{rgb}{.2,.1,0}
\definecolor{flColor}{rgb}{0.7,0.3,0.3}

\definecolor{logoOrange}{RGB}{108, 18, 30}
\definecolor{logoGreen}{RGB}{85, 153, 89}
\definecolor{logoPurple}{RGB}{200, 208, 30}

\definecolor{logoBlue}{RGB}{4, 2, 25}
\definecolor{logoPeach}{RGB}{255, 159, 102}
\definecolor{logoCyan}{RGB}{66, 206, 244}
\definecolor{logoRed}{rgb}{.3,0,0}

\definecolor{inOne}{rgb}{0.122, 0.435, 0.698}% Rule colour
\definecolor{inTwo}{rgb}{0.122, 0.698, 0.435}% Rule colour

\definecolor{outOne}{rgb}{0.435, 0.698, 0.122}% Rule colour
\definecolor{outTwo}{rgb}{0.698, 0.435, 0.122}% Rule colour

\usepackage[many]{tcolorbox}% http://ctan.org/pkg/tcolorbox

\usepackage{transparent}

\newenvironment{cframed}{\begin{mdframed}[linecolor=logoPeach,linewidth=0.4mm]}{\end{mdframed}}

\newenvironment{ccframed}{\begin{mdframed}[backgroundcolor=logoGreen!5,linecolor=logoCyan!50!black,linewidth=0.4mm]}{\end{mdframed}}

\usepackage{aurical}
\usepackage[T1]{fontenc}

\usepackage{relsize}

\newcommand{\pseudoIndent}{

\vspace{1pt}\hspace{8pt}}

\newcommand{\YPDFI}{{\fontfamily{fvs}\selectfont YPDF-Interactive}}

%
\newcommand{\deconum}[1]{{\protect\raisebox{-1pt}{{\LARGE #1}}}}



\newcommand{\VersatileUX}{{\color{red!85!black}{\Fontauri Versatile}}%
{{\fontfamily{qhv}\selectfont\smaller UX}}}

\newcommand{\NDPCloud}{{\color{red!15!black}%
{\fontfamily{qhv}\selectfont {\smaller NDP C{\smaller LOUD}}}}}

\newcommand{\lfNDPCloud}{{\color{red!15!black}%
{\fontfamily{qhv}\selectfont N{\smaller DP C{\smaller LOUD}}}}}

\newcommand{\textds}[1]{{\fontfamily{lmdh}\selectfont{%
\raisebox{-1pt}{#1}}}}

\newcommand{\dsC}{{\textds{ds}{\fontfamily{qhv}\selectfont \raisebox{-1pt}
{\color{red!15!black}{C}}}}}

\newcommand{\HTXN}{\resizebox{!}{8pt}{\AcronymText{HTXN}}}
\newcommand{\lHTXN}{\resizebox{!}{8.5pt}{\AcronymText{HTXN}}}

\newcommand{\lMOSAIC}{\resizebox{!}{8.5pt}{\AcronymText{MOSAIC}}}

\newcommand{\XML}{\resizebox{!}{8pt}{\AcronymText{XML}}}
\newcommand{\RDF}{\resizebox{!}{8pt}{\AcronymText{RDF}}}

\newcommand{\CLang}{\resizebox{!}{8pt}{\AcronymText{C}}}

\newcommand{\HNaN}{\resizebox{!}{8pt}{\AcronymText{HN%
\textsc{a}N}}}


\newcommand{\MeshLab}{\resizebox{!}{8pt}{\AcronymText{MeshLab}}}
\newcommand{\IQmol}{\resizebox{!}{8pt}{\AcronymText{IQmol}}}

\newcommand{\GUI}{\resizebox{!}{8pt}{\AcronymText{GUI}}}

\newcommand{\OS}{\resizebox{!}{8.5pt}{\AcronymText{OS}}}


\newcommand{\API}{\resizebox{!}{8pt}{\AcronymText{API}}}

\newcommand{\IDE}{\resizebox{!}{8pt}{\AcronymText{IDE}}}

\newcommand{\ThreeD}{\resizebox{!}{8pt}{\AcronymText{3D}}}

\newcommand{\FAIR}{\resizebox{!}{8pt}{\AcronymText{FAIR}}}

\newcommand{\UI}{\resizebox{!}{8pt}{\AcronymText{UI}}}
%\newcommand{\NDPCloud}{\resizebox{!}{8pt}{%
%\AcronymText{NDP-Cloud}}}

%\newcommand{\lNDPCloud}{\resizebox{!}{8.5pt}{%
%\AcronymText{NDP-Cloud}}}

\newcommand{\lNDPCloud}{\lfNDPCloud}


\newcommand{\QNetworkManager}{\resizebox{!}{8pt}{\AcronymText{QNetworkManager}}}
\newcommand{\QTextDocument}{\resizebox{!}{8pt}{\AcronymText{QTextDocument}}}
\newcommand{\QWebEngineView}{\resizebox{!}{8pt}{\AcronymText{QWebEngineView}}}
\newcommand{\HTTP}{\resizebox{!}{8pt}{\AcronymText{HTTP}}}


\newcommand{\lAcronymTextNC}[2]{{\fontfamily{fvs}\selectfont {\Large{#1}}{\large{#2}}}}

\newcommand{\AcronymTextNC}[1]{{\fontfamily{fvs}\selectfont {\large #1}}}


\colorlet{orr}{orange!60!red}

\newcommand{\textscc}[1]{{\color{orr!35!black}{{%
						\fontfamily{Cabin-TLF}\fontseries{b}\selectfont{\textsc{\scriptsize{#1}}}}}}}


\newcommand{\textsccserif}[1]{{\color{orr!35!black}{{%
				\scriptsize{\textbf{#1}}}}}}


\newcommand{\AcronymText}[1]{{\textscc{#1}}}

\newcommand{\AcronymTextser}[1]{{\textsccserif{#1}}}


\newcommand{\mAcronymText}[1]{{\textscc{\normalsize{#1}}}}

\newcommand{\NAThree}{\resizebox{!}{8pt}{\AcronymText{NA3}}}
\newcommand{\NCN}{\resizebox{!}{8pt}{\AcronymText{NCN}}}
\newcommand{\AThreeR}{\resizebox{!}{8pt}{\AcronymText{A3R}}}

\newcommand{\lAThreeR}{\resizebox{!}{9pt}{\AcronymText{A3R}}}

\newcommand{\lNAThree}{\resizebox{!}{9pt}{\AcronymText{NA3}}}

\newcommand{\NGML}{\resizebox{!}{8pt}{\AcronymText{NGML}}}

\newcommand{\Cpp}{\resizebox{!}{8.5pt}{\AcronymText{C++}}}

\newcommand{\WhiteDB}{\resizebox{!}{8pt}{\AcronymText{WhiteDB}}}

\colorlet{drp}{darkRed!70!purple}

%\newcommand{\MOSAIC}{{\color{drp}{\AcronymTextNC{\scriptsize{MOSAIC}}}}}

\newcommand{\MOSAIC}{\resizebox{!}{8pt}{\AcronymText{MOSAIC}}}


\newcommand{\mMOSAIC}{{\color{drp}{\AcronymTextNC{\normalsize{MOSAIC}}}}}

\newcommand{\MOSAICVM}{\mMOSAIC-\mAcronymText{VM}}

\newcommand{\sMOSAICVM}{\resizebox{!}{8pt}{\MOSAICVM}}
\newcommand{\sMOSAIC}{\resizebox{!}{8pt}{\MOSAIC}}


%\newcommand{\lMOSAIC}{{\color{drp}{\lAcronymTextNC{M}{OSAIC}}}}
\newcommand{\lfMOSAIC}{\resizebox{!}{9pt}{{\color{drp}{\lAcronymTextNC{M}{OSAIC}}}}}

\newcommand{\Mosaic}{\resizebox{!}{8pt}{\MOSAIC}}
\newcommand{\MosaicPortal}{{\color{drp}{\AcronymTextNC{MOSAIC Portal}}}}

\newcommand{\RnD}{\resizebox{!}{7.5pt}{\AcronymText{R\&D}}}
\newcommand{\QtCpp}{\resizebox{!}{8.5pt}{\AcronymText{Qt/C++}}}
\newcommand{\Qt}{\resizebox{!}{9pt}{\AcronymText{Qt}}}
\newcommand{\QtSQL}{\resizebox{!}{8pt}{\AcronymText{QtSQL}}}

\newcommand{\HTML}{\resizebox{!}{8pt}{\AcronymText{HTML}}}
\newcommand{\PDF}{\resizebox{!}{8pt}{\AcronymText{PDF}}}

\newcommand{\p}{

\vspace{1.2em}}

\newcommand{\q}[1]{{\fontfamily{qcr}\selectfont ``}#1{\fontfamily{qcr}\selectfont ''}} 

%\newcommand{\deconum}[1]{{\textcircled{#1}}}


\renewcommand{\thesection}{\protect\mbox{\deconum{\Roman{section}}}}
\renewcommand{\thesubsection}{\arabic{section}.\arabic{subsection}}

\newcommand{\llMOSAIC}{\mbox{{\LARGE MOSAIC}}}
%\newcommand{\lfMOSAIC}{\mbox{M\small{OSAIC}}}

\newcommand{\llMosaic}{\llMOSAIC}
\newcommand{\lMosaic}{\lMOSAIC}
\newcommand{\lfMosaic}{\lfMOSAIC}


\newcommand{\llWC}{\mbox{{\LARGE WhiteCharmDB}}}

\newcommand{\llwh}{\mbox{{\LARGE White}}}
\newcommand{\llch}{\mbox{{\LARGE CharmDB}}}

\usepackage{enumitem}

\setlist[description]{%
  topsep=30pt,               % space before start / after end of list
  itemsep=5pt,               % space between items
  font={\bfseries\sffamily}, % set the label font
%  font={\bfseries\sffamily\color{red}}, % if colour is needed
}

\setlist[enumerate]{%
  topsep=3pt,               % space before start / after end of list
  itemsep=-2pt,               % space between items
  font={\bfseries\sffamily}, % set the label font
%  font={\bfseries\sffamily\color{red}}, % if colour is needed
}

%\usepackage{tcolorbox}

\newcommand{\slead}[1]{%
\noindent{\raisebox{2pt}{\relscale{1.15}{{{%
\fcolorbox{logoCyan!50!black}{logoGreen!5}{#1}
}}}}}\hspace{.5em}}


\let\OldLaTeX\LaTeX

\renewcommand{\LaTeX}{\resizebox{!}{8pt}{\color{orr!35!black}{\OldLaTeX}}}

\newcommand{\LargeLaTeX}{\resizebox{!}{8.5pt}{\color{orr!35!black}{\OldLaTeX}}}


\setlength\parindent{24pt}
%%\usepakage{newfile}

\usepackage{hyperref}

\usepackage{etoolbox}

\usepackage{zref-user}

\newwrite\sdiFile
\immediate\openout\sdiFile=\jobname.sdi.txt

\newcommand{\p}[1]{

\vspace{10pt}#1}

\newif\iftabng
\tabngfalse


\usepackage{letltxmacro}
\LetLtxMacro{\oldmmsemi}{\;}
\LetLtxMacro{\oldtbplus}{\+}
\LetLtxMacro{\oldtbgt}{\>}
\LetLtxMacro{\oldmmgt}{\+}

\newcommand{\+}{\iftabng\oldtbplus\else\sss\fi}

\renewcommand{\>}{\iftabng\oldtbplus\else
\ifmmode\oldmmgt\else\sse\sss\fi\fi}

%\renewcommand{\>}{\sse\sss}

\renewcommand{\;}{\relax\ifmmode\oldmmsemi\else\sse\fi}

\newcommand{\writeSDI}[1]{\immediate\write\sdiFile#1}

\newcommand{\emblink}[2]{\href{\#sdi:#1--#2}{\#sdi:#1--#2}}

%\newcount\sdiCounter
%\def\advsdiCounter{\global\advance\sdiCounter by1}

%\newcount\sdiCounterP
%\def\advsdiCounterP{\global\advance\sdiCounterP by1}

%\newcounter{sdiCounter}
\newcounter{sdiCounterP}[page]
\newcounter{sdiCounter}

\def\topt#1{\expandafter\the\dimexpr\dimexpr#1sp\relax\relax}

\makeatletter
%\catcode`\*=10
\newcommand{\sss}{%
\stepcounter{sdiCounterP}
\stepcounter{sdiCounter}
\pdfsavepos\write\sdiFile{!/ SDI_Sentence_Start} 
\write\sdiFile\expandafter{\expandafter$%
\expandafter i\expandafter:%
\expandafter\space\the\c@sdiCounter}
\write\sdiFile\expandafter{\expandafter$%
\expandafter o\expandafter:%
\expandafter\space\the\c@sdiCounterP}
\write\sdiFile\expandafter{\expandafter$%
\expandafter p\expandafter:%
\expandafter\space\thepage^^J%
$x: \topt\pdflastxpos^^J%
$y: \topt\pdflastypos^^J%
/!^^J%
<<>^^J%
}}
%\catcode`\%=14
\makeatother

\makeatletter
\newcommand{\sse}{%
\pdfsavepos\write\sdiFile{!/ SDI_Sentence_End} 
\write\sdiFile\expandafter{\expandafter$%
\expandafter i\expandafter:%
\expandafter\space\the\c@sdiCounter}
\write\sdiFile\expandafter{\expandafter$%
\expandafter o\expandafter:%
\expandafter\space\the\c@sdiCounterP}
\write\sdiFile\expandafter{\expandafter$%
\expandafter p\expandafter:%
\expandafter\space\thepage^^J%
$x: \topt\pdflastxpos^^J%
$y: \topt\pdflastypos^^J%
/!^^J%
<<>^^J%
}}
\makeatother



\newcommand{\lun}[1]{{\fontfamily{qcr}\selectfont{%
\LARGE{\textbf{\underline{#1}}}}}}


\begin{document}
	
{\linespread{1.1}\selectfont

\vspace*{-7em}

\begin{center}
%{\relscale{1.2}{\fontfamily{qcr}\fontseries{b}\selectfont 
%{\colorbox{black}{\color{blue}{\llWC{} Database Engine \\and 
%\llMOSAIC{} Native Application Toolkit}}}}}

\colorlet{ctmp}{logoPeach!20!gray}
\colorlet{ctmpp}{ctmp!90!yellow}
\colorlet{ctmppp}{ctmpp!50!black}
\colorlet{ctmpppp}{ctmppp!90!logoRed}

\vspace{1em}

%{\colorbox{darkBlGreen!30!darkRed}{%
\begin{tcolorbox}
[
%%enhanced,
%%frame hidden,
%interior hidden
arc=2pt,outer arc=0pt,
enhanced jigsaw,
width=.7\textwidth,
colback=ctmpppp!60,
colframe=logoRed!30!darkRed,
drop shadow=logoPurple!50!darkRed,
%boxsep=0pt,
%left=0pt,
%right=0pt,
%top=2pt,
]
\begin{minipage}{\textwidth}	
\begin{center}		
{\setlength{\fboxsep}{18pt}
	\relscale{1.4}{{\fontfamily{qcr}\fontseries{b}\selectfont%
{NCN/A3R Native Application Framework

\\
 (Native-Cloud/Native, Application as a Resource)}}}}
\end{center}
\end{minipage}
\end{tcolorbox}
\end{center}

\vspace{-1em}

%\noindent\lun{Overview}
\fontfamily{ptm}\fontsize{13pt}{18pt}\selectfont
{\sectsp}
\p{\NCN{}/\AThreeR{} (hereafter \NAThree{}) is a \Qt{}-based 
application-development framework which prioritizes 
hybrid solutions combining cloud and desktop/native 
components.  The \NCN{} (Native-CloudNative) 
model refers to desktop client applications that 
are integrated with Cloud/Native back-ends; 
by sharing code libraries and data formats across 
both end-points, \NCN{} solutions are more streamlined 
than native front-ends with generic back-ends, or 
Cloud/Native back-ends with web-application clients.  
The \AThreeR{} (Application-as-a-Resource) model promotes 
self-contained, downloadable applications that can 
be distributed in source-code fashion and compiled 
with few (if any) non-\Qt{} dependencies.  The 
combined \NAThree{} framework yields a 
comprehensive application-development toolkit 
with numerous components to streamline the 
implementation of \Qt{} applications 
(\NAThree{} can also be used as a template 
for implementations based on frameworks other 
than \Qt{}, such as wxWidgets or 
Operating-System-specific options).           
}

\vspace{2.25em}
\noindent\lun{The Current Status of \Qt{} Cloud Integration}
\p{There has been considerable demand in the native-application sector for a systematized Cloud Services model designed to interoperate with cross-platform native applications.  Cloud/Native components can augment the functionality of native/desktop software by providing remote storage for user data; enabling users to share content for collaborative work; maintaining domain-specific repositories (i.e., spaces of resources whose format is specialized so that only select applications can access them properly); and upgrading or extending applications without re-install.  
We use the term \q{Native Cloud/Native} to describe hybrid applications whose server and client endpoints are both internally native -- in contrast to conventional Cloud/Native where native servers are paired with (potentially) non-native clients. 
}

\p{Cloud/Native support in existing native-application frameworks is fairly primitive.  Since \Qt{} is by far the most widely-used such framework, the \Qt{} case is instructive.  In 2013 (following an earlier beta phase) the \Qt{} company introduced \p{Qt Cloud Services}, which provided a convenient, Qt-aware cloud-hosting platform for \Qt{} accounts (in the company's words: \p{Qt Cloud Beta has solved an immense need for Qt developers when it comes to backend-as-a-service and believe that there is an even greater need to provide the Qt ecosystem with an all-in-one Qt solution for cloud computing}).  However --- to the consternation of the Qt community --- this project was discontinued several years later (retroactively we can identify some design flaws which might have hindered the project).  Meanwhile, OpenShift discontinued their free-tier Cloud/Native hosting last year, and another company with Cloud/Native options, Arukas, is folding at the end of this month.  This means that \Qt{} developers have limited options even for hosting hand-rolled \Qt{} cloud solutions (which can be done by compiling \Qt{} into a Ubuntu container)
}

\p{Considering the prominence of both \Qt{} and Cloud/Native technologies in the contemporary computing landscape, it is disconcerting that no standard framework or hosting service provides a cloud platform which works with \Qt{} \q{out-of-the-box.}  The existence of such a platform would be a boon to software in sectors like scientific computing, bioinformatics, bioimaging, pharmaceuticals, academic publishing, and other fields (where due to complex \GUI{} and/or data-analytic requirements) the software is predominantly native-compiled and desktop-oriented.}
 
\p{Of course, many desktop applications have some web integration, but the current architecture forces the client-facing and web-facing components of the application to be almost completely separate, which adds to development time and expense.  Moreover, current native-application environments do not fully leverage Cloud/Native services; they may well be implemented via more old-fashioned non-cloud servers.  The great possibility of Native Cloud/Native is a peer-to-peer client-server relationship, sharing libraries and data formats on both ends; and the infrastructure to bring the benefits of Cloud Computing (e.g. faster development and deployment, and less expensive hosting, as compared to non-cloud web services) to the native-application sector.}

\noindent\lun{Native Cloud/Native in the context of \lNAThree{}}
\p{In light of the limitations just identified, 
LTS intends to contribute tools or hosting arrangements that would bring some of the capabilities of \Qt{} Cloud Services back to the market.  The simplest commercial model for such a product is to licence a containerized \Qt{}-based \HTTP{} server that can run as a local application during testing and development, before being deployed to a container hosting service.  
We have implemented a prototype server along these lines 
that we call \NDPCloud{} (for \q{Native-Driven Platform}).  
\lNDPCloud{} is fully self-contained in a \Qt{} context 
(it bundles portions of the Node.js code base and 
utilizes the \Qt{} network module, so it requires no 
external \HTTP{} or sockets libraries).  One significant 
benefit of \NDPCloud{} is that it is fully transparent: 
all of the code for parsing and routing \HTTP{} 
requests can be loaded into \IDE{}s (such as \Qt{} creator)
and examined by the debugger.  Another benefit is that 
project-specific libraries can be compiled into both 
\NDPCloud{} instances and client front-ends; therefore, 
clients and servers can share procedures for serializing 
and deserializing domain-specific data structures.  
For development and prototyping, \NDPCloud{} 
can be launched as an ordinary (non-virtual) 
\OS{}; further testing can then be performed 
running \NDPCloud{} as a local Docker container, 
before eventually deploying the application to 
a remote Docker hosting environment. 
}

\p{Via \NDPCloud{}, \NCN{} applications can be 
deployed on any Docker cloud service, 
such as OpenShift.  In this guise LTS has 
no direct involvement with the hosting service 
(although \NDPCloud{} includes some tools 
to streamline cloud deployment).  Ideally, 
however, LTS would like to secure its own 
hosting capabilities, perhaps by using an 
LTS-specific container depoyed on OpenShift 
or a similar platform.  LTS would allocate 
cloud assets to \NDPCloud{} licensees 
(e.g. a limited free-tier hosting plan) 
for testing and development.  A further 
possibility is to provide free 
hosting, subject to data-space constraints, 
to scientific institutions.  The dwindling 
availability of free-tier Cloud/Native options 
is a hinderance to projects' adoption of 
Cloud/Native solutions for sharing and 
disseminating scientific data; this can 
result in researchers hosting data sets on 
platforms such as Mendeley or DataVerse, 
which have limited functionality or 
customizability compared to Cloud/Native 
containers.  Use-cases for \NAThree{} in 
the context of scientific data sets are explained in 
the discussion of \AThreeR{} below.}

\vspace{-.5em}
\noindent\lun{Application Development via \lAThreeR{}}
\p{The \AThreeR{} model facilitates implementation of 
standalone native applications, whose data models and 
\UI{} logistics are described via integrated metadata.
As much as possible, \AThreeR{} applications 
are entirely self-contained, so that all application 
code and data can be packaged into single 
downloadable resource.  In a \Qt{} environment, 
\Qt{} modules are available for concerns such 
as networking, database management, \Cpp{} reflection, 
\XML{} or \JSON{} parsing, or embedded web viewers, so that 
\AThreeR{} can leverage these capabilities without 
requiring separate library installs.  For many use-cases, 
then, an entire desktop application can be deployed 
in source-code form, to be compiled and launched 
via a single click within \Qt{} creator.  
Self-contained in this manner, \AThreeR{} 
applications can be treated as single 
resource units --- to some degree analogous 
to Docker containers, but achieving their 
autonomy by leveraging the \Qt{} ecosystem 
rather than by virtualization.  
}

\p{\lAThreeR{} applications are also autonomous resources 
by virtue of detailed metadata bundled with application code.  
This metadata provides a summary of application-specific 
data models, capabilities, \UI{} features, and 
user documentation.  The metadata may be accessed 
by human users or by automated tools to help 
users become familiar with a 
newly-acquired \AThreeR{} resource.}

\p{The \AThreeR{} architecture is especially 
warranted when applications are designed to 
work with one or several non-standardized, 
domain-specific data formats (including those 
unique to an individual data set).  In these 
scenarios, \AThreeR{} applications provide 
both libraries for parsing and manipulating 
the domain-specific formats and a \q{reference 
implementation} documenting the proper 
visualization and User Experience optimal 
for the unique data structures involved.  
This structural profiling is advanced not 
only by domain-specific \GUI{} components 
implemented within \AThreeR{} applications, 
but also by \AThreeR{} metadata which 
describes application-specific data types 
and interface requirements germane to 
any software components which work with 
such data types.}

\vspace{-.5em}
\noindent\lun{\lNCN{} and \lAThreeR{} in consort}
\p{While \NCN{} applications need not use \AThreeR{}, 
or vice-versa, the two models are organically 
paired together.  This can take the form of 
\NCN{} servers hosting \AThreeR{} applications 
as resources, and/or \AThreeR{} software 
connecting to \NCN{} instances as a domain-specific 
cloud back-end.  The \AThreeR{} metadata 
paradigm, based on \q{Hyperegraph Ontologies}, 
provides tools to streamline the encoding and 
distribution of application-specific data types.  
This model thereby accelerates the process of 
implementing cloud services procedurally 
aligned with \AThreeR{} components, because 
complementary \AThreeR{} and \NCN{} endpoints 
can share the same data-type libraries.  
Moreover, \AThreeR{} interface definitions 
can serve as references for implementing 
compatible \NCN{} server-side code; 
the interface specification documents which 
client-side procedures will handle any 
server-originating data structures, so the 
server-side data providers can be constructed 
accordingly.
}

\p{To ensure rigorous alignment between client and server endpoints, \NAThree{} employs a data modeling paradigm based on hypergraphs; the mathematical framework for the relevant new hypergraph model (which adds some additional structure to the theory of Hypergraph Categories) is provisionally outlined in a chapter of a book which I edited that will soon be published (we can share this material, or alternatively a more thorough unpublished explication, as desired).  In practical terms, the advantage of this model is that \Qt{}-specific data structures can be conveniently serialized and shared with or through cloud services; meanwhile the relevant data structures --- and their interface and procedural requirements --- are rigorously characterized, to support application testing, code-verification, documentation, and systematic User Interface development.  
\lNAThree{} concretely operationalizes theories 
which have been advanced in the scientific-computing 
and knowledge-engineering community toward a 
more conceptually refined and \q{multi-scale} 
Semantic Web.  \lAThreeR{} extends \q{Semantic Web 
alternatives} such as Conceptual Space Markup 
Language (\CSML{}) and Categorial Informatics, 
while also providing a self-contained \Cpp{} 
Hypergraph library comparable in some respects 
to AtomSpace (part of the 
OpenCOG platform) or to HypegraphDB (Hypergraphs 
and Conceptual Spaces have been proposed in 
combination as a comprehensive foundation 
for computational semantics, notably in 
the article \textit{Interacting Conceptual Spaces} 
which arose from an Oxford University reading 
group on Category Theory and formal grammar).
}

\p{A good example of an \AThreeR{} use-case is 
that of hosting scientific data sets.  According 
to the emerging \q{Research Object} paradigm, 
scientific data should be published alongside 
code which ensures that subsequent readers and 
researchers have the tools they need to 
access, analyze, and visualize the data set, 
including double-checking statistical analyses 
and/or replicating experiments.  With 
\AThreeR{}, data sets can be self-contained 
Research Object \q{bundles} while still 
being provisioned with full-featured 
desktop applications tailored to their 
specific information profile.  
LTS is actively developing a framework (called DataSet Creator, or \dsC{}) for building data sets that paired with academic publications and with native software implemented in custom fashion for each data set (we can provide links to several data sets published as demonstration examples of this technology).  The hosting and implementation of data sets along these lines offers concrete examples of \NAThree{} solutions and an opportunity to promote \NAThree{}-style development in the scientific and publishing communities. 
}
 
\p{As this use-case illustrates, 
our novel \NCN{} model is based on rigorous serialization and interface-definition paradigms; in comparison, \Qt{} Cloud Services tended to reuse structures more appropriate for non-native contexts (e.g. \JSON{}), which arguably limited client-to-server interoperability.  LTS's \NAThree{} model, by contrast, is combined with data modeling and serialization features that bring their own benefits to application projects over all; as such, this model does not only provide cloud-integration capabilities, but can be used as an overarching application-development framework.}

%\vspace{2em}
%\noindent\lun{Complex Exolinks and Interoperability}


\end{document}

