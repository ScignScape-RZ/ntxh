\documentclass[10pt,letterpaper]{article}

\usepackage{eso-pic}

\AddToShipoutPictureBG{%

\ifnum\value{page}>0{
\AtTextUpperLeft{
\makebox[18.5cm][r]{
\raisebox{-2.3cm}{%
{\transparent{0.3}{\includegraphics[width=0.29\textwidth]{e-logo.png}}	}} } }
}\fi
}

\AddToShipoutPicture{%
{
 {\color{blGreen!70!red}\transparent{0.9}{\put(0,0){\rule{.55cm}{\paperheight}}}}%
 {\color{darkRed!70!purple}\transparent{1}\put(6,0){{\rule{.3cm}{\paperheight}}}}
% {\color{logoPeach!80!cyan}\transparent{0.5}{\put(0,700){\rule{1cm}{.6cm}}}}%
% {\color{darkRed!60!cyan}\transparent{0.7}\put(0,706){{\rule{1cm}{.6cm}}}}
% \put(18,726){\thepage}
% \transparent{0.8}
}
}



\AddToShipoutPicture{%

\ifnum\value{page}>0


{\color{blGreen!70!red}\transparent{0.9}{\put(300,8){\rule{0.5\paperwidth}{.3cm}}}}%
{\color{inOne}\transparent{0.8}{\put(300,10){\rule{0.5\paperwidth}{.3cm}}}}%
{\color{inTwo}\transparent{0.3}\put(300,13){{\rule{0.5\paperwidth}{.3cm}}}}

\put(301,16){%
\transparent{0.7}{
\includegraphics[width=0.2\textwidth]{logo.png}} }

{\color{blGreen!70!red}\transparent{0.9}{\put(5.6,5){\rule{0.5\paperwidth}{.4cm}}}}%
{\color{inOne}\transparent{1}{\put(5.6,10){\rule{0.5\paperwidth}{.4cm}}}}%
{\color{inTwo}\transparent{0.3}\put(5.6,15){{\rule{0.5\paperwidth}{.4cm}}}}

\fi
}

%\pagestyle{empty} % no page number
%\parskip 7.2pt    % space between paragraphs
%\parindent 12pt   % indent for new paragraph
%\textwidth 4.5in  % width of text
%\columnsep 0.8in  % separation between columns

\setlength{\footskip}{20pt}

\usepackage[paperheight=12.3in,paperwidth=9in]{geometry}
\geometry{left=.75in,top=.7in,right=.6in,bottom=1.4in} %margins

\newcommand{\sectsp}{\vspace{12pt}}

\usepackage{graphicx}
\usepackage{color,framed}

\usepackage{float}

\usepackage{mdframed}


\usepackage{setspace}
\newcommand{\rpdfNotice}[1]{\begin{onehalfspacing}{

\Large #1

}\end{onehalfspacing}}

\usepackage{xcolor}

\usepackage[hyphenbreaks]{breakurl}
\usepackage[hyphens]{url}

\usepackage[official]{eurosym}

\usepackage{hyperref}
\newcommand{\rpdfLink}[1]{\href{#1}{\small{#1}}}
\newcommand{\dblHref}[1]{\href{#1}{\small{\burl{#1}}}}
\newcommand{\browseHref}[2]{\href{#1}{\Large #2}}

\hypersetup{
    colorlinks=true,
    linkcolor=cyan,
    filecolor=magenta,
    urlcolor=blue,
}

\urlstyle{same}

\definecolor{blGreen}{rgb}{.2,.7,.3}
\definecolor{darkRed}{rgb}{.2,.0,.1}

\definecolor{darkBlGreen}{rgb}{.1,.3,.2}

\definecolor{oldBlColor}{rgb}{.2,.7,.3}

\definecolor{blColor}{rgb}{.1,.3,.2}

\definecolor{elColor}{rgb}{.2,.1,0}
\definecolor{flColor}{rgb}{0.7,0.3,0.3}

\definecolor{logoOrange}{RGB}{108, 18, 30}
\definecolor{logoGreen}{RGB}{85, 153, 89}
\definecolor{logoPurple}{RGB}{200, 208, 30}

\definecolor{logoBlue}{RGB}{4, 2, 25}
\definecolor{logoPeach}{RGB}{255, 159, 102}
\definecolor{logoCyan}{RGB}{66, 206, 244}
\definecolor{logoRed}{rgb}{.3,0,0}

\definecolor{inOne}{rgb}{0.122, 0.435, 0.698}% Rule colour
\definecolor{inTwo}{rgb}{0.122, 0.698, 0.435}% Rule colour

\definecolor{outOne}{rgb}{0.435, 0.698, 0.122}% Rule colour
\definecolor{outTwo}{rgb}{0.698, 0.435, 0.122}% Rule colour

\usepackage[many]{tcolorbox}% http://ctan.org/pkg/tcolorbox

\usepackage{transparent}

\newenvironment{cframed}{\begin{mdframed}[linecolor=logoPeach,linewidth=0.4mm]}{\end{mdframed}}

\newenvironment{ccframed}{\begin{mdframed}[backgroundcolor=logoGreen!5,linecolor=logoCyan!50!black,linewidth=0.4mm]}{\end{mdframed}}

\usepackage{aurical}
\usepackage[T1]{fontenc}

\usepackage{relsize}

\newcommand{\pseudoIndent}{

\vspace{1pt}\hspace{8pt}}

\newcommand{\YPDFI}{{\fontfamily{fvs}\selectfont YPDF-Interactive}}

%
\newcommand{\deconum}[1]{{\protect\raisebox{-1pt}{{\LARGE #1}}}}



\newcommand{\VersatileUX}{{\color{red!85!black}{\Fontauri Versatile}}%
{{\fontfamily{qhv}\selectfont\smaller UX}}}

\newcommand{\NDPCloud}{{\color{red!15!black}%
{\fontfamily{qhv}\selectfont {\smaller NDP C{\smaller LOUD}}}}}

\newcommand{\lfNDPCloud}{{\color{red!15!black}%
{\fontfamily{qhv}\selectfont N{\smaller DP C{\smaller LOUD}}}}}

\newcommand{\textds}[1]{{\fontfamily{lmdh}\selectfont{%
\raisebox{-1pt}{#1}}}}

\newcommand{\dsC}{{\textds{ds}{\fontfamily{qhv}\selectfont \raisebox{-1pt}
{\color{red!15!black}{C}}}}}

\newcommand{\HTXN}{\resizebox{!}{8pt}{\AcronymText{HTXN}}}
\newcommand{\lHTXN}{\resizebox{!}{8.5pt}{\AcronymText{HTXN}}}


\newcommand{\CSML}{\resizebox{!}{8.5pt}{\AcronymText{CSML}}}

\newcommand{\lMOSAIC}{\resizebox{!}{8.5pt}{\AcronymText{MOSAIC}}}

\newcommand{\XML}{\resizebox{!}{8pt}{\AcronymText{XML}}}
\newcommand{\RDF}{\resizebox{!}{8pt}{\AcronymText{RDF}}}

\newcommand{\JSON}{\resizebox{!}{8pt}{\AcronymText{JSON}}}

\newcommand{\CLang}{\resizebox{!}{8pt}{\AcronymText{C}}}

\newcommand{\HNaN}{\resizebox{!}{8pt}{\AcronymText{HN%
\textsc{a}N}}}


\newcommand{\MeshLab}{\resizebox{!}{8pt}{\AcronymText{MeshLab}}}
\newcommand{\IQmol}{\resizebox{!}{8pt}{\AcronymText{IQmol}}}

\newcommand{\GUI}{\resizebox{!}{8pt}{\AcronymText{GUI}}}

\newcommand{\OS}{\resizebox{!}{8.5pt}{\AcronymText{OS}}}

\newcommand{\GNome}{\resizebox{!}{8.5pt}{\AcronymText{GNome}}}


\newcommand{\CPU}{\resizebox{!}{8.5pt}{\AcronymText{CPU}}}
\newcommand{\MQTT}{\resizebox{!}{8.5pt}{\AcronymText{MQTT}}}
\newcommand{\CoAP}{\resizebox{!}{8.5pt}{\AcronymText{CoAP}}}
\newcommand{\IoT}{\resizebox{!}{8.5pt}{\AcronymText{IoT}}}

\newcommand{\KDE}{\resizebox{!}{8.5pt}{\AcronymText{KDE}}}
\newcommand{\MFC}{\resizebox{!}{8.5pt}{\AcronymText{MFC}}}

\newcommand{\QGIS}{\resizebox{!}{8.5pt}{\AcronymText{QGIS}}}



\newcommand{\API}{\resizebox{!}{8pt}{\AcronymText{API}}}

\newcommand{\IDE}{\resizebox{!}{8pt}{\AcronymText{IDE}}}

\newcommand{\ThreeD}{\resizebox{!}{8pt}{\AcronymText{3D}}}

\newcommand{\FAIR}{\resizebox{!}{8pt}{\AcronymText{FAIR}}}

\newcommand{\UI}{\resizebox{!}{8pt}{\AcronymText{UI}}}
%\newcommand{\NDPCloud}{\resizebox{!}{8pt}{%
%\AcronymText{NDP-Cloud}}}

%\newcommand{\lNDPCloud}{\resizebox{!}{8.5pt}{%
%\AcronymText{NDP-Cloud}}}

\newcommand{\lNDPCloud}{\lfNDPCloud}


\newcommand{\QNetworkManager}{\resizebox{!}{8pt}{\AcronymText{QNetworkManager}}}
\newcommand{\QTextDocument}{\resizebox{!}{8pt}{\AcronymText{QTextDocument}}}
\newcommand{\QWebEngineView}{\resizebox{!}{8pt}{\AcronymText{QWebEngineView}}}
\newcommand{\HTTP}{\resizebox{!}{8pt}{\AcronymText{HTTP}}}


\newcommand{\lAcronymTextNC}[2]{{\fontfamily{fvs}\selectfont {\Large{#1}}{\large{#2}}}}

\newcommand{\AcronymTextNC}[1]{{\fontfamily{fvs}\selectfont {\large #1}}}

\newcommand{\lunvs}{\vspace{2.5em}}
\newcommand{\lunvss}{\vspace{2em}}
\newcommand{\lunvsa}{\vspace{.25em}}

\colorlet{orr}{orange!60!red}

\newcommand{\textscc}[1]{{\color{orr!35!black}{{%
						\fontfamily{Cabin-TLF}\fontseries{b}\selectfont{\textsc{\scriptsize{#1}}}}}}}


\newcommand{\textsccserif}[1]{{\color{orr!35!black}{{%
				\scriptsize{\textbf{#1}}}}}}


\newcommand{\AcronymText}[1]{{\textscc{#1}}}

\newcommand{\AcronymTextser}[1]{{\textsccserif{#1}}}


\newcommand{\mAcronymText}[1]{{\textscc{\normalsize{#1}}}}

\newcommand{\NAThree}{\resizebox{!}{8pt}{\AcronymText{NA3}}}
\newcommand{\NCN}{\resizebox{!}{8pt}{\AcronymText{NCN}}}
\newcommand{\AThreeR}{\resizebox{!}{8pt}{\AcronymText{A3R}}}

\newcommand{\lQt}{\resizebox{!}{10.5pt}{\AcronymText{Qt}}}

\newcommand{\lNCN}{\resizebox{!}{9pt}{\AcronymText{NCN}}}
\newcommand{\lAThreeR}{\resizebox{!}{9pt}{\AcronymText{A3R}}}
\newcommand{\lNAThree}{\resizebox{!}{9pt}{\AcronymText{NA3}}}

\newcommand{\lsNCN}{\resizebox{!}{10.5pt}{\AcronymText{NCN}}}
\newcommand{\lsAThreeR}{\resizebox{!}{10.5pt}{\AcronymText{A3R}}}
\newcommand{\lsNAThree}{\resizebox{!}{10.5pt}{\AcronymText{NA3}}}

\newcommand{\NGML}{\resizebox{!}{8pt}{\AcronymText{NGML}}}

\newcommand{\Cpp}{\resizebox{!}{8.5pt}{\AcronymText{C++}}}

\newcommand{\WhiteDB}{\resizebox{!}{8pt}{\AcronymText{WhiteDB}}}

\colorlet{drp}{darkRed!70!purple}

%\newcommand{\MOSAIC}{{\color{drp}{\AcronymTextNC{\scriptsize{MOSAIC}}}}}

\newcommand{\MOSAIC}{\resizebox{!}{8pt}{\AcronymText{MOSAIC}}}


\newcommand{\mMOSAIC}{{\color{drp}{\AcronymTextNC{\normalsize{MOSAIC}}}}}

\newcommand{\MOSAICVM}{\mMOSAIC-\mAcronymText{VM}}

\newcommand{\sMOSAICVM}{\resizebox{!}{8pt}{\MOSAICVM}}
\newcommand{\sMOSAIC}{\resizebox{!}{8pt}{\MOSAIC}}


%\newcommand{\lMOSAIC}{{\color{drp}{\lAcronymTextNC{M}{OSAIC}}}}
\newcommand{\lfMOSAIC}{\resizebox{!}{9pt}{{\color{drp}{\lAcronymTextNC{M}{OSAIC}}}}}

\newcommand{\Mosaic}{\resizebox{!}{8pt}{\MOSAIC}}
\newcommand{\MosaicPortal}{{\color{drp}{\AcronymTextNC{MOSAIC Portal}}}}

\newcommand{\RnD}{\resizebox{!}{7.5pt}{\AcronymText{R\&D}}}
\newcommand{\QtCpp}{\resizebox{!}{8.5pt}{\AcronymText{Qt/C++}}}
\newcommand{\Qt}{\resizebox{!}{9pt}{\AcronymText{Qt}}}
\newcommand{\SQL}{\resizebox{!}{8pt}{\AcronymText{SQL}}}

\newcommand{\HTML}{\resizebox{!}{8pt}{\AcronymText{HTML}}}
\newcommand{\PDF}{\resizebox{!}{8pt}{\AcronymText{PDF}}}

\newcommand{\p}{

\vspace{1em}}

\newcommand{\q}[1]{{\fontfamily{qcr}\selectfont ``}#1{\fontfamily{qcr}\selectfont ''}} 

%\newcommand{\deconum}[1]{{\textcircled{#1}}}


\renewcommand{\thesection}{\protect\mbox{\deconum{\Roman{section}}}}
\renewcommand{\thesubsection}{\arabic{section}.\arabic{subsection}}

\newcommand{\llMOSAIC}{\mbox{{\LARGE MOSAIC}}}
%\newcommand{\lfMOSAIC}{\mbox{M\small{OSAIC}}}

\newcommand{\llMosaic}{\llMOSAIC}
\newcommand{\lMosaic}{\lMOSAIC}
\newcommand{\lfMosaic}{\lfMOSAIC}


\newcommand{\llWC}{\mbox{{\LARGE WhiteCharmDB}}}

\newcommand{\llwh}{\mbox{{\LARGE White}}}
\newcommand{\llch}{\mbox{{\LARGE CharmDB}}}

\usepackage{enumitem}

\setlist[description]{%
  topsep=30pt,               % space before start / after end of list
  itemsep=5pt,               % space between items
  font={\bfseries\sffamily}, % set the label font
%  font={\bfseries\sffamily\color{red}}, % if colour is needed
}

\setlist[enumerate]{%
  topsep=3pt,               % space before start / after end of list
  itemsep=-2pt,               % space between items
  font={\bfseries\sffamily}, % set the label font
%  font={\bfseries\sffamily\color{red}}, % if colour is needed
}

%\usepackage{tcolorbox}

\newcommand{\slead}[1]{%
\noindent{\raisebox{2pt}{\relscale{1.15}{{{%
\fcolorbox{logoCyan!50!black}{logoGreen!5}{#1}
}}}}}\hspace{.5em}}


\let\OldLaTeX\LaTeX

\renewcommand{\LaTeX}{\resizebox{!}{8pt}{\color{orr!35!black}{\OldLaTeX}}}

\newcommand{\LargeLaTeX}{\resizebox{!}{8.5pt}{\color{orr!35!black}{\OldLaTeX}}}


\setlength\parindent{36pt}
%%\usepakage{newfile}

\usepackage{hyperref}

\usepackage{etoolbox}

\usepackage{zref-user}

\newwrite\sdiFile
\immediate\openout\sdiFile=\jobname.sdi.txt

\newcommand{\p}[1]{

\vspace{10pt}#1}

\newif\iftabng
\tabngfalse


\usepackage{letltxmacro}
\LetLtxMacro{\oldmmsemi}{\;}
\LetLtxMacro{\oldtbplus}{\+}
\LetLtxMacro{\oldtbgt}{\>}
\LetLtxMacro{\oldmmgt}{\+}

\newcommand{\+}{\iftabng\oldtbplus\else\sss\fi}

\renewcommand{\>}{\iftabng\oldtbplus\else
\ifmmode\oldmmgt\else\sse\sss\fi\fi}

%\renewcommand{\>}{\sse\sss}

\renewcommand{\;}{\relax\ifmmode\oldmmsemi\else\sse\fi}

\newcommand{\writeSDI}[1]{\immediate\write\sdiFile#1}

\newcommand{\emblink}[2]{\href{\#sdi:#1--#2}{\#sdi:#1--#2}}

%\newcount\sdiCounter
%\def\advsdiCounter{\global\advance\sdiCounter by1}

%\newcount\sdiCounterP
%\def\advsdiCounterP{\global\advance\sdiCounterP by1}

%\newcounter{sdiCounter}
\newcounter{sdiCounterP}[page]
\newcounter{sdiCounter}

\def\topt#1{\expandafter\the\dimexpr\dimexpr#1sp\relax\relax}

\makeatletter
%\catcode`\*=10
\newcommand{\sss}{%
\stepcounter{sdiCounterP}
\stepcounter{sdiCounter}
\pdfsavepos\write\sdiFile{!/ SDI_Sentence_Start} 
\write\sdiFile\expandafter{\expandafter$%
\expandafter i\expandafter:%
\expandafter\space\the\c@sdiCounter}
\write\sdiFile\expandafter{\expandafter$%
\expandafter o\expandafter:%
\expandafter\space\the\c@sdiCounterP}
\write\sdiFile\expandafter{\expandafter$%
\expandafter p\expandafter:%
\expandafter\space\thepage^^J%
$x: \topt\pdflastxpos^^J%
$y: \topt\pdflastypos^^J%
/!^^J%
<<>^^J%
}}
%\catcode`\%=14
\makeatother

\makeatletter
\newcommand{\sse}{%
\pdfsavepos\write\sdiFile{!/ SDI_Sentence_End} 
\write\sdiFile\expandafter{\expandafter$%
\expandafter i\expandafter:%
\expandafter\space\the\c@sdiCounter}
\write\sdiFile\expandafter{\expandafter$%
\expandafter o\expandafter:%
\expandafter\space\the\c@sdiCounterP}
\write\sdiFile\expandafter{\expandafter$%
\expandafter p\expandafter:%
\expandafter\space\thepage^^J%
$x: \topt\pdflastxpos^^J%
$y: \topt\pdflastypos^^J%
/!^^J%
<<>^^J%
}}
\makeatother



\newcommand{\lun}[1]{{\fontfamily{qcr}\selectfont{%
\LARGE{\textbf{\underline{#1}}}}}}


\begin{document}
	
{\linespread{1.1}\selectfont

\vspace*{-5em}

\begin{center}
%{\relscale{1.2}{\fontfamily{qcr}\fontseries{b}\selectfont 
%{\colorbox{black}{\color{blue}{\llWC{} Database Engine \\and 
%\llMOSAIC{} Native Application Toolkit}}}}}

\colorlet{ctmp}{logoPeach!20!gray}
\colorlet{ctmpp}{ctmp!90!yellow}
\colorlet{ctmppp}{ctmpp!50!black}
\colorlet{ctmpppp}{ctmppp!90!logoRed}

\vspace{.5em}

%{\colorbox{darkBlGreen!30!darkRed}{%
\begin{tcolorbox}
[
%%enhanced,
%%frame hidden,
%interior hidden
arc=2pt,outer arc=0pt,
enhanced jigsaw,
width=.7\textwidth,
colback=ctmpppp!60,
colframe=logoRed!30!darkRed,
drop shadow=logoPurple!50!darkRed,
%boxsep=0pt,
%left=0pt,
%right=0pt,
%top=2pt,
]
\begin{minipage}{\textwidth}	
\begin{center}		
{\setlength{\fboxsep}{18pt}
	\relscale{1.4}{{\fontfamily{qcr}\fontseries{b}\selectfont%
{\vspace{8pt}\larger{NA3} Investor Prospectus}\vspace{2pt}}}}
\end{center}
\end{minipage}
\end{tcolorbox}
\end{center}

\vspace{-1.25em}

%\noindent\lun{Overview}
\fontfamily{ptm}\fontsize{13pt}{18pt}\selectfont
{\sectsp}
\p{The \NAThree{} platform encompasses both 
server-side components (the \NCN{}, or 
\q{Native/Cloud-Native} framework) and client-side 
components (the \AThreeR{}, or \q{Application-as-a-Resource} 
framework).  This paper will discuss how 
Linguistic Technology Systems, Inc. (LTS) envisions 
marketing and productizing elements of both frameworks, 
so as clarify our implementation strategy and 
business model for \NAThree{}.}

\p{The current document will focus on \NAThree{}'s 
default implementation via the \Qt{} platform.  
\lQt{} is the most widely used native cross-platform 
application development framework, with approximately 
one million active developers, over 5,000 client 
companies, and tens of mullions of downloads of 
recent \Qt{} versions (metrics according to the 
\Qt{} Group).  Companies employing \Qt{} form a natural 
customer base for \NAThree{}.  
The \Qt{}-based market, however, 
is even more substantial than official figures indicate, for 
the following reasons: 

\begin{itemize}[leftmargin=18pt,labelsep=12pt]

\item{} \Qt{} is a commonly used platform for 
embedded systems, touchscreen devices, Cyber-Physical 
devices, and other User Interface environments 
apart from conventional desktop software, with 
particular prominence in such sectors as aeronautics, 
automotive, or military cockpits or consoles; 
wearable devices; biomedical equipment; and 
hybrid mobile/desktop applications.  Most of these 
technologies depend on networking for data management 
and persistence, because embedded systems (in comparison 
to native software on desktop computers) have 
limited \CPU{} and file-system capabilities.

\pseudoIndent{} Despite the significant market presence of 
\Qt{} in the area of User Interface front-ends 
for embedded/touchscreen systems, there is not a 
comparably well-developed \Qt{} server or Cloud-Native 
ecosystem.  The \Qt{} Company launched \q{\Qt{} Cloud 
Services} in 2013, but discontinued this project several years 
later.  The \Qt{} platform provides convenient client libraries 
for Cyber-Physical and web networking protocols such as 
\HTTP{}, \MQTT{}, and \CoAP{}, so \Qt{} front-ends can 
be implemented to network with generic web or \IoT{} 
servers.  Nevertheless, many \Qt{} developers 
have expressed a desire for an integrated environment 
where \Qt{} tools and protocols are usable on both 
server and client sides.  Consequently 
(as the \Qt{} team itself acknowledged when 
promoting \Qt{} Cloud Services: 
\q{\Qt{} Cloud Beta has solved an immense need for \Qt{} developers when it comes to backend-as-a-service and believe that there is an even greater need to provide the \Qt{} ecosystem with an all-in-one \Qt{} solution for cloud computing}), the market 
for hybrid \Qt{}-cloud solutions remains untapped.

\pseudoIndent{} In the context of \NAThree{}, the most 
direct use-case for applications managing embedded systems 
and/or Cyber-Physical data would be integrated solutions 
which include desktop software for realtime or post-hoc 
access to and analysis of Cyber-Physical/\IoT{} information.  
\lNAThree{} is implemented to prioritize desktop clients, 
but it is also quite applicable in the case of 
\q{triple-endpoint} hybrids which integrate embedded 
devices that generate raw data, servers which receive and 
store this data, and desktop clients which manage the data. 

\item{} \Qt{} is the basis 
for one of the two main User Interface technologies employed 
on Linux systems (the other being \GNome{}), so a 
substantial percentage of Linux users run software on 
\Qt{} (often without realizing as much).  Consequently, 
any software implemented to target Linux/\Qt{} Operating 
System environments (such as \KDE{}) is a viable 
candidate to benefit from \NCN{} back-ends or 
\AThreeR{} components.

\item{} \Qt{} is used for many scientific computing 
applications, in numerous disciplines; a representative 
sample includes  
CERN ROOT (CERN's physics/subatomic analytics 
platform), IQmol (for chemistry/molecular physics), 
medInria (for radiology), TeXstudio (for \LaTeX{} processing), 
Mendeley Desktop (for Reference/Citation Management), \QGIS{} (for Geoinformatics), 
MeshLab (for \ThreeD{} modeling), 
OpenSCAD (for \ThreeD{} geometry), 
Octave (a MATLAB emulator), 
ParaView (for data visualization), 
and the \Qt{} Creator \IDE{} (Integrated Development Environment).  
Most of these applications would not be included 
in the \Qt{} company's official user metrics because 
they are maintained by academic or research 
institutions with an open-source \Qt{} license; 
nevertheless, institutions allocate resources 
for keeping their technical applications up-to-date 
with current computing trends.  This kind of 
scientific software provides natural targets for 
integrating desktop applications with \NCN{} back-ends.  
Similarly, \NAThree{} would be a useful toolkit 
for implementing new technical software driven 
by scientific innovations.
\end{itemize}
}

\p{Having emphasized use-cases in the \Qt{} market, 
it is worth adding that the unique technical 
innovations expressed via \NAThree{} have applications 
beyond the \Qt{} ecosystem.  In most cases, \Qt{} 
data types and protocols have corresponding 
equivalents in other application frameworks, 
both cross-platform and Operating-System-specific, 
such as wxWidgets (cross-platform), Xcode 
(Apple) or \MFC{} (Microsoft Foundation Classes).  
A reasonable estimate is that porting \NAThree{} 
to non-\Qt{} platforms would comprise a 
six-month project for a two- or three-person 
development team.}  

\p{In order to stay focused on currently-implemented 
prototypes and the near-term business model, however, 
the remainder of this presentation will restrict 
attention to the \Qt{}-specific version of \NAThree{}.

{\lunvs}
\noindent\lun{\lsNAThree{} Revenue and Marketing}
{\lunvsa}
\p{History suggests that most commercial 
software products generate revenue, in their 
early stages, primarily from customer-specific 
customizations, but then eventually derive 
their most valuable profit-stream from 
commercial licenses.  Special-license customizations 
help mold the product into a widely-usable standard 
version, given the natural feedback loop which emerges 
as the product's development team  
implements project-specific deployments, observing 
\q{in the field} which features are most useful and 
how these features are best made available to 
developers.  As a \q{standardized} version of the 
product rounds into shape, an increasing user-base 
who simply downloads and develops solutions with the 
standard version gradually overtakes, in terms of 
licensing revenue, the organizations who contract 
for customizations.}

\p{Taking the official \q{\Qt{} partners} cohort as a 
representative cross-section of the \Qt{} market, 
we can find companies whose revenue is driven 
by custom software development (ICS, Woboq, KDAB, Base2, 
Bitfactor, Sequality), by commercial licensing 
(Wind River, VNC Automotive, FrogLogic, NXP), 
by realtime and/or platform services (Mender, Timesys, Mapbox), 
as well as hardware/mircroprocessor providers (Toradex, ARM, 
Texas Instruments).  It is probably true that 
companies whose products depend in whole or 
in significant part on \Qt{} generate revenue from 
customization and consulting more than in other 
technology sectors.  This may reflect the position  
of front-end technology in relation to software in general: 
many software projects begin with new kinds of 
data or new user-interaction models, and only 
later address the need for implementing high-quality 
\GUI{}s.  
Given that desktop-style front-end development 
is a rather specialized subdiscipline, many 
companies end up hiring \Qt{}-focused companies 
as service contractors, which in turn supports a robust 
ecosystem of \Qt{} partner companies 
(data from sources such as 
Glassdoor suggest that larger \Qt{} consulting 
partners have revenues roughly comparable to 
\Qt{} itself, indicating that the worldwide 
\Qt{} consulting/contracting market falls in 
the US \$150-\$250 Million range; factoring 
commercial licenses, \Qt{}-enabled 
hardware, and \Qt{}-based software reasonably 
projects the overall \Qt{} market to roughly one-half 
billion US dollars).}

\p{This being said, financial records released by the 
\Qt{} company itself suggest that commercial 
(\q{Developer} and \q{Distribution}) licenses 
are \Qt{}'s largest revenues source (targets 
released in 2018 indicate that The \Qt{} Group Plc 
aims for 60\% revenue from licenses, 20\% from 
consulting, and 20\% for \q{support and maintenance}, 
which is an offshoot of developer licenses; 
total net revenue across these sources from 2018, 
the most recent figures available, was 
\euro{}45.6 Million, just over 
US \$57 Million at 2018 rates).}

\p{It is premature to estimate a comparable partonomy 
of revenue share for \NAThree{}, but we can 
identify four distinct profit streams appropriate for 
\NAThree{} as an integrated platform: 

\begin{enumerate}[itemsep=13pt,
	topsep=14pt,leftmargin=14pt]
\item \textbf{Customization} \hspace{.5em} Custom-implemented applications 
using project-specific versions of \NCN{} and/or \AThreeR{}.

\item \textbf{Licensing}  \hspace{.5em} Commercial licenses required for 
any deployment of \NCN{} outside LTS-controled 
servers and/or any deployment of \AThreeR{} 
applications (or of software including 
\AThreeR{} components for such 
development requirements as databases, data modeling, 
scripting, data serializing/deserialization, 
and text parsing) in a commercial context.

\item \textbf{Hosting}  \hspace{.5em} LTS anticipates running proprietary 
containers via a Cloud-Native service such as 
OpenShift, and then leasing access to this service 
to \NAThree{} users.  LTS can offer integrated hosting and consulting 
wherein LTS fully implements and maintains a back-end 
paired to any desktop/native client software.
Because the expertise involved 
in building native desktop applications is very different 
from the techniques required to deploy a Cloud-Native container 
image, the option of delegating all 
backend responsibilities to LTS may 
appeal to \Qt{}-oriented development teams.

\item \textbf{Sponsorship}  \hspace{.5em} As discussed below, LTS anticipates 
running a data-sharing platform which would be a 
publicly-visible introduction to LTS's in-house 
\NCN{} service (whereas other sub- or para-containers 
would be leased to third parties and provide 
publicly-visible content only at their discretion).  
This \q{demo} container, while being a vehicle for 
the general public to learn about \NAThree{}, 
would also host research data sets and 
would therefore be a resource in the public 
interest, allowing LTS to receive compensation 
from companies financially supporting the 
portal because of its merits as a technology 
benefitting science or scholarship.
\end{enumerate}

\vspace{-12pt}
\p{The remainder of this summary, to elaborate 
further on the hosting and sponsorship 
possibilities, will focus on cloud services 
expressly maintained by LTS (in contrast 
to commercial \NCN{} instances whose licensees 
host the \NCN{} code on their own).  In 
\NCN{} parlance, sub- or para-containers 
are units within a larger \NCN{} environment 
utilizing isolated \HTTP{} access protocols 
and data/file storage.  Each such partial container 
can be twinned with a specific \AThreeR{} 
application, providing a cloud end-point for 
storing application-specific data and/or sharing 
such data between different executable instances 
of the application.  Potentially, then, 
any \AThreeR{} project developed by LTS 
may have a corresponding presence on 
LTS's cloud resources.} 
     
\p{At the same time, the \AThreeR{} (Application-as-a-Resource) 
model also envisions desktop applications as 
self-contained, shareable units, which can 
be hosted on web servers (including \NCN{} instances) 
as zip and metadata files.  Therefore, LTS's 
\NCN{} deployment can serve as an access-point 
for users acquiring or obtaining information about 
\AThreeR{} applications (including data sets 
published as \q{Research Objects} using \AThreeR{} 
within their code base).}

{\lunvss}
\noindent\lun{\lsNAThree{} and the Research Object Protocol}
{\lunvsa}
\p{Open-access data sets conforming to the Research 
Object protocol are a good example of use-cases 
where the \AThreeR{} development strategies may 
be beneficial.  According to the Research Object protocol, 
data sets are paired with code and metadata to help 
subsequent researchers use and interact with the 
published data.  Via \AThreeR{}, the Research Object 
can be implemented as a standalone, desktop-style 
\q{dataset application} whose data models and 
\GUI{} components are uniquely designed for the 
associated data set, reflecting its scientific 
and theoretical provenance (experimental 
setup, data-acquisition methodology, 
data-structural rationale, etc.).  \lAThreeR{} 
employs unusually rigorous modeling for application 
components and data types, which makes it 
particularly appropriate for this Research Object 
context wherein a data set's technology --- its 
structural organization and custom code base 
--- becomes itself a scientific artifact.}

\p{Academic data-hosting is also a sector where 
LTS has a marketing head-start, insofar 
as LTS founder Dr. Amy Neustein serves as editor 
of the International Journal of Speech Technology 
and has authored or edited 14 academic/technical 
volumes.  LTS is currently in discussions with
several publishers to make \AThreeR{} tools available 
to authors for document and/or data-set preparation, collaborations which LTS is pursuing partly 
to introduce \NAThree{} within the scientific 
community and partly to curate and spur the emergence 
of an \AThreeR{} application corpus.}
 
\p{\lAThreeR{} data sets serve two distinct 
purposes in the context of marketing \NAThree{}.  
On the one hand, these data sets may be 
published on respected scientific platforms (notwithstanding 
their being hosted on an \NCN{} service), which 
provides a forum for promoting \NAThree{} 
to the scientific, academic, and Information Technology 
communities (through included \AThreeR{} code as 
well as documentation that will explain both 
the \NCN{} and \AThreeR{} frameworks).  Second, 
open-access \AThreeR{} data sets model a specific 
non-commercial version of \AThreeR{}, which serves as a 
baseline demonstration of \AThreeR{} features.  
The open-source \AThreeR{} implementation provides 
rudimentary support for scripting, data persistence, 
cloud integration, \ThreeD{} graphics, 
embedded web viewers, \GUI{}-based unit and integration testing, 
multi-application networking/workflows, and other features 
often desired for contemporary application development.  
Licensee developers (or LTS itself in a consulting/contracting  
role) can then extend whichever of these features are relevant 
for a commercial project.  The minimal dataset applications 
serve to concretize the overall structure of 
\AThreeR{} software, helping developers and/or 
acquisition teams visualize 
the practical benefits of \AThreeR{} and also 
decide on which \AThreeR{} features they 
will use on a commercial-grade scale in their project.}

\p{Generalizing to \NAThree{} overall, the 
concrete example of \AThreeR{} data sets helps 
illustrate \NCN{} features, because server-side 
capabilities are best understood in terms of 
how they complement client-side User Experience.  
Desirable cloud-integration features include 
persisting one user's application state across 
computers (home/school/office, etc.), sharing 
data in application-specific formats between users, 
collaborative editing, non-local backup, web-searching application corpora, and 
dynamic or non-recompile upgrades to running applications.  
Although such capabilities in the context of 
\NAThree{} depend on properly implemented 
\NCN{} services, \makebox{they can be tangibly demonstrated 
for prospective customers through prototypical 
\AThreeR{} front-ends.}}

\p{In this sense open-access data sets created via 
\AThreeR{} serve as demonstrations and 
concrete examples of \NAThree{} technology, and 
an opportunity to (indirectly) market \NAThree{} 
in the relevant scientific, publishing, and  
computer-scientific communities.  Currently 
three \AThreeR{} data sets have been published 
in the fields of linguistics, speech technology, 
and biomedical Cyber-Physical systems.  Several 
additional data sets will be made publicly 
available in conjunction with the upcoming 
publication of the volume 
(edited by Amy Neustein) 
\textit{Advances in Ubiquitous Computing: 
Cyber-Physical Systems, Smart Cities and 
Ecological Monitoring}, part of Elsevier's 
\q{Ubiquitous Sensing for Healthcare} series.}

\p{One chapter in this Cyber-Physical Systems 
book presents a theoretical outline of the 
tools which form the core of the 
\AThreeR{} framework (a more detailed 
unpublished manuscript covering similar 
topics, \q{Hypergraph 
Data Modeling and a Hypergraph Virtual Machine}, 
is also available on request).  As desired, LTS can 
provide technical information about \NAThree{} 
components, screenshots of \AThreeR{} applications, 
sample data sets, or otherwise respond 
to questions or comments from companies 
who may consider licensing from and/or partnering with 
LTS in conjunction with \NAThree{}.}

%\vspace{2em}
%\noindent\lun{Complex Exolinks and Interoperability}


\end{document}

