\section{Appendix 2: A Hypergraph Vaccine and Immunization Ontology}
\p{The development of a SARS-CoV-2 vaccine is, clearly, 
an addition area of prime focus for Covid-19 research.  
As with radiological outcomes and Clinical Effectiveness 
research, vaccine ontologies are multi-modal because 
the scope of vaccine research spans numerous scientific 
disciplines, including fields where quantitative or 
image-based analysis is methodologically significant.  
Data generated during clinical trials is only one 
part of our overall understanding of the 
immunological properties of different vaccines.}

\p{Vaccine data can be partitioned into information generated 
\textit{before} a vaccine is proven effective --- that is, 
data generated during clinical trials --- and then information 
generated once a vaccine is adopted for immunization campaigns.  
Data in the former (clinical trial) phase tends to fit 
the structure of patient cohorts, demographics, and outcomes, 
but data in the vaccine \q{deployment} phase can be more 
epidemiological and sociological.  In particular, 
modeling immunization drives often depends on gathering 
geographical and political information --- researchers 
need to identify localities where vaccinations are performed, 
and to model the governmental or organizational initiatives 
which support immunization.  Insofar as vaccines attempt 
to prevent the spread of specific diseases, data models 
also need to represent epidemiological details notating 
how well a vaccine has limited disease outbreaks among 
some target population.}

\p{Aside from these geographic, epidemiological, and clinical 
data profiles, vaccine research also addresses the 
immunological mechanisms which determine how a particular 
vaccine works biologically, as well as the level of 
immunity which the vaccine provides for individual 
patients.  Understanding these phenomena requires 
observing patients' immunological response upon receiving 
the vaccine, which generally involves blood and serum 
analysis using a variety of modalities, including 
chemical, genomic, and molecular imagining.  
Consequently, a general-purpose vaccine ontology needs 
the capacity to represent a broad spectrum of 
data which may be presented as evidence of a 
vaccine's immunological effectiveness.}

\p{As with the \RadLex{} and \SeDI{} ontologies 
for radiology, most vaccine-ontology work appears 
focused on reference ontologies, notably the 
\q{VIOLIN} (Vaccine Investigation and Online Information Network) 
Vaccine Ontology (see \bhref{http://www.violinet.org/vaccineontology/}).  
Our strategy for developing
a hypergraph application ontology based on VIOLIN is 
analogous to that of progressing from \RadLex{} and \SeDI{} 
to a hypergraph diagnostic-imaging ontology --- specifically, 
identifying how \RDF{} ontology classes and properties 
can map to annotations on different hypergraph sites.}


