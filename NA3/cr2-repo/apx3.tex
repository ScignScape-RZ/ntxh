\section{Appendix 2: A Hypergraph Vaccine and Immunization Ontology}
\p{As with radiological outcomes and Clinical Effectiveness 
Research, vaccine ontologies are multi-modal because 
the scope of vaccine research spans numerous scientific 
disciplines, including fields where quantitative or 
image-based analysis is methodologically significant.  
As such, data generated during clinical trials is only one 
part of our overall understanding of the 
immunological properties of different vaccines.  
Vaccine data can be partitioned into information generated 
\textit{before} a vaccine is proven effective (that is, 
data generated during clinical trials, including 
challenge trials) and information 
generated \textit{after} a vaccine is adopted for immunization 
campaigns.  Data in the former (clinical trial) phase tends to fit 
the structure of patient cohorts, demographics, and outcomes, 
but data in the vaccine \q{deployment} phase can be more 
epidemiological/sociological by virtue of post-vaccination 
analyses of the effects on the broad population.  In particular, 
modeling immunization drives often depends on gathering 
geographical and political information --- that is, researchers 
need to identify localities where vaccinations are performed 
and to be able to model the governmental/organizational initiatives 
which support immunization.}

\p{Aside from these geographic, epidemiological, and clinical 
data profiles, vaccine research also addresses the 
immunological mechanisms which determine how a particular 
vaccine works biologically, as well as the level of 
immunity which the vaccine provides for individual 
patients.  Understanding these phenomena requires 
observing the immunological response of patients upon 
their receiving the vaccine, which generally involves blood and 
serum analysis --- using a variety of modalities, including 
chemical, genomic, and molecular imaging.  
Consequently, a general-purpose vaccine ontology needs 
the capacity to represent a broad spectrum of 
data which may be presented as evidence of a 
vaccine's immunological effectiveness.  
As such, we will develop a novel ontology, which 
we call \q{Hypergraph Ontology for 
Vaccines and Immunization} (\hVaIm{}) 
so as to integrate vaccinology data 
with \CRtwo{}, which represents one example of an ontology 
meeting these requirements.}

\p{As with the \RadLex{} and \SeDI{} ontologies 
for radiology, most vaccine-ontology work appears 
focused on reference ontologies, notably the 
\q{\VIOLIN{}} Vaccine Ontology 
(Vaccine Investigation and Online Information Network; 
see \bhref{http://www.violinet.org/vaccineontology/}).  
Our strategy for developing
a hypergraph application ontology based on VIOLIN is 
analogous to that of progressing from \RadLex{} and \SeDI{} 
to a hypergraph diagnostic-imaging ontology --- specifically, 
identifying how \RDF{} ontology classes and properties 
can map to annotations on different hypergraph sites.}

\p{Biomedical imaging at the cellular and molecular 
level has become an increasingly important part of 
vaccine research, insofar as new imaging techniques 
allow scientists to directly observe immunological 
responses at the scale of antigens, adjuvants, and 
immune cells:  
\q{monitoring vaccine components, such as antigen 
or adjuvants, and immune cell dynamics at the site of
vaccination or draining lymph nodes can provide important
information to understand more about the vaccine response.} 
As such, \q{a variety of imaging modalities including bioluminescence
imaging, nuclear medicine imaging (such as positron 
emission tomography [PET], single photon emission computed
tomography [SPECT]), and magnetic resonance imaging (MRI)
can provide in vivo non-invasive imaging for visualizing 
immune cell kinetics} 
(see \bhref{https://pubmed.ncbi.nlm.nih.gov/31406689/}).}

\p{The evolution of vaccine research, in effect, is an inversion 
that of diagnostic imaging.  That is, in the context of radiology 
image-based technology became established in previous 
decades and has only recently been augmented with 
data-integration models that allow images to be cross-referenced 
with clinical and outcomes data.  Conversely, in the vaccine 
context, the clinical and epidemiological/sociodemographic 
dimensions of vaccine effectiveness have been modeled 
for decades, but only recently has sophisticated 
medical imaging become part of the vaccine engineering 
arsenal.  Despite these inverted paths, 
one can observe that contemporary radiological and vaccine research 
have a similar technological infrastructure, characterized 
by a digital synthesis of graphical, quantitative, and 
clinical information.}

\p{Given the interdisciplinary nature of 
vaccine research, software engineers developing tools 
for curating data sets involving vaccine research 
should anticipate a similar diversity in the 
information present within repositories which 
include vaccine-related research.  That is 
to say, data-set software applicable in the 
vaccinology context should be able to read 
data in diverse formats which represent the 
scope of vaccine investigations (clinical trials, 
cytometry, blood/serum analysis, bioimaging, etc.).  
Similarly, developers should provide a suite of 
software tools offering interactive visualization 
of this data spectrum, including imaging tools.  
In short, a comprehensive software ecosystem 
for vaccine-related data sets will have an architecture 
similar to that for radiology, suggesting that 
the Procedural Application Ontology model described 
in Appenndix 1 can also be used to describe 
vaccinological research.  
In short, \hVaIm{} and \HDICOM{} will function 
as sibling ontologies, reflecting their 
common role as protocols structuring inter-application 
communications within data repositories which 
include both vaccinological and radiological 
resources.  In particular, both of 
these ontologies will be integrated into 
software tools providing access to 
\CRtwo{} data sets publishing information 
about diagnostic imaging and vaccine engineering 
in the context of Covid-19.}


%, providing analogous 
%models of application data as well as inter-application 
%networking protocols.  }

%As such, software published 
%as part of the \CRtwo{} and \AIMConc{} will then use these 
%ontologies to guide the operations of 
%vaccinology and radiology research tools.}


