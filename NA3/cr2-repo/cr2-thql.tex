\documentclass[11pt,letterpaper]{article}


% pmml  arff  openannotation

%\usepackage[condensed,math]{anttor}
%\usepackage[T1]{fontenc}

%\usepackage[T1]{fontenc}
%\usepackage{tgtermes}

\usepackage[hang,flushmargin]{footmisc}

\usepackage{titlesec}

%\usepackage{sectsty}
%\sectionfont{\fontsize{13}{4}\selectfont}

\titleformat{\section}
  {\normalfont\fontsize{13}{15}\bfseries}{\thesection}{1em}{}

\titlespacing*{\section}
{0pt}{4ex plus 1ex minus .5ex}{.9ex plus .2ex}

%\usepackage{mathptmx}

\usepackage{eso-pic}

%\setlength\parindent{0pt}

\AddToShipoutPictureBG{%

\ifnum\value{page}>1{
\AtTextUpperLeft{
\makebox[20.5cm][r]{
\raisebox{-1.95cm}{%
{\transparent{0.3}{\includegraphics[width=0.29\textwidth]{e-logo.png}}	}} } }
}\fi
}

\AddToShipoutPicture{%
{
 {\color{blGreen!70!red}\transparent{0.9}{\put(0,0){\rule{3pt}{\paperheight}}}}%
 {\color{darkRed!70!purple}\transparent{1}\put(3,0){{\rule{4pt}{\paperheight}}}}
% {\color{logoPeach!80!cyan}\transparent{0.5}{\put(0,700){\rule{1cm}{.6cm}}}}%
% {\color{darkRed!60!cyan}\transparent{0.7}\put(0,706){{\rule{1cm}{.6cm}}}}
% \put(18,726){\thepage}
% \transparent{0.8}
}
}

\AddToShipoutPicture{%
\ifnum\value{page}=1
\put(257.5,932){%
	\transparent{0.7}{
		\includegraphics[width=0.2\textwidth]{logo.png}}}
\put(59,903){\textbf{{\fontfamily{phv}\fontsize{14}{14}\selectfont{}WHITE PAPER}}}
\fi
}	



\AddToShipoutPicture{%
\ifnum\value{page}>1
{\color{blGreen!70!red}\transparent{0.9}{\put(300,8){\rule{0.5\paperwidth}{.3cm}}}}%
{\color{inOne}\transparent{0.8}{\put(300,10){\rule{0.5\paperwidth}{.3cm}}}}%
{\color{inTwo}\transparent{0.3}\put(300,13){{\rule{0.5\paperwidth}{.3cm}}}}

\put(301,16){%
\transparent{0.7}{
\includegraphics[width=0.2\textwidth]{logo.png}} }

\ifnum\value{page}=10
\put(12,51){
 {\setlength{\fboxsep}{.65em}

   {\color{white}{\parbox{11cm}{\vspace{7pt}\framebox{\begin{minipage}{.46\textwidth}
	  {\color{black}{\textit{For more information please contact:}}\\\textbf{\color{blGreen!40!blbl}{Amy Neustein, Ph.D., Founder and CEO}}\\  
       \textbf{{\color{blGreen!20!black}Linguistic Technology Systems \\
      amy.neustein@verizon.net \textbullet{} \textbf{(917) 817-2184} }}}
	 \end{minipage}}}}}}
}
\fi

{\color{blGreen!70!red}\transparent{0.9}{\put(5.6,5){\rule{0.5\paperwidth}{.4cm}}}}%
{\color{inOne}\transparent{1}{\put(5.6,10){\rule{0.5\paperwidth}{.4cm}}}}%
{\color{inTwo}\transparent{0.3}\put(5.6,15){{\rule{0.5\paperwidth}{.4cm}}}}

\fi
}

%\pagestyle{empty} % no page number
%\parskip 7.2pt    % space between paragraphs
%\parindent 12pt   % indent for new paragraph
%\textwidth 4.5in  % width of text
%\columnsep 0.8in  % separation between columns

%\setlength{\footskip}{7pt}

\usepackage[paperheight=14in,paperwidth=8.5in]{geometry}
\geometry{left=.74in,top=.5in,right=.74in,bottom=1.2in} %margins

\usepackage{etoolbox}% http://ctan.org/pkg/etoolbox
\makeatletter
% \patchcmd{<cmd>}{<search>}{<replace>}{<success>}{<failure>}
\patchcmd{\@part}{\par}{\quad}{}{}
\patchcmd{\@part}{\huge}{\Large}{}{}
\makeatother

\renewcommand{\partname}{\hspace{-1em}Part}

\renewcommand*\thepart{\Roman{part}:}

\renewcommand{\thepage}{\raisebox{2pt}{\arabic{page}}}

\renewcommand{\footnoterule}{%
	\kern -3pt
	\hrule width .92\textwidth height .5pt
	\kern 10pt
}


\usepackage[hyphens]{url}
\newcommand{\biburl}[1]{ {\fontfamily{gar}\selectfont{\textcolor[rgb]{.2,.6,0}%
{\scriptsize {\url{#1}}}}}}

%\linespread{1.3}

\newcommand{\sectsp}{\vspace{12pt}}

\usepackage{graphicx}
\usepackage{color,framed}

\usepackage{textcomp}

\usepackage{float}

\usepackage{mdframed}


\usepackage{setspace}
\newcommand{\rpdfNotice}[1]{\begin{onehalfspacing}{

\Large #1

}\end{onehalfspacing}}

\usepackage{xcolor}

\usepackage[hyphenbreaks]{breakurl}
\usepackage[hyphens]{url}

\usepackage{hyperref}
\newcommand{\rpdfLink}[1]{\href{#1}{\small{#1}}}
\newcommand{\dblHref}[1]{\href{#1}{\small{\burl{#1}}}}
\newcommand{\browseHref}[2]{\href{#1}{\Large #2}}

\colorlet{blCyan}{cyan!50!blue}

\definecolor{darkRed}{rgb}{.2,.0,.1}


\definecolor{blGreen}{rgb}{.2,.7,.3}

\definecolor{darkBlGreen}{rgb}{.1,.3,.2}

\definecolor{oldBlColor}{rgb}{.2,.7,.3}

\definecolor{blColor}{rgb}{.1,.3,.2}

\definecolor{elColor}{rgb}{.2,.1,0}
\definecolor{flColor}{rgb}{0.7,0.3,0.3}

\definecolor{logoOrange}{RGB}{108, 18, 30}
\definecolor{logoGreen}{RGB}{85, 153, 89}
\definecolor{logoPurple}{RGB}{200, 208, 30}

\definecolor{logoBlue}{RGB}{4, 2, 25}
\definecolor{logoPeach}{RGB}{255, 159, 102}
\definecolor{logoCyan}{RGB}{66, 206, 244}
\definecolor{logoRed}{rgb}{.3,0,0}

\newcommand{\colorq}[1]{{\color{logoOrange!70!black}{\q{\small\textbf{#1}}}}}

\definecolor{inOne}{rgb}{0.122, 0.435, 0.698}% Rule colour
\definecolor{inTwo}{rgb}{0.122, 0.698, 0.435}% Rule colour

\definecolor{outOne}{rgb}{0.435, 0.698, 0.122}% Rule colour
\definecolor{outTwo}{rgb}{0.698, 0.435, 0.122}% Rule colour

\colorlet{linkcolor}{flColor!60!red}


\hypersetup{
	colorlinks=true,
	citecolor=blCyan!40!green,
	filecolor=magenta!30!logoBlue,
	urlcolor=blue,
    linkcolor=linkcolor!70!black,
%    allcolors=blCyan!40!green
}


\usepackage[many]{tcolorbox}% http://ctan.org/pkg/tcolorbox

\usepackage{transparent}

\newlength{\bsep}
\setlength{\bsep}{-1pt}
\let\xbibitem\bibitem
\renewcommand{\bibitem}[2]{\vspace{\bsep}\xbibitem{#1}{#2}}

\newenvironment{cframed}{\begin{mdframed}[linecolor=logoPeach,linewidth=0.4mm]}{\end{mdframed}}

\newenvironment{ccframed}{\begin{mdframed}[backgroundcolor=logoGreen!5,linecolor=logoCyan!50!black,linewidth=0.4mm]}{\end{mdframed}}


%\usepackage[T1]{fontenc}

%\usepackage{aurical}
% \Fontauri

\usepackage{gfsdidot}
\usepackage[T1]{fontenc}

%\makeatletter
%\f@family,  cmr, T1, n, m,
%\f@encoding,
%\f@shape,
%\f@series,
%\makeatother



%\usepackage{LibreBodoni}

%\usepackage{fontspec}
%\setmainfont{QTBengal}

\usepackage{relsize}

\newcommand{\bref}[1]{\hspace*{1pt}\textbf{\ref{#1}}}

\newcommand{\pseudoIndent}{

\vspace{10pt}\hspace*{12pt}}

\newcommand{\YPDFI}{{\fontfamily{fvs}\selectfont YPDF-Interactive}}

%
\newcommand{\deconum}[1]{{\protect\raisebox{-1pt}{{\LARGE #1}}}}

\newcommand{\visavis}{vis-\`a-vis}

\newcommand{\VersatileUX}{{\color{red!85!black}{\Fontauri Versatile}}%
{{\fontfamily{qhv}\selectfont\smaller UX}}}

\newcommand{\NDPCloud}{{\color{red!15!black}%
{\fontfamily{qhv}\selectfont {\smaller NDP C{\smaller LOUD}}}}}

\newcommand{\MThreeK}{{\color{blGreen!45!black}%
{\fontfamily{qhv}\fontsize{10}{8}\selectfont {M3K}}}}


\newcommand{\lfNDPCloud}{{\color{red!15!black}%
{\fontfamily{qhv}\selectfont N{\smaller DP C{\smaller LOUD}}}}}

\newcommand{\textds}[1]{{\fontfamily{lmdh}\selectfont{%
\raisebox{-1pt}{#1}}}}

\newcommand{\dsC}{{\textds{ds}{\fontfamily{qhv}\selectfont \raisebox{-1pt}
{\color{red!15!black}{C}}}}}

\definecolor{tcolor}{RGB}{24,52,61}

\newcommand{\CCpp}{\resizebox{!}{7pt}{\AcronymText{C}}/\Cpp{}}
\newcommand{\NoSQL}{\resizebox{!}{7pt}{\AcronymText{NoSQL}}}
\newcommand{\SQL}{\resizebox{!}{7pt}{\AcronymText{SQL}}}

\newcommand{\SPARQL}{\resizebox{!}{7pt}{\AcronymText{SPARQL}}}

\newcommand{\NCBI}{\resizebox{!}{7pt}{\AcronymText{NCBI}}}

\newcommand{\HTXN}{\resizebox{!}{7pt}{\AcronymText{HTXN}}}

\newcommand{\TDM}{\resizebox{!}{7pt}{\AcronymText{TDM}}}

\newcommand{\lHTXN}{\resizebox{!}{7.5pt}{\AcronymText{H}}%
\resizebox{!}{6.5pt}{\AcronymText{TXN}}}

\newcommand{\lsHTXN}{\resizebox{!}{9.5pt}{\AcronymText{\textcolor{tcolor}{HTXN}}}}

\newcommand{\LAF}{\resizebox{!}{7pt}{\AcronymText{LAF}}}

\newcommand{\UDpipe}{\resizebox{!}{7pt}{\AcronymText{UDpipe}}}

\newcommand{\C}{\resizebox{!}{7pt}{\AcronymText{C}}}


\usepackage{mdframed}

\newcommand{\cframedboxpanda}[1]{\begin{mdframed}[linecolor=yellow!70!blue,linewidth=0.4mm]#1\end{mdframed}}


\newcommand{\PVD}{\resizebox{!}{7pt}{\AcronymText{PVD}}}

\newcommand{\THQL}{\resizebox{!}{7pt}{\AcronymText{THQL}}}
\newcommand{\lTHQL}{\resizebox{!}{7.5pt}{\AcronymText{THQL}}}

\newcommand{\SDK}{\resizebox{!}{7pt}{\AcronymText{SDK}}}
\newcommand{\NLP}{\resizebox{!}{7pt}{\AcronymText{NLP}}}

\newcommand{\AXF}{\resizebox{!}{7pt}{\AcronymText{AXF}}}

\newcommand{\lAXF}{\resizebox{!}{7.5pt}{\AcronymText{A}}%
\resizebox{!}{6.5pt}{\AcronymText{XF}}}


\newcommand{\lsAXF}{\resizebox{!}{8.5pt}{\AcronymText{AXF}}}

\newcommand{\AXFD}{\resizebox{!}{7pt}{\AcronymText{AXFD}}}

\newcommand{\lAXFD}{\resizebox{!}{7.5pt}{\AcronymText{A}}%
\resizebox{!}{6.5pt}{\AcronymText{XFD}}}


\newcommand{\IJST}{\resizebox{!}{7pt}{\AcronymText{IJST}}}

\newcommand{\BioC}{\resizebox{!}{7pt}{\AcronymText{BioC}}}

\newcommand{\CoNLL}{\resizebox{!}{7pt}{\AcronymText{CoNLL}}}
\newcommand{\CoNLLU}{\resizebox{!}{7pt}{\AcronymText{CoNLL-U}}}

\newcommand{\sapp}{\resizebox{!}{7pt}{\AcronymText{Sapien+}}}
\newcommand{\lsapp}{\resizebox{!}{8.5pt}{\AcronymText{Sapien+}}}
\newcommand{\lssapp}{\resizebox{!}{9.5pt}{\AcronymText{Sapien+}}}

\newcommand{\ePub}{\resizebox{!}{7pt}{\AcronymText{ePub}}}

%\lsLPF


\newcommand{\GIT}{\resizebox{!}{7pt}{\AcronymText{GIT}}}

%\definecolor{atColor}{RGB}{11, 71, 17}


\DeclareMathVersion{fordg}
\SetSymbolFont{letters}{fordg}{OML}{cmr}{b}{n}

\definecolor{atColor}{RGB}{50, 22, 40}
\newcommand{\ATextClr}[1]{\textcolor{atColor}{\textbf{#1}}}

\newcommand{\DgDb}{{\mathversion{fordg}%
\makebox{\raisebox{-3pt}{\resizebox{!}{11pt}{\ATextClr{%
\rotatebox{17}{$\varsigma$}}}}\hspace{-4pt}%
\resizebox{!}{6.5pt}{\ATextClr{D\hspace{-2pt}B}}}}}

\newcommand{\lDgDb}{{\mathversion{fordg}%
\resizebox{!}{12pt}{\ATextClr{%
\rotatebox{17}{$\varsigma$}}}}\hspace{-4pt}%
\resizebox{!}{6.5pt}{\ATextClr{D\hspace{-2pt}B}}}}}

\newcommand{\URL}{\resizebox{!}{7pt}{\AcronymText{URL}}}
\newcommand{\CSML}{\resizebox{!}{7pt}{\AcronymText{CSML}}}
\newcommand{\LPF}{\resizebox{!}{7pt}{\AcronymText{LPF}}}
\newcommand{\lLPF}{\resizebox{!}{8.5pt}{\AcronymText{LPF}}}
\newcommand{\lsLPF}{\resizebox{!}{9.5pt}{\AcronymText{LPF}}}

\makeatletter

\newcommand*\getX[1]{\expandafter\getX@i#1\@nil}

\newcommand*\getY[1]{\expandafter\getY@i#1\@nil}
\def\getX@i#1,#2\@nil{#1}
\def\getY@i#1,#2\@nil{#2}
\makeatother
	
\newcommand{\rectann}[9]{%
\path [draw=#1,draw opacity=#2,line width=#3, fill=#4, fill opacity = #5, even odd rule] %
(#6) rectangle(\getX{#6}+#7,\getY{#6}+#8)
({\getX{#6}+((#7-(#7*#9))/2)},{\getY{#6}+((#8-(#8*#9))/2)}) rectangle %
({\getX{#6}+((#7-(#7*#9))/2)+#7*#9},{\getY{#6}+((#8-(#8*#9))/2)+#8*#9});}


\definecolor{pfcolor}{RGB}{94, 54, 73}

\newcommand{\EPF}{\resizebox{!}{7pt}{\AcronymText{ETS{\color{pfcolor}pf}}}}
\newcommand{\lEPF}{\resizebox{!}{8.5pt}{\AcronymText{ETS{\color{pfcolor}pf}}}}
\newcommand{\lsEPF}{\resizebox{!}{9.5pt}{\AcronymText{ETS{\color{pfcolor}pf}}}}


\newcommand{\XPDF}{\resizebox{!}{7pt}{\AcronymText{XPDF}}}

\newcommand{\GRE}{\resizebox{!}{7pt}{\AcronymText{GRE}}}
\newcommand{\CAS}{\resizebox{!}{7pt}{\AcronymText{CAS}}}

\newcommand{\lMOSAIC}{%
\resizebox{!}{8pt}{\AcronymText{M}}%
\resizebox{!}{6pt}{\AcronymText{OSAIC}}}

\newcommand{\XML}{\resizebox{!}{7pt}{\AcronymText{XML}}}
\newcommand{\RDF}{\resizebox{!}{7pt}{\AcronymText{RDF}}}
\newcommand{\DOM}{\resizebox{!}{7pt}{\AcronymText{DOM}}}

\newcommand{\Covid}{\resizebox{!}{7pt}{\AcronymText{Covid-19}}}

\newcommand{\CLang}{\resizebox{!}{7pt}{\AcronymText{C}}}

\newcommand{\HNaN}{\resizebox{!}{7pt}{\AcronymText{HN%
\textsc{a}N}}}

\newcommand{\JSON}{\resizebox{!}{7pt}{\AcronymText{JSON}}}
\newcommand{\UV}{\resizebox{!}{7pt}{\AcronymText{UV}}}


\newcommand{\MeshLab}{\resizebox{!}{7pt}{\AcronymText{MeshLab}}}
\newcommand{\IQmol}{\resizebox{!}{7pt}{\AcronymText{IQmol}}}

\newcommand{\SGML}{\resizebox{!}{7pt}{\AcronymText{SGML}}}

\newcommand{\WhiteDB}{\makebox{WhiteDB}}

\newcommand{\ASCII}{\resizebox{!}{7pt}{\AcronymText{ASCII}}}

\newcommand{\GUI}{\resizebox{!}{7pt}{\AcronymText{GUI}}}

\newcommand{\API}{\resizebox{!}{7pt}{\AcronymText{API}}}

\newcommand{\JATS}{\resizebox{!}{7pt}{\AcronymText{JATS}}}


\newcommand{\SDI}{\resizebox{!}{7pt}{\AcronymText{SDI}}}
\newcommand{\SDIV}{\resizebox{!}{7pt}{\AcronymText{SDIV}}}

\definecolor{atColor}{RGB}{50, 22, 40}
\newcommand{\ATextClr}[1]{\textcolor{atColor}{\textbf{#1}}}

\newcommand{\DgDb}{\makebox{\raisebox{-3pt}{\resizebox{!}{11pt}{\ATextClr{%
\rotatebox{17}{$\varsigma$}}}}\hspace{-4pt}%
\resizebox{!}{6.5pt}{\ATextClr{D\hspace{-2pt}B}}}}

\newcommand{\lDgDb}{\makebox{\raisebox{-3pt}{%
\resizebox{!}{12pt}{\ATextClr{%
\rotatebox{17}{$\varsigma$}}}}\hspace{-4pt}%
\resizebox{!}{6.5pt}{\ATextClr{D\hspace{-2pt}B}}}}


\newcommand{\IDE}{\resizebox{!}{7pt}{\AcronymText{IDE}}}

\newcommand{\ThreeD}{\resizebox{!}{7pt}{\AcronymText{3D}}}

\newcommand{\FAIR}{\resizebox{!}{7pt}{\AcronymText{FAIR}}}

\newcommand{\QNetworkManager}{\resizebox{!}{7pt}{\AcronymText{QNetworkManager}}}
\newcommand{\QTextDocument}{\resizebox{!}{7pt}{\AcronymText{QTextDocument}}}
\newcommand{\QWebEngineView}{\resizebox{!}{7pt}{\AcronymText{QWebEngineView}}}
\newcommand{\HTTP}{\resizebox{!}{7pt}{\AcronymText{HTTP}}}


\newcommand{\lAcronymTextNC}[2]{{\fontfamily{fvs}\selectfont {\Large{#1}}{\large{#2}}}}

\newcommand{\AcronymTextNC}[1]{{\fontfamily{fvs}\selectfont {\large #1}}}


\colorlet{orr}{orange!60!red}

\newcommand{\textscc}[1]{{\color{orr!35!black}{{%
						\fontfamily{Cabin-TLF}\fontseries{b}\selectfont{\textsc{\scriptsize{#1}}}}}}}


\newcommand{\textsccserif}[1]{{\color{orr!35!black}{{%
				\scriptsize{\textbf{#1}}}}}}


\newcommand{\iXPDF}{\resizebox{!}{7pt}{\textsccserif{%
\textit{XPDF}}}}

\newcommand{\iEPF}{\resizebox{!}{7pt}{\textsccserif{%
\textit{ETSpf}}}}

\newcommand{\iSDI}{\resizebox{!}{7pt}{\textsccserif{%
\textit{SDI}}}}

\newcommand{\iHTXN}{\resizebox{!}{7pt}{\textsccserif{%
\textit{HTXN}}}}


\newcommand{\AcronymText}[1]{{\textscc{#1}}}

\newcommand{\AcronymTextser}[1]{{\textsccserif{#1}}}


\newcommand{\mAcronymText}[1]{{\textscc{\normalsize{#1}}}}

\newcommand{\FASTA}{{\resizebox{!}{7pt}{\AcronymText{FASTA}}}}
\newcommand{\SRA}{{\resizebox{!}{7pt}{\AcronymText{SRA}}}}
\newcommand{\DNA}{{\resizebox{!}{7pt}{\AcronymText{DNA}}}}
\newcommand{\MAP}{{\resizebox{!}{7pt}{\AcronymText{MAP}}}}
\newcommand{\EPS}{{\resizebox{!}{7pt}{\AcronymText{EPS}}}}
\newcommand{\CSV}{{\resizebox{!}{7pt}{\AcronymText{CSV}}}}
\newcommand{\PDB}{{\resizebox{!}{7pt}{\AcronymText{PDB}}}}

\newcommand{\XOCS}{{\resizebox{!}{7pt}{\AcronymText{XOCS}}}}

\newcommand{\HGXF}{{\resizebox{!}{7pt}{\AcronymText{HGXF}}}}
\newcommand{\lHGXF}{{\resizebox{!}{7.5pt}{\AcronymText{HGXF}}}}
\newcommand{\sHGXF}{{\resizebox{!}{5.5pt}{\AcronymText{HGXF}}}}

\newcommand{\CRtwo}{{\resizebox{!}{7pt}{\AcronymText{CR2}}}}
\newcommand{\lCRtwo}{{\resizebox{!}{7.5pt}{\AcronymText{CR2}}}}
\newcommand{\sCRtwo}{{\resizebox{!}{5.5pt}{\AcronymText{CR2}}}}

\newcommand{\ChemXML}{{\resizebox{!}{7pt}{\AcronymText{ChemXML}}}}

\newcommand{\TeXMECS}{\resizebox{!}{7pt}{\AcronymText{TeXMECS}}}

% pmml  arff  openannotation

\newcommand{\PMML}{\resizebox{!}{7pt}{\AcronymText{PMML}}}
\newcommand{\ARFF}{\resizebox{!}{7pt}{\AcronymText{ARFF}}}
\newcommand{\IeXML}{\resizebox{!}{7pt}{\AcronymText{IeXML}}}


\newcommand{\NGML}{\resizebox{!}{7pt}{\AcronymText{NGML}}}

\newcommand{\Cpp}{\resizebox{!}{7pt}{\AcronymText{C++}}}

%\newcommand{\\WhiteDB{}}{\resizebox{!}{7pt}{\AcronymText{\WhiteDB{}}}}

\colorlet{drp}{darkRed!70!purple}

%\newcommand{\MOSAIC}{{\color{drp}{\AcronymTextNC{\scriptsize{MOSAIC}}}}}

\newcommand{\MOSAIC}{\resizebox{!}{7pt}{\AcronymText{MOSAIC}}}


\newcommand{\mMOSAIC}{{\color{drp}{\AcronymTextNC{\normalsize{MOSAIC}}}}}

\newcommand{\MOSAICVM}{\mMOSAIC-\mAcronymText{VM}}

\newcommand{\sMOSAICVM}{\resizebox{!}{7pt}{\MOSAICVM}}
\newcommand{\sMOSAIC}{\resizebox{!}{7pt}{\MOSAIC}}

\newcommand{\LDOM}{\resizebox{!}{7pt}{\AcronymText{LDOM}}}
\newcommand{\Cnineteen}{\resizebox{!}{7pt}{\AcronymText{CORD-19}}}

\newcommand{\lCnineteen}{\resizebox{!}{7.5pt}{\AcronymText{CORD-19}}}


\newcommand{\MOL}{\resizebox{!}{7pt}{\AcronymText{MOL}}}

\newcommand{\ACL}{\resizebox{!}{7pt}{\AcronymText{ACL}}}

\newcommand{\LXCR}{\resizebox{!}{7pt}{\AcronymText{LXCR}}}
\newcommand{\lLXCR}{\resizebox{!}{8.5pt}{\AcronymText{LXCR}}}
\newcommand{\lsLXCR}{\resizebox{!}{9.5pt}{\AcronymText{LXCR}}}

%\newcommand{\lMOSAIC}{{\color{drp}{\lAcronymTextNC{M}{OSAIC}}}}
\newcommand{\lfMOSAIC}{\resizebox{!}{9pt}{{\color{drp}{\lAcronymTextNC{M}{OSAIC}}}}}

\newcommand{\Mosaic}{\resizebox{!}{7pt}{\MOSAIC}}
\newcommand{\MosaicPortal}{{\color{drp}{\AcronymTextNC{MOSAIC Portal}}}}

\newcommand{\RnD}{\resizebox{!}{7pt}{\AcronymText{R\&D}}}

\newcommand{\lQt}{\resizebox{!}{8.5pt}{\AcronymText{Qt}}}
\newcommand{\QtCpp}{\resizebox{!}{8.5pt}{\AcronymText{Qt/C++}}}
\newcommand{\Qt}{\resizebox{!}{7pt}{\AcronymText{Qt}}}

\newcommand{\QtSQL}{\resizebox{!}{7pt}{\AcronymText{QtSQL}}}

\newcommand{\HTML}{\resizebox{!}{7pt}{\AcronymText{HTML}}}
\newcommand{\PDF}{\resizebox{!}{7pt}{\AcronymText{PDF}}}

\newcommand{\R}{\resizebox{!}{7pt}{\AcronymText{R}}}
\newcommand{\SciXML}{\resizebox{!}{7pt}{\AcronymText{SciXML}}}



\newcommand{\lGRE}{\resizebox{!}{7.5pt}{\AcronymText{GRE}}}

\newcommand{\p}[1]{

\vspace{.6em}#1}

\newcommand{\q}[1]{{\fontfamily{qcr}\selectfont ``}#1{\fontfamily{qcr}\selectfont ''}} 

%\newcommand{\deconum}[1]{{\textcircled{#1}}}

\renewcommand{\thesection}{\protect\hspace{-1.5em}}
%\renewcommand{\thesection}{\protect\mbox{\deconum{\Roman{section}}}}
\renewcommand{\thesubsection}{\arabic{section}.\arabic{subsection}}

\newcommand{\llMOSAIC}{\mbox{{\LARGE MOSAIC}}}
%\newcommand{\lfMOSAIC}{\mbox{M\small{OSAIC}}}

\newcommand{\llMosaic}{\llMOSAIC}
\newcommand{\lMosaic}{\lMOSAIC}
\newcommand{\lfMosaic}{\lfMOSAIC}

%\newcommand{\dsC}{}

\newcommand{\textds}[1]{{\fontfamily{lmdh}\selectfont{%
\raisebox{-1pt}{#1}}}}

\newcommand{\ltextds}[1]{{\fontfamily{lmdh}\fontsize{12}{11}\selectfont{%
\raisebox{-1pt}{#1}}}}

\newcommand{\dsC}{{\textds{ds}{\fontfamily{qhv}\selectfont \raisebox{-1pt}{C}}}}
\newcommand{\ldsC}{{\textds{ds}{\fontfamily{qhv}\selectfont \raisebox{-1pt}{C}}}}

\newcommand{\llWC}{\mbox{{\LARGE WhiteCharmDB}}}

\newcommand{\llwh}{\mbox{{\LARGE White}}}
\newcommand{\llch}{\mbox{{\LARGE CharmDB}}}

\usepackage{enumitem}
%\usepackage{listings}

\colorlet{dsl}{purple!20!brown}
\colorlet{dslr}{dsl!50!blue}

\setlist[description]{%
  topsep=10pt,
  labelsep=22pt, leftmargin=5pt,
  itemsep=5pt,               % space between items
  %font={\bfseries\sffamily}, % set the label font
  font=\normalfont\bfseries\color{dslr!50!black}, % if colour is needed
}

\setlist[enumerate]{%
  topsep=3pt,               % space before start / after end of list
  itemsep=-2pt,               % space between items
  font={\bfseries\sffamily}, % set the label font
%  font={\bfseries\sffamily\color{red}}, % if colour is needed
}

%\usepackage{tcolorbox}

\newcommand{\slead}[1]{%
\noindent{\raisebox{2pt}{\relscale{1.15}{{{%
\fcolorbox{logoCyan!50!black}{logoGreen!5}{#1}
}}}}}\hspace{.5em}}


\let\OldLaTeX\LaTeX

\renewcommand{\LaTeX}{\resizebox{!}{7pt}{\color{orr!35!black}{\OldLaTeX}}}

\let\OldTeX\TeX

\renewcommand{\TeX}{\resizebox{!}{7pt}{\color{orr!35!black}{\OldTeX}}}


\newcommand{\LargeLaTeX}{\resizebox{!}{8.5pt}{\color{orr!35!black}{\OldLaTeX}}}

\setlength\parindent{0pt}
%\setlength\parindent{24pt}
%%\usepakage{newfile}

\usepackage{hyperref}

\usepackage{etoolbox}

\usepackage{zref-user}

\newwrite\sdiFile
\immediate\openout\sdiFile=\jobname.sdi.txt

\newcommand{\p}[1]{

\vspace{10pt}#1}

\newif\iftabng
\tabngfalse


\usepackage{letltxmacro}
\LetLtxMacro{\oldmmsemi}{\;}
\LetLtxMacro{\oldtbplus}{\+}
\LetLtxMacro{\oldtbgt}{\>}
\LetLtxMacro{\oldmmgt}{\+}

\newcommand{\+}{\iftabng\oldtbplus\else\sss\fi}

\renewcommand{\>}{\iftabng\oldtbplus\else
\ifmmode\oldmmgt\else\sse\sss\fi\fi}

%\renewcommand{\>}{\sse\sss}

\renewcommand{\;}{\relax\ifmmode\oldmmsemi\else\sse\fi}

\newcommand{\writeSDI}[1]{\immediate\write\sdiFile#1}

\newcommand{\emblink}[2]{\href{\#sdi:#1--#2}{\#sdi:#1--#2}}

%\newcount\sdiCounter
%\def\advsdiCounter{\global\advance\sdiCounter by1}

%\newcount\sdiCounterP
%\def\advsdiCounterP{\global\advance\sdiCounterP by1}

%\newcounter{sdiCounter}
\newcounter{sdiCounterP}[page]
\newcounter{sdiCounter}

\def\topt#1{\expandafter\the\dimexpr\dimexpr#1sp\relax\relax}

\makeatletter
%\catcode`\*=10
\newcommand{\sss}{%
\stepcounter{sdiCounterP}
\stepcounter{sdiCounter}
\pdfsavepos\write\sdiFile{!/ SDI_Sentence_Start} 
\write\sdiFile\expandafter{\expandafter$%
\expandafter i\expandafter:%
\expandafter\space\the\c@sdiCounter}
\write\sdiFile\expandafter{\expandafter$%
\expandafter o\expandafter:%
\expandafter\space\the\c@sdiCounterP}
\write\sdiFile\expandafter{\expandafter$%
\expandafter p\expandafter:%
\expandafter\space\thepage^^J%
$x: \topt\pdflastxpos^^J%
$y: \topt\pdflastypos^^J%
/!^^J%
<<>^^J%
}}
%\catcode`\%=14
\makeatother

\makeatletter
\newcommand{\sse}{%
\pdfsavepos\write\sdiFile{!/ SDI_Sentence_End} 
\write\sdiFile\expandafter{\expandafter$%
\expandafter i\expandafter:%
\expandafter\space\the\c@sdiCounter}
\write\sdiFile\expandafter{\expandafter$%
\expandafter o\expandafter:%
\expandafter\space\the\c@sdiCounterP}
\write\sdiFile\expandafter{\expandafter$%
\expandafter p\expandafter:%
\expandafter\space\thepage^^J%
$x: \topt\pdflastxpos^^J%
$y: \topt\pdflastypos^^J%
/!^^J%
<<>^^J%
}}
\makeatother




\newcommand{\lun}[1]{\raisebox{-4pt}{\fontfamily{qcr}\selectfont{%
\LARGE{\textbf{\textcolor{tcolor}{#1}}}}}\vspace{-2pt}}

\newcommand{\inditem}{\itemindent10pt\item}

\usepackage{soul}

\definecolor{hlcolor}{RGB}{114, 54, 203}
\colorlet{hlcol}{hlcolor!35}
\sethlcolor{hlcol}

\makeatletter
\def\SOUL@hlpreamble{%
	\setul{}{3ex}%         !!!change this value!!! default is 2.5ex
	\let\SOUL@stcolor\SOUL@hlcolor
	\SOUL@stpreamble
}
\makeatother

\usepackage{scrextend}
%\vspace*{3em}
\newenvironment{mldescription}{\vspace{1em}%
  \begin{addmargin}[4pt]{1em}
    \setlength{\parindent}{-1em}%
    \newcommand*{\mlitem}[1][]{\vspace{5pt}\par\medskip%
%\colorbox{hlcolor}{\textbf{##1}}\quad}\indent
\hl{ \textbf{##1} }\quad}\indent
}{%
  \end{addmargin}
  \medskip
}

\usepackage{marginnote}

\newcommand{\mnote}[1]{%
\vspace*{-2em}
\reversemarginpar
\raisebox{1em}{\marginnote{\parbox{4em}{%
\begin{mdframed}[innerleftmargin=4pt,
	innerrightmargin=1pt,innertopmargin=1pt,
	linecolor=red!20!cyan,userdefinedwidth=4em,
	topline=false,
	rightline=false]
{{\fontfamily{ppl}\fontsize{12}{0}\selectfont
		\textit{#1}}}
\end{mdframed}}
	}[3em]}}

\newcommand{\mnotel}[1]{%
\vspace*{-2em}
\reversemarginpar
\raisebox{-4em}{\marginnote{\parbox{4em}{%
\begin{mdframed}[innerleftmargin=4pt,
	innerrightmargin=1pt,innertopmargin=1pt,
	linecolor=red!20!cyan,userdefinedwidth=4em,
	topline=false,
	rightline=false]
{{\fontfamily{ppl}\fontsize{12}{0}\selectfont
		\textit{#1}}}
\end{mdframed}}
	}[3em]}}

\newcommand{\mnoteh}[3]{%
	\vspace*{#1}
	\reversemarginpar
	\raisebox{#2}{\marginnote{\parbox{4em}{%
				\begin{mdframed}[innerleftmargin=4pt,
					innerrightmargin=1pt,innertopmargin=1pt,
					linecolor=red!20!cyan,userdefinedwidth=4em,
					topline=false,
					rightline=false]
					{{\fontfamily{ppl}\fontsize{12}{0}\selectfont
							\textit{#3}}}
				\end{mdframed}}
			}[3em]}}


\newcommand{\mnoteb}[1]{%
	\vspace*{1em}
	\reversemarginpar
	\raisebox{1em}{\marginnote{\parbox{4em}{%
				\begin{mdframed}[innerleftmargin=4pt,
					innerrightmargin=1pt,innertopmargin=1pt,
					linecolor=red!20!cyan,userdefinedwidth=4em,
					topline=false,
					rightline=false]
					{{\fontfamily{ppl}\fontsize{12}{0}\selectfont
							\textit{#1}}}
				\end{mdframed}}
			}[3em]}}
	
\usepackage{wrapfig}

\usetikzlibrary{arrows, decorations.markings}
\usetikzlibrary{shapes.arrows}

\newcommand{\curicon}[2]{%
	\node at (#1,#2) [
	draw=black,
	%minimum width=2ex,
	inner sep=.7pt,
	fill=white,
	single arrow,
	single arrow head extend=3pt,
	single arrow head indent=1.5pt,
	single arrow tip angle=45,
	line join=bevel,
	minimum height=4.6mm,
	rotate=115
	] {};
}

\makeatletter
\def\@cite#1#2{[\textbf{#1\if@tempswa , #2\fi}]}
\def\@biblabel#1{[\textbf{#1}]}
\makeatother


%\let\origref\ref
%\renewcommand{\ref}[1]{{\LARGE #1}}

%\def\ref#1{\textbf{\origref{{\LARGE #1}}}}

\setlength{\footnotesep}{0pt}

\renewcommand{\thefootnote}{\textcolor{logoGreen!80!logoBlue}{{\fontfamily{qcr}\fontseries{b}\fontsize{10}{4}\selectfont\arabic{footnote}}}}


\newcommand{\LVee}{{\colorbox{cyan!40!yellow}{\textcolor{red!70!navy}{\textbf{\LARGE$\vee$}}}}}
\newcommand{\LWedge}{{\colorbox{cyan!40!yellow}{\textcolor{red!70!navy}{\textbf{\LARGE$\wedge$}}}}}

\renewcommand{\LVee}{}
\renewcommand{\LWedge}{}


\urlstyle{same}

%\setmainfont{QTChanceryType}

\begin{document}

\setlength{\skip\footins}{18pt}	
	
{\linespread{1.2}\selectfont

\vspace*{1.5em}

\begin{center}
%{\relscale{1.2}{\fontfamily{qcr}\fontseries{b}\selectfont 
%{\colorbox{black}{\color{blue}{\llWC{} Database Engine \\and 
%\llMOSAIC{} Native Application Toolkit}}}}}

\colorlet{ctmp}{logoPeach!20!gray}
\colorlet{ctmpp}{ctmp!90!yellow}
\colorlet{ctmppp}{ctmpp!50!black}
\colorlet{ctmpppp}{ctmppp!90!logoRed}
\colorlet{ctmcyan}{ctmpppp!70!cyan}

\colorlet{ctmppppy}{ctmppp!60!orange}

\vspace{6.7em}

%{\colorbox{darkBlGreen!30!darkRed}{%
\begin{tcolorbox}
[
%%enhanced,
%%frame hidden,
%interior hidden
arc=2pt,outer arc=0pt,
enhanced jigsaw,
width=\textwidth,
colback=ctmppppy!40,
%colback=ctmcyan!50,
colframe=logoRed!30!darkRed,
drop shadow=logoPurple!50!darkRed,
%boxsep=0pt,
%left=0pt,
%right=0pt,
%top=2pt,
]
%\hspace{22pt}
\begin{minipage}{\textwidth}	
\begin{center}	
{\setlength{\fboxsep}{32pt}
	\relscale{1.2}{{\fontfamily{qcr}\fontseries{b}\selectfont%
{New Database Engineering and 
Archive Construction Technology to Accelerate Covid-19 Research}
}}}
\end{center}
\end{minipage}
\end{tcolorbox}
\end{center}

\vspace{1em}
\vspace*{12pt}
\begin{center}
\parbox{.88\textwidth}{%
{\fontfamily{fvs}\fontsize{9}{9}\selectfont   
LTS (Linguistic Technology Systems) is founded by 
Amy Neustein, PhD, Series Editor of {\textbf{Speech Technology 
and Text Mining in Medicine and Health Care}} (de Gruyter); 
Editor of {\textbf{Advances in Ubiquitous Computing: 
Cyber-Physical Systems, Smart Cities, 
and Ecological \makebox{Monitoring}}} 
(Elsevier, 2020); and 
co-author (with Nathaniel Christen) 
of {\textbf{Cross-Disciplinary Data Integration Models
for the Emerging Covid-19 Data Ecosystem}} 
(Elsevier, forthcoming).}}\end{center}

\vspace*{1em}	

\section{Introduction}
\p{LTS is curating a Cross-Disciplinary Archive/Repository 
for Covid-19 Research (\CRtwo{}), a collection 
of open-access research data sets related to 
SARS-CoV-2 and Covid-19.  \lCRtwo{} is being developed 
as a supplement to our forthcoming Elsevier volume 
examining Covid-19 research from the perspective 
of text and data mining.  In building this repository, 
LTS will develop and demonstrate 
new database engineering and dataset/repository construction 
technologies.  Part I of this paper will outline the repository;  
and Part II will discuss the new technologies 
in greater detail.  Such database engineering and archive 
construction technologies have deep market penetration 
in many areas, including but not limited to 
pharmaceuticals, manufacturing, fintech, 
healthcare, educational software, and bioinformatics.}

\p{Some of the \CRtwo{} code/data will be hosted on 
GitHub at \href{https://github.com/Mosaic-DigammaDB/CRCR}{Mosaic-DigammaDB/CRCR} 
(for data aggregation) and 
\href{https://github.com/Mosaic-DigammaDB/CRCR}{Mosaic-DigammaDB/LingTechSys} 
(for code previews).  
The latter repository includes preview code sampling 
database engine features, such as the logic for constructing a 
database in shared memory, encoding data types for persistence, 
and so forth.  
The data management tools developed 
for the \CRtwo{} repository have a broad range of use-cases, and 
can be customized for different projects.  
Companies or research groups interested in a more 
substantial code preview are invited to contact LTS
to discuss their projects and requirements in greater detail.}

\p{The main challenge when curating a 
data repository such as \CRtwo{} is reconciling 
heterogeneous data formats.  
In response to this challenge, LTS has 
focused on hypergraph-based data models 
which can unify many different information 
structures into one common structure.  
In particular, \CRtwo{} will introduce 
a special \q{Hypergraph Exchange Format} (\HGXF{}) 
which can take the place of disparate 
tabular or graph file formats 
(comma-separated values, numeric python, 
spreadsheets, graph-network serializations, 
etc.), so as to merge data sets into a common 
\textit{machine-readable} archive.  In addition, \CRtwo{} will 
introduce a new protocol for engineering 
hypergraph databases, called 
\q{Transparent Hypergraph Query Language}
(\THQL{}).  Databases conformant to the \THQL{} model will 
be able to export data in hypergraph-based 
formats, thereby generating data sets 
which can be used as published, citable 
Research Objects.  \lCRtwo{} will demonstrate 
\THQL{} via a new database engine called \q{DigammaDB} 
(or \DgDb{}) which 
serves as a \q{Reference Implementation} for 
\THQL{}.  In \CRtwo{}, \DgDb{} functions as 
a prototype and reference example for \THQL{}, used 
to curate data sets before their final 
form is exported into the main data repository.  
LTS can also customize commercial versions of a \THQL{} 
engine tailored to the requirements of 
individual projects.} 
    
\p{\lTHQL{} is designed with a priority on 
application development.  In particular, 
any instantiation of \THQL{} should 
provide data persistence capabilities 
through (as much as possible) self-contained 
code libraries that can be included in 
source-code form within an overall application.  
\lTHQL{} is designed to integrate seamlessly 
with native, desktop-style standalone 
applications.  In short, \THQL{} 
represents an unprecedented combination of 
native desktop-style software development 
and hypergraph database engineering.} 

\p{Complementing \THQL{}'s application-development 
focus, \CRtwo{} will also introduce \q{Dataset Creator,} 
(\dsC{}), a new tool for curating research data sets.  
The main feature of Dataset Creator is 
its use of native software components 
(called \q{Dataset Applications}), 
allowing researchers to view, manipulate, and 
reuse research data.  In sum, the typical 
Research Object built with \dsC{} 
will include self-contained source code 
implementing a customized desktop application 
providing access to the accompanying data set.  
In addition to \GUI{} code, each 
Dataset Application will supply \q{data-access} 
code libraries for 
parsing the raw data-set files, so as to obtain 
the information visualized within the 
\GUI{} classes of the Dataset Application.  
These data-access libraries offer machine-readable access 
to the raw data, permitting subsequent researchers to 
reuse the data-access software libraries so as to transform, 
filter, or analyze the published data in 
the context of replication studies and/or 
novel research projects.} 
  
\p{Development of \THQL{} and \dsC{} is 
concomitant with \CRtwo{}; the \CRtwo{} repository will 
provide a practical test-bed for validating 
this new technology.  Accordingly, the 
following sections will describe \CRtwo{} 
in greater detail, followed afterward by 
sections offering more information about 
\THQL{} and \dsC{}.}

\part{The Covid-19 Repository}
\section{The Cross-Disciplinary Repository for Covid-19 Research}
%\section{Background}
\p{The sudden emergence of Covid-19 as a global crisis has 
cast a spotlight on computational and technological challenges 
which, in the absence of a catastrophic pandemic, would 
rarely rise to public attention.  In particular, an effective 
response to the dangers of \makebox{SARS-CoV-2} requires coordinated 
policy making integrating diverse modes of scientific inquiry.  
Genomic, biomolecular, epidemiological, socio-demographic, clinical, 
and radiological information are all pertinent to Covid-19.  
In this environment, it is important that the 
empirical foundations for expert recommendations --- which 
in turn drive public policies of enormous social and 
economic consequence --- be transparently documented 
and critically examined.  The proper synergy between government 
and science depends on data centralization: given 
the gaps in our current Covid-19 knowledge, it is 
understandable that different jurisdictions will craft responses to 
the pandemic in different ways.  There is no central authority 
with sufficient epistemic force to legitimize homogeneous 
mandates across the entire country.  However, such 
policy differences should be a consequence of alternative interpretations of 
scientific knowledge or the diverse needs of local communities  
--- rather than being a haphazard consequence of governments 
working with divergent, competing, and poorly integrated data.}

\p{The current administration, along with numerous corporate and academic 
entities, has clearly recognized the need for a more 
centralized paradigm for sharing Covid-19 data.  For example, 
the White House spearheaded a scientific initiative to 
develop \Cnineteen{}, an open-access corpus of over 46,000 
peer-reviewed publications related to Covid-19, which 
were transformed into a common machine-readable representation 
so as to promote text and data mining.  Similarly, 
large institutions such as Google, Johns Hopkins, and 
Springer Nature have all implemented some form of coronavirus 
data-sharing platform targeted to both scientists and 
policy makers.  However, these two aspects of the 
corporate/academic contributions to Covid-19 data sharing 
(exemplified by the \Cnineteen{} White House initiative and by 
institution-generated portals, respectively) 
have been incomplete, for opposite but complementary 
reasons.  Specifically, \lCnineteen{} is highly structured and tightly 
integrated, but it focuses primarily on text mining and 
scientific documents, not \textit{research} data.  
While it is possible to find data 
sets about Covid-19 through \Cnineteen{}, the 
techniques to do so are both cumbersome and non-scalable.  
On the other hand, projects such as the Johns Hopkins coronavirus 
\q{dashboard} provide accessible data sets, yet these 
projects are isolated and do not offer the level of 
structure and integration evinced by \Cnineteen{}.  In 
short, an optimal Covid-19 research platform 
would merge the structural text-mining rigor of 
\Cnineteen{} with the data-centric focus of  
isolated projects that share Covid-19 data 
with the scientific community, policy makers, 
and the general public.}

\p{The benefit of \CRtwo{} 
is that it can accelerate Covid-19 research by 
(1) pooling a diverse collection of data sets into a 
single resource which scientists can utilize; 
(2) serving as the prototype for larger research 
portals that can aggregate new Covid-19 data 
that will emerge from hospitals, labs, and 
academic institutions in the future; (3) formalizing 
a framework for aggregating patient narratives 
to accurately capture first-hand subjective symptomatology of 
the patient suffering from Covid-19; and 
(4) accelerating the implementation of novel 
data-integration and software-development 
technologies which can contribute to scientific 
progress \visavis{} Covid-19 
in particular, and biomedical/scientific computing 
methodology in general.  These principles, in turn, 
will shape the design of \CRtwo{}.  An ideal data-sharing 
ecosystem should merge data from multiple sources, but should do so 
in a fashion which yields a machine-readable totality, 
analogous to \Cnineteen{}'s structuration with respect to 
text mining.  The merit of \CRtwo{} therefore lies not 
only in the data which it will encompass but also in 
novel technology that it will concretize for constructing 
data repositories adhering to these goals 
--- aggregating data, but also instantiating novel 
data-integration and database engineering strategies.}

\p{Given these varying goals, \CRtwo{} can provide value at different 
scales of realization.  Relatively small data 
sets serve several scientific and computational 
purposes: (1) they can provide researchers 
with a mental picture of how data in different 
disciplines, projects, and experiments is structured; 
(2) they can serve as a prototype and testing 
kernel for technologies implemented to manipulate 
data in relevant formats and encodings; and 
(3) they can lay the foundation for data-integration 
strategies.  For example, when designing a 
representation format and/or implementing code 
to merge different data formats into a single 
structure (or meta-structure), it is useful 
to work with small, representative examples 
of the data structures involved, so as not 
to complicate the integration logic with 
computational details solely oriented to 
scaling up the data-management logistics.  
As a result, \CRtwo{} can provide a 
testbed for implementing data-integration 
technologies which can scale up as needed.  
To fulfill this mission, \CRtwo{} can aggregate 
relatively small data sets which have 
previously been published on academic and research 
portals, such as Springer Nature, Dryad, and DataVerse.  
At the same time, a more substantial 
(and not necessarily fully open-access) Covid-19 
data-set collection would also be beneficial to the 
scientific and policy-making community.  Ideally, then, 
\CRtwo{} will be paired with a larger technology which shares 
a similar implementational strategy but with different 
accession paradigms, allowing for an open-ended 
collection of Covid-19 data which users may 
selectively access (instead of a single package 
that users may acquire as an integrated resource).  
The common denominator in both cases 
(whether the focus is on relatively smaller or 
larger data sets) is the 
importance of deploying novel and contemporary 
data-integration techniques to centralize 
Covid-19 research as much as possible.  
Accordingly, this summary will briefly explain 
how \CRtwo{} can accelerate Covid-19 data integration 
on both a practical and technological level.}
 
\section{Methodology for Covid-19 Data Integration}

\p{As indicated above, pertinent Covid-19 data is drawn 
from multiple scientific disciplines.  On a technological level, 
Covid-19 data is documented via a wide array of file 
types and data formats.  This diversity presents technological 
challenges: if a Covid-19 information space encompasses 
files representing 25 different incompatible 
formats, users would need 25 different technologies 
to fully benefit from this data.  In many 
cases, however, data incompatibilities are 
merely superficial --- an important subset of 
Covid-19 data, for example, has a common 
tabular meta-model, even if the data is 
realized in discordant technologies (spreadsheets, 
relational databases, comma-separated-value or 
Numeric Python files, and so forth).  Applying \CRtwo{}'s technology, 
one level of data integration can thus 
be achieved simply by encoding tabular 
structure into a common representation: any field in a table 
can be accessed via a record number and a column name and/or 
index.  In some cases, more rigorous integration is also 
possible --- for example, by identifying situations where 
columns in one table correspond semantically or conceptually 
to those in another table.  In either case, 
it is reasonable to assume that a single abstract data 
format lies behind surface data-expression in patterns 
such as spreadsheets and comma-separated values 
(\CSV{}), so that all files in an 
archive encoding spreadsheet-like data can be 
migrated to a common model.}

\p{Other forms of clinical and epidemiological inputs are often 
more amenable to graph-like representations.  For instance, 
trajectories of viral transmission through 
person-to-person contact is obviously an instance 
of social network analysis.  Similarly, models of 
clinical treatments and outcomes can take graph-like 
form insofar as there are causal or institutional 
relations between discrete medical events: 
a certain clinical observation \textit{causes} a 
care team to request a laboratory analysis, 
which \textit{yields} results that \textit{factor} 
into the team's decision to \textit{administer} some 
treatment (e.g., a drug \textit{from} a particular 
provider \textit{with} a specific chemical structure), which 
observationally \textit{results} in the patient improving 
and eventually \textit{being} discharged.  In short, 
patient-care information often takes the form 
--- at least conceptually --- of a network comprised 
of different \q{events,} each event involving some 
observation, action, intervention, or decision made 
by care providers, and where the important data 
lies in how the events are interconnected: both their 
logical relationships (e.g., cause/effect) and their 
temporal dynamics (how long before a drug leads to a 
patient's improvement; how much time elapses between admission to 
a hospital and discharge).  These graph-like representations 
are a natural formalization of \q{patient-centered} data 
models.}

\p{Using \CRtwo{}'s associated software 
(for example, importing Covid-19 data sets into a 
\THQL{} database), a higher level of 
data integration can then be 
achieved by merging tabular and graph-like models into a 
single \textit{hypergraph} format.  A 
significant subset of Covid-19 data (or, more generally, 
any clinical/biomedical information) 
conforms to either tabular or graph structures; 
thus it is feasible to unify all of this information 
into a common framework.  A graph-plus-table 
architecture is generally considered some form of 
Hypergraph model, and indeed \CRtwo{} adopts a hypergraph 
paradigm to merge many different sorts of information into a 
common structure.  In particular, \CRtwo{} introduces 
a new \q{Hypergraph Exchange Format} (\HGXF{}) which 
can provide a text encoding of many files that, 
when originally published, embodied a diverse 
array of file-types requiring a corresponding 
array of different technologies.  \lCRtwo{} 
will include specialized computer code that 
would enable machine-readability of the \HGXF{} files, 
and use them to create hypergraph-database instances.  
In short, \CRtwo{} will promote Covid-19 data integration 
by translating a wide range of files into a common 
\HGXF{} format.\footnote{\CRtwo{} data sets 
are not required to compile all files 
to a hypergraph format; in particular, 
sciences requiring substantial quantitative 
analysis --- e.g., biomechanics or genomics --- 
express data via encodings optimized for relevant 
mathematical operations, and have parser libraries 
optimized for these specific formats.  For 
these files \sCRtwo{} will generally provide an 
\sHGXF{} encoding supplying data 
\textit{about} the original file, with information 
concerning the file type, preferred software components 
for viewing/manipulating its data, etc.,  
so that the contents of non-\sHGXF{} files 
can be indirectly included into the \sCRtwo{} 
hypergraph-based ecosystem.}}

\section{Hypergraph Data Models and Multi-Application Networks}

\p{As outlined thus far, most Covid-19 data via the \CRtwo{} 
technology can be wholly or partially integrated 
into a single hypergraph framework, which accordingly simplifies 
the process of designing software applications and 
algorithms to analyze and manipulate this data.  
Specifically, software components can employ 
a single code library to obtain, read, consume, 
and store data, rather than needing to re-implement 
this logic for a large number of different file formats 
and/or database models.}

\p{Quality software (especially in the clinical and 
biomedical context) demands a balance between 
applications which are either too broad or too narrow 
in scope.  On the one hand, doctors often complain 
that homogeneous Electronic Health Record systems (where 
every digital record or observation is managed by a single 
all-encompassing application) are unwieldy and hard to work 
with.  This is understandable, because the clinical tasks 
of health care workers with different specializations can be very 
different.  On the other hand, doctors also complain about 
software and information systems which are so balkanized 
that they must repeatedly switch between different, non-interoperable 
applications.  In short, clinical, diagnostic, and research software 
should be neither too homogeneous nor too isolated; finding the 
proper balance between these extremes is, no doubt, 
a major challenge to the usability of electronic health 
systems going forward.}

\p{Against this background \CRtwo{} demonstrates novel 
solutions to this problem: it focuses on the dimensions 
of data acquisition and management that are specific 
to individual scientific or medical specializations, 
while also identifying requirements that are 
consistent across domains.  Scientific software 
generally needs to hone in 
on the data visualization and analytic 
requirements of particular disciplines; for example, 
biochemists use different programs than 
astrophysicists.  However, much 
of the code underlying scientific applications 
has nothing to do with these high-level 
models or theories, but is simply a 
fulfillment of basic data-management 
functionality --- data storage, accession, provenance, 
searching, user validation, and so forth.  
In effect, the computational requirements 
of scientific and biomedical software can 
be partitioned into two classes: (1) 
domain-specific logic which reflects the 
quantitative or theoretical models of 
narrow scientific fields; and (2) 
data-management logistics which can be 
realized within a central access hub, rather 
than being re-implemented by each application 
in isolation.}

\p{In short, \CRtwo{} architecture  
conceives of a central hub responsible for storing 
data and serving as a common access point 
--- providing the \q{gateway} where authorized 
users can gain access to heterogeneous information 
spaces utilized by an array of domain-specific 
software applications.  Since peer applications  
would not be directly responsible for data persistence 
or user identity management, they can focus 
on their specific data analysis and visualization 
capabilities.  The central hub, serving multiple 
peer applications, is then a heterogeneous data space 
managing information from multiple applications while 
also tracking information about the applications 
themselves: helping users to identify and launch the 
software which is most directly relevant to their 
clinical or research needs at the moment.  Meanwhile, 
because peer applications are jointly connected to a 
central hub, it is possible to implement scientific 
workflows where one application may send and receive 
data from its peers, allowing applications to complement 
each others' capabilities.}

\p{This multi-application networking architecture 
has precedents in some of the current database and 
engineering technologies.  For example, many hospitals and 
medical institutions employ some version of a 
\q{Data Lake,} pooling disparate data sources into 
a heterogeneous aggregate which is then accessed 
by multiple client applications.  Similarly, Machine Learning 
and Artificial Intelligence often adopts \q{software agents} 
or analytic modules in contexts such as Online 
Analytic Processing, which again represent semi-autonomous 
software components sharing an originary 
data hub.  Web applications, too, often act as domain-specific 
subsidiaries deferring operational requirements, such 
as user authentication or transaction processing, to a 
central web service.  The limitation of multi-application 
networks in these existing contexts are that the 
software agents involved are generally \q{lightweight,} 
with relatively primitive user-interface design.  
By contrast, the hypergraph technology introduced with \CRtwo{} 
will support multi-application networking in the context of more 
substantial desktop-style scientific applications.  In sum, 
the novel hypergraph technology developed by LTS offers a 
hybrid of the development methodologies 
employed for desktop scientific software and those 
applicable to multi-agent heterogeneous data stores, like 
a Semantic Data Lake.  To accomplish these goals, 
\CRtwo{} will utilize 
a new hypergraph database engine, coded in the \Cpp{} 
programming language, which has a unique focus on 
supporting native \GUI{} applications from the ground 
up, including persisting application state and 
storing application documentation within 
the database itself.}

\section{A New Paradigm for Data Sharing and Data Transparency}
\p{One exceptional feature of Covid-19 research is the 
extent of public attention focused on scientific discoveries 
about the disease.  Academic and commercial research teams 
find themselves in an unprecedented situation where 
there is unusual pressure to accelerate the 
Research and Development process, and a concomitant demand for 
a novel level of transparency and openness.  For example, 
vaccine development protocols are being fast-forwarded 
to take months instead of years, and information about 
the development process (such as trial results and 
scheduling) will likely be shared with the public much 
more than is standard practice.  This new reality, 
in turn, calls for a commensurate evolution in 
the technology for public data-sharing.}

\p{In conventional biomedical R\&D, much of the research 
data is proprietary, and revealed only in restricted 
contexts to select parties (such as the Food and Drug 
Administration).  Data which \textit{is} then publicly shared 
tends to be tied to published research papers 
in peer-reviewed literature, primarily read 
by a relatively small, specialist audience.  All of this is 
changing with SARS-CoV-2: companies pursuing 
Covid-19 R\&D (in the context of vaccine trials, for example) 
are facing pressure to publicly share their results 
as soon, and as transparently, as possible; and 
policy makers, scientists, and journalists are no 
less looking for quick access to research data directly, 
rather than circuitously through academic publications.}

\p{\lCRtwo{} will introduce the new Dataset Creator technology 
targeted toward this new environment of direct, transparent 
data-access (\dsC{} will be discussed in greater detail below).  
Data sets created 
via this technology therefore implement the \q{Research Object Protocol,} 
which mandates that research data be bundled with code allowing 
scientists to analyze and manipulate the information in the 
corresponding data set.  The Research Object framework was 
designed by a consortium of academic and governmental 
entities, such as the National Institutes of Health, to 
promote a paradigm for data publishing which prioritizes 
multi-faceted research tools over \q{raw} data that 
can be difficult to reuse in the absence of supporting code.  
In particular, Research Objects 
should be (as much as possible) \textit{self-contained,} 
which means that scientists do not need external 
software dependencies to access and study the data --- any special 
code which is a prerequisite to using this data should 
be included, alongside the raw data, as part of the 
Research Object itself.}

\p{Dataset Creator enables 
standalone, self-contained, and full-featured 
native/desktop applications to be uniquely implemented 
for each data set, distributed in source-code fashion 
along with raw research data (\dsC{} Dataset Applications 
use \Qt{} by default to provide native \GUI{} classes, 
tailored to the relevant Research Object).  Adopting such a 
data-curation method makes data sets easier to 
use across a wide range of scientific disciplines, 
because the data sets are freed from having to rely on domain-specific 
software (software which may be commonly used in one scientific 
field but is unfamiliar outside that field).  
In addition, Research Objects composed with \dsC{} can 
be integrated into Multi-Application Networks  
(which are described in the previous section) because 
the dataset applications are autonomous native \GUI{} applications that 
can easily interoperate via \Qt{} messaging protocols.}

\p{Of course, most of the \CRtwo{} data sets are 
previously-published work composed via older technology. 
Many of these resources, created with a wide range of software 
products, predate (or fail to apply) contemporary specifications 
such as the Research Object Protocol; not 
every \CRtwo{} data set will have the full set of 
features described in this section.  However, 
\CRtwo{} will try to maximize the value of each data set 
by translating them into a \Qt{}-based format --- in 
particular, \CRtwo{} will provide \Qt{} code for 
reading \HGXF{} files, as well as a \Qt{}-based hypergraph 
representation library.  Following the 
data integration methods outlined earlier, much of the 
\CRtwo{} data can be merged into a \Qt{}-based framework, 
which can facilitate the implementation of 
new, more sophisticated Dataset Applications 
as the information in \CRtwo{} gets reused 
for subsequent research.  \lCRtwo{} will also include 
\Qt{}-based software, such as a customized \PDF{} viewer, which 
will help researchers utilize the corpus in its 
entirety.  For example, \CRtwo{}'s \PDF{} viewer 
will include special code to connect \PDF{} files 
with data sets via \q{micro-citations,} as discussed in the 
next section.}

\section{Supporting Data Micro-Citations to Improve Machine Readability}

\p{The \CRtwo{} database engine supports annotating individual components 
of a database --- a technology sometimes referred to as 
\q{micro-citation.}  Data micro-citations are references to 
integral parts of a data set, such as an individual table, 
or a single row/record or column in a table.  Micro-citations allow 
these integral parts within the data set to be cited by and linked 
to publications, for purposes of machine readability and 
attribution.  As an example, preliminary vaccine trials often 
target a patient cohort selected for demographic or medical 
criteria matching the population who would most benefit from 
the vaccine.  These criteria for selecting the cohort for 
the vaccine study are usually described in the texts of the 
articles.  However, these criteria are also identified 
within the data set by socio-demographic 
data which is part of the information generated 
by the trial. \hspace{-6pt} By making these connections between criteria 
discussed in the article and those 
represented in the corresponding data set explicit, 
text and data mining can be \textit{merged} as analytic 
tools targeting a data repository, so that 
%\lCRtwo{} is therefore 
%designed to optimize the convenience and rigor of micro-citations 
%both within data sets and within research papers.  
machine reading is able to mine not just article text but the 
corresponding data.}  

%, accelerating the 
%extraction of current scientific information that is 
%of use to scientists and policy makers.}

%\vspace{1em}
\p{One reason why micro-citations are important is that they 
clarify the scientific meaning attributed to data set  
elements by connecting these elements to scientific concepts and 
\q{controlled vocabularies} (such as a list of drug names, 
diseases, proteins, etc.).  
For instance, micro-citations allow table columns to be 
mapped to statistical parameters, enabling their 
empirical properties (such as min/max values and distribution) 
to be queried by text and data mining software.  Likewise, 
\CRtwo{} enables dimensional 
and measurement annotations to describe the empirical and experimental 
significance of the measured or calculated quantities which 
are stored in a database.  Such quantity dimensions model the 
conceptual roles which particular parameters perform: 
e.g., the axiation \q{mJ/cm$^2$}(millijoule per 
square centimeter) indicates the intensity of 
ultraviolet light --- any table (or other 
data aggregate) having a column or field with this dimension 
is intrinsically associated with observations or experiments 
pertaining to \UV{} light.  Consequently, to 
locate data sets relevant for research about the clinical uses of 
antiviral \UV{} radiation, 
one method is to search for data fields dimensionalized 
in terms of joule or millijoule per square centimeter.  
As this example illustrates, data micro-citation 
--- via annotations on data fields, statistical parameters, 
and table columns --- is an important data-mining tool.  
In short, constructing micro-citations within a database 
serves two distinct benefits: (1) to aid data mining; and 
(2) to enable granular links (joining specific 
parts of articles to corresponding parts of the data set in 
the data set repository --- analogous to hyperlinks 
between web pages) to be established between 
publications and data sets, making it \textit{easier} for 
researchers to find the specific information most 
relevant to their own research.}

\section{Adding Patient Narratives to Covid-19 Data}

\p{In addition to aggregating published data sets, \CRtwo{} 
may be used as a repository for collecting new Covid-19 
information.  With that in mind, we are prioritizing the 
design of a standard for storing and accessing 
natural-language text representing patients' subjective 
symptom descriptions, which is quite useful for 
diagnostic/prognostic assessments of patients 
infected by Covid-19.}

\p{Just as \CRtwo{} envisions a curation of published 
data sets for data mining to improve machine-readability 
of Covid-19 research, LTS also sees the 
benefit of a repository of patient narratives prepared 
for text mining, to improve machine readability of the 
open-ended symptom descriptions offered by patients.  
While \CRtwo{} does not need to specify how these 
narratives should be collected, it will implement 
a common representational format 
so that patient narratives can be pooled, similar to  
to how \Cnineteen{} research texts are merged and 
encoded with a system that permits annotation.}

\p{In modeling patient narratives, this technology 
will be oriented toward the scientific-computing 
ecosystem outlined in the previous section.  
In particular, we assume that \GUI{}-based desktop 
applications will be the primary instruments for 
data collection and analysis; this means that 
the encoding of patient narratives may, at times, 
need to be paired with \GUI{} or multi-media content.  For example, 
the software for patients to submit 
medical history information could also allow 
them to pair (text-form) narratives with 
graphics indicating the location of their pain or 
discomfort.  Furthermore, the software could 
allow narratives to be accompanied 
by an audio file where patients could cough/speak into 
a microphone.  Given this 
range of possible inputs, patient-narrative 
encodings must be flexible enough to 
include diverse multi-media content.}

\p{As described earlier, an information space 
adapted for multiple peer applications should encompass 
capabilities for saving application state 
(the current visual appearance of the program), which 
includes features for modeling instances of 
\GUI{} classes.  This technology provides the 
necessary infrastructure for managing patient 
narratives.  For example, consider a multi-media 
intake form where patients may describe symptoms by 
placing icons (representing pain or discomfort) 
against anatomic silhouettes (head/body, back/front, 
extremities, and so forth).  
As patients use such a multi-media form, \GUI{} application 
state corresponds to the patient's subjective symptomology; 
in this way the graphics-based representation of symptoms 
could then be incorporated into the overall patient 
narrative.  This is an example of how 
application-persistence logic can be marshaled to 
the related project of curating patient narratives.}

%\pagestyle{empty}

%\p{Further documentation of text-encoding methodology applicable 
%to both patient narratives and publications associated 
%with \CRtwo{} research data is available on the \CRtwo{} 
%web site, such as \href{https://raw.githubusercontent.com/Mosaic-DigammaDB/CRCR/master/%cr2.pdf}{here} (this is a downloadable \PDF{} link; 
%visit the repository to see the larger archive structure).}

\part{The new Database Engineering and Data-Set 
Construction Technology}

\section{Native/Desktop Application Development with THQL}

\p{As described earlier, \THQL{} is a database 
engineering protocol which prioritizes 
data-persistence components that can be 
included in source code fashion within application-development 
projects.  \lTHQL{} is \q{transparent} in that 
all layers of data persistence and 
query processing logic are provided 
via self-contained source-code libraries.  
The complete database functionality can 
then be statically examined via the source-code 
files, and dynamically examined by running the 
client application through a debugger.  Moreover, 
because all \THQL{} source code is bundled with 
application code, \THQL{} can be configured to 
integrate seamlessly into its environing client-application  
logic.  For example, \THQL{} can be 
extended to natively recognize client-specific 
datatypes as data fields, or to execute 
client algorithms as query parameters.}

\p{As a query \textit{language,} \THQL{} can 
be instantiated either by special languages 
with their own syntax and semantics 
(analogous to \SQL{} or \SPARQL{}), or 
as an interface and pattern specified 
for a conventional language, such as 
\Cpp{}.  In the latter guise, 
\THQL{} provides a common protocol for 
essential database tasks, such as 
constructing, updating, querying, 
and backing up database instances.  
Each procedure comprising the 
\THQL{} protocol is assigned a 
specific role, so that the protocol 
can be abstractly modeled as a 
set of data-management roles 
mapped to corresponding procedure 
implementations.  On this basis, 
custom query languages can be 
constructed by exposing each role-procedure 
to a scripting interface.  For example, 
\THQL{}'s Reference Implementation 
(DigammaDB) exposes the protocol-specific 
procedures via a set of pointers to \Cpp{} functions.  
Consequently, parsing a query language is then rendered a 
straightforward process of mapping query 
expressions to the requisite procedures, 
whose corresponding handle can then be 
obtained via \Cpp{} interop.}

\p{As suggested by its name, \THQL{} is 
centered around the operations to 
define and store hypergraph-form data: 
information which has several levels 
or scales of structuration.  
This means that the \THQL{} protocol 
includes procedures for registering 
individual data fields 
(representing, in general, primitive 
types such as integers and decimals) in a 
database; aggregating fields into 
\q{hypernodes,} or groups of interrelated 
information; connecting pairs of 
hypernodes by identifying a specific 
connection which they have; 
adding contextual details or annotations 
(via so-called \textit{frames} and \textit{channels}) 
which refine assertions of hypernode connections; 
and constructing \q{proxies} to database elements 
(e.g. hypernodes, frames, and subgraphs) which 
can be referenced (via unique identifiers) 
as individual data fields.  Since proxies 
can then be aggregated into hypernodes in 
turn, \THQL{} graphs can have, if desired, 
arbitrarily deep nested structures.}

\p{\THQL{} also recognizes additional structures 
corresponding to conceptual details described 
earlier --- for example, fields within 
hypernodes can be linked with 
dimensional attributes (e.g. scales and 
units of measurement) and identified 
as micro-citation targets.  \lTHQL{} likewise 
supports a genre of controlled vocabularies 
applying to hypernode-types and/or connection labels 
(called \q{dominions,} for \q{Domain-Specific Mini-Ontologies}).  
Consequently, a graph can then be, if desired, 
configured to only accept hypernodes 
which conform to one of the dominion-defined 
types; and/or to only allow connections to be 
asserted between hypernodes when these connections 
can be labeled from a dominion-specific list 
of connectors.  Less restrictively, graphs 
can be defined in a more free-form style 
but use dominions to filter or 
query nodes and edges.}

\p{Another feature of \THQL{}, relevant 
to application integration, is the notion of 
configuring each database to support different 
\q{modes} of data persistence.  It is possible
to use \THQL{} for completely in-memory data management, 
with no direct data persistence at all.  This would be 
an appropriate solution when data can be read all at once 
from a static source, such as a data set.  In this guise, 
\THQL{} would be used to build a structural model 
of the data set, which can then be queried by application 
code.  Conversely, it is possible to employ \THQL{} as a 
continuously-updated data store, where changes to 
an underlying \THQL{} graph are persisted to disk 
as soon as they are registered.  Between these extremes, 
\THQL{} graphs can hold dynamically changing representations 
of a persistent database, which are only incorporated 
into the underlying database when instructed by 
the client application.  To support these different 
operational modes, \THQL{} engines need the capability 
to represent each data type in several different 
formats, tailored to different stages of processing 
through which values are routed before they can be 
stored persistently.}

\p{In DigammaDB (the \THQL{} Reference Implementation), 
persistent data storage is implemented via the \WhiteDB{} 
database engine.  \WhiteDB{} is a hybrid 
graph/record database which allocates a persistent 
data store in shared memory (allowing each database to be 
accessed from multiple applications).  \lDgDb{} encodes 
hypernodes in \WhiteDB{} records (although programmers 
can interface with the underlying \WhiteDB{} instances if 
desired).  \lDgDb{} can then use \WhiteDB{}'s index and 
query mechanism as the foundation for its own 
higher-level query system.  \lDgDb{} provides a 
convenient interface for binary-encoding user-defined \Cpp{} 
types, so that arbitrary application-level data can be 
stored via \THQL{} (we can supply more documentation 
describing hypergraph-encoding with WhiteDB if needed).  
In addition --- as an alternative 
to writing serializers for bonafide \Cpp{} types --- 
it is possible to construct \q{ghost} types, which are 
\Qt{} data structures built via the \q{QVariant} class, 
so that all (or most) hypernodes in the corresponding 
database have a single \Cpp{} type --- this technique 
is appropriate when, for example, the purpose of a \THQL{} instance 
is to read and then update \HGXF{} files with new information.   
To support different \THQL{} operational modes, \DgDb{} organizes 
a stage-structure based on encoding \WhiteDB{} values: 
at the ground level, values are simply pointers 
to in-memory \Cpp{} objects.  At an intermediate level, 
values are encoded (via the QDataStream class in \Qt{}) 
into structures which recognize the \WhiteDB{} encoding 
scheme but do not themselves interact with \WhiteDB{}. 
Finally, values may be recorded as \WhiteDB{} fields 
and records ready for persistent storage.}

\p{\WhiteDB{} also allows databases to be shared 
(including being sent over a network) by storing all 
database information in a special file format.  
\lDgDb{} instances can be shared via this 
same mechanism, although another option is to 
export the contents of a \DgDb{} database to 
\HGXF{} files, which in turn form the core 
of a research data set representing the 
database contents at a specific moment in time.  
In this guise \lDgDb{} works in conjunction 
with \dsC{}, serving as the engine to 
construct a data set through which research data 
curated via a DigammaDB database can be published.  Our 
\dsC{} technology will be described in the next section.}

\section{Dataset Creator}
\p{Dataset Creator (\dsC{}) is a framework for 
constructing data sets which include computer code based on the 
\Qt{} application-development platform.  
Dataset Creator takes advantage of the \Qt{} platform 
to construct Research Objects with exceptional 
\GUI{} and data-mining capabilities.  \lQt{}, the 
leading native cross-platform development toolkit, 
is a comprehensive framework encompassing 
a thorough inventory of programming features 
--- networking, \GUI{} implementation, file 
management, data visualization, \ThreeD{} graphics, 
and so forth.  Data sets based on \Qt{} require 
users to obtain a copy of the \Qt{} platform, but 
\Qt{} is free for non-commercial use and easy to 
install --- importantly, \Qt{} is wholly contained 
in its own folder and does not affect any other 
files on the user's computers (in this manner \Qt{} is 
different than most software packages, which usually 
demand a \q{system install}).}

\p{By leveraging the \Qt{} platform, \dsC{} enables 
standalone, self-contained, and full-featured 
native/ desktop applications to be uniquely implemented 
for each data set, distributed in source-code fashion 
along with raw research data.  Adopting such a 
data-curation method makes data sets easier to 
use across a wide range of scientific disciplines, 
because the data sets are freed from having to rely on domain-specific 
software (software which may be commonly used in one scientific 
field but is unfamiliar outside that field).  
In addition, Research Objects composed with \dsC{} can 
be integrated into Multi-Application Networks  
(which are described above) because 
the dataset applications are autonomous native \GUI{} applications that 
can easily interoperate via \Qt{} messaging protocols.}

\p{Because every data set is unique, each Dataset Application 
will necessarily include some code specific to that one 
Research Object.  However, \dsC{} will provide a core 
code base and file layout which is shared by all 
\dsC{} data sets by default.  This common core is 
structured in part by the goal of developing 
Dataset Applications in a \Qt{} context; for 
instance, \dsC{} projects are structured to use \Qt{}'s \q{qmake} 
build system as the primary tool for compiling 
data-set code.  The common \dsC{} code therefore 
includes qmake project files which support 
compiling application with several build configurations.  
In this framework, data-set users are classified into several 
different roles --- in addition to ordinary users 
(specifically, researchers who want to work with 
and draw information from 
data sets but have no development connection to 
these data sets themselves), \dsC{} recognizes 
roles for authors, editors, testers, and other 
users who are responsible for bringing data 
sets into publication-ready form to begin with.  
Depending on the administrative role, data set 
code can be compiled with additional 
features (e.g., unit testing features).}

\p{Another core component of \dsC{} is \LaTeX{} 
code that authors may use when preparing documents 
accompanying their data set.  These \LaTeX{} files 
encompass special functionality for defining code 
annotations and semantically significant points in 
article text, such as sentence and paragraph 
boundaries.  This \LaTeX{} code can be used 
in conjunction with a pre-processor that 
generates \LaTeX{} files from a special input 
language.  The goal of these text-processing 
technologies is to improve the interoperability 
between research papers and data sets.  In particular, 
the \LaTeX{} pre-processors (and subsequent 
\LaTeX{}-to-\PDF{} converters) generate \HGXF{} 
files which store information about textual and 
\PDF{} viewport coordinates specifying the location of semantically 
meaningful elements such as sentences and annotations.  
These files are then zipped and embedded in 
\PDF{} files.  With \dsC{}, the resulting \PDF{} files can 
then be loaded into a customized \PDF{} viewer capable of 
reading the embedded \HGXF{} data.  This allows the 
\PDF{} application to utilize the embedded information 
so as to provide a more interactive reading 
experience --- for instance, viewing annotations 
or copying sentences via context menus, where viewport 
data maps cursor position to textual elements visible 
on the \PDF{} page.  These features provide an 
application-level interface between the \PDF{} viewers 
(considered as \GUI{} components) and the corresponding 
\GUI{} components in Dataset Applications.}

\p{With proper customization, both the \PDF{} viewers 
and the \dsC{} Dataset Applications can interoperate, 
with \PDF{} context menus calling up windows in 
the Dataset Application's \GUI{} implementation, 
and vice-versa.  For example, researchers reading 
the \PDF{} version of a scientific article can 
launch the Dataset Application to explore some 
detail mentioned in the text.  This is an example of 
where micro-citations are practically useful: 
any microcitable element in a document (such as a table, 
column, row/record, or analytic procedure formalized as a 
procedural asset associated with a data set) 
can be linked to a corresponding \GUI{} element 
in the Dataset Application.  For example, a statistical 
parameter --- mentioned by name in the text, and perhaps 
represented in serialization within raw data --- can 
be mapped to a \GUI{} table column, and specifically 
the column header; this is then an annotation target, 
in the sense that for readers to gain more information it 
is proper to link mentions of the relevant scientific 
concept in article text to the column header as a 
graphical element that can be made visible.  The 
link is operationalized by implementing a procedure 
to show the \GUI{} window where the table is located, 
and ensure that the column header lies in the visible 
portion of the screen, as a response to readers on the 
\PDF{} side signaling a desire for information on 
the annotated text element.  In the opposite direction, 
database elements can be annotated with 
links that can used by the Dataset Application to launch 
a \PDF{} window opened to the page and location where 
the corresponding concept is discussed in the article.}

\p{In order to properly model this semantic, 
viewport, and data set data integration, 
\dsC{} uses a new document-representation format 
called \HTXN{} (Hypergraph Text Encoding Protocol).  
With \dsC{}, \HTXN{} files are not only associated 
with data set assets; they are also machine-readable 
document encodings that can be introduced into 
publication repositories and other corpora 
oriented toward machine-readability.  Authors can 
host \HTXN{} files within data sets and link to 
them via services such as CrossRef, thereby 
ensuring that a highly structured, machine-readable 
version of their papers is available for 
text and data mining.  The \HTXN{} protocol is 
also useful for encoding natural-language content 
which becomes part of a data set as data assets 
in themselves; for example, patient narratives.} 

\p{Further documentation of text-encoding methodology (applicable 
to both patient narratives and publications associated 
with \CRtwo{} research data) is available on the \CRtwo{} 
web site: such as \href{https://raw.githubusercontent.com/Mosaic-DigammaDB/CRCR/master/cr2.pdf}{here} (this is a downloadable \PDF{} link; 
visit the repository to see the larger archive structure).  
The document referenced in this hyperlink contains more information 
on \dsC{}, \HGXF{}, \HTXN{}, and the other technologies 
discussed here.}

%\p{ }

\end{document}


