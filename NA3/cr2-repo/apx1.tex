%\part{Appendices}
\section{Hypergraph Ontologies}
\p{The notion of \textit{hypergraph} ontologies extends 
the idea of \textit{ontology} as this appears in the context of 
the Resource Description Framework (\RDF{}).  In 
the Semantic Web, an \q{ontology} is essentially a 
graph schema, defining metadata for any information 
that can be serialized or represented in graph-like 
fashion, as well as criteria which graphs must satisfy 
when they are used to encode a specific sort of 
data.  To encode information about a \textit{clinical 
trial}, for instance, one may require that graphs include 
one node (representing the trial itself) which is linked 
to other nodes representing patients enrolled in the 
trial, as well as one or two nodes representing 
the trial's start and end dates.  These requirements 
constitute both a structural mandate on the graphs 
--- specifying how the nodes should be connected --- and a 
\textit{semantic} requirement on the nodes, each of which must 
be tagged with metadata clarifying that the corresponding 
node embodies a trial, person, or date.  An ontology 
then supplies a fixed vocabulary with which to rigorously 
declare this necessary metadata and the relevant 
graph-construction rules.  Insofar as graphs are employed 
to encode data, ontologies are analogous to 
Document Type Declarations (\DTD{}s) in the context 
of \XML{}.}

\p{In a conventional \RDF{} ontology, metadata is primarily 
associated with graph nodes and edges.  In particular, 
nodes are referenced to Uniform Reference Identifiers 
(\URI{}s), such as web addresses, and edges are labeled 
with concepts formally defined in one or more ontologies.  
Concepts which are used to annotate graph-edges, and which are 
given a fixed meaning in some controled vocabulary, are 
often called \q{properties.}  One special \q{is-a} property 
is often used to connect nodes with concepts that 
classify entities into one of many categories defined 
in an ontology, often called \q{classes.}  As such, most 
\RDF{} ontologies are primarily composed of 
\textit{classes} and \textit{properties}, each assigned 
a unique label.  The purpose of metadata for a given graph 
is then to link nodes to classes (for example, specifying 
that one node represents a clinical trial and the second 
represents a patient), and furthermore link edges to properties 
(for example, specifying that a patient-node is connected 
to a trial-node in that the patient is \textit{enrolled in} 
the trial).}

\p{Hypergraph ontologies are similar to conventional 
\RDF{} ontologies in that they likewise provide 
constraints and metadata for graphs.  However, 
hypergraph ontologies are more complex because hypergraphs 
are likewise more complex than ordinary graphs.  In particular, 
hypergraphs have different layers of structure:  
whereas \RDF{} nodes are intended to represent a 
single concept or value (such as a number, date, 
personal name, or \URL{}), a \textit{hypernode}, within a 
graph, typically encompasses multiple pieces 
of information inside it (often called \textit{hyponodes}, 
\textit{projections}, \textit{inner elements}, 
\textit{roles}, or just \textit{nodes}).\footnote{
The term \q{projections} is used by HyperGraphDB (see 
\bhref{http://www.hypergraphdb.org/?project=hypergraphdb&page=RefCustomTypes}); 
\q{roles} is used by Grakn.ai 
(see \bhref{https://dev.grakn.ai/docs/schema/concepts}); 
\q{inner entity} is used by the biointelligence project 
(see \bhref{https://www.ncbi.nlm.nih.gov/pmc/articles/PMC3555311/}), 
where the corresponding notion of \q{external entity} 
refers to what in other contexts might be called 
other hypernodes linked to a given hypernode via hyperedges.}  
In general, when analyzing hypergraphs it is necessary to 
distinguish at least two \q{tiers} of nodes, 
\textit{hypernodes} and \textit{hyponodes}, such that 
hyponodes are contained within hypernodes.  As a result, 
hypergraph ontologies need a corresponding distinction 
for node and edge annotations: insofar as nodes are 
categorized via classes, and edges via properties, 
it is neccesary to stipulate whether these classifications 
apply to hypernodes, hyponodes, or some combination of the two.}

\p{A further complication (contributing to the complexity of hypergraph 
ontologies compared to \RDF{} ontologies) arises because, 
even though hypergraphs represent nested or hierarchical structures,  
these hierarchies are often partial or overlapping. 
For example, a patient is \textit{part of} a 
clinical trial, but a patient is also included in 
other collections as well; for instance, a patient 
may be enrolled in a specific health plan 
(for insurance coverage).  One technique for modeling 
overlapping hierarchical data via hypergraphs is to 
employ \q{proxies,} which are digital identifiers encoding a 
multi-faceted concept into a single value that can be 
part of a hypernode (proxies are similar to 
\q{foreign keys} in \SQL{}).  Therefore, each patient, represented 
by its own hypernode, has an identifier which can be a 
proxy-value for the patient; for example, a value 
assigned to a hyponode becomes included 
in the hypernode encoding the list of patients enrolled 
in a clinical trial, or in the hypernode 
encoding the list of patients enrolled 
in a specific health plan.  Hypernodes can then be linked to 
other hypernodes by virtue of proxies (e.g., the 
trial-to-patient connection), and also by 
virtue of overlap (e.g., the set of all patients 
both enrolled in a given clinical trial \textit{and} 
enrolled in a given health plan).}

\p{In sum, compared to \RDF{} --- where there is one 
single sort of node-to-node relationship, based on 
whether or not an edge exists betweel nodes and 
how this edge is labeled --- hypergraphs are more 
flexible/expressive because they have 
multiple genres of node-to-node relationships: 
the relation between hypernodes and their inner 
hypnodes; between hypernodes and one another; 
between hyponodes in different hypernodes; 
and variations on each of these relation-types 
wherein relations are defined indirectly through 
proxies.  Moreover, in addition to hypernodes 
and hyponodes, hypergraphs afford additional 
levels of detail, such as \textit{frames}, 
\textit{channels}, and \textit{axiations}.
All of these details provide different 
\q{sites} where hypergraph annotations and 
metadata may be defined.\footnote{This means that 
formats for describing hypergraph ontologies 
have to be more expressive than \RDF{} ontologies, 
because \RDF{} ontologies need only to classify 
metadata as node-annotations or edge-annotations; 
by contrast, hypergraph ontologies need to distribute 
annotations among multiple sites of graph structure.}}

\p{An additional distinction within the Semantic Web 
is the contrast between \textit{reference ontologies} and 
\textit{application ontologies}.  In general, \textit{reference 
ontologies} are general-purpose schema intended to 
establish conventions shared by many different applications, 
to ensure that a large collection of data-producing software 
in a given domain is interoperable.  By contrast, 
\textit{application ontologies} are narrower in scope 
because they are more tightly integrated into applications that 
directly send and receive data.  Ontologies 
of either variety are used by software to interoperate 
with other software: so long as two applications are 
using the same ontologies, it is possible to 
ensure that one application can understand the 
data produced by a second, and vice-versa.  
However, such inter-operability is only a potential; 
it is the responsibility of programmers to 
actually implement code which produces and/or 
consumes data that conforms to the relevant 
ontology specifications.  In general, application 
ontologies are structured in such a way that 
these concrete implementations are more straightforward 
to produce, compared with reference ontologies.  
Reference ontologies offer little guidance to 
developers \visavis{} how to directly support the 
ontology within application code.  Conversely, 
application ontologies more rigorously describe the 
data structures which applications must implement 
in order to properly manipulate data that is structured 
according to the specifications of the ontology.}

\p{Although it is theoretically possible to encode 
data directly via \RDF{} graphs, it is far more 
common for applications to employ tabular and/or 
hierarchical formats such as spreadsheets, Protocol Buffers,
\XML{}, or \JSON{}.  As a result, the role of ontologies 
for constraining data structures (so that they adhere 
to common standards) is indirect.  It is useful to 
remember that ontologies are, at their most basic level, 
Controled Vocabularies; as such, ontology constraints often 
amount to stipulating a set of acceptable terms for a data 
value, column header, or annotation.  For a trivial example, 
our calendar recognizes 12 month names and 7 day names, 
which constrain the set of values permissible for \q{month} 
and \q{day} within a calendar date.  These terms are 
so commonplace that a \q{date ontology} is unnecessary, 
but in scientific or technical domains it becomes 
necessary to define vocabularies of allowable 
names or labels for specific data fields that 
representing some scientific value or measurement.  For instance, 
the Ontology of Vaccine Adverse Events (see 
\bhref{https://jbiomedsem.biomedcentral.com/articles/10.1186/2041-1480-4-40}) provides a nomenclature for use in Adverse Events Reporting, 
so that researchers or clinicians can describe symptoms via canonically 
recognized terms rather than through informal 
text descriptions.  In general, ontologies constrain 
data sets by stipulating that particular individual 
values within the overall data collection have names or descriptions 
whose associated set of possible values is prescribed \textit{a priori} 
by the applicable ontology.  However, the relationship between 
ontologies and concrete data sets must itself be 
documented, which is where application ontologies 
can become relevant --- application ontologies provide 
a bridge between reference ontologies and the 
applications which use them (along with the data 
generated and shared by those applications).}

\p{Most of the commonly used ontologies published 
by large-scale Semantic Web projects, such as the 
\OBO{} (Open Biomedical Ontology) Foundry, are 
\textit{reference} ontologies (see for example the 
\OBO{} tutorial at \bhref{https://github.com/jamesaoverton/obo-tutorial/blob/master/docs/obo.md}).  By contrast, hypergraph schema are 
closer in spirit to \textit{application} ontologies.\footnote{% 
although sometimes different terminology is used to 
express this idea (see the HyperGraphDB tutorial 
at \bhref{http://www.hypergraphdb.org/?project=hypergraphdb&page=IntroStoreData}).}
in particular, there is often a direct correlation between 
hypernode types and the data types understood by 
code libraries implemented in a given programming language, 
and these type-level correlations need to be modeled 
by hypergraph ontologies (the ontology makes 
explicit how types in the ontology and in the application 
are interconnected).  As a result, hypergraph 
ontologies require capabilities to model the type 
systems of programming languages.\footnote{By contrast, conventional 
\RDF{} ontologies do not intrinsically interact with 
type systems outside a limited group of 
\textbf{rdf:type}s}.  Moreover, because programming 
language types cover a range of use-cases --- including 
functional and \GUI{} types --- hypergraph 
ontologies need to directly model details 
about how applications interact with human users, 
including the visual objects (windows and their 
components) which users see on-screen.  They 
also model the \q{operational semantics,} (viz., patterns 
of human-software interaction) which govern 
how one uses the application to achieve their 
desired goals.  For example, if a software 
tool exists, in part, to help researchers define 
a patient cohort from an aggregate of clinical 
data, a relevant application ontology can include 
a description of the concrete steps used to achieve this end.}

\p{Because hypergraph ontologies are more directly 
structured around software operations and 
programming-language type systems, these ontologies 
function as a more rigorous guide to those implementing 
novel software solutions than reference ontologies 
intended primarily for Semantic Web data-sharing.  Hypergraph 
ontologies can, therefore, be especially important in 
the context of multi-application networks, where collections of 
distinct software applications share a common technological 
and information infrastructure.  In general, hypergraph 
ontologies are more appropriate for modeling data 
in contexts where custom software implementation is 
an important operational requirement of the overall 
data space --- that is, a particular data ecosystem 
is engineered with a priority for enabling developers 
to implement semi-autonomous software \q{agents} that 
provide specialized views or analyses of a data 
space shared by many such applications.}

%\p{}

