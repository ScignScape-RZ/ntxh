\documentclass[10pt,letterpaper]{article}


% pmml  arff  openannotation

\usepackage[T1]{fontenc}
\usepackage{tgtermes}

\usepackage[hang,flushmargin]{footmisc}

\usepackage{titlesec}

%\usepackage{sectsty}
%\sectionfont{\fontsize{13}{4}\selectfont}

\titleformat{\section}
  {\normalfont\fontsize{13}{15}\bfseries}{\thesection}{1em}{}

\titlespacing*{\section}
{0pt}{2ex plus 1ex minus .5ex}{-.4ex plus .2ex}

%\usepackage{mathptmx}

\usepackage{eso-pic}

%\setlength\parindent{0pt}

\AddToShipoutPictureBG{%

\ifnum\value{page}>1{
\AtTextUpperLeft{
\makebox[20.5cm][r]{
\raisebox{-1.95cm}{%
{\transparent{0.3}{\includegraphics[width=0.29\textwidth]{e-logo.png}}	}} } }
}\fi
}

\AddToShipoutPicture{%
{
 {\color{blGreen!70!red}\transparent{0.9}{\put(0,0){\rule{3pt}{\paperheight}}}}%
 {\color{darkRed!70!purple}\transparent{1}\put(3,0){{\rule{4pt}{\paperheight}}}}
% {\color{logoPeach!80!cyan}\transparent{0.5}{\put(0,700){\rule{1cm}{.6cm}}}}%
% {\color{darkRed!60!cyan}\transparent{0.7}\put(0,706){{\rule{1cm}{.6cm}}}}
% \put(18,726){\thepage}
% \transparent{0.8}
}
}

\AddToShipoutPicture{%
\ifnum\value{page}=1
\put(257.5,918){%
	\transparent{0.7}{
		\includegraphics[width=0.2\textwidth]{logo.png}}}
\put(59,948){\textbf{{\fontfamily{phv}\fontsize{14}{14}\selectfont{}EXECUTIVE SUMMARY}}}
\fi
}	



\AddToShipoutPicture{%
\ifnum\value{page}>1
{\color{blGreen!70!red}\transparent{0.9}{\put(300,8){\rule{0.5\paperwidth}{.3cm}}}}%
{\color{inOne}\transparent{0.8}{\put(300,10){\rule{0.5\paperwidth}{.3cm}}}}%
{\color{inTwo}\transparent{0.3}\put(300,13){{\rule{0.5\paperwidth}{.3cm}}}}

\put(301,16){%
\transparent{0.7}{
\includegraphics[width=0.2\textwidth]{logo.png}} }

{\color{blGreen!70!red}\transparent{0.9}{\put(5.6,5){\rule{0.5\paperwidth}{.4cm}}}}%
{\color{inOne}\transparent{1}{\put(5.6,10){\rule{0.5\paperwidth}{.4cm}}}}%
{\color{inTwo}\transparent{0.3}\put(5.6,15){{\rule{0.5\paperwidth}{.4cm}}}}

\fi
}

%\pagestyle{empty} % no page number
%\parskip 7.2pt    % space between paragraphs
%\parindent 12pt   % indent for new paragraph
%\textwidth 4.5in  % width of text
%\columnsep 0.8in  % separation between columns

%\setlength{\footskip}{7pt}

\usepackage[paperheight=14in,paperwidth=8.5in]{geometry}
\geometry{left=.5in,top=.6in,right=.45in,bottom=1.15in} %margins

\renewcommand{\thepage}{\raisebox{2pt}{\arabic{page}}}

\renewcommand{\footnoterule}{%
	\kern -3pt
	\hrule width .92\textwidth height .5pt
	\kern 10pt
}


\usepackage[hyphens]{url}
\newcommand{\biburl}[1]{ {\fontfamily{gar}\selectfont{\textcolor[rgb]{.2,.6,0}%
{\scriptsize {\url{#1}}}}}}

%\linespread{1.3}

\newcommand{\sectsp}{\vspace{12pt}}

\usepackage{graphicx}
\usepackage{color,framed}

\usepackage{textcomp}

\usepackage{float}

\usepackage{mdframed}


\usepackage{setspace}
\newcommand{\rpdfNotice}[1]{\begin{onehalfspacing}{

\Large #1

}\end{onehalfspacing}}

\usepackage{xcolor}

\usepackage[hyphenbreaks]{breakurl}
\usepackage[hyphens]{url}

\usepackage{hyperref}
\newcommand{\rpdfLink}[1]{\href{#1}{\small{#1}}}
\newcommand{\dblHref}[1]{\href{#1}{\small{\burl{#1}}}}
\newcommand{\browseHref}[2]{\href{#1}{\Large #2}}

\colorlet{blCyan}{cyan!50!blue}

\definecolor{darkRed}{rgb}{.2,.0,.1}


\definecolor{blGreen}{rgb}{.2,.7,.3}

\definecolor{darkBlGreen}{rgb}{.1,.3,.2}

\definecolor{oldBlColor}{rgb}{.2,.7,.3}

\definecolor{blColor}{rgb}{.1,.3,.2}

\definecolor{elColor}{rgb}{.2,.1,0}
\definecolor{flColor}{rgb}{0.7,0.3,0.3}

\definecolor{logoOrange}{RGB}{108, 18, 30}
\definecolor{logoGreen}{RGB}{85, 153, 89}
\definecolor{logoPurple}{RGB}{200, 208, 30}

\definecolor{logoBlue}{RGB}{4, 2, 25}
\definecolor{logoPeach}{RGB}{255, 159, 102}
\definecolor{logoCyan}{RGB}{66, 206, 244}
\definecolor{logoRed}{rgb}{.3,0,0}

\newcommand{\colorq}[1]{{\color{logoOrange!70!black}{\q{\small\textbf{#1}}}}}

\definecolor{inOne}{rgb}{0.122, 0.435, 0.698}% Rule colour
\definecolor{inTwo}{rgb}{0.122, 0.698, 0.435}% Rule colour

\definecolor{outOne}{rgb}{0.435, 0.698, 0.122}% Rule colour
\definecolor{outTwo}{rgb}{0.698, 0.435, 0.122}% Rule colour

\colorlet{linkcolor}{flColor!60!red}


\hypersetup{
	colorlinks=true,
	citecolor=blCyan!40!green,
	filecolor=magenta!30!logoBlue,
	urlcolor=blue,
    linkcolor=linkcolor!70!black,
%    allcolors=blCyan!40!green
}


\usepackage[many]{tcolorbox}% http://ctan.org/pkg/tcolorbox

\usepackage{transparent}

\newlength{\bsep}
\setlength{\bsep}{-1pt}
\let\xbibitem\bibitem
\renewcommand{\bibitem}[2]{\vspace{\bsep}\xbibitem{#1}{#2}}

\newenvironment{cframed}{\begin{mdframed}[linecolor=logoPeach,linewidth=0.4mm]}{\end{mdframed}}

\newenvironment{ccframed}{\begin{mdframed}[backgroundcolor=logoGreen!5,linecolor=logoCyan!50!black,linewidth=0.4mm]}{\end{mdframed}}

\usepackage{aurical}
\usepackage[T1]{fontenc}

\usepackage{relsize}

\newcommand{\bref}[1]{\hspace*{1pt}\textbf{\ref{#1}}}

\newcommand{\pseudoIndent}{

\vspace{10pt}\hspace*{12pt}}

\newcommand{\YPDFI}{{\fontfamily{fvs}\selectfont YPDF-Interactive}}

%
\newcommand{\deconum}[1]{{\protect\raisebox{-1pt}{{\LARGE #1}}}}

\newcommand{\visavis}{vis-\`a-vis}

\newcommand{\VersatileUX}{{\color{red!85!black}{\Fontauri Versatile}}%
{{\fontfamily{qhv}\selectfont\smaller UX}}}

\newcommand{\NDPCloud}{{\color{red!15!black}%
{\fontfamily{qhv}\selectfont {\smaller NDP C{\smaller LOUD}}}}}

\newcommand{\MThreeK}{{\color{blGreen!45!black}%
{\fontfamily{qhv}\fontsize{10}{8}\selectfont {M3K}}}}


\newcommand{\lfNDPCloud}{{\color{red!15!black}%
{\fontfamily{qhv}\selectfont N{\smaller DP C{\smaller LOUD}}}}}

\newcommand{\textds}[1]{{\fontfamily{lmdh}\selectfont{%
\raisebox{-1pt}{#1}}}}

\newcommand{\dsC}{{\textds{ds}{\fontfamily{qhv}\selectfont \raisebox{-1pt}
{\color{red!15!black}{C}}}}}

\definecolor{tcolor}{RGB}{24,52,61}

\newcommand{\CCpp}{\resizebox{!}{7pt}{\AcronymText{C}}/\Cpp{}}
\newcommand{\NoSQL}{\resizebox{!}{7pt}{\AcronymText{NoSQL}}}
\newcommand{\SQL}{\resizebox{!}{7pt}{\AcronymText{SQL}}}

\newcommand{\NCBI}{\resizebox{!}{7pt}{\AcronymText{NCBI}}}

\newcommand{\HTXN}{\resizebox{!}{7pt}{\AcronymText{HTXN}}}

\newcommand{\TDM}{\resizebox{!}{7pt}{\AcronymText{TDM}}}

\newcommand{\lHTXN}{\resizebox{!}{7.5pt}{\AcronymText{H}}%
\resizebox{!}{6.5pt}{\AcronymText{TXN}}}

\newcommand{\lsHTXN}{\resizebox{!}{9.5pt}{\AcronymText{\textcolor{tcolor}{HTXN}}}}

\newcommand{\LAF}{\resizebox{!}{7pt}{\AcronymText{LAF}}}

\newcommand{\UDpipe}{\resizebox{!}{7pt}{\AcronymText{UDpipe}}}

\newcommand{\C}{\resizebox{!}{7pt}{\AcronymText{C}}}


\usepackage{mdframed}

\newcommand{\cframedboxpanda}[1]{\begin{mdframed}[linecolor=yellow!70!blue,linewidth=0.4mm]#1\end{mdframed}}


\newcommand{\PVD}{\resizebox{!}{7pt}{\AcronymText{PVD}}}

\newcommand{\THQL}{\resizebox{!}{7pt}{\AcronymText{THQL}}}
\newcommand{\lTHQL}{\resizebox{!}{7.5pt}{\AcronymText{THQL}}}

\newcommand{\SDK}{\resizebox{!}{7pt}{\AcronymText{SDK}}}
\newcommand{\NLP}{\resizebox{!}{7pt}{\AcronymText{NLP}}}

\newcommand{\AXF}{\resizebox{!}{7pt}{\AcronymText{AXF}}}

\newcommand{\lAXF}{\resizebox{!}{7.5pt}{\AcronymText{A}}%
\resizebox{!}{6.5pt}{\AcronymText{XF}}}


\newcommand{\lsAXF}{\resizebox{!}{8.5pt}{\AcronymText{AXF}}}

\newcommand{\AXFD}{\resizebox{!}{7pt}{\AcronymText{AXFD}}}

\newcommand{\lAXFD}{\resizebox{!}{7.5pt}{\AcronymText{A}}%
\resizebox{!}{6.5pt}{\AcronymText{XFD}}}


\newcommand{\IJST}{\resizebox{!}{7pt}{\AcronymText{IJST}}}

\newcommand{\BioC}{\resizebox{!}{7pt}{\AcronymText{BioC}}}

\newcommand{\CoNLL}{\resizebox{!}{7pt}{\AcronymText{CoNLL}}}
\newcommand{\CoNLLU}{\resizebox{!}{7pt}{\AcronymText{CoNLL-U}}}

\newcommand{\sapp}{\resizebox{!}{7pt}{\AcronymText{Sapien+}}}
\newcommand{\lsapp}{\resizebox{!}{8.5pt}{\AcronymText{Sapien+}}}
\newcommand{\lssapp}{\resizebox{!}{9.5pt}{\AcronymText{Sapien+}}}

\newcommand{\ePub}{\resizebox{!}{7pt}{\AcronymText{ePub}}}

%\lsLPF


\newcommand{\GIT}{\resizebox{!}{7pt}{\AcronymText{GIT}}}

%\definecolor{atColor}{RGB}{11, 71, 17}
\definecolor{atColor}{RGB}{50, 22, 40}
\newcommand{\ATextClr}[1]{\textcolor{atColor}{\textbf{#1}}}

\newcommand{\DgDb}{\makebox{\raisebox{-3pt}{\resizebox{!}{11pt}{\ATextClr{%
\rotatebox{17}{$\varsigma$}}}}\hspace{-4pt}%
\resizebox{!}{6.5pt}{\ATextClr{D\hspace{-2pt}B}}}}

\newcommand{\lDgDb}{\makebox{\raisebox{-3pt}{%
\resizebox{!}{12pt}{\ATextClr{%
\rotatebox{17}{$\varsigma$}}}}\hspace{-4pt}%
\resizebox{!}{6.5pt}{\ATextClr{D\hspace{-2pt}B}}}}

\newcommand{\URL}{\resizebox{!}{7pt}{\AcronymText{URL}}}
\newcommand{\CSML}{\resizebox{!}{7pt}{\AcronymText{CSML}}}
\newcommand{\LPF}{\resizebox{!}{7pt}{\AcronymText{LPF}}}
\newcommand{\lLPF}{\resizebox{!}{8.5pt}{\AcronymText{LPF}}}
\newcommand{\lsLPF}{\resizebox{!}{9.5pt}{\AcronymText{LPF}}}

\makeatletter

\newcommand*\getX[1]{\expandafter\getX@i#1\@nil}

\newcommand*\getY[1]{\expandafter\getY@i#1\@nil}
\def\getX@i#1,#2\@nil{#1}
\def\getY@i#1,#2\@nil{#2}
\makeatother
	
\newcommand{\rectann}[9]{%
\path [draw=#1,draw opacity=#2,line width=#3, fill=#4, fill opacity = #5, even odd rule] %
(#6) rectangle(\getX{#6}+#7,\getY{#6}+#8)
({\getX{#6}+((#7-(#7*#9))/2)},{\getY{#6}+((#8-(#8*#9))/2)}) rectangle %
({\getX{#6}+((#7-(#7*#9))/2)+#7*#9},{\getY{#6}+((#8-(#8*#9))/2)+#8*#9});}


\definecolor{pfcolor}{RGB}{94, 54, 73}

\newcommand{\EPF}{\resizebox{!}{7pt}{\AcronymText{ETS{\color{pfcolor}pf}}}}
\newcommand{\lEPF}{\resizebox{!}{8.5pt}{\AcronymText{ETS{\color{pfcolor}pf}}}}
\newcommand{\lsEPF}{\resizebox{!}{9.5pt}{\AcronymText{ETS{\color{pfcolor}pf}}}}


\newcommand{\XPDF}{\resizebox{!}{7pt}{\AcronymText{XPDF}}}

\newcommand{\GRE}{\resizebox{!}{7pt}{\AcronymText{GRE}}}
\newcommand{\CAS}{\resizebox{!}{7pt}{\AcronymText{CAS}}}

\newcommand{\lMOSAIC}{%
\resizebox{!}{8pt}{\AcronymText{M}}%
\resizebox{!}{6pt}{\AcronymText{OSAIC}}}

\newcommand{\XML}{\resizebox{!}{7pt}{\AcronymText{XML}}}
\newcommand{\RDF}{\resizebox{!}{7pt}{\AcronymText{RDF}}}
\newcommand{\DOM}{\resizebox{!}{7pt}{\AcronymText{DOM}}}

\newcommand{\Covid}{\resizebox{!}{7pt}{\AcronymText{Covid-19}}}

\newcommand{\CLang}{\resizebox{!}{7pt}{\AcronymText{C}}}

\newcommand{\HNaN}{\resizebox{!}{7pt}{\AcronymText{HN%
\textsc{a}N}}}

\newcommand{\JSON}{\resizebox{!}{7pt}{\AcronymText{JSON}}}

\newcommand{\MeshLab}{\resizebox{!}{7pt}{\AcronymText{MeshLab}}}
\newcommand{\IQmol}{\resizebox{!}{7pt}{\AcronymText{IQmol}}}

\newcommand{\SGML}{\resizebox{!}{7pt}{\AcronymText{SGML}}}

\newcommand{\ASCII}{\resizebox{!}{7pt}{\AcronymText{ASCII}}}

\newcommand{\GUI}{\resizebox{!}{7pt}{\AcronymText{GUI}}}

\newcommand{\API}{\resizebox{!}{7pt}{\AcronymText{API}}}

\newcommand{\JATS}{\resizebox{!}{7pt}{\AcronymText{JATS}}}


\newcommand{\SDI}{\resizebox{!}{7pt}{\AcronymText{SDI}}}
\newcommand{\SDIV}{\resizebox{!}{7pt}{\AcronymText{SDIV}}}



\newcommand{\IDE}{\resizebox{!}{7pt}{\AcronymText{IDE}}}

\newcommand{\ThreeD}{\resizebox{!}{7pt}{\AcronymText{3D}}}

\newcommand{\FAIR}{\resizebox{!}{7pt}{\AcronymText{FAIR}}}

\newcommand{\QNetworkManager}{\resizebox{!}{7pt}{\AcronymText{QNetworkManager}}}
\newcommand{\QTextDocument}{\resizebox{!}{7pt}{\AcronymText{QTextDocument}}}
\newcommand{\QWebEngineView}{\resizebox{!}{7pt}{\AcronymText{QWebEngineView}}}
\newcommand{\HTTP}{\resizebox{!}{7pt}{\AcronymText{HTTP}}}


\newcommand{\lAcronymTextNC}[2]{{\fontfamily{fvs}\selectfont {\Large{#1}}{\large{#2}}}}

\newcommand{\AcronymTextNC}[1]{{\fontfamily{fvs}\selectfont {\large #1}}}


\colorlet{orr}{orange!60!red}

\newcommand{\textscc}[1]{{\color{orr!35!black}{{%
						\fontfamily{Cabin-TLF}\fontseries{b}\selectfont{\textsc{\scriptsize{#1}}}}}}}


\newcommand{\textsccserif}[1]{{\color{orr!35!black}{{%
				\scriptsize{\textbf{#1}}}}}}


\newcommand{\iXPDF}{\resizebox{!}{7pt}{\textsccserif{%
\textit{XPDF}}}}

\newcommand{\iEPF}{\resizebox{!}{7pt}{\textsccserif{%
\textit{ETSpf}}}}

\newcommand{\iSDI}{\resizebox{!}{7pt}{\textsccserif{%
\textit{SDI}}}}

\newcommand{\iHTXN}{\resizebox{!}{7pt}{\textsccserif{%
\textit{HTXN}}}}


\newcommand{\AcronymText}[1]{{\textscc{#1}}}

\newcommand{\AcronymTextser}[1]{{\textsccserif{#1}}}


\newcommand{\mAcronymText}[1]{{\textscc{\normalsize{#1}}}}

\newcommand{\FASTA}{{\resizebox{!}{7pt}{\AcronymText{FASTA}}}}
\newcommand{\SRA}{{\resizebox{!}{7pt}{\AcronymText{SRA}}}}
\newcommand{\DNA}{{\resizebox{!}{7pt}{\AcronymText{DNA}}}}
\newcommand{\MAP}{{\resizebox{!}{7pt}{\AcronymText{MAP}}}}
\newcommand{\EPS}{{\resizebox{!}{7pt}{\AcronymText{EPS}}}}
\newcommand{\CSV}{{\resizebox{!}{7pt}{\AcronymText{CSV}}}}
\newcommand{\PDB}{{\resizebox{!}{7pt}{\AcronymText{PDB}}}}

\newcommand{\XOCS}{{\resizebox{!}{7pt}{\AcronymText{XOCS}}}}

\newcommand{\HGXF}{{\resizebox{!}{7pt}{\AcronymText{HGXF}}}}
\newcommand{\lHGXF}{{\resizebox{!}{7.5pt}{\AcronymText{HGXF}}}}
\newcommand{\sHGXF}{{\resizebox{!}{5.5pt}{\AcronymText{HGXF}}}}

\newcommand{\CRtwo}{{\resizebox{!}{7pt}{\AcronymText{CR2}}}}
\newcommand{\lCRtwo}{{\resizebox{!}{7.5pt}{\AcronymText{CR2}}}}
\newcommand{\sCRtwo}{{\resizebox{!}{5.5pt}{\AcronymText{CR2}}}}

\newcommand{\ChemXML}{{\resizebox{!}{7pt}{\AcronymText{ChemXML}}}}

\newcommand{\TeXMECS}{\resizebox{!}{7pt}{\AcronymText{TeXMECS}}}

% pmml  arff  openannotation

\newcommand{\PMML}{\resizebox{!}{7pt}{\AcronymText{PMML}}}
\newcommand{\ARFF}{\resizebox{!}{7pt}{\AcronymText{ARFF}}}
\newcommand{\IeXML}{\resizebox{!}{7pt}{\AcronymText{IeXML}}}


\newcommand{\NGML}{\resizebox{!}{7pt}{\AcronymText{NGML}}}

\newcommand{\Cpp}{\resizebox{!}{7pt}{\AcronymText{C++}}}

\newcommand{\WhiteDB}{\resizebox{!}{7pt}{\AcronymText{WhiteDB}}}

\colorlet{drp}{darkRed!70!purple}

%\newcommand{\MOSAIC}{{\color{drp}{\AcronymTextNC{\scriptsize{MOSAIC}}}}}

\newcommand{\MOSAIC}{\resizebox{!}{7pt}{\AcronymText{MOSAIC}}}


\newcommand{\mMOSAIC}{{\color{drp}{\AcronymTextNC{\normalsize{MOSAIC}}}}}

\newcommand{\MOSAICVM}{\mMOSAIC-\mAcronymText{VM}}

\newcommand{\sMOSAICVM}{\resizebox{!}{7pt}{\MOSAICVM}}
\newcommand{\sMOSAIC}{\resizebox{!}{7pt}{\MOSAIC}}

\newcommand{\LDOM}{\resizebox{!}{7pt}{\AcronymText{LDOM}}}
\newcommand{\Cnineteen}{\resizebox{!}{7pt}{\AcronymText{CORD-19}}}

\newcommand{\lCnineteen}{\resizebox{!}{7.5pt}{\AcronymText{CORD-19}}}


\newcommand{\MOL}{\resizebox{!}{7pt}{\AcronymText{MOL}}}

\newcommand{\ACL}{\resizebox{!}{7pt}{\AcronymText{ACL}}}

\newcommand{\LXCR}{\resizebox{!}{7pt}{\AcronymText{LXCR}}}
\newcommand{\lLXCR}{\resizebox{!}{8.5pt}{\AcronymText{LXCR}}}
\newcommand{\lsLXCR}{\resizebox{!}{9.5pt}{\AcronymText{LXCR}}}

%\newcommand{\lMOSAIC}{{\color{drp}{\lAcronymTextNC{M}{OSAIC}}}}
\newcommand{\lfMOSAIC}{\resizebox{!}{9pt}{{\color{drp}{\lAcronymTextNC{M}{OSAIC}}}}}

\newcommand{\Mosaic}{\resizebox{!}{7pt}{\MOSAIC}}
\newcommand{\MosaicPortal}{{\color{drp}{\AcronymTextNC{MOSAIC Portal}}}}

\newcommand{\RnD}{\resizebox{!}{7pt}{\AcronymText{R\&D}}}

\newcommand{\lQt}{\resizebox{!}{8.5pt}{Qt}}
\newcommand{\QtCpp}{\resizebox{!}{8.5pt}{\AcronymText{Qt/C++}}}
\newcommand{\Qt}{\resizebox{!}{7pt}{\AcronymText{Qt}}}

\newcommand{\QtSQL}{\resizebox{!}{7pt}{\AcronymText{QtSQL}}}

\newcommand{\HTML}{\resizebox{!}{7pt}{\AcronymText{HTML}}}
\newcommand{\PDF}{\resizebox{!}{7pt}{\AcronymText{PDF}}}

\newcommand{\R}{\resizebox{!}{7pt}{\AcronymText{R}}}
\newcommand{\SciXML}{\resizebox{!}{7pt}{\AcronymText{SciXML}}}



\newcommand{\lGRE}{\resizebox{!}{7.5pt}{\AcronymText{GRE}}}

\newcommand{\p}[1]{

\vspace{.7em}#1}

\newcommand{\q}[1]{{\fontfamily{qcr}\selectfont ``}#1{\fontfamily{qcr}\selectfont ''}} 

%\newcommand{\deconum}[1]{{\textcircled{#1}}}

\renewcommand{\thesection}{\protect\hspace{-1.5em}}
%\renewcommand{\thesection}{\protect\mbox{\deconum{\Roman{section}}}}
\renewcommand{\thesubsection}{\arabic{section}.\arabic{subsection}}

\newcommand{\llMOSAIC}{\mbox{{\LARGE MOSAIC}}}
%\newcommand{\lfMOSAIC}{\mbox{M\small{OSAIC}}}

\newcommand{\llMosaic}{\llMOSAIC}
\newcommand{\lMosaic}{\lMOSAIC}
\newcommand{\lfMosaic}{\lfMOSAIC}


\newcommand{\llWC}{\mbox{{\LARGE WhiteCharmDB}}}

\newcommand{\llwh}{\mbox{{\LARGE White}}}
\newcommand{\llch}{\mbox{{\LARGE CharmDB}}}

\usepackage{enumitem}
%\usepackage{listings}

\colorlet{dsl}{purple!20!brown}
\colorlet{dslr}{dsl!50!blue}

\setlist[description]{%
  topsep=10pt,
  labelsep=22pt, leftmargin=5pt,
  itemsep=5pt,               % space between items
  %font={\bfseries\sffamily}, % set the label font
  font=\normalfont\bfseries\color{dslr!50!black}, % if colour is needed
}

\setlist[enumerate]{%
  topsep=3pt,               % space before start / after end of list
  itemsep=-2pt,               % space between items
  font={\bfseries\sffamily}, % set the label font
%  font={\bfseries\sffamily\color{red}}, % if colour is needed
}

%\usepackage{tcolorbox}

\newcommand{\slead}[1]{%
\noindent{\raisebox{2pt}{\relscale{1.15}{{{%
\fcolorbox{logoCyan!50!black}{logoGreen!5}{#1}
}}}}}\hspace{.5em}}


\let\OldLaTeX\LaTeX

\renewcommand{\LaTeX}{\resizebox{!}{7pt}{\color{orr!35!black}{\OldLaTeX}}}

\let\OldTeX\TeX

\renewcommand{\TeX}{\resizebox{!}{7pt}{\color{orr!35!black}{\OldTeX}}}


\newcommand{\LargeLaTeX}{\resizebox{!}{8.5pt}{\color{orr!35!black}{\OldLaTeX}}}

\setlength\parindent{0pt}
%\setlength\parindent{24pt}
%%\usepakage{newfile}

\usepackage{hyperref}

\usepackage{etoolbox}

\usepackage{zref-user}

\newwrite\sdiFile
\immediate\openout\sdiFile=\jobname.sdi.txt

\newcommand{\p}[1]{

\vspace{10pt}#1}

\newif\iftabng
\tabngfalse


\usepackage{letltxmacro}
\LetLtxMacro{\oldmmsemi}{\;}
\LetLtxMacro{\oldtbplus}{\+}
\LetLtxMacro{\oldtbgt}{\>}
\LetLtxMacro{\oldmmgt}{\+}

\newcommand{\+}{\iftabng\oldtbplus\else\sss\fi}

\renewcommand{\>}{\iftabng\oldtbplus\else
\ifmmode\oldmmgt\else\sse\sss\fi\fi}

%\renewcommand{\>}{\sse\sss}

\renewcommand{\;}{\relax\ifmmode\oldmmsemi\else\sse\fi}

\newcommand{\writeSDI}[1]{\immediate\write\sdiFile#1}

\newcommand{\emblink}[2]{\href{\#sdi:#1--#2}{\#sdi:#1--#2}}

%\newcount\sdiCounter
%\def\advsdiCounter{\global\advance\sdiCounter by1}

%\newcount\sdiCounterP
%\def\advsdiCounterP{\global\advance\sdiCounterP by1}

%\newcounter{sdiCounter}
\newcounter{sdiCounterP}[page]
\newcounter{sdiCounter}

\def\topt#1{\expandafter\the\dimexpr\dimexpr#1sp\relax\relax}

\makeatletter
%\catcode`\*=10
\newcommand{\sss}{%
\stepcounter{sdiCounterP}
\stepcounter{sdiCounter}
\pdfsavepos\write\sdiFile{!/ SDI_Sentence_Start} 
\write\sdiFile\expandafter{\expandafter$%
\expandafter i\expandafter:%
\expandafter\space\the\c@sdiCounter}
\write\sdiFile\expandafter{\expandafter$%
\expandafter o\expandafter:%
\expandafter\space\the\c@sdiCounterP}
\write\sdiFile\expandafter{\expandafter$%
\expandafter p\expandafter:%
\expandafter\space\thepage^^J%
$x: \topt\pdflastxpos^^J%
$y: \topt\pdflastypos^^J%
/!^^J%
<<>^^J%
}}
%\catcode`\%=14
\makeatother

\makeatletter
\newcommand{\sse}{%
\pdfsavepos\write\sdiFile{!/ SDI_Sentence_End} 
\write\sdiFile\expandafter{\expandafter$%
\expandafter i\expandafter:%
\expandafter\space\the\c@sdiCounter}
\write\sdiFile\expandafter{\expandafter$%
\expandafter o\expandafter:%
\expandafter\space\the\c@sdiCounterP}
\write\sdiFile\expandafter{\expandafter$%
\expandafter p\expandafter:%
\expandafter\space\thepage^^J%
$x: \topt\pdflastxpos^^J%
$y: \topt\pdflastypos^^J%
/!^^J%
<<>^^J%
}}
\makeatother




\newcommand{\lun}[1]{\raisebox{-4pt}{\fontfamily{qcr}\selectfont{%
\LARGE{\textbf{\textcolor{tcolor}{#1}}}}}\vspace{-2pt}}

\newcommand{\inditem}{\itemindent10pt\item}

\usepackage{soul}

\definecolor{hlcolor}{RGB}{114, 54, 203}
\colorlet{hlcol}{hlcolor!35}
\sethlcolor{hlcol}

\makeatletter
\def\SOUL@hlpreamble{%
	\setul{}{3ex}%         !!!change this value!!! default is 2.5ex
	\let\SOUL@stcolor\SOUL@hlcolor
	\SOUL@stpreamble
}
\makeatother

\usepackage{scrextend}
%\vspace*{3em}
\newenvironment{mldescription}{\vspace{1em}%
  \begin{addmargin}[4pt]{1em}
    \setlength{\parindent}{-1em}%
    \newcommand*{\mlitem}[1][]{\vspace{5pt}\par\medskip%
%\colorbox{hlcolor}{\textbf{##1}}\quad}\indent
\hl{ \textbf{##1} }\quad}\indent
}{%
  \end{addmargin}
  \medskip
}

\usepackage{marginnote}

\newcommand{\mnote}[1]{%
\vspace*{-2em}
\reversemarginpar
\raisebox{1em}{\marginnote{\parbox{4em}{%
\begin{mdframed}[innerleftmargin=4pt,
	innerrightmargin=1pt,innertopmargin=1pt,
	linecolor=red!20!cyan,userdefinedwidth=4em,
	topline=false,
	rightline=false]
{{\fontfamily{ppl}\fontsize{12}{0}\selectfont
		\textit{#1}}}
\end{mdframed}}
	}[3em]}}

\newcommand{\mnotel}[1]{%
\vspace*{-2em}
\reversemarginpar
\raisebox{-4em}{\marginnote{\parbox{4em}{%
\begin{mdframed}[innerleftmargin=4pt,
	innerrightmargin=1pt,innertopmargin=1pt,
	linecolor=red!20!cyan,userdefinedwidth=4em,
	topline=false,
	rightline=false]
{{\fontfamily{ppl}\fontsize{12}{0}\selectfont
		\textit{#1}}}
\end{mdframed}}
	}[3em]}}

\newcommand{\mnoteh}[3]{%
	\vspace*{#1}
	\reversemarginpar
	\raisebox{#2}{\marginnote{\parbox{4em}{%
				\begin{mdframed}[innerleftmargin=4pt,
					innerrightmargin=1pt,innertopmargin=1pt,
					linecolor=red!20!cyan,userdefinedwidth=4em,
					topline=false,
					rightline=false]
					{{\fontfamily{ppl}\fontsize{12}{0}\selectfont
							\textit{#3}}}
				\end{mdframed}}
			}[3em]}}


\newcommand{\mnoteb}[1]{%
	\vspace*{1em}
	\reversemarginpar
	\raisebox{1em}{\marginnote{\parbox{4em}{%
				\begin{mdframed}[innerleftmargin=4pt,
					innerrightmargin=1pt,innertopmargin=1pt,
					linecolor=red!20!cyan,userdefinedwidth=4em,
					topline=false,
					rightline=false]
					{{\fontfamily{ppl}\fontsize{12}{0}\selectfont
							\textit{#1}}}
				\end{mdframed}}
			}[3em]}}
	
\usepackage{wrapfig}

\usetikzlibrary{arrows, decorations.markings}
\usetikzlibrary{shapes.arrows}

\newcommand{\curicon}[2]{%
	\node at (#1,#2) [
	draw=black,
	%minimum width=2ex,
	inner sep=.7pt,
	fill=white,
	single arrow,
	single arrow head extend=3pt,
	single arrow head indent=1.5pt,
	single arrow tip angle=45,
	line join=bevel,
	minimum height=4.6mm,
	rotate=115
	] {};
}

\makeatletter
\def\@cite#1#2{[\textbf{#1\if@tempswa , #2\fi}]}
\def\@biblabel#1{[\textbf{#1}]}
\makeatother


%\let\origref\ref
%\renewcommand{\ref}[1]{{\LARGE #1}}

%\def\ref#1{\textbf{\origref{{\LARGE #1}}}}

\setlength{\footnotesep}{0pt}

\renewcommand{\thefootnote}{\textcolor{logoGreen!80!logoBlue}{{\fontfamily{qcr}\fontseries{b}\fontsize{10}{4}\selectfont\arabic{footnote}}}}


\newcommand{\LVee}{{\colorbox{cyan!40!yellow}{\textcolor{red!70!navy}{\textbf{\LARGE$\vee$}}}}}
\newcommand{\LWedge}{{\colorbox{cyan!40!yellow}{\textcolor{red!70!navy}{\textbf{\LARGE$\wedge$}}}}}

\renewcommand{\LVee}{}
\renewcommand{\LWedge}{}


\urlstyle{same}

%\setmainfont{QTChanceryType}

\begin{document}

\setlength{\skip\footins}{18pt}	
	
{\linespread{1.2}\selectfont

\vspace*{3em}

\begin{center}
%{\relscale{1.2}{\fontfamily{qcr}\fontseries{b}\selectfont 
%{\colorbox{black}{\color{blue}{\llWC{} Database Engine \\and 
%\llMOSAIC{} Native Application Toolkit}}}}}

\colorlet{ctmp}{logoPeach!20!gray}
\colorlet{ctmpp}{ctmp!90!yellow}
\colorlet{ctmppp}{ctmpp!50!black}
\colorlet{ctmpppp}{ctmppp!90!logoRed}
\colorlet{ctmcyan}{ctmpppp!70!cyan}

\colorlet{ctmppppy}{ctmppp!60!orange}

\vspace{1em}

%{\colorbox{darkBlGreen!30!darkRed}{%
\begin{tcolorbox}
[
%%enhanced,
%%frame hidden,
%interior hidden
arc=2pt,outer arc=0pt,
enhanced jigsaw,
width=\textwidth,
colback=ctmppppy!40,
%colback=ctmcyan!50,
colframe=logoRed!30!darkRed,
drop shadow=logoPurple!50!darkRed,
%boxsep=0pt,
%left=0pt,
%right=0pt,
%top=2pt,
]
%\hspace{22pt}
\begin{minipage}{\textwidth}	
\begin{center}	
{\setlength{\fboxsep}{32pt}
	\relscale{1.27}{{\fontfamily{qcr}\fontseries{b}\selectfont%
{Developing a Data Mining Repository to Accelerate Covid-19 Research 
for the Scientific Community and Public Policy Makers}
}}}
\end{center}
\end{minipage}
\end{tcolorbox}
\end{center}

\vspace*{1.15em}
\hspace{-7pt}\parbox{1.02\textwidth}{%
{\fontfamily{pzc}\selectfont   
LTS is founded by Amy Neustein, PhD, 
Series Editor of {\bf Speech Technology and 
Text Mining in Medicine and Health Care} (de Gruyter); 
Editor of {\bf Advances in Ubiquitous Computing: 
Cyber-Physical Systems, Smart Cities, 
and Ecological Monitoring} 
(Elsevier, 2020); and 
co-author (with Nathaniel Christen) 
of {\bf Cross-Disciplinary Data Integration Models
for the Emerging Covid-19 Data Ecosystem} 
(Elsevier, forthcoming).}}
\vspace*{.75em}	

\section{Introduction}
\p{The LTS Cross-Disciplinary Repository for Covid-19 Research 
(\CRtwo{}) is a collection 
of open-access research data sets related to 
SARS-COV-2 and Covid-19, which is being developed 
as a supplement to the forthcoming Elsevier volume 
examining Covid-19 research from the perspective 
of text and data mining technologies that 
cut across disiplines.  The benefit of \CRtwo{} 
is that it can accelerate Covid-19 research by 
(1) pooling a diverse collection of data sets into a 
single resource which scientists can utilize; 
(2) serving as the prototype for larger research 
portals that can aggregate new Covid-19 data 
that will emerge from hospitals, labs, and 
academic institutions in the future; (3) formalizing 
a framework for aggregating patient narratives 
to accurately capture first-hand subjective symptomatology of 
the patient suffering from Covid-19; and 
(4) accelerating the implementation of novel 
data-integration and software-development 
technologies which can contribute to scientific 
progress \visavis{} Covid-19 
in particular, and biomedical/scientific computing 
methodology in general.
The software used to curate \CRtwo{} data has 
diverse applications for software and database engineering, 
and provides solutions to technical problems with a 
broad reach in the private sector.  Further 
documentation of the \CRtwo{} technology and 
products may be found on the development repository 
for the aggregation of \CRtwo{} data 
(\href{https://github.com/Mosaic-DigammaDB/CRCR}{Mosaic-DigammaDB/CRCR}).}

\section{Background}
\p{The sudden emergence of Covid-19 as a global crisis has 
cast a spotlight on computational and technological challenges 
which, in the absence of a catastrophic pandemic, would 
rarely rise to public attention.  In particular, an effective 
response to the dangers of \makebox{SARS-COV-2} requires coordinated 
policy making integrating diverse modes of scientific inquiry.  
Genomic, biomolecular, epidemiological, socio-demographic, clinical, 
and radiological information are all pertinent to Covid-19.  
In this environment, it is important that the 
empirical foundations for expert recommendations --- which 
in turn drive public policies of enormous social and 
economic consequence --- be transparently documented 
and critically examined.  The proper synergy between government 
and science depends on data centralization: given 
the gaps in our current Covid-19 knowledge, it is 
understandable that different jurisdictions will craft responses to 
the pandemic in different ways.  There is no central authority 
with sufficient epistemic force to legitimize homogeneous 
mandates across the entire country.  However, such 
policy differences should be a consequence of alternative interpretations of 
scientific knowledge or the diverse needs of local communities  
--- rather than being a haphazard consequence of governments 
working with divergent, competing, and poorly integrated data.}

\p{The current administration, along with numerous corporate and academic 
entities, has clearly recognized the need for a more 
centralized paradigm for sharing Covid-19 data.  For example, 
the White House spearheaded a scientific initiative to 
develop \Cnineteen{}, an open-access corpus of over 46,000 
peer-reviewed publications related to Covid-19, which 
were transformed into a common machine-readable representation 
so as to promote text and data mining.  Similarly, 
large institutions such as Google, Johns Hopkins, and 
Springer Nature have all implemented some form of coronavirus 
data-sharing platform targeted to both scientists and 
policy makers.  However, these two aspects of the 
corporate/academic contributions to Covid-19 data sharing 
(exemplified by the \Cnineteen{} White House initiative and by 
institution-generated portals, respectively) 
have been incomplete, for opposite but complementary 
reasons.  Specifically, \lCnineteen{} is highly structured and tightly 
integrated, but it focuses primarily on text mining and 
scientific documents, not \textit{research} data.  
While it is possible to find data 
sets about Covid-19 through \Cnineteen{}, the 
techniques to do so are both cumbersome and non-scalable.  
On the other hand, projects such as the Johns Hopkins coronavirus 
\q{dashboard} provide accessible data sets, yet these 
projects are isolated and do not offer the level of 
structure and integration evinced by \Cnineteen{}.  In 
short, an optimal Covid-19 research platform 
would merge the structural text-mining rigor of 
\Cnineteen{} with the data-centric focus of  
isolated projects that share Covid-19 data 
with the scientific community, policy makers, 
and the general public.}

\section{Description}
\p{The design of \CRtwo{} derives from the principles outlined 
in the previous paragraph.  In particular, an ideal data-sharing 
ecosystem should merge data from multiple sources, but should do so 
in a fashion which yields a machine-readable totality, 
analogous to \Cnineteen{}'s structuration with respect to 
text mining.  The merit of \CRtwo{} therefore lies not 
only in the data which it will encompass but also in 
novel technology that it will concretize for constructing 
data repositories adhering to these principles.  
Accordingly, \CRtwo{} can provide value at different 
scales of realization.  Relatively small data 
sets serve several scientific and computational 
purposes: (1) they can provide researchers 
with a mental picture of how data in different 
disciplines, projects, and experiments is structured; 
(2) they can serve as a prototype and testing 
kernel for technologies implemented to manipulate 
data in relevant formats and encodings; and 
(3) they can lay the foundation for data-integration 
strategies.  For example, when designing a 
representation format and/or implementing code 
to merge different data formats into a single 
structure (or meta-structure), it is useful 
to work with small, representative examples 
of the data structures involved, so as not 
to complicate the integration logic with 
computational details solely oriented to 
scaling up the data-management logistics.  
As a result, \CRtwo{} can provide a 
testbed for implementing data-integration 
technologies which can scale up as needed.  
To fulfill this mission, \CRtwo{} can aggregate 
relatively small data sets which have 
previously been published on academic and research 
portals, such as Springer Nature, Dryad, and DataVerse.  
At the same time, a more substantial 
(and not necessarily fully open-access) Covid-19 
data-set collection would also be beneficial to the 
scientific and policy-making community.  Ideally, then, 
\CRtwo{} will be paired with a larger technology which shares 
a similar implementational strategy but with different 
accession paradigms, allowing for an open-ended 
collection of Covid-19 data which users may 
selectively access (instead of a single package 
that users may acquire as an integrated resource).  
The common denominator in both cases 
(whether the focus is on relatively smaller or 
larger data sets) is the 
importance of deploying novel and contemporary 
data-integration techniques to centralize 
Covid-19 research as much as possible.  
Accordingly, this summary will briefly explain 
how \CRtwo{} can accelerate Covid-19 data integration 
on both a practical and technological level.}
 
\section{Methodology for Covid-19 Data Integration}

\p{As indicated above, pertinent Covid-19 data is drawn 
from multiple scientific disciplines.  On a technological level, 
Covid-19 data is documented via a wide array of file 
types and data formats.  This diversity presents technological 
challenges: if a Covid-19 information space encompasses 
files representing 25 different incompatible 
formats, users would need 25 different technologies 
to fully benefit from this data.  In many 
cases, however, data incompatibilities are 
merely superficial --- an important subset of 
Covid-19 data, for example, has a common 
tabular meta-model, even if the data is 
realized in discordant technologies (spreadsheets, 
relational databases, comma-separated-value or 
Numeric Python files, and so forth).  Applying \CRtwo{}'s technology, 
one level of data integration can thus 
be achieved simply by encoding tabular 
structure into a common representation: any field in a table 
can be accessed via a record number and a column name and/or 
index.  In some cases, more rigorous integration is also 
possible --- for example, by identifying situations where 
columns in one table correspond semantically or conceptually 
to those in another table.  In either case, 
it is reasonable to assume that a single abstract data 
format lies behind surface data-expression in patterns 
such as spreadsheets and comma-separated values 
(\CSV{}), so that all files in an 
archive encoding spreadsheet-like data can be 
migrated to a common model.}

\p{Other forms of clinical and epidemiological inputs are often 
more amenable to graph-like representations.  For instance, 
trajectories of viral transmission through 
person-to-person contact is obviously an instance 
of social network analysis.  Similarly, models of 
clinical treatments and outcomes can take graph-like 
form insofar as there are causal or institutional 
relations between discrete medical events: 
a certain clinical observation \textit{causes} a 
care team to request a laboratory analysis, 
which \textit{yields} results that \textit{factor} 
into the team's decision to \textit{administer} some 
treatment (e.g., a drug \textit{from} a particular 
provider \textit{with} a specific chemical structure), which 
observationally \textit{results} in the patient improving 
and eventually \textit{being} discharged.  In short, 
patient-care information often takes the form 
--- at least conceptually --- of a network comprised 
of different \q{events,} each event involving some 
observation, action, intervention, or decision made 
by care providers, and where the important data 
lies in how the events are interconnected: both their 
logical relationships (e.g., cause/effect) and their 
temporal dynamics (how long before a drug leads to a 
patient's improvement; how much time elapses between admission to 
a hospital and discharge).  These graph-like representations 
are a natural formalization of \q{patient-centered} data 
models.}

\p{Using \CRtwo{} associated software, a higher level of 
data integration can then be 
achieved by merging tabular and graph-like models into a 
single \textit{hypergraph} format.  A 
significant subset of Covid-19 data (or, more generally, 
any clinical/biomedical information) 
conforms to either tabular or graph structures; 
thus it is feasible to unify all of this information 
into a common framework.  A graph-plus-table 
architecture is generally considered some form of 
Hypergraph model, and indeed \CRtwo{} adopts a hypergraph 
paradigm to merge many different sorts of information into a 
common structure.  In particular, \CRtwo{} introduces 
a new \q{Hypergraph Exchange Format} (\HGXF{}) which 
can provide a text encoding of many files that, 
when originally published, embodied a diverse 
array of file-types requiring a corresponding 
array of different technologies.  \lCRtwo{} 
will include specialized computer code that 
would enable machine-readability of the \HGXF{} files, 
and use them to create hypergraph-database instances.  
In short, \CRtwo{} will promote Covid-19 data integration 
by translating a wide range of files into a common 
\HGXF{} format, something that has not yet been done 
before.\footnote{Not every format relevant to Covid-19 can be 
realistically translated to \sHGXF{}.  In particular, 
scienctific fields requiring substantial quantitative 
analysis --- e.g., biomechanics or genomics --- 
express data via encodings optimized for relevant 
mathematical operations.  In this scenario, 
\sCRtwo{} will not attempt to migrate \textit{all} 
of a data file to \sHGXF{}.  However, even for 
these files \sCRtwo{} will generally provide a 
supplemental \sHGXF{} encoding supplying data 
\textit{about} the original file, with information 
about the file type, preferred software components 
for viewing/manipulating its data, and so forth.  
In this manner the contents of non-\sHGXF{} files 
can be indirectly included into the \sCRtwo{} 
hypergraph-based ecosystem.}}

\section{Hypergraph Data Models and Multi-Application Networks}

\p{As has been outlined thus far, via the \CRtwo{} 
technology most Covid-19 data can be wholly or partially integrated 
into a single hypergraph framework, which accordingly simplifies 
the process of designing software applications and 
algorithms to analyze and manipulate this data.  
Specifically, software components can employ 
a single code library to obtain, read, consume, 
and store data, rather than needing to re-implement 
this logic for a large number of different file formats 
and/or database models.}

\p{Quality software (especially in the clinical and 
biomedical context) demands a balance between 
applications which are either too broad or too narrow 
in scope.  On the one hand, doctors often complain 
that homogeneous Electronic Health Record systems (where 
every digital record or observation is managed by a single 
all-encompassing application) are unwieldy and hard to work 
with.  This is understandable, because the clinical tasks 
of health care workers with different specializations can be very 
different.  On the other hand, doctors also complain about 
software and information systems which are so balkanized 
that they must repeatedly switch between different, non-interoperable 
applications.  In short, clinical, diagnostic, and research software 
should be neither too homogeneous nor too isolated; finding the 
proper balance between these extremes is, no doubt, 
a major challenge to the usability of electronic health 
systems going forward.}

\p{Against this background \CRtwo{} demonstrates novel 
solutions to this problem: it focuses on the dimensions 
of data acquisition and management that are specific 
to individual scientific or medical specializations, 
while also identifying requirements that are 
consistent across domains.  Scientific software 
generally needs to hone in 
on the data visualization and analytic 
requirements of particular disciplines; for example, 
biochemists use different programs than 
astrophysicists.  However, much 
of the code underlying scientific applications 
has nothing to do with these high-level 
models or theories, but is simply a 
fulfillment of basic data-management 
functionality --- data storage, accession, provenance, 
searching, user validation, and so forth.  
In effect, the computational requirements 
of scientific and biomedical software can 
be partitioned into two classes: (1) 
domain-specific logic which reflects the 
quantitative or theoretical models of 
narrow scientific fields; and (2) 
data-management logistics which can be 
realized within a central access hub, rather 
than being re-implemented by each application 
in isolation.}

\p{In short, the architecture enabled by \CRtwo{} 
conceives of a central hub which is responsible for storing 
data and for serving as a common access point 
--- providing the \q{gateway} where authorized 
users can gain access to heterogeneous information 
spaces utilized by an array of domain-specific 
software applications.  Since peer applications  
would not be directly responsible for data persistence 
or user identity management, they can focus 
on their specific data analysis and visualization 
capabilities.  The central hub, serving multiple 
peer applications, is then a heterogeneous data space 
managing information from multiple applications while 
also tracking information about the applications 
themselves: helping users to identify and launch the 
software which is most directly relevant to their 
clinical or research needs at the moment.  Meanwhile, 
because peer applications are jointly connected to a 
central hub, it is possible to implement scientific 
workflows where one application may send and receive 
data from its peers, allowing applications to complement 
each others' capabilities.}

\p{This multi-application networking architecture 
has precedents in some of the current database and 
engineering technologies.  For example, many hospitals and 
medical institutions employ some version of a 
\q{Data Lake,} pooling disparate data sources into 
a heterogeneous aggregate which is then accessed 
by multiple client applications.  Similarly, Machine Learning 
and Artificial Intelligence often adopts \q{software agents} 
or analytic modules in contexts such as Online 
Analytic Processing, which again represent semi-autonomous 
software components sharing an originary 
data hub.  Web applications, too, often act as domain-specific 
subsidiaries deferring operational requirements, such 
as user authentication or transaction processing, to a 
central web service.  The limitation of multi-application 
networks in these existing contexts are that the 
software agents involved are generally \q{lightweight,} 
with relatively primitive user-interface design.  
By contrast, the hypergraph technology introduced with \CRtwo{} 
will support multi-application networking in the context of more 
substantial desktop-style scientific applications.  In sum, 
the novel hypergraph technology developed by LTS offers a 
hybrid of the development methodologies 
employed for desktop scientific software and those 
applicable to multi-agent heterogeneous data stores, like 
a Semantic Data Lake.  To accomplish these goals, 
\CRtwo{} will utilize 
a new hypergraph database engine, coded in the \Cpp{} 
programming language, which has a unique focus on 
supporting native \GUI{} applications from the ground 
up, including persisting application state and 
storing application documentation within 
the database itself.}

\section{Adding Patient Narratives to Covid-19 Data}

\p{In addition to aggregating published data sets, \CRtwo{} 
may be used as a repository for collecting new Covid-19 
information.  With that in mind, we are prioritizing the 
design of a standard for storing and accessing 
natural-language text representing patients' subjective 
symptom descriptions, which is quite useful for 
diagnostic/prognostic assessments of patients 
infected by Covid-19.}

\p{Just as \CRtwo{} envisions a curation of published 
data sets for data mining to improve machine-readability 
of Covid-19 research, LTS also sees the 
benefit of a repository of patient narratives prepared 
for text mining, to improve machine readability of the 
open-ended symptom descriptions offered by patients.  
While \CRtwo{} does not need to specify how these 
narratives should be collected, it will implement 
a common representational format 
so that patient narratives can be pooled, similar to  
to how \Cnineteen{} research texts are merged and 
encoded with a system that permits annotation.}

\p{In modeling patient narratives, this technology 
will be oriented toward the scientific-computing 
ecosystem outlined in the previous section.  
In particular, we assume that \GUI{}-based desktop 
applications will be the primary instruments for 
data collection and analysis; this means that 
the encoding of patient narratives may, at times, 
need to be paired with \GUI{} or multi-media content.  For example, 
the software fot patients to submit 
medical history information could also allow 
them to pair (text-form) narratives with 
graphics indicating the location of pain or 
discomfort.  Furthermore, the software could 
allow narratives to be accompanied 
by an audio file where patients could cough/speak into 
a microphone.  A patient-narrative 
encoding must therefore, in light of this 
range of possible inputs, be flexible enough to 
include diverse multi-media content.}

\p{As described earlier, an information space 
adapted for multiple peer applications should encompass 
capabilities for saving application state 
(the current visual appearance of the program), which 
includes features for modeling instances of 
\GUI{} classes.  This technology provides the 
necessary infrastructure for managing patient 
narratives.  For example, consider a multi-media 
intake form where patients may describe symptoms by 
placing icons (representing pain or discomfort) 
against anatomic silhouettes (head/body, back/front, 
extremities, and so forth).  
As patients use such a multi-media form, \GUI{} application 
state corresponds to the patient's subjective symptomology, 
so that their graphics-based represention of symptoms 
could be incorporated into the overall patient 
narrative.  This is an example of how 
application-persistence logic can be marshaled to 
the related project of curating patient narratives.}

\p{Further documentation of text-encoding methodology applicable 
to both patient narratives and publications associated 
with \CRtwo{} research data is available on the \CRtwo{} 
web site, such as \href{http://raw.githubusercontent.com/Mosaic-DigammaDB/CRCR/master/CR2.pdf}{here} (this is a downloadable \PDF{} link; 
visit the repository to see the larger archive structure).}

\end{document}


