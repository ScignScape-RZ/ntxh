
\section{Appendix 2: A Hypergraph Ontology for Radiological Outomes}

\p{Diagnoistic imaging and radiological data sets are 
an important aspect of Covid-19 data because radiological 
analysis of the chest, particularly Computed Tomography 
scans, is a prime supplier of evidence that a patient is 
suffering from Covid-19.  The Radiological Society of 
North America (\RSNA{}) has announced an initiative to 
create an \q{\AI{} Imaging Data Repository} which will 
\q{support research on using medical imaging to screen for, 
diagnose and treat COVID-19} 
(see \bhref{https://www.rsna.org/covid-19}).  In 
accord with previously curated \RSNA{} data sets, 
this repository will include images along with 
annotations encoding analytic results indicating 
image features that suggest a Covid-19 diagnosis.  
Radiological data is intrinsically multi-modal, 
because radiological findings typically 
include image graphics as well as image 
annotations, diagnostic codes, and a certain 
amount of basic clinical data.  Recent 
\q{clinical effectiveness} initiatives have 
attempted to expand the scope of radiological 
data by incorporating more, and more detailed, 
clinical information.  These initiatives imply 
that radiological data will become increasingly 
multi-modal in nature in the future, which 
in turn will shape software that manages 
radiological data such as the \RSNA{} Covid-19 
repository.}

\p{Any radiological analysis is connected to 
a clinical context insofar as diagnostic 
imaging has to be prescribed by a specific 
doctor (or medical team), which means that 
the patient examined is already situated 
in a medical/institutional environment.  
The diagnostic tests, and subsequent findings, 
therefore can get absorbed into the larger 
space of data which accrues to a patient 
over the trajectory of their clinical history.  
The manner in which both radiological and 
clinical software are structured, however, 
often makes it possible to  

}


\p{}

\p{}

