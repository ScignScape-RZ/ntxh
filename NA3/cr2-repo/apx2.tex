\vspace{-1em}
\section{Appendix 2: A Hypergraph Ontology for Radiological Outcomes}
\vspace{-1em}
\p{Diagnoistic imaging and radiological data sets are 
an important aspect of Covid-19 data because radiological 
analysis of the chest, particularly Computed Tomography 
scans, is a prime supplier of evidence that a patient is 
suffering from Covid-19.  The Radiological Society of 
North America (\RSNA{}) has announced an initiative to 
create an \q{\AI{} Imaging Data Repository} which will 
\q{support research on using medical imaging to screen for, 
diagnose and treat COVID-19} 
(see \bhref{https://www.rsna.org/covid-19}).  In 
accord with previously curated \RSNA{} data sets, 
this repository will include image series paired with 
annotations encoding analytic results indicating 
image features that suggest a Covid-19 diagnosis.  
Radiological data is intrinsically multi-modal, 
because radiological findings typically 
include image graphics as well as image 
annotations, diagnostic codes, and a certain 
amount of basic clinical data.  Recent 
\q{comparative effectiveness} initiatives have 
attempted to expand the scope of radiological 
data by incorporating more, and more detailed, 
clinical information.  These initiatives imply 
that radiological data will become increasingly 
multi-modal in nature in the future, which 
in turn will shape software that manages 
radiological data such as the \RSNA{} Covid-19 
repository.}

\p{Any radiological analysis is connected to 
a clinical context insofar as diagnostic 
imaging has to be prescribed by a specific 
doctor (or medical team), which means that 
the patient examined is already situated 
in a medical/institutional environment.  
The diagnostic tests, and subsequent findings, 
therefore can get absorbed into the larger 
space of data which accrues to a patient 
over the trajectory of their clinical history.  
The manner in which both radiological and 
clinical software are structured, however, 
often makes it difficult to cross-reference 
radiological data (particularly in the 
context of image data sets, which may be 
drawn from a diverse patient population) 
with corresponding details describing 
patients' treatment protocols and medical 
outcomes once a radiological diagnosis has 
been performed.  For a concrete example (see 
\bhref{https://www.ncbi.nlm.nih.gov/pmc/articles/PMC5119276/pdf/nihms-822829.pdf}), 
\q{to retrieve the radiation treatment plan} 
associated with a \CT{} scan, one \q{has to search for the 
RTSTRUCT object based on the specific CT scan, and from there 
search for the RTPLAN object based on the RTSTRUCT object [which] is an inefficient operation because all RTSTRUCT [and] RTPLAN files for the patient need to be processed to find the correct treatment plan.}  
Even within the context of analyzing radiological 
data sets for radiology-specific data, established 
\PACS{} software is not designed for large-scale query 
evaluation linked to clinical data.  As pointed out 
by the Semantic \DICOM{} documentation (see 
\bhref{https://semantic-dicom.com/starting-page/}), 
queries like \q{display all patients with a
bronchial carcinoma bigger than 50 cm$^3$} cannot 
be processed by \PACS{} systems: \q{although there are various powerful clinical applications to process image data and image data series to create significant clinical analyses, none of these analytic results can be merged with the clinical data of a single patient.  However, retrieving this data is an essential prerequisite to perform such search queries.}  
These inefficiencies limit researchers' ability to 
perform data mining which incorporates diagnostic 
images, and also to prepare well-integrated data 
sets for advancing diagnostic and Comparative Effectiveness 
research.}

\p{In a well-integrated research framework, treatment plans 
should be linked both to the diagnostic data which 
motivated them (including radiological images) and 
to reports of treatment outcomes.  As a concrete 
example, consider the choice to use \IORT{} 
(Intraoperative Radiation Therapy) as an alternative to 
conventional cancer treatments (see 
\bhref{https://www.cuimc.columbia.edu/news/new-cancer-treatment-personalized-radiation-therapy-during-cancer-surgery}).  
Researchers can ask two different questions related to the 
effectiveness of \IORT{}: (1) which diagnostic 
cues indicate that a given patient may benefit from 
\IORT{} in lieu of longer-term therapies, and 
(2) which treatment is more effective at 
promoting better survival rates.  These two lines 
of inquiry both require an integrated data querying 
mechanism that recognizes data from image analysis, 
from clinical outcomes, and from descriptions of 
medical treatment, each kind of data to be potentially 
included as a source of parameters with which to define 
patient cohorts or to search for statistical correlations.  
The relatively new \IORT{} treatment is an example of 
why integrated research is important --- oncologists 
still have limited understanding of which diagnostic 
clues are statistically associated with positive 
outcomes from \IORT{} as compared with conventional therapy.}

\p{Of course, Covid-19 is an even newer research priority, 
and initiatives such as the \RSNA{} repository are just 
beginning to aggregate radiological data in a manner which 
permits broad-scale analysis to be performed.  This new 
data presents opportunities for the use of semantic 
annotation technology, including ontologies such 
as \SeDI{} or ViSion (see \bhref{https://epos.myesr.org/esr/viewing/index.php?module=viewing_poster&task=&pi=155548}).  Although 
radiological ontologies are primarily designed 
for use in laboratory software itself --- i.e., the 
programs radiologists use directly to derive and 
present their findings --- radiological research 
in general introduces further application domains, 
including software for image analysis, for 
integrating radiological and clinical data, and 
for examining radiological data sets.}


\p{Insofar as \PACS{} and \DICOM{} systems need 
to be paired with applications in these other 
categories, it could be valuable to 
develop radiological \q{application} ontologies, 
alongside the canonical \q{reference} ontologies 
associated with diagnostic imaging.  Starting with 
the Covid-19 context, but designed for more 
general applications as well, one component of \CRtwo{} 
is a novel radiology-focused \textit{application} 
ontology, specifically the \q{Hypergraph Ontology 
for Diagnostic Imaging, Clinical 
Outcomes, and Data Mining} (\HDICOM{}).  The goal 
of \HDICOM{} is to connect \RDF{}-based 
radiological ontologies with software applications 
designed to manipulate radiological (and corresponding 
treatment and outcomes) data.  Using the more 
expressive Hypergraph Ontology framework, developers 
via \HDICOM{} can model how software objects 
are connected to classes and properties standardized 
within vocabularies such as \LexRad{}.  
The \HDICOM{} system will also help developers 
describe hypergraph and data-mining models 
applicable to the information managed by 
their specific applications.}

%\p{In the context of data mining, there are different 
%operations which belong to a data-mining pipeline, 
%distinctions which are appropriate to make within an 
%Application Ontology.  For example, we can distinguish 
%between \textit{integrative} data mining (merging 
%heterogeneous data into a common representation so as 
%to enlarge the available search space), 
%\textit{selective} data mining (filtering a large data 
%space to select a smaller aggregate that can be 
%directly managed by a specific application; e.g., a 
%patient cohort),  
%\textit{holistic} data mining (large-scale analysis 
%to uncover statistical patterns or correlations), 
%and \textit{investigative} data mining (executing 
%multi-parameter queries against a large, heterogeneous, 
%and/or multi-modal data space).
%}

\p{Within data mining and image analysis, hypergraph 
models are used in different ways for different 
algorithms.  In the context of Covid-19 radiology, 
\bhref{https://arxiv.org/pdf/2005.04043.pdf} 
describes an algorithm for assessing the probability 
of SARS-CoV-2 infection from chest \CT{} scans, 
where hypernodes represent high-dimensional 
vectors (191 dimensions overall) and hyperedges 
represent k-nearest-neighbors; here each hypernode 
represents an entire image, mapped to a 191-dimensional 
feature-vector.  On the other hand, other image-analysis 
methods use hypernodes to designate smaller segments 
\textit{within} the image, where hyperedges 
designate geometric adjacency and/or feature-space 
similarities.  Whatever the algorithm, hypergraph 
analyses would employ a hypergraph library 
to store preliminary data for analysis and/or for 
serialization within a data set.  One benefit of a 
Hypergraph Application Ontology is therefore that 
these data structures used internally to implement 
analytic methods can be directly expressed within 
the ontology, wheres \RDF{} ontologies can only 
model hypergraphs indirectly.}

%\p{}

