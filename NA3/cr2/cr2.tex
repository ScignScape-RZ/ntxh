\documentclass[11pt,letterpaper]{article}


% pmml  arff  openannotation

\usepackage[T1]{fontenc}
\usepackage{tgtermes}

\usepackage[hang,flushmargin]{footmisc}

\usepackage{titlesec}

\titlespacing*{\section}
{0pt}{4.5ex plus 1ex minus .5ex}{.9ex plus .2ex}

%\usepackage{mathptmx}

\usepackage{eso-pic}

%\setlength\parindent{0pt}

\AddToShipoutPictureBG{%

\ifnum\value{page}>1{
\AtTextUpperLeft{
\makebox[20.5cm][r]{
\raisebox{-1.95cm}{%
{\transparent{0.3}{\includegraphics[width=0.29\textwidth]{e-logo.png}}	}} } }
}\fi
}

\AddToShipoutPicture{%
{
 {\color{blGreen!70!red}\transparent{0.9}{\put(0,0){\rule{3pt}{\paperheight}}}}%
 {\color{darkRed!70!purple}\transparent{1}\put(3,0){{\rule{4pt}{\paperheight}}}}
% {\color{logoPeach!80!cyan}\transparent{0.5}{\put(0,700){\rule{1cm}{.6cm}}}}%
% {\color{darkRed!60!cyan}\transparent{0.7}\put(0,706){{\rule{1cm}{.6cm}}}}
% \put(18,726){\thepage}
% \transparent{0.8}
}
}

\AddToShipoutPicture{%
\ifnum\value{page}=1
\put(257.5,918){%
	\transparent{0.7}{
		\includegraphics[width=0.2\textwidth]{logo.png}}}
\fi
}	



\AddToShipoutPicture{%
\ifnum\value{page}>1
{\color{blGreen!70!red}\transparent{0.9}{\put(300,8){\rule{0.5\paperwidth}{.3cm}}}}%
{\color{inOne}\transparent{0.8}{\put(300,10){\rule{0.5\paperwidth}{.3cm}}}}%
{\color{inTwo}\transparent{0.3}\put(300,13){{\rule{0.5\paperwidth}{.3cm}}}}

\put(301,16){%
\transparent{0.7}{
\includegraphics[width=0.2\textwidth]{logo.png}} }

{\color{blGreen!70!red}\transparent{0.9}{\put(5.6,5){\rule{0.5\paperwidth}{.4cm}}}}%
{\color{inOne}\transparent{1}{\put(5.6,10){\rule{0.5\paperwidth}{.4cm}}}}%
{\color{inTwo}\transparent{0.3}\put(5.6,15){{\rule{0.5\paperwidth}{.4cm}}}}

\fi
}

%\pagestyle{empty} % no page number
%\parskip 7.2pt    % space between paragraphs
%\parindent 12pt   % indent for new paragraph
%\textwidth 4.5in  % width of text
%\columnsep 0.8in  % separation between columns

%\setlength{\footskip}{7pt}

\usepackage[paperheight=14in,paperwidth=8.5in]{geometry}
\geometry{left=.7in,top=.6in,right=.65in,bottom=1.35in} %margins

\renewcommand{\thepage}{\raisebox{2pt}{\arabic{page}}}

\renewcommand{\footnoterule}{%
	\kern -3pt
	\hrule width .92\textwidth height .5pt
	\kern 10pt
}


\usepackage[hyphens]{url}
\newcommand{\biburl}[1]{ {\fontfamily{gar}\selectfont{\textcolor[rgb]{.2,.6,0}%
{\scriptsize {\url{#1}}}}}}

%\linespread{1.3}

\newcommand{\sectsp}{\vspace{12pt}}

\usepackage{graphicx}
\usepackage{color,framed}

\usepackage{textcomp}

\usepackage{float}

\usepackage{mdframed}


\usepackage{setspace}
\newcommand{\rpdfNotice}[1]{\begin{onehalfspacing}{

\Large #1

}\end{onehalfspacing}}

\usepackage{xcolor}

\usepackage[hyphenbreaks]{breakurl}
\usepackage[hyphens]{url}

\usepackage{hyperref}
\newcommand{\rpdfLink}[1]{\href{#1}{\small{#1}}}
\newcommand{\dblHref}[1]{\href{#1}{\small{\burl{#1}}}}
\newcommand{\browseHref}[2]{\href{#1}{\Large #2}}

\colorlet{blCyan}{cyan!50!blue}

\definecolor{darkRed}{rgb}{.2,.0,.1}


\definecolor{blGreen}{rgb}{.2,.7,.3}

\definecolor{darkBlGreen}{rgb}{.1,.3,.2}

\definecolor{oldBlColor}{rgb}{.2,.7,.3}

\definecolor{blColor}{rgb}{.1,.3,.2}

\definecolor{elColor}{rgb}{.2,.1,0}
\definecolor{flColor}{rgb}{0.7,0.3,0.3}

\definecolor{logoOrange}{RGB}{108, 18, 30}
\definecolor{logoGreen}{RGB}{85, 153, 89}
\definecolor{logoPurple}{RGB}{200, 208, 30}

\definecolor{logoBlue}{RGB}{4, 2, 25}
\definecolor{logoPeach}{RGB}{255, 159, 102}
\definecolor{logoCyan}{RGB}{66, 206, 244}
\definecolor{logoRed}{rgb}{.3,0,0}

\newcommand{\colorq}[1]{{\color{logoOrange!70!black}{\q{\small\textbf{#1}}}}}

\definecolor{inOne}{rgb}{0.122, 0.435, 0.698}% Rule colour
\definecolor{inTwo}{rgb}{0.122, 0.698, 0.435}% Rule colour

\definecolor{outOne}{rgb}{0.435, 0.698, 0.122}% Rule colour
\definecolor{outTwo}{rgb}{0.698, 0.435, 0.122}% Rule colour

\colorlet{linkcolor}{flColor!60!red}


\hypersetup{
	colorlinks=true,
	citecolor=blCyan!40!green,
	filecolor=magenta!30!logoBlue,
	urlcolor=blue,
    linkcolor=linkcolor!70!black,
%    allcolors=blCyan!40!green
}


\usepackage[many]{tcolorbox}% http://ctan.org/pkg/tcolorbox

\usepackage{transparent}

\newlength{\bsep}
\setlength{\bsep}{-1pt}
\let\xbibitem\bibitem
\renewcommand{\bibitem}[2]{\vspace{\bsep}\xbibitem{#1}{#2}}

\newenvironment{cframed}{\begin{mdframed}[linecolor=logoPeach,linewidth=0.4mm]}{\end{mdframed}}

\newenvironment{ccframed}{\begin{mdframed}[backgroundcolor=logoGreen!5,linecolor=logoCyan!50!black,linewidth=0.4mm]}{\end{mdframed}}

\usepackage{aurical}
\usepackage[T1]{fontenc}

\usepackage{relsize}

\newcommand{\bref}[1]{\hspace*{1pt}\textbf{\ref{#1}}}

\newcommand{\pseudoIndent}{

\vspace{10pt}\hspace*{12pt}}

\newcommand{\YPDFI}{{\fontfamily{fvs}\selectfont YPDF-Interactive}}

%
\newcommand{\deconum}[1]{{\protect\raisebox{-1pt}{{\LARGE #1}}}}

\newcommand{\visavis}{vis-\`a-vis}

\newcommand{\VersatileUX}{{\color{red!85!black}{\Fontauri Versatile}}%
{{\fontfamily{qhv}\selectfont\smaller UX}}}

\newcommand{\NDPCloud}{{\color{red!15!black}%
{\fontfamily{qhv}\selectfont {\smaller NDP C{\smaller LOUD}}}}}

\newcommand{\MThreeK}{{\color{blGreen!45!black}%
{\fontfamily{qhv}\fontsize{10}{8}\selectfont {M3K}}}}


\newcommand{\lfNDPCloud}{{\color{red!15!black}%
{\fontfamily{qhv}\selectfont N{\smaller DP C{\smaller LOUD}}}}}

\newcommand{\textds}[1]{{\fontfamily{lmdh}\selectfont{%
\raisebox{-1pt}{#1}}}}

\newcommand{\dsC}{{\textds{ds}{\fontfamily{qhv}\selectfont \raisebox{-1pt}
{\color{red!15!black}{C}}}}}

\definecolor{tcolor}{RGB}{24,52,61}

\newcommand{\CCpp}{\resizebox{!}{7pt}{\AcronymText{C}}/\Cpp{}}
\newcommand{\NoSQL}{\resizebox{!}{7pt}{\AcronymText{NoSQL}}}
\newcommand{\SQL}{\resizebox{!}{7pt}{\AcronymText{SQL}}}

\newcommand{\NCBI}{\resizebox{!}{7pt}{\AcronymText{NCBI}}}

\newcommand{\HTXN}{\resizebox{!}{7pt}{\AcronymText{HTXN}}}

\newcommand{\TDM}{\resizebox{!}{7pt}{\AcronymText{TDM}}}

\newcommand{\lHTXN}{\resizebox{!}{7.5pt}{\AcronymText{H}}%
\resizebox{!}{6.5pt}{\AcronymText{TXN}}}

\newcommand{\lsHTXN}{\resizebox{!}{9.5pt}{\AcronymText{\textcolor{tcolor}{HTXN}}}}

\newcommand{\LAF}{\resizebox{!}{7pt}{\AcronymText{LAF}}}

\newcommand{\UDpipe}{\resizebox{!}{7pt}{\AcronymText{UDpipe}}}

\newcommand{\C}{\resizebox{!}{7pt}{\AcronymText{C}}}


\usepackage{mdframed}

\newcommand{\cframedboxpanda}[1]{\begin{mdframed}[linecolor=yellow!70!blue,linewidth=0.4mm]#1\end{mdframed}}


\newcommand{\PVD}{\resizebox{!}{7pt}{\AcronymText{PVD}}}

\newcommand{\THQL}{\resizebox{!}{7pt}{\AcronymText{THQL}}}
\newcommand{\lTHQL}{\resizebox{!}{7.5pt}{\AcronymText{THQL}}}

\newcommand{\SDK}{\resizebox{!}{7pt}{\AcronymText{SDK}}}
\newcommand{\NLP}{\resizebox{!}{7pt}{\AcronymText{NLP}}}

\newcommand{\AXF}{\resizebox{!}{7pt}{\AcronymText{AXF}}}

\newcommand{\lAXF}{\resizebox{!}{7.5pt}{\AcronymText{A}}%
\resizebox{!}{6.5pt}{\AcronymText{XF}}}


\newcommand{\lsAXF}{\resizebox{!}{8.5pt}{\AcronymText{AXF}}}

\newcommand{\AXFD}{\resizebox{!}{7pt}{\AcronymText{AXFD}}}

\newcommand{\lAXFD}{\resizebox{!}{7.5pt}{\AcronymText{A}}%
\resizebox{!}{6.5pt}{\AcronymText{XFD}}}


\newcommand{\IJST}{\resizebox{!}{7pt}{\AcronymText{IJST}}}

\newcommand{\BioC}{\resizebox{!}{7pt}{\AcronymText{BioC}}}

\newcommand{\CoNLL}{\resizebox{!}{7pt}{\AcronymText{CoNLL}}}
\newcommand{\CoNLLU}{\resizebox{!}{7pt}{\AcronymText{CoNLL-U}}}

\newcommand{\sapp}{\resizebox{!}{7pt}{\AcronymText{Sapien+}}}
\newcommand{\lsapp}{\resizebox{!}{8.5pt}{\AcronymText{Sapien+}}}
\newcommand{\lssapp}{\resizebox{!}{9.5pt}{\AcronymText{Sapien+}}}

\newcommand{\ePub}{\resizebox{!}{7pt}{\AcronymText{ePub}}}

%\lsLPF


\newcommand{\GIT}{\resizebox{!}{7pt}{\AcronymText{GIT}}}

%\definecolor{atColor}{RGB}{11, 71, 17}
\definecolor{atColor}{RGB}{50, 22, 40}
\newcommand{\ATextClr}[1]{\textcolor{atColor}{\textbf{#1}}}

\newcommand{\DgDb}{\makebox{\raisebox{-3pt}{\resizebox{!}{11pt}{\ATextClr{%
\rotatebox{17}{$\varsigma$}}}}\hspace{-4pt}%
\resizebox{!}{6.5pt}{\ATextClr{D\hspace{-2pt}B}}}}

\newcommand{\lDgDb}{\makebox{\raisebox{-3pt}{%
\resizebox{!}{12pt}{\ATextClr{%
\rotatebox{17}{$\varsigma$}}}}\hspace{-4pt}%
\resizebox{!}{6.5pt}{\ATextClr{D\hspace{-2pt}B}}}}

\newcommand{\URL}{\resizebox{!}{7pt}{\AcronymText{URL}}}
\newcommand{\CSML}{\resizebox{!}{7pt}{\AcronymText{CSML}}}
\newcommand{\LPF}{\resizebox{!}{7pt}{\AcronymText{LPF}}}
\newcommand{\lLPF}{\resizebox{!}{8.5pt}{\AcronymText{LPF}}}
\newcommand{\lsLPF}{\resizebox{!}{9.5pt}{\AcronymText{LPF}}}

\makeatletter

\newcommand*\getX[1]{\expandafter\getX@i#1\@nil}

\newcommand*\getY[1]{\expandafter\getY@i#1\@nil}
\def\getX@i#1,#2\@nil{#1}
\def\getY@i#1,#2\@nil{#2}
\makeatother
	
\newcommand{\rectann}[9]{%
\path [draw=#1,draw opacity=#2,line width=#3, fill=#4, fill opacity = #5, even odd rule] %
(#6) rectangle(\getX{#6}+#7,\getY{#6}+#8)
({\getX{#6}+((#7-(#7*#9))/2)},{\getY{#6}+((#8-(#8*#9))/2)}) rectangle %
({\getX{#6}+((#7-(#7*#9))/2)+#7*#9},{\getY{#6}+((#8-(#8*#9))/2)+#8*#9});}


\definecolor{pfcolor}{RGB}{94, 54, 73}

\newcommand{\EPF}{\resizebox{!}{7pt}{\AcronymText{ETS{\color{pfcolor}pf}}}}
\newcommand{\lEPF}{\resizebox{!}{8.5pt}{\AcronymText{ETS{\color{pfcolor}pf}}}}
\newcommand{\lsEPF}{\resizebox{!}{9.5pt}{\AcronymText{ETS{\color{pfcolor}pf}}}}


\newcommand{\XPDF}{\resizebox{!}{7pt}{\AcronymText{XPDF}}}

\newcommand{\GRE}{\resizebox{!}{7pt}{\AcronymText{GRE}}}
\newcommand{\CAS}{\resizebox{!}{7pt}{\AcronymText{CAS}}}

\newcommand{\lMOSAIC}{%
\resizebox{!}{8pt}{\AcronymText{M}}%
\resizebox{!}{6pt}{\AcronymText{OSAIC}}}

\newcommand{\XML}{\resizebox{!}{7pt}{\AcronymText{XML}}}
\newcommand{\RDF}{\resizebox{!}{7pt}{\AcronymText{RDF}}}
\newcommand{\DOM}{\resizebox{!}{7pt}{\AcronymText{DOM}}}

\newcommand{\Covid}{\resizebox{!}{7pt}{\AcronymText{Covid-19}}}

\newcommand{\CLang}{\resizebox{!}{7pt}{\AcronymText{C}}}

\newcommand{\HNaN}{\resizebox{!}{7pt}{\AcronymText{HN%
\textsc{a}N}}}

\newcommand{\JSON}{\resizebox{!}{7pt}{\AcronymText{JSON}}}

\newcommand{\MeshLab}{\resizebox{!}{7pt}{\AcronymText{MeshLab}}}
\newcommand{\IQmol}{\resizebox{!}{7pt}{\AcronymText{IQmol}}}

\newcommand{\SGML}{\resizebox{!}{7pt}{\AcronymText{SGML}}}

\newcommand{\ASCII}{\resizebox{!}{7pt}{\AcronymText{ASCII}}}

\newcommand{\GUI}{\resizebox{!}{7pt}{\AcronymText{GUI}}}

\newcommand{\API}{\resizebox{!}{7pt}{\AcronymText{API}}}

\newcommand{\JATS}{\resizebox{!}{7pt}{\AcronymText{JATS}}}


\newcommand{\SDI}{\resizebox{!}{7pt}{\AcronymText{SDI}}}
\newcommand{\SDIV}{\resizebox{!}{7pt}{\AcronymText{SDIV}}}



\newcommand{\IDE}{\resizebox{!}{7pt}{\AcronymText{IDE}}}

\newcommand{\ThreeD}{\resizebox{!}{7pt}{\AcronymText{3D}}}

\newcommand{\FAIR}{\resizebox{!}{7pt}{\AcronymText{FAIR}}}

\newcommand{\QNetworkManager}{\resizebox{!}{7pt}{\AcronymText{QNetworkManager}}}
\newcommand{\QTextDocument}{\resizebox{!}{7pt}{\AcronymText{QTextDocument}}}
\newcommand{\QWebEngineView}{\resizebox{!}{7pt}{\AcronymText{QWebEngineView}}}
\newcommand{\HTTP}{\resizebox{!}{7pt}{\AcronymText{HTTP}}}


\newcommand{\lAcronymTextNC}[2]{{\fontfamily{fvs}\selectfont {\Large{#1}}{\large{#2}}}}

\newcommand{\AcronymTextNC}[1]{{\fontfamily{fvs}\selectfont {\large #1}}}


\colorlet{orr}{orange!60!red}

\newcommand{\textscc}[1]{{\color{orr!35!black}{{%
						\fontfamily{Cabin-TLF}\fontseries{b}\selectfont{\textsc{\scriptsize{#1}}}}}}}


\newcommand{\textsccserif}[1]{{\color{orr!35!black}{{%
				\scriptsize{\textbf{#1}}}}}}


\newcommand{\iXPDF}{\resizebox{!}{7pt}{\textsccserif{%
\textit{XPDF}}}}

\newcommand{\iEPF}{\resizebox{!}{7pt}{\textsccserif{%
\textit{ETSpf}}}}

\newcommand{\iSDI}{\resizebox{!}{7pt}{\textsccserif{%
\textit{SDI}}}}

\newcommand{\iHTXN}{\resizebox{!}{7pt}{\textsccserif{%
\textit{HTXN}}}}


\newcommand{\AcronymText}[1]{{\textscc{#1}}}

\newcommand{\AcronymTextser}[1]{{\textsccserif{#1}}}


\newcommand{\mAcronymText}[1]{{\textscc{\normalsize{#1}}}}

\newcommand{\FASTA}{{\resizebox{!}{7pt}{\AcronymText{FASTA}}}}
\newcommand{\SRA}{{\resizebox{!}{7pt}{\AcronymText{SRA}}}}
\newcommand{\DNA}{{\resizebox{!}{7pt}{\AcronymText{DNA}}}}
\newcommand{\MAP}{{\resizebox{!}{7pt}{\AcronymText{MAP}}}}
\newcommand{\EPS}{{\resizebox{!}{7pt}{\AcronymText{EPS}}}}
\newcommand{\CSV}{{\resizebox{!}{7pt}{\AcronymText{CSV}}}}
\newcommand{\PDB}{{\resizebox{!}{7pt}{\AcronymText{PDB}}}}

\newcommand{\XOCS}{{\resizebox{!}{7pt}{\AcronymText{XOCS}}}}

\newcommand{\HGXF}{{\resizebox{!}{7pt}{\AcronymText{HGXF}}}}
\newcommand{\lHGXF}{{\resizebox{!}{7.5pt}{\AcronymText{HGXF}}}}

\newcommand{\CRtwo}{{\resizebox{!}{7pt}{\AcronymText{CR2}}}}
\newcommand{\lCRtwo}{{\resizebox{!}{7.5pt}{\AcronymText{CR2}}}}

\newcommand{\ChemXML}{{\resizebox{!}{7pt}{\AcronymText{ChemXML}}}}

\newcommand{\TeXMECS}{\resizebox{!}{7pt}{\AcronymText{TeXMECS}}}

% pmml  arff  openannotation

\newcommand{\PMML}{\resizebox{!}{7pt}{\AcronymText{PMML}}}
\newcommand{\ARFF}{\resizebox{!}{7pt}{\AcronymText{ARFF}}}
\newcommand{\IeXML}{\resizebox{!}{7pt}{\AcronymText{IeXML}}}


\newcommand{\NGML}{\resizebox{!}{7pt}{\AcronymText{NGML}}}

\newcommand{\Cpp}{\resizebox{!}{7pt}{\AcronymText{C++}}}

\newcommand{\WhiteDB}{\resizebox{!}{7pt}{\AcronymText{WhiteDB}}}

\colorlet{drp}{darkRed!70!purple}

%\newcommand{\MOSAIC}{{\color{drp}{\AcronymTextNC{\scriptsize{MOSAIC}}}}}

\newcommand{\MOSAIC}{\resizebox{!}{7pt}{\AcronymText{MOSAIC}}}


\newcommand{\mMOSAIC}{{\color{drp}{\AcronymTextNC{\normalsize{MOSAIC}}}}}

\newcommand{\MOSAICVM}{\mMOSAIC-\mAcronymText{VM}}

\newcommand{\sMOSAICVM}{\resizebox{!}{7pt}{\MOSAICVM}}
\newcommand{\sMOSAIC}{\resizebox{!}{7pt}{\MOSAIC}}

\newcommand{\LDOM}{\resizebox{!}{7pt}{\AcronymText{LDOM}}}
\newcommand{\Cnineteen}{\resizebox{!}{7pt}{\AcronymText{CORD-19}}}

\newcommand{\lCnineteen}{\resizebox{!}{7.5pt}{\AcronymText{CORD-19}}}


\newcommand{\MOL}{\resizebox{!}{7pt}{\AcronymText{MOL}}}

\newcommand{\ACL}{\resizebox{!}{7pt}{\AcronymText{ACL}}}

\newcommand{\LXCR}{\resizebox{!}{7pt}{\AcronymText{LXCR}}}
\newcommand{\lLXCR}{\resizebox{!}{8.5pt}{\AcronymText{LXCR}}}
\newcommand{\lsLXCR}{\resizebox{!}{9.5pt}{\AcronymText{LXCR}}}

%\newcommand{\lMOSAIC}{{\color{drp}{\lAcronymTextNC{M}{OSAIC}}}}
\newcommand{\lfMOSAIC}{\resizebox{!}{9pt}{{\color{drp}{\lAcronymTextNC{M}{OSAIC}}}}}

\newcommand{\Mosaic}{\resizebox{!}{7pt}{\MOSAIC}}
\newcommand{\MosaicPortal}{{\color{drp}{\AcronymTextNC{MOSAIC Portal}}}}

\newcommand{\RnD}{\resizebox{!}{7pt}{\AcronymText{R\&D}}}

\newcommand{\lQt}{\resizebox{!}{8.5pt}{Qt}}
\newcommand{\QtCpp}{\resizebox{!}{8.5pt}{\AcronymText{Qt/C++}}}
\newcommand{\Qt}{\resizebox{!}{7pt}{\AcronymText{Qt}}}

\newcommand{\QtSQL}{\resizebox{!}{7pt}{\AcronymText{QtSQL}}}

\newcommand{\HTML}{\resizebox{!}{7pt}{\AcronymText{HTML}}}
\newcommand{\PDF}{\resizebox{!}{7pt}{\AcronymText{PDF}}}

\newcommand{\R}{\resizebox{!}{7pt}{\AcronymText{R}}}
\newcommand{\SciXML}{\resizebox{!}{7pt}{\AcronymText{SciXML}}}



\newcommand{\lGRE}{\resizebox{!}{7.5pt}{\AcronymText{GRE}}}

\newcommand{\p}[1]{

\vspace{.75em}#1}

\newcommand{\q}[1]{{\fontfamily{qcr}\selectfont ``}#1{\fontfamily{qcr}\selectfont ''}} 

%\newcommand{\deconum}[1]{{\textcircled{#1}}}


\renewcommand{\thesection}{\protect\mbox{\deconum{\Roman{section}}}}
\renewcommand{\thesubsection}{\arabic{section}.\arabic{subsection}}

\newcommand{\llMOSAIC}{\mbox{{\LARGE MOSAIC}}}
%\newcommand{\lfMOSAIC}{\mbox{M\small{OSAIC}}}

\newcommand{\llMosaic}{\llMOSAIC}
\newcommand{\lMosaic}{\lMOSAIC}
\newcommand{\lfMosaic}{\lfMOSAIC}


\newcommand{\llWC}{\mbox{{\LARGE WhiteCharmDB}}}

\newcommand{\llwh}{\mbox{{\LARGE White}}}
\newcommand{\llch}{\mbox{{\LARGE CharmDB}}}

\usepackage{enumitem}
%\usepackage{listings}

\colorlet{dsl}{purple!20!brown}
\colorlet{dslr}{dsl!50!blue}

\setlist[description]{%
  topsep=10pt,
  labelsep=22pt, leftmargin=5pt,
  itemsep=5pt,               % space between items
  %font={\bfseries\sffamily}, % set the label font
  font=\normalfont\bfseries\color{dslr!50!black}, % if colour is needed
}

\setlist[enumerate]{%
  topsep=3pt,               % space before start / after end of list
  itemsep=-2pt,               % space between items
  font={\bfseries\sffamily}, % set the label font
%  font={\bfseries\sffamily\color{red}}, % if colour is needed
}

%\usepackage{tcolorbox}

\newcommand{\slead}[1]{%
\noindent{\raisebox{2pt}{\relscale{1.15}{{{%
\fcolorbox{logoCyan!50!black}{logoGreen!5}{#1}
}}}}}\hspace{.5em}}


\let\OldLaTeX\LaTeX

\renewcommand{\LaTeX}{\resizebox{!}{7pt}{\color{orr!35!black}{\OldLaTeX}}}

\let\OldTeX\TeX

\renewcommand{\TeX}{\resizebox{!}{7pt}{\color{orr!35!black}{\OldTeX}}}


\newcommand{\LargeLaTeX}{\resizebox{!}{8.5pt}{\color{orr!35!black}{\OldLaTeX}}}

\setlength\parindent{0pt}
%\setlength\parindent{24pt}
%%\usepakage{newfile}

\usepackage{hyperref}

\usepackage{etoolbox}

\usepackage{zref-user}

\newwrite\sdiFile
\immediate\openout\sdiFile=\jobname.sdi.txt

\newcommand{\p}[1]{

\vspace{10pt}#1}

\newif\iftabng
\tabngfalse


\usepackage{letltxmacro}
\LetLtxMacro{\oldmmsemi}{\;}
\LetLtxMacro{\oldtbplus}{\+}
\LetLtxMacro{\oldtbgt}{\>}
\LetLtxMacro{\oldmmgt}{\+}

\newcommand{\+}{\iftabng\oldtbplus\else\sss\fi}

\renewcommand{\>}{\iftabng\oldtbplus\else
\ifmmode\oldmmgt\else\sse\sss\fi\fi}

%\renewcommand{\>}{\sse\sss}

\renewcommand{\;}{\relax\ifmmode\oldmmsemi\else\sse\fi}

\newcommand{\writeSDI}[1]{\immediate\write\sdiFile#1}

\newcommand{\emblink}[2]{\href{\#sdi:#1--#2}{\#sdi:#1--#2}}

%\newcount\sdiCounter
%\def\advsdiCounter{\global\advance\sdiCounter by1}

%\newcount\sdiCounterP
%\def\advsdiCounterP{\global\advance\sdiCounterP by1}

%\newcounter{sdiCounter}
\newcounter{sdiCounterP}[page]
\newcounter{sdiCounter}

\def\topt#1{\expandafter\the\dimexpr\dimexpr#1sp\relax\relax}

\makeatletter
%\catcode`\*=10
\newcommand{\sss}{%
\stepcounter{sdiCounterP}
\stepcounter{sdiCounter}
\pdfsavepos\write\sdiFile{!/ SDI_Sentence_Start} 
\write\sdiFile\expandafter{\expandafter$%
\expandafter i\expandafter:%
\expandafter\space\the\c@sdiCounter}
\write\sdiFile\expandafter{\expandafter$%
\expandafter o\expandafter:%
\expandafter\space\the\c@sdiCounterP}
\write\sdiFile\expandafter{\expandafter$%
\expandafter p\expandafter:%
\expandafter\space\thepage^^J%
$x: \topt\pdflastxpos^^J%
$y: \topt\pdflastypos^^J%
/!^^J%
<<>^^J%
}}
%\catcode`\%=14
\makeatother

\makeatletter
\newcommand{\sse}{%
\pdfsavepos\write\sdiFile{!/ SDI_Sentence_End} 
\write\sdiFile\expandafter{\expandafter$%
\expandafter i\expandafter:%
\expandafter\space\the\c@sdiCounter}
\write\sdiFile\expandafter{\expandafter$%
\expandafter o\expandafter:%
\expandafter\space\the\c@sdiCounterP}
\write\sdiFile\expandafter{\expandafter$%
\expandafter p\expandafter:%
\expandafter\space\thepage^^J%
$x: \topt\pdflastxpos^^J%
$y: \topt\pdflastypos^^J%
/!^^J%
<<>^^J%
}}
\makeatother




\newcommand{\lun}[1]{\raisebox{-4pt}{\fontfamily{qcr}\selectfont{%
\LARGE{\textbf{\textcolor{tcolor}{#1}}}}}\vspace{-2pt}}

\newcommand{\inditem}{\itemindent10pt\item}

\usepackage{soul}

\definecolor{hlcolor}{RGB}{114, 54, 203}
\colorlet{hlcol}{hlcolor!35}
\sethlcolor{hlcol}

\makeatletter
\def\SOUL@hlpreamble{%
	\setul{}{3ex}%         !!!change this value!!! default is 2.5ex
	\let\SOUL@stcolor\SOUL@hlcolor
	\SOUL@stpreamble
}
\makeatother

\usepackage{scrextend}
%\vspace*{3em}
\newenvironment{mldescription}{\vspace{1em}%
  \begin{addmargin}[4pt]{1em}
    \setlength{\parindent}{-1em}%
    \newcommand*{\mlitem}[1][]{\vspace{5pt}\par\medskip%
%\colorbox{hlcolor}{\textbf{##1}}\quad}\indent
\hl{ \textbf{##1} }\quad}\indent
}{%
  \end{addmargin}
  \medskip
}

\usepackage{marginnote}

\newcommand{\mnote}[1]{%
\vspace*{-2em}
\reversemarginpar
\raisebox{1em}{\marginnote{\parbox{4em}{%
\begin{mdframed}[innerleftmargin=4pt,
	innerrightmargin=1pt,innertopmargin=1pt,
	linecolor=red!20!cyan,userdefinedwidth=4em,
	topline=false,
	rightline=false]
{{\fontfamily{ppl}\fontsize{12}{0}\selectfont
		\textit{#1}}}
\end{mdframed}}
	}[3em]}}

\newcommand{\mnotel}[1]{%
\vspace*{-2em}
\reversemarginpar
\raisebox{-4em}{\marginnote{\parbox{4em}{%
\begin{mdframed}[innerleftmargin=4pt,
	innerrightmargin=1pt,innertopmargin=1pt,
	linecolor=red!20!cyan,userdefinedwidth=4em,
	topline=false,
	rightline=false]
{{\fontfamily{ppl}\fontsize{12}{0}\selectfont
		\textit{#1}}}
\end{mdframed}}
	}[3em]}}

\newcommand{\mnoteh}[3]{%
	\vspace*{#1}
	\reversemarginpar
	\raisebox{#2}{\marginnote{\parbox{4em}{%
				\begin{mdframed}[innerleftmargin=4pt,
					innerrightmargin=1pt,innertopmargin=1pt,
					linecolor=red!20!cyan,userdefinedwidth=4em,
					topline=false,
					rightline=false]
					{{\fontfamily{ppl}\fontsize{12}{0}\selectfont
							\textit{#3}}}
				\end{mdframed}}
			}[3em]}}


\newcommand{\mnoteb}[1]{%
	\vspace*{1em}
	\reversemarginpar
	\raisebox{1em}{\marginnote{\parbox{4em}{%
				\begin{mdframed}[innerleftmargin=4pt,
					innerrightmargin=1pt,innertopmargin=1pt,
					linecolor=red!20!cyan,userdefinedwidth=4em,
					topline=false,
					rightline=false]
					{{\fontfamily{ppl}\fontsize{12}{0}\selectfont
							\textit{#1}}}
				\end{mdframed}}
			}[3em]}}
	
\usepackage{wrapfig}

\usetikzlibrary{arrows, decorations.markings}
\usetikzlibrary{shapes.arrows}

\newcommand{\curicon}[2]{%
	\node at (#1,#2) [
	draw=black,
	%minimum width=2ex,
	inner sep=.7pt,
	fill=white,
	single arrow,
	single arrow head extend=3pt,
	single arrow head indent=1.5pt,
	single arrow tip angle=45,
	line join=bevel,
	minimum height=4.6mm,
	rotate=115
	] {};
}

\makeatletter
\def\@cite#1#2{[\textbf{#1\if@tempswa , #2\fi}]}
\def\@biblabel#1{[\textbf{#1}]}
\makeatother


%\let\origref\ref
%\renewcommand{\ref}[1]{{\LARGE #1}}

%\def\ref#1{\textbf{\origref{{\LARGE #1}}}}


\renewcommand{\thefootnote}{\textcolor{logoGreen!80!logoBlue}{{\fontfamily{qcr}\fontseries{b}\fontsize{10}{4}\selectfont\arabic{footnote}}}}


\newcommand{\LVee}{{\colorbox{cyan!40!yellow}{\textcolor{red!70!navy}{\textbf{\LARGE$\vee$}}}}}
\newcommand{\LWedge}{{\colorbox{cyan!40!yellow}{\textcolor{red!70!navy}{\textbf{\LARGE$\wedge$}}}}}

\renewcommand{\LVee}{}
\renewcommand{\LWedge}{}


\urlstyle{same}

%\setmainfont{QTChanceryType}

\begin{document}

\setlength{\skip\footins}{18pt}	
	
{\linespread{1.2}\selectfont

\vspace*{3em}

\begin{center}
%{\relscale{1.2}{\fontfamily{qcr}\fontseries{b}\selectfont 
%{\colorbox{black}{\color{blue}{\llWC{} Database Engine \\and 
%\llMOSAIC{} Native Application Toolkit}}}}}

\colorlet{ctmp}{logoPeach!20!gray}
\colorlet{ctmpp}{ctmp!90!yellow}
\colorlet{ctmppp}{ctmpp!50!black}
\colorlet{ctmpppp}{ctmppp!90!logoRed}
\colorlet{ctmcyan}{ctmpppp!70!cyan}

%\vspace{2em}


%{\colorbox{darkBlGreen!30!darkRed}{%
\begin{tcolorbox}
[
%%enhanced,
%%frame hidden,
%interior hidden
arc=2pt,outer arc=0pt,
enhanced jigsaw,
width=.92\textwidth,
colback=ctmcyan!50,
colframe=logoRed!30!darkRed,
drop shadow=logoPurple!50!darkRed,
%boxsep=0pt,
%left=0pt,
%right=0pt,
%top=2pt,
]
%\hspace{22pt}
\begin{minipage}{.99\textwidth}	
\begin{center}	
{\setlength{\fboxsep}{28pt}
	\relscale{1.2}{{\fontfamily{qcr}\fontseries{b}\selectfont%
{The Cross-Disciplinary Repository for Covid-19 Research}
}}}
\end{center}
\end{minipage}
\end{tcolorbox}
\end{center}

\vspace*{1.15em}
\begin{center}
\parbox{.9\textwidth}{%
{\fontfamily{pzc}\selectfont   
LTS is founded by Amy Neustein, PhD, 
Series Editor of {\bf Speech Technology and 
Text Mining in Medicine and Health Care} (de Gruyter); 
Editor of {\bf Advances in Ubiquitous Computing: 
Cyber-Physical Systems, Smart Cities, 
and Ecological Monitoring} 
(Elsevier, 2020); and 
co-author (with Nathaniel Christen) 
of {\bf Cross-Disciplinary Data Integration Models
for the Emerging Covid-19 Data Ecosystem} 
(Elsevier, forthcoming).}}
\end{center}
\vspace*{.75em}	

\p{The Cross-Disciplinary Repository for Covid-19 Research (hereafter called \CRtwo{}) is an archive of open-access research 
data related to Covid-19 and SARS-COV-2.  This repository 
will accompany the forthcoming volume \textit{Cross-Disciplinary Data 
Integration Models for the Emerging Covid-19 Data 
Ecosystem}.  The current paper will discuss 
the goals and structure of \CRtwo{}.  It will also 
explain how scientists or organizations can contribute 
data to \CRtwo{}.  The authors are hoping to 
find and/or republish data sets representing a 
diverse spectrum of scientific fields which 
contribute to our medical, biological, and 
policy-oriented knowledge about the Covid-19 pandemic.  
Our mission is to centralize Covid-19 data more 
effectively than is achieved by existing platforms, 
either general-purpose data hosting frameworks 
(Dryad, DataVerse, Mendeley) or 
research corpora focused on Covid-19, but 
not necessarily placing a priority on data mining 
and data integration (an example of the second 
category is \Cnineteen{}, 
which is a large corpus of scientific publications 
related to Covid-19).  Although a significant body 
of relevant scientific literature can be found on 
platforms such as Dryad and \Cnineteen{}, mining 
these resources for usable raw data can take 
considerable effort.  \lCRtwo{} seeks to 
streamline this process by aggregating 
Covid data into a central corpus and by 
integrating data sets, as much as possible, 
into a common representation and technological 
infrastructure.}

\p{With that said, the mission of \CRtwo{} is not 
specifically to curate a large volume of data for 
its own sake.  Indeed, the core of \CRtwo{} will be 
available as a package that can be downloaded and 
used on an ordinary computer --- analogous to \Cnineteen{}, 
which covers over 33,000 research papers but 
is encoded in such a way that the total size of 
the corpus is not prohibitively large.  We also believe 
a more substantial 
(and not necessarily fully open-access) Covid-19 
information space would be beneficial to the 
scientific community, but this larger project is 
not the direct mission of \CRtwo{}.  Ideally, \CRtwo{} can be 
paired with a larger technology sharing a similar 
implementational strategy but with different 
accession paradigms, allowing for an open-ended 
collection of Covid-19 data which users may 
selectively access, instead of a single package 
that users may assess as an integrated resource.  
However, the focus for \CRtwo{} --- and for 
the \textit{Cross-Disciplinary Integration} volume 
--- is to serve as a precursor to such a larger 
ecosystem and/or to use Covid-19 as a case-study 
in how such larger technologies may 
best be engineered.}

\p{One consequence of this background is that small 
data sets can be equally valuable for \CRtwo{} as 
larger ones --- indeed, given a volume of data 
which would be impractical to include in one 
downloadable package, it is better to extract a 
representative sample for inclusion in \CRtwo{}, 
with links to allow researchers to access the entire 
resource as desired.  Relatively small data 
sets (including samples standing in for larger 
ones) serve several scientific and computational 
purposes: (1) they can provide researchers 
with a mental picture of how data in different 
disciplines, projects, and experiments is structured; 
(2) they can serve as a prototype and testing 
kernel for technologies implemented to manipulate 
data in relevant formats and encodings; and 
(3) they can lay the foundation for data-integration 
strategies.  For example, when designing a 
representation format and/or implementing code 
to merge different data formats into a single 
structure (or meta-structure), it is useful 
to work with small, representative examples 
of the data structures involved, so as not 
to complicate the integration logic with 
computational details solely oriented to 
scaling up the data-management logistics.}

\p{Data associated with Covid-19 comes in many 
different structures and formats.  This diversity 
in and of itself presents technological challenges: 
if a Covid-19 information space encompasses 
files representing 25 different incompatible 
formats, users need 25 different technologies 
to fully utilize this data.  The point is not 
to envision one single application used for 
all Covid-19 data --- to the contrary, 
scientific software generally needs to hone in 
on the data visualization and analytic 
requirements of particular disciplines; 
biochemists use different programs than 
astrophysicists.  However, nor is it 
appropriate for software components to be so 
balkanized that common data-management 
functionality is duplicated across 
many isolated projects.  The ideal data-management 
architecture is a compromise between the 
domain-specific needs of scientific applications 
and the desire for a centralized technology 
to handle data logistics at the foundational 
level where disciplinary domains are not 
consequential: data accession, provenance, 
user validation, searching, and so forth.}

\p{In short, the ideal technology for a 
cross-disciplinary information space involves a 
central technological core which is 
common to, and serves as an entry point 
for, diverse domain-specific resources.  
This entry point can provide preliminary 
functionality such as user validation and 
\q{logging in} (to the degree that access 
to parts of a repository may be restricted), 
searching for data and/or data sets, file 
management, updating information, and so on.  
The central software may then be complemented 
with narrower applications that provide 
more complex capabilities, but within the 
context of specific disciplines or scientific 
workflows.}

\p{In the Covid-19 context, some of the relevant 
data (despite superficial differences) has a 
common table-like structure, which may be manifested 
in different ways (comma-separated values, spreadsheets, 
word-document tables, Numeric Python, \XML{}, \JSON{}, 
\SQL{} dumps, etc.).  
In general, disparate tabular data can be straightforwardly 
merged.  One level of integration simply encodes tabular 
structure into a common representation: any field in a table 
can be accessed via a record number and a column name and/or 
index.  In some cases, more rigorous integration is also 
possible, for example by identifying situations where 
columns in one table correspond semantically or conceptually 
to those in a second table.  In either case, 
it is reasonable to assume that a single abstract data 
format lies behind surface data-expression in forms 
like spreadsheets and \CSV{}, so that all files in an 
archive encoding spreadsheet-like data can be 
migrated to a common model.}

\p{Other forms of 
clinical and epidemiological inputs are more 
amenable to graph-like representations: for example, 
trajectories of viral transmission through 
person-to-person contact is obviously an instance 
of social network analysis.  Similarly, models of 
clinical treatments and outcomes can take graph-like 
form insofar as there are causal or institutional 
relations between discrete medical events: 
a certain clinical observation \textit{causes} a 
care team to request a laboratory analysis, 
which \textit{yields} results that \textit{factor} 
into the team's decision to \textit{administer} some 
treatment (say, a drug \textit{from} some provider 
\textit{with} some chemical structure), which 
observationally \textit{results} in the patient improving 
and eventually \textit{being} discharged.  In short, 
patient-care information often takes the form 
--- at least conceptually --- of a network comprised 
of different \q{events}, each event involving some 
observation, action, intervention, or decision made 
by care providers, and where the important data 
lies in how the events are interconnected: both their 
logical relationships (e.g., cause/effect) and their 
temporal dynamics (how long before a drug leads to a 
patient's improvement; how long before admission to 
a hospital and discharge).  These graph-like representations 
are a natural formalization of \q{patient-centered} data 
models as outlined, for example, in \cite{EranBellin}.}     


\p{As the previous two paragraphs have outlined, a 
significant subset of Covid-19 data (or, more generally, 
any clinical/biomedical information) 
conforms to either tabular or graph structures, 
and so it is feasible to unify all of this information 
into a common framework to the degree that one 
works with a meta-model which incorporates both 
record-sets and graph structures (node-sets and edge-sets) 
in its representational arsenal.  A graph-plus-table 
architecture is generally considered some form of 
Hypergraph model, and indeed \CRtwo{} uses a hypergraph 
paradigm to merge many different sorts of information into a 
common structure.  In particular, \CRtwo{} introduces 
a new \q{Hypergraph Exchange Format} (\HGXF{}) which 
can provide a text encoding of many files that, 
when originally published, embodied a diverse 
array of file-types requiring a corresponding 
array of different technologies.  \lCRtwo{} 
will include computer code to read \HGXF{} files 
and use them to create hypergraph-database instances.}

\p{Not every format relevant to Covid-19 can be 
realistically translated to \HGXF{}.  In particular, 
fields requiring substantial quantitative 
analysis --- e.g., biomechanics or genomics --- 
express data via encodings optimized for relevant 
mathematical operations.  In this scenario, 
\CRtwo{} will not attempt to migrate \textit{all} 
of a data file to \HGXF{}.  However, even for 
these files \CRtwo{} will generally provide a 
supplemental \HGXF{} encoding supplying data 
\textit{about} the original file, with information 
about the file type, preferred software components 
for viewing/manipulating its data, and so forth.  
In this manner the contents of non-\HGXF{} files 
can be indirectly included into the \CRtwo{} 
hypergraph-based ecosystem.}

\p{As this summary has implied, hypergraph data 
models and the \HGXF{} format are a significant 
aspect of how \CRtwo{} is designed and how it 
technologically achieves its stated goals.  The 
\HGXF{} format will therefore be examined 
in greater detail in the following section.}

\section{DigammaDB and the HGXF Format}
\p{\lCRtwo{} will introduce a new database engine for 
preparing the information provided within the 
repository (called DigammaDB, or \DgDb{} for short) 
as well as the Hypergraph Exchange Format (\HGXF{}).  In 
general, \HGXF{} files can be generated from 
\DgDb{} instances, capturing the state of the 
database at a moment in time.  \lDgDb{} could therefore 
be used to store research data.  When scientists 
choose to publish data, they would then output the 
information from their database into \HGXF{}, using 
the resulting files as the published raw data.  
In the \CRtwo{} context, \DgDb{} will be used 
to merge data from multiple files into a single 
database, yielding \HGXF{} output forming most of the 
raw data republished in \CRtwo{}.}

\p{Operationally, \DgDb{} is designed to emulate the 
programming interface provided by several existing 
databases and hypergraph libraries --- for 
instance, HyperGraphDB (\cite{BorislavIordanov}), WhiteDB
(\cite{EnarReilent}), and HgLib (\cite[\textnormal{p. 9}]{Erable}).  
That is, 
\DgDb{} is designed so that existing code using 
these technologies can be adopted for \DgDb{} 
with relatively little effort.  \lDgDb{} also 
introduces some new concepts and structuring features, 
which will reviewed below.  In addition to 
prior technologies such as HyperGraphDB, \DgDb{} 
draws on theoretical work connected to hypergraphs 
and their value as multi-paradigm, general-purpose 
metamodels.  The reach of hypergraph theory encompasses 
several paradigms for modeling scientific data, 
such as Conceptual Graph Semantics (see \cite{MattSelway}) 
and Conceptual Space Theory (which is linked to hypergraphs 
in work summarized by \cite{InteractingConceptualSpaces}, 
where Conceptual Space semantics is paired with 
hypergraph categorial grammar).  \lDgDb{} therefore 
introduces modeling elements designed to capture 
scientific details (such as dimensional analysis and units 
of measurement).  When discussing graph structures and 
programming techniques, \DgDb{} draws terminology 
from these scientific perspectives as well as from 
existing Hypergraph database engines.}
	
\subsection{Scientific Features of DigammaDB}
\p{\lDgDb{} has no specific connection to SARS-COV-2 or 
Covid-19; it is conceived as a general-purpose database 
engine that can facilitate application development 
across many domains and industries.  However, \DgDb{} is 
designed with exceptional attention to scientific research 
and software, with respect to metadata, \GUI{} integration, 
its representation of files and file-types, 
and interoperability with technologies related 
to publishing and open-access research data.  
A full enumeration of \DgDb{}'s scientific 
focus is outside the scope of this outline, but 
certain features are specifically relevant 
to \CRtwo{}, so they can be discussed here: 

\begin{description}

\item[From-the-Ground-Up \lQt{} Integration]  All 
\DgDb{} classes natively interoperate with 
\Qt{} (a leading cross-platform application-development 
and \GUI{} framework).  Programmers then have 
access to features like QDataStream and 
\Qt{} meta-objects for binary serialization of \Cpp{} 
objects for persistence in the database.  Robust 
\Qt{} support also makes it easy to design 
\GUI{} classes for interacting with \DgDb{} data, 
and for employing \DgDb{} as a technology for 
managing and storing application state.  This is 
important in the scientific computing context 
because most scientific software is designed 
as native desktop-style applications needing 
special-purpose \GUI{} classes (in this 
environment it is usually not possible to 
adopt techniques such as \HTML{} page 
templates which are commonplace in the 
web-programming context for creating user 
views onto database objects).  Therefore, 
implementing custom \GUI{} logic is an 
important part of the development requirements 
for scientific applications, and it is helpful 
to interface with a database engine prioritizing 
that aspect of software engineering.  

\item[Projections and Scientific Annotations]  
\lDgDb{} adopts the HyperGraphDB notion of 
\textit{projections}, or descriptions of 
individual fields within a complex object 
that can automate the extrapolation of 
binary-serialization algorithms (which 
in turn allows complex objects to be 
stored in \DgDb{} alongside primitive values 
like strings and numbers).  \lDgDb{} also 
uses projections as a basis for introducing 
metadata associated with Conceptual Spaces 
and Conceptual Space Markup Language (\CSML{}) 
as mentioned earlier.  In particular, projections 
can be annotated with statistical/quantitative 
details such as dimensions (e.g., base quantities 
and units of measurement), scales or \q{measurement levels} 
(in \CSML{} terms, particularly the 
distinction between nominal, ordinal, interval, and ratio 
axes), minima and maxima, probability distributions, 
and the aggregation of isolated units into complex 
dimensions, or \CSML{} \q{domains} (vectors, tensors, 
area/volume spans, and so forth).

\item[Data Microcitations]  Most scientific 
data is formally or informally linked to 
peer-reviewed publications which present 
research findings.  Published data sets 
make these connections explicit, by joining 
document identifiers and \URL{}s with 
the corresponding identifiers for designating 
and locating data sets.  These connections 
are however coarse-grained, applying only to 
documents and data sets in their entirety.  
By contrast, an emerging trend in publishing is 
to link portions of publications --- such 
as individual sentences, paragraphs, or figure 
illustrations --- with smaller parts of a data 
set (which could be an individual record/sample, 
or a table column, or some conceptually 
related group of records or columns).  To support 
microcitations, there must be some formal 
mechanism to individuate small parts of a data set.  
This structure is provided internally by 
\DgDb{}: there are multiple microcitable 
\q{zones} in a database, which introduce 
encodings for microcitation targets that 
become exposed to publications via \HGXF{}.  
Here \q{zones} refer to aspects or 
portions of a database that may be cited 
individually, apart from the database 
as a whole: records/atoms, types, 
projections, subgraphs, software 
components implementing a digamma-application 
interface (analogous to HGApplication in 
HyperGraphDB), among others.  These entities can 
be given identifiers which are unique not only in the context of a single database, but if needed over a larger corpus (e.g. an archive of research data) so that the relevant atoms/types/projections can be referenced from publication texts (and similar resources).  Such references can also be linked to \GUI{}s; one can for instance say that the data currently visible in a \GUI{} window or dialog box is a view onto a specific 
database atom, information that could be used for debugging, storing application state, inter-application networking (e.g. linking \PDF{} viewers for research papers with applications to view research data), 
and so forth.

\item[Memory and Persistence Models]  \lDgDb{} can be 
run in several different \q{modes}, which determine 
how data is stored in memory while an application 
is running and/or is stored in files.  If desired, 
\DgDb{} can be used as an \q{in-memory} database which 
does not maintain persistent file state at all.  
This feature can be useful in a research-data context 
where all information derives from data files.  
In this scenario, a \DgDb{} instance can be 
loaded from parsed serializations (e.g., \HGXF{} 
files) without using other file-system resources.  
The database engine can then be adopted as a 
tool for accessing and manipulating the \q{infoset} 
derived from these data files, analogous to 
\XML{} libraries interfacing with a Document Object Model.

\item[Scientific-Computing Interface Description]  Following 
HyperGraphDB, \DgDb{} establishes a \q{type system} 
unique to each database, which is responsible for 
translating application-level types to database atoms.  
The type system used by \DgDb{} is actually 
more detailed (compared to HyperGraphDB) and 
allows for the description of procedure signatures 
and the declaration of conceptually related 
procedural groupings, which collectively 
enable interface definition as persistent 
data within the database itself.  \lDgDb{} 
likewise supports a notion of \q{meta-procedures}, 
which are indirectly derived from Alexandru Telea's 
\textit{metaclasses} and \textit{dataflow interfaces} 
(see \cite{TeleaVISSION}).  Moreover, \DgDb{} 
introduces a notion of \q{channels}, which are a 
higher-level graph structuring feature 
(see \cite[\textnormal{Chapter 3}]{NeusteinCPS}) that may, 
in particular, be applied for the semantic 
annotation of function signatures so as 
to refine interface documentations.  
Collectively, these features enable 
each \DgDb{} database to store information 
about procedures used to access and 
interact with their stored information, 
to facilitate the implementation of 
scientific workflows and \GUI{}s for 
accessing and using \DgDb{} data.

\end{description}
}


\subsection{Programming with HGXF Files}

\p{The prior outline described some of the principle 
features of \DgDb{}, which in turn influenced 
the design of \HGXF{}, insofar as one goal of 
\HGXF{} is to serialize \DgDb{} instances.  However, 
\HGXF{} files are also generic resources for 
serializing research data (or any other technical 
information), and \DgDb{} includes libraries 
for using \HGXF{} (not necessarily in the context 
of \DgDb{} instances).  Therefore, \HGXF{} deserves 
a brief overview in its own right.}

\p{Like any hypergraph meta-model, \HGXF{} represents 
information on two levels.  On the one hand, 
each \q{node} is (in genral) a synthesis of multiple 
smaller parts.  In \HGXF{}, the parts of a hypernode 
are \textit{structure fields} and \textit{array fields}, 
where the former are roughly analogous to named columns, 
and the latter are usually either units within 
expandable collections (such as lists, queues, and dictionaries) 
or fixed-size arrays of numeric fields with similar 
dimensional qualities (e.g., vectors representing point in space).  
\lHGXF{} files define \q{types}, which are internal to 
the file (but may be associated with \DgDb{} types) and 
describe the layout of fields within the containing 
hypernode (in particular, the number and names of 
structure fields, and restrictions on the number of 
array fields, if any).  \lHGXF{} types may also 
provide hints about how to best encode the parsed 
data in running memory.}

\p{The structure of \HGXF{} files is specified by 
a Reference Implementation using \Cpp{} to 
parse and manipulate \HGXF{} data.  In general, 
this code library mixes functional and object-oriented 
programming styles.  The central procedures in 
this library are ones to isolate single hypernodes, 
by graph traversal or queries, and then procedures 
to operate on array and structure fields contained 
within hypernodes.  The basic interface for this 
latter context involves passing criteria for selecting 
one or more fields, along with a callback function 
which will be called with the vector of matching 
fields (or else a procedure to operate on fields 
one at a time).  In this functional style, \HGXF{} 
hypernodes can be seen as monads encapsulating 
access to their internal structure and array 
fields (or \q{hyponodes}).}

\p{On a higher level, applying not to the 
scope of a single hypernode but to groups of 
hypernodes, \HGXF{} introduces several 
higher-level structuring elements (mostly 
derived from corresponding \DgDb{} features). 
In particular, \textit{channels} are 
aggregates of edges, and \textit{frames} are 
contexts for defining edges and nodes.  
In \DgDb{}, every node and edge is constructed 
in the context of a frame, which \textit{may} 
(but need not) offer either additional 
semantic detailing or callback functionality 
to add code affecting how graph data is 
constructed.  All graphs have a \q{default} 
frame that implies no special semantic 
context (i.e., all edges are asserted relations 
deemed to apply in general, with no 
contextual filter) and that utilizes 
general-purpose algorithms for edge and 
node construction.  Developers can 
introduce more specialized frames as desired, 
and construct nodes and edges in these 
tailored contexts.  Such frames can then be 
identified in \HGXF{} as the context where a 
given node or edge serialization applies.  
Hypernodes may also be annotated within frames 
to identify a conceptual, operational, or semantic 
role of a node within some larger node-collection.  
Indeed, \lDgDb{} introduces the concept of a 
\textit{virtual frame} to capture logical patterns 
implicit in certain node-annotations which are not 
explicitly designated by frame objects allocated 
within the database.}

\p{The default behavior for edge-construction 
(which can be overridden within individual frames) 
conceives edges as \RDF{}-style \q{triples}, 
where a directed pair of nodes is annotated 
with a simple label (which may or may not 
be defined within a controlled vocabulary) and/or 
a more complex object; edges can be marked with 
hypernodes of their own.  The source and target 
hypernodes of a hyperedge correspond to the 
source and target node-sets defined in most 
mathematical treatments of hypergraph theory.  
In libraries such as HgLib, which (compared to 
graph/hypergraph databases) are more directly based 
on this mathematical theory, hyperedges are defined 
by providing one or two (for undirected or directed 
edges respectively) sets of nodes.  \lDgDb{} supports 
this construction as well, essentially forming 
hypernodes \q{on the fly}, but in this case the 
fields within these auto-constructed hypernodes are 
\q{proxies} for other hypernodes (possibly three or more) 
linked by the edge.  Proxies can also be 
defined for more complex objects, such as 
frames and subgraphs, allowing multi-dimensional, 
nested graph structures if these are desired for 
a particular domain model.} 

\p{\lHGXF{} also supports microcitations, which are 
discussed above.  By designating parts of 
a data set as microcitation targets, \HGXF{} can 
be integrated with annotation systems designed 
for scientific publications.  These features 
allow document annotations --- which are 
used, for instance, to connect text in a 
research paper with scientific concepts and 
Ontologies --- to connect also with microcitable 
elements within data sets and Research Objects 
which serialize data via \HGXF{}.  In this 
use-case, \HGXF{} works in conjunction with 
a related protocol, the Annotation Exchange Format 
(\AXF{}), which is also initially developed to 
support the \CRtwo{} implementation.  In \CRtwo{}, 
\AXF{} will be used to connect Covid-19 research 
data with publications where the corresponding 
data sets are introduced to the scientific 
community.  The \AXF{} format and design principles 
will be discussed further in the next section.}


\section{The AXF Platform}	
\p{The \AXF{} Platform (hereafter called \AXF{}) 
is a toolkit for hosting full-text, open access 
publications, with an emphasis on scientific, academic, 
and technical documents.  At the core of an \AXF{} 
Publication Repository is a collection of files 
in a machine-readable \AXF{} Document Format (\AXFD{}), 
which are paired with human-readable \PDF{} documents 
as well as supplemental multi-media and metadata files.  
Depending on institutional requirements, an \AXF{} repository 
may be the primary storage resource for the contained 
publications, or an adjunct resource whose documents are 
linked to publications hosted elsewhere.  In the second 
scenario, the primary goal of an \AXF{} repository is to 
host manuscripts in \AXFD{} format, along with software 
to aid viewing and text-mining of the associated publications.}

\p{\lAXFD{} therefore has two distinct purposes: (1) to 
aid in text and data mining (\TDM{}) of full 
publication text (along with research data that may be 
linked to publications), and (2) to enhance the reader 
experience, given e-Reader software (canonically, 
\PDF{} viewers) which are programmed to consume 
\AXF{} information.  To (1) aid in text mining, \AXFD{} documents 
can be compiled into different structured representations, 
yielding document versions that can be registered 
on services such as CrossREF \TDM{} and SemanticScholar.  Given 
a Document Object Identifier, text-mining tools can therefore 
readily obtain a highly structured, machine readable version 
of the publication, which may then be used as the basis 
for further text-mining and \NLP{} operations.  Simultaneously, 
to (2) improve reader experience, the \AXF{} platform generates 
numeric data linking semantically significant text locations 
to \PDF{} viewport coordinates 
(such text locations include annotation, quotation, or citation 
start/end points and paragraph or sentence boundaries --- 
collectively dubbed a Semantic Document Infoset, or \SDI{}).  
This \SDI{}+Viewport (\SDIV{}) information can then be used by 
\PDF{} applications to provide contexts 
for word searches, to localize context 
menus, to activate multi-media features at different points 
in the text, and in general to make \PDF{} files more 
interactive.  Data sets composed with the aid of \AXF{} 
tools may include source code for a \PDF{} viewer (an extension 
to \XPDF{}) capable of leveraging \AXF{} data.}

\p{In addition to the \AXFD{} document format, the \AXF{} platform 
includes the Annotation Exchange Format itself, a protocol for 
defining and sharing annotations on full-text publications.  
\lAXF{} differs from other annotation-representation 
strategies by (1) providing more detail concerning 
the location of annotated text segments, in the 
surrounding publication context, and (2) supplying 
annotation data in multimedia or microcitation 
formats which extend beyond conventional 
\q{controlled vocabularies}.  In terms of (1) 
publication context --- that is, the annotation 
\textit{target} (using terminology from the 
Linguistic Annotation Framework, or \LAF{}) --- \AXF{} 
represents \SDIV{} information as introduced 
above; this data supplements the node/index coordinates 
used by traditional annotation mechanisms.  
With respect to (2) annotation metadata --- 
or the annotation \textit{body} (again using 
\LAF{} terminology) 
--- \AXF{} introduces models for multimedia 
assets, software components, and data set 
content, which may be linked to annotation 
targets.  With this additional metadata, 
annotations may be used in application-development 
environments, not only for text mining.}

%Linguistic Technology Systems has developed \AXF{} to 
%address shortcomings of existing annotation standards, particularly 
%in the context of annotation environments integrating application, 
%semantic, and multimedia content.  The goal of \AXF{} is to represent 
%annotations at a novel level of detail.  In particular 
%(as will be described below), the \textit{target} of the annotation 
%should be represented in both semantic and interactive contexts 
%(to include, for instance, both sentence-level context and 
%\PDF{} viewport coordinates).  Meanwhile, the \textit{body} 
%of the annotation (using terms originating with the 
%Linguistic Annotation Framework) can be linked to data 
%\q{microcitations}, multimedia assets, or 
%controlled vocabularies.  \lAXF{} is optimized for 
%\AXFD{} manuscripts --- i.e., documents encoded according to 
%the native \AXF{} protocol --- but \AXF{} can potentially 
%be used as an exchange format for document-annotation 
%operations in a variety of contexts (named entity recognizers, 
%science-data \API{}s, publication metadata sharing, and so forth.}


\p{The following sections will (1) outline \AXF{} and \AXFD{} 
in greater detail, (2) describe how \AXF{} repositories 
can unify publications sharing similar themes, scholarly 
disciplines, or coding requirements, and (3) 
describe features for data-set publication in 
the context of \AXF{} repositories.}


\subsection{AXF Documents}

\p{The \AXFD{} format for describing document content and 
structure is designed to be a \q{Pivot Representation} in the 
sense of \LAF{} (see \cite{IdeSuderman}).  In particular, \AXFD{} can represent 
the structure of both \XML{} (including several 
\XML{} flavors used in publishing) and \LaTeX{}.  
Technically, \AXFD{} does not prescribe any 
specific input format; instead, a document is 
considered an instance of \AXFD{} if it can be 
compiled into a Document Object satisfying 
interface requirements.  A \Cpp{} reference implementation 
anchors the \AXF{} Document Object Model; nodes in 
this implementation have facets combining 
\LAF{}, \XML{}, and \LaTeX{}.  In practice, 
\AXFD{} manuscripts are then converted via \LaTeX{} to 
\PDF{}, and simultaneously compiled to \XML{} representations 
so as to generate machine-readable, structured full-text 
versions of the manuscripts.  Authors can choose to 
compose \AXFD{} papers to conform with several common 
publication \XML{} standards, such as \JATS{} 
(Journal Article Tag Suite), \SciXML{} \cite{CJRupp}, and 
\IeXML{} \cite{DietrichRebholzSchuhman} 
(the latter is an annotation-oriented 
\XML{} language used by the BeCAS project 
\cite{TiagoNunes}).  Authors or editors 
may further define or model scientific 
concepts via strategies focused on 
computer simulation (in the broad sense 
as defined, for instance, in \cite{EricWinsberg}) 
or statistics (e.g., Predictive Model Markup Language 
or Attribute-Relation File Format), 
and/or conceptual semantics, such as Conceptual Space Markup 
Language (which was inspired by the linguist 
Peter G\"ardenfors, but developed in a 
scientific/mathematical framework in e.g.
\cite{RaubalAdams}, \cite{RaubalAdamsCSML},
\cite{Zenker}, and \cite{InteractingConceptualSpaces}).}

\p{One distinct feature of \AXF{} is that \LaTeX{} and 
\XML{} generation are chained in a pipeline: the \LaTeX{} 
and subsequent \PDF{} generation steps yield auxiliary data, 
which includes \PDF{} viewport data, that can be 
subsequently incorporated into \XML{} views onto 
the documents.  Specifically, 
\AXFD{}-generated \LaTeX{} files include notations for 
semantic annotations and for sentence boundaries, implemented 
via \LaTeX{} commands which, as one processing step, 
write \PDF{} coordinates to auxiliary files.  The resulting 
data is then read by a \Cpp{} program which collates 
annotations and sentence-boundaries into a vector of 
data structures indexed by \PDF{} page numbers, creating a 
distinct file for each page, and zips those files into an 
archive which can be distributed alongside (or embedded inside) 
the \PDF{} publications.  Simultaneously, sentences, 
paragraphs, annotations, and other semantically significant 
content (such as quotations and citations) are assigned 
unique ids and compiled into their own data structures 
(from which machine-readable \XML{} full-text may be generated).  
These \XML{} files may then be hosted and/or registered on 
\TDM{}-oriented services such as CrossRef.  At the same 
time, unique identifiers unify this \XML{} data 
(focused on text mining) with \PDF{} viewport data 
(focused on reader experience).  The goal of such 
integration is to incorporate text-mining results 
so as to enhance reader experience.  For example, 
Named Entity Recognizers might flag a word-sequence 
as matching a concept within a controlled vocabulary.  
Via the relevant paper's \SDI{} model, this annotation 
may be placed in a proper semantic context --- for 
example, obtaining the text of the sentence where the 
Named Entity occurs.  This semantic information may then 
be used by a \PDF{} viewer --- e.g., providing a 
context menu option to select the sentence text, when 
the context menu is activated within the rectangular 
coordinates of the annotation itself.}

%
\begin{figure}

\caption{Data Microcitations via Tabular Columns}
\label{fig:xl}

\begin{tikzpicture}

\node[inner sep=0pt] (x1) at (0,0)
    {\includegraphics[width=120mm, 
    	trim={0mm 0mm 0mm 0mm},clip]
    	{pics/xl.png}};
    
\end{tikzpicture}   
\end{figure}



\begin{figure}

\caption{Linking PDF Files with Scientific Applications}
\label{fig:il}

\begin{tikzpicture}

\node[inner sep=0pt] (x1) at (0,0)
    {\includegraphics[width=90mm, 
    	trim={0mm 0mm 0mm 0mm},clip]
    	{pics/xl.png}};

\node[inner sep=0pt] (i1) at (10,0)
    {\includegraphics[width=90mm, 
    	trim={0mm 0mm 0mm 0mm},clip]
    	{pics/il.png}};

\curicon{1.9}{1}
    
\end{tikzpicture}   
\end{figure}


\p{As a representation of annotation data structures, 
\AXF{} ensures that \SDI{} and viewport data is included 
among annotations wherever this data is available.  
This facilitates the integration between text-mining 
tools and \PDF{} viewer software, which in turn 
enhances reader experience.  As mentioned earlier, every 
annotation can be placed in a semantic context (e.g., 
the text of the surrounding sentence), which provides 
useful reader features such as one-click copying of 
sentences to the clipboard.  Other reader-experience 
enhancements involve multimedia assets.  As a concrete example, 
suppose a paper includes mention of a chemical; that particular 
keyword can accordingly be flagged for annotation.  
As one encoding of the corresponding scientific concept, 
the annotation can include the chemical's Chemical Abstract 
Service Reference Number, via which it is possible to obtain 
Protein Data Bank (\PDB{}) files to view the relevant 
molecular structure in \ThreeD{}.  In sum, 
annotations supply a constellation of data 
--- in this example, concepts may be linked not only to 
identifiers in cheminformatic ontologies, but also to \CAS{} 
reference numbers and thereby to \ThreeD{} graphics files 
--- which facilitate interactive User Experience at the 
application level, not only document classification at the 
corpus level.  Once a chemical compound (mentioned in a 
publication) is linked to a \PDB{} file (or any other 
\ThreeD{} format) the \PDF{} viewer may include 
options to for the reader to connect to software or 
web applications where the corresponding visuals can 
be rendered.  Via \AXF{}, the relevant document-to-software 
connections are asserted not only on the overall document 
level, but on the granular scale of the precise character 
and \PDF{} viewport coordinates where the relevant 
annotation is grounded (Figure~\bref{fig:il} illustrates 
such capabilities in the context of a chemistry publication 
--- specifically, test-preparation materials for the 
Chemistry \GRE{} exam).}

\p{To support this kind of multimedia functionality, 
\AXF{} standardizes a Plugin Framework, dubbed \q{\Mosaic{}}, 
allowing programmers to embed code which can parse and 
respond to \AXF{} annotations in different scientific and 
document-viewer applications.  \lMosaic{} allows different 
applications to inter-operate; in particular, 
\PDF{} viewers can share data with scientific applications 
that can render files in domain-specific formats such as 
\PDB{}.  This application networking protocol is considered 
part of the \AXF{} annotation model, because application-oriented 
information is computationally relevant for many concepts 
encountered in scientific and technical environments.  For instance, 
one aspect of cheminformatic data is that many chemical 
compounds are modeled by \PDB{}, \MOL{}, or \ChemXML{} files, 
which in turn are associated with software applications that 
can load those file types.  Such inter-application 
networking data is relevant to \PDF{} viewers 
when they display manuscripts with annotations that suggest 
links to special file types and their applications; the 
viewers can employ this information to launch and/or communicate 
with the corresponding software.  \lAXF{} is designed to 
facilitate implementation of application-networking protocols 
as an operational continuation of processes related to obtaining 
and consuming annotation data.}

\p{The \AXF{} document model, at the manuscript-structure 
level, is paired with a novel \q{Hypergraph Text Encoding 
Protocol} (\HTXN{}) operating at the character-encoding level.  
Within the \HTXN{} protocol, an annotation target is 
a character-index interval in the context of an 
\HTXN{} character stream.  On that basis, 
\HTXN{} treats documents as graphs whose nodes 
are ranges in a character stream, where text can 
be recovered as an operation on one or more nodes 
(e.g., the text of a sentence is derived from a 
pair of nodes representing the sentence's start 
and end).  \lHTXN{} code-points 
are distinguished in terms of their semantic 
role, which may be more granular than their 
visible appearance --- for example, 
a period glyph is assigned different code-points 
depending on whether it marks a sentence-ending 
punctuation, an abbreviation, a decimal point, 
or part of an ellipsis.  Procedures are then implemented to 
represent text in different formats, such as 
\ASCII{}, Unicode, \XML{}, or \LaTeX{}.  In 
contrast to a format such as Web Annotations, 
any particular human-readable text presentation 
(including \ASCII{}) is considered a \textit{derived} 
property of the annotation, not a foundational 
representation.}

\p{\lAXFD{} manuscripts do not need to utilize \HTXN{} 
for character data, but \HTXN{} simplifies certain \AXF{} 
operations, such as identifying sentence boundaries.  
In particular, \HTXN{} provides distinct code-points for 
end-of-sentence punctuation, so that sentence-boundary 
detection reduces to a trivial search for those particular 
code-points.  Proper \HTXN{} encoding requires that authors 
follow certain simple heuristics --- e.g., that end-of-sentence 
periods should be followed by two spaces and/or a newline, 
whereas other uses of a period character should precede at 
most one space.  Aside from the goal of preparing documents 
for text-mining machine-readability, such conventions are 
appropriate even for basic typesetting, because non-punctuation 
characters have their own kerning rules (this is why 
\LaTeX{} provides a distinct command for non-punctuation glyphs 
that would otherwise be read as punctuation characters).  \lHTXN{} 
hides these typesetting details within its character-encoding 
schema, which is then useful both for producing professional-caliber 
\LaTeX{} output and for identifying \SDI{} details (such as 
sentence boundaries) which with less rigorously structured 
text would need elaborate text-mining or \NLP{} algorithms.}

\p{Each \AXFD{} document is, in sum, associated with an 
aggregate of character-encoding, annotation, document-structure, 
and \PDF{} viewport information.  The \AXF{} platform uses 
code libraries to pull this information together as a 
runtime object system, so that any application which loads 
an \AXFD{} manuscript can execute queries against the 
corresponding collection of \AXF{} objects (queries such 
as obtaining the sentence text around an annotation, obtaining 
the concave-octagonal viewport coordinates for a sentence,\footnote{
In the general case, sentence coordinates are concave octagons because 
they incorporate the line height of their start and end lines; 
in the general case sentences share start and end lines with other 
sentences, while also including whole lines vertically positioned 
between these extrema.  A sentence octagon roughly corresponds with 
the screen area where a mouse/pointer action should be understood as 
occurring in the context of that sentence from the user's point 
of view --- implying that the user would benefit from context menu 
options pertaining specifically to that sentence, such as 
copy-to-clipboard.} obtaining application-networking 
information for an annotation, 
etc.)  In addition to such runtime data, \AXF{} platforms can 
compile the full suite of information into machine-readable 
files for text and data mining.  These files, collected across a 
corpus of multiple documents, then form the backbone of an 
\AXF{} publication repository, as will be discussed next.}

\subsection{AXF Publication Repositories}
\p{The \AXF{} platform is designed for hosting collections 
of publications sharing a common academic or technical 
focus.  When \AXF{} is used in the context of a 
general-purpose text and/or data repository, 
the \AXF{} platform is designed to work with collections that 
are organized into separate projects or topics, 
each giving rise to an archive or corpus of publications.  
Insofar as these corpora internally share a common theme 
or focus, they can be associated with their own 
ontologies, code libraries, annotation models, and 
application-networking protocols, based on the sorts 
of applications and data structures commonly used 
in the corresponding scholarly discipline.  
In some cases, publishers may choose to package 
an entire archive of research papers (perhaps along with 
research data) as a single downloadable resource.  
\lAXF{} allows publishers to construct such Research Archives 
following the structure of existing examples such as 
the \ACL{} (Association for Computational Linguistics) 
Anthology or the recent \Cnineteen{} corpus.   This 
latter archive is a useful case-study in both the 
possibilities and limitations of existing 
publication-repository technology, 
so it is reviewed here in more detail.}

\p{\lCnineteen{}, curated by the Allen Institute for 
Artificial Intelligence, was spearheaded by 
a White House initiative to centralize scientific 
research related to \Covid{} (see \cite{CORD}).  The  
collection was formulated with the explicit goal 
of promoting both \textit{text mining} and 
\textit{data mining} solutions 
to advance coronavirus research, so that 
\Cnineteen{} is intended to be used both as a document 
archive for text mining and as a repository for 
finding and obtaining coronavirus data for subsequent 
research.  Although novel research is being 
incorporated into \Cnineteen{}, many of the articles 
reproduced in this corpus are older publications related 
to coronaviruses and to SARS in general, not just to the 
current pandemic.  As a result, the full-text versions 
of these publications were retroactively aggregated 
into a single archive due to the unanticipated emergence 
of a coronavirus crisis, with the full text often obtained 
from \PDF{} files rather than from structured 
representations (such as \JATS{}) explicitly intended 
for text mining.}

\p{This archival methodology results in \Cnineteen{} being 
limited as a \TDM{} framework.  These limitations include 
the following:

\begin{description}
\item[Transcription Errors]  
Transcription errors can easily result from trying to 
read scientific data and notations based on 
\PDF{} files --- or on full-text representations using 
relatively unstructured formats such as \XOCS{} 
(the response-encoding format for the ScienceDirect 
\API{}).  Transcription errors cause the machine-readable 
text archive to misrepresent the structure 
and content of documents.  For instance, 
there are cases in \Cnineteen{} 
of scientific notation and terminology 
being improperly encoded.  As a concrete example, \colorq{2{\textquotesingle}-C-ethynyl} is encoded incorrectly in one \Cnineteen{} file 
as \makebox{\colorq{2 0 -C-ethynyl}} (see \cite{Eyer} for 
the human-readable publication where this error is 
observed; the corresponding index in the corpus is \textcolor{blGreen!45!black}{9555f44156bc5f2c6ac191dda2fb651501a7bd7b.json}).  
To help address these sorts of errors --- 
which could stymie text searches 
against the \Cnineteen{} corpus --- 
it is obviously preferable to archive structured, 
machine-readable versions of publications, using 
a platform such as \AXF{}.  
    
\item[Converting Between Data Formats]
Although the \Cnineteen{} corpus is published 
as \JSON{} files, many text-mining tools such 
as those reviewed in \cite{NeusteinText} recognize 
inputs or produce outputs in alternative formats, 
such as \XML{}, \BioC{}, \CoNLL{} (Conference on Natural 
Language Learning), or \JSON{} trees with 
different schema than \Cnineteen{}.  For this 
reason, rather than providing data with one single 
representational format, it is better to 
encode the data along with code libraries that can 
express the data in different formats as needed 
for different \TDM{} ecosystems.

\item[Inconsistent Annotations]  
The structure of \Cnineteen{} allows text segments to 
be defined via a combination of \JSON{} file names, 
paragraph ids, and character indices.  This indexing 
schema is used for representing certain internal 
details of individual articles, such as citations, 
but is not explicitly defined as an annotation 
target structure for standoff annotations against 
the archive as a whole.  This problem could 
also be rectified with code libraries that map 
index targets to file handles and character pointers. 

\item[Limited Support for Research Data-Mining]  Even though 
many papers in \Cnineteen{} are paired with 
published data sets, there is currently no tool for 
locating  research \textit{data} 
through \Cnineteen{}. 
For example, the collection of manuscripts available 
through the Springer Nature portal linked 
from \Cnineteen{} includes over 30 \Covid{} data sets,
but researchers can only discover that these data 
sets exist by looking for a \q{supplemental materials} or 
a \q{data availability} addendum near the end of each article.
These Springer Nature data sets encompass a wide array of file types 
and formats, including \FASTA{} (which stands for Fast-All, 
a genomics format), \SRA{} (Sequence Read Archive, for 
\DNA{} sequencing), \PDB{} (Protein Data Bank,  
representing the \ThreeD{} geometry of protein 
molecules), \MAP{} (Electron Microscopy Map), \EPS{} 
(Embedded Postscript), \CSV{} (comma-separated values), 
and tables represented in Microsoft Word 
and Excel formats.  To make this data more 
readily accessible in the context of \Cnineteen{}, it 
would be appropriate to (1) maintain an index of 
data sets linked to \Cnineteen{} articles 
and (2) merge these resources into a common representation 
(such as \XML{}) wherever possible.  This research-data 
curation can then be treated as a supplement to 
text-mining operations.  In particular, queries 
against the full-text publications could be evaluated 
\textit{also} as queries against the relevant set 
collection of research data sets.    

\item[Wrappers for Network Requests]  Scientific 
use of \Cnineteen{} will often require communicating 
with remote servers.  For example, genomics 
information in the \Covid{} data sets (such as 
those mentioned above that are available through 
Springer Nature) is generally 
provided in the form of accession numbers which 
are used to query online genomics services.  
Similarly, text mining algorithms often 
rely on dedicated servers to perform 
Natural Language Processing; these services 
might take requests in \BioC{} format and respond 
with \CoNLL{} data.  As another case study epidemiological 
studies of \Covid{} may need to access \API{}s or data 
sets such as the John Hopkins University \q{dashboard} 
(see \href{https://coronavirus.jhu.edu/map.html}{https://coronavirus.jhu.edu/map.html}, which is paired with a \GIT{} archive 
updated almost daily).  To reduce the amount 
of \q{biolerplate code} which developers need 
for these networking requirements, an archive's 
text-mining code could provide a unified framework with which 
to construct web-\API{} queries, 
one that could be used across 
disparate scientific disciplines 
(genomics, \NLP{}, epidemiology, and so forth). 
\end{description}}

\p{Many of these limitations observed in \Cnineteen{} 
reflect the fact that this corpus was prepared 
as raw (text) data, without any supporting code.  
By contrast, recent initiatives --- such as the Research Object 
protocol (see \cite{KhalidBelhajjame}) 
and \FAIR{} (\q{Findable, Accessible, 
Interoperable, Reusable}; see \cite{TrifanOliveira}) --- 
encourage authors to publish code and data together, 
so that the computing environment needed to process 
published data is provided within the data set itself.  
The \Cnineteen{} limitations accordingly provide an 
example of why Research Objects, rather than raw data 
sets, should be preferred for data publication in the 
future.  The Research Object model is usually defined in 
the context of a single publication, but the paradigm 
applies equally well to corpora encompassing many 
single articles.  That is, \AXF{} is structured so 
that Research Archives can be designed as higher-scale 
Research Objests, wherein the document collection is bundled 
with supporting code and an overall computing and 
software-development environment.  Such archive-specific 
\SDK{}s would include \AXF{}-specific code as well 
as libraries or applications often utilized in 
the academic disciplines relevant to the archival 
subject areas.  The \AXF{} platform especially 
promotes the design of domain-specific \SDK{} 
which are \textit{standalone} 
and \textit{self-contained}, with minimal external 
dependencies.  As much as possible, users should 
not have to install external software to utilize 
data provided along with an \AXF{} repository; 
instead, the needed data-management tools should be 
provided in source-code form within the archive itself.}

\p{Each \AXF{} repository, then, should 
bundle numerous applications used for database 
storage, data visualization, and scripting.  
The goal of this application package would be to 
provide researchers with a self-contained computing 
platform optimized for scientific research 
and findings related to the archived publications.  
Archival \SDK{}s should try 
to eliminate almost all scenarios where 
programmers would need to perform a \q{system 
install}; for the most part, the entire 
computing platform (including scripting 
and database capabilities) should be compiled 
from source \q{out-of-the-box}.  While the 
actual libraries and applications bundled with 
an archive would depend on its topical 
focus, the following is an example of components 
that would be appropriate in many different \SDK{}: 

\begin{itemize}[itemsep=-1pt]

\item \XPDF{}: A \PDF{} viewer for reading full-text articles 
(augmented with \Cnineteen{} features, such as integration 
with biomedical ontologies);

\item \Qt{}: The \Qt{} library is a cross-platform 
Application-Development framework and 
\GUI{} toolkit commonly used for scientific applications 
(\XPDF{} is one example of a \Qt{}-based document 
viewer).  Almost any data set can be accompanied 
with \Qt{} code for data visualization, so that readers 
would not have to install additional software.  For its 
part, \Qt{} can be freely obtained and, once 
downloaded, resides wholly in its own folder 
(there is no install step which modifies the 
user's system); as such, \Qt{} along with individual 
archive \SDK{}s function as standalone packages, 
although optimally the \SDK{}s would be updated along with 
new \Qt{} versions.  
  
\item AngelScript: An embeddable scripting engine 
that could be used for analytic processing 
of data generated by text and data mining operations 
on \Cnineteen{} (see \cite{AS});

\item WhiteDB: A persistent database 
engine that supports both relational 
and \NoSQL{}-style architectures 
(see above);

\item MeshLab: A general-purpose \ThreeD{} graphics 
viewer;

\item LaTeXML: a \LaTeX{}-to-\XML{} converter;

\item PositLib: a library for use in high-precision computations based on the \q{Universal Number} format, 
which is more accurate than traditional floating-point 
encoding in some scientific contexts 
(see \cite{JohnGustafson}). 
\end{itemize}

To this list one might add components specific to various 
scientific fields: \IQmol{} for chemistry and molecular 
biology, for example, or open-source libraries such 
as EpiFire or Simpact (for Epidemiology), \UDpipe{} (for 
\CoNLL{}), and so forth.  Here again the priority 
would be for self-contained components with few 
external dependencies --- particularly libraries 
programmed in \C{} or \Cpp{}, which are the 
languages best positioned to be a common denominator 
across diverse research projects (of course, many 
scientific \Cpp{} libraries have wrappers for 
languages like \R{} or Python that researchers may be 
more comfortable using).  
In general, Research Archive code should be 
(1) \textit{self-contained} (with few or no external 
dependencies, as emphasized above); 
(2) \textit{transparent} (meaning that 
all computing operations should be implemented by 
source code within the bundle that can be examined 
as code files and within a debugging session); 
and (3) interactive (meaning that the bundle does not 
only include raw data but also software to interactively 
view and manipulate this data).  Research Archives which 
embrace these priorities attempt to provide data visualization, 
persistence, and analysis through \GUI{}, database, and 
scripting engines that can be embedded as source 
code in the archive itself. }
  
\p{It is worth noting that a data-mining platform requires 
\textit{machine-readable} open-access research data
(which is a more stringent requirement than simply 
pairing publications with data that can only 
be understood by domain-specific 
software).  For example, radiological imaging can be a source 
of \Covid{} data insofar as patterns of lung 
scarring, such as \q{ground-glass opacity,} are a leading 
indicator of the disease.  Consequently, diagnostic 
images of \Covid{} patients are a relevant kind of 
content for inclusion in a \Covid{} data set 
(see \cite{Shi} as a case-study).  However, 
diagnostic images are not in themselves 
\q{machine readable.}  When medical imaging is 
used in a quantitative context (e.g., applying 
Machine Learning for diagnostic pathology), it is necessary 
to perform Image Analysis to convert the raw data 
--- in this case, radiological graphics --- into 
quantitative aggregates.  For instance, by using image 
segmentation to demarcate geometric boundaries one 
is able to define diagnostically relevant features (such 
as opacity) represented as a scalar field over the segments.  
In short, even after research data is openly published, 
it may be necessary to perform 
additional analysis on the data for it to be 
a full-fledged component of a 
machine-readable information space.\footnote{%
\raisebox{-10pt}{\hspace{3pt}\parbox{.9\textwidth}{This does not mean that diagnostic images (or 
other graphical data) should not be placed in a 
data set; only that computational reuse of such 
data will usually involve certain numeric 
processing, such as image segmentation.  
Insofar as this subsequent analysis is performed, 
the resulting data should wherever possible 
be added to the underlying image data as a 
supplement to the data set.}}}  To 
deal with this sort of situation, \AXF{} equips 
\SDK{}s with a \textit{procedural 
data-modeling vocabulary} that would both identify the 
interrelationships between data representations 
and define the workflows needed to 
convert research data into machine-readable data sets.} 

\p{Another concern in developing an integrated Research Arcive
data collection is that of indexing documents and research findings  
for both text mining \textit{and} data mining.  
In particular, \AXF{} introduces a 
system of \textit{microcitations} that apply 
to portions of manuscripts \textit{as well as} data sets.  
In the publishing context, a microcitation is defined as a 
reference to a partially isolated fragment of a larger 
document, such as a table or figure illustration, or a 
sentence or paragraph defining a technical term, 
or (in mathematics) the statement/proof of a definition, axiom, 
or theorem.  In data publishing, \q{data citations} are 
unique references to data sets in their entirety or to 
their smaller parts.  A data microcitation is then a 
fine-grained reference into a data set.  For example, 
a data microcitation can consist of one 
column in a spreadsheet, 
one statistical parameter in a quantitative analysis, 
or \q{the precise data records actually used in a study} 
(in the words adopted by the Federation of Earth Science Information Partners to define microcitations; 
see \cite{ESIP}).  As a concrete example, 
a concept such as \q{expiratory flow} appears in \Cnineteen{} 
both as a table column in research data and as a medical concept 
discussed in research papers; a unified microcitation framework 
should therefore map \textit{\color{drp}{expiratory flow}} as a keyphrase 
to both textual locations and data set parameters.  
Similarly, a concept such as 
\textit{\color{drp}{2{\textquotesingle}-C-ethynyl}} (mentioned earlier, in the context of transcription errors) 
should be identified both as a phrase in 
article texts and as a molecular component 
present within compounds whose scientific 
properties are investigated through \Cnineteen{} 
research data.  In so doing, a search for this 
concept would then trigger both publication and 
data-set matches at the same time.}

\p{Further discussion on data microcitations depends on 
how data sets are structured, which is addressed in the next section.}


\subsection{Research Objects and Data Microcitations}
\p{The design of \AXF{} assumes that many Research Archives, 
comprising of multiple publications sharing an academic 
focus, will also include open-access research data.  
The primary motivation for publishing 
research data is to ensure transparency and 
reusability --- open-access data allows readers to verify 
that scientific/technical claims are warranted, 
and/or to reuse or incorporate existing data into 
new research.  Open-access data has other purposes 
as well: data sets, for example, can serve 
as pedagogic tools helping readers understand 
publications' concepts experimentally and 
interactively (a good example is \cite{AlexandruTelea}, 
which pairs data sets with its individual chapters to 
illustrate principles in data visualization).  
Moreover, the theory informing how data sets are 
organized can serve as a technical exposition 
of research principles and methodology.  For all 
of these reasons, open-access data is an 
increasingly important part of the publishing 
ecosystem.  This means 
that well-curated archives will often need to 
prepare data sets for data 
mining, alongside the preparation of text materials for 
text mining.  }

   \begin{frame}{\ft{Oxy}}

        \begin{annotatedFigure}{0pt}{0pt}
            {\includegraphics[scale=1]{texs/oxy.png}}
            
  \node [text width=7.6cm,align=justify,fill=logoCyan!20, draw=logoBlue, 
  draw opacity=0.5,line width=1mm, fill opacity=0.9]
   at (0.74,0.465){\textbf{Dataset Applications make extensive 
   use of context menus to organize functionality and provide 
   advanced interactivity.  In this screenshot a context menu 
   action (\circled{1}) has been selected which alters the 2D 
   display, visually emphasizing a restricted set 
   of data points (\circled{2}) and contracting all others (\circled{3}).}};

            \annotatedFigureBox{0.73,0.26}{0.96,0.328}{1}{0.89,0.326}%            
            \annotatedFigureBox{0.55,0.4}{0.55,0.44}{3}{0.55,0.44}            
            \annotatedFigureBox{0.598,0.45}{0.598,0.48}{2}{0.598,0.45}    
      %      \annotatedFigureBox{0.222,0.284}{0.3743,0.4934}{B}{0.3743,0.4934}%tr
      %      \annotatedFigureBox{0.555,0.784}{0.6815,0.874}{C}{0.555,0.784}%bl
      %      \annotatedFigureBox{0.557,0.322}{0.8985,0.5269}{D}{0.8985,0.5269}%tr
  
        \end{annotatedFigure}

   %     \caption{Expanded Sample (A)}
    %    \label{fig:teaser}

    \end{frame}


\p{In contrast to text mining, however, it is not feasible, 
in the general case, to 
assign one single format (like \AXFD{} or \JATS{}) for all 
data sets published within an archive.  Precisely how data 
sets can be annotated depends on the data models, programming 
languages, and analytic methodologies which they utilize.  
Because this variability prohibits a single data-annotation 
protocol from being required, \AXF{} adopts a strategy of 
defining a rigorous protocol for \Cpp{} code bases, which 
can then be emulated by other languages.}

\p{As outlined above, data citations refer to parts within a 
data set --- such as individual data records, but also larger-scale 
aggregates such as table columns or statistical parameters.  
The complication when defining data citations is that a 
concept such as a table column, although it may have an 
obvious technical status as a discrete conceptual unit from 
the point of view of scientists curating, studying or reusing 
a data set, does not necessarily correspond to a single 
coding entity that could be isolated as an annotation 
target.  It is therefore the responsibility of 
\textit{code base annotations} to provide annotations for 
computational units --- such as data types, procedures, and 
\GUI{} components --- that have an annotatable \textit{conceptual} 
status relative to the data set on which the code operates.  
Often this will involve mapping one concept to 
several computational units (for instance, several 
procedure implementations).}

\p{For a concrete example of these points concerning 
data citations, consider the data set pictured 
in Figure~\bref{fig:oxy}, representing cyber-physical measurements 
used to calculate oxygenated airflow.  The data-set 
application (interactive-visualization code deployed within 
the Research Object) displays tabular data via a tree widget 
(which functions as a generalized, multi-scale spreadsheet 
table), with tabular columns expressing quantities --- such 
as air flow and oxygen levels --- in several formats (raw 
measures as well as sample rankings and min-max 
percentages).  Conceptually, these columns have distinct 
methodological roles and therefore can be microcited; 
indeed, the application links the columns to article 
text where the corresponding concepts are presented 
(see Figure~\bref{fig:about}).  However, the implementation does not introduce 
a distinct \Cpp{} object uniquely designating individual 
columns.  Instead, the individual columns can be annotated 
in terms of \Cpp{} methods providing column-specific 
functionality.  In the current example, these methods 
primarily take the form of features linked to context-menu 
actions (copying column data to the clipboard, sorting data 
by one column, etc.).  In general, rather than a rigid 
protocol for data-set annotations, \AXF{} proposes 
heuristic guidelines for how best to map programming 
constructs to scientifically salient data-set concepts.}

\begin{frame}{\ft{About}}

	\pdfpageheight 30cm

        \begin{annotatedFigure}
            {\includegraphics[scale=1]{about.png}}
            
  \node [text width=12cm,align=justify,fill=logoCyan!20, draw=logoBlue, 
  draw opacity=0.5,line width=1mm, fill opacity=0.9]
   at (0.58,0.76){\textbf{Context menus also allow users to 
   obtain information and explanations about individual parts of the 
   data set, such as individual statistical parameters.  In this 
   screenshot, the user has right-clicked on a data column and 
   chosen a context menu action which shows, via a dialog box, 
   a precis of the quantities represened in that column and their 
   significance for the data set as a whole.}};

            \annotatedFigureBox{0.2,0.12}{0.812,0.645}{1}{0.81,0.645}%            
      %      \annotatedFigureBox{0.222,0.284}{0.3743,0.4934}{B}{0.3743,0.4934}%tr
      %      \annotatedFigureBox{0.555,0.784}{0.6815,0.874}{C}{0.555,0.784}%bl
      %      \annotatedFigureBox{0.557,0.322}{0.8985,0.5269}{D}{0.8985,0.5269}%tr
  
        \end{annotatedFigure}

\end{frame}

\p{Defining an annotation schema for data sets can potentially 
be an organic outgrowth of software-development methodology 
--- viz., the engineering steps, 
such as implementing unit tests, which are essential 
to deploying a commercial-grade application.  
This point is illustrated in 
Figure~\bref{fig:testing}, which shows a \GUI{}-based testing environment for 
the data set depicted in Figures~\bref{fig:oxy} and 
\bref{fig:about}.  For this data set, 
the context menu actions providing column-specific functionality 
are also discrete capabilities which can be covered by 
unit tests, so the set of procedures mapped to the citeable 
concept correspond with a set of unit-test requirements.  
In this data set, these procedures are also exposed to 
scripting engines via the \Qt{} meta-object system.  In general, 
there is often a structural correlation between 
scripting, unit testing, and microcitation, so that 
an applications' scripting and testing protocol can serve 
as the basis for annotation schema.  For data sets which use 
in-memory or persistent databases, evaluable queries against 
these databases provide an additional grounding for annotations.  
In general, data-annotation should be engineered on the 
basis of a dataset applications' scripting, testing, and/or 
query-evaluation code.  However, this is only a heuristic 
guideline, and \AXF{} does not presuppose any data-annotation 
scheme \textit{a priori}.}

\begin{figure}

\caption{Test Suites for Dataset Applications}
\label{fig:testing}

\begin{tikzpicture}

\node[inner sep=0pt] (x1) at (0,0)
    {\includegraphics[width=7in, 
    	trim={0mm 0mm 0mm 0mm},clip]
    	{pics/testing.png}};
    
\end{tikzpicture}   
\end{figure}


%a file in formats such as \CoNLL{} or \PMML{} 
%(Predictive Model Markup Language).}
%\p{Operationally, \AXF{} is modeled most 
%directly on the BeCAS \API{} \cite{TiagoNunes} 
%and the Linguistic Annotation Framework (\LAF{}).\footnote{%
%such as \PMML{}, \ARFF{} (Attribute-Relation File Format), 
%interface based on BeCAS.  Moreover, 

\section{Conclusion}
\p{This outline has focused on \AXF{}, 
\HGXF{}, and DigammaDB, which are complementary 
technologies designed in particular for
creating and managing scientific or technical 
data sets.  Although applicable for many 
use-cases, the initial motivation for 
implementing these technologies in their current 
form is to enable the curation of a Covid-19 
data repository in conjunction with the volume 
\textit{Cross-Disciplinary Data 
Integration Models for the Emerging Covid-19 Data 
Ecosystem}}.

\p{\lAXF{} uses a two-tier node structure similar to 
\LAF{}; at one level is an extensible text-encoding 
methodology (outlined earlier in the context of 
\HTXN{}), while a higher level defines annotations in terms 
of directed hypergraphs.  Programmatically, 
\AXF{} aims in the canonical case for a level of 
detail that is intermediate between linked-data-oriented 
projects like Web Annotations (which tend to focus 
mostly on isolatable semantic resources such 
as citations and named entities) and \NLP{}-oriented 
paradigms such as \LAF{}.  That is, \AXF{} does not 
natively serialize fine-grained \NLP{} data at the 
level of individual words (the kind of data asserting 
semantic and morphosyntactic details: lemmatization, Part of Speech, 
dependency relations, and so forth), although it does support 
queries which return sentences as (unparsed) word-sequences.  
On the other hand, \AXF{} offers \text{some} granular information about 
small-scale linguistic units, such as the 
role of non-alphanumeric characters, or 
\PDF{} coordinates of sentence start and end points.  
In short, \AXF{} occupies a unique space in the 
landscape of annotation tools at the intersection 
of application-development, \NLP{}, and document-preparation 
requirements.}

\p{\lHGXF{} is first and foremost a tool for 
serializing \lDgDb{} data sets, but can 
also be employed as a generic system for 
describing research data.  Finally, \lDgDb{} is a 
\Cpp{} database engine which brings sophisticated 
hypergraph modeling capabilities to the \Cpp{} and 
\Qt{} ecosystem.  Unlike many high-level databases, 
\DgDb{} is mostly self-contained and \q{transparent}: 
it can be distributed in source-code fashion 
along with applications and/or research data sets.}

\p{Although a number of open-access Covid-19 data sets have 
already been published, the \lCRtwo{} repository can be a 
site for sharing new Covid-19 data as well as republishing 
existing data in the rigorous data-modeling framework 
established for \CRtwo{}.  Scientists can propose new 
data sets for inclusion in the corpus through an 
official GitHub repository.  Depending on the size and 
nature of these resources, the submitted information 
will then be translated in whole or in part into 
hypergraph data models and the \HGXF{} format; moreover, 
open-access publications, if applicable, may be 
annotated with \AXF{} to properly link data sets to 
their corresponding research papers.  \lCRtwo{} can 
then be used as a portal for hosting Covid-19 related 
manuscripts, alongside the collection of research 
objects tailored to the pandemic and its proper 
medical and governmental response.}

\vspace{-.75em}
%\noindent\lun{ETS\textsc{pf} for Scientific and Technical Applications}

\setlength{\bsep}{-2pt}
%\setlength{\parskip}{0pt}
%\setlength{\itemsep}{-2pt}

\makeatletter
    \clubpenalty10000
    \@clubpenalty \clubpenalty
    \widowpenalty10000
\makeatother

\begin{thebibliography}{99}
\vspace{1.5em}
{\fontsize{10}{11}\selectfont

\bibitem{RaubalAdams}
Benjamin Adams and Martin Raubal, 
\q{A Metric Conceptual Space Algebra}.
\biburl{https://pdfs.semanticscholar.org/521a/cbab9658df27acd9f40bba2b9445f75d681c.pdf}

\bibitem{RaubalAdamsCSML}
Benjamin Adams and Martin Raubal, 
\q{Conceptual Space Markup Language (CSML): Towards the Cognitive Semantic Web}.
\biburl{http://idwebhost-202-147.ethz.ch/Publications/RefConferences/ICSC_2009_AdamsRaubal_Camera-FINAL.pdf}

\bibitem{KhalidBelhajjame}{%
	Khalid Belhajjame, \textit{et. al.},
	\q{Workflow-centric research objects:  First class citizens in scholarly discourse}.	\biburl{https://pages.semanticscholar.org/coronavirus-research}}

\bibitem{EranBellin}{%
	Eran Bellin, 
	\q{Riddles in Accountable Healthcare}.	\biburl{https://streamlinehealth.net/wp-content/uploads/2016/02/Riddles-In-Accountable-Care_Defining-Your-Outcomes.pdf} (summary presentation)} 


\bibitem{InteractingConceptualSpaces}
Joe Bolt, \i{et. al.}, 
\cq{Interacting Conceptual Spaces I:
Grammatical Composition of Concepts}.
\biburl{https://arxiv.org/pdf/1703.08314.pdf}

\bibitem{CORD}{%
	\q{COVID-19 Open Research Dataset (CORD-19)}. 2020. Version 2020-03-13. Retrieved from https://pages.semanticscholar.org/coronavirus-research. Accessed 2020-03-20. doi:10.5281/zenodo.3715506
	\biburl{https://pages.semanticscholar.org/coronavirus-research}}

\bibitem{Eyer}{%
	Lud\"ek Eyer, \textit{et. al.},
	\q{Nucleoside analogs as a rich source of antiviral agents active against arthropod-borne flaviviruses}.
	\biburl{https://www.ncbi.nlm.nih.gov/pmc/articles/PMC5890575/}}

\bibitem{Erable}{%
	Project-Team ERABLE: \q{Activity Report 2018}
(European Research Team in Algorithms and
Biology, Formal and Experimental).
	\biburl{https://raweb.inria.fr/rapportsactivite/RA2018/erable/erable.pdf}}

\bibitem{Zenker}
Peter G\"ardenfors and Frank Zenker,  
\cq{Theory Change as Dimensional Change: Conceptual Spaces 
	Applied to the Dynamics of Empirical Theories}.
\intitle{Synthese 190(6)}, pp. 1039-1058, 2013.  
\biburl{http://lup.lub.lu.se/record/1775234}

\bibitem{JohnGustafson}{%
	John Gustafson,
	\q{Beating Floating Point at its Own Game: Posit Arithmetic}, 
	\biburl{http://www.johngustafson.net/pdfs/BeatingFloatingPoint.pdf}}

\bibitem{IdeSuderman}{%
	Nancy Ide and Keith Suderman,
	\q{GrAF: A Graph-based Format for Linguistic Annotations}.
	\biburl{https://www.cs.vassar.edu/~ide/papers/LAW.pdf}}

\bibitem{BorislavIordanov}{%
	Borislav Iordanov,
	\q{HyperGraphDB: A Generalized Graph Database}.
	\biburl{http://www.hypergraphdb.org/docs/hypergraphdb.pdf}}

\bibitem{AS}{%
	Andreas J\"onsson,
	\q{AngelCode Scripting Library}, 
	\biburl{www.AngelCode.com/AngelScript/}}

\bibitem{NeusteinCPS}{%
	Amy Neustein, \textit{ed.},
	\textit{Advances in Ubiquitous Computing: Cyber-Physical 
Systems, Smart Cities and Ecological Monitoring}, 
	\biburl{https://www.elsevier.com/books/advances-in-ubiquitous-computing/neustein/978-0-12-816801-1}}


\bibitem{NeusteinText}{%
	Amy Neustein, \textit{et. al.},
	\q{Application of Text Mining to Biomedical Knowledge Extraction: Analyzing Clinical Narratives and Medical Literature}, 
	\biburl{https://www.researchgate.net/publication/262372604_Application_of_Text_Mining_to_Biomedical_Knowledge_Extraction_Analyzing_Clinical_Narratives_and_Medical_Literature}}

\bibitem{TiagoNunes}{%
	Tiago Nunes, \textit{et. al.},
	\q{BeCAS: biomedical
concept recognition services and visualization}.
	\biburl{https://www.ncbi.nlm.nih.gov/pubmed/23736528}}

\bibitem{ESIP}{%
Mark A. Parsons and Ruth Duerr,
\q{Data Identifiers, Versioning, and Micro-citation}, 
\biburl{https://www.thelancet.com/action/showPdf?pii=S1473-3099\%2820\%2930086-4}}

\bibitem{EnarReilent}{%
	Enar Reilent,
	\q{Whiteboard Architecture for the Multi-agent Sensor Systems}, 
	\biburl{https://www.thelancet.com/action/showPdf?pii=S1473-3099\%2820\%2930086-4}}

\bibitem{Shi}{%
	Heshui Shi, \textit{et. al.},
	\q{Radiological findings from 81 patients with COVID-19 
		pneumonia in Wuhan, China: a descriptive study}.
	\biburl{https://www.thelancet.com/action/showPdf?pii=S1473-3099\%2820\%2930086-4}}

\bibitem{DietrichRebholzSchuhman}{%
	Dietrich Rebholz-Schuhman, \textit{et. al.},
	\q{IeXML: towards an annotation framework for biomedical semantic types enabling interoperability of text processing modules}.
	\biburl{https://www.semanticscholar.org/paper/IeXML\%3A-towards-an-annotation-framework-for-semantic-Rebholz-Schuhmann-Kirsch/1d72a56b6576117c62f388a5f2193965e4c7e293}}

\bibitem{CJRupp}{%
	C. J. Rupp, \textit{et. al.},
	\q{Flexible Interfaces in the Application of Language Technology to an eScience Corpus}.
	\biburl{https://www.cl.cam.ac.uk/~sht25/papers/Rupp_et_al.pdf}}

\bibitem{MattSelway}{%
	Matt Selway, 
	\q{Formal Models from Controlled Natural
Language via Cognitive Grammar and
Configuration}.
	\biburl{https://aise.unisa.edu.au/wp/wp-content/papercite-data/pdf/selwaythesis2016.pdf}}

\bibitem{AlexandruTelea}{%
	Alexandru Telea,
	\textit{Data Visualization --- Principles and Practice}.
	\biburl{https://www.semanticscholar.org/paper/Data-visualization-principles-and-practice-Telea/c853f4b8aee67cd749e03b3a4413769792222776}}

\bibitem{TeleaVISSION}{%
	Alexandru Telea and Jarke J. van Wijk,
	\q{VISSION: An Object Oriented Dataflow System
for Simulation and Visualization}.
	\biburl{https://www.rug.nl/research/portal/files/3178139/1999ProcVisSymTelea.pdf}}


\bibitem{TrifanOliveira}{%
	Alina Trifan and Jos\'e Lu\'\i{}s Oliveira,
	\q{FAIRness in Biomedical Data Discovery}.
	\biburl{https://www.researchgate.net/publication/331775411_FAIRness_in_Biomedical_Data_Discovery}}

\bibitem{EricWinsberg}{%
	Eric Winsberg, \textit{Science in the Age of Computer Simulation}.
	\biburl{https://www.press.uchicago.edu/ucp/books/book/chicago/S/bo9003670.html}}

}
\end{thebibliography}

\end{document}


