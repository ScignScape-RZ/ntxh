
\part{Outline of the AXFI Data Model}

\section{Types of Image Annotations}
\p{The most fundamental kinds of image annotations 
are geometric primitives such as points, lines, 
and polygons.  Many other forms of annotations 
are possible, however, mostly supplemental 
data which is associated with these primitives 
or, in some cases, with the image as a whole.  
In general, \AXFI{} classifies annotations into 
the following groups:

\begin{description}
\item[Geometric Primitives]  These annotations 
delineate spatial/geometric regions in zero, one, 
or two dimensions (or potentially higher dimensions 
when working in non-\TwoD{} contexts).  At a minimum, 
\AXFI{} data should support points, lines, 
polygons, polylines (considered a superkind of 
polygons where a polygon is a closed polyline), circles, 
and ellipses.  Additionally, \AXFI{} recognizes 
generic \q{closed} and \q{open} \textit{curves}, which 
are nonlinear one-dimensional regions (or boundaries 
of two-dimensional regions) that are neither 
elliptical or circular arcs.  Depending on 
context, \AXFI{} can be extended to provide 
more detailed subkinds of curves generated 
by different sorts of mathematical equations, 
along with notations for the formulae 
(e.g., b-splines) that generate a particular 
curve.\footnote{In formal statements, 
this and other \AXFI{} documentation will 
use the term \q{kind,} as well as \q{superkind} 
and \q{subkind,} to indicate groups of values/entities 
identified via a classification.  Vocabulary 
based on \q{kind} is preferred to \q{type} or 
\q{class} because of the distinct meanings these
latter terms have in computational contexts 
which are also discussed in reference to \sAXFI{}.}  
More information about encoding 
ellipse data as well as other sorts of 
curves is outlined below (section ).  

A variation on the polygon/polygon-line alternative 
are polygon-lines demarcating spatial regions 
whose boundaries extend to one or more sides/corners 
of the image (e.g., a \q{quadrant} is defined 
via one central point with line segments 
parallal to and implictly 
extended to the edges).  These poly-lines 
may not explicitly represent the overall image 
boundary as part of their own boundary.  
Conceptually, treating the image-border itself 
as a different \textit{sort} of boundary than 
those explicitly notated may be more accurate 
than eliding these distinctions.  This 
applies also to segment boundaries.  For instance, 
the sand/ocean border in a beach scene represents 
a physical discontinuity between two different 
material substances, and so it records an 
observable detail of the depicted scene.  
On the other hand, the edge of the sand-region 
at the bottom of the image is not a physical 
boundary, but an artifact of the camery position; 
we assume the beach itself extends further than 
what the camera captures.  All told, then, 
in addition to polygons (or curves or segment-boundaries) 
being \textit{open} or \textit{closed}, \AXFI{} 
recognizes a third form of closure dubbed 
\q{incomplete,} meaning that a spatial region is 
geometrically closed by the edge of the image but 
that this closure has no observational or semantic 
significance.  \lAXFI{} allows a notation that 
a spatial region is \q{closed by fiat} when the closure 
results from the intrusion of the global image boundary.  
A polygonal line may be closed by fiat when it does 
not explicitly include lines along the global boundary, 
but forms a closed polygon when such lines are 
included in practice; the result is called a 
\textit{fiat polygon} (similarly an annotation 
can be classified as a \textit{fiat curve} or 
\textit{fiat segment}.
   

\item[Segments and Regions]  A \textit{segment} is 
considered to be, canonically, an integral 
subimage with a semantically precise 
separation of \q{foreground} from \q{background}.  
There may be vagueness or approximation 
in how the segment is precisely individuated 
from its surroundings, but these ambiguities 
are considered to be practical limitations 
due to limited computer power, limiting pixel 
resolution, and so forth.  A \textit{region} 
or \textit{region of interest} is similar to a 
segment, but defined more loosely; regions 
can have vague descriptors, and spatially 
disconneted parts of an image can be treated 
as part of the same region.  In a photograph 
of a flock of birds, say, there may be 
multiple segments, each outlining one single bird.  
However, the flock as a whole may be outlined 
by one \textit{region}, which could (without 
being deemed approximative) include some of the 
background sky.  A \textit{segment} can be seen 
as subkind of \textit{region} with stricter 
granular and topological requirements. 

A \textit{partition} of an image is a segmentation 
which exhaustively classifies every point into 
one or another segment.  A \textit{selective} 
segmentation, unlike a complete partition, only 
isolates certain segments or regions of 
interests.  \lAXFI{} adopts these terms 
as parameters that can characterize 
segmentation processes, and by extension the 
resulting segments/regions. 

\item[Locations]  In \AXFI{}, \textit{locations} 
are considered to be designations of areas 
within an image (of varying dimensions) which 
are significant by virtue of their directional, 
morphological, or topological relations to 
the rest of the image, rather than by virtue 
of their intrinsic shape.  Conceptually, 
a \textit{location} is in many cases similar 
to a \textit{point}, but \AXFI{} does not require 
locations to be zero-dimensional.  A location 
may be designated by a small disk, or even a
region/segment.  The distinguishing feature 
of locations is that their spatial shape or extent 
are not semantically significant; instead, 
what is important about locations is their 
position in the image and how this position 
relates to surrounding image content.  For 
instance, a location might be the point/position 
where two roads intersect, or it could 
be the leftmost point on the segment-boundary of 
a car's fender or a bird's wings, or the 
geometric center of a car's body or a bird's torso.

\lAXFI{} distinguishes location-annotations from 
secondary annotations used to identify locations 
(insofar as these may be visually distinct).  For 
instance, a position may be modeled as a single 
point, but visually conveyed by an arrow, or by 
a circle hovering above the relevant point.

\item[Focal Points]  The concept of \textit{focal 
points} integrates locations and geometric primitives 
for certain analytic tasks.  In general, focal points 
are important for positional rather than morphological 
regions, similar to locations.  However, focal points 
may function more like segments or geometric objects 
when used as part of an analytic objective.  For 
instance, points embodying the center of a car's body 
or a bird's torso could be used to count the number 
of birds or cars appearing in a photograph.  
In this case the concept of \q{geometric center} has 
no meaningful morphological properties, so it 
is analogous to a location.  On the other hand, the 
center may be treated as a proxy for, or a most 
significant component of, a segment or region; in this 
sense the point is \textit{part of} a segment.  
This mereological aspect makes focal points 
act conceptually more like geometric primitives than 
like locations.  In short, the classification 
\textit{focal point} is available for spatial 
objects which behave somewhere between locations 
and regions/points, and particularly when they 
are used in some proxying or indicative relation 
to other regions (e.g. for counting).  
   
\textit[Secondary Images]  In some contexts, such 
as segmentation, image-transforms are used to 
convey image-processing operations, in contrast 
to geometric-style annotations that can be 
defined via vertex coordinates.  A crisp 
segment can be defined via a two-toned image 
with the same dimensions as the annotated 
image, but with only two logical colors 
(colors are called \q{logical} insofar as the 
relevant detail is how the colors compare to 
one another, irregardless of the optical colors 
used to render them\footnote{In \Qt{}, for instance, 
the \textbf{QColorConstants::Color0} and 
\textbf{QColorConstants::Color1} color values 
are considered to be special \q{colors} 
(i.e., \textbf{QColor} instances) which define 
the foreground and background of a two-toned 
image; they are not formally assigned to a 
fixed visual color, like black or white.}).  
A \q{fuzzy} segment can likewise be defined 
via a greyscale image (anything which 
is pure-background becoming either pure white, 
or pure transparent, depending on context).  
Secondary images can be used to denote annotations 
which are too granular to be summarized by 
any mathematical expression.  In these cases, 
the actual annotation data should identify the 
secondary image (e.g., via a file path) and 
explain how it relates to the primary 
(or \q{ground}) image, wheres the secondary 
file itself fills in the annotation details.      

Secondary images may be derived from ground 
images merely to present annotations, but they 
may also be intermediate analysands which are 
themselves annotated.  In the latter case, 
annotations on secondary images are usually 
meaningful also as annotations on ground 
images, so the interrelations between both 
images should be notated in the annotation data.

\item[Proscriptive Annotations]  \lAXFI{} 
allows for the designation of certain annotations 
as \q{proscriptive} when they do not 
formally delineate a geometric object, but 
indirectly express such an object's shape 
or how it may be derived.  A canonical 
example is the use of color --- perhaps 
color-enhanced secondary images --- to 
designate segments or regions of interest.  
For instance, one common segmentation technique 
uses color simplification; smoothing out color 
patches can facilitate image partitioning 
by reinforcing the boundaries between different 
regions.  With sufficient color manipulation, 
image-segments can potentially be described 
chromatically; a region may for instance 
be identified as \q{the area all of whose 
pixels display a red channel above} some 
threshold.  \lAXFI{} data should therefore 
allow such indirect designations of 
segments (and spatial/geometric regions in 
general) to be encoded as annotations.

The concept of proscriptive annotations is not 
only chromatic; one can imagine other scenarios 
where morphological features, such as symmetries, 
may be employed to similar effect.  In describing 
a chess board, for instance, a set of image-regions 
(each square on the board) can be derived 
via translational transforms of an initial 
two-segment kernel (one black square and one white).  
This representation is possible because of 
translational symmetries in the underlying image.  
Such geometric transforms and symmetries can 
sometimes be used to \q{generate} meaningful 
image segments, so they should be notated 
in \AXFI{} data when appropriate.  

\item[Semantic Labels]  Some annotations supply data 
about other annotations (what \AIM{} calls 
\q{Annotation on Annotation} as opposed to 
\q{Annotation on Image}).  In general, we 
can supply a label, description, or semantic 
classification indicating what an image segment 
or region is \q{about}, i.e., what it \q{depicts} 
(a bird, a flock of birds, a car, a traffic 
jam, and so forth).  Such meta-annotation may be 
seen as a semantic postlude to an imaging  
process, but it may also be seen as providing further 
detail about the process itself.  For instance, 
suppose segmentation isolates a bird from the 
sky: the epilogue to this operation is an ability 
(for a human user or a computer algorithm) to 
classify the segment as \q{bird.}  However, we 
can also say that the given fact of the segment 
capturing the optics of a bird (with its given 
anatomical outline and coloration) is what 
allowed the segmentation to be possible in the 
first place.  Therefore, the label \q{bird} 
serves both to utilize the segmentation for 
further analytic/classificatory purposes and 
also, potentially, to characterize the 
inner workings of effective feasibility of 
achieving the desired processing objective.  
It should be kept in mind, in short, that 
label annotations are not exclusively 
intended to be used for practical labeling 
in the contexts of tasks such as subject-marking 
images in corpora; labeling could also 
be used to notate how semantic properties 
of the image make segmentation (and other 
processing objectives) more or less 
feasible, or computationally intensive/error-prone. 

The semantics of labels themselves is outside 
the scope of \AXFI{}.  \lAXFI{} makes no effort 
to proscibe what sorts of terms are useful 
as labels (\q{bird,} \q{car,} \q{ocean,} 
\q{tumor,} etc.), or whether labels are 
simple strings/words or have internal structure 
of their own.  \lAXFI{} does, however, 
make the following minimal stipulations about 
labels.  First, \AXFI{} distinguishes 
\q{mass,} \q{count,} and \q{plural} label 
targets --- e.g., \textit{ocean}, \textit{one bird}, 
\textit{two birds}.  These distinctions are 
directly relevant to how segments are bounded and 
counted; for instance, labeling a body of water 
as \textit{a lake} (e.g. on a satellite image where 
the lake's full shore is visible) versus \textit{ocean} 
(continuing to the horizon) implies different 
desiderata for how the labeled region spatially extends 
in relation to the surrounding image.  Secondly, 
with respect to semantic labels, \AXFI{} recommends 
that applications and libraries enable 
labels to be typed values --- instances of 
application-specific data types --- as well 
as simple character strings (like \q{bird.})  
This is consistent with \AXFI{}'s general 
approach to application integration and 
type systems, to be discussed further in 
Part II. 

In addition to labels being second-order annotations 
--- assertions attached to an image segment, 
for example, or any other image feature deemed 
semantically relevant in some context --- labels 
may be applied to the image as a whole.  In this
context labels would be an example of 
\textit{metadata} annotations (discussed below, 
item ??), which apply to the image itself 
rather than any proper part. 

%\item[Data Transformations]    

\item[Image Metadata]   This category 
of annotation concerns any information 
\q{about} an image which is apart from 
the specific image content --- including 
computational/encoding details such 
as color depth, pixel dimensions, and image 
file format, as well as acquisiton details 
such as the make and settings of the camera 
whose photograph yields the current image.  
In theory, any information prerequisite to 
displaying an image may potentially 
be relevant to how annotations are 
used and interpreted.  Therefore, 
\AXFI{} should support representations of 
image metadata where such information is needed 
to fully specify annotation data; the metadata 
would then, typically, be considered an 
annotation on the entire image.  It is 
outside the scope of \AXFI{} to present 
image-related data which is applicable  
to the sharing or rendering of images but 
not, in a particular context, important to 
annotaitons themselves --- unlike \DICOM{}, 
for instance, which is designed as a 
standard for ensuring that images are correctly 
rendered and therefore includes detailed 
specifications of image formats and 
prerequisties, \AXFI{} introduces 
image metadata only where such information 
can be considered intrinsic to the 
specific objectives and operations of image 
annotation.  

\item[Acquisition Context]  Among the class of 
annotations which apply to the image as a whole, 
\AXFI{} recognizes a distinct kind of annotation 
covering what \AIM{} calls the \q{context,} 
or the purpose and concrete goal which compels 
users to initiate image-processing operations.  
In general, some of this contextual data 
should be preserved as a supplement to 
the overall information generated by image-processing 
operations.  Often image processing itself is only 
one stage in a multi-faceted scientific workflow, 
and contextual information which exists prior to 
the image-annotation step may still be important 
for subsequent steps.  For example, clinical data 
which motivates a radiographic study should remain 
linked to images and annotations which result from 
that study; this is why \DICOMSR{} includes 
clinical as well as imaging data.  More details 
about \AXFI{} context descriptions are provided 
below (section ?). 
\end{description}

This outline coveres the different kinds 
of annotation data recognized by \AXFI{}.  
The following section will fill in some 
missing details by addressing other 
sorts of information, apart from annotations 
themselves, which might be included in an 
\AXFI{} data package.}

\section[Non-Annotation Data in AXFI]{Non-Annotation Data in \protect\llAXFI{}}

\p{This section covers background information which may 
be addressed by \AXFI{}-modeled objects to ensure 
that annotations are properly understood.

\begin{description}

\item[Dimensions and Coordinate Systems]  Many different coordinate 
systems may come into play during an overall image-processing 
task.  Even restricting attention to \TwoD{} graphics --- 
where each pixel is mathematically locatable with just two 
numbers --- different algorithms or code libraries 
define \TwoD{} coordinates in different ways, so that 
representations of individual points need to be mapped 
across several different transformations.  Coordinate systems 
differ in terms of whether locations are specified with 
integer or real-valued (via binary approximation) magnitudes, 
and whether designations of shapes and geometric structures 
are oriented to bounding-box points or centers; whether 
magnitudes increase left-to-right/top-to-bottom or 
vice-versa; and so forth.  The application library with which 
an image is rendered may have a different system then 
the analytic library with which it is processed.  
These differences become significant insofar as, 
for example, annotation data should be modeled 
so that applications can display visualizations of 
annotations and respond to \GUI{} events so that 
users can examine or manipulate annotations. 

During the course of image proessing, certain operations 
moreover generate new sorts of coordinates; for example, 
segmentation may result in one part of an image 
being isolated in such a way that a separate coordinate 
system can be used for the interior of that region.  
The \q{Insight Toolkit} (\ITK{}), for example, 
distinguishes \q{object space} (internal to individual 
geometric objects) from \q{world space.}

Furthermore, additional coordinate systems may be 
relevant to the context or empirical situation 
wherein an image is acquired.  Geospatial 
images, for example, can associate image-space 
locations with geographical coordinates.  In some 
contexts, the coordinate frame applicable to 
some image can be contextualized with reference 
to parameters of the acquisition equipment; 
consider microscope configuration.  The ground 
image may also be the result of intermediate 
processing subsequent to data acquisition; 
consider Flow Cytometry transformations.  
In these cases, the coordinates and dimensions 
considered \q{internal} to the image can be 
associated with other dimensions pertaining 
to the real-world context and/or image-acquisition 
protocol.

To make these connections explicit, \AXFI{} 
employs a distinct category of \textit{dimension} 
objects, which can ground multiple parameters 
related to dimensions/axes, their scales, units 
of measurement, kinds of magnitude, transformations, 
and so forth.  One or more dimension objects 
can then be merged into an object representing 
a specific \textit{coordinate system}. 

\item[Vertices]  A vertex is a position which 
can be expressed in one or more coordinate systems.  



\end{description}
}

\p{}






