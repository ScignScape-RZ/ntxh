
\part{Outline of the AXFI Data Model}

\section{Types of Image Annotations}
\p{The most fundamental kinds of image annotations 
are geometric primitives such as points, lines, 
and polygons.  Many other forms of annotations 
are possible, however, mostly supplemental 
data which is associated with these primitives 
or, in some cases, with the image as a whole.  
In general, \AXFI{} classifies annotations into 
the following groups:

\begin{description}
\item[Geometric Primitives]  These annotations 
delineate spatial/geometric regions in zero, one, 
or two dimensions (or potentially higher dimensions 
when working in non-\TwoD{} contexts).  At a minimum, 
\AXFI{} data should support points, lines, 
polygons, polylines (considered a superkind of 
polygons where a polygon is a closed polyline), circles, 
and ellipses.  Additionally, \AXFI{} recognizes 
generic \q{closed} and \q{open} \textit{curves}, which 
are nonlinear one-dimensional regions (or boundaries 
of two-dimensional regions) that are neither 
elliptical or circular arcs.  Depending on 
context, \AXFI{} can be extended to provide 
more detailed subkinds of curves generated 
by different sorts of mathematical equations, 
along with notations for the formulae 
(e.g., b-splines) that generate a particular 
curve.\footnote{In formal statements, 
this and other \AXFI{} documentation will 
use the term \q{kind,} as well as \q{superkind} 
and \q{subkind,} to indicate groups of values/entities 
identified via a classification.  Vocabulary 
based on \q{kind} is preferred to \q{type} or 
\q{class} because of the distinct meanings these
latter terms have in computational contexts 
which are also discussed in reference to \sAXFI{}.}  
More information about encoding 
ellipse data as well as other sorts of 
curves is outlined below (section ).  

A variation on the polygon/polygon-line alternative 
are polygon-lines demarcating spatial regions 
whose boundaries extend to one or more sides/corners 
of the image.   

\item[Segments and Regions]  A \textit{segment} is 
considered to be, canonically, an integral 
subimage with a semantically precise 
separation of \q{foreground} from \q{background}.  
There may be vagueness or approximation 
in how the segment is precisely individuated 
from its surroundings, but these ambiguities 
are considered to be practical limitations 
due to limited computer power, limiting pixel 
resolution, and so forth.  A \textit{region} 
or \textit{region of interest} is similar to a 
segment, but defined more loosely; regions 
can have vague descriptors, and spatially 
disconneted parts of an image can be treated 
as part of the same region.  In a photograph 
of a flock of birds, say, there may be 
multiple segments, each outlining one single bird.  
However, the flock as a whole may be outlined 
by one \textit{region}, which could (without 
being deemed approximative) include some of the 
background sky.  A \textit{segment} can be seen 
as subkind of \textit{region} with stricter 
granular and topological requirements. 

\item[Locations]  In \AXFI{}, \textit{locations} 
are considered to be designations of areas 
within an image (of varying dimensions) which 
are significant by virtue of their directional, 
morphological, or topological relations to 
the rest of the image, rather than by virtue 
of their intrinsic shape.  Conceptually, 
a \textit{location} is in many cases similar 
to a \textit{point}, but \AXFI{} does not require 
locations to be zero-dimensional.  A location 
may be designated by a small disk, or even a
region/segment.  The distinguishing feature 
of locations is that their spatial shape or extent 
are not semantically significant; instead, 
what is important about locations is their 
position in the image and how this position 
relates to surrounding image content.  For 
instance, a location might be the point/position 
where two roads intersect, or it could 
be the leftmost point on the segment-boundary of 
a car's fender or a bird's wings, or the 
geometric center of a car's body or a bird's torso.

\lAXFI{} distinguishes location-annotations from 
secondary annotations used to identify locations 
(insofar as these may be visually distinct).  For 
instance, a position may be modeled as a single 
point, but visually conveyed by an arrow, or by 
a circle hovering above the relevant point.

\item[Focal Points]  The concept of \textit{focal 
points} integrates locations and geometric primitives 
for certain analytic tasks.  In general, focal points 
are important for positional rather than morphological 
regions, similar to locations.  However, focal points 
may function more like segments or geometric objects 
when used as part of an analytic objective.  For 
instance, points embodying the center of a car's body 
or a bird's torso could be used to count the number 
of birds or cars appearing in a photograph.  
In this case the concept of \q{geometric center} has 
no meaningful morphological properties, so it 
is analogous to a location.  On the other hand, the 
center may be treated as a proxy for, or a most 
significant component of, a segment or region; in this 
sense the point is \textit{part of} a segment.  
This mereological aspect makes focal points 
act conceptually more like geometric primitives than 
like locations.  In short, the classification 
\textit{focal point} is available for spatial 
objects which behave somewhere between locations 
and regions/points, and particularly when they 
are used in some proxying or indicative relation 
to other regions (e.g. for counting).  
   
\textit[Secondary Images]  In some contexts, such 
as segmentation, image-transforms are used to 
convey image-processing operations, in contrast 
to geometric-style annotations that can be 
defined via vertex coordinates.  A crisp 
segment can be defined via a two-toned image 
with the same dimensions as the annotated 
image, but with only two logical colors 
(colors are called \q{logical} insofar as the 
relevant detail is how the colors compare to 
one another, irregardless of the optical colors 
used to render them\footnote{In \Qt{}, for instance, 
the \textbf{QColorConstants::Color0} and 
\textbf{QColorConstants::Color1} color values 
are considered to be special \q{colors} 
(i.e., \textbf{QColor} instances) which define 
the foreground and background of a two-toned 
image; they are not formally assigned to a 
fixed visual color, like black or white.}).  
A \q{fuzzy} segment can likewise be defined 
via a greyscale image (anything which 
is pure-background becoming either pure white, 
or pure transparent, depending on context).  
Secondary images can be used to denote annotations 
which are too granular to be summarized by 
any mathematical expression.  In these cases, 
the actual annotation data should identify the 
secondary image (e.g., via a file path) and 
explain how it relates to the primary 
(or \q{ground}) image, wheres the secondary 
file itself fills in the annotation details.      

\end{description}

 }





