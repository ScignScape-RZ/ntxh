
\section{Extending AXFI}
\label{sec:SFour}
\p{In some contexts, \AXFI{} can be used as an 
alternative format for serializing data conformant 
to other standards related to image annotation.  
When these use-cases call for refining the 
\AXFI{} specifications, the architecture of 
\AXFI{} Object Libraries and \GTagML{} can be 
employed to make these extensions compatible 
with the core \AXFI{} model, and to 
facilitate application integration.  This 
section will consider several extensions 
along these lines allowing for \AXFI{}  
being adopted as an alternative to \GatingML{} 
(in the Flow Cytometry context), to 
\ArcGIS{} (for geospatial data), and 
to \DICOMSR{} or \AIM{} (in the biomedical 
imaging context).}

\p{Given any \AXFI{} extension which sustains the design 
principles of \AXFI{} itself, the extension's 
central technology should be an \AXFI{} Object Library 
which includes object classes modeling the relevant 
extended semantics as well as the primary, 
geometric-based \AXFI{} data.  It is recommended 
that the Object Library be implemented in a programming 
language and environment which minimizes external 
dependencies.  In particular, \C{} or \Cpp{} are 
preferred to scripting languages unless the 
library can be developed in \q{pure} scriptin code  
(without importing packages 
or modules which are mostly composed in 
\C{}/\Cpp{}).  For \Cpp{} libraries, it is recommended 
that most dependencies be limited to a single 
overarching framework, like \textbf{boost} or 
\Qt{}, without \textit{mixing} dependencies.  
In \Qt{}, for instance, the \IO{}, Regular Expression, 
and \XML{} support (and so on) is sufficiently mature 
and full-featured that there should be no need for 
dependencies on \textbf{boost::filesystem}, 
\textbf{Boost.Regex}, or etc.}

\p{For sake of discussion --- and to remain 
consistent with the default \DSPIN{} implementation 
--- assume therefore that prototypical 
\AXFI{} Object Libraries are implemented 
in \Qt{}-based \Cpp{} code.  The core of any 
new (potentially extended) library, then, 
is a \textit{co-serialization interface} 
which includes procedures for serializing 
and deserializing data structures 
(including \Cpp{} class instances assuming 
certain template procedures are instantiated 
for those classes), as well as \q{querying} 
encoded data as part of the deserialization 
process.  For \DSPIN{}, the co-serialization 
logic is implemented using LTS's \ConceptsDB{}, 
a hypergraph database engine.  An overview 
of this technology is outside the scope of 
\AXFI{}, but the key detail is that 
\ConceptsDB{} allows data structures to be 
modeled and then encoded via certain hypergraph 
constructions, and that the resulting hypergraph 
models are manipulated via an interface which 
incorporates Functional Programming techniques 
(more so than conventional Object-Oriented 
styles).  An \AXFI{} Object Library mimicking 
\DSPIN{}, accordingly, would implement 
logic to serialize and deserialize \AXFI{} objects 
according to this hypergraph-based interface.}

\p{Such an Object Library would then have, at 
its core, a suite of annotation objects 
(plus objects expressing an extended semantics) 
and would have procedures for serializing and 
deserializing these objects as hypergraph data 
structures.  This co-serialization interface 
would then be augmented to support \GTagML{} by 
re-implementing the (de)serialization procedures to generate/parse 
\GTagML{} instead of the \ConceptsDB{} 
hypergraph format.  The \GTagML{} interface 
should be tested to confirm that it 
functions equivalently to that of \ConceptsDB{}.  
Once that is verified, the generated \GTagML{} 
code can be provisioned with \q{grounding} 
metadata, that is, with parameters indicating 
the relation of the serialization content to 
the data types and procedures of the Object 
Library.  In the simplest case, a 
\GTagML{} node is \q{grounded} by being 
marked as encoding one instance of an 
object class.}

\p{An important aspect of an \AXFI{} Object 
Library, then, is functional equivalence 
between the \ConceptsDB{} serialization 
and the \GTagML{} serialization.  This notion 
of functional equivalence may then be 
extended when the purpose of an Object 
Library is to incorporate features from 
other data standards, or to allow \AXFI{} 
to replace other data formats.  For 
example, Object Libraries designed 
to adopt \AXFI{} in lieu of \GatingML{}, 
\AIM{}, or \DICOMSR{}, respectively, 
would need to demonstrate that 
they serialize and deserialize structures 
in a manner functionally equivalent to 
the formats being emulated.}

\p{In these scenarios, Object Libraries 
need to orient themselves not only to standards 
for existing markup formats, but also to 
existing libraries where data in such formats 
is generated and consumed.  For example, 
the de facto reference implementation for 
reading \GatingML{} data appears to be the 
\R{}/Bioconductor package \textbf{flowUtils}.  
Likewise, the \AIM{} data format was initially engineered 
via an open-source \Cpp{} library (\textbf{AIMLib}), 
which is paired with \textbf{AIMConverter}, a utility to 
express \AIM{} data via \DICOMSR{} (rather than 
\AIM{} \XML{}).  For \DICOMSR{}, the principle 
\Cpp{} library (used e.g. by \textbf{AIMConverter}) 
is \DCMTK{}, which is divided into multiple 
modules.  The \textbf{dcmsr} module is the 
core \DCMTK{} implementation of \DICOMSR{}, although 
\textbf{dcmpstate} (\DICOM{} Presentation State) 
is also relevant for image annotation.  
Finally, in the \GIS{} context, the 
\ArcGIS{} Runtime \SDK{} for \Qt{} is a 
standard model for managing \GIS{} annotations 
in a \Cpp{} context.}

\p{In short, these four libaries 
--- \textbf{flowUtils}, \ArcGIS{}, 
\textbf{AIMLib}, and \textbf{dcmcr} --- 
supply the most canonical implementations that would be 
referenced by \AXFI{} Object Libraries incorporating 
Flow Cytometry, \GIS{}, \AIM{}, and \DICOMSR{} data 
repectively.  Recall that \AXFI{} paradigmatically 
aims for alignment at the level of creator and consumer 
libraries, rather than at the level of markup 
format --- the \AXFI{} analogs for these libraries 
need not use the same serialization formats; 
indeed, if they use \GTagML{}, they will not use 
an \XML{}-based language at all.  However, 
the respective \AXFI{} Object Libraries 
can try to emulate the programming interface 
offered by the \q{target} libraries (\textbf{flowUtils}, 
\ArcGIS{}, \textbf{AIMLib}, \textbf{dcmcr}).  This 
entails duplicating procedure and type 
names as much as possible, and constructing 
the interface such that the steps needed 
to serialize and deserialize data are largely 
congruent when a programmer is 
using the \AXFI{}-compatible library to 
the emulated target libary.}

\p{When emulating external components, 
\AXFI{} Object Libraries should be attentive 
to the connections between emulation targets 
and their surrounding computational context.  
In the case of \GatingML{}, for example, 
gating data is only meaningful in the 
presence of cytometry information (typically 
conveyed via \FCS{} files) establishing 
the baseline data (analogoug to a ground 
image) against which gates are defined.  
To preserve autonomy, an \AXFI{} Object Library 
which supports \FCM{} gating would not necessarily 
read or display \FCS{} files itself; instead, 
such a library would be deployed as embedded 
within a larger \FCM{} programming context, 
which would be responsible for managing 
\FCS{} data and also any \GUI{} operations 
(including responding to user events 
on gates).}

\p{The gating-specific Object Library 
therefore needs to define a protocol for 
sharing gating information with this host 
application.  Such would be an example of 
an \q{\AXFI{} Host Library} as described in 
\makebox{Part I}.  The Host Library would serve as a 
bridge between the Object Library and the host 
application.  When the Object Library acquires 
\AXFI{} (including gating) data, it would 
construct gating objects and pass them to the 
host software, leveraging the fact that the Host Library provides 
a runtime overlap between the Object Library's 
and host application's data types and accessible 
computational resource (e.g., heap allocations).  
In the opposite direction, the host application 
could communicate gate modifications (due 
to user \GUI{} actions, for instance, such 
as dragging gate handles) back to the 
\AXFI{} Object Library if it is needed 
to update (re-serialize) the gating data.} 

\p{Architecturally, then, there are 
three distinct components involved: 
an \AXFI{} Object Library; a host 
application; and an \AXFI{} Host Library 
which encircles a communication protocol 
and capability overlap between those 
two.  The Object and Host libraries 
should be designed so that communication 
between the Object Library and a host 
application is rigorous and transparent, 
and so that the communication protocol can 
be reused in different contexts.  Moreover, 
the Host Library should be designed so 
that implementational objectives can 
be clarified --- criteria for the 
data management and \HCI{} features expected 
of the host application.  In the case of 
\FCM{} gating, this would include formal 
models of how gates are visualized, 
what sorts of user events (e.g., mouse-events) 
are intrinsic to the semantics of the 
\FCM{} system, and how user events translate 
into data modifications which lead to 
the gating data being updated.}

\p{Even though \FCM{} gating is mathematically 
different than annotating photographic 
images, many of these architectural details 
are functionaly parellel between the \FCM{} 
and imaging (according to traditional 
image-acquisition processes) contexts.  
This is one reason why developing \FCM{} gating 
annotations as a special case of \AXFI{} is 
productive; it allows the overall application-integration 
between annotation, serialization, and \HCI{}/\GUI{} 
features to be systematically profiled, in parallel 
to analogous models applicable to other image-related 
annotation domains.  Complementary comments apply 
to \GIS{}, which, like \FCM{}, evinces 
annotation requirements which are geometrically 
similar to systems such as \AIM{}, but 
apply to an expanded notion of \q{imaging} 
(and image acquisition).}

\p{The value 
of treating these various imaging and 
annotation domains as alternative manifestation 
of one underlying computational space is that 
the requirements shared across these 
domains --- particularly at the \HCI{} and 
\GUI{} levels --- then become foregrounded, 
whereas they are largely left implicit or 
unstatated in existing annotation frameworks.  
When formulating a multi-purpose 
image-annotation model, the relevant standardization 
applies not only to annotation \textit{data}, 
but also to annotations as user-interaction artifacts: 
how they are displayed, the event model which 
systematizes how human users interact with 
annotations, and the logic for interconnecting 
event handlers with underlying annotation data.  
The \AXFI{} architecture is organized such 
that this event model is grounded in 
\AXFI{} Host Libraries, meaning that 
event-processing is guided by a 
canonical protocol.  Such \HCI{} focus 
is harder to achieve for formats 
such as \GatingML{} or \AIM{} whose 
standardization is driven by canonical 
serialization models rather than by 
canonical library models.}

%\p{}

