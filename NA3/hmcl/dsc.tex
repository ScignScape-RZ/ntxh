   \begin{frame}{\ft{Oxy}}

        \begin{annotatedFigure}{0pt}{0pt}
            {\includegraphics[scale=1]{texs/oxy.png}}
            
  \node [text width=7.6cm,align=justify,fill=logoCyan!20, draw=logoBlue, 
  draw opacity=0.5,line width=1mm, fill opacity=0.9]
   at (0.74,0.465){\textbf{Dataset Applications make extensive 
   use of context menus to organize functionality and provide 
   advanced interactivity.  In this screenshot a context menu 
   action (\circled{1}) has been selected which alters the 2D 
   display, visually emphasizing a restricted set 
   of data points (\circled{2}) and contracting all others (\circled{3}).}};

            \annotatedFigureBox{0.73,0.26}{0.96,0.328}{1}{0.89,0.326}%            
            \annotatedFigureBox{0.55,0.4}{0.55,0.44}{3}{0.55,0.44}            
            \annotatedFigureBox{0.598,0.45}{0.598,0.48}{2}{0.598,0.45}    
      %      \annotatedFigureBox{0.222,0.284}{0.3743,0.4934}{B}{0.3743,0.4934}%tr
      %      \annotatedFigureBox{0.555,0.784}{0.6815,0.874}{C}{0.555,0.784}%bl
      %      \annotatedFigureBox{0.557,0.322}{0.8985,0.5269}{D}{0.8985,0.5269}%tr
  
        \end{annotatedFigure}

   %     \caption{Expanded Sample (A)}
    %    \label{fig:teaser}

    \end{frame}


\p{As outlined above, data citations refer to parts within a 
data set --- such as individual data records, but also larger-scale 
aggregates such as table columns or statistical parameters.  
The complication when defining data citations is that a 
concept such as a table column, although it may have an 
obvious technical status as a discrete conceptual unit from 
the point of view of scientists curating, studying or reusing 
a data set, does not necessarily correspond to a single 
coding entity that could be isolated as an annotation 
target.  It is therefore the responsibility of 
\textit{code base annotations} to provide annotations for 
computational units --- such as data types, procedures, and 
\GUI{} components --- that have an annotatable \textit{conceptual} 
status relative to the data set on which the code operates.  
Often this will involve mapping one concept to 
several computational units (for instance, several 
procedure implementations).}

\p{For a concrete example of these points concerning 
data citations, consider the data set pictured 
in Figure~\bref{fig:oxy}, representing cyber-physical measurements 
used to calculate oxygenated airflow.  The data-set 
application (interactive-visualization code deployed within 
the Research Object) displays tabular data via a tree widget 
(which functions as a generalized, multi-scale spreadsheet 
table), with tabular columns expressing quantities --- such 
as air flow and oxygen levels --- in several formats (raw 
measures as well as sample rankings and min-max 
percentages).  Conceptually, these columns have distinct 
methodological roles and therefore can be microcited; 
indeed, the application links the columns to article 
text where the corresponding concepts are presented 
(see Figure~\bref{fig:about}).  However, the implementation does not introduce 
a distinct \Cpp{} object uniquely designating individual 
columns.  Instead, the individual columns can be annotated 
in terms of \Cpp{} methods providing column-specific 
functionality.  In the current example, these methods 
primarily take the form of features linked to context-menu 
actions (copying column data to the clipboard, sorting data 
by one column, etc.).  In general, rather than a rigid 
protocol for data-set annotations, \AXF{} proposes 
heuristic guidelines for how best to map programming 
constructs to scientifically salient data-set concepts.}

\begin{frame}{\ft{About}}

	\pdfpageheight 30cm

        \begin{annotatedFigure}
            {\includegraphics[scale=1]{about.png}}
            
  \node [text width=12cm,align=justify,fill=logoCyan!20, draw=logoBlue, 
  draw opacity=0.5,line width=1mm, fill opacity=0.9]
   at (0.58,0.76){\textbf{Context menus also allow users to 
   obtain information and explanations about individual parts of the 
   data set, such as individual statistical parameters.  In this 
   screenshot, the user has right-clicked on a data column and 
   chosen a context menu action which shows, via a dialog box, 
   a precis of the quantities represened in that column and their 
   significance for the data set as a whole.}};

            \annotatedFigureBox{0.2,0.12}{0.812,0.645}{1}{0.81,0.645}%            
      %      \annotatedFigureBox{0.222,0.284}{0.3743,0.4934}{B}{0.3743,0.4934}%tr
      %      \annotatedFigureBox{0.555,0.784}{0.6815,0.874}{C}{0.555,0.784}%bl
      %      \annotatedFigureBox{0.557,0.322}{0.8985,0.5269}{D}{0.8985,0.5269}%tr
  
        \end{annotatedFigure}

\end{frame}

\p{Defining an annotation schema for data sets can potentially 
be an organic outgrowth of software-development methodology 
--- viz., the engineering steps, 
such as implementing unit tests, which are essential 
to deploying a commercial-grade application.  
This point is illustrated in 
Figure~\bref{fig:testing}, which shows a \GUI{}-based testing environment for 
the data set depicted in Figures~\bref{fig:oxy} and 
\bref{fig:about}.  For this data set, 
the context menu actions providing column-specific functionality 
are also discrete capabilities which can be covered by 
unit tests, so the set of procedures mapped to the citeable 
concept correspond with a set of unit-test requirements.  
In this data set, these procedures are also exposed to 
scripting engines via the \Qt{} meta-object system.  In general, 
there is often a structural correlation between 
scripting, unit testing, and microcitation, so that 
an applications' scripting and testing protocol can serve 
as the basis for annotation schema.  For data sets which use 
in-memory or persistent databases, evaluable queries against 
these databases provide an additional grounding for annotations.  
In general, data-annotation should be engineered on the 
basis of a dataset applications' scripting, testing, and/or 
query-evaluation code.  However, this is only a heuristic 
guideline, and \AXF{} does not presuppose any data-annotation 
scheme \textit{a priori}.}

\begin{figure}

\caption{Test Suites for Dataset Applications}
\label{fig:testing}

\begin{tikzpicture}

\node[inner sep=0pt] (x1) at (0,0)
    {\includegraphics[width=7in, 
    	trim={0mm 0mm 0mm 0mm},clip]
    	{pics/testing.png}};
    
\end{tikzpicture}   
\end{figure}


