\documentclass[11pt,letterpaper]{article}


% pmml  arff  openannotation

%\usepackage[condensed,math]{anttor}
%\usepackage[T1]{fontenc}

%\usepackage[T1]{fontenc}
%\usepackage{tgtermes}

\usepackage[hang,flushmargin]{footmisc}

\usepackage{titlesec}

%\usepackage{sectsty}
%\sectionfont{\fontsize{13}{4}\selectfont}

\titleformat{\section}
  {\normalfont\fontsize{13}{15}\bfseries}{\thesection}{1em}{}

\titlespacing*{\section}
{0pt}{7ex plus 1ex minus .1ex}{2ex plus .2ex}

%\usepackage{mathptmx}

\usepackage{eso-pic}

%\setlength\parindent{0pt}

\AddToShipoutPictureBG{%

\ifnum\value{page}>1{
\AtTextUpperLeft{
\makebox[20.5cm][r]{
\raisebox{-1.95cm}{%
{\transparent{0.3}{\includegraphics[width=0.29\textwidth]{e-logo.png}}	}} } }
}\fi
}

\AddToShipoutPicture{%
{
 {\color{blGreen!70!red}\transparent{0.9}{\put(0,0){\rule{3pt}{\paperheight}}}}%
 {\color{darkRed!70!purple}\transparent{1}\put(3,0){{\rule{4pt}{\paperheight}}}}
% {\color{logoPeach!80!cyan}\transparent{0.5}{\put(0,700){\rule{1cm}{.6cm}}}}%
% {\color{darkRed!60!cyan}\transparent{0.7}\put(0,706){{\rule{1cm}{.6cm}}}}
% \put(18,726){\thepage}
% \transparent{0.8}
}
}

\AddToShipoutPicture{%
\ifnum\value{page}>1
{\color{blGreen!70!red}\transparent{0.9}{\put(300,8){\rule{0.5\paperwidth}{.3cm}}}}%
{\color{inOne}\transparent{0.8}{\put(300,10){\rule{0.5\paperwidth}{.3cm}}}}%
{\color{inTwo}\transparent{0.3}\put(300,13){{\rule{0.5\paperwidth}{.3cm}}}}

{\color{blGreen!70!red}\transparent{0.9}{\put(5.6,5){\rule{0.5\paperwidth}{.4cm}}}}%
{\color{inOne}\transparent{1}{\put(5.6,10){\rule{0.5\paperwidth}{.4cm}}}}%
{\color{inTwo}\transparent{0.3}\put(5.6,15){{\rule{0.5\paperwidth}{.4cm}}}}

\put(301,16){%
\transparent{0.7}{
\includegraphics[width=0.2\textwidth]{logo.png}} }
\fi
}

\AddToShipoutPicture{%
\ifnum\value{page}=1
\put(257.5,729){%
	\transparent{0.7}{
		\includegraphics[width=0.2\textwidth]{logo.png}}}
%\put(59,953){}
\fi
}	


%\pagestyle{empty} % no page number
%\parskip 7.2pt    % space between paragraphs
%\parindent 12pt   % indent for new paragraph
%\textwidth 4.5in  % width of text
%\columnsep 0.8in  % separation between columns

%\setlength{\footskip}{7pt}

\usepackage[paperheight=11in,paperwidth=8.5in]{geometry}
\geometry{left=.74in,top=.5in,right=.74in,bottom=1.2in} %margins

\usepackage{etoolbox}% http://ctan.org/pkg/etoolbox
\makeatletter
% \patchcmd{<cmd>}{<search>}{<replace>}{<success>}{<failure>}
\patchcmd{\@part}{\par}{\quad}{}{}
\patchcmd{\@part}{\huge}{\Large}{}{}
\makeatother

\renewcommand{\partname}{\hspace{-1em}Part}

\renewcommand*\thepart{\Roman{part}:}

\renewcommand{\thepage}{\raisebox{2pt}{\arabic{page}}}

\renewcommand{\footnoterule}{%
	\kern -3pt
	\hrule width .92\textwidth height .5pt
	\kern 10pt
}


\usepackage[hyphens]{url}
\newcommand{\biburl}[1]{ {\fontfamily{gar}\selectfont{\textcolor[rgb]{.2,.6,0}%
{\scriptsize {\url{#1}}}}}}

%\linespread{1.3}

\newcommand{\sectsp}{\vspace{12pt}}

\usepackage{graphicx}
\usepackage{color,framed}

\usepackage{textcomp}

\usepackage{float}

\usepackage{mdframed}


\usepackage{setspace}
\newcommand{\rpdfNotice}[1]{\begin{onehalfspacing}{

\Large #1

}\end{onehalfspacing}}

\usepackage{xcolor}

\usepackage[hyphenbreaks]{breakurl}
\usepackage[hyphens]{url}

\usepackage{hyperref}
\newcommand{\rpdfLink}[1]{\href{#1}{\small{#1}}}
\newcommand{\dblHref}[1]{\href{#1}{\small{\burl{#1}}}}
\newcommand{\browseHref}[2]{\href{#1}{\Large #2}}

\colorlet{blCyan}{cyan!50!blue}

\definecolor{darkRed}{rgb}{.2,.0,.1}


\definecolor{blGreen}{rgb}{.2,.7,.3}

\definecolor{darkBlGreen}{rgb}{.1,.3,.2}

\definecolor{oldBlColor}{rgb}{.2,.7,.3}

\definecolor{blColor}{rgb}{.1,.3,.2}

\definecolor{elColor}{rgb}{.2,.1,0}
\definecolor{flColor}{rgb}{0.7,0.3,0.3}

\definecolor{logoOrange}{RGB}{108, 18, 30}
\definecolor{logoGreen}{RGB}{85, 153, 89}
\definecolor{logoPurple}{RGB}{200, 208, 30}

\definecolor{logoBlue}{RGB}{4, 2, 25}
\definecolor{logoPeach}{RGB}{255, 159, 102}
\definecolor{logoCyan}{RGB}{66, 206, 244}
\definecolor{logoRed}{rgb}{.3,0,0}

\newcommand{\colorq}[1]{{\color{logoOrange!70!black}{\q{\small\textbf{#1}}}}}

\definecolor{inOne}{rgb}{0.122, 0.435, 0.698}% Rule colour
\definecolor{inTwo}{rgb}{0.122, 0.698, 0.435}% Rule colour

\definecolor{outOne}{rgb}{0.435, 0.698, 0.122}% Rule colour
\definecolor{outTwo}{rgb}{0.698, 0.435, 0.122}% Rule colour

\colorlet{linkcolor}{flColor!60!red}


\hypersetup{
	colorlinks=true,
	citecolor=blCyan!40!green,
	filecolor=magenta!30!logoBlue,
	urlcolor=blue,
    linkcolor=linkcolor!70!black,
%    allcolors=blCyan!40!green
}


\usepackage[many]{tcolorbox}% http://ctan.org/pkg/tcolorbox

\usepackage{transparent}

\newlength{\bsep}
\setlength{\bsep}{-1pt}
\let\xbibitem\bibitem
\renewcommand{\bibitem}[2]{\vspace{\bsep}\xbibitem{#1}{#2}}

\newenvironment{cframed}{\begin{mdframed}[linecolor=logoPeach,linewidth=0.4mm]}{\end{mdframed}}

\newenvironment{ccframed}{\begin{mdframed}[backgroundcolor=logoGreen!5,linecolor=logoCyan!50!black,linewidth=0.4mm]}{\end{mdframed}}


%\usepackage[T1]{fontenc}

%\usepackage{aurical}
% \Fontauri

\usepackage{gfsdidot}
\usepackage[T1]{fontenc}

%\makeatletter
%\f@family,  cmr, T1, n, m,
%\f@encoding,
%\f@shape,
%\f@series,
%\makeatother



%\usepackage{LibreBodoni}

%\usepackage{fontspec}
%\setmainfont{QTBengal}

\usepackage{relsize}

\newcommand{\bref}[1]{\hspace*{1pt}\textbf{\ref{#1}}}

\newcommand{\pseudoIndent}{

\vspace{10pt}\hspace*{12pt}}

\newcommand{\YPDFI}{{\fontfamily{fvs}\selectfont YPDF-Interactive}}

%
\newcommand{\deconum}[1]{{\protect\raisebox{-1pt}{{\LARGE #1}}}}

\newcommand{\visavis}{vis-\`a-vis}

\newcommand{\VersatileUX}{{\color{red!85!black}{\Fontauri Versatile}}%
{{\fontfamily{qhv}\selectfont\smaller UX}}}

\newcommand{\NDPCloud}{{\color{red!15!black}%
{\fontfamily{qhv}\selectfont {\smaller NDP C{\smaller LOUD}}}}}

\newcommand{\MThreeK}{{\color{blGreen!45!black}%
{\fontfamily{qhv}\fontsize{10}{8}\selectfont {M3K}}}}


\newcommand{\lfNDPCloud}{{\color{red!15!black}%
{\fontfamily{qhv}\selectfont N{\smaller DP C{\smaller LOUD}}}}}

\newcommand{\textds}[1]{{\fontfamily{lmdh}\selectfont{%
\raisebox{-1pt}{#1}}}}

%\newcommand{\dsC}{{\textds{ds}{\fontfamily{qhv}\selectfont \raisebox{-1pt}
%{\color{red!15!black}{C}}}}}

\definecolor{tcolor}{RGB}{24,52,61}



\newcommand{\ParaView}{\resizebox{!}{7pt}{\AcronymText{ParaView}}}
\newcommand{\Octave}{\resizebox{!}{7pt}{\AcronymText{Octave}}}
\newcommand{\ROOT}{\resizebox{!}{7pt}{\AcronymText{ROOT}}}
\newcommand{\CERN}{\resizebox{!}{7pt}{\AcronymText{CERN}}}
\newcommand{\MQFour}{\resizebox{!}{7pt}{\AcronymText{MQ4}}}
\newcommand{\VISSION}{\resizebox{!}{7pt}{\AcronymText{VISSION}}}



\newcommand{\CCpp}{\resizebox{!}{7pt}{\AcronymText{C}}/\Cpp{}}
\newcommand{\NoSQL}{\resizebox{!}{7pt}{\AcronymText{NoSQL}}}
\newcommand{\SQL}{\resizebox{!}{7pt}{\AcronymText{SQL}}}

\newcommand{\SPARQL}{\resizebox{!}{7pt}{\AcronymText{SPARQL}}}

\newcommand{\NCBI}{\resizebox{!}{7pt}{\AcronymText{NCBI}}}

\newcommand{\HTXN}{\resizebox{!}{7pt}{\AcronymText{HTXN}}}

\newcommand{\OWL}{\resizebox{!}{7pt}{\AcronymText{OWL}}}

\newcommand{\TDM}{\resizebox{!}{7pt}{\AcronymText{TDM}}}

\newcommand{\lHTXN}{\resizebox{!}{7.5pt}{\AcronymText{H}}%
\resizebox{!}{6.5pt}{\AcronymText{TXN}}}

\newcommand{\lsHTXN}{\resizebox{!}{9.5pt}{\AcronymText{\textcolor{tcolor}{HTXN}}}}

\newcommand{\LAF}{\resizebox{!}{7pt}{\AcronymText{LAF}}}

\newcommand{\UDpipe}{\resizebox{!}{7pt}{\AcronymText{UDpipe}}}

\newcommand{\C}{\resizebox{!}{7pt}{\AcronymText{C}}}


\usepackage{mdframed}

\newcommand{\cframedboxpanda}[1]{\begin{mdframed}[linecolor=yellow!70!blue,linewidth=0.4mm]#1\end{mdframed}}


\newcommand{\PVD}{\resizebox{!}{7pt}{\AcronymText{PVD}}}

\newcommand{\SDK}{\resizebox{!}{7pt}{\AcronymText{SDK}}}
\newcommand{\NLP}{\resizebox{!}{7pt}{\AcronymText{NLP}}}

\newcommand{\AXF}{\resizebox{!}{7pt}{\AcronymText{AXF}}}

\newcommand{\lAXF}{\resizebox{!}{7.5pt}{\AcronymText{A}}%
\resizebox{!}{6.5pt}{\AcronymText{XF}}}


\newcommand{\lsAXF}{\resizebox{!}{8.5pt}{\AcronymText{AXF}}}

\newcommand{\AXFD}{\resizebox{!}{7pt}{\AcronymText{AXFD}}}

\newcommand{\CBICA}{\resizebox{!}{7pt}{\AcronymText{CBICA}}}

\newcommand{\IORT}{\resizebox{!}{7pt}{\AcronymText{IORT}}}


\newcommand{\SeDI}{\resizebox{!}{7pt}{\AcronymText{SeDI}}}
\newcommand{\RSNA}{\resizebox{!}{7pt}{\AcronymText{RSNA}}}

\newcommand{\CER}{\resizebox{!}{7pt}{\AcronymText{CER}}}
\newcommand{\PACS}{\resizebox{!}{7pt}{\AcronymText{PACS}}}

\newcommand{\DICOM}{\resizebox{!}{7pt}{\AcronymText{DICOM}}}

\newcommand{\CT}{\resizebox{!}{7pt}{\AcronymText{CT}}}

\newcommand{\LOINC}{\resizebox{!}{7pt}{\AcronymText{LOINC}}}


\newcommand{\TAGML}{\resizebox{!}{7pt}{\AcronymText{TAGML}}}

\newcommand{\CPAP}{\resizebox{!}{7pt}{\AcronymText{CPAP}}}

\newcommand{\RadLex}{\resizebox{!}{7pt}{\AcronymText{RadLex}}}


\newcommand{\OMOP}{\resizebox{!}{7pt}{\AcronymText{OMOP}}}
\newcommand{\PCORnet}{\resizebox{!}{7pt}{\AcronymText{PCORnet}}}
\newcommand{\FHIR}{\resizebox{!}{7pt}{\AcronymText{FHIR}}}

\newcommand{\CaPTk}{\resizebox{!}{7pt}{\AcronymText{CaPTk}}}

\newcommand{\VIOLIN}{\resizebox{!}{7pt}{\AcronymText{VIOLIN}}}



\newcommand{\lAXFD}{\resizebox{!}{7.5pt}{\AcronymText{A}}%
\resizebox{!}{6.5pt}{\AcronymText{XFD}}}


\newcommand{\IJST}{\resizebox{!}{7pt}{\AcronymText{IJST}}}

\newcommand{\Jupyter}{\resizebox{!}{7pt}{\AcronymText{Jupyter}}}
\newcommand{\Python}{\resizebox{!}{7pt}{\AcronymText{Python}}}
\newcommand{\IDN}{\resizebox{!}{7pt}{\AcronymText{IDN}}}
\newcommand{\JPG}{\resizebox{!}{7pt}{\AcronymText{JPG}}}
\newcommand{\PNG}{\resizebox{!}{7pt}{\AcronymText{PNG}}}
\newcommand{\TIFF}{\resizebox{!}{7pt}{\AcronymText{TIFF}}}
\newcommand{\REPL}{\resizebox{!}{7pt}{\AcronymText{REPL}}}


\newcommand{\BioC}{\resizebox{!}{7pt}{\AcronymText{BioC}}}

\newcommand{\CoNLL}{\resizebox{!}{7pt}{\AcronymText{CoNLL}}}
\newcommand{\CoNLLU}{\resizebox{!}{7pt}{\AcronymText{CoNLL-U}}}

\newcommand{\sapp}{\resizebox{!}{7pt}{\AcronymText{Sapien+}}}
\newcommand{\lsapp}{\resizebox{!}{8.5pt}{\AcronymText{Sapien+}}}
\newcommand{\lssapp}{\resizebox{!}{9.5pt}{\AcronymText{Sapien+}}}

\newcommand{\ePub}{\resizebox{!}{7pt}{\AcronymText{ePub}}}

%\lsLPF


\newcommand{\GIT}{\resizebox{!}{7pt}{\AcronymText{GIT}}}

%\definecolor{atColor}{RGB}{11, 71, 17}


\DeclareMathVersion{fordg}
\SetSymbolFont{letters}{fordg}{OML}{cmr}{b}{n}

\definecolor{atcColor}{RGB}{96, 17, 12}
\newcommand{\ATextCClr}[1]{\textcolor{atcColor}{\textbf{#1}}}

\newcommand{\AIMConc}{\resizebox{!}{7.5pt}{\ATextCClr{AIM-Concepts}}}
\newcommand{\lAIMConc}{\resizebox{!}{8pt}{\ATextCClr{AIM-Concepts}}}

\newcommand{\HGXF}{{\resizebox{!}{7.5pt}{\ATextCClr{HGXF}}}}
\newcommand{\lHGXF}{{\resizebox{!}{8pt}{\ATextCClr{HGXF}}}}
\newcommand{\sHGXF}{{\resizebox{!}{6pt}{\ATextCClr{HGXF}}}}

\newcommand{\MdsX}{\resizebox{!}{7.5pt}{\ATextCClr{MdsX}}}
\newcommand{\lsMdsX}{\resizebox{!}{9pt}{\ATextCClr{MdsX}}}


\newcommand{\HMCL}{{\resizebox{!}{7.5pt}{\ATextCClr{HMCL}}}}
\newcommand{\DSPIN}{{\resizebox{!}{7.5pt}{\ATextCClr{D-SPIN}}}}


\newcommand{\CRtwo}{{\resizebox{!}{7.5pt}{\ATextCClr{CR2}}}}
\newcommand{\lCRtwo}{{\resizebox{!}{8pt}{\ATextCClr{CR2}}}}
\newcommand{\sCRtwo}{{\resizebox{!}{6pt}{\ATextCClr{CR2}}}}


\newcommand{\THQL}{\resizebox{!}{7.5pt}{\ATextCClr{THQL}}}
\newcommand{\lTHQL}{\resizebox{!}{8pt}{\ATextCClr{THQL}}}

\newcommand{\HDICOM}{\resizebox{!}{7.5pt}{\ATextCClr{{\large h}-DICOM}}}

\newcommand{\hVaImm}{\resizebox{!}{7.5pt}{\ATextCClr{{\large h}-VaImm}}}


\newcommand{\PhaonVI}{\resizebox{!}{7.5pt}{\ATextCClr{Phaon-VI}}}



\definecolor{atColor}{RGB}{50, 22, 40}
\newcommand{\ATextClr}[1]{\textcolor{atColor}{\textbf{#1}}}

\newcommand{\DgDb}{{\mathversion{fordg}%
\makebox{\raisebox{-3pt}{\resizebox{!}{11pt}{\ATextClr{%
\rotatebox{17}{$\varsigma$}}}}\hspace{-4pt}%
\resizebox{!}{6.5pt}{\ATextClr{D\hspace{-2pt}B}}}}}


\newcommand{\lDgDb}{{\mathversion{fordg}%
\resizebox{!}{12pt}{\ATextClr{%
\rotatebox{17}{$\varsigma$}}}}\hspace{-4pt}%
\resizebox{!}{6.5pt}{\ATextClr{D\hspace{-2pt}B}}}}}

\newcommand{\URL}{\resizebox{!}{7pt}{\AcronymText{URL}}}
\newcommand{\CSML}{\resizebox{!}{7pt}{\AcronymText{CSML}}}
\newcommand{\LPF}{\resizebox{!}{7pt}{\AcronymText{LPF}}}
\newcommand{\lLPF}{\resizebox{!}{8.5pt}{\AcronymText{LPF}}}
\newcommand{\lsLPF}{\resizebox{!}{9.5pt}{\AcronymText{LPF}}}

\newcommand{\AI}{\resizebox{!}{7.5pt}{\AcronymText{AI}}}
\newcommand{\lAI}{\resizebox{!}{8pt}{\AcronymText{AI}}}

\makeatletter

\newcommand*\getX[1]{\expandafter\getX@i#1\@nil}

\newcommand*\getY[1]{\expandafter\getY@i#1\@nil}
\def\getX@i#1,#2\@nil{#1}
\def\getY@i#1,#2\@nil{#2}
\makeatother
	
\newcommand{\rectann}[9]{%
\path [draw=#1,draw opacity=#2,line width=#3, fill=#4, fill opacity = #5, even odd rule] %
(#6) rectangle(\getX{#6}+#7,\getY{#6}+#8)
({\getX{#6}+((#7-(#7*#9))/2)},{\getY{#6}+((#8-(#8*#9))/2)}) rectangle %
({\getX{#6}+((#7-(#7*#9))/2)+#7*#9},{\getY{#6}+((#8-(#8*#9))/2)+#8*#9});}


\definecolor{pfcolor}{RGB}{94, 54, 73}

\newcommand{\EPF}{\resizebox{!}{7pt}{\AcronymText{ETS{\color{pfcolor}pf}}}}
\newcommand{\lEPF}{\resizebox{!}{8.5pt}{\AcronymText{ETS{\color{pfcolor}pf}}}}
\newcommand{\lsEPF}{\resizebox{!}{9.5pt}{\AcronymText{ETS{\color{pfcolor}pf}}}}


\newcommand{\XPDF}{\resizebox{!}{7pt}{\AcronymText{XPDF}}}

\newcommand{\GRE}{\resizebox{!}{7pt}{\AcronymText{GRE}}}
\newcommand{\CAS}{\resizebox{!}{7pt}{\AcronymText{CAS}}}

\newcommand{\lMOSAIC}{%
\resizebox{!}{8pt}{\AcronymText{M}}%
\resizebox{!}{6pt}{\AcronymText{OSAIC}}}

\newcommand{\XML}{\resizebox{!}{7pt}{\AcronymText{XML}}}
\newcommand{\RDF}{\resizebox{!}{7pt}{\AcronymText{RDF}}}
\newcommand{\DOM}{\resizebox{!}{7pt}{\AcronymText{DOM}}}

\newcommand{\Covid}{\resizebox{!}{7pt}{\AcronymText{Covid-19}}}

\newcommand{\CLang}{\resizebox{!}{7pt}{\AcronymText{C}}}

\newcommand{\HNaN}{\resizebox{!}{7pt}{\AcronymText{HN%
\textsc{a}N}}}

\newcommand{\JSON}{\resizebox{!}{7pt}{\AcronymText{JSON}}}
\newcommand{\UV}{\resizebox{!}{7pt}{\AcronymText{UV}}}


\newcommand{\MeshLab}{\resizebox{!}{7pt}{\AcronymText{MeshLab}}}
\newcommand{\IQmol}{\resizebox{!}{7pt}{\AcronymText{IQmol}}}

\newcommand{\SGML}{\resizebox{!}{7pt}{\AcronymText{SGML}}}

\newcommand{\WhiteDB}{\makebox{WhiteDB}}

\newcommand{\ASCII}{\resizebox{!}{7pt}{\AcronymText{ASCII}}}

\newcommand{\DSPIN}{\resizebox{!}{7pt}{\AcronymText{D-SPIN}}}


\newcommand{\GUI}{\resizebox{!}{7pt}{\AcronymText{GUI}}}

\newcommand{\URI}{\resizebox{!}{7pt}{\AcronymText{URI}}}
\newcommand{\DTD}{\resizebox{!}{7pt}{\AcronymText{DTD}}}

\newcommand{\API}{\resizebox{!}{7pt}{\AcronymText{API}}}

\newcommand{\JATS}{\resizebox{!}{7pt}{\AcronymText{JATS}}}


\newcommand{\SDI}{\resizebox{!}{7pt}{\AcronymText{SDI}}}
\newcommand{\SDIV}{\resizebox{!}{7pt}{\AcronymText{SDIV}}}

\definecolor{atColor}{RGB}{50, 22, 40}
\newcommand{\ATextClr}[1]{\textcolor{atColor}{\textbf{#1}}}

\newcommand{\DgDb}{\makebox{\raisebox{-3pt}{\resizebox{!}{11pt}{\ATextClr{%
\rotatebox{17}{$\varsigma$}}}}\hspace{-4pt}%
\resizebox{!}{6.5pt}{\ATextClr{D\hspace{-2pt}B}}}}

\newcommand{\lDgDb}{\makebox{\raisebox{-3pt}{%
\resizebox{!}{12pt}{\ATextClr{%
\rotatebox{17}{$\varsigma$}}}}\hspace{-4pt}%
\resizebox{!}{6.5pt}{\ATextClr{D\hspace{-2pt}B}}}}


\newcommand{\IDE}{\resizebox{!}{7pt}{\AcronymText{IDE}}}


\newcommand{\ViSion}{\resizebox{!}{7pt}{\AcronymText{ViSion}}}

\newcommand{\CWL}{\resizebox{!}{7pt}{\AcronymText{CWL}}}

\newcommand{\ThreeD}{\resizebox{!}{7pt}{\AcronymText{3D}}}
\newcommand{\TwoD}{\resizebox{!}{7pt}{\AcronymText{2D}}}

\newcommand{\medInria}{\resizebox{!}{7pt}{\AcronymText{medInria}}}
\newcommand{\ThreeDimViewer}{\resizebox{!}{7pt}{\AcronymText{3DimViewer}}}

\newcommand{\FAIR}{\resizebox{!}{7pt}{\AcronymText{FAIR}}}

\newcommand{\QNetworkManager}{\resizebox{!}{7pt}{\AcronymText{QNetworkManager}}}
\newcommand{\QTextDocument}{\resizebox{!}{7pt}{\AcronymText{QTextDocument}}}
\newcommand{\QWebEngineView}{\resizebox{!}{7pt}{\AcronymText{QWebEngineView}}}
\newcommand{\HTTP}{\resizebox{!}{7pt}{\AcronymText{HTTP}}}


\newcommand{\lAcronymTextNC}[2]{{\fontfamily{fvs}\selectfont {\Large{#1}}{\large{#2}}}}

\newcommand{\AcronymTextNC}[1]{{\fontfamily{fvs}\selectfont {\large #1}}}


\colorlet{orr}{orange!60!red}

\newcommand{\textscc}[1]{{\color{orr!35!black}{{%
						\fontfamily{Cabin-TLF}\fontseries{b}\selectfont{\textsc{\scriptsize{#1}}}}}}}


\newcommand{\textsccserif}[1]{{\color{orr!35!black}{{%
				\scriptsize{\textbf{#1}}}}}}


\newcommand{\iXPDF}{\resizebox{!}{7pt}{\textsccserif{%
\textit{XPDF}}}}

\newcommand{\iEPF}{\resizebox{!}{7pt}{\textsccserif{%
\textit{ETSpf}}}}

\newcommand{\iSDI}{\resizebox{!}{7pt}{\textsccserif{%
\textit{SDI}}}}

\newcommand{\iHTXN}{\resizebox{!}{7pt}{\textsccserif{%
\textit{HTXN}}}}


\newcommand{\AcronymText}[1]{{\textscc{#1}}}

\newcommand{\AcronymTextser}[1]{{\textsccserif{#1}}}


\newcommand{\mAcronymText}[1]{{\textscc{\normalsize{#1}}}}

\newcommand{\FASTA}{{\resizebox{!}{7pt}{\AcronymText{FASTA}}}}
\newcommand{\SRA}{{\resizebox{!}{7pt}{\AcronymText{SRA}}}}
\newcommand{\DNA}{{\resizebox{!}{7pt}{\AcronymText{DNA}}}}
\newcommand{\MAP}{{\resizebox{!}{7pt}{\AcronymText{MAP}}}}
\newcommand{\EPS}{{\resizebox{!}{7pt}{\AcronymText{EPS}}}}
\newcommand{\CSV}{{\resizebox{!}{7pt}{\AcronymText{CSV}}}}
\newcommand{\PDB}{{\resizebox{!}{7pt}{\AcronymText{PDB}}}}


\newcommand{\OBO}{{\resizebox{!}{7pt}{\AcronymText{OBO}}}}

\newcommand{\XOCS}{{\resizebox{!}{7pt}{\AcronymText{XOCS}}}}

\newcommand{\ChemXML}{{\resizebox{!}{7pt}{\AcronymText{ChemXML}}}}

\newcommand{\TeXMECS}{\resizebox{!}{7pt}{\AcronymText{TeXMECS}}}

% pmml  arff  openannotation

\newcommand{\PMML}{\resizebox{!}{7pt}{\AcronymText{PMML}}}
\newcommand{\ARFF}{\resizebox{!}{7pt}{\AcronymText{ARFF}}}
\newcommand{\IeXML}{\resizebox{!}{7pt}{\AcronymText{IeXML}}}


\newcommand{\WebGL}{\resizebox{!}{7pt}{\AcronymText{WebGL}}}


\newcommand{\Cpp}{\resizebox{!}{7pt}{\AcronymText{C++}}}

%\newcommand{\\WhiteDB{}}{\resizebox{!}{7pt}{\AcronymText{\WhiteDB{}}}}

\colorlet{drp}{darkRed!70!purple}

%\newcommand{\MOSAIC}{{\color{drp}{\AcronymTextNC{\scriptsize{MOSAIC}}}}}

\newcommand{\MOSAIC}{\resizebox{!}{7pt}{\AcronymText{MOSAIC}}}


\newcommand{\mMOSAIC}{{\color{drp}{\AcronymTextNC{\normalsize{MOSAIC}}}}}

\newcommand{\MOSAICVM}{\mMOSAIC-\mAcronymText{VM}}

\newcommand{\sMOSAICVM}{\resizebox{!}{7pt}{\MOSAICVM}}
\newcommand{\sMOSAIC}{\resizebox{!}{7pt}{\MOSAIC}}

\newcommand{\LDOM}{\resizebox{!}{7pt}{\AcronymText{LDOM}}}
\newcommand{\Cnineteen}{\resizebox{!}{7pt}{\AcronymText{CORD-19}}}

\newcommand{\lCnineteen}{\resizebox{!}{7.5pt}{\AcronymText{CORD-19}}}


\newcommand{\MOL}{\resizebox{!}{7pt}{\AcronymText{MOL}}}

\newcommand{\ACL}{\resizebox{!}{7pt}{\AcronymText{ACL}}}

\newcommand{\LXCR}{\resizebox{!}{7pt}{\AcronymText{LXCR}}}
\newcommand{\lLXCR}{\resizebox{!}{8.5pt}{\AcronymText{LXCR}}}
\newcommand{\lsLXCR}{\resizebox{!}{9.5pt}{\AcronymText{LXCR}}}

%\newcommand{\lMOSAIC}{{\color{drp}{\lAcronymTextNC{M}{OSAIC}}}}
\newcommand{\lfMOSAIC}{\resizebox{!}{9pt}{{\color{drp}{\lAcronymTextNC{M}{OSAIC}}}}}

\newcommand{\Mosaic}{\resizebox{!}{7pt}{\MOSAIC}}
\newcommand{\MosaicPortal}{{\color{drp}{\AcronymTextNC{MOSAIC Portal}}}}

\newcommand{\RnD}{\resizebox{!}{7pt}{\AcronymText{R\&D}}}

\newcommand{\lQt}{\resizebox{!}{8.5pt}{\AcronymText{Qt}}}
\newcommand{\QtCpp}{\resizebox{!}{8.5pt}{\AcronymText{Qt/C++}}}
\newcommand{\Qt}{\resizebox{!}{7pt}{\AcronymText{Qt}}}

\newcommand{\QtSQL}{\resizebox{!}{7pt}{\AcronymText{QtSQL}}}

\newcommand{\HTML}{\resizebox{!}{7pt}{\AcronymText{HTML}}}
\newcommand{\PDF}{\resizebox{!}{7pt}{\AcronymText{PDF}}}

\newcommand{\R}{\resizebox{!}{7pt}{\AcronymText{R}}}
\newcommand{\SciXML}{\resizebox{!}{7pt}{\AcronymText{SciXML}}}



\newcommand{\lGRE}{\resizebox{!}{7.5pt}{\AcronymText{GRE}}}

\newcommand{\p}[1]{

\vspace{1em}#1}

\newcommand{\q}[1]{{\fontfamily{qcr}\selectfont ``}#1{\fontfamily{qcr}\selectfont ''}} 

%\newcommand{\deconum}[1]{{\textcircled{#1}}}

\renewcommand{\thesection}{\protect\hspace{-1.5em}}
%\renewcommand{\thesection}{\protect\mbox{\deconum{\Roman{section}}}}
\renewcommand{\thesubsection}{\arabic{section}.\arabic{subsection}}

\newcommand{\llMOSAIC}{\mbox{{\LARGE MOSAIC}}}
%\newcommand{\lfMOSAIC}{\mbox{M\small{OSAIC}}}

\newcommand{\llMosaic}{\llMOSAIC}
\newcommand{\lMosaic}{\lMOSAIC}
\newcommand{\lfMosaic}{\lfMOSAIC}

%\newcommand{\dsC}{}

\newcommand{\textds}[1]{{\fontfamily{lmdh}\selectfont{%
\raisebox{-1pt}{#1}}}}

\newcommand{\ltextds}[1]{{\fontfamily{lmdh}\fontsize{12}{11}\selectfont{%
\raisebox{-1pt}{#1}}}}

\newcommand{\dsC}{{\textds{ds}{\fontfamily{qhv}\selectfont \raisebox{-1pt}{C}}}}
\newcommand{\ldsC}{{\textds{ds}{\fontfamily{qhv}\selectfont \raisebox{-1pt}{C}}}}

\newcommand{\llWC}{\mbox{{\LARGE WhiteCharmDB}}}

\newcommand{\llwh}{\mbox{{\LARGE White}}}
\newcommand{\llch}{\mbox{{\LARGE CharmDB}}}

\usepackage{enumitem}
%\usepackage{listings}

\colorlet{dsl}{purple!20!brown}
\colorlet{dslr}{dsl!50!blue}

\setlist[description]{%
  topsep=11pt,
  labelsep=22pt, leftmargin=10pt,
  itemsep=13pt,               % space between items
  %font={\bfseries\sffamily}, % set the label font
  font=\normalfont\bfseries\color{dslr!50!black}, % if colour is needed
}

\setlist[enumerate]{%
  topsep=3pt,               % space before start / after end of list
  itemsep=-2pt,               % space between items
  font={\bfseries\sffamily}, % set the label font
%  font={\bfseries\sffamily\color{red}}, % if colour is needed
}

%\usepackage{tcolorbox}

\newcommand{\slead}[1]{%
\noindent{\raisebox{2pt}{\relscale{1.15}{{{%
\fcolorbox{logoCyan!50!black}{logoGreen!5}{#1}
}}}}}\hspace{.5em}}


\let\OldLaTeX\LaTeX

\renewcommand{\LaTeX}{\resizebox{!}{7pt}{\color{orr!35!black}{\OldLaTeX}}}

\let\OldTeX\TeX

\renewcommand{\TeX}{\resizebox{!}{7pt}{\color{orr!35!black}{\OldTeX}}}


\newcommand{\LargeLaTeX}{\resizebox{!}{8.5pt}{\color{orr!35!black}{\OldLaTeX}}}

%\setlength\parindent{0pt}
\setlength\parindent{24pt}
%%\usepakage{newfile}

\usepackage{hyperref}

\usepackage{etoolbox}

\usepackage{zref-user}

\newwrite\sdiFile
\immediate\openout\sdiFile=\jobname.sdi.txt

\newcommand{\p}[1]{

\vspace{10pt}#1}

\newif\iftabng
\tabngfalse


\usepackage{letltxmacro}
\LetLtxMacro{\oldmmsemi}{\;}
\LetLtxMacro{\oldtbplus}{\+}
\LetLtxMacro{\oldtbgt}{\>}
\LetLtxMacro{\oldmmgt}{\+}

\newcommand{\+}{\iftabng\oldtbplus\else\sss\fi}

\renewcommand{\>}{\iftabng\oldtbplus\else
\ifmmode\oldmmgt\else\sse\sss\fi\fi}

%\renewcommand{\>}{\sse\sss}

\renewcommand{\;}{\relax\ifmmode\oldmmsemi\else\sse\fi}

\newcommand{\writeSDI}[1]{\immediate\write\sdiFile#1}

\newcommand{\emblink}[2]{\href{\#sdi:#1--#2}{\#sdi:#1--#2}}

%\newcount\sdiCounter
%\def\advsdiCounter{\global\advance\sdiCounter by1}

%\newcount\sdiCounterP
%\def\advsdiCounterP{\global\advance\sdiCounterP by1}

%\newcounter{sdiCounter}
\newcounter{sdiCounterP}[page]
\newcounter{sdiCounter}

\def\topt#1{\expandafter\the\dimexpr\dimexpr#1sp\relax\relax}

\makeatletter
%\catcode`\*=10
\newcommand{\sss}{%
\stepcounter{sdiCounterP}
\stepcounter{sdiCounter}
\pdfsavepos\write\sdiFile{!/ SDI_Sentence_Start} 
\write\sdiFile\expandafter{\expandafter$%
\expandafter i\expandafter:%
\expandafter\space\the\c@sdiCounter}
\write\sdiFile\expandafter{\expandafter$%
\expandafter o\expandafter:%
\expandafter\space\the\c@sdiCounterP}
\write\sdiFile\expandafter{\expandafter$%
\expandafter p\expandafter:%
\expandafter\space\thepage^^J%
$x: \topt\pdflastxpos^^J%
$y: \topt\pdflastypos^^J%
/!^^J%
<<>^^J%
}}
%\catcode`\%=14
\makeatother

\makeatletter
\newcommand{\sse}{%
\pdfsavepos\write\sdiFile{!/ SDI_Sentence_End} 
\write\sdiFile\expandafter{\expandafter$%
\expandafter i\expandafter:%
\expandafter\space\the\c@sdiCounter}
\write\sdiFile\expandafter{\expandafter$%
\expandafter o\expandafter:%
\expandafter\space\the\c@sdiCounterP}
\write\sdiFile\expandafter{\expandafter$%
\expandafter p\expandafter:%
\expandafter\space\thepage^^J%
$x: \topt\pdflastxpos^^J%
$y: \topt\pdflastypos^^J%
/!^^J%
<<>^^J%
}}
\makeatother




\newcommand{\lun}[1]{\raisebox{-4pt}{\fontfamily{qcr}\selectfont{%
\LARGE{\textbf{\textcolor{tcolor}{#1}}}}}\vspace{-2pt}}

\newcommand{\inditem}{\itemindent10pt\item}

\usepackage{soul}

\definecolor{hlcolor}{RGB}{114, 54, 203}
\colorlet{hlcol}{hlcolor!35}
\sethlcolor{hlcol}

\makeatletter
\def\SOUL@hlpreamble{%
	\setul{}{3ex}%         !!!change this value!!! default is 2.5ex
	\let\SOUL@stcolor\SOUL@hlcolor
	\SOUL@stpreamble
}
\makeatother

\usepackage{scrextend}
%\vspace*{3em}
\newenvironment{mldescription}{\vspace{1em}%
  \begin{addmargin}[4pt]{1em}
    \setlength{\parindent}{-1em}%
    \newcommand*{\mlitem}[1][]{\vspace{5pt}\par\medskip%
%\colorbox{hlcolor}{\textbf{##1}}\quad}\indent
\hl{ \textbf{##1} }\quad}\indent
}{%
  \end{addmargin}
  \medskip
}

\usepackage{marginnote}

\newcommand{\mnote}[1]{%
\vspace*{-2em}
\reversemarginpar
\raisebox{1em}{\marginnote{\parbox{4em}{%
\begin{mdframed}[innerleftmargin=4pt,
	innerrightmargin=1pt,innertopmargin=1pt,
	linecolor=red!20!cyan,userdefinedwidth=4em,
	topline=false,
	rightline=false]
{{\fontfamily{ppl}\fontsize{12}{0}\selectfont
		\textit{#1}}}
\end{mdframed}}
	}[3em]}}

\newcommand{\mnotel}[1]{%
\vspace*{-2em}
\reversemarginpar
\raisebox{-4em}{\marginnote{\parbox{4em}{%
\begin{mdframed}[innerleftmargin=4pt,
	innerrightmargin=1pt,innertopmargin=1pt,
	linecolor=red!20!cyan,userdefinedwidth=4em,
	topline=false,
	rightline=false]
{{\fontfamily{ppl}\fontsize{12}{0}\selectfont
		\textit{#1}}}
\end{mdframed}}
	}[3em]}}

\newcommand{\mnoteh}[3]{%
	\vspace*{#1}
	\reversemarginpar
	\raisebox{#2}{\marginnote{\parbox{4em}{%
				\begin{mdframed}[innerleftmargin=4pt,
					innerrightmargin=1pt,innertopmargin=1pt,
					linecolor=red!20!cyan,userdefinedwidth=4em,
					topline=false,
					rightline=false]
					{{\fontfamily{ppl}\fontsize{12}{0}\selectfont
							\textit{#3}}}
				\end{mdframed}}
			}[3em]}}


\newcommand{\mnoteb}[1]{%
	\vspace*{1em}
	\reversemarginpar
	\raisebox{1em}{\marginnote{\parbox{4em}{%
				\begin{mdframed}[innerleftmargin=4pt,
					innerrightmargin=1pt,innertopmargin=1pt,
					linecolor=red!20!cyan,userdefinedwidth=4em,
					topline=false,
					rightline=false]
					{{\fontfamily{ppl}\fontsize{12}{0}\selectfont
							\textit{#1}}}
				\end{mdframed}}
			}[3em]}}
	
\usepackage{wrapfig}

\usetikzlibrary{arrows, decorations.markings}
\usetikzlibrary{shapes.arrows}

\newcommand{\curicon}[2]{%
	\node at (#1,#2) [
	draw=black,
	%minimum width=2ex,
	inner sep=.7pt,
	fill=white,
	single arrow,
	single arrow head extend=3pt,
	single arrow head indent=1.5pt,
	single arrow tip angle=45,
	line join=bevel,
	minimum height=4.6mm,
	rotate=115
	] {};
}

\makeatletter
\def\@cite#1#2{[\textbf{#1\if@tempswa , #2\fi}]}
\def\@biblabel#1{[\textbf{#1}]}
\makeatother


%\let\origref\ref
%\renewcommand{\ref}[1]{{\LARGE #1}}

%\def\ref#1{\textbf{\origref{{\LARGE #1}}}}

\setlength{\footnotesep}{0pt}

\renewcommand{\thefootnote}{\textcolor{logoGreen!80!logoBlue}{{\fontfamily{qcr}\fontseries{b}\fontsize{10}{4}\selectfont\arabic{footnote}}}}


\newcommand{\LVee}{{\colorbox{cyan!40!yellow}{\textcolor{red!70!navy}{\textbf{\LARGE$\vee$}}}}}
\newcommand{\LWedge}{{\colorbox{cyan!40!yellow}{\textcolor{red!70!navy}{\textbf{\LARGE$\wedge$}}}}}

\renewcommand{\LVee}{}
\renewcommand{\LWedge}{}


\urlstyle{same}

\usepackage[preserveurlmacro]{breakurl}

\newcommand{\bhref}[1]{\href{#1}{\burl{#1}}}

\newcommand{\dhref}[1]{\href{#1}{\url{#1}}}

%\setmainfont{QTChanceryType}

\begin{document}

\setlength{\skip\footins}{18pt}	
	
{\linespread{1.25}\selectfont

\vspace*{1.5em}

\begin{center}
%{\relscale{1.2}{\fontfamily{qcr}\fontseries{b}\selectfont 
%{\colorbox{black}{\color{blue}{\llWC{} Database Engine \\and 
%\llMOSAIC{} Native Application Toolkit}}}}}

\colorlet{ctmp}{logoPeach!20!gray}
\colorlet{ctmpp}{ctmp!90!yellow}
\colorlet{ctmppp}{ctmpp!50!black}
\colorlet{ctmpppp}{ctmppp!90!logoRed}
\colorlet{ctmcyan}{ctmpppp!70!cyan}

\colorlet{ctmppppy}{ctmppp!60!orange}




%{\colorbox{darkBlGreen!30!darkRed}{%
\begin{tcolorbox}
[
%%enhanced,
%%frame hidden,
%interior hidden
arc=2pt,outer arc=0pt,
enhanced jigsaw,
width=\textwidth,
colback=ctmppppy!40,
%colback=ctmcyan!50,
colframe=logoRed!30!darkRed,
drop shadow=logoPurple!50!darkRed,
%boxsep=0pt,
%left=0pt,
%right=0pt,
%top=2pt,
]
%\hspace{22pt}
\begin{minipage}{\textwidth}	
\begin{center}	
{\setlength{\fboxsep}{32pt}
	\relscale{1.2}{{\fontfamily{qcr}\fontseries{b}\selectfont%
{The MOSAIC Data-Set Explorer (\lsMdsX{}): Initial 
Developer's Overview}
}}}
\end{center}
\end{minipage}
\end{tcolorbox}
\end{center}

\p{The \MOSAIC{} Data-Set Explorer (\MdsX{}) is a suite of 
code libraries which can be used to 
build native, desktop-style applications 
for viewing data sets.  An \MdsX{} 
\q{data-set application} is an application 
customized and tailored to a particular 
data set, or a repository including multiple 
data sets.  An \MdsX{} \textit{notebook} is 
a particular strategy for organizing 
data-set applications, or parts of 
data-set applications.  In addition to 
serving as standalone applications in themselves, 
\MdsX{} notebooks can be embedded in other 
applications; for instance, in scientific software.}

\p{\lMdsX{} is paired with \MOSAIC{} \textit{portal}, 
a code library for hosting data sets and 
publications.  The \MOSAIC{} portal code includes 
custom \LaTeX{} commands for building annotated, 
indexed \PDF{} files, and a custom \PDF{} viewer 
which can read \MOSAIC{} annotations and 
utilize their information to interoperate 
with data-set applications.  \lMOSAIC{} data-set 
applications can also customize this \PDF{} 
viewer to add functionality specific to 
its data models and scientific subject-matter.}

\p{In general, \MOSAIC{} applications are 
designed to be distributed in source-code fashion.  
They are, by default, written in \Cpp{} and based on 
the \Qt{} application-development framework.  
\lMOSAIC{} is designed so that sophisticated 
data-set applications can be built with few 
(or no) external dependencies apart from \Qt{} 
itself.  In the typical scenario, users would 
build and run \MdsX{} applications inside 
the \Qt{} Creator Integrated Development Environment 
(\IDE{}).  However, \MdsX{} applications can 
also be configured so that they (or some 
functionality they provide) can be run 
from a command line --- which allows them to 
participate in multi-application workflows --- 
or bundled as plugins or source-code extensions 
into larger software components.}

\p{The \MOSAIC{} portal is developed by 
Linguistic Technology Systems (LTS), who 
can develop and/or host data/publication 
repositories across disciplines (see 
contact details at the end of this 
paper for more information).  One 
repository under development with 
\MOSAIC{} is the \q{Cross-Disciplinary 
Repository for Covid-19 Research} 
(\CRtwo{}), a collection of data sets 
related to Covid-19, that is paired with a 
forthcoming Elsevier volume, 
\textit{Cross-Disciplinary Data Integration 
and Conceptual Space Models for Covid-19}. 
In addition to \MOSAIC{}, LTS also provides a 
Dataset Creator (\dsC{}), which is a 
\Qt Creator plugin helping researchers 
and programmers curate data sets and 
implement data-set applications.}

\p{\lMOSAIC{} is designed to bridge the 
gap between scientific software and 
scientific data sets.  While increasing 
volumes of open-access research data is 
becoming available to readers, researchers, 
and scientists, this data is not always 
published in a manner which facilitates 
reuse and interoperability with scientific 
software --- the kinds of applications 
that scientists themselves use to conduct 
and examine experiments or simulations.  
Moreover, the software-development ecosystem 
which is evolving around data publishing 
(as far as exchange protocols, file formats, 
development tools, and so forth) is  
methodologically removed from the engineering 
norms and principles of most scientific 
software.  As such, a technical gap 
exists between the data-publishing and 
scientific-computing ecosystems seen 
as software-engineering domains.  
\lMOSAIC{} aims to be a suite of tools 
which can help bridge that gap.}


\section{Data Sets and Data Publishing: the current picture}
\p{Recent years have seen an increasing emphasis, 
in the academic and scientific worlds, on 
\textit{data publishing} --- sharing research 
data and experimental results/protocols via 
web portals complementing those that host 
scientific papers.  Published data sets 
now take a position alongside books and articles 
as primary publicly-accessible outputs of 
scientific projects.  Coinciding with this 
increased volume of raw data, there has 
also emerged an ecosystem of tools allowing 
researchers to find, view, explore, and 
reuse data sets.  These tools enhance the 
value of published data, because they decrease 
the amount of effort which scientists need 
to make use of data sets in productive ways.}

\p{Unfortunately, however, this ecosystem 
of tools does not include extensive work 
on software \textit{applications} for 
accessing and using published data sets.  
Prominent publishes (Elsevier, Springer, Wiley, 
de Gruyter, etc.) have all developed 
suites of components for manipulating data 
sets and data/code repositories, including 
\API{}s, search portals, Semantic Web ontologies
and other forms of Controlled Vocabularies, and 
cloud-based computing or visualization engines, 
founded on technologies such as Jupyter, Docker, 
and \WebGL{}.  However, none of these 
publishers actually provide \textit{applications} 
for accessing data sets outside of the 
online resources where data sets are indexed.  
While these online portals can provide a 
basic overview of the data sets, publishers 
do not provide tools to help researchers 
rigorously use any data sets once they 
are downloaded.  Moreover, the ecosystem 
for manipulating published research is largely 
disconnected from the software  
applications which scientists actually 
use to do research.  The ability to work 
with data-publishing tools has not been 
implemented within most 
scientific-computing environments.}

\p{These lacunae may be explained in part by publishers' 
and scientists' hopes of creating cloud-hosted 
environments that can themselves serve as fully featured 
scientific-computing frameworks, with the ability 
to run code, evaluate queries, interactively 
display \TwoD{} and \ThreeD{} graphics, and 
maintain user and session state so that 
researchers can suspend and resume their work 
at different times.  In these cloud environments, 
users can run computations and generate complex 
graphics on remote processing units, with relatively 
little data or code-execution stored or performed 
on their own computers.  Such employment of remote, 
virtual programming environments is sometimes necessary 
when interacting with extremely large data repositories; 
and can be a convenient way to explore data sets 
in general, especially if a user is unsure whether 
or not a given data set is in fact germane to 
their research.  Investigating data via cloud 
services spares the researcher from having to 
download the data set directly (along with 
the additional software and requirements which 
are often needed to make downloaded data functionally 
accessible).  However, cloud-based data access is 
limited in important ways, which makes relying 
solely on cloud services to provide the 
filaments of a research-data ecosystm a very bad 
idea.  The first problem is that cloud services 
are, despite their technical features, essentially 
just web applications under the hood; as such, 
they are susceptible to the same User Expericne 
degradation as any other web service --- subpar 
performance due to network latency, poor connectivity, 
and the simple fact that web-based graphics can 
never be as responsive or as compelling as 
desktop software, which can interact directly 
with the local operating system and react 
instantaneously to user actions.  The second, 
more serious problem is that could-computing 
environments are computationally and 
architecturally different than the native-application 
contexts where scientific software usually 
operates.  Insofar as researchers develop 
new analytic techniques, implement new algorithms, 
or write custom code to process the data 
generated by a new experiment, these 
computational resources are usually 
formulated in a local-processing environment that 
cannot be translated, without extra 
effort, to the cloud.}

\p{To be sure, scientists can sometimes 
\q{package} their experimental and analytic 
methods into a coherent framework, such as a 
Jupyter notebook, which serves as both a 
demonstration and a precis of their 
research work.  Indeed, tools such 
as Jupyter (which packages code, data, 
and graphics into a self-contained Python-based 
programming environment) are useful in part 
because the content shared via these 
systems (e.g. Jupyter \q{notebooks}) needs 
to be deliberately curated; building a 
notebook is a kind of summarial follow-up to 
actual research work.  The intellectual discipline 
involved in packaging up one's research via 
such tools may be a valuable stage in the 
scientific process, but even then the programming 
environment where research code and data 
is publicly shared is fundamentally different 
than the environment where the research is 
actually carried out.  As a consequence, 
sharing research indirectly via cloud 
services and or \q{notebook}-oriented 
frameworks like Jupyter is not really 
conducive to either reuse or replication.  
To actually replicate a course of 
investigation, it is more thorough 
to employ the same (or at least 
functionally equivalent) software for 
data acquisition, analysis, and validation 
as the original software; and to incorporate 
published data in new projects, the data 
should be shared in such a way that 
the original research data, code, and 
protocols can be absorbed into a new 
research context, including the software 
used by the research team.  Cloud-based 
services, which provide only an overview 
of research data, with limited analytic 
and imaging/visualization functionality 
compared to actual scientific software, 
do not substantially promote data 
replication and reuse insofar as 
these cloud services are functionally 
disconnected from scientific applications 
themselves.}

\p{This is the motivation behind x, 
a \textit{native,} \textit{desktop-style} application 
for accessing research data of different 
kinds --- within the overall space of 
published data sets we can find specific 
variations, such as \textit{data repositories} 
comprising multiple data sets; \textit{image corpora} 
designed as test beds for Machine Vision and 
diagnostic-imaging methods; \textit{simulations} 
which involve not only raw data but digital 
experiments that can be re-run as a way 
to access the data; and so forth.  Each of 
these various kinds of data sets present 
different sorts of interactive specifications 
which must be implemented by the data-set explorer 
software.  While executed as a native application 
--- not a cloud service --- \MdsX{} nevertheless incoporates 
the important ideas from contemporary data publishing 
(including ideas originating in the cloud context): 
workflow models, notebooks, access to publishers' 
\API{}s, etc.  Another feature of \MdsX{} is that it 
can be run as a standalone application \textit{or} 
embedded in other applications --- such as 
the software which researchers are already 
using.  In short, \MdsX{} can be seen as akin to 
a cloud-based data-publishing platform 
where the \q{cloud} is replaced by a 
scientific-computing application.  Instead of 
being hosted remotely (\q{on the cloud}), 
\MdsX{} is hosted within a local desktop application.  
This host application may be pre-existing 
program, or a custom host implemented 
to allow \MdsX{} data-sets to be explored in standalone 
fashion (with a default implementation that 
can, as desired, be modified for individual 
data sets/repositories).}


\section[The Structure of MdsX Notebooks]{The Structure of \protect\lsMdsX{} Notebooks}

\p{A common feature of software through which users 
study and reuse research data sets is some 
form of \q{interactive notebooks,} or 
digital resources combining data, code, 
and graphics/visuals.  The main feature of 
notebook-oriented design is the idea of 
interactive code editing, where changes in 
the code directly leads to changes in a visual 
display (such as a plot or diagram) which is 
viewed alongside the code.  This setup allows 
developers to present or demonstrate data 
sets, and associated code, in an exploratory 
and interactive manner.}

\p{The exact details of how \q{notebooks} are designed 
and implemented varies between different technologies, 
although the concept is most clearly associated 
with \Jupyter{}, which is a coding and presentation 
environment based on \Python{}.  Whatever the 
underlying programming environment, notebooks 
--- or as \MdsX{} uses the term, \q{interactive/digital 
notebooks} (\IDN{}s) --- have several software-engineering 
requirements, including a scripting environment and 
a data-visualization layer, wherein data sets 
or numeric models are transformed into \TwoD{} or 
\ThreeD{} graphics (charts, diagrams, etc.).  
Moreover, the scripting layer needs to be 
connected to the data-visualization layer so 
that scripts can modify the data-to-graphics 
transformations.  A further requirement is 
functionality to load pre-existing data sets from 
saved files or from a web resource.}

\p{Beyond these general features, \IDN{} 
programming can take different forms and 
prioritize different styles of user 
interaction.  The \MdsX{} approach recognizes 
that it is often more convenient to 
interact with applications through 
\GUI{} actions --- buttons, tabs, context 
menus, and so forth --- than by typing 
in commands (whether or not these are 
executed immediately in \REPL{}, or 
\q{read-eval-print-loop}, fashion, or are 
stored in scripts).  As such, \IDN{}s 
should not differ in design too noticeably 
from conventional \GUI{} windows or dialog 
boxes --- they should not be little more 
than \q{\REPL{}s with plots.}  On the other 
hand, rigorous \GUI{} programming 
calls for a carefully organized 
set of mappings from potential user 
actions to application responses.  
Whether on the scripting level or 
the \GUI{} coding, in short, 
implementations need a level of abstraction 
more general than the underlying event-handling 
and procedure-calling logic which forms 
the application's concrete operational 
behavior.  This semi-abstracted layer 
can be described in terms of \q{meta-classes,} 
\q{meta-objects,} \q{tools,} \q{transitions,} 
\q{services,} and so forth: the common 
denominator in different contexts is some 
notion of a structure which can be called 
a \q{meta-procedure,} similar to an ordinary 
computational procedure in having inputs and 
outputs, but embodying a level of abstraction 
somewhat removed from concrete procedures.  
In particular, meta-procedures are not 
directly implemented; instead, some algorithm 
is necessary to determine, given a description 
of a meta-procedure with its outputs and 
context, what concrete procedure (or set of 
procedures) should actually run.  Moreover, 
meta-procedures need some notion of delayed 
execution: there is a logical gap between 
\q{marking} (using the language of petri-net 
theory), i.e., fully specifying the input 
parameters consumed by a meta-procedure, and a 
meta-procedure's actual execution.  
As such, meta-procedural markings can be 
built up in stages, with input data coming 
from multiple sources (including scripts and 
\GUI{} elements).  For a concrete example, 
consider the process of filling out a web 
form, wherein entries typed in to the form 
fields are validated, one at a time, 
before the form can be submitted.  
In these cases, the step-by-step process of 
entering and validating individual fields 
corresponds to incremental marking, and 
\q{hitting the submit button} corresponds 
to meta-procedure execution.}

\p{In short --- although different systems 
use different terminology --- any \IDN{} 
programming environment needs a mechanism 
to incrementally define and execute 
meta-procedure calls.  The 
implementational foundations of that 
mechanism (hypergraphs, workflow engines, 
state monads, etc.) depend on the underlying 
programming environment.  The \MdsX{} approach 
borrows ideas primarily from HyperGraphDB and 
SeCo, which is a notebook-programming environment 
based on HyperGraphDB.  As in SeCo, units of 
marking and execution are called \textit{cells}.  
The main difference between \MdsX{} and SeCo (apart 
from \Cpp{} instead of \Java{} being 
the underlying programming language) is 
that \MdsX{} cells are not intended, in the general 
case, to be typed in by programmers directly.  
Instead, \MdsX{} cells are normally constructed 
behind the scenes, on the basis 
of \GUI{} component state, user actions, 
or scripting input.  However, once 
constructed, they can be manipulated 
like SeCo cells, both in terms of 
functionality and in terms of rationale: 
they can be used as a log of user 
actions, for undo/redo, for defining 
workflows, for generating scripts, 
and so on.  In particular, the 
mappings from \GUI{} actions to 
application handlers can be defined 
(and extended) by annotating the 
relevant \GUI{} elements with 
meta-procedure cells. This also 
allows data sets to be 
annotated with micro-citations 
(which are discussed below).}

\p{As a \Cpp{} environment, \MdsX{} 
uses an embedded \q{virtual machine} 
to interpret meta-procedure cells; 
application-level event handlers are 
not automatically exposed to 
a scripting interface as they would be 
in a \JVM{} or \Python{} environment.  
However, \MdsX{} also supports scripting 
via a choice of languages, similar to 
SeCo.  The primary scripting language 
used with \MdsX{} is AngelScript, although 
other \C{}/\Cpp{} based languages 
(Embeddable Common Lisp, \ChaiScript{}, 
etc.) can work as well.  To support 
various scripting languages, modules 
loaded into \MdsX{} need to provide a 
meta-procedural interface declaration, 
and the desired scripting language also 
needs a bridge to work with these 
declarations (which is generally usable 
across all datasets and modules).  
Such a bridge will be provided by 
default for AngelScript and \ECL{} 
(Embeddable Common Lisp), and similar 
tools could be implemented for other languages.}

\p{The typical \MdsX{} notebook combines, at a 
minimum, some graphical element --- 
such as an image to be analyzed and/or 
a plot/diagram to be populated with 
data --- along with a user-interface 
\q{panel} for interacting with 
the graphics, and the overall application.  
This panel partially takes the place of 
a script-composition or \REPL{} frame, although 
such a frame is implicitly present, normally 
behind the scenes (users can view it 
if desired).  Notebooks can then 
load data files, and representations of 
the loaded data (e.g., text serializations) 
may then also become part of the notebook 
content, able to be visualized in their 
own frame.  Notebooks in general then can 
have four varieties of frames (graphics 
views, interaction panels, data panels, 
and meta-procedure logs) although not 
every available frame may be explicitly 
constructed and/or visible at a given point 
in the user's session.  There may also be 
multiple instances of graphics frames.  
In any case, the layout and state of 
these various frames --- what frames are 
visible, and their current content 
--- define notebook \textit{state} which 
can be saved, restored, and shared.  
Loading a data set into a \MdsX{} notebook 
therefore involves loading a particular 
initial state, defined as part of the 
data set, arranged in part to serve as 
a useful starting-point for users to 
explore and visualize the relevant data.  
Each of these kinds of frames corresponds 
to a particular aspect of software 
implementations, requiring its own 
strategies and paradigms.  The following 
sections will review these various 
programming concerns one at a time.}


\section{Image Analysis and Data Visualization}

\p{The central graphical element of an 
\MdsX{} notebook is either a \TwoD{} or \ThreeD{} 
image loaded from an image file (in formats 
such as \PNG{}, \JPG{}, \DICOM{}, \TIFF{}, 
etc.), or else a \TwoD{} or \ThreeD{} plot, 
chart, or diagram.  The functionality of 
the notebook will therefore differ depending 
on whether the central graphics is an image 
loaded from a file (called 
an \q{image-based} notebook), or a data visualization 
constructed from a data set or some 
mathematical formulae (called a \q{diagram-based 
notebook}).  A third option is an \q{object-based} 
notebook, whose central viewport contains a 
structured display of information, mostly in 
textual form (for instance, a table or tree view), 
but this section will focus on graphics-oriented 
notebooks (Figure~\ref{fig:oxy} shows a contrast 
between an object-based view, in the background, 
and a graphics-based view, which has been 
opened floating above it).}

\begin{figure}

\caption{A graphics-based view (foreground) and 
object-based view (background)}
\label{fig:oxy}

\begin{tikzpicture}

\node[inner sep=0pt] (x1) at (0,0)
    {\includegraphics[width=180mm, 
    	trim={0mm 0mm 0mm 0mm},clip]
    	{pics/oxy.png}};
    
\end{tikzpicture}   
\end{figure}



\p{Diagram-based \MdsX{} notebooks can be implemented with 
different diagram-plotting engines; the default 
implementations support \Qt{} Charts (a built-in \Qt{} 
module) as well as the qtcustomplot and 
JKQCustomPlotter libaries.  In short, diagram-based 
notebooks need to implement subclasses of 
\MdsX{} frames for a navigation panel, graphics view, 
meta-procedure view, and data view, as well as a 
\q{\MdsX{} diagram} subclass occupying the diagram view.  
In this case, the primary responsibility of 
the meta-procedural layer is to interface with 
functionality provided by the diagram/plotting 
engine.  Since most coding details are derived from 
these engine's object models and classes, they lie 
mostly outside the scope of this paper.}

\p{Image-based notebooks, on the other hand, 
need to integrate several different 
areas of functionality.  As such, setting 
up the image view is only one step in 
constructing such a notebook; additional 
programming is needed to support image annotation, 
analysis, and feature extraction.  
In order to integrate these different layers 
of functionality, \MdsX{} provides a 
\q{Data Structure Protocol for Image-Analysis Networking} 
(\DSPIN{}) which defines communication rules 
between image-related software subsystems 
in Object-Oriented terms.  \DSPIN{} 
objects are comprised of four 
more specific objects or layers, describing 
different aspects of the shared image and 
how it should be processed.  These 
four layers are defined as follows:

\begin{description} 

\item[Metadata Layer]  This object presents 
metadata describing the image format and acquisition.  
If the surrounding \DSPIN{} object represents an 
image series (rather than a single image), the 
metadata object should also declare the 
size of the collection and how individual 
images should be referenced.  The metadata 
should include a file path or resource identifier 
asserting where the image can be acquired from 
(which in the case of a series can be a zipped 
folder or a list of resource paths).  More 
specific metadata depends on the image or images' 
graphical format; to properly load images in 
most formats (such as \PNG{}, \JPEG{}, \TIFF{}, 
and \DICOM{}) applications need to specify 
detail such as dimensions, resolution, and 
color depth.  Of course, some of this information 
is stored internally within the image file 
(depending on its format), but certain formats 
require some metadata to be shared along with 
the image itself (moreover, it is often convenient 
to have basic information available without needing 
to extract it from binary image data).  The details 
on which form of metadata are appropriate for 
which image format can be determined based 
on image-viewing code libraries, such as 
\textbf{libpng}, \textbf{libtiff}, or 
\DICOM{} clients.  If both end-points of a 
\DSPIN{} communication have the same 
libraries installed, the sending application 
will have a clear idea of how much supplemental 
data is needed over and above what will be 
read from image files directly.  If there 
are uncertainties in library alignment between 
the two end-points, the sending application should 
consider serializing a more detailed summary of 
the image providing any information that would 
ordinarily be read from the image file. 
  
\item[Annotation Layer]
Almost all image analysis --- whether done by 
humans or by softare --- results in either 
some form of statistical representation of 
an image's properties, or a complex of 
data which presents information about 
(and may visually overlay) the image, 
particularly in the form of annotations.  
Image annotations are arrows, line segments, 
or \TwoD{} closed shapes that call attention 
to some point or region inside the image, 
usually with some additional label or commentary.  
The basis of each annotation is therefore 
some zero-dimensional or two-dimensional 
region (or a set of zero-dimensional 
control points; or, occasionally, a one-dimensonsal  
line or curve), so annotations require a mechanism 
for designating regions.  The same 
issues apply to asserting feature-vectors 
with respect to an image region rather than 
the image as a whole; accordingly, 
both annotations and feature vectors 
can be seen as equivalent varieties of 
constructions which isolate and then 
define data structures on zero-, one-, 
and/or two-dimensional subimages 
(feature vectors on the entire image 
can accordingly be treated as a special 
case).    

\item[Contextual Layer]
Contextual information associated with an 
image can include metadata 
or supplemental details that are not 
directly relevant to the image, but 
convey facts about how the image connects 
to a broader context where it was obtained, 
and for what purpose.  An example of 
contextual data would be the part of 
\DICOM{} headers that include patient 
or clinical information, rather than 
metadata about image format or 
dimensions.  


\item[Procedural Layer]

\end{description}
}


\p{In \MdsX{}, \DSPIN{} objects are 
associated with image-based notebooks in 
that the notebook components 
(the navigation panel and graphics, data, 
and meta-procedural controllers) jointly refer 
to a common \DSPIN{} object for all 
image-related data.  Image-analysis routines 
conducted within the notebook then may yield 
additional data structures bundled into the 
overarching \DSPIN{} object.  This 
overarching object can then be exported or 
saved, alongside (and as an extension of) 
notebook state.}

\p{Analytic operations available through an 
image-based notebook may be provided by 
the notebook itself, or by a host application 
where \MdsX{} is embedded.  In the latter 
case, the notebook needs to construct the 
proper calls to the host application, using 
the meta-procedural controller as a bridge 
to ambient capabilities.  For instance, if 
an \MdsX{} notebook is developed as a plugin 
to \CaPTk{}, the notebook would interface 
with the host \CaPTk{} application via the 
formats and programming constructs which 
\CaPTk{} recognizes (specifically, 
the Common Workflow Language and the 
\Qt{} signal/slot mechanism).  
This specific scenario --- embedding \MdsX{} 
in \CaPTk{} --- is employed as a demonstration 
and case-study for embedding notebooks 
in host application in general.  
The \CaPTk{} workflow protocol also forms a 
basis for the meta-procedural view and 
controllers, discussed next.}

\section[MdsX Meta-Procedure Controllers]{\protect\lsMdsX{} Meta-Procedure Controllers}

\p{The meta-procedural layer of an \MdsX{} notebook 
is responsible for handling events generated 
by a corresponding navigational panel, or at 
least those events which have a 
non-trivial impact on notebook/session 
state and data.  The visual representation of 
meta-procedural commands and history is 
provided by a meta-procedural \q{view,} which 
is normally invisible, but notebooks may 
choose to allow users to \q{unhide} this view.  
The meta-procedural controller is responsible 
for generating the meta-procedural view (if 
applicable) and responding to user events 
within this view; it is also responsible for 
maintaining an inventory of objects summarizing 
available metaprocedures, \GUI{} elements, 
and the mappings between them.}

\p{In general, the \GUI{} elements in these 
meta-procedural mappings are referred to as 
\q{visual objects,} and are represented 
in the meta-procedural controller context via 
application-unique identifiers (not raw pointers).  
Similarly, \q{meta-procedural objects} encapsulate 
information about meta-procedures themselves.  
This controller does not directly connect 
\GUI{} events to event handlers; instead, it 
receives information about these connections when 
the notebook is loaded.  The controller is however 
responsible for implementing \textit{incremental 
execution} wrapping event callbacks (or any other 
relevant procedure).  Incremental execution 
means that the controller may create 
temporary \q{execution contexts} and incrementally 
build up the data which, given a sufficiently 
complete \q{marking,} can lead to the 
meta-procedure being \q{fired.}  Each 
preliminary stage --- that is, each pre-firing 
addition to the execution context --- may in 
turn be generated by events originating elsewhere 
in the application (canonically, the 
navigation panel), and the meta-procedural controller 
should model both the history of these pre-firing stages 
and the origin and nature of the events which 
triggered them.  An execution context 
may then be \textit{reified}, representing the 
cumulative pre-firing stages as a data structure 
that can be matched to a meta-procedure's 
outcomes: for example, noting the 
inputs or steps producing the specific appearance 
of a diagram or image in the graphics view, 
or the parameters configured to instantiate 
a workflow model.  Reified meta-procedural 
execution contexts can then be shared as 
objects with components responsible for 
sharing or preserving information about the 
notebook.  For instance, a notebook graphic may be 
included in a publication; the reified context 
could then be associated with that image as an 
annotation, and used to reconstruct notebook 
state if the notebook is launched from a 
document viewer in the context of the 
published graphic.}

\p{In general, the information represented 
by the meta-procedural controller is not only 
relevant for the reactive operations of the 
notebook, responding to user actions; 
it also serves to document the notebook's 
properties, and potentially to connect 
the notebook with data sets and/or 
publications.  Many operations which can be 
performed within a notebook are associated 
with a given scientific or theoretical 
concept, or a statistical parameter modeled 
within a data set.  As such, it is possible 
for the notebook to maintain a list of these 
concepts, so as to create an interactive 
glossary or to interoperate with a document 
viewer.  For instance, Figure~\ref{fig:oxy} shows a 
context-menu action based on the concept 
of \q{oxygenated air flow,} which is also 
discussed in the scientific article 
on which the depicted data set is based.  
This concept also has a visual expression in 
one table column shown (in the background) on 
Figure~\ref{fig:oxy}.  As demonstrated 
in Figure~\ref{fig:about}, the data-set application includes 
code to explain technical concepts in pop-up 
dialog boxes, and also to link to the 
page/paragraph in the article where that correspinding 
concept is first (or most thoroughly) defined/mentioned.  
Establishing these conceptual connections 
between an \MdsX{} notebook, data set, and 
technical publication is facilitated by 
annotating both meta-procedural capabilities 
and \GUI{} elements with references to 
relevant technical/scientific concepts; these 
annotations, when defined, are represented through the 
meta-procedural controller.}
\begin{frame}{\ft{About}}

	\pdfpageheight 30cm

        \begin{annotatedFigure}
            {\includegraphics[scale=1]{about.png}}
            
  \node [text width=12cm,align=justify,fill=logoCyan!20, draw=logoBlue, 
  draw opacity=0.5,line width=1mm, fill opacity=0.9]
   at (0.58,0.76){\textbf{Context menus also allow users to 
   obtain information and explanations about individual parts of the 
   data set, such as individual statistical parameters.  In this 
   screenshot, the user has right-clicked on a data column and 
   chosen a context menu action which shows, via a dialog box, 
   a precis of the quantities represened in that column and their 
   significance for the data set as a whole.}};

            \annotatedFigureBox{0.2,0.12}{0.812,0.645}{1}{0.81,0.645}%            
      %      \annotatedFigureBox{0.222,0.284}{0.3743,0.4934}{B}{0.3743,0.4934}%tr
      %      \annotatedFigureBox{0.555,0.784}{0.6815,0.874}{C}{0.555,0.784}%bl
      %      \annotatedFigureBox{0.557,0.322}{0.8985,0.5269}{D}{0.8985,0.5269}%tr
  
        \end{annotatedFigure}

\end{frame}


\p{}

\section{DigammaDB and the HGXF Format}
\p{\lCRtwo{} will introduce a new database engine for 
preparing the information provided within the 
repository (called DigammaDB, or \DgDb{} for short) 
as well as the Hypergraph Exchange Format (\HGXF{}).  In 
general, \HGXF{} files can be generated from 
\DgDb{} instances, capturing the state of the 
database at a moment in time.  \lDgDb{} could therefore 
be used to store research data.  When scientists 
choose to publish data, they would then output the 
information from their database into \HGXF{}, using 
the resulting files as the published raw data.  
In the \CRtwo{} context, \DgDb{} will be used 
to merge data from multiple files into a single 
database, yielding \HGXF{} output forming most of the 
raw data republished in \CRtwo{}.}

\p{Operationally, \DgDb{} is designed to emulate the 
programming interface provided by several existing 
databases and hypergraph libraries --- for 
instance, HyperGraphDB (\cite{BorislavIordanov}), WhiteDB
(\cite{EnarReilent}), and HgLib (\cite[\textnormal{p. 9}]{Erable}).  
That is, 
\DgDb{} is designed so that existing code using 
these technologies can be adopted for \DgDb{} 
with relatively little effort.  \lDgDb{} also 
introduces some new concepts and structuring features, 
which will reviewed below.  In addition to 
prior technologies such as HyperGraphDB, \DgDb{} 
draws on theoretical work connected to hypergraphs 
and their value as multi-paradigm, general-purpose 
metamodels.  The reach of hypergraph theory encompasses 
several paradigms for modeling scientific data, 
such as Conceptual Graph Semantics (see \cite{MattSelway}) 
and Conceptual Space Theory (which is linked to hypergraphs 
in work summarized by \cite{InteractingConceptualSpaces}, 
where Conceptual Space semantics is paired with 
hypergraph categorial grammar).  \lDgDb{} therefore 
introduces modeling elements designed to capture 
scientific details (such as dimensional analysis and units 
of measurement).  When discussing graph structures and 
programming techniques, \DgDb{} draws terminology 
from these scientific perspectives as well as from 
existing Hypergraph database engines.}
	
\subsection{Scientific Features of DigammaDB}
\p{\lDgDb{} has no specific connection to SARS-COV-2 or 
Covid-19; it is conceived as a general-purpose database 
engine that can facilitate application development 
across many domains and industries.  However, \DgDb{} is 
designed with exceptional attention to scientific research 
and software, with respect to metadata, \GUI{} integration, 
its representation of files and file-types, 
and interoperability with technologies related 
to publishing and open-access research data.  
A full enumeration of \DgDb{}'s scientific 
focus is outside the scope of this outline, but 
certain features are specifically relevant 
to \CRtwo{}, so they can be discussed here: 

\begin{description}

\item[From-the-Ground-Up \lQt{} Integration]  All 
\DgDb{} classes natively interoperate with 
\Qt{} (a leading cross-platform application-development 
and \GUI{} framework).  Programmers then have 
access to features like QDataStream and 
\Qt{} meta-objects for binary serialization of \Cpp{} 
objects for persistence in the database.  Robust 
\Qt{} support also makes it easy to design 
\GUI{} classes for interacting with \DgDb{} data, 
and for employing \DgDb{} as a technology for 
managing and storing application state.  This is 
important in the scientific computing context 
because most scientific software is designed 
as native desktop-style applications needing 
special-purpose \GUI{} classes (in this 
environment it is usually not possible to 
adopt techniques such as \HTML{} page 
templates which are commonplace in the 
web-programming context for creating user 
views onto database objects).  Therefore, 
implementing custom \GUI{} logic is an 
important part of the development requirements 
for scientific applications, and it is helpful 
to interface with a database engine prioritizing 
that aspect of software engineering.  

\item[Projections and Scientific Annotations]  
\lDgDb{} adopts the HyperGraphDB notion of 
\textit{projections}, or descriptions of 
individual fields within a complex object 
that can automate the extrapolation of 
binary-serialization algorithms (which 
in turn allows complex objects to be 
stored in \DgDb{} alongside primitive values 
like strings and numbers).  \lDgDb{} also 
uses projections as a basis for introducing 
metadata associated with Conceptual Spaces 
and Conceptual Space Markup Language (\CSML{}) 
as mentioned earlier.  In particular, projections 
can be annotated with statistical/quantitative 
details such as dimensions (e.g., base quantities 
and units of measurement), scales or \q{measurement levels} 
(in \CSML{} terms, particularly the 
distinction between nominal, ordinal, interval, and ratio 
axes), minima and maxima, probability distributions, 
and the aggregation of isolated units into complex 
dimensions, or \CSML{} \q{domains} (vectors, tensors, 
area/volume spans, and so forth).

\item[Data Microcitations]  Most scientific 
data is formally or informally linked to 
peer-reviewed publications which present 
research findings.  Published data sets 
make these connections explicit, by joining 
document identifiers and \URL{}s with 
the corresponding identifiers for designating 
and locating data sets.  These connections 
are however coarse-grained, applying only to 
documents and data sets in their entirety.  
By contrast, an emerging trend in publishing is 
to link portions of publications --- such 
as individual sentences, paragraphs, or figure 
illustrations --- with smaller parts of a data 
set (which could be an individual record/sample, 
or a table column, or some conceptually 
related group of records or columns).  To support 
microcitations, there must be some formal 
mechanism to individuate small parts of a data set.  
This structure is provided internally by 
\DgDb{}: there are multiple microcitable 
\q{zones} in a database, which introduce 
encodings for microcitation targets that 
become exposed to publications via \HGXF{}.  
Here \q{zones} refer to aspects or 
portions of a database that may be cited 
individually, apart from the database 
as a whole: records/atoms, types, 
projections, subgraphs, software 
components implementing a digamma-application 
interface (analogous to HGApplication in 
HyperGraphDB), among others.  These entities can 
be given identifiers which are unique not only in the context of a single database, but if needed over a larger corpus (e.g. an archive of research data) so that the relevant atoms/types/projections can be referenced from publication texts (and similar resources).  Such references can also be linked to \GUI{}s; one can for instance say that the data currently visible in a \GUI{} window or dialog box is a view onto a specific 
database atom, information that could be used for debugging, storing application state, inter-application networking (e.g. linking \PDF{} viewers for research papers with applications to view research data), 
and so forth.

\item[Memory and Persistence Models]  \lDgDb{} can be 
run in several different \q{modes}, which determine 
how data is stored in memory while an application 
is running and/or is stored in files.  If desired, 
\DgDb{} can be used as an \q{in-memory} database which 
does not maintain persistent file state at all.  
This feature can be useful in a research-data context 
where all information derives from data files.  
In this scenario, a \DgDb{} instance can be 
loaded from parsed serializations (e.g., \HGXF{} 
files) without using other file-system resources.  
The database engine can then be adopted as a 
tool for accessing and manipulating the \q{infoset} 
derived from these data files, analogous to 
\XML{} libraries interfacing with a Document Object Model.

\item[Scientific-Computing Interface Description]  Following 
HyperGraphDB, \DgDb{} establishes a \q{type system} 
unique to each database, which is responsible for 
translating application-level types to database atoms.  
The type system used by \DgDb{} is actually 
more detailed (compared to HyperGraphDB) and 
allows for the description of procedure signatures 
and the declaration of conceptually related 
procedural groupings, which collectively 
enable interface definition as persistent 
data within the database itself.  \lDgDb{} 
likewise supports a notion of \q{meta-procedures}, 
which are indirectly derived from Alexandru Telea's 
\textit{metaclasses} and \textit{dataflow interfaces} 
(see \cite{TeleaVISSION}).  Moreover, \DgDb{} 
introduces a notion of \q{channels}, which are a 
higher-level graph structuring feature 
(see \cite[\textnormal{Chapter 3}]{NeusteinCPS}) that may, 
in particular, be applied for the semantic 
annotation of function signatures so as 
to refine interface documentations.  
Collectively, these features enable 
each \DgDb{} database to store information 
about procedures used to access and 
interact with their stored information, 
to facilitate the implementation of 
scientific workflows and \GUI{}s for 
accessing and using \DgDb{} data.

\end{description}
}


\subsection{Programming with HGXF Files}

\p{The prior outline described some of the principle 
features of \DgDb{}, which in turn influenced 
the design of \HGXF{}, insofar as one goal of 
\HGXF{} is to serialize \DgDb{} instances.  However, 
\HGXF{} files are also generic resources for 
serializing research data (or any other technical 
information), and \DgDb{} includes libraries 
for using \HGXF{} (not necessarily in the context 
of \DgDb{} instances).  Therefore, \HGXF{} deserves 
a brief overview in its own right.}

\p{Like any hypergraph meta-model, \HGXF{} represents 
information on two levels.  On the one hand, 
each \q{node} is (in genral) a synthesis of multiple 
smaller parts.  In \HGXF{}, the parts of a hypernode 
are \textit{structure fields} and \textit{array fields}, 
where the former are roughly analogous to named columns, 
and the latter are usually either units within 
expandable collections (such as lists, queues, and dictionaries) 
or fixed-size arrays of numeric fields with similar 
dimensional qualities (e.g., vectors representing point in space).  
\lHGXF{} files define \q{types}, which are internal to 
the file (but may be associated with \DgDb{} types) and 
describe the layout of fields within the containing 
hypernode (in particular, the number and names of 
structure fields, and restrictions on the number of 
array fields, if any).  \lHGXF{} types may also 
provide hints about how to best encode the parsed 
data in running memory.}

\p{The structure of \HGXF{} files is specified by 
a Reference Implementation using \Cpp{} to 
parse and manipulate \HGXF{} data.  In general, 
this code library mixes functional and object-oriented 
programming styles.  The central procedures in 
this library are ones to isolate single hypernodes, 
by graph traversal or queries, and then procedures 
to operate on array and structure fields contained 
within hypernodes.  The basic interface for this 
latter context involves passing criteria for selecting 
one or more fields, along with a callback function 
which will be called with the vector of matching 
fields (or else a procedure to operate on fields 
one at a time).  In this functional style, \HGXF{} 
hypernodes can be seen as monads encapsulating 
access to their internal structure and array 
fields (or \q{hyponodes}).}

\p{On a higher level, applying not to the 
scope of a single hypernode but to groups of 
hypernodes, \HGXF{} introduces several 
higher-level structuring elements (mostly 
derived from corresponding \DgDb{} features). 
In particular, \textit{channels} are 
aggregates of edges, and \textit{frames} are 
contexts for defining edges and nodes.  
In \DgDb{}, every node and edge is constructed 
in the context of a frame, which \textit{may} 
(but need not) offer either additional 
semantic detailing or callback functionality 
to add code affecting how graph data is 
constructed.  All graphs have a \q{default} 
frame that implies no special semantic 
context (i.e., all edges are asserted relations 
deemed to apply in general, with no 
contextual filter) and that utilizes 
general-purpose algorithms for edge and 
node construction.  Developers can 
introduce more specialized frames as desired, 
and construct nodes and edges in these 
tailored contexts.  Such frames can then be 
identified in \HGXF{} as the context where a 
given node or edge serialization applies.  
Hypernodes may also be annotated within frames 
to identify a conceptual, operational, or semantic 
role of a node within some larger node-collection.  
Indeed, \lDgDb{} introduces the concept of a 
\textit{virtual frame} to capture logical patterns 
implicit in certain node-annotations which are not 
explicitly designated by frame objects allocated 
within the database.}

\p{The default behavior for edge-construction 
(which can be overridden within individual frames) 
conceives edges as \RDF{}-style \q{triples}, 
where a directed pair of nodes is annotated 
with a simple label (which may or may not 
be defined within a controlled vocabulary) and/or 
a more complex object; edges can be marked with 
hypernodes of their own.  The source and target 
hypernodes of a hyperedge correspond to the 
source and target node-sets defined in most 
mathematical treatments of hypergraph theory.  
In libraries such as HgLib, which (compared to 
graph/hypergraph databases) are more directly based 
on this mathematical theory, hyperedges are defined 
by providing one or two (for undirected or directed 
edges respectively) sets of nodes.  \lDgDb{} supports 
this construction as well, essentially forming 
hypernodes \q{on the fly}, but in this case the 
fields within these auto-constructed hypernodes are 
\q{proxies} for other hypernodes (possibly three or more) 
linked by the edge.  Proxies can also be 
defined for more complex objects, such as 
frames and subgraphs, allowing multi-dimensional, 
nested graph structures if these are desired for 
a particular domain model.} 

\p{\lHGXF{} also supports microcitations, which are 
discussed above.  By designating parts of 
a data set as microcitation targets, \HGXF{} can 
be integrated with annotation systems designed 
for scientific publications.  These features 
allow document annotations --- which are 
used, for instance, to connect text in a 
research paper with scientific concepts and 
Ontologies --- to connect also with microcitable 
elements within data sets and Research Objects 
which serialize data via \HGXF{}.  In this 
use-case, \HGXF{} works in conjunction with 
a related protocol, the Annotation Exchange Format 
(\AXF{}), which is also initially developed to 
support the \CRtwo{} implementation.  In \CRtwo{}, 
\AXF{} will be used to connect Covid-19 research 
data with publications where the corresponding 
data sets are introduced to the scientific 
community.  The \AXF{} format and design principles 
will be discussed further in the next section.}


\section{The AXF Platform}	
\p{The \AXF{} Platform (hereafter called \AXF{}) 
is a toolkit for hosting full-text, open access 
publications, with an emphasis on scientific, academic, 
and technical documents.  At the core of an \AXF{} 
Publication Repository is a collection of files 
in a machine-readable \AXF{} Document Format (\AXFD{}), 
which are paired with human-readable \PDF{} documents 
as well as supplemental multi-media and metadata files.  
Depending on institutional requirements, an \AXF{} repository 
may be the primary storage resource for the contained 
publications, or an adjunct resource whose documents are 
linked to publications hosted elsewhere.  In the second 
scenario, the primary goal of an \AXF{} repository is to 
host manuscripts in \AXFD{} format, along with software 
to aid viewing and text-mining of the associated publications.}

\p{\lAXFD{} therefore has two distinct purposes: (1) to 
aid in text and data mining (\TDM{}) of full 
publication text (along with research data that may be 
linked to publications), and (2) to enhance the reader 
experience, given e-Reader software (canonically, 
\PDF{} viewers) which are programmed to consume 
\AXF{} information.  To (1) aid in text mining, \AXFD{} documents 
can be compiled into different structured representations, 
yielding document versions that can be registered 
on services such as CrossREF \TDM{} and SemanticScholar.  Given 
a Document Object Identifier, text-mining tools can therefore 
readily obtain a highly structured, machine readable version 
of the publication, which may then be used as the basis 
for further text-mining and \NLP{} operations.  Simultaneously, 
to (2) improve reader experience, the \AXF{} platform generates 
numeric data linking semantically significant text locations 
to \PDF{} viewport coordinates 
(such text locations include annotation, quotation, or citation 
start/end points and paragraph or sentence boundaries --- 
collectively dubbed a Semantic Document Infoset, or \SDI{}).  
This \SDI{}+Viewport (\SDIV{}) information can then be used by 
\PDF{} applications to provide contexts 
for word searches, to localize context 
menus, to activate multi-media features at different points 
in the text, and in general to make \PDF{} files more 
interactive.  Data sets composed with the aid of \AXF{} 
tools may include source code for a \PDF{} viewer (an extension 
to \XPDF{}) capable of leveraging \AXF{} data.}

\p{In addition to the \AXFD{} document format, the \AXF{} platform 
includes the Annotation Exchange Format itself, a protocol for 
defining and sharing annotations on full-text publications.  
\lAXF{} differs from other annotation-representation 
strategies by (1) providing more detail concerning 
the location of annotated text segments, in the 
surrounding publication context, and (2) supplying 
annotation data in multimedia or microcitation 
formats which extend beyond conventional 
\q{controlled vocabularies}.  In terms of (1) 
publication context --- that is, the annotation 
\textit{target} (using terminology from the 
Linguistic Annotation Framework, or \LAF{}) --- \AXF{} 
represents \SDIV{} information as introduced 
above; this data supplements the node/index coordinates 
used by traditional annotation mechanisms.  
With respect to (2) annotation metadata --- 
or the annotation \textit{body} (again using 
\LAF{} terminology) 
--- \AXF{} introduces models for multimedia 
assets, software components, and data set 
content, which may be linked to annotation 
targets.  With this additional metadata, 
annotations may be used in application-development 
environments, not only for text mining.}

%Linguistic Technology Systems has developed \AXF{} to 
%address shortcomings of existing annotation standards, particularly 
%in the context of annotation environments integrating application, 
%semantic, and multimedia content.  The goal of \AXF{} is to represent 
%annotations at a novel level of detail.  In particular 
%(as will be described below), the \textit{target} of the annotation 
%should be represented in both semantic and interactive contexts 
%(to include, for instance, both sentence-level context and 
%\PDF{} viewport coordinates).  Meanwhile, the \textit{body} 
%of the annotation (using terms originating with the 
%Linguistic Annotation Framework) can be linked to data 
%\q{microcitations}, multimedia assets, or 
%controlled vocabularies.  \lAXF{} is optimized for 
%\AXFD{} manuscripts --- i.e., documents encoded according to 
%the native \AXF{} protocol --- but \AXF{} can potentially 
%be used as an exchange format for document-annotation 
%operations in a variety of contexts (named entity recognizers, 
%science-data \API{}s, publication metadata sharing, and so forth.}


\p{The following sections will (1) outline \AXF{} and \AXFD{} 
in greater detail, (2) describe how \AXF{} repositories 
can unify publications sharing similar themes, scholarly 
disciplines, or coding requirements, and (3) 
describe features for data-set publication in 
the context of \AXF{} repositories.}


\subsection{AXF Documents}

\p{The \AXFD{} format for describing document content and 
structure is designed to be a \q{Pivot Representation} in the 
sense of \LAF{} (see \cite{IdeSuderman}).  In particular, \AXFD{} can represent 
the structure of both \XML{} (including several 
\XML{} flavors used in publishing) and \LaTeX{}.  
Technically, \AXFD{} does not prescribe any 
specific input format; instead, a document is 
considered an instance of \AXFD{} if it can be 
compiled into a Document Object satisfying 
interface requirements.  A \Cpp{} reference implementation 
anchors the \AXF{} Document Object Model; nodes in 
this implementation have facets combining 
\LAF{}, \XML{}, and \LaTeX{}.  In practice, 
\AXFD{} manuscripts are then converted via \LaTeX{} to 
\PDF{}, and simultaneously compiled to \XML{} representations 
so as to generate machine-readable, structured full-text 
versions of the manuscripts.  Authors can choose to 
compose \AXFD{} papers to conform with several common 
publication \XML{} standards, such as \JATS{} 
(Journal Article Tag Suite), \SciXML{} \cite{CJRupp}, and 
\IeXML{} \cite{DietrichRebholzSchuhman} 
(the latter is an annotation-oriented 
\XML{} language used by the BeCAS project 
\cite{TiagoNunes}).  Authors or editors 
may further define or model scientific 
concepts via strategies focused on 
computer simulation (in the broad sense 
as defined, for instance, in \cite{EricWinsberg}) 
or statistics (e.g., Predictive Model Markup Language 
or Attribute-Relation File Format), 
and/or conceptual semantics, such as Conceptual Space Markup 
Language (which was inspired by the linguist 
Peter G\"ardenfors, but developed in a 
scientific/mathematical framework in e.g.
\cite{RaubalAdams}, \cite{RaubalAdamsCSML},
\cite{Zenker}, and \cite{InteractingConceptualSpaces}).}

\p{One distinct feature of \AXF{} is that \LaTeX{} and 
\XML{} generation are chained in a pipeline: the \LaTeX{} 
and subsequent \PDF{} generation steps yield auxiliary data, 
which includes \PDF{} viewport data, that can be 
subsequently incorporated into \XML{} views onto 
the documents.  Specifically, 
\AXFD{}-generated \LaTeX{} files include notations for 
semantic annotations and for sentence boundaries, implemented 
via \LaTeX{} commands which, as one processing step, 
write \PDF{} coordinates to auxiliary files.  The resulting 
data is then read by a \Cpp{} program which collates 
annotations and sentence-boundaries into a vector of 
data structures indexed by \PDF{} page numbers, creating a 
distinct file for each page, and zips those files into an 
archive which can be distributed alongside (or embedded inside) 
the \PDF{} publications.  Simultaneously, sentences, 
paragraphs, annotations, and other semantically significant 
content (such as quotations and citations) are assigned 
unique ids and compiled into their own data structures 
(from which machine-readable \XML{} full-text may be generated).  
These \XML{} files may then be hosted and/or registered on 
\TDM{}-oriented services such as CrossRef.  At the same 
time, unique identifiers unify this \XML{} data 
(focused on text mining) with \PDF{} viewport data 
(focused on reader experience).  The goal of such 
integration is to incorporate text-mining results 
so as to enhance reader experience.  For example, 
Named Entity Recognizers might flag a word-sequence 
as matching a concept within a controlled vocabulary.  
Via the relevant paper's \SDI{} model, this annotation 
may be placed in a proper semantic context --- for 
example, obtaining the text of the sentence where the 
Named Entity occurs.  This semantic information may then 
be used by a \PDF{} viewer --- e.g., providing a 
context menu option to select the sentence text, when 
the context menu is activated within the rectangular 
coordinates of the annotation itself.}

%
\begin{figure}

\caption{Data Microcitations via Tabular Columns}
\label{fig:xl}

\begin{tikzpicture}

\node[inner sep=0pt] (x1) at (0,0)
    {\includegraphics[width=120mm, 
    	trim={0mm 0mm 0mm 0mm},clip]
    	{pics/xl.png}};
    
\end{tikzpicture}   
\end{figure}



\begin{figure}

\caption{Linking PDF Files with Scientific Applications}
\label{fig:il}

\begin{tikzpicture}

\node[inner sep=0pt] (x1) at (0,0)
    {\includegraphics[width=90mm, 
    	trim={0mm 0mm 0mm 0mm},clip]
    	{pics/xl.png}};

\node[inner sep=0pt] (i1) at (10,0)
    {\includegraphics[width=90mm, 
    	trim={0mm 0mm 0mm 0mm},clip]
    	{pics/il.png}};

\curicon{1.9}{1}
    
\end{tikzpicture}   
\end{figure}


\p{As a representation of annotation data structures, 
\AXF{} ensures that \SDI{} and viewport data is included 
among annotations wherever this data is available.  
This facilitates the integration between text-mining 
tools and \PDF{} viewer software, which in turn 
enhances reader experience.  As mentioned earlier, every 
annotation can be placed in a semantic context (e.g., 
the text of the surrounding sentence), which provides 
useful reader features such as one-click copying of 
sentences to the clipboard.  Other reader-experience 
enhancements involve multimedia assets.  As a concrete example, 
suppose a paper includes mention of a chemical; that particular 
keyword can accordingly be flagged for annotation.  
As one encoding of the corresponding scientific concept, 
the annotation can include the chemical's Chemical Abstract 
Service Reference Number, via which it is possible to obtain 
Protein Data Bank (\PDB{}) files to view the relevant 
molecular structure in \ThreeD{}.  In sum, 
annotations supply a constellation of data 
--- in this example, concepts may be linked not only to 
identifiers in cheminformatic ontologies, but also to \CAS{} 
reference numbers and thereby to \ThreeD{} graphics files 
--- which facilitate interactive User Experience at the 
application level, not only document classification at the 
corpus level.  Once a chemical compound (mentioned in a 
publication) is linked to a \PDB{} file (or any other 
\ThreeD{} format) the \PDF{} viewer may include 
options to for the reader to connect to software or 
web applications where the corresponding visuals can 
be rendered.  Via \AXF{}, the relevant document-to-software 
connections are asserted not only on the overall document 
level, but on the granular scale of the precise character 
and \PDF{} viewport coordinates where the relevant 
annotation is grounded (Figure~\bref{fig:il} illustrates 
such capabilities in the context of a chemistry publication 
--- specifically, test-preparation materials for the 
Chemistry \GRE{} exam).}

\p{To support this kind of multimedia functionality, 
\AXF{} standardizes a Plugin Framework, dubbed \q{\Mosaic{}}, 
allowing programmers to embed code which can parse and 
respond to \AXF{} annotations in different scientific and 
document-viewer applications.  \lMosaic{} allows different 
applications to inter-operate; in particular, 
\PDF{} viewers can share data with scientific applications 
that can render files in domain-specific formats such as 
\PDB{}.  This application networking protocol is considered 
part of the \AXF{} annotation model, because application-oriented 
information is computationally relevant for many concepts 
encountered in scientific and technical environments.  For instance, 
one aspect of cheminformatic data is that many chemical 
compounds are modeled by \PDB{}, \MOL{}, or \ChemXML{} files, 
which in turn are associated with software applications that 
can load those file types.  Such inter-application 
networking data is relevant to \PDF{} viewers 
when they display manuscripts with annotations that suggest 
links to special file types and their applications; the 
viewers can employ this information to launch and/or communicate 
with the corresponding software.  \lAXF{} is designed to 
facilitate implementation of application-networking protocols 
as an operational continuation of processes related to obtaining 
and consuming annotation data.}

\p{The \AXF{} document model, at the manuscript-structure 
level, is paired with a novel \q{Hypergraph Text Encoding 
Protocol} (\HTXN{}) operating at the character-encoding level.  
Within the \HTXN{} protocol, an annotation target is 
a character-index interval in the context of an 
\HTXN{} character stream.  On that basis, 
\HTXN{} treats documents as graphs whose nodes 
are ranges in a character stream, where text can 
be recovered as an operation on one or more nodes 
(e.g., the text of a sentence is derived from a 
pair of nodes representing the sentence's start 
and end).  \lHTXN{} code-points 
are distinguished in terms of their semantic 
role, which may be more granular than their 
visible appearance --- for example, 
a period glyph is assigned different code-points 
depending on whether it marks a sentence-ending 
punctuation, an abbreviation, a decimal point, 
or part of an ellipsis.  Procedures are then implemented to 
represent text in different formats, such as 
\ASCII{}, Unicode, \XML{}, or \LaTeX{}.  In 
contrast to a format such as Web Annotations, 
any particular human-readable text presentation 
(including \ASCII{}) is considered a \textit{derived} 
property of the annotation, not a foundational 
representation.}

\p{\lAXFD{} manuscripts do not need to utilize \HTXN{} 
for character data, but \HTXN{} simplifies certain \AXF{} 
operations, such as identifying sentence boundaries.  
In particular, \HTXN{} provides distinct code-points for 
end-of-sentence punctuation, so that sentence-boundary 
detection reduces to a trivial search for those particular 
code-points.  Proper \HTXN{} encoding requires that authors 
follow certain simple heuristics --- e.g., that end-of-sentence 
periods should be followed by two spaces and/or a newline, 
whereas other uses of a period character should precede at 
most one space.  Aside from the goal of preparing documents 
for text-mining machine-readability, such conventions are 
appropriate even for basic typesetting, because non-punctuation 
characters have their own kerning rules (this is why 
\LaTeX{} provides a distinct command for non-punctuation glyphs 
that would otherwise be read as punctuation characters).  \lHTXN{} 
hides these typesetting details within its character-encoding 
schema, which is then useful both for producing professional-caliber 
\LaTeX{} output and for identifying \SDI{} details (such as 
sentence boundaries) which with less rigorously structured 
text would need elaborate text-mining or \NLP{} algorithms.}

\p{Each \AXFD{} document is, in sum, associated with an 
aggregate of character-encoding, annotation, document-structure, 
and \PDF{} viewport information.  The \AXF{} platform uses 
code libraries to pull this information together as a 
runtime object system, so that any application which loads 
an \AXFD{} manuscript can execute queries against the 
corresponding collection of \AXF{} objects (queries such 
as obtaining the sentence text around an annotation, obtaining 
the concave-octagonal viewport coordinates for a sentence,\footnote{
In the general case, sentence coordinates are concave octagons because 
they incorporate the line height of their start and end lines; 
in the general case sentences share start and end lines with other 
sentences, while also including whole lines vertically positioned 
between these extrema.  A sentence octagon roughly corresponds with 
the screen area where a mouse/pointer action should be understood as 
occurring in the context of that sentence from the user's point 
of view --- implying that the user would benefit from context menu 
options pertaining specifically to that sentence, such as 
copy-to-clipboard.} obtaining application-networking 
information for an annotation, 
etc.)  In addition to such runtime data, \AXF{} platforms can 
compile the full suite of information into machine-readable 
files for text and data mining.  These files, collected across a 
corpus of multiple documents, then form the backbone of an 
\AXF{} publication repository, as will be discussed next.}

\subsection{AXF Publication Repositories}
\p{The \AXF{} platform is designed for hosting collections 
of publications sharing a common academic or technical 
focus.  When \AXF{} is used in the context of a 
general-purpose text and/or data repository, 
the \AXF{} platform is designed to work with collections that 
are organized into separate projects or topics, 
each giving rise to an archive or corpus of publications.  
Insofar as these corpora internally share a common theme 
or focus, they can be associated with their own 
ontologies, code libraries, annotation models, and 
application-networking protocols, based on the sorts 
of applications and data structures commonly used 
in the corresponding scholarly discipline.  
In some cases, publishers may choose to package 
an entire archive of research papers (perhaps along with 
research data) as a single downloadable resource.  
\lAXF{} allows publishers to construct such Research Archives 
following the structure of existing examples such as 
the \ACL{} (Association for Computational Linguistics) 
Anthology or the recent \Cnineteen{} corpus.   This 
latter archive is a useful case-study in both the 
possibilities and limitations of existing 
publication-repository technology, 
so it is reviewed here in more detail.}

\p{\lCnineteen{}, curated by the Allen Institute for 
Artificial Intelligence, was spearheaded by 
a White House initiative to centralize scientific 
research related to \Covid{} (see \cite{CORD}).  The  
collection was formulated with the explicit goal 
of promoting both \textit{text mining} and 
\textit{data mining} solutions 
to advance coronavirus research, so that 
\Cnineteen{} is intended to be used both as a document 
archive for text mining and as a repository for 
finding and obtaining coronavirus data for subsequent 
research.  Although novel research is being 
incorporated into \Cnineteen{}, many of the articles 
reproduced in this corpus are older publications related 
to coronaviruses and to SARS in general, not just to the 
current pandemic.  As a result, the full-text versions 
of these publications were retroactively aggregated 
into a single archive due to the unanticipated emergence 
of a coronavirus crisis, with the full text often obtained 
from \PDF{} files rather than from structured 
representations (such as \JATS{}) explicitly intended 
for text mining.}

\p{This archival methodology results in \Cnineteen{} being 
limited as a \TDM{} framework.  These limitations include 
the following:

\begin{description}
\item[Transcription Errors]  
Transcription errors can easily result from trying to 
read scientific data and notations based on 
\PDF{} files --- or on full-text representations using 
relatively unstructured formats such as \XOCS{} 
(the response-encoding format for the ScienceDirect 
\API{}).  Transcription errors cause the machine-readable 
text archive to misrepresent the structure 
and content of documents.  For instance, 
there are cases in \Cnineteen{} 
of scientific notation and terminology 
being improperly encoded.  As a concrete example, \colorq{2{\textquotesingle}-C-ethynyl} is encoded incorrectly in one \Cnineteen{} file 
as \makebox{\colorq{2 0 -C-ethynyl}} (see \cite{Eyer} for 
the human-readable publication where this error is 
observed; the corresponding index in the corpus is \textcolor{blGreen!45!black}{9555f44156bc5f2c6ac191dda2fb651501a7bd7b.json}).  
To help address these sorts of errors --- 
which could stymie text searches 
against the \Cnineteen{} corpus --- 
it is obviously preferable to archive structured, 
machine-readable versions of publications, using 
a platform such as \AXF{}.  
    
\item[Converting Between Data Formats]
Although the \Cnineteen{} corpus is published 
as \JSON{} files, many text-mining tools such 
as those reviewed in \cite{NeusteinText} recognize 
inputs or produce outputs in alternative formats, 
such as \XML{}, \BioC{}, \CoNLL{} (Conference on Natural 
Language Learning), or \JSON{} trees with 
different schema than \Cnineteen{}.  For this 
reason, rather than providing data with one single 
representational format, it is better to 
encode the data along with code libraries that can 
express the data in different formats as needed 
for different \TDM{} ecosystems.

\item[Inconsistent Annotations]  
The structure of \Cnineteen{} allows text segments to 
be defined via a combination of \JSON{} file names, 
paragraph ids, and character indices.  This indexing 
schema is used for representing certain internal 
details of individual articles, such as citations, 
but is not explicitly defined as an annotation 
target structure for standoff annotations against 
the archive as a whole.  This problem could 
also be rectified with code libraries that map 
index targets to file handles and character pointers. 

\item[Limited Support for Research Data-Mining]  Even though 
many papers in \Cnineteen{} are paired with 
published data sets, there is currently no tool for 
locating  research \textit{data} 
through \Cnineteen{}. 
For example, the collection of manuscripts available 
through the Springer Nature portal linked 
from \Cnineteen{} includes over 30 \Covid{} data sets,
but researchers can only discover that these data 
sets exist by looking for a \q{supplemental materials} or 
a \q{data availability} addendum near the end of each article.
These Springer Nature data sets encompass a wide array of file types 
and formats, including \FASTA{} (which stands for Fast-All, 
a genomics format), \SRA{} (Sequence Read Archive, for 
\DNA{} sequencing), \PDB{} (Protein Data Bank,  
representing the \ThreeD{} geometry of protein 
molecules), \MAP{} (Electron Microscopy Map), \EPS{} 
(Embedded Postscript), \CSV{} (comma-separated values), 
and tables represented in Microsoft Word 
and Excel formats.  To make this data more 
readily accessible in the context of \Cnineteen{}, it 
would be appropriate to (1) maintain an index of 
data sets linked to \Cnineteen{} articles 
and (2) merge these resources into a common representation 
(such as \XML{}) wherever possible.  This research-data 
curation can then be treated as a supplement to 
text-mining operations.  In particular, queries 
against the full-text publications could be evaluated 
\textit{also} as queries against the relevant set 
collection of research data sets.    

\item[Wrappers for Network Requests]  Scientific 
use of \Cnineteen{} will often require communicating 
with remote servers.  For example, genomics 
information in the \Covid{} data sets (such as 
those mentioned above that are available through 
Springer Nature) is generally 
provided in the form of accession numbers which 
are used to query online genomics services.  
Similarly, text mining algorithms often 
rely on dedicated servers to perform 
Natural Language Processing; these services 
might take requests in \BioC{} format and respond 
with \CoNLL{} data.  As another case study epidemiological 
studies of \Covid{} may need to access \API{}s or data 
sets such as the John Hopkins University \q{dashboard} 
(see \href{https://coronavirus.jhu.edu/map.html}{https://coronavirus.jhu.edu/map.html}, which is paired with a \GIT{} archive 
updated almost daily).  To reduce the amount 
of \q{biolerplate code} which developers need 
for these networking requirements, an archive's 
text-mining code could provide a unified framework with which 
to construct web-\API{} queries, 
one that could be used across 
disparate scientific disciplines 
(genomics, \NLP{}, epidemiology, and so forth). 
\end{description}}

\p{Many of these limitations observed in \Cnineteen{} 
reflect the fact that this corpus was prepared 
as raw (text) data, without any supporting code.  
By contrast, recent initiatives --- such as the Research Object 
protocol (see \cite{KhalidBelhajjame}) 
and \FAIR{} (\q{Findable, Accessible, 
Interoperable, Reusable}; see \cite{TrifanOliveira}) --- 
encourage authors to publish code and data together, 
so that the computing environment needed to process 
published data is provided within the data set itself.  
The \Cnineteen{} limitations accordingly provide an 
example of why Research Objects, rather than raw data 
sets, should be preferred for data publication in the 
future.  The Research Object model is usually defined in 
the context of a single publication, but the paradigm 
applies equally well to corpora encompassing many 
single articles.  That is, \AXF{} is structured so 
that Research Archives can be designed as higher-scale 
Research Objests, wherein the document collection is bundled 
with supporting code and an overall computing and 
software-development environment.  Such archive-specific 
\SDK{}s would include \AXF{}-specific code as well 
as libraries or applications often utilized in 
the academic disciplines relevant to the archival 
subject areas.  The \AXF{} platform especially 
promotes the design of domain-specific \SDK{} 
which are \textit{standalone} 
and \textit{self-contained}, with minimal external 
dependencies.  As much as possible, users should 
not have to install external software to utilize 
data provided along with an \AXF{} repository; 
instead, the needed data-management tools should be 
provided in source-code form within the archive itself.}

\p{Each \AXF{} repository, then, should 
bundle numerous applications used for database 
storage, data visualization, and scripting.  
The goal of this application package would be to 
provide researchers with a self-contained computing 
platform optimized for scientific research 
and findings related to the archived publications.  
Archival \SDK{}s should try 
to eliminate almost all scenarios where 
programmers would need to perform a \q{system 
install}; for the most part, the entire 
computing platform (including scripting 
and database capabilities) should be compiled 
from source \q{out-of-the-box}.  While the 
actual libraries and applications bundled with 
an archive would depend on its topical 
focus, the following is an example of components 
that would be appropriate in many different \SDK{}: 

\begin{itemize}[itemsep=-1pt]

\item \XPDF{}: A \PDF{} viewer for reading full-text articles 
(augmented with \Cnineteen{} features, such as integration 
with biomedical ontologies);

\item \Qt{}: The \Qt{} library is a cross-platform 
Application-Development framework and 
\GUI{} toolkit commonly used for scientific applications 
(\XPDF{} is one example of a \Qt{}-based document 
viewer).  Almost any data set can be accompanied 
with \Qt{} code for data visualization, so that readers 
would not have to install additional software.  For its 
part, \Qt{} can be freely obtained and, once 
downloaded, resides wholly in its own folder 
(there is no install step which modifies the 
user's system); as such, \Qt{} along with individual 
archive \SDK{}s function as standalone packages, 
although optimally the \SDK{}s would be updated along with 
new \Qt{} versions.  
  
\item AngelScript: An embeddable scripting engine 
that could be used for analytic processing 
of data generated by text and data mining operations 
on \Cnineteen{} (see \cite{AS});

\item WhiteDB: A persistent database 
engine that supports both relational 
and \NoSQL{}-style architectures 
(see above);

\item MeshLab: A general-purpose \ThreeD{} graphics 
viewer;

\item LaTeXML: a \LaTeX{}-to-\XML{} converter;

\item PositLib: a library for use in high-precision computations based on the \q{Universal Number} format, 
which is more accurate than traditional floating-point 
encoding in some scientific contexts 
(see \cite{JohnGustafson}). 
\end{itemize}

To this list one might add components specific to various 
scientific fields: \IQmol{} for chemistry and molecular 
biology, for example, or open-source libraries such 
as EpiFire or Simpact (for Epidemiology), \UDpipe{} (for 
\CoNLL{}), and so forth.  Here again the priority 
would be for self-contained components with few 
external dependencies --- particularly libraries 
programmed in \C{} or \Cpp{}, which are the 
languages best positioned to be a common denominator 
across diverse research projects (of course, many 
scientific \Cpp{} libraries have wrappers for 
languages like \R{} or Python that researchers may be 
more comfortable using).  
In general, Research Archive code should be 
(1) \textit{self-contained} (with few or no external 
dependencies, as emphasized above); 
(2) \textit{transparent} (meaning that 
all computing operations should be implemented by 
source code within the bundle that can be examined 
as code files and within a debugging session); 
and (3) interactive (meaning that the bundle does not 
only include raw data but also software to interactively 
view and manipulate this data).  Research Archives which 
embrace these priorities attempt to provide data visualization, 
persistence, and analysis through \GUI{}, database, and 
scripting engines that can be embedded as source 
code in the archive itself. }
  
\p{It is worth noting that a data-mining platform requires 
\textit{machine-readable} open-access research data
(which is a more stringent requirement than simply 
pairing publications with data that can only 
be understood by domain-specific 
software).  For example, radiological imaging can be a source 
of \Covid{} data insofar as patterns of lung 
scarring, such as \q{ground-glass opacity,} are a leading 
indicator of the disease.  Consequently, diagnostic 
images of \Covid{} patients are a relevant kind of 
content for inclusion in a \Covid{} data set 
(see \cite{Shi} as a case-study).  However, 
diagnostic images are not in themselves 
\q{machine readable.}  When medical imaging is 
used in a quantitative context (e.g., applying 
Machine Learning for diagnostic pathology), it is necessary 
to perform Image Analysis to convert the raw data 
--- in this case, radiological graphics --- into 
quantitative aggregates.  For instance, by using image 
segmentation to demarcate geometric boundaries one 
is able to define diagnostically relevant features (such 
as opacity) represented as a scalar field over the segments.  
In short, even after research data is openly published, 
it may be necessary to perform 
additional analysis on the data for it to be 
a full-fledged component of a 
machine-readable information space.\footnote{%
\raisebox{-10pt}{\hspace{3pt}\parbox{.9\textwidth}{This does not mean that diagnostic images (or 
other graphical data) should not be placed in a 
data set; only that computational reuse of such 
data will usually involve certain numeric 
processing, such as image segmentation.  
Insofar as this subsequent analysis is performed, 
the resulting data should wherever possible 
be added to the underlying image data as a 
supplement to the data set.}}}  To 
deal with this sort of situation, \AXF{} equips 
\SDK{}s with a \textit{procedural 
data-modeling vocabulary} that would both identify the 
interrelationships between data representations 
and define the workflows needed to 
convert research data into machine-readable data sets.} 

\p{Another concern in developing an integrated Research Arcive
data collection is that of indexing documents and research findings  
for both text mining \textit{and} data mining.  
In particular, \AXF{} introduces a 
system of \textit{microcitations} that apply 
to portions of manuscripts \textit{as well as} data sets.  
In the publishing context, a microcitation is defined as a 
reference to a partially isolated fragment of a larger 
document, such as a table or figure illustration, or a 
sentence or paragraph defining a technical term, 
or (in mathematics) the statement/proof of a definition, axiom, 
or theorem.  In data publishing, \q{data citations} are 
unique references to data sets in their entirety or to 
their smaller parts.  A data microcitation is then a 
fine-grained reference into a data set.  For example, 
a data microcitation can consist of one 
column in a spreadsheet, 
one statistical parameter in a quantitative analysis, 
or \q{the precise data records actually used in a study} 
(in the words adopted by the Federation of Earth Science Information Partners to define microcitations; 
see \cite{ESIP}).  As a concrete example, 
a concept such as \q{expiratory flow} appears in \Cnineteen{} 
both as a table column in research data and as a medical concept 
discussed in research papers; a unified microcitation framework 
should therefore map \textit{\color{drp}{expiratory flow}} as a keyphrase 
to both textual locations and data set parameters.  
Similarly, a concept such as 
\textit{\color{drp}{2{\textquotesingle}-C-ethynyl}} (mentioned earlier, in the context of transcription errors) 
should be identified both as a phrase in 
article texts and as a molecular component 
present within compounds whose scientific 
properties are investigated through \Cnineteen{} 
research data.  In so doing, a search for this 
concept would then trigger both publication and 
data-set matches at the same time.}

\p{Further discussion on data microcitations depends on 
how data sets are structured, which is addressed in the next section.}


\subsection{Research Objects and Data Microcitations}
\p{The design of \AXF{} assumes that many Research Archives, 
comprising of multiple publications sharing an academic 
focus, will also include open-access research data.  
The primary motivation for publishing 
research data is to ensure transparency and 
reusability --- open-access data allows readers to verify 
that scientific/technical claims are warranted, 
and/or to reuse or incorporate existing data into 
new research.  Open-access data has other purposes 
as well: data sets, for example, can serve 
as pedagogic tools helping readers understand 
publications' concepts experimentally and 
interactively (a good example is \cite{AlexandruTelea}, 
which pairs data sets with its individual chapters to 
illustrate principles in data visualization).  
Moreover, the theory informing how data sets are 
organized can serve as a technical exposition 
of research principles and methodology.  For all 
of these reasons, open-access data is an 
increasingly important part of the publishing 
ecosystem.  This means 
that well-curated archives will often need to 
prepare data sets for data 
mining, alongside the preparation of text materials for 
text mining.  }

%   \begin{frame}{\ft{Oxy}}

        \begin{annotatedFigure}{0pt}{0pt}
            {\includegraphics[scale=1]{texs/oxy.png}}
            
  \node [text width=7.6cm,align=justify,fill=logoCyan!20, draw=logoBlue, 
  draw opacity=0.5,line width=1mm, fill opacity=0.9]
   at (0.74,0.465){\textbf{Dataset Applications make extensive 
   use of context menus to organize functionality and provide 
   advanced interactivity.  In this screenshot a context menu 
   action (\circled{1}) has been selected which alters the 2D 
   display, visually emphasizing a restricted set 
   of data points (\circled{2}) and contracting all others (\circled{3}).}};

            \annotatedFigureBox{0.73,0.26}{0.96,0.328}{1}{0.89,0.326}%            
            \annotatedFigureBox{0.55,0.4}{0.55,0.44}{3}{0.55,0.44}            
            \annotatedFigureBox{0.598,0.45}{0.598,0.48}{2}{0.598,0.45}    
      %      \annotatedFigureBox{0.222,0.284}{0.3743,0.4934}{B}{0.3743,0.4934}%tr
      %      \annotatedFigureBox{0.555,0.784}{0.6815,0.874}{C}{0.555,0.784}%bl
      %      \annotatedFigureBox{0.557,0.322}{0.8985,0.5269}{D}{0.8985,0.5269}%tr
  
        \end{annotatedFigure}

   %     \caption{Expanded Sample (A)}
    %    \label{fig:teaser}

    \end{frame}


\p{In contrast to text mining, however, it is not feasible, 
in the general case, to 
assign one single format (like \AXFD{} or \JATS{}) for all 
data sets published within an archive.  Precisely how data 
sets can be annotated depends on the data models, programming 
languages, and analytic methodologies which they utilize.  
Because this variability prohibits a single data-annotation 
protocol from being required, \AXF{} adopts a strategy of 
defining a rigorous protocol for \Cpp{} code bases, which 
can then be emulated by other languages.}

\p{As outlined above, data citations refer to parts within a 
data set --- such as individual data records, but also larger-scale 
aggregates such as table columns or statistical parameters.  
The complication when defining data citations is that a 
concept such as a table column, although it may have an 
obvious technical status as a discrete conceptual unit from 
the point of view of scientists curating, studying or reusing 
a data set, does not necessarily correspond to a single 
coding entity that could be isolated as an annotation 
target.  It is therefore the responsibility of 
\textit{code base annotations} to provide annotations for 
computational units --- such as data types, procedures, and 
\GUI{} components --- that have an annotatable \textit{conceptual} 
status relative to the data set on which the code operates.  
Often this will involve mapping one concept to 
several computational units (for instance, several 
procedure implementations).}

\p{For a concrete example of these points concerning 
data citations, consider the data set pictured 
in Figure~\bref{fig:oxy}, representing cyber-physical measurements 
used to calculate oxygenated airflow.  The data-set 
application (interactive-visualization code deployed within 
the Research Object) displays tabular data via a tree widget 
(which functions as a generalized, multi-scale spreadsheet 
table), with tabular columns expressing quantities --- such 
as air flow and oxygen levels --- in several formats (raw 
measures as well as sample rankings and min-max 
percentages).  Conceptually, these columns have distinct 
methodological roles and therefore can be microcited; 
indeed, the application links the columns to article 
text where the corresponding concepts are presented 
(see Figure~\bref{fig:about}).  However, the implementation does not introduce 
a distinct \Cpp{} object uniquely designating individual 
columns.  Instead, the individual columns can be annotated 
in terms of \Cpp{} methods providing column-specific 
functionality.  In the current example, these methods 
primarily take the form of features linked to context-menu 
actions (copying column data to the clipboard, sorting data 
by one column, etc.).  In general, rather than a rigid 
protocol for data-set annotations, \AXF{} proposes 
heuristic guidelines for how best to map programming 
constructs to scientifically salient data-set concepts.}

%\begin{frame}{\ft{About}}

	\pdfpageheight 30cm

        \begin{annotatedFigure}
            {\includegraphics[scale=1]{about.png}}
            
  \node [text width=12cm,align=justify,fill=logoCyan!20, draw=logoBlue, 
  draw opacity=0.5,line width=1mm, fill opacity=0.9]
   at (0.58,0.76){\textbf{Context menus also allow users to 
   obtain information and explanations about individual parts of the 
   data set, such as individual statistical parameters.  In this 
   screenshot, the user has right-clicked on a data column and 
   chosen a context menu action which shows, via a dialog box, 
   a precis of the quantities represened in that column and their 
   significance for the data set as a whole.}};

            \annotatedFigureBox{0.2,0.12}{0.812,0.645}{1}{0.81,0.645}%            
      %      \annotatedFigureBox{0.222,0.284}{0.3743,0.4934}{B}{0.3743,0.4934}%tr
      %      \annotatedFigureBox{0.555,0.784}{0.6815,0.874}{C}{0.555,0.784}%bl
      %      \annotatedFigureBox{0.557,0.322}{0.8985,0.5269}{D}{0.8985,0.5269}%tr
  
        \end{annotatedFigure}

\end{frame}

\p{Defining an annotation schema for data sets can potentially 
be an organic outgrowth of software-development methodology 
--- viz., the engineering steps, 
such as implementing unit tests, which are essential 
to deploying a commercial-grade application.  
This point is illustrated in 
Figure~\bref{fig:testing}, which shows a \GUI{}-based testing environment for 
the data set depicted in Figures~\bref{fig:oxy} and 
\bref{fig:about}.  For this data set, 
the context menu actions providing column-specific functionality 
are also discrete capabilities which can be covered by 
unit tests, so the set of procedures mapped to the citeable 
concept correspond with a set of unit-test requirements.  
In this data set, these procedures are also exposed to 
scripting engines via the \Qt{} meta-object system.  In general, 
there is often a structural correlation between 
scripting, unit testing, and microcitation, so that 
an applications' scripting and testing protocol can serve 
as the basis for annotation schema.  For data sets which use 
in-memory or persistent databases, evaluable queries against 
these databases provide an additional grounding for annotations.  
In general, data-annotation should be engineered on the 
basis of a dataset applications' scripting, testing, and/or 
query-evaluation code.  However, this is only a heuristic 
guideline, and \AXF{} does not presuppose any data-annotation 
scheme \textit{a priori}.}

\begin{figure}

\caption{Test Suites for Dataset Applications}
\label{fig:testing}

\begin{tikzpicture}

\node[inner sep=0pt] (x1) at (0,0)
    {\includegraphics[width=7in, 
    	trim={0mm 0mm 0mm 0mm},clip]
    	{pics/testing.png}};
    
\end{tikzpicture}   
\end{figure}




\end{document}
