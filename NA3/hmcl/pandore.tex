\documentclass[11pt,letterpaper]{article}


% pmml  arff  openannotation

%\usepackage[condensed,math]{anttor}
%\usepackage[T1]{fontenc}

%\usepackage[T1]{fontenc}
%\usepackage{tgtermes}

\usepackage[hang,flushmargin]{footmisc}

\usepackage{titlesec}

%\usepackage{sectsty}
%\sectionfont{\fontsize{13}{4}\selectfont}

\titleformat{\section}
  {\normalfont\fontsize{13}{15}\bfseries}{\thesection}{1em}{}

\let\OldPart\part

\renewcommand{\part}[1]{\OldPart{#1}%
%{\textcolor{darkRed}\hrule}
\vspace{-.5em}}

\titlespacing*{\section}
{0pt}{4ex plus 1ex minus .1ex}{-0.2ex plus .2ex}

\titlespacing*{\subsection}
{0pt}{3.5ex plus 1ex minus .1ex}{-.8ex plus .2ex}

%\usepackage{mathptmx}

\usepackage{eso-pic}

%\setlength\parindent{0pt}

\AddToShipoutPictureBG{%

\ifnum\value{page}>1{
\AtTextUpperLeft{
\makebox[20.5cm][r]{
\raisebox{-1.95cm}{%
{\transparent{0.3}{\includegraphics[width=0.29\textwidth]{e-logo.png}}	}} } }
}\fi
}

\AddToShipoutPicture{%
{
 {\color{blGreen!70!red}\transparent{0.9}{\put(0,0){\rule{3pt}{\paperheight}}}}%
 {\color{darkRed!70!purple}\transparent{1}\put(3,0){{\rule{4pt}{\paperheight}}}}
% {\color{logoPeach!80!cyan}\transparent{0.5}{\put(0,700){\rule{1cm}{.6cm}}}}%
% {\color{darkRed!60!cyan}\transparent{0.7}\put(0,706){{\rule{1cm}{.6cm}}}}
% \put(18,726){\thepage}
% \transparent{0.8}
}
}

\AddToShipoutPicture{%
\ifnum\value{page}=1
\put(257.5,942){%
	\transparent{0.7}{
		\includegraphics[width=0.2\textwidth]{logo.png}}}
\put(59,953){\textbf{{\fontfamily{phv}\fontsize{14}{14}\selectfont{}WHITE PAPER}}}
\fi
}	



\AddToShipoutPicture{%
\ifnum\value{page}>1
{\color{blGreen!70!red}\transparent{0.9}{\put(300,8){\rule{0.5\paperwidth}{.3cm}}}}%
{\color{inOne}\transparent{0.8}{\put(300,10){\rule{0.5\paperwidth}{.3cm}}}}%
{\color{inTwo}\transparent{0.3}\put(300,13){{\rule{0.5\paperwidth}{.3cm}}}}

%\ifnum\value{page}<39
\put(301,16){%
\transparent{0.7}{
\includegraphics[width=0.2\textwidth]{logo.png}}
}
%\fi

%\pgfmathparse{equal(\value{page},15)||equal(\value{page},20)?int(1):int(0)}

\ifnum\value{page}=44   %\pgfmathresult>0\relax
\put(182,139){
 {\setlength{\fboxsep}{.65em}\fontsize{9}{10}\selectfont

   {\color{white}{\parbox{11cm}{\vspace{7pt}\framebox{\begin{minipage}{.46\textwidth}
	  {\color{black}{\textit{For more information please contact:}}\\\textbf{\color{blGreen!40!blbl}{Amy Neustein, Ph.D., Founder and CEO}}\\  
       \textbf{{\color{blGreen!20!black}Linguistic Technology Systems \\
      amy.neustein@verizon.net \textbullet{} \textbf{(917) 817-2184} }}}
	 \end{minipage}}}}}}
}
\fi

{\color{blGreen!70!red}\transparent{0.9}{\put(5.6,5){\rule{0.5\paperwidth}{.4cm}}}}%
{\color{inOne}\transparent{1}{\put(5.6,10){\rule{0.5\paperwidth}{.4cm}}}}%
{\color{inTwo}\transparent{0.3}\put(5.6,15){{\rule{0.5\paperwidth}{.4cm}}}}

\fi
}

%\pagestyle{empty} % no page number
%\parskip 7.2pt    % space between paragraphs
%\parindent 12pt   % indent for new paragraph
%\textwidth 4.5in  % width of text
%\columnsep 0.8in  % separation between columns

%\setlength{\footskip}{7pt}

\usepackage[paperheight=14in,paperwidth=8.5in]{geometry}
\geometry{left=.83in,top=.5in,right=.81in,bottom=1.2in} %margins

\usepackage{etoolbox}% http://ctan.org/pkg/etoolbox
\makeatletter
% \patchcmd{<cmd>}{<search>}{<replace>}{<success>}{<failure>}
\patchcmd{\@part}{\par}{\quad}{}{}
\patchcmd{\@part}{\huge}{\Large}{}{}
\makeatother

\renewcommand{\partname}{\hspace{-1em}Part}

\renewcommand*\thepart{\Roman{part}:}

\renewcommand{\thepage}{\raisebox{2pt}{\arabic{page}}}

\renewcommand{\footnoterule}{%
	\kern -3pt
	\hrule width .92\textwidth height .5pt
	\kern 10pt
}


\usepackage[hyphens]{url}
\newcommand{\biburl}[1]{ {\fontfamily{gar}\selectfont{\textcolor[rgb]{.2,.6,0}%
{\scriptsize {\url{#1}}}}}}

%\linespread{1.3}

\newcommand{\sectsp}{\vspace{12pt}}

\usepackage{graphicx}
\usepackage{color,framed}

\usepackage{textcomp}

\usepackage{float}

\usepackage{mdframed}


\usepackage{setspace}
\newcommand{\rpdfNotice}[1]{\begin{onehalfspacing}{

\Large #1

}\end{onehalfspacing}}

\usepackage{xcolor}

\usepackage[hyphenbreaks]{breakurl}
\usepackage[hyphens]{url}

\usepackage{hyperref}
\newcommand{\rpdfLink}[1]{\href{#1}{\small{#1}}}
\newcommand{\dblHref}[1]{\href{#1}{\small{\burl{#1}}}}
\newcommand{\browseHref}[2]{\href{#1}{\Large #2}}

\colorlet{blCyan}{cyan!50!blue}

\definecolor{darkRed}{rgb}{.2,.0,.1}


\definecolor{blGreen}{rgb}{.2,.7,.3}

\definecolor{darkBlGreen}{rgb}{.1,.3,.2}

\definecolor{oldBlColor}{rgb}{.2,.7,.3}

\definecolor{blColor}{rgb}{.1,.3,.2}

\definecolor{elColor}{rgb}{.2,.1,0}
\definecolor{flColor}{rgb}{0.7,0.3,0.3}

\definecolor{logoOrange}{RGB}{108, 18, 30}
\definecolor{logoGreen}{RGB}{85, 153, 89}
\definecolor{logoPurple}{RGB}{200, 208, 30}

\definecolor{logoBlue}{RGB}{4, 2, 25}
\definecolor{logoPeach}{RGB}{255, 159, 102}
\definecolor{logoCyan}{RGB}{66, 206, 244}
\definecolor{logoRed}{rgb}{.3,0,0}

\newcommand{\colorq}[1]{{\color{logoOrange!70!black}{\q{\small\textbf{#1}}}}}

\definecolor{inOne}{rgb}{0.122, 0.435, 0.698}% Rule colour
\definecolor{inTwo}{rgb}{0.122, 0.698, 0.435}% Rule colour

\definecolor{outOne}{rgb}{0.435, 0.698, 0.122}% Rule colour
\definecolor{outTwo}{rgb}{0.698, 0.435, 0.122}% Rule colour

\colorlet{linkcolor}{flColor!60!red}


\hypersetup{
	colorlinks=true,
	citecolor=blCyan!40!green,
	filecolor=magenta!30!logoBlue,
	urlcolor=blue,
    linkcolor=linkcolor!70!black,
%    allcolors=blCyan!40!green
}


\usepackage[many]{tcolorbox}% http://ctan.org/pkg/tcolorbox

\usepackage{transparent}

\newlength{\bsep}
\setlength{\bsep}{-1pt}
\let\xbibitem\bibitem
\renewcommand{\bibitem}[2]{\vspace{\bsep}\xbibitem{#1}{#2}}

\newenvironment{cframed}{\begin{mdframed}[linecolor=logoPeach,linewidth=0.4mm]}{\end{mdframed}}

\newenvironment{ccframed}{\begin{mdframed}[backgroundcolor=logoGreen!5,linecolor=logoCyan!50!black,linewidth=0.4mm]}{\end{mdframed}}


%\usepackage[T1]{fontenc}

%\usepackage{aurical}
% \Fontauri

\usepackage{gfsdidot}
\usepackage[T1]{fontenc}

%\makeatletter
%\f@family,  cmr, T1, n, m,
%\f@encoding,
%\f@shape,
%\f@series,
%\makeatother



%\usepackage{LibreBodoni}

%\usepackage{fontspec}
%\setmainfont{QTBengal}

\usepackage{relsize}

\newcommand{\bref}[1]{\hspace*{1pt}\textbf{\ref{#1}}}

\newcommand{\pseudoIndent}{

\vspace{10pt}\hspace*{12pt}}

\newcommand{\YPDFI}{{\fontfamily{fvs}\selectfont YPDF-Interactive}}

%
\newcommand{\deconum}[1]{{\protect\raisebox{-1pt}{{\LARGE #1}}}}

\newcommand{\visavis}{vis-\`a-vis}

\newcommand{\VersatileUX}{{\color{red!85!black}{\Fontauri Versatile}}%
{{\fontfamily{qhv}\selectfont\smaller UX}}}

\newcommand{\NDPCloud}{{\color{red!15!black}%
{\fontfamily{qhv}\selectfont {\smaller NDP C{\smaller LOUD}}}}}

\newcommand{\MThreeK}{{\color{blGreen!45!black}%
{\fontfamily{qhv}\fontsize{10}{8}\selectfont {M3K}}}}


\newcommand{\lfNDPCloud}{{\color{red!15!black}%
{\fontfamily{qhv}\selectfont N{\smaller DP C{\smaller LOUD}}}}}

\newcommand{\textds}[1]{{\fontfamily{lmdh}\selectfont{%
\raisebox{-1pt}{#1}}}}

\newcommand{\dsC}{{\textds{ds}{\fontfamily{qhv}\selectfont \raisebox{-1pt}
{\color{red!15!black}{C}}}}}

\definecolor{tcolor}{RGB}{24,52,61}

\newcommand{\CCpp}{\resizebox{!}{7pt}{\AcronymText{C}}/\Cpp{}}
\newcommand{\NoSQL}{\resizebox{!}{7pt}{\AcronymText{NoSQL}}}
\newcommand{\SQL}{\resizebox{!}{7pt}{\AcronymText{SQL}}}

\newcommand{\SPARQL}{\resizebox{!}{7pt}{\AcronymText{SPARQL}}}

\newcommand{\NCBI}{\resizebox{!}{7pt}{\AcronymText{NCBI}}}

\newcommand{\HTXN}{\resizebox{!}{7pt}{\ATexttclr{HTXN}}}

\newcommand{\TDM}{\resizebox{!}{7pt}{\AcronymText{TDM}}}

\newcommand{\lHTXN}{\resizebox{!}{7.5pt}{\ATexttclr{H}}%
\resizebox{!}{6.5pt}{\ATexttclr{TXN}}}

\newcommand{\lsHTXN}{\resizebox{!}{9.5pt}{\ATexttclr{HTXN}}}

\newcommand{\LAF}{\resizebox{!}{7pt}{\AcronymText{LAF}}}

\newcommand{\UDpipe}{\resizebox{!}{7pt}{\AcronymText{UDpipe}}}

\newcommand{\C}{\resizebox{!}{7pt}{\AcronymText{C}}}

\newcommand{\FCS}{\resizebox{!}{7pt}{\AcronymText{FCS}}}

\newcommand{\GAVI}{\resizebox{!}{7pt}{\AcronymText{GAVI}}}
\newcommand{\ArcGIS}{\resizebox{!}{7pt}{\AcronymText{ArcGIS}}}
\newcommand{\QGIS}{\resizebox{!}{7pt}{\AcronymText{QGIS}}}

\newcommand{\GIS}{\resizebox{!}{7pt}{\AcronymText{GIS}}}
\newcommand{\AngelScript}{\resizebox{!}{7pt}{\AcronymText{AngelScript}}}



\usepackage{mdframed}

\newcommand{\cframedboxpanda}[1]{\begin{mdframed}[linecolor=yellow!70!blue,linewidth=0.4mm]#1\end{mdframed}}


\newcommand{\PVD}{\resizebox{!}{7pt}{\AcronymText{PVD}}}

\newcommand{\SDK}{\resizebox{!}{7pt}{\AcronymText{SDK}}}
\newcommand{\NLP}{\resizebox{!}{7pt}{\AcronymText{NLP}}}

\newcommand{\AXF}{\resizebox{!}{7pt}{\ATexttclr{AXF}}}

\newcommand{\HyperGraphDB}{\resizebox{!}{7pt}{\AcronymText{HyperGraphDB}}}

\newcommand{\AllegroGraph}{\resizebox{!}{7pt}{\AcronymText{AllegroGraph}}}

\newcommand{\Grakenai}{\resizebox{!}{7pt}{\AcronymText{Graken.ai}}}


\newcommand{\lAXF}{\resizebox{!}{7.5pt}{\ATexttclr{A}}%
\resizebox{!}{6.5pt}{\ATexttclr{XF}}}


\newcommand{\lsAXF}{\resizebox{!}{8.5pt}{\ATexttclr{AXF}}}

\newcommand{\AXFD}{\resizebox{!}{7pt}{\ATexttclr{AXFD}}}

\newcommand{\CBICA}{\resizebox{!}{7pt}{\AcronymText{CBICA}}}

\newcommand{\IORT}{\resizebox{!}{7pt}{\AcronymText{IORT}}}

\newcommand{\openCyto}{\resizebox{!}{7pt}{\AcronymText{openCyto}}}
\newcommand{\FACs}{\resizebox{!}{7pt}{\AcronymText{FACsanadu}}}



\newcommand{\SeDI}{\resizebox{!}{7pt}{\AcronymText{SeDI}}}
\newcommand{\RSNA}{\resizebox{!}{7pt}{\AcronymText{RSNA}}}

\newcommand{\CER}{\resizebox{!}{7pt}{\AcronymText{CER}}}
\newcommand{\PACS}{\resizebox{!}{7pt}{\AcronymText{PACS}}}

\newcommand{\DICOM}{\resizebox{!}{7pt}{\AcronymText{DICOM}}}
\newcommand{\lsDICOM}{\resizebox{!}{9pt}{\AcronymText{DICOM}}}

\newcommand{\CytometryML}{\resizebox{!}{7pt}{\AcronymText{CytometryML}}}

\newcommand{\FCM}{\resizebox{!}{7pt}{\AcronymText{FCM}}}

\newcommand{\OMETIFF}{\resizebox{!}{7pt}{\AcronymText{OME-TIFF}}}
\newcommand{\OME}{\resizebox{!}{7pt}{\AcronymText{OME}}}


\newcommand{\CT}{\resizebox{!}{7pt}{\AcronymText{CT}}}

\newcommand{\LOINC}{\resizebox{!}{7pt}{\AcronymText{LOINC}}}

\newcommand{\RadLex}{\resizebox{!}{7pt}{\AcronymText{RadLex}}}


\newcommand{\OMOP}{\resizebox{!}{7pt}{\AcronymText{OMOP}}}
\newcommand{\PCORnet}{\resizebox{!}{7pt}{\AcronymText{PCORnet}}}
\newcommand{\FHIR}{\resizebox{!}{7pt}{\AcronymText{FHIR}}}

\newcommand{\CaPTk}{\resizebox{!}{7pt}{\AcronymText{CaPTk}}}

\newcommand{\VIOLIN}{\resizebox{!}{7pt}{\AcronymText{VIOLIN}}}



\newcommand{\lAXFD}{\resizebox{!}{7.5pt}{\ATexttclr{A}}%
\resizebox{!}{6.5pt}{\ATexttclr{XFD}}}


\newcommand{\IJST}{\resizebox{!}{7pt}{\AcronymText{IJST}}}

\newcommand{\BioC}{\resizebox{!}{7pt}{\AcronymText{BioC}}}

\newcommand{\CoNLL}{\resizebox{!}{7pt}{\AcronymText{CoNLL}}}
\newcommand{\CoNLLU}{\resizebox{!}{7pt}{\AcronymText{CoNLL-U}}}

\newcommand{\sapp}{\resizebox{!}{7pt}{\AcronymText{Sapien+}}}
\newcommand{\lsapp}{\resizebox{!}{8.5pt}{\AcronymText{Sapien+}}}
\newcommand{\lssapp}{\resizebox{!}{9.5pt}{\AcronymText{Sapien+}}}

\newcommand{\ePub}{\resizebox{!}{7pt}{\AcronymText{ePub}}}

%\lsLPF


\newcommand{\GIT}{\resizebox{!}{7pt}{\AcronymText{GIT}}}

%\definecolor{atColor}{RGB}{11, 71, 17}


\DeclareMathVersion{fordg}
\SetSymbolFont{letters}{fordg}{OML}{cmr}{b}{n}

\definecolor{atcColor}{RGB}{96, 17, 12}
%\textcolor{tcolor}{

\newcommand{\ATextCClr}[1]{\textcolor{atcColor}{\textbf{#1}}}

\newcommand{\ATexttclr}[1]{\textcolor{tcolor}{\textbf{#1}}}

\newcommand{\AIMConc}{\resizebox{!}{7.5pt}{\ATextCClr{AIM-Concepts}}}
\newcommand{\lAIMConc}{\resizebox{!}{8pt}{\ATextCClr{AIM-Concepts}}}

\newcommand{\HGXF}{{\resizebox{!}{7.5pt}{\ATexttclr{HGXF}}}}
\newcommand{\lHGXF}{{\resizebox{!}{8pt}{\ATexttclr{HGXF}}}}
\newcommand{\sHGXF}{{\resizebox{!}{6pt}{\ATexttclr{HGXF}}}}

\newcommand{\CRtwo}{{\resizebox{!}{7.5pt}{\ATextCClr{CR2}}}}
\newcommand{\lCRtwo}{{\resizebox{!}{8pt}{\ATextCClr{CR2}}}}
\newcommand{\sCRtwo}{{\resizebox{!}{6pt}{\ATextCClr{CR2}}}}


\newcommand{\THQL}{\resizebox{!}{7.5pt}{\ATexttclr{THQL}}}
\newcommand{\lTHQL}{\resizebox{!}{8pt}{\ATexttclr{THQL}}}

\newcommand{\HDICOM}{\resizebox{!}{7.5pt}{\ATexttclr{{\large h}-DICOM}}}

\newcommand{\hVaImm}{\resizebox{!}{7.5pt}{\ATexttclr{{\large h}-VaImm}}}


\newcommand{\PhaonVI}{\resizebox{!}{7.5pt}{\ATexttclr{Phaon-VI}}}



\definecolor{atColor}{RGB}{50, 22, 40}
\newcommand{\ATextClr}[1]{\textcolor{atColor}{\textbf{#1}}}

\newcommand{\DgDb}{{\mathversion{fordg}%
\makebox{\raisebox{-3pt}{\resizebox{!}{11pt}{\ATextClr{%
\rotatebox{17}{$\varsigma$}}}}\hspace{-4pt}%
\resizebox{!}{6.5pt}{\ATextClr{D\hspace{-2pt}B}}}}}


\newcommand{\lDgDb}{{\mathversion{fordg}%
\resizebox{!}{12pt}{\ATextClr{%
\rotatebox{17}{$\varsigma$}}}}\hspace{-4pt}%
\resizebox{!}{6.5pt}{\ATextClr{D\hspace{-2pt}B}}}}}

\newcommand{\URL}{\resizebox{!}{7pt}{\AcronymText{URL}}}
\newcommand{\CSML}{\resizebox{!}{7pt}{\AcronymText{CSML}}}
\newcommand{\LPF}{\resizebox{!}{7pt}{\AcronymText{LPF}}}
\newcommand{\lLPF}{\resizebox{!}{8.5pt}{\AcronymText{LPF}}}
\newcommand{\lsLPF}{\resizebox{!}{9.5pt}{\AcronymText{LPF}}}

\newcommand{\AI}{\resizebox{!}{7.5pt}{\AcronymText{AI}}}
\newcommand{\lAI}{\resizebox{!}{8pt}{\AcronymText{AI}}}

\newcommand{\Jupyter}{\resizebox{!}{7pt}{\AcronymText{Jupyter}}}
\newcommand{\Python}{\resizebox{!}{7pt}{\AcronymText{Python}}}
\newcommand{\IDN}{\resizebox{!}{7pt}{\AcronymText{IDN}}}
\newcommand{\JPG}{\resizebox{!}{7pt}{\AcronymText{JPG}}}
\newcommand{\JPEG}{\resizebox{!}{7pt}{\AcronymText{JPEG}}}
\newcommand{\PNG}{\resizebox{!}{7pt}{\AcronymText{PNG}}}
\newcommand{\TIFF}{\resizebox{!}{7pt}{\AcronymText{TIFF}}}
\newcommand{\REPL}{\resizebox{!}{7pt}{\AcronymText{REPL}}}

\newcommand{\Pandore}{\resizebox{!}{7pt}{\AcronymText{Pandore}}}

\newcommand{\MIFlowCyt}{\resizebox{!}{7pt}{\AcronymText{MIFlowCyt}}}
\newcommand{\GatingML}{\resizebox{!}{7pt}{\AcronymText{Gating-ML}}}
\newcommand{\flowCL}{\resizebox{!}{7pt}{\AcronymText{flowCL}}}


\makeatletter

\newcommand*\getX[1]{\expandafter\getX@i#1\@nil}

\newcommand*\getY[1]{\expandafter\getY@i#1\@nil}
\def\getX@i#1,#2\@nil{#1}
\def\getY@i#1,#2\@nil{#2}
\makeatother
	
\newcommand{\rectann}[9]{%
\path [draw=#1,draw opacity=#2,line width=#3, fill=#4, fill opacity = #5, even odd rule] %
(#6) rectangle(\getX{#6}+#7,\getY{#6}+#8)
({\getX{#6}+((#7-(#7*#9))/2)},{\getY{#6}+((#8-(#8*#9))/2)}) rectangle %
({\getX{#6}+((#7-(#7*#9))/2)+#7*#9},{\getY{#6}+((#8-(#8*#9))/2)+#8*#9});}


\definecolor{pfcolor}{RGB}{94, 54, 73}

\newcommand{\EPF}{\resizebox{!}{7pt}{\AcronymText{ETS{\color{pfcolor}pf}}}}
\newcommand{\lEPF}{\resizebox{!}{8.5pt}{\AcronymText{ETS{\color{pfcolor}pf}}}}
\newcommand{\lsEPF}{\resizebox{!}{9.5pt}{\AcronymText{ETS{\color{pfcolor}pf}}}}


\newcommand{\XPDF}{\resizebox{!}{7pt}{\AcronymText{XPDF}}}

\newcommand{\GRE}{\resizebox{!}{7pt}{\AcronymText{GRE}}}
\newcommand{\CAS}{\resizebox{!}{7pt}{\AcronymText{CAS}}}

\newcommand{\lMOSAIC}{%
\resizebox{!}{8pt}{\ATexttclr{M}}%
\resizebox{!}{6pt}{\ATexttclr{OSAIC}}}

\newcommand{\XML}{\resizebox{!}{7pt}{\AcronymText{XML}}}
\newcommand{\RDF}{\resizebox{!}{7pt}{\AcronymText{RDF}}}
\newcommand{\DOM}{\resizebox{!}{7pt}{\AcronymText{DOM}}}

\newcommand{\Java}{\resizebox{!}{7pt}{\AcronymText{Java}}}


\newcommand{\ParaView}{\resizebox{!}{7pt}{\AcronymText{ParaView}}}
\newcommand{\Octave}{\resizebox{!}{7pt}{\AcronymText{Octave}}}
\newcommand{\ROOT}{\resizebox{!}{7pt}{\AcronymText{ROOT}}}
\newcommand{\CERN}{\resizebox{!}{7pt}{\AcronymText{CERN}}}
\newcommand{\MQFour}{\resizebox{!}{7pt}{\AcronymText{MQ4}}}
\newcommand{\VISSION}{\resizebox{!}{7pt}{\AcronymText{VISSION}}}

\newcommand{\ReproZip}{\resizebox{!}{7pt}{\AcronymText{ReproZip}}}
\newcommand{\BioCoder}{\resizebox{!}{7pt}{\AcronymText{BioCoder}}}

\newcommand{\Covid}{\resizebox{!}{7pt}{\AcronymText{Covid-19}}}


\newcommand{\HMCL}{{\resizebox{!}{7.5pt}{\ATexttclr{HMCL}}}}
\newcommand{\DSPIN}{{\resizebox{!}{7.5pt}{\ATexttclr{D-SPIN}}}}

\newcommand{\Pandore}{{\resizebox{!}{7.5pt}{\ATexttclr{Pandore}}}}

\newcommand{\lDSPIN}{{\resizebox{!}{8pt}{\ATexttclr{D-SPIN}}}}
\newcommand{\lsDSPIN}{{\resizebox{!}{8.5pt}{\ATexttclr{D-SPIN}}}}

\newcommand{\CLang}{\resizebox{!}{7pt}{\AcronymText{C}}}

\newcommand{\HNaN}{\resizebox{!}{7pt}{\AcronymText{HN%
\textsc{a}N}}}

\newcommand{\JSON}{\resizebox{!}{7pt}{\AcronymText{JSON}}}
\newcommand{\UV}{\resizebox{!}{7pt}{\AcronymText{UV}}}


\newcommand{\PET}{\resizebox{!}{7pt}{\AcronymText{PET}}}
\newcommand{\MRI}{\resizebox{!}{7pt}{\AcronymText{MRI}}}


\newcommand{\MeshLab}{\resizebox{!}{7pt}{\AcronymText{MeshLab}}}
\newcommand{\IQmol}{\resizebox{!}{7pt}{\AcronymText{IQmol}}}

\newcommand{\SGML}{\resizebox{!}{7pt}{\AcronymText{SGML}}}

\newcommand{\WhiteDB}{\resizebox{!}{7pt}{\AcronymText{\makebox{WhiteDB}}}}

\newcommand{\CrossRef}{\resizebox{!}{7pt}{\AcronymText{CrossRef}}}

\newcommand{\ASCII}{\resizebox{!}{7pt}{\AcronymText{ASCII}}}

\newcommand{\GUI}{\resizebox{!}{7pt}{\AcronymText{GUI}}}
\newcommand{\UI}{\resizebox{!}{7pt}{\AcronymText{UI}}}


\newcommand{\URI}{\resizebox{!}{7pt}{\AcronymText{URI}}}
\newcommand{\DTD}{\resizebox{!}{7pt}{\AcronymText{DTD}}}

\newcommand{\API}{\resizebox{!}{7pt}{\AcronymText{API}}}

\newcommand{\JATS}{\resizebox{!}{7pt}{\AcronymText{JATS}}}


\newcommand{\SDI}{\resizebox{!}{7pt}{\AcronymText{SDI}}}
\newcommand{\SDIV}{\resizebox{!}{7pt}{\AcronymText{SDIV}}}

\definecolor{atColor}{RGB}{50, 22, 40}
\newcommand{\ATextClr}[1]{\textcolor{atColor}{\textbf{#1}}}

\newcommand{\DgDb}{\makebox{\raisebox{-3pt}{\resizebox{!}{11pt}{\ATextClr{%
\rotatebox{17}{$\varsigma$}}}}\hspace{-4pt}%
\resizebox{!}{6.5pt}{\ATextClr{D\hspace{-2pt}B}}}}

\newcommand{\lDgDb}{\makebox{\raisebox{-3pt}{%
\resizebox{!}{12pt}{\ATextClr{%
\rotatebox{17}{$\varsigma$}}}}\hspace{-4pt}%
\resizebox{!}{6.5pt}{\ATextClr{D\hspace{-2pt}B}}}}


\newcommand{\IDE}{\resizebox{!}{7pt}{\AcronymText{IDE}}}

\newcommand{\OWL}{\resizebox{!}{7pt}{\AcronymText{OWL}}}

\newcommand{\Kaggle}{\resizebox{!}{7pt}{\AcronymText{Kaggle}}}


\newcommand{\ViSion}{\resizebox{!}{7pt}{\AcronymText{ViSion}}}

\newcommand{\CWL}{\resizebox{!}{7pt}{\AcronymText{CWL}}}

\newcommand{\ThreeD}{\resizebox{!}{7pt}{\AcronymText{3D}}}
\newcommand{\TwoD}{\resizebox{!}{7pt}{\AcronymText{2D}}}

\newcommand{\medInria}{\resizebox{!}{7pt}{\AcronymText{medInria}}}
\newcommand{\ThreeDimViewer}{\resizebox{!}{7pt}{\AcronymText{3DimViewer}}}

\newcommand{\FAIR}{\resizebox{!}{7pt}{\AcronymText{FAIR}}}

\newcommand{\MIAPEGI}{\resizebox{!}{7pt}{\AcronymText{MIAPE-GI}}}
\newcommand{\MIAPE}{\resizebox{!}{7pt}{\AcronymText{MIAPE}}}



\newcommand{\QNetworkManager}{\resizebox{!}{7pt}{\AcronymText{QNetworkManager}}}
\newcommand{\QTextDocument}{\resizebox{!}{7pt}{\AcronymText{QTextDocument}}}
\newcommand{\QWebEngineView}{\resizebox{!}{7pt}{\AcronymText{QWebEngineView}}}
\newcommand{\HTTP}{\resizebox{!}{7pt}{\AcronymText{HTTP}}}


\newcommand{\lAcronymTextNC}[2]{{\fontfamily{fvs}\selectfont {\Large{#1}}{\large{#2}}}}

\newcommand{\AcronymTextNC}[1]{{\fontfamily{fvs}\selectfont {\large #1}}}


\colorlet{orr}{orange!60!red}

\newcommand{\textscc}[1]{{\color{orr!35!black}{{%
						\fontfamily{Cabin-TLF}\fontseries{b}\selectfont{\textsc{\scriptsize{#1}}}}}}}


\newcommand{\textsccserif}[1]{{\color{orr!35!black}{{%
				\scriptsize{\textbf{#1}}}}}}


\newcommand{\iXPDF}{\resizebox{!}{7pt}{\textsccserif{%
\textit{XPDF}}}}

\newcommand{\iEPF}{\resizebox{!}{7pt}{\textsccserif{%
\textit{ETSpf}}}}

\newcommand{\iSDI}{\resizebox{!}{7pt}{\textsccserif{%
\textit{SDI}}}}

\newcommand{\iHTXN}{\resizebox{!}{7pt}{\textsccserif{%
\textit{HTXN}}}}


\newcommand{\AcronymText}[1]{{\textscc{#1}}}

\newcommand{\AcronymTextser}[1]{{\textsccserif{#1}}}


\newcommand{\mAcronymText}[1]{{\textscc{\normalsize{#1}}}}

\newcommand{\FASTA}{{\resizebox{!}{7pt}{\AcronymText{FASTA}}}}
\newcommand{\SRA}{{\resizebox{!}{7pt}{\AcronymText{SRA}}}}
\newcommand{\DNA}{{\resizebox{!}{7pt}{\AcronymText{DNA}}}}
\newcommand{\MAP}{{\resizebox{!}{7pt}{\AcronymText{MAP}}}}
\newcommand{\EPS}{{\resizebox{!}{7pt}{\AcronymText{EPS}}}}
\newcommand{\CSV}{{\resizebox{!}{7pt}{\AcronymText{CSV}}}}
\newcommand{\PDB}{{\resizebox{!}{7pt}{\AcronymText{PDB}}}}

\newcommand{\WebGL}{{\resizebox{!}{7pt}{\AcronymText{WebGL}}}}
\newcommand{\Docker}{{\resizebox{!}{7pt}{\AcronymText{Docker}}}}


\newcommand{\OBO}{{\resizebox{!}{7pt}{\AcronymText{OBO}}}}

\newcommand{\XOCS}{{\resizebox{!}{7pt}{\AcronymText{XOCS}}}}

\newcommand{\ChemXML}{{\resizebox{!}{7pt}{\AcronymText{ChemXML}}}}

\newcommand{\TeXMECS}{\resizebox{!}{7pt}{\AcronymText{TeXMECS}}}

% pmml  arff  openannotation

\newcommand{\PMML}{\resizebox{!}{7pt}{\AcronymText{PMML}}}
\newcommand{\ARFF}{\resizebox{!}{7pt}{\AcronymText{ARFF}}}
\newcommand{\IeXML}{\resizebox{!}{7pt}{\AcronymText{IeXML}}}

\newcommand{\SeCo}{\resizebox{!}{7pt}{\AcronymText{SeCo}}}


\newcommand{\HDFFive}{\resizebox{!}{7pt}{\AcronymText{HDF5}}}

\newcommand{\NGML}{\resizebox{!}{7pt}{\ATexttclr{NGML}}}


\newcommand{\Cpp}{\resizebox{!}{7pt}{\AcronymText{C++}}}

%\newcommand{\\WhiteDB{}}{\resizebox{!}{7pt}{\AcronymText{\WhiteDB{}}}}

\colorlet{drp}{darkRed!70!purple}

%\newcommand{\MOSAIC}{{\color{drp}{\AcronymTextNC{\scriptsize{MOSAIC}}}}}

\newcommand{\MOSAIC}{\resizebox{!}{7pt}{\ATexttclr{MOSAIC}}}


\newcommand{\mMOSAIC}{{\color{drp}{\ATexttclr{\normalsize{MOSAIC}}}}}

\newcommand{\lmMOSAIC}{\ATexttclr{\Large{M}\normalsize{OSAIC}}}

\newcommand{\MOSAICVM}{\mMOSAIC-\AcronymTextNC{VM}}
\newcommand{\MOSAICSR}{\resizebox{!}{7pt}{%
\mMOSAIC-\AcronymTextNC{\textcolor{drp!25!black}{SR}}}}

\newcommand{\MOSAICSD}{\resizebox{!}{7pt}{%
\mMOSAIC-\AcronymTextNC{\textcolor{drp!25!black}{SD}}}}

\newcommand{\MOSAICGIS}{\resizebox{!}{7pt}{%
\mMOSAIC-\AcronymTextNC{\textcolor{drp!25!black}{GIS}}}}


\newcommand{\lMOSAICSR}{\resizebox{!}{7pt}{%
\lmMOSAIC-\AcronymTextNC{\textcolor{drp!25!black}{SR}}}}

\newcommand{\lMOSAICSD}{\resizebox{!}{7pt}{%
\lmMOSAIC-\AcronymTextNC{\textcolor{drp!25!black}{SD}}}}


\newcommand{\lsMOSAICSR}{\resizebox{!}{9.5pt}{%
\lMOSAIC-}\resizebox{!}{9pt}{\AcronymTextNC{\textcolor{drp!25!black}{SR}}}}

%\newcommand{\lMOSAICSR}{\resizebox{!}{7pt}{%
%\mMOSAIC-\AcronymTextNC{SR}}}


\newcommand{\sMOSAICVM}{\resizebox{!}{7pt}{\MOSAICVM}}
\newcommand{\sMOSAIC}{\resizebox{!}{7pt}{\MOSAIC}}

\newcommand{\LDOM}{\resizebox{!}{7pt}{\AcronymText{LDOM}}}
\newcommand{\Cnineteen}{\resizebox{!}{7pt}{\AcronymText{CORD-19}}}

\newcommand{\lCnineteen}{\resizebox{!}{7.5pt}{\AcronymText{CORD-19}}}


\newcommand{\MOL}{\resizebox{!}{7pt}{\AcronymText{MOL}}}

\newcommand{\ACL}{\resizebox{!}{7pt}{\AcronymText{ACL}}}

\newcommand{\LXCR}{\resizebox{!}{7pt}{\AcronymText{LXCR}}}
\newcommand{\lLXCR}{\resizebox{!}{8.5pt}{\AcronymText{LXCR}}}
\newcommand{\lsLXCR}{\resizebox{!}{9.5pt}{\AcronymText{LXCR}}}

%\newcommand{\lMOSAIC}{{\color{drp}{\lAcronymTextNC{M}{OSAIC}}}}
\newcommand{\lfMOSAIC}{\resizebox{!}{9pt}{{\color{drp}{\lAcronymTextNC{M}{OSAIC}}}}}

\newcommand{\Mosaic}{\resizebox{!}{7pt}{\MOSAIC}}
\newcommand{\MosaicPortal}{{\color{drp}{\AcronymTextNC{MOSAIC Portal}}}}

\newcommand{\RnD}{\resizebox{!}{7pt}{\AcronymText{R\&D}}}

\newcommand{\MIBBI}{\resizebox{!}{7pt}{\AcronymText{MIBBI}}}

\newcommand{\JVM}{\resizebox{!}{7pt}{\AcronymText{JVM}}}
\newcommand{\ECL}{\resizebox{!}{7pt}{\AcronymText{ECL}}}

\newcommand{\ChaiScript}{\resizebox{!}{7pt}{\AcronymText{ChaiScript}}}

\newcommand{\TCP}{\resizebox{!}{7pt}{\AcronymText{TCP}}}

\newcommand{\lQt}{\resizebox{!}{8.5pt}{\AcronymText{Qt}}}
\newcommand{\QtCpp}{\resizebox{!}{8.5pt}{\AcronymText{Qt/C++}}}
\newcommand{\Qt}{\resizebox{!}{7pt}{\AcronymText{Qt}}}

\newcommand{\QtSQL}{\resizebox{!}{7pt}{\AcronymText{QtSQL}}}

\newcommand{\HTML}{\resizebox{!}{7pt}{\AcronymText{HTML}}}
\newcommand{\PDF}{\resizebox{!}{7pt}{\AcronymText{PDF}}}

\newcommand{\R}{\resizebox{!}{7pt}{\AcronymText{R}}}
\newcommand{\SciXML}{\resizebox{!}{7pt}{\AcronymText{SciXML}}}

\newcommand{\MPF}{\resizebox{!}{7pt}{\ATexttclr{MPF}}}


\newcommand{\lGRE}{\resizebox{!}{7.5pt}{\AcronymText{GRE}}}

\newcommand{\p}[1]{

\vspace{.9em}#1}

\newcommand{\q}[1]{{\fontfamily{qcr}\selectfont ``}#1{\fontfamily{qcr}\selectfont ''}} 

%\newcommand{\deconum}[1]{{\textcircled{#1}}}

\renewcommand{\thesection}{\protect\hspace{-1.5em}}
%\renewcommand{\thesection}{\protect\mbox{\deconum{\Roman{section}}}}
\renewcommand{\thesubsection}{\protect\hspace{-1em}}

\newcommand{\llMOSAIC}{\mbox{{\LARGE MOSAIC}}}
%\newcommand{\lfMOSAIC}{\mbox{M\small{OSAIC}}}

\newcommand{\llMosaic}{\llMOSAIC}
\newcommand{\lMosaic}{\lMOSAIC}
\newcommand{\lfMosaic}{\lfMOSAIC}

%\newcommand{\dsC}{}

\newcommand{\textds}[1]{{\fontfamily{lmdh}\selectfont{%
\raisebox{-1pt}{#1}}}}

\newcommand{\ltextds}[1]{{\fontfamily{lmdh}\fontsize{12}{11}\selectfont{%
\raisebox{-1pt}{#1}}}}

\newcommand{\dsC}{{\textds{ds}{\fontfamily{qhv}\selectfont \raisebox{-1pt}{C}}}}
\newcommand{\ldsC}{{\textds{ds}{\fontfamily{qhv}\selectfont \raisebox{-1pt}{C}}}}

\newcommand{\MdsX}{\resizebox{!}{9pt}{\ATexttclr{\raisebox{-1pt}{{\fontfamily{lmdh}\selectfont M}}\fontfamily{qhv}\selectfont dsX}}}
\newcommand{\lsMdsX}{\resizebox{!}{10.5pt}{\ATexttclr{\raisebox{-1pt}{{\fontfamily{lmdh}\selectfont M}}\fontfamily{qhv}\selectfont dsX}}}
\newcommand{\lMdsX}{\resizebox{!}{9.5pt}{\ATexttclr{\raisebox{-1pt}{{\fontfamily{lmdh}\selectfont M}}\fontfamily{qhv}\selectfont dsX}}}


\newcommand{\llWC}{\mbox{{\LARGE WhiteCharmDB}}}

\newcommand{\llwh}{\mbox{{\LARGE White}}}
\newcommand{\llch}{\mbox{{\LARGE CharmDB}}}

\usepackage{enumitem}
%\usepackage{listings}

\colorlet{dsl}{purple!20!brown}
\colorlet{dslr}{dsl!50!blue}

\setlist[description]{%
  topsep=11pt,
  labelsep=22pt, leftmargin=10pt,
  itemsep=9pt,               % space between items
  %font={\bfseries\sffamily}, % set the label font
  font=\normalfont\bfseries\color{dslr!50!black}, % if colour is needed
}

\setlist[enumerate]{%
  topsep=3pt,               % space before start / after end of list
  itemsep=-2pt,               % space between items
  font={\bfseries\sffamily}, % set the label font
%  font={\bfseries\sffamily\color{red}}, % if colour is needed
}

%\usepackage{tcolorbox}

\newcommand{\slead}[1]{%
\noindent{\raisebox{2pt}{\relscale{1.15}{{{%
\fcolorbox{logoCyan!50!black}{logoGreen!5}{#1}
}}}}}\hspace{.5em}}


\let\OldLaTeX\LaTeX

\renewcommand{\LaTeX}{\resizebox{!}{7pt}{\color{orr!35!black}{\OldLaTeX}}}

\let\OldTeX\TeX

\renewcommand{\TeX}{\resizebox{!}{7pt}{\color{orr!35!black}{\OldTeX}}}


\newcommand{\LargeLaTeX}{\resizebox{!}{8.5pt}{\color{orr!35!black}{\OldLaTeX}}}

\setlength\parindent{0pt}
%\setlength\parindent{24pt}
%%\usepakage{newfile}

\usepackage{hyperref}

\usepackage{etoolbox}

\usepackage{zref-user}

\newwrite\sdiFile
\immediate\openout\sdiFile=\jobname.sdi.txt

\newcommand{\p}[1]{

\vspace{10pt}#1}

\newif\iftabng
\tabngfalse


\usepackage{letltxmacro}
\LetLtxMacro{\oldmmsemi}{\;}
\LetLtxMacro{\oldtbplus}{\+}
\LetLtxMacro{\oldtbgt}{\>}
\LetLtxMacro{\oldmmgt}{\+}

\newcommand{\+}{\iftabng\oldtbplus\else\sss\fi}

\renewcommand{\>}{\iftabng\oldtbplus\else
\ifmmode\oldmmgt\else\sse\sss\fi\fi}

%\renewcommand{\>}{\sse\sss}

\renewcommand{\;}{\relax\ifmmode\oldmmsemi\else\sse\fi}

\newcommand{\writeSDI}[1]{\immediate\write\sdiFile#1}

\newcommand{\emblink}[2]{\href{\#sdi:#1--#2}{\#sdi:#1--#2}}

%\newcount\sdiCounter
%\def\advsdiCounter{\global\advance\sdiCounter by1}

%\newcount\sdiCounterP
%\def\advsdiCounterP{\global\advance\sdiCounterP by1}

%\newcounter{sdiCounter}
\newcounter{sdiCounterP}[page]
\newcounter{sdiCounter}

\def\topt#1{\expandafter\the\dimexpr\dimexpr#1sp\relax\relax}

\makeatletter
%\catcode`\*=10
\newcommand{\sss}{%
\stepcounter{sdiCounterP}
\stepcounter{sdiCounter}
\pdfsavepos\write\sdiFile{!/ SDI_Sentence_Start} 
\write\sdiFile\expandafter{\expandafter$%
\expandafter i\expandafter:%
\expandafter\space\the\c@sdiCounter}
\write\sdiFile\expandafter{\expandafter$%
\expandafter o\expandafter:%
\expandafter\space\the\c@sdiCounterP}
\write\sdiFile\expandafter{\expandafter$%
\expandafter p\expandafter:%
\expandafter\space\thepage^^J%
$x: \topt\pdflastxpos^^J%
$y: \topt\pdflastypos^^J%
/!^^J%
<<>^^J%
}}
%\catcode`\%=14
\makeatother

\makeatletter
\newcommand{\sse}{%
\pdfsavepos\write\sdiFile{!/ SDI_Sentence_End} 
\write\sdiFile\expandafter{\expandafter$%
\expandafter i\expandafter:%
\expandafter\space\the\c@sdiCounter}
\write\sdiFile\expandafter{\expandafter$%
\expandafter o\expandafter:%
\expandafter\space\the\c@sdiCounterP}
\write\sdiFile\expandafter{\expandafter$%
\expandafter p\expandafter:%
\expandafter\space\thepage^^J%
$x: \topt\pdflastxpos^^J%
$y: \topt\pdflastypos^^J%
/!^^J%
<<>^^J%
}}
\makeatother



\newcommand{\lun}[1]{\raisebox{-4pt}{\fontfamily{qcr}\selectfont{%
\LARGE{\textbf{\textcolor{tcolor}{#1}}}}}\vspace{-2pt}}

\newcommand{\inditem}{\itemindent10pt\item}

\usepackage{soul}

\definecolor{hlcolor}{RGB}{114, 54, 203}
\colorlet{hlcol}{hlcolor!35}
\sethlcolor{hlcol}

\makeatletter
\def\SOUL@hlpreamble{%
	\setul{}{3ex}%         !!!change this value!!! default is 2.5ex
	\let\SOUL@stcolor\SOUL@hlcolor
	\SOUL@stpreamble
}
\makeatother

\usepackage{scrextend}
%\vspace*{3em}
\newenvironment{mldescription}{\vspace{1em}%
  \begin{addmargin}[4pt]{1em}
    \setlength{\parindent}{-1em}%
    \newcommand*{\mlitem}[1][]{\vspace{5pt}\par\medskip%
%\colorbox{hlcolor}{\textbf{##1}}\quad}\indent
\hl{ \textbf{##1} }\quad}\indent
}{%
  \end{addmargin}
  \medskip
}

\usepackage{marginnote}

\newcommand{\mnote}[1]{%
\vspace*{-2em}
\reversemarginpar
\raisebox{1em}{\marginnote{\parbox{4em}{%
\begin{mdframed}[innerleftmargin=4pt,
	innerrightmargin=1pt,innertopmargin=1pt,
	linecolor=red!20!cyan,userdefinedwidth=4em,
	topline=false,
	rightline=false]
{{\fontfamily{ppl}\fontsize{12}{0}\selectfont
		\textit{#1}}}
\end{mdframed}}
	}[3em]}}

\newcommand{\mnotel}[1]{%
\vspace*{-2em}
\reversemarginpar
\raisebox{-4em}{\marginnote{\parbox{4em}{%
\begin{mdframed}[innerleftmargin=4pt,
	innerrightmargin=1pt,innertopmargin=1pt,
	linecolor=red!20!cyan,userdefinedwidth=4em,
	topline=false,
	rightline=false]
{{\fontfamily{ppl}\fontsize{12}{0}\selectfont
		\textit{#1}}}
\end{mdframed}}
	}[3em]}}

\newcommand{\mnoteh}[3]{%
	\vspace*{#1}
	\reversemarginpar
	\raisebox{#2}{\marginnote{\parbox{4em}{%
				\begin{mdframed}[innerleftmargin=4pt,
					innerrightmargin=1pt,innertopmargin=1pt,
					linecolor=red!20!cyan,userdefinedwidth=4em,
					topline=false,
					rightline=false]
					{{\fontfamily{ppl}\fontsize{12}{0}\selectfont
							\textit{#3}}}
				\end{mdframed}}
			}[3em]}}


\newcommand{\mnoteb}[1]{%
	\vspace*{1em}
	\reversemarginpar
	\raisebox{1em}{\marginnote{\parbox{4em}{%
				\begin{mdframed}[innerleftmargin=4pt,
					innerrightmargin=1pt,innertopmargin=1pt,
					linecolor=red!20!cyan,userdefinedwidth=4em,
					topline=false,
					rightline=false]
					{{\fontfamily{ppl}\fontsize{12}{0}\selectfont
							\textit{#1}}}
				\end{mdframed}}
			}[3em]}}
	
\usepackage{wrapfig}

\usetikzlibrary{arrows, decorations.markings}
\usetikzlibrary{shapes.arrows}

\newcommand{\curicon}[2]{%
	\node at (#1,#2) [
	draw=black,
	%minimum width=2ex,
	inner sep=.7pt,
	fill=white,
	single arrow,
	single arrow head extend=3pt,
	single arrow head indent=1.5pt,
	single arrow tip angle=45,
	line join=bevel,
	minimum height=4.6mm,
	rotate=115
	] {};
}

\makeatletter
\def\@cite#1#2{[\textbf{#1\if@tempswa , #2\fi}]}
\def\@biblabel#1{[\textbf{#1}]}
\makeatother


%\let\origref\ref
%\renewcommand{\ref}[1]{{\LARGE #1}}

%\def\ref#1{\textbf{\origref{{\LARGE #1}}}}

\setlength{\footnotesep}{0pt}

\renewcommand{\thefootnote}{\textcolor{logoGreen!80!logoBlue}{{\fontfamily{qcr}\fontseries{b}\fontsize{10}{4}\selectfont\arabic{footnote}}}}


\newcommand{\LVee}{{\colorbox{cyan!40!yellow}{\textcolor{red!70!navy}{\textbf{\LARGE$\vee$}}}}}
\newcommand{\LWedge}{{\colorbox{cyan!40!yellow}{\textcolor{red!70!navy}{\textbf{\LARGE$\wedge$}}}}}

\renewcommand{\LVee}{}
\renewcommand{\LWedge}{}


\urlstyle{same}

\usepackage[preserveurlmacro]{breakurl}

\newcommand{\bhref}[1]{\href{#1}{\burl{#1}}}

%\setmainfont{QTChanceryType}

\begin{document}

\setlength{\skip\footins}{18pt}	
	
{\linespread{1.25}\selectfont

\vspace*{1.5em}

\begin{center}
%{\relscale{1.2}{\fontfamily{qcr}\fontseries{b}\selectfont 
%{\colorbox{black}{\color{blue}{\llWC{} Database Engine \\and 
%\llMOSAIC{} Native Application Toolkit}}}}}

\colorlet{ctmp}{logoPeach!20!gray}
\colorlet{ctmpp}{ctmp!90!yellow}
\colorlet{ctmppp}{ctmpp!50!black}
\colorlet{ctmpppp}{ctmppp!90!logoRed}
\colorlet{ctmcyan}{ctmpppp!70!cyan}

\colorlet{ctmppppy}{ctmppp!60!orange}

%{\colorbox{darkBlGreen!30!darkRed}{%
\begin{tcolorbox}
[
%%enhanced,
%%frame hidden,
%interior hidden
arc=2pt,outer arc=0pt,
enhanced jigsaw,
width=\textwidth,
colback=ctmppppy!40,
%colback=ctmcyan!50,
colframe=logoRed!30!darkRed,
drop shadow=logoPurple!50!darkRed,
%boxsep=0pt,
%left=0pt,
%right=0pt,
%top=2pt,
]
%\hspace{22pt}
\begin{minipage}{\textwidth}	
\begin{center}	
{\setlength{\fboxsep}{32pt}
	\relscale{1.2}{{\fontfamily{qcr}\fontseries{b}\selectfont%
{New Database Engineering and 
Archive Construction Technology to Accelerate 
Bio-Imaging, Biomedical Engineering, and Covid-19 Research}
}}}
\end{center}
\end{minipage}
\end{tcolorbox}
\end{center}

%\vspace{1em}
\vspace*{2pt}
\begin{center}
\parbox{.85\textwidth}{%
{\fontfamily{fvs}\fontsize{9}{9}\selectfont   
LTS (Linguistic Technology Systems) is founded by 
Amy Neustein, Ph.D., Series Editor of {\textbf{Speech Technology 
and Text Mining in Medicine and Health Care}} (de Gruyter); 
Editor of {\textbf{\makebox{Advances} in Ubiquitous Computing: 
Cyber-Physical Systems, Smart Cities, 
and Ecological \makebox{Monitoring}}} 
(Elsevier, 2020); 
co-author (with Nathaniel Christen) 
of {\textbf{Cross-Disciplinary Data Integration 
and Conceptual Space Models
for Covid-19}} 
(Elsevier, 2021); and co-editor of 
{\textbf{Medical Image Processing and Machine Learning}}
(Institution of Engineering and 
Technology, forthcoming).}}\end{center}

\vspace*{14pt}	

\textcolor{darkRed}{\textbf{Team}\vspace{3pt}
\hrule}
\begin{description}

\item[Principal Investigator:]  Dr. James A. Rodger,  
Professor of Management Information 
Systems and Decision Sciences at Indiana University of 
Pennsylvania
\vspace{-7pt}
\item[Administrative Officer:]  Dr. Amy Neustein, 
founder of Linguistic Technology Systems (LTS)
\vspace{-7pt}
\item[Contributors]  
\begin{itemize}\item[]

\item Nathaniel Christen, Lead Software Architect, LTS

\item Professor Amita Nandal, 
Department of Department of Computer and 
Communication Engineering at Manipal University, Jaipur

\item Professor Arvind Dhaka, 
Department of Department of Computer and 
Communication Engineering at Manipal University, Jaipur; 
recently visiting scholar at 
University of Varna, Bulgaria

\item Professor Todor Ganchev, Vice Rector of Research 
at University of Varna, Bulgaria

\end{itemize}

\end{description}

\vspace{-3pt}

\textcolor{darkRed}{\hrule}

\p{The \q{\MOSAIC{} Data-Set Explorer} (\MdsX{}) 
and \q{\MOSAIC{} Structured Reporting} (\MOSAICSR{}) 
are tools to help authors develop interactive 
presentations supplementing academic documents 
(\MOSAIC{} is an acronym for \q{Multi-Paradigm 
Ontologies for Scientific and Technical Publications}).
With \MdsX{}, interactive presentations take the 
form of software applications that provide 
access to data sets, analytic techniques, 
or other digitally representable artifacts 
to document or encapsulate research work.  
With \MOSAICSR{}, authors can implement or 
reuse code libraries that report on 
research/experiment methods, workflows, and 
protocols.  Conceptually, \MOSAICSR{} 
is functionally similar to the various 
domain-speciic recommendations collectively 
gathered into the \q{Minimum Information for Biological and Biomedical Investigations} (\MIBBI{}) specifications, 
and indeed one use-case for \MOSAICSR{} is that of 
implementing object models instantiating \MIBBI{} 
policies.  In some contexts, \MOSAICSR{} and 
\MIBBI{} overlap, because elements of 
scientific workflows are sometimes 
algorithms implemented within a code package 
concretizing authors' research.}

\p{\lMOSAICSR{} can express both computational 
workflows that are fully encapsulated by published 
code as well as real-world protocols concerning 
laboratory equipment and physical materials 
or samples under investigation.  In the latter 
guise, \MOSAICSR{} code can employ or instantiate 
standardized terminologies and data structures 
for describing experiments --- such as 
\MIBBI{} policies or \BioCoder{} functions.  
In this case, the role of \MOSAICSR{} code is 
to serve as a serialization/deserialization 
endpoint for sharing research metadata.  
Conversely, when workflows are fully implemented 
within software developed as part of a body 
of research, \MOSAICSR{} can provide a functional 
interface allowing this code to be embedded 
in scientific software.  For these cases, 
\MOSAICSR{} provides a framework for modeling 
how a software component specific to a given 
research project exposes its functionality to 
host and/or networked peer applications.  There 
are also scenarios where both scenarios are 
relevant --- the \MOSAICSR{} code would simultaneously 
document real-world experimental protocols and 
construct a digital interface as part of 
a workflow which is part digital and part 
\q{real-world.}}

\p{This paper will focus on one specific application 
of \MOSAICSR{} in the context of image analysis 
and bioimaging --- specifically, a \q{Data Structure 
Protocol for Image-Analysis Networking} (\DSPIN{}), 
which both extends and adds a narrower focus 
to \MOSAICSR{}.  Software using the \DSPIN{} 
protocol provides a description of image 
processing capabilities which have been 
utilized and/or are functionally exposed 
by code and data associated with a research 
project.  This includes \q{structured reporting} 
of research objectives as well as a concrete 
interface for invoking analyses associated 
with the research (either new algorithms or 
techniques used to obtain reported findings).
\lDSPIN{}, in turn, is based on \CaPTk{} 
(the Cancer Imaging and Phenomics Toolkit) 
and \Pandore{} (an image-processing environment 
which includes both data models and interactive 
software).  The \Pandore{} project encompasses 
an ontology of \q{Image Processing Objectives} 
that provides a structural basis for \DSPIN{}.  
For information about how different objectives 
are merged into workflows, \DSPIN{} 
adopts protocols from \CaPTk{}, particularly 
with respect to implementating 
image-analysis capabilities as 
extensions to a core application, and 
\CaPTk{}'s implementation of the 
Common Workflow Language (\CWL{}).  In effect, 
\DSPIN{} formalizes the data models and 
prototypes adopted by \Pandore{} and 
\CaPTk{} so as to concretize \MOSAICSR{} 
for the specific domain of image processing 
and Computer Vision.  The following sections 
will therefore outline \DSPIN{} features 
in the context of \MOSAICSR{} design principles 
and objectives.}

\section{Meta-Procedural Modeling in 
\protect\lsDSPIN{} and \protect\lsMOSAICSR{}}

\p{Most approaches to modeling research workflows 
involve some concept of \q{meta-objects},\footnote{See 
the \VISSION{} system: \bhref{https://pdfs.semanticscholar.org/1ad7/c459dc4f89f87719af1d7a6f30e6f58dff17.pdf}.}, 
\q{tools} (in the terminology of \CWL{}), 
and \q{transitions} (in the language of Petri Net 
theory).  In \MOSIACSR{}, the analogous concept is 
that of \textit{meta-procedures}, which are analogous 
to ordinary computational procedures but 
add extra sources of information concerning 
input and output parameters.  In general, 
rather than simply passing an imput value 
into an executable routine, metaprocedures 
define steps which can be taken to 
acquire the proper values when needed.  
Aside from ordinary runtime values, the 
most important input sources are methods 
defined on \GUI{} components; command-line 
parameters; file contents; and 
not-yet-evaluated expressions (perhaps 
encapsulated in scripts or function pointers).  
A meta-procedure formulation abstracts the 
acquisition of inputs (or \q{channels}) 
from the concrete procedure or procedures 
which are eventually executed.  Therefore, 
a \MOSAIC{} meta-procedure definition 
has two separate parts: a preamble where 
input sources are described, and an 
executive sequence where concrete procedures 
are indicated.  A \textit{meta-evaluator} 
then operates in accord with these 
definitions, concretizing the input values 
and launching the actual procedure(s).  
For \DSPIN{}, meta-procedures can be defined 
using a framework based on \BioCoder{},\footnote{See 
\bhref{https://jbioleng.biomedcentral.com/articles/10.1186/1754-1611-4-13}} 
but adopted to the imaging and Computer Vision 
context.}

\p{Image analysis methods are often 
described in academic literature in 
terms of methematical formulae and/or 
characterizations of computational environments 
(such as Graphical Processing Units); 
it requires additional construction to 
translate these overviews into actual 
computer code.  Once 
Computer Vision innovations are in fact 
concretely implemented, there is then 
an additional stage of development 
requisite for users to actually enact 
the computations described in the 
research.  Although it is theoretically 
possible to demonstrate novel methods 
within fully self-contained autonomous 
applications, it is more convenient for 
users if research code is integrated 
with existing imaging software.  
The \DSPIN{} interface can then help 
connect new code to existing applications, 
allowing users to access new code's functionaloty 
through \GUI{} actions, command-line invocations, 
or inter-application messaging protocols.}

\p{In addition to practically enabling 
application embedding, \DSPIN{} models 
represent research methods and theories, 
contributing to transparency and 
reusabiltity according to the 
\MIBBI{} and \FAIR{} (Findable, Accessible, 
Interoperable, Reusable) standards.\footnote{See 
\bhref{https://www.researchgate.net/publication/331775411_FAIRness_in_Biomedical_Data_Discovery}.}
This can be achieved, in part, by implementing 
data structures conforming to \PANDORE{} 
Image Processing Objectives.  However, \DSPIN{} 
embeds this logic in an Object-Oriented 
context which allows imaging-specific workflow 
notations to be paired with specifications 
outside of imaging processing in the narrowest 
sense.  This allows \DSPIN{} to be available 
for hybrid computational objectives representations 
which are only partially covered by the imaging 
domain --- analogous to the \MIAPEGI{} 
(Gel Electrophoresis Informatics) component 
of \MIAPE{} (Minimum Information About a Proteomics 
Experiment).  The following section will 
discuss several domains where \DSPIN{} has 
been explicitly integrated with code 
libraries codifying \MIBBI{}-style research protocols.}

\section{\protect\lsDSPIN{} in Contexts Supplemental to Image Processing.}

\subsection{Image Flow Cytometry}
\p{One important use-case for biomedical image processing 
is to analyze cellular microscopy in conjuncton with 
cytmetric experiments which investigate cells and 
cellular-scale entities (such as proteins) indirectly.  
Conceptually, image analysis and flow cytometry 
(\FCM{}) analysis 
are mathematically similar, and some commercial 
cytometry software has been extended with 
image-processing capabilities.  The overlap 
between cytometric and image analysis has also 
inspired attempts to merge cytometry standards, 
such as \MIFlowCyt{} (the Minimum Information about a Flow Cytometry Experiment policy within \MIBBI{}), with bioimaging 
standards such as \DICOM{} (Digital Imaging and 
Communications in Medicine).  One such proposal is 
due to Robert Leif, who argues that \q{The large overlap between imaging and flow cytometry provides strong evidence that both modalities should be covered by the same standard} 
and has formalized an \XML{} language 
(\CytometryML{}) to serve as that 
overarching bridge.\footnote{See 
\bhref{https://spie.org/Publications/Proceedings/Paper/10.1117/12.2295220?SSO=1}.}  The \DSPIN{} project builds 
off this work by introducing its own \FCM{}/\DICOM{} 
hybrid, although in an object-oriented rather 
than \XML{}-based context (discussed further in 
the next section).  As a reference implementation 
for this \DSPIN{} extension, the project 
also provides a pure-\Cpp{} cytometry 
library based on the \openCyto{} and 
\FACsanadu{} libraries, but eliminating 
external dependencies such as \R{} and 
\Java{}.  The \FCM{}/\DICOM{} bridge is 
implemented in this context via a 
\DSPIN{} supplement to \DICOM{} based 
on \q{Semantic \DICOM{},} which is an 
effort to standardize query processing 
within \PACS{} (Picture Archiving and 
Communications Service) workstations and 
to more effectively integrate \DICOM{} with 
clinical data.}

\subsection{A Semantic \protect\lsDICOM{} Object Model}

\p{As a formal representation of imaging workflows, 
\DSPIN{} would reasonably be paired in 
many contexts with \DICOM{}, insofar as \DICOM{} 
represents the canonical standard for exchanging 
medical image data.  For its applications within 
the medical-imaging context, therefore, 
\DSPIN{} provides object-oriented accessors 
to \DICOM{} data such that image-objectives and 
\DICOM{} object models can interoperate.  
This object-oriented foundation also provides a 
basis on which to further integrate clinical 
data in the form of \q{Semantic} \PACS{} 
models.\footnote{See \bhref{https://www.ncbi.nlm.nih.gov/pmc/articles/PMC5119276/}.}  In general, Semantic \PACS{} applications 
draw clinical data from \DICOM{} headers, though it 
would be consistent with the Semantic \DICOM{} 
paradigm to expose this information via 
sufficiently well-structured supplemental 
information packages as well.  This 
arrangement then forms the basis of a  
\MOSAIC{}-Semantic \DICOM{} (\MOSAICSD{}) 
framework which integrates Semantic \DICOM{} 
with \DSPIN{} --- specifically, which defines 
a central \DICOM{} object model affixed 
to both clinical and 
image-processing object models.}

\p{The \MOSAICSD{} object system applies 
not only to \DICOM{} integrated with 
statements of image-processing objectives, 
but also to other biomedical contexts where 
image analysis should be integrated 
with other analytic modalities and 
also with clinical or epidemiological 
information.  For example, Flow 
Cytometry overlaps with clinical data tracking because 
one of \FCM{}'s essential investigative roles 
is to examine patients' immunological 
response to diseases and/or interventions.  
In the context of Covid-19, say --- with respect to 
achieving a deeper understanding of how 
and why SARS-CoV-2 symptoms present differently 
in different patients --- \q{The starting point will likely be a deep characterization of the immune system in patients with different stage of the disease}.\footnote{See \bhref{https://onlinelibrary.wiley.com/doi/full/10.1002/cyto.a.24002}.}  
That is, \FCM{} observations need to be 
matched with clinical data in order to 
classify (and consider statistical correlations 
between) immunology and clinical facts 
(risk factors, sociodemographics, disease 
progression, and so forth).  This analysis 
intrinsically assumes that \FCM{} data can 
be transparently linked with all relevant 
clinical data, but such integration is 
difficult even in \DICOM{}, where \DICOM{} 
headers are specifically for preserving 
patient data across picture-sharing networks  
(there is no analogous \q{header} 
component in \FCS{}, the 
Flow Cytometry Standard).  
In short, \DSPIN{} extensions to 
non-imaging domains can promote 
data integration insofar as \DSPIN{} 
Clinical Object Models, based on 
Semantic \DICOM{}, provide 
an affixation point for clinical data 
in analytic contexts which are 
operationally related to image analysis, 
not just to image analysis per se.}

\subsection{Geo-Imaging and Geographic Information Systems}
\p{Another area where \DSPIN{} provides structured 
object models is that of Geographic Information Systems 
(\GIS{}), and specifically \GIS{} annotations.  
There is a direct link between image processing 
and \GIS{} insofar as identifying geotaggable 
features is one dimension or application of 
Computer Vision.  Effectively manipulating 
geoimaging data requires mathematical 
translations between several different 
coordinate systems, in both two and 
three dimensions.  These coordinate 
transforms --- as well as semantic interpretations 
of geoimage segments (buildings, land features, 
roads, etc.) --- can serve as the basis for 
an object model affixing image-processing 
objectives to \GIS{} workflows.}

\p{In conventional \GIS{} annotation, 
data structures are linked to both 
geospatial coordinates and to 
iconography, or other visual cues, 
allowing locations of interest to be 
indicated on maps.  The actual geotagged 
data structures can be derived from 
any domain; as such any object model 
can be integrated with \GIS{} annotations 
so long as one can assign spatial interpretations 
to the phenomena computationally encapsulated 
by the domain in question.  In a medical 
context, for instance, geotagged data 
may represent the scope of a vaccination campaign 
or the extent of an epidemic, along with 
relevant geographic or civic features 
(villages, medical clinics, national borders, 
and so on).  The actual map as a visible 
digital artefact therefore serves as a 
virtual glue where clinical, geographical, 
governmental, and \GUI{} data are all 
sutured together.  Insofar as geoimaging 
involved analysis of photographed land 
features and/or urban environments, 
image processing information represents a 
further object model that can be 
added to the mix.  Even when dealing solely 
with virtual maps, however --- rather 
than with satellite images or other geospatial 
photography --- the analysis and application-level 
rendering of map features is sufficiently 
similar to image process that a rigorous 
model of \GIS{} integration belongs 
properly within \DSPIN{}, specifically 
within a \MOSAICGIS{} extension.}

\section{Markup Serialization and \protect\q{Grounding}}

\p{The data-integration mechanisms discussed 
in this paper thus far have focused on 
object models and object-oriented programming 
techniques.  While composing 
special-purpose object-based libraries 
specifically tailored to individual  
data-integration problems is a powerful 
tool for solbing such problems, in practice 
data integration initiatives are 
often organized around standards 
for sharing or serializing conformation 
data structures.  As a result, an important 
aspect of data integration is implementing 
proper data-serialization technology.  
In the biomedical and bioimaging context, 
most serialization operates through 
\XML{}, so discussions of \XML{} languages 
are a good foundation for examining 
serialization concerns in general.}

\p{To be sure, standardizing a data 
format by stipulating how \XML{} files 
may encode the data is simpler than 
defining an analogous specification 
in terms of executable computer code.  
For sake of discussion, data structures 
defining a novel image-analysis method 
can serve as a case-study in 
standardization-through-serialization: 
one way to document the shape of any 
relevant data is to explain how an \XML{} 
document will be structured insofar as 
it encodes data accordingly.  Potentially, 
such specification can be a single 
\XML{} \DTD{} file, or an \XML{} sample, 
providing a convenient reference point 
for developers to grasp the underlying 
data model.}

\p{However, the structure of \XML{} document 
does not, in itself, present a clear picture of 
how the information which the document 
represents is semantically organized.  Even 
though \XML{} is processed by computer programs, 
it is not even evident from an \XML{} document or 
schema which \XML{} elements (if any) correspond 
to data types recognized by applications which 
read and/or write the corresponding \XML{} code.  
For example, the \XML{} portion of \OMETIFF{} 
(the principal Open Microscopy Environment imaging 
format) includes an explicit \textbf{Image} element 
(which gathers up all significant image metadata); 
an application reading \OMETIFF{} files 
might therefore introduce a single datatype 
--- analogous to \textbf{itk::Image} from 
the Insight Toolkit imaging libraries --- 
bundling the data in that part of the \OME{} \XML{}.  
In this case there will be a one-to-one correspondance 
between \XML{} structure and application-level 
data types, at least for that one \textbf{Image} node.  
On the other hand, software reading \OMETIFF{} 
information may not manipulate images directly, 
but rather pull out other kinds of metadata, 
such as an experimenter's name or description 
of the microscopic setup.  In this case, 
the application may not have an explicit 
\q{object} representing the image itself, 
but it may still read information about the 
image from child nodes of the \XML{} \textbf{Image} 
element.}

\p{In short, applications can read or use data 
from an \XML{} document in different ways, 
so the document's structure does not itself 
provide a clear picture of how code which 
reads the \XML{} is organized.  This uncertainty 
is significant insofar as one wishes to use 
\XML{} specifications as an indirect strategy 
for documenting parameters and features 
of the data structures which are serialized 
via the relevant \XML{} language.  In effect, 
\XML{} serialization operates on two levels: 
on the one hand, the specific \XML{} document 
provides an encoding of data conforming to a 
given structure; but, at a more abstract 
level, one can model the relationship 
between the surface-level \XML{} node 
structure and the application-level data structures 
thereby serialized.  In \MOSAIC{}, such 
\q{meta-serialization} --- a term used 
to suggest the idea of providing meta-data 
\textit{about} a serialization  
    
 }

\p{}



\end{document}


