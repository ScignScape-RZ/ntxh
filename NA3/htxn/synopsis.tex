\documentclass[10pt,letterpaper]{article}

\usepackage{eso-pic}

\AddToShipoutPictureBG{%

\ifnum\value{page}>0{
\AtTextUpperLeft{
\makebox[18.5cm][r]{
\raisebox{-2.3cm}{%
{\transparent{0.3}{\includegraphics[width=0.29\textwidth]{e-logo.png}}	}} } }
}\fi
}

\AddToShipoutPicture{%
{
 {\color{blGreen!70!red}\transparent{0.9}{\put(0,0){\rule{.55cm}{\paperheight}}}}%
 {\color{darkRed!70!purple}\transparent{1}\put(6,0){{\rule{.3cm}{\paperheight}}}}
% {\color{logoPeach!80!cyan}\transparent{0.5}{\put(0,700){\rule{1cm}{.6cm}}}}%
% {\color{darkRed!60!cyan}\transparent{0.7}\put(0,706){{\rule{1cm}{.6cm}}}}
% \put(18,726){\thepage}
% \transparent{0.8}
}
}



\AddToShipoutPicture{%

\ifnum\value{page}>0


{\color{blGreen!70!red}\transparent{0.9}{\put(300,8){\rule{0.5\paperwidth}{.3cm}}}}%
{\color{inOne}\transparent{0.8}{\put(300,10){\rule{0.5\paperwidth}{.3cm}}}}%
{\color{inTwo}\transparent{0.3}\put(300,13){{\rule{0.5\paperwidth}{.3cm}}}}

\put(301,16){%
\transparent{0.7}{
\includegraphics[width=0.2\textwidth]{logo.png}} }

{\color{blGreen!70!red}\transparent{0.9}{\put(5.6,5){\rule{0.5\paperwidth}{.4cm}}}}%
{\color{inOne}\transparent{1}{\put(5.6,10){\rule{0.5\paperwidth}{.4cm}}}}%
{\color{inTwo}\transparent{0.3}\put(5.6,15){{\rule{0.5\paperwidth}{.4cm}}}}

\fi
}

%\pagestyle{empty} % no page number
%\parskip 7.2pt    % space between paragraphs
%\parindent 12pt   % indent for new paragraph
%\textwidth 4.5in  % width of text
%\columnsep 0.8in  % separation between columns

\usepackage[paperheight=13in,paperwidth=8.25in]{geometry}
\geometry{left=.8in,top=.6in,right=.6in,bottom=1.2in} %margins

\newcommand{\sectsp}{\vspace{12pt}}

\usepackage{graphicx}
\usepackage{color,framed}

\usepackage{float}

\usepackage{mdframed}


\usepackage{setspace}
\newcommand{\rpdfNotice}[1]{\begin{onehalfspacing}{

\Large #1

}\end{onehalfspacing}}

\usepackage{xcolor}

\usepackage[hyphenbreaks]{breakurl}
\usepackage[hyphens]{url}

\usepackage{hyperref}
\newcommand{\rpdfLink}[1]{\href{#1}{\small{#1}}}
\newcommand{\dblHref}[1]{\href{#1}{\small{\burl{#1}}}}
\newcommand{\browseHref}[2]{\href{#1}{\Large #2}}

\hypersetup{
    colorlinks=true,
    linkcolor=cyan,
    filecolor=magenta,
    urlcolor=blue,
}

\urlstyle{same}

\definecolor{blGreen}{rgb}{.2,.7,.3}
\definecolor{darkRed}{rgb}{.2,.0,.1}

\definecolor{darkBlGreen}{rgb}{.1,.3,.2}

\definecolor{oldBlColor}{rgb}{.2,.7,.3}

\definecolor{blColor}{rgb}{.1,.3,.2}

\definecolor{elColor}{rgb}{.2,.1,0}
\definecolor{flColor}{rgb}{0.7,0.3,0.3}

\definecolor{logoOrange}{RGB}{108, 18, 30}
\definecolor{logoGreen}{RGB}{85, 153, 89}
\definecolor{logoPurple}{RGB}{200, 208, 30}

\definecolor{logoBlue}{RGB}{4, 2, 25}
\definecolor{logoPeach}{RGB}{255, 159, 102}
\definecolor{logoCyan}{RGB}{66, 206, 244}
\definecolor{logoRed}{rgb}{.3,0,0}

\definecolor{inOne}{rgb}{0.122, 0.435, 0.698}% Rule colour
\definecolor{inTwo}{rgb}{0.122, 0.698, 0.435}% Rule colour

\definecolor{outOne}{rgb}{0.435, 0.698, 0.122}% Rule colour
\definecolor{outTwo}{rgb}{0.698, 0.435, 0.122}% Rule colour

\usepackage[many]{tcolorbox}% http://ctan.org/pkg/tcolorbox

\usepackage{transparent}

\newenvironment{cframed}{\begin{mdframed}[linecolor=logoPeach,linewidth=0.4mm]}{\end{mdframed}}

\newenvironment{ccframed}{\begin{mdframed}[backgroundcolor=logoGreen!5,linecolor=logoCyan!50!black,linewidth=0.4mm]}{\end{mdframed}}

\usepackage{aurical}
\usepackage[T1]{fontenc}

\usepackage{relsize}

\newcommand{\pseudoIndent}{

\vspace{1pt}\hspace{8pt}}

\newcommand{\YPDFI}{{\fontfamily{fvs}\selectfont YPDF-Interactive}}

%
\newcommand{\deconum}[1]{{\protect\raisebox{-1pt}{{\LARGE #1}}}}



\newcommand{\VersatileUX}{{\color{red!85!black}{\Fontauri Versatile}}%
{{\fontfamily{qhv}\selectfont\smaller UX}}}

\newcommand{\NDPCloud}{{\color{red!15!black}%
{\fontfamily{qhv}\selectfont {\smaller NDP C{\smaller LOUD}}}}}

\newcommand{\lfNDPCloud}{{\color{red!15!black}%
{\fontfamily{qhv}\selectfont N{\smaller DP C{\smaller LOUD}}}}}

\newcommand{\textds}[1]{{\fontfamily{lmdh}\selectfont{%
\raisebox{-1pt}{#1}}}}

\newcommand{\dsC}{{\textds{ds}{\fontfamily{qhv}\selectfont \raisebox{-1pt}
{\color{red!15!black}{C}}}}}

\newcommand{\HTXN}{\resizebox{!}{8pt}{\AcronymText{HTXN}}}
\newcommand{\lHTXN}{\resizebox{!}{8.5pt}{\AcronymText{HTXN}}}

\newcommand{\lMOSAIC}{\resizebox{!}{8.5pt}{\AcronymText{MOSAIC}}}

\newcommand{\XML}{\resizebox{!}{8pt}{\AcronymText{XML}}}
\newcommand{\RDF}{\resizebox{!}{8pt}{\AcronymText{RDF}}}

\newcommand{\GUI}{\resizebox{!}{8pt}{\AcronymText{GUI}}}

\newcommand{\API}{\resizebox{!}{8pt}{\AcronymText{API}}}

\newcommand{\IDE}{\resizebox{!}{8pt}{\AcronymText{IDE}}}

\newcommand{\ThreeD}{\resizebox{!}{8pt}{\AcronymText{3D}}}

\newcommand{\FAIR}{\resizebox{!}{8pt}{\AcronymText{FAIR}}}

\newcommand{\QNetworkManager}{\resizebox{!}{8pt}{\AcronymText{QNetworkManager}}}
\newcommand{\QTextDocument}{\resizebox{!}{8pt}{\AcronymText{QTextDocument}}}
\newcommand{\QWebEngineView}{\resizebox{!}{8pt}{\AcronymText{QWebEngineView}}}
\newcommand{\HTTP}{\resizebox{!}{8pt}{\AcronymText{HTTP}}}


\newcommand{\lAcronymTextNC}[2]{{\fontfamily{fvs}\selectfont {\Large{#1}}{\large{#2}}}}

\newcommand{\AcronymTextNC}[1]{{\fontfamily{fvs}\selectfont {\large #1}}}


\colorlet{orr}{orange!60!red}

\newcommand{\textscc}[1]{{\color{orr!35!black}{{%
						\fontfamily{Cabin-TLF}\fontseries{b}\selectfont{\textsc{\scriptsize{#1}}}}}}}


\newcommand{\textsccserif}[1]{{\color{orr!35!black}{{%
				\scriptsize{\textbf{#1}}}}}}


\newcommand{\AcronymText}[1]{{\textscc{#1}}}

\newcommand{\AcronymTextser}[1]{{\textsccserif{#1}}}


\newcommand{\mAcronymText}[1]{{\textscc{\normalsize{#1}}}}

\newcommand{\TeXMECS}{\resizebox{!}{8pt}{\AcronymText{TeXMECS}}}

\newcommand{\NGML}{\resizebox{!}{8pt}{\AcronymText{NGML}}}

\newcommand{\Cpp}{\resizebox{!}{8.5pt}{\AcronymText{C++}}}

\newcommand{\WhiteDB}{\resizebox{!}{8pt}{\AcronymText{WhiteDB}}}

\colorlet{drp}{darkRed!70!purple}

%\newcommand{\MOSAIC}{{\color{drp}{\AcronymTextNC{\scriptsize{MOSAIC}}}}}

\newcommand{\MOSAIC}{\resizebox{!}{8pt}{\AcronymText{MOSAIC}}}


\newcommand{\mMOSAIC}{{\color{drp}{\AcronymTextNC{\normalsize{MOSAIC}}}}}

\newcommand{\MOSAICVM}{\mMOSAIC-\mAcronymText{VM}}

\newcommand{\sMOSAICVM}{\resizebox{!}{8pt}{\MOSAICVM}}
\newcommand{\sMOSAIC}{\resizebox{!}{8pt}{\MOSAIC}}


%\newcommand{\lMOSAIC}{{\color{drp}{\lAcronymTextNC{M}{OSAIC}}}}
\newcommand{\lfMOSAIC}{\resizebox{!}{9pt}{{\color{drp}{\lAcronymTextNC{M}{OSAIC}}}}}

\newcommand{\Mosaic}{\resizebox{!}{8pt}{\MOSAIC}}
\newcommand{\MosaicPortal}{{\color{drp}{\AcronymTextNC{MOSAIC Portal}}}}

\newcommand{\RnD}{\resizebox{!}{7.5pt}{\AcronymText{R\&D}}}
\newcommand{\QtCpp}{\resizebox{!}{8.5pt}{\AcronymText{Qt/C++}}}
\newcommand{\Qt}{\resizebox{!}{9pt}{\AcronymText{Qt}}}
\newcommand{\QtSQL}{\resizebox{!}{8pt}{\AcronymText{QtSQL}}}

\newcommand{\HTML}{\resizebox{!}{8pt}{\AcronymText{HTML}}}
\newcommand{\PDF}{\resizebox{!}{8pt}{\AcronymText{PDF}}}

\newcommand{\p}{

\vspace{.6em}}

\newcommand{\q}[1]{{\fontfamily{qcr}\selectfont ``}#1{\fontfamily{qcr}\selectfont ''}} 

%\newcommand{\deconum}[1]{{\textcircled{#1}}}


\renewcommand{\thesection}{\protect\mbox{\deconum{\Roman{section}}}}
\renewcommand{\thesubsection}{\arabic{section}.\arabic{subsection}}

\newcommand{\llMOSAIC}{\mbox{{\LARGE MOSAIC}}}
%\newcommand{\lfMOSAIC}{\mbox{M\small{OSAIC}}}

\newcommand{\llMosaic}{\llMOSAIC}
\newcommand{\lMosaic}{\lMOSAIC}
\newcommand{\lfMosaic}{\lfMOSAIC}


\newcommand{\llWC}{\mbox{{\LARGE WhiteCharmDB}}}

\newcommand{\llwh}{\mbox{{\LARGE White}}}
\newcommand{\llch}{\mbox{{\LARGE CharmDB}}}

\usepackage{enumitem}

\setlist[description]{%
  topsep=30pt,               % space before start / after end of list
  itemsep=5pt,               % space between items
  font={\bfseries\sffamily}, % set the label font
%  font={\bfseries\sffamily\color{red}}, % if colour is needed
}

\setlist[enumerate]{%
  topsep=3pt,               % space before start / after end of list
  itemsep=-2pt,               % space between items
  font={\bfseries\sffamily}, % set the label font
%  font={\bfseries\sffamily\color{red}}, % if colour is needed
}

%\usepackage{tcolorbox}

\newcommand{\slead}[1]{%
\noindent{\raisebox{2pt}{\relscale{1.15}{{{%
\fcolorbox{logoCyan!50!black}{logoGreen!5}{#1}
}}}}}\hspace{.5em}}


\let\OldLaTeX\LaTeX

\renewcommand{\LaTeX}{\resizebox{!}{8pt}{\color{orr!35!black}{\OldLaTeX}}}

\newcommand{\LargeLaTeX}{\resizebox{!}{8.5pt}{\color{orr!35!black}{\OldLaTeX}}}


\setlength\parindent{34pt}
%%\usepakage{newfile}

\usepackage{hyperref}

\usepackage{etoolbox}

\usepackage{zref-user}

\newwrite\sdiFile
\immediate\openout\sdiFile=\jobname.sdi.txt

\newcommand{\p}[1]{

\vspace{10pt}#1}

\newif\iftabng
\tabngfalse


\usepackage{letltxmacro}
\LetLtxMacro{\oldmmsemi}{\;}
\LetLtxMacro{\oldtbplus}{\+}
\LetLtxMacro{\oldtbgt}{\>}
\LetLtxMacro{\oldmmgt}{\+}

\newcommand{\+}{\iftabng\oldtbplus\else\sss\fi}

\renewcommand{\>}{\iftabng\oldtbplus\else
\ifmmode\oldmmgt\else\sse\sss\fi\fi}

%\renewcommand{\>}{\sse\sss}

\renewcommand{\;}{\relax\ifmmode\oldmmsemi\else\sse\fi}

\newcommand{\writeSDI}[1]{\immediate\write\sdiFile#1}

\newcommand{\emblink}[2]{\href{\#sdi:#1--#2}{\#sdi:#1--#2}}

%\newcount\sdiCounter
%\def\advsdiCounter{\global\advance\sdiCounter by1}

%\newcount\sdiCounterP
%\def\advsdiCounterP{\global\advance\sdiCounterP by1}

%\newcounter{sdiCounter}
\newcounter{sdiCounterP}[page]
\newcounter{sdiCounter}

\def\topt#1{\expandafter\the\dimexpr\dimexpr#1sp\relax\relax}

\makeatletter
%\catcode`\*=10
\newcommand{\sss}{%
\stepcounter{sdiCounterP}
\stepcounter{sdiCounter}
\pdfsavepos\write\sdiFile{!/ SDI_Sentence_Start} 
\write\sdiFile\expandafter{\expandafter$%
\expandafter i\expandafter:%
\expandafter\space\the\c@sdiCounter}
\write\sdiFile\expandafter{\expandafter$%
\expandafter o\expandafter:%
\expandafter\space\the\c@sdiCounterP}
\write\sdiFile\expandafter{\expandafter$%
\expandafter p\expandafter:%
\expandafter\space\thepage^^J%
$x: \topt\pdflastxpos^^J%
$y: \topt\pdflastypos^^J%
/!^^J%
<<>^^J%
}}
%\catcode`\%=14
\makeatother

\makeatletter
\newcommand{\sse}{%
\pdfsavepos\write\sdiFile{!/ SDI_Sentence_End} 
\write\sdiFile\expandafter{\expandafter$%
\expandafter i\expandafter:%
\expandafter\space\the\c@sdiCounter}
\write\sdiFile\expandafter{\expandafter$%
\expandafter o\expandafter:%
\expandafter\space\the\c@sdiCounterP}
\write\sdiFile\expandafter{\expandafter$%
\expandafter p\expandafter:%
\expandafter\space\thepage^^J%
$x: \topt\pdflastxpos^^J%
$y: \topt\pdflastypos^^J%
/!^^J%
<<>^^J%
}}
\makeatother



\newcommand{\lun}[1]{{\fontfamily{psv}\selectfont{\LARGE{\underline{#1}}}}}


\begin{document}
	
{\linespread{1.1}\selectfont

\vspace*{-7em}

\begin{center}
%{\relscale{1.2}{\fontfamily{qcr}\fontseries{b}\selectfont 
%{\colorbox{black}{\color{blue}{\llWC{} Database Engine \\and 
%\llMOSAIC{} Native Application Toolkit}}}}}

\colorlet{ctmp}{logoPeach!20!gray}
\colorlet{ctmpp}{ctmp!90!yellow}
\colorlet{ctmppp}{ctmpp!50!black}
\colorlet{ctmpppp}{ctmppp!90!logoRed}

\vspace{2.5em}

%{\colorbox{darkBlGreen!30!darkRed}{%
\begin{tcolorbox}
[
%%enhanced,
%%frame hidden,
%interior hidden
arc=2pt,outer arc=0pt,
enhanced jigsaw,
width=.7\textwidth,
colback=ctmpppp!60,
colframe=logoRed!30!darkRed,
drop shadow=logoPurple!50!darkRed,
%boxsep=0pt,
%left=0pt,
%right=0pt,
%top=2pt,
]
\begin{minipage}{\textwidth}	
\begin{center}		
{\setlength{\fboxsep}{6pt}
	\relscale{1.4}{{\fontfamily{qcr}\fontseries{b}\selectfont%
{HTXN/MOSAIC Synopsis}}}}
\end{center}
\end{minipage}
\end{tcolorbox}

\vspace{10pt}



%}}

\end{center}

\vspace{-2.5em}

%\noindent\lun{Overview}
\fontfamily{ptm}\fontsize{13pt}{18pt}\selectfont
{\sectsp}
\p{This paper will review the Hypergraph Text 
Encoding Protocol (\HTXN{}) and \MOSAIC{} 
(Multi-Paradigm Ontologies for Scientific, Academic, 
and Technical Publications), which is a model 
for describing published data sets and academic/research 
publications.  \lMOSAIC{} defines a protocol 
for embedding document viewers, along with access 
to publishers' metatadata, within scientific and 
technical applications.  This protocol can be 
specialized for individual publishing companies, 
so that each company may have its own customized 
embedded platform targeted at the software  
which scientists use for their research.}

\p{\lMOSAIC{} is designed with an emphasis on 
(1) representing the interconnections between 
publications and data sets and 
(2) describing the software used to properly 
access data sets.  \lMOSAIC{} prioritizes 
self-contained \q{Research Objects} which 
grant informative, interactive access to 
research data without requiring users to 
download external software, either by 
providing a comprehensive code base 
(called a \q{dataset application}) along 
with the raw data, or by using a tool such 
as ReproZip to bundle external dependencies 
into a single executable package.  \lMOSAIC{} 
data (which can be embedded in data sets 
as one branch of metadata conformant to 
the Research Object Protocol, and/or 
exposed via a \MOSAIC{} \API{}) describes 
data/code integration (function signatures, 
test suites, data types, file/executable 
formats and paths, and User Interface 
documentation) as well as data/text integration 
(cross-references between data sets and 
publications via text anchors and embedded data).}

\p{The \MOSAIC{} \API{} is not a single web service but 
rather a prototype that can be reused in different 
publishing contexts.  The \HTXN{} reference implementation 
includes both \MOSAIC{} \API{} client libraries and 
a demonstration server-side utility hosting \MOSAIC{} 
content.  The \MOSAIC{} \API{} client may be 
natively embedded in \QtCpp{} applications, 
via an embedding protocol that can be extended 
to other scientific and publishing-related 
\API{}s (for example, Elsevier's 
\API{} suite and Developer Portal).  The \Qt{} platform undergirds  
many scientific and publishing applications 
--- TeXstudio (for \LaTeX{} processing), 
Mendeley Desktop (for Reference/Citation Management), 
CERN ROOT (CERN's physics/subatomic analytics 
platform), IQmol (for chemistry/molecular physics), 
medInria (for radiology), QGIS (for Geoinformatics), 
MeshLab (for \ThreeD{} modeling), 
OpenSCAD (for \ThreeD{} geometry), 
Octave (a MATLAB emulator), 
ParaView (for data visualization), 
Qt Creator \IDE{} (Integrated Development Environment), 
to name a few.  With \API{} client plugins or 
extensions, any \Qt{}-based application 
can become a platform for accessing 
publication-related information, 
so that the unique capabilities of 
domain-specific technical software can 
augment the reading and preparation of 
technical publications.}

\p{The \lHTXN{} format, in turn, serves as an illustrative 
case-study for embedded 
\API{} clients.  This format was designed with 
four main goals: (1) to construct documents which 
can be expressed in multiple formats 
(such as \LaTeX{}, \XML{}, and \RDF{}); 
(2) to have a flexible grammar and parsing 
engine so that the input language can be adapted 
for different environments; (3) to achieve 
fine-grained integration and cross-referencing between 
documents and data sets; and (4) to support 
a reference implementation encompassing both 
parsers from text files to \HTXN{} (with 
output-generators for \LaTeX{} or \XML{}) and 
a demonstrative server-side \API{} engine.  Data integration is 
facilitated through \textit{Complex Microcitations}, 
which are citable document anchors that embed 
structured data either encoding a data asset directly 
or referencing an external structured resource.  
Complex Microcitations may encode data in 
special-purpose formats pertinent to publications' 
data sets; in these scenarios, \HTXN{} processors 
may be embedded within domain-specific 
applications that can handle the relevant 
domain-specific data.}

\p{Wherever possible, technologies for 
publications should be integrated with technologies 
working with data sets.  Such integration has numerous 
benefits for authors and publishers, including:

\begin{enumerate}[leftmargin=20pt,labelsep=9pt]
\item{} Authors' contributions in the form of 
published data can be recognized as essential 
facets of their research.  For example, 
the novelty of accompanying data can be 
identified as vouching for the originality of 
the research documented in a publication.

\item{} Data sets enhance publications' value 
according to the \q{\FAIR{}} criteria 
(Findable, Accessible, Interoperable, Reusable).  
Data sets make publications more 
\q{discoverable} because publications, 
by virtue of associated research information, can be 
linked to searchable archives of data repositories.

\item{} The formats describing structured 
data comprising and/or summarizing data sets are 
themselves illustrations and encapsulations of 
scientific principles.  Readers who become 
familiar with domain-specific data formats 
and scientific visual/analytic tools 
gain insights into the research paradigms  
and experimental/analytical protocols through 
which shareable data is generated.  

\item{} Publications (in 
formats like \PDF{}) can be used as supplements 
to data sets.  In particular, technology for 
reading documents (such as \PDF{} viewers), 
when embedded in domain-specific technical applications, 
ensure that the proper  
graphical and analytic capabilities 
--- \ThreeD{} models, molecular visualization, 
panoramic photography viewers, physical simulations, 
linguistic annotation, statistical reparameterization, 
etc. --- are available to 
publications' authors, reviewers, and readers.
\end{enumerate}
}

\p{\lHTXN{} has several features augmenting 
these benefits --- for instance, \lHTXN{} 
documents can be paired with scripts demonstrating 
data set features, which can help reviewers 
understand the organization and features of 
publications' data sets, helping to ensure 
that data sets are properly examined and considered 
during the peer-review process.  In the context of 
publication-related \API{}s, then, a typical scenario 
would unfold as follows: \HTXN{} parsers 
are embedded within special-purpose scientific 
or technical applications (take \Qt{} Creator as a 
case-study).  The \HTXN{} documents will 
include domain-specific metadata targeted at the 
host application (which, for \Qt{} Creator, 
may also be a project loaded in the current \IDE{}
session).  Calls to publication-related \API{}s, 
within the host application, can then link 
the metadata situated in local \HTXN{} files to the 
corresponding points in publication-related resources, 
such as the documents' page proofs or 
document/data-set metadata.  
Via this architecture, technical software 
such as \Qt{} Creator, with publisher-specific 
plugins, can serve as document-preparation 
applications usable across multiple stages in the 
publication workflow.}

\end{document}

