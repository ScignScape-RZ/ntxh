\documentclass[10pt,letterpaper]{article}

\usepackage{eso-pic}

\AddToShipoutPictureBG{%

\ifnum\value{page}>0{
\AtTextUpperLeft{
\makebox[18.5cm][r]{
\raisebox{-2.3cm}{%
{\transparent{0.3}{\includegraphics[width=0.29\textwidth]{e-logo.png}}	}} } }
}\fi
}

\AddToShipoutPicture{%
{
 {\color{blGreen!70!red}\transparent{0.9}{\put(0,0){\rule{.55cm}{\paperheight}}}}%
 {\color{darkRed!70!purple}\transparent{1}\put(6,0){{\rule{.3cm}{\paperheight}}}}
% {\color{logoPeach!80!cyan}\transparent{0.5}{\put(0,700){\rule{1cm}{.6cm}}}}%
% {\color{darkRed!60!cyan}\transparent{0.7}\put(0,706){{\rule{1cm}{.6cm}}}}
% \put(18,726){\thepage}
% \transparent{0.8}
}
}



\AddToShipoutPicture{%

\ifnum\value{page}>0


{\color{blGreen!70!red}\transparent{0.9}{\put(300,8){\rule{0.5\paperwidth}{.3cm}}}}%
{\color{inOne}\transparent{0.8}{\put(300,10){\rule{0.5\paperwidth}{.3cm}}}}%
{\color{inTwo}\transparent{0.3}\put(300,13){{\rule{0.5\paperwidth}{.3cm}}}}

\put(301,16){%
\transparent{0.7}{
\includegraphics[width=0.2\textwidth]{logo.png}} }

{\color{blGreen!70!red}\transparent{0.9}{\put(5.6,5){\rule{0.5\paperwidth}{.4cm}}}}%
{\color{inOne}\transparent{1}{\put(5.6,10){\rule{0.5\paperwidth}{.4cm}}}}%
{\color{inTwo}\transparent{0.3}\put(5.6,15){{\rule{0.5\paperwidth}{.4cm}}}}

\fi
}

%\pagestyle{empty} % no page number
%\parskip 7.2pt    % space between paragraphs
%\parindent 12pt   % indent for new paragraph
%\textwidth 4.5in  % width of text
%\columnsep 0.8in  % separation between columns

\usepackage{geometry}
\geometry{left=.9in,top=1in,right=.8in,bottom=1.4in} %margins

\newcommand{\sectsp}{\vspace{6pt}}

\usepackage{graphicx}
\usepackage{color,framed}

\usepackage{float}

\usepackage{mdframed}


\usepackage{setspace}
\newcommand{\rpdfNotice}[1]{\begin{onehalfspacing}{

\Large #1

}\end{onehalfspacing}}

\usepackage{xcolor}

\usepackage[hyphenbreaks]{breakurl}
\usepackage[hyphens]{url}

\usepackage{hyperref}
\newcommand{\rpdfLink}[1]{\href{#1}{\small{#1}}}
\newcommand{\dblHref}[1]{\href{#1}{\small{\burl{#1}}}}
\newcommand{\browseHref}[2]{\href{#1}{\Large #2}}

\hypersetup{
    colorlinks=true,
    linkcolor=cyan,
    filecolor=magenta,
    urlcolor=blue,
}

\urlstyle{same}

\definecolor{blGreen}{rgb}{.2,.7,.3}
\definecolor{darkRed}{rgb}{.2,.0,.1}

\definecolor{darkBlGreen}{rgb}{.1,.3,.2}

\definecolor{oldBlColor}{rgb}{.2,.7,.3}

\definecolor{blColor}{rgb}{.1,.3,.2}

\definecolor{elColor}{rgb}{.2,.1,0}
\definecolor{flColor}{rgb}{0.7,0.3,0.3}

\definecolor{logoOrange}{RGB}{108, 18, 30}
\definecolor{logoGreen}{RGB}{85, 153, 89}
\definecolor{logoPurple}{RGB}{200, 208, 30}

\definecolor{logoBlue}{RGB}{4, 2, 25}
\definecolor{logoPeach}{RGB}{255, 159, 102}
\definecolor{logoCyan}{RGB}{66, 206, 244}
\definecolor{logoRed}{rgb}{.3,0,0}

\definecolor{inOne}{rgb}{0.122, 0.435, 0.698}% Rule colour
\definecolor{inTwo}{rgb}{0.122, 0.698, 0.435}% Rule colour

\definecolor{outOne}{rgb}{0.435, 0.698, 0.122}% Rule colour
\definecolor{outTwo}{rgb}{0.698, 0.435, 0.122}% Rule colour

\usepackage[many]{tcolorbox}% http://ctan.org/pkg/tcolorbox

\usepackage{transparent}

\newenvironment{cframed}{\begin{mdframed}[linecolor=logoPeach,linewidth=0.4mm]}{\end{mdframed}}

\newenvironment{ccframed}{\begin{mdframed}[backgroundcolor=logoGreen!5,linecolor=logoCyan!50!black,linewidth=0.4mm]}{\end{mdframed}}

\usepackage{aurical}
\usepackage[T1]{fontenc}

\usepackage{relsize}

\newcommand{\pseudoIndent}{

\vspace{1pt}\hspace{8pt}}

\newcommand{\YPDFI}{{\fontfamily{fvs}\selectfont YPDF-Interactive}}

%
\newcommand{\deconum}[1]{{\protect\raisebox{-1pt}{{\LARGE #1}}}}



\newcommand{\VersatileUX}{{\color{red!85!black}{\Fontauri Versatile}}%
{{\fontfamily{qhv}\selectfont\smaller UX}}}

\newcommand{\NDPCloud}{{\color{red!15!black}%
{\fontfamily{qhv}\selectfont {\smaller NDP C{\smaller LOUD}}}}}

\newcommand{\lfNDPCloud}{{\color{red!15!black}%
{\fontfamily{qhv}\selectfont N{\smaller DP C{\smaller LOUD}}}}}

\newcommand{\textds}[1]{{\fontfamily{lmdh}\selectfont{%
\raisebox{-1pt}{#1}}}}

\newcommand{\dsC}{{\textds{ds}{\fontfamily{qhv}\selectfont \raisebox{-1pt}
{\color{red!15!black}{C}}}}}

\newcommand{\HTXN}{\resizebox{!}{8pt}{\AcronymText{HTXN}}}
\newcommand{\XML}{\resizebox{!}{8pt}{\AcronymText{XML}}}
\newcommand{\RDF}{\resizebox{!}{8pt}{\AcronymText{RDF}}}

\newcommand{\GUI}{\resizebox{!}{8pt}{\AcronymText{GUI}}}

\newcommand{\QNetworkManager}{\resizebox{!}{8pt}{\AcronymText{QNetworkManager}}}
\newcommand{\QTextDocument}{\resizebox{!}{8pt}{\AcronymText{QTextDocument}}}
\newcommand{\QWebEngineView}{\resizebox{!}{8pt}{\AcronymText{QWebEngineView}}}
\newcommand{\HTTP}{\resizebox{!}{8pt}{\AcronymText{HTTP}}}


\newcommand{\lAcronymTextNC}[2]{{\fontfamily{fvs}\selectfont {\small{#1}}{\scriptsize{#2}}}}

\colorlet{orr}{orange!60!red}

\newcommand{\textscc}[1]{{\color{orr!35!black}{{%
						\fontfamily{Cabin-TLF}\fontseries{b}\selectfont{\textsc{\scriptsize{#1}}}}}}}


\newcommand{\textsccserif}[1]{{\color{orr!35!black}{{%
				\scriptsize{\textbf{#1}}}}}}


\newcommand{\AcronymText}[1]{{\textscc{#1}}}

\newcommand{\AcronymTextser}[1]{{\textsccserif{#1}}}


\newcommand{\mAcronymText}[1]{{\textscc{\normalsize{#1}}}}

\newcommand{\TeXMECS}{\resizebox{!}{8pt}{\AcronymText{TeXMECS}}}

\newcommand{\NGML}{\resizebox{!}{8pt}{\AcronymText{NGML}}}

\newcommand{\Cpp}{\resizebox{!}{8.5pt}{\AcronymText{C++}}}

\newcommand{\WhiteDB}{\resizebox{!}{8pt}{\AcronymText{WhiteDB}}}

\colorlet{drp}{darkRed!70!purple}

\newcommand{\MOSAIC}{{\color{drp}{\AcronymTextNC{\scriptsize{MOSAIC}}}}}

\newcommand{\mMOSAIC}{{\color{drp}{\AcronymTextNC{\normalsize{MOSAIC}}}}}

\newcommand{\MOSAICVM}{\mMOSAIC-\mAcronymText{VM}}

\newcommand{\sMOSAICVM}{\resizebox{!}{8pt}{\MOSAICVM}}
\newcommand{\sMOSAIC}{\resizebox{!}{8pt}{\MOSAIC}}


\newcommand{\lMOSAIC}{{\color{drp}{\lAcronymTextNC{M}{OSAIC}}}}
\newcommand{\lfMOSAIC}{\resizebox{!}{9pt}{{\color{drp}{\lAcronymTextNC{M}{OSAIC}}}}}

\newcommand{\Mosaic}{\resizebox{!}{8pt}{\MOSAIC}}
\newcommand{\MosaicPortal}{{\color{drp}{\AcronymTextNC{MOSAIC Portal}}}}

\newcommand{\RnD}{\resizebox{!}{7.5pt}{\AcronymText{R\&D}}}
\newcommand{\QtCpp}{\resizebox{!}{8.5pt}{\AcronymText{Qt/C++}}}
\newcommand{\Qt}{\resizebox{!}{9pt}{\AcronymText{Qt}}}
\newcommand{\QtSQL}{\resizebox{!}{8pt}{\AcronymText{QtSQL}}}

\newcommand{\HTML}{\resizebox{!}{8pt}{\AcronymText{HTML}}}
\newcommand{\PDF}{\resizebox{!}{8pt}{\AcronymText{PDF}}}

\newcommand{\p}{

\vspace{.8em}}

\newcommand{\q}[1]{{\fontfamily{qcr}\selectfont ``}#1{\fontfamily{qcr}\selectfont ''}} 

%\newcommand{\deconum}[1]{{\textcircled{#1}}}


\renewcommand{\thesection}{\protect\mbox{\deconum{\Roman{section}}}}
\renewcommand{\thesubsection}{\arabic{section}.\arabic{subsection}}

\newcommand{\llMOSAIC}{\mbox{{\LARGE MOSAIC}}}
%\newcommand{\lfMOSAIC}{\mbox{M\small{OSAIC}}}

\newcommand{\llMosaic}{\llMOSAIC}
\newcommand{\lMosaic}{\lMOSAIC}
\newcommand{\lfMosaic}{\lfMOSAIC}


\newcommand{\llWC}{\mbox{{\LARGE WhiteCharmDB}}}

\newcommand{\llwh}{\mbox{{\LARGE White}}}
\newcommand{\llch}{\mbox{{\LARGE CharmDB}}}

\usepackage{enumitem}

\setlist[description]{%
  topsep=30pt,               % space before start / after end of list
  itemsep=5pt,               % space between items
  font={\bfseries\sffamily}, % set the label font
%  font={\bfseries\sffamily\color{red}}, % if colour is needed
}

%\usepackage{tcolorbox}

\newcommand{\slead}[1]{%
\noindent{\raisebox{2pt}{\relscale{1.15}{{{%
\fcolorbox{logoCyan!50!black}{logoGreen!5}{#1}
}}}}}\hspace{.5em}}


\let\OldLaTeX\LaTeX

\renewcommand{\LaTeX}{\resizebox{!}{8pt}{\color{orr!35!black}{\OldLaTeX}}}

\newcommand{\LargeLaTeX}{\resizebox{!}{8.5pt}{\color{orr!35!black}{\OldLaTeX}}}


\setlength\parindent{34pt}
%%\usepakage{newfile}

\usepackage{hyperref}

\usepackage{etoolbox}

\usepackage{zref-user}

\newwrite\sdiFile
\immediate\openout\sdiFile=\jobname.sdi.txt

\newcommand{\p}[1]{

\vspace{10pt}#1}

\newif\iftabng
\tabngfalse


\usepackage{letltxmacro}
\LetLtxMacro{\oldmmsemi}{\;}
\LetLtxMacro{\oldtbplus}{\+}
\LetLtxMacro{\oldtbgt}{\>}
\LetLtxMacro{\oldmmgt}{\+}

\newcommand{\+}{\iftabng\oldtbplus\else\sss\fi}

\renewcommand{\>}{\iftabng\oldtbplus\else
\ifmmode\oldmmgt\else\sse\sss\fi\fi}

%\renewcommand{\>}{\sse\sss}

\renewcommand{\;}{\relax\ifmmode\oldmmsemi\else\sse\fi}

\newcommand{\writeSDI}[1]{\immediate\write\sdiFile#1}

\newcommand{\emblink}[2]{\href{\#sdi:#1--#2}{\#sdi:#1--#2}}

%\newcount\sdiCounter
%\def\advsdiCounter{\global\advance\sdiCounter by1}

%\newcount\sdiCounterP
%\def\advsdiCounterP{\global\advance\sdiCounterP by1}

%\newcounter{sdiCounter}
\newcounter{sdiCounterP}[page]
\newcounter{sdiCounter}

\def\topt#1{\expandafter\the\dimexpr\dimexpr#1sp\relax\relax}

\makeatletter
%\catcode`\*=10
\newcommand{\sss}{%
\stepcounter{sdiCounterP}
\stepcounter{sdiCounter}
\pdfsavepos\write\sdiFile{!/ SDI_Sentence_Start} 
\write\sdiFile\expandafter{\expandafter$%
\expandafter i\expandafter:%
\expandafter\space\the\c@sdiCounter}
\write\sdiFile\expandafter{\expandafter$%
\expandafter o\expandafter:%
\expandafter\space\the\c@sdiCounterP}
\write\sdiFile\expandafter{\expandafter$%
\expandafter p\expandafter:%
\expandafter\space\thepage^^J%
$x: \topt\pdflastxpos^^J%
$y: \topt\pdflastypos^^J%
/!^^J%
<<>^^J%
}}
%\catcode`\%=14
\makeatother

\makeatletter
\newcommand{\sse}{%
\pdfsavepos\write\sdiFile{!/ SDI_Sentence_End} 
\write\sdiFile\expandafter{\expandafter$%
\expandafter i\expandafter:%
\expandafter\space\the\c@sdiCounter}
\write\sdiFile\expandafter{\expandafter$%
\expandafter o\expandafter:%
\expandafter\space\the\c@sdiCounterP}
\write\sdiFile\expandafter{\expandafter$%
\expandafter p\expandafter:%
\expandafter\space\thepage^^J%
$x: \topt\pdflastxpos^^J%
$y: \topt\pdflastypos^^J%
/!^^J%
<<>^^J%
}}
\makeatother



\newcommand{\lun}[1]{{\fontfamily{psv}\selectfont{\LARGE{\underline{#1}}}}}


\begin{document}
	
{\linespread{1.1}\selectfont

\vspace*{-7em}

\begin{center}
%{\relscale{1.2}{\fontfamily{qcr}\fontseries{b}\selectfont 
%{\colorbox{black}{\color{blue}{\llWC{} Database Engine \\and 
%\llMOSAIC{} Native Application Toolkit}}}}}

\colorlet{ctmp}{logoPeach!20!gray}
\colorlet{ctmpp}{ctmp!90!yellow}
\colorlet{ctmppp}{ctmpp!50!black}
\colorlet{ctmpppp}{ctmppp!90!logoRed}

\vspace{1em}

%{\colorbox{darkBlGreen!30!darkRed}{%
\begin{tcolorbox}
[
%%enhanced,
%%frame hidden,
%interior hidden
arc=2pt,outer arc=0pt,
enhanced jigsaw,
width=.7\textwidth,
colback=ctmpppp!60,
colframe=logoRed!30!darkRed,
drop shadow=logoPurple!50!darkRed,
%boxsep=0pt,
%left=0pt,
%right=0pt,
%top=2pt,
]
\begin{minipage}{\textwidth}	
\begin{center}		
{\setlength{\fboxsep}{18pt}
	\relscale{1.4}{{\fontfamily{qcr}\fontseries{b}\selectfont%
{Hypergraph Text Encoding Protocol (\HTXN{})}}}}
\end{center}
\end{minipage}
\end{tcolorbox}

\vspace{11pt}



%}}

\end{center}

\vspace{1em}

\noindent\lun{Overview}
\fontfamily{ptm}\fontsize{13pt}{18pt}\selectfont
{\sectsp}
\p{The Hypergraph Text Encoding Protocol is a 
framework for detailed encoding of text 
documents.  \HTXN{} is not a markup language, 
although the \HTXN{} \q{Reference Implementation} 
includes a new markup format (called \NGML{}, or 
\q{Next Generation Markup Language}) which 
is one way to create \HTXN{} documents.  
Instead, \HTXN{} is a binary encoding that can 
represent the structure of several different 
markup formats, including \XML{}, \LaTeX{}, and 
\RDF{}.  \HTXN{} allows document collections to 
have a common representation, even if the 
original documents are in different 
formats.  \HTXN{} also facilitates converting 
between formats; for example, one document 
can be represented in both \LaTeX{} and \XML{}.}

\p{The motivations for \HTXN{} are both practical and technological.  
Practically speaking, mismatch between document formats 
can be a noticeable impediment to editing and fine-tuning 
publications, and integrating them into digital 
ecosystems.  For example, converting from \LaTeX{} to 
\XML{} discards document anchors and auxiliary data 
that could be used to externally cite specific 
locations in a publication (\q{Micro-Citations} go 
hand-in-hand with fine-grained textual annotations, because 
\PDF{} viewers can, given proper metadata, browse to 
the exact line where a person, data structure, or 
technical concept appears in a document).  Technologically, 
different formats each have their own unique virtues.  
\LargeLaTeX{}, in many people's opinion, creates the 
most visually compelling documents, and also has sophisticated 
auxiliary-file and data-generation features 
(this auxiliary data is increasingly important 
in a publishing industry where text-based 
content, such as \PDF{} files, are only one part of an 
integrated digital platform).  \LargeLaTeX{}, however, has no 
established proofing and change-tracking protocol.  
\XML{} is architecturally well-designed for multi-party 
editing and tracking, but typically yields 
mediocre (\HTML{}-based) renderings.  \RDF{}, by supporting concurrent 
and overlapping markup, is more expressive than 
\XML{} and \LaTeX{}, but it lacks \LaTeX{}'s pre-processing 
(auxiliary data) capabilities as well as \XML{}'s 
post-processing document-interaction functionality.}

\p{Rather than accept the limitations of any one format, 
\HTXN{} embraces a philosophy of encoding documents 
in its own internal system, then generating 
\q{views} onto the document in a range of structures, 
so as to benefit from different formats' features 
at different stages of finalizing a publication.  
An \HTXN{} file may be converted to \LaTeX{} to 
generate auxiliary data, then converted to 
\XML{} for collaborative editing, and finally 
converted to \RDF{} (or another graph representation) 
for searching and indexing.}

\vspace{2em}
\noindent\lun{Features}
{\sectsp}
\p{Apart from its overarching philosophy, 
\HTXN{}'s distinguishing features include:

\begin{enumerate}
\item{} A unified document-structure model which accommodates 
the idiosyncracies of different markup styles, 
such as attribute child nodes in \XML{}, 
optional arguments in \LaTeX{}, subject defaults in 
\RDF{}, and concurrent/overlapping markup as in \TeXMECS{}. 

\item{} An encoding built from the ground up to support 
multi-party editing and collaboration.  Internally, 
\HTXN{} uses \q{stand-off} annotations which logically 
separate markup content from the underlying text.  
Multiple sources can thereby contribute distinct 
markup structure, which may be used, for example, 
to differentiate document changes made by 
authors from those made by editors.

\item{} A convenient architecture for \q{subordinate} documents which 
expose some restricted portion of their \q{parent} documents.  
This is also useful for editing.  Starting with an overall 
\HTXN{} parent, one can selectively create a subordinate document 
which (for example) includes only editable text, yielding 
a simplified version of the document suitable for 
textual revisions.  After editing, changes made to the 
subordinate document are then folded into its parent.

\item{} A detailed system for pairing documents with external 
data and data sets.  In particular, \HTXN{} supports document 
anchors allowing structured data within a data set to 
reference locations in accompanying publications.  These 
back-references could be used, for example, to identify 
points in a publication where there is a definition, 
explanation, review, or visualization of technical concepts 
and/or data aggregates introduced in a data set.  Unlike 
\LaTeX{} references or \HTML{} anchors, these \HTXN{} anchors 
are data-specific and logically separated from textual 
cross-references. 

\item{} Several special features for data-based annotations, 
or ascribing metadata or interpretations to text 
segments which conform to structured data types 
(proper names, acronyms, dates, times, magnitudes, etc.).  
This includes numeric types typically unrepresented 
in most markup formats, such as range-delimited 
integer types and \q{Universal Numbers}, or \q{posits}, 
a recent alternative to floating-point decimals.  

\item{} Functionality for creating data sets directly 
from the text itself.  For instance, example 
sentences used within linguistic publications, 
to illustrate semantic or syntactic principles, 
can be compiled as a sentence-corpus to produce 
a distinct data set associated with the 
original publication.  Larger corpora can then 
be compiled from collections of linguistic 
texts.  Similar techniques can be employed 
in analogous \q{digital humanities} contexts; 
for example, extracting a corpus of discussions 
about cultural artifacts by extracting sentences 
or paragraphs pertaining to an 
externally identifiable object (such as an 
artwork held in a museum).   

\item{} A framework for \q{complex microcitations}, meaning 
individually citable locations in a document which involve 
multipart data structures.  A case study would be linguistic 
examples that display (alongside or within a sentence) supplemental 
visual cues such as intonation, prosody, lexical category, 
morphology, or grammatical-structural annotations.  
Such material needs special typesetting instructions, but 
the visual format in turn depends on formal sentence 
models.  For another example, graphics such as charts 
or plots may represent a specific statistical 
view on a data set; the construction of that distinct 
perspective marshals formal data, such as dependent 
and independent variables, axes, scales, scaling 
factors (e.g. scalar multipliers or logarithms), and 
ranges.  In these examples, textual content is not only 
a presentation layer which merely exposits or 
reviews scientific frameworks; instead, 
the code describing textual structure and appearance 
reflects scientific paradigms and is therefore 
a theoretical artifact in its own right.
In particular, textual anchors and 
typesetting reflects not only raw text but structured 
data whose format and provenance represents specific 
scientific models.  Here it is valuable to 
cross-reference text with data sources:  
typesetting code may be generated directly from a 
data set, while the generated instructions (in \LaTeX{}, 
say) can trigger anchor labels and page references 
to be \q{forward-referenced} into the data set via 
auxiliary files.  Complete cross-references 
thereby integrate raw and textual data, 
yielding \textit{bi-directional} complex microcitations.  
 
\item{} A \Qt{}-based Reference Implementation which conveniently 
documents the \HTXN{} specifications.  With no external 
dependencies apart from \Qt{} Creator (a \Cpp{} Integrated 
Development Environment), authors can construct \HTXN{} documents 
via \NGML{} and, if desired, examine all parsing and encoding code 
(both statically and, dynamically, during document creation).  
The Reference Implementation also supports importing and 
conversions to \HTXN{} from \XML{}, using \Qt{}'s \XML{} module. 

\item{} Options for stashing compiled \HTXN{} documents in a 
database, so that saving files containing text serialization 
(in \NGML{}, for example) is not a prerequisite for maintaining 
persistent copies of \HTXN{} documents.  The Reference 
Implementation supports persistent 
\HTXN{} documents via the \WhiteDB{} database engine.

\item{} A flexible parsing engine which allows input languages 
for generating \HTXN{} documents to be customized and extended.
\end{enumerate}
}

\vspace{2em}
\noindent\lun{Architecture}
{\sectsp}
\p{Internally, \HTXN{} uses a Hypergraph representation where 
nodes are ranges pointing toward character streams in a 
separate multi-layered buffer (zero-length ranges represent 
anchors and locations within documents).  Ranges 
(and so therefore their corresponding nodes) can overlap 
or encompass one another in any combination.  The data 
establishing nodes' ranges and markup specifications (such as 
tag/command names) is separated from the text layers.}

\p{Also external to the text layers are extra data that affects 
the interpretation of individual characters (for 
example, to accommodate special symbols).  By factoring 
out this supplemental information, each \HTXN{} character 
has a consistent bit-length (typically one or two bytes, 
or a 10-bit pack corresponding to two base-32 digits).  
In general, each \HTXN{} character also has a visual 
representation (a \q{glyph}), because non-visual markup is 
modeled via standoff annotations rather than via content 
embedded in the text itself.  In particular, \HTXN{} does 
not internally use \XML{} entity references, Unicode points, 
diacritics, or other variant-length constructions where 
multiple physical characters correspond to one single 
text character.  Instead, any text character needing 
such supplemental detail is flagged via special codes, 
and an external table is used to retrieve that character's 
specifications.  The rationale for this design is to fully 
separate the markup-related processes of defining and 
annotating \textit{ranges}, from the text-encoding requirements 
concerning foreign letters, special symbols, and other 
nonstandard characters.}

\p{Instead of statically modeling character encodings, 
\HTXN{} provides an \textit{interface-based} encoding protocol.  
This means that \HTXN{} will recognize any encoding of a 
character, a character-set, a character-stream, 
and a character-stream offset range, so long as it is 
possible to obtain certain specific pieces of 
information with regard to the 
object in question.  At the lowest level, this 
means that character-sets and character-streams 
are implemented via \Cpp{} classes which inherit 
from an abstract base class.  Users can then 
build their own character sets by implementing 
their own \Cpp{} classes.  At a higher level, one 
character-set class can provide multiple 
character-set instances by loading data 
from an external file.  In either case, 
\HTXN{}'s character-encoding mechanism is 
orthogonal to encodings used for \XML{}, 
\LaTeX{}, Unicode, or any other markup/encoding format 
(though character sets can provide conversions 
to these formats as needed).  This promotes 
\HTXN{}'s capabilities to convert between and 
to integrate multiple markup protocols, 
such as \LaTeX{} and \XML{}.  More generally, it 
allows programmers to use Object-Oriented 
coding techniques to fine-tune the text-encoding 
process more aggressively than is possible 
with most markup and text-encoding formats.}

\vspace{2em}
\noindent\lun{Document Editing}
{\sectsp}
\p{One benefit of \HTXN{} encoding concerns how 
\HTXN{} may be integrated with those application  
which allow multiple users to edit the text at hand.}

\p{As already described, a parent \HTXN{} 
document may be paired with a subordinate 
document providing restricted access to the 
parent.  This schema applies in contexts 
where only \textit{restricted editing} is 
appropriate.  In contrast to 
\textit{unrestricted editing}, restricted editing 
proceeds in the context of rules limiting 
which editing actions are possible.  In particular, 
restrictions may stipulate that only certain 
document content --- such as text that will be 
visible in the final publication --- may be 
modified.  Restrictions may also limit the degree 
to which a document's graph structure is altered.  
Added nodes, for example, may be required to fit 
inside other nodes (disabling interwoven markup that 
is normally accepted in \HTXN{}).  Added nodes 
may also be restricted to a subset of tags/commands, 
such as those providing visual markup 
(like italics and boldface) for published text, 
or metadata which permits annotations without changing 
the text or its appearance.}

\p{In order to enforce such restrictions, editors 
may be prevented from modifying \HTXN{} documents 
as raw text files; this in turn demands software 
which presents views onto documents' contents 
other than through text serialization.  For these use 
cases, \HTXN{} natively supports integration with 
\Qt{} applications.  In particular, subordinate 
\HTXN{} documents can be rendered via the 
\QTextDocument{} class, which allows restricted editing 
of visible textual content without directly 
changing the underlying \HTXN{} data.  
With \QTextDocument{}, editing actions can be 
initiated through a responsive native front end 
supporting context menus, secondary windows, 
undo/redo, and other convenient features 
associated with native desktop \GUI{}s.  
Alternatively, subordinate documents may be 
converted to \HTML{} and visualized through 
\QWebEngineView{}, which offers similar 
interactive functionality.}

\p{Each edit initiated through \QTextDocument{} 
or \QWebEngineView{} may then generate 
a data structure through which the 
changes are registered in subordinate and parent 
\HTXN{} documents.  These data structures, describing 
individual edits and groups of edits, may then be 
shared over \HTTP{} via \QNetworkManager{} and 
related classes.  This allows multiple users 
to synchronize their local document views in 
collaborative-editing contexts, and simultaneously allows 
publishers to track changes and maintain an official 
version for publication.  Maintaining one 
canonical document --- so that 
multiple users (e.g. authors, volume editors, series editors, 
and production editors) may update the document in real time 
--- is obviously more efficient than users 
sending successive copies of their document back 
and forth.  This, of course, is one reason 
why web-based collaborative editing is popular.  
The improvements introduced by \HTXN{} are first that 
co-edited documents are actually subordinate versions 
of parent documents which have more complex typesetting 
and metadata capabilities.  In so doing, the changes 
introduced in the subordinate document are thereby 
(as desired) reflected in the parent document.  Second, the 
editing \GUI{}s can be part of native-compiled applications 
with a full range of desktop-style functionality 
rather than web portals 
which have a more restricted interactive experience. 
In short, with the \HTXN{} framework  
the publication (books and journals) production process 
would be accelerated by virtue of \HTXN{}'s 
native-application features.}

\end{document}

