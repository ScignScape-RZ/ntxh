\documentclass[10pt,letterpaper]{article}

\usepackage{eso-pic}

\AddToShipoutPictureBG{%

\ifnum\value{page}>0{
\AtTextUpperLeft{
\makebox[18.5cm][r]{
\raisebox{-2.3cm}{%
{\transparent{0.3}{\includegraphics[width=0.29\textwidth]{e-logo.png}}	}} } }
}\fi
}

\AddToShipoutPicture{%
{
 {\color{blGreen!70!red}\transparent{0.9}{\put(0,0){\rule{.55cm}{\paperheight}}}}%
 {\color{darkRed!70!purple}\transparent{1}\put(6,0){{\rule{.3cm}{\paperheight}}}}
% {\color{logoPeach!80!cyan}\transparent{0.5}{\put(0,700){\rule{1cm}{.6cm}}}}%
% {\color{darkRed!60!cyan}\transparent{0.7}\put(0,706){{\rule{1cm}{.6cm}}}}
% \put(18,726){\thepage}
% \transparent{0.8}
}
}



\AddToShipoutPicture{%

\ifnum\value{page}>0


{\color{blGreen!70!red}\transparent{0.9}{\put(300,8){\rule{0.5\paperwidth}{.3cm}}}}%
{\color{inOne}\transparent{0.8}{\put(300,10){\rule{0.5\paperwidth}{.3cm}}}}%
{\color{inTwo}\transparent{0.3}\put(300,13){{\rule{0.5\paperwidth}{.3cm}}}}

\put(301,16){%
\transparent{0.7}{
\includegraphics[width=0.2\textwidth]{logo.png}} }

{\color{blGreen!70!red}\transparent{0.9}{\put(5.6,5){\rule{0.5\paperwidth}{.4cm}}}}%
{\color{inOne}\transparent{1}{\put(5.6,10){\rule{0.5\paperwidth}{.4cm}}}}%
{\color{inTwo}\transparent{0.3}\put(5.6,15){{\rule{0.5\paperwidth}{.4cm}}}}

\fi
}

%\pagestyle{empty} % no page number
%\parskip 7.2pt    % space between paragraphs
%\parindent 12pt   % indent for new paragraph
%\textwidth 4.5in  % width of text
%\columnsep 0.8in  % separation between columns

\setlength{\footskip}{23pt}

\usepackage[paperheight=14.5in,paperwidth=9.1in]{geometry}
\geometry{left=.9in,top=1.1in,right=.6in,bottom=1.4in} %margins

\newcommand{\sectsp}{\vspace{12pt}}

\usepackage{graphicx}
\usepackage{color,framed}

\usepackage{float}

\usepackage{mdframed}


\usepackage{setspace}
\newcommand{\rpdfNotice}[1]{\begin{onehalfspacing}{

\Large #1

}\end{onehalfspacing}}

\usepackage{xcolor}

\usepackage[hyphenbreaks]{breakurl}
\usepackage[hyphens]{url}

\usepackage{hyperref}
\newcommand{\rpdfLink}[1]{\href{#1}{\small{#1}}}
\newcommand{\dblHref}[1]{\href{#1}{\small{\burl{#1}}}}
\newcommand{\browseHref}[2]{\href{#1}{\Large #2}}

\hypersetup{
    colorlinks=true,
    linkcolor=cyan,
    filecolor=magenta,
    urlcolor=blue,
}

\urlstyle{same}

\definecolor{blGreen}{rgb}{.2,.7,.3}
\definecolor{darkRed}{rgb}{.2,.0,.1}

\definecolor{darkBlGreen}{rgb}{.1,.3,.2}

\definecolor{oldBlColor}{rgb}{.2,.7,.3}

\definecolor{blColor}{rgb}{.1,.3,.2}

\definecolor{elColor}{rgb}{.2,.1,0}
\definecolor{flColor}{rgb}{0.7,0.3,0.3}

\definecolor{logoOrange}{RGB}{108, 18, 30}
\definecolor{logoGreen}{RGB}{85, 153, 89}
\definecolor{logoPurple}{RGB}{200, 208, 30}

\definecolor{logoBlue}{RGB}{4, 2, 25}
\definecolor{logoPeach}{RGB}{255, 159, 102}
\definecolor{logoCyan}{RGB}{66, 206, 244}
\definecolor{logoRed}{rgb}{.3,0,0}

\definecolor{inOne}{rgb}{0.122, 0.435, 0.698}% Rule colour
\definecolor{inTwo}{rgb}{0.122, 0.698, 0.435}% Rule colour

\definecolor{outOne}{rgb}{0.435, 0.698, 0.122}% Rule colour
\definecolor{outTwo}{rgb}{0.698, 0.435, 0.122}% Rule colour

\usepackage[many]{tcolorbox}% http://ctan.org/pkg/tcolorbox

\usepackage{transparent}

\newenvironment{cframed}{\begin{mdframed}[linecolor=logoPeach,linewidth=0.4mm]}{\end{mdframed}}

\newenvironment{ccframed}{\begin{mdframed}[backgroundcolor=logoGreen!5,linecolor=logoCyan!50!black,linewidth=0.4mm]}{\end{mdframed}}

\usepackage{aurical}
\usepackage[T1]{fontenc}

\usepackage{relsize}

\newcommand{\pseudoIndent}{

\vspace{1pt}\hspace{8pt}}

\newcommand{\YPDFI}{{\fontfamily{fvs}\selectfont YPDF-Interactive}}

%
\newcommand{\deconum}[1]{{\protect\raisebox{-1pt}{{\LARGE #1}}}}



\newcommand{\VersatileUX}{{\color{red!85!black}{\Fontauri Versatile}}%
{{\fontfamily{qhv}\selectfont\smaller UX}}}

\newcommand{\NDPCloud}{{\color{red!15!black}%
{\fontfamily{qhv}\selectfont {\smaller NDP C{\smaller LOUD}}}}}

\newcommand{\lfNDPCloud}{{\color{red!15!black}%
{\fontfamily{qhv}\selectfont N{\smaller DP C{\smaller LOUD}}}}}

\newcommand{\textds}[1]{{\fontfamily{lmdh}\selectfont{%
\raisebox{-1pt}{#1}}}}

\newcommand{\dsC}{{\textds{ds}{\fontfamily{qhv}\selectfont \raisebox{-1pt}
{\color{red!15!black}{C}}}}}

\newcommand{\HTXN}{\resizebox{!}{8pt}{\AcronymText{HTXN}}}
\newcommand{\lHTXN}{\resizebox{!}{8.5pt}{\AcronymText{HTXN}}}
\newcommand{\lsHTXN}{\resizebox{!}{9.5pt}{\AcronymText{HTXN}}}

\newcommand{\sapp}{\resizebox{!}{8pt}{\AcronymText{Sapien+}}}
\newcommand{\lsapp}{\resizebox{!}{8.5pt}{\AcronymText{Sapien+}}}
\newcommand{\lssapp}{\resizebox{!}{9.5pt}{\AcronymText{Sapien+}}}

\newcommand{\XPDF}{\resizebox{!}{8.5pt}{\AcronymText{XPDF}}}
\newcommand{\GRE}{\resizebox{!}{8.5pt}{\AcronymText{GRE}}}

\newcommand{\lMOSAIC}{\resizebox{!}{8.5pt}{\AcronymText{MOSAIC}}}

\newcommand{\XML}{\resizebox{!}{8pt}{\AcronymText{XML}}}
\newcommand{\RDF}{\resizebox{!}{8pt}{\AcronymText{RDF}}}

\newcommand{\CLang}{\resizebox{!}{8pt}{\AcronymText{C}}}

\newcommand{\HNaN}{\resizebox{!}{8pt}{\AcronymText{HN%
\textsc{a}N}}}


\newcommand{\MeshLab}{\resizebox{!}{8pt}{\AcronymText{MeshLab}}}
\newcommand{\IQmol}{\resizebox{!}{8pt}{\AcronymText{IQmol}}}

\newcommand{\GUI}{\resizebox{!}{8pt}{\AcronymText{GUI}}}

\newcommand{\API}{\resizebox{!}{8pt}{\AcronymText{API}}}

\newcommand{\IDE}{\resizebox{!}{8pt}{\AcronymText{IDE}}}

\newcommand{\ThreeD}{\resizebox{!}{8pt}{\AcronymText{3D}}}

\newcommand{\FAIR}{\resizebox{!}{8pt}{\AcronymText{FAIR}}}

\newcommand{\QNetworkManager}{\resizebox{!}{8pt}{\AcronymText{QNetworkManager}}}
\newcommand{\QTextDocument}{\resizebox{!}{8pt}{\AcronymText{QTextDocument}}}
\newcommand{\QWebEngineView}{\resizebox{!}{8pt}{\AcronymText{QWebEngineView}}}
\newcommand{\HTTP}{\resizebox{!}{8pt}{\AcronymText{HTTP}}}


\newcommand{\lAcronymTextNC}[2]{{\fontfamily{fvs}\selectfont {\Large{#1}}{\large{#2}}}}

\newcommand{\AcronymTextNC}[1]{{\fontfamily{fvs}\selectfont {\large #1}}}


\colorlet{orr}{orange!60!red}

\newcommand{\textscc}[1]{{\color{orr!35!black}{{%
						\fontfamily{Cabin-TLF}\fontseries{b}\selectfont{\textsc{\scriptsize{#1}}}}}}}


\newcommand{\textsccserif}[1]{{\color{orr!35!black}{{%
				\scriptsize{\textbf{#1}}}}}}


\newcommand{\AcronymText}[1]{{\textscc{#1}}}

\newcommand{\AcronymTextser}[1]{{\textsccserif{#1}}}


\newcommand{\mAcronymText}[1]{{\textscc{\normalsize{#1}}}}

\newcommand{\TeXMECS}{\resizebox{!}{8pt}{\AcronymText{TeXMECS}}}

\newcommand{\NGML}{\resizebox{!}{8pt}{\AcronymText{NGML}}}

\newcommand{\Cpp}{\resizebox{!}{8.5pt}{\AcronymText{C++}}}

\newcommand{\WhiteDB}{\resizebox{!}{8pt}{\AcronymText{WhiteDB}}}

\colorlet{drp}{darkRed!70!purple}

%\newcommand{\MOSAIC}{{\color{drp}{\AcronymTextNC{\scriptsize{MOSAIC}}}}}

\newcommand{\MOSAIC}{\resizebox{!}{8pt}{\AcronymText{MOSAIC}}}


\newcommand{\mMOSAIC}{{\color{drp}{\AcronymTextNC{\normalsize{MOSAIC}}}}}

\newcommand{\MOSAICVM}{\mMOSAIC-\mAcronymText{VM}}

\newcommand{\sMOSAICVM}{\resizebox{!}{8pt}{\MOSAICVM}}
\newcommand{\sMOSAIC}{\resizebox{!}{8pt}{\MOSAIC}}


%\newcommand{\lMOSAIC}{{\color{drp}{\lAcronymTextNC{M}{OSAIC}}}}
\newcommand{\lfMOSAIC}{\resizebox{!}{9pt}{{\color{drp}{\lAcronymTextNC{M}{OSAIC}}}}}

\newcommand{\Mosaic}{\resizebox{!}{8pt}{\MOSAIC}}
\newcommand{\MosaicPortal}{{\color{drp}{\AcronymTextNC{MOSAIC Portal}}}}

\newcommand{\RnD}{\resizebox{!}{7.5pt}{\AcronymText{R\&D}}}
\newcommand{\QtCpp}{\resizebox{!}{8.5pt}{\AcronymText{Qt/C++}}}
\newcommand{\Qt}{\resizebox{!}{9pt}{\AcronymText{Qt}}}
\newcommand{\QtSQL}{\resizebox{!}{8pt}{\AcronymText{QtSQL}}}

\newcommand{\HTML}{\resizebox{!}{8pt}{\AcronymText{HTML}}}
\newcommand{\PDF}{\resizebox{!}{8pt}{\AcronymText{PDF}}}

\newcommand{\p}{

\vspace{1.2em}}

\newcommand{\q}[1]{{\fontfamily{qcr}\selectfont ``}#1{\fontfamily{qcr}\selectfont ''}} 

%\newcommand{\deconum}[1]{{\textcircled{#1}}}


\renewcommand{\thesection}{\protect\mbox{\deconum{\Roman{section}}}}
\renewcommand{\thesubsection}{\arabic{section}.\arabic{subsection}}

\newcommand{\llMOSAIC}{\mbox{{\LARGE MOSAIC}}}
%\newcommand{\lfMOSAIC}{\mbox{M\small{OSAIC}}}

\newcommand{\llMosaic}{\llMOSAIC}
\newcommand{\lMosaic}{\lMOSAIC}
\newcommand{\lfMosaic}{\lfMOSAIC}


\newcommand{\llWC}{\mbox{{\LARGE WhiteCharmDB}}}

\newcommand{\llwh}{\mbox{{\LARGE White}}}
\newcommand{\llch}{\mbox{{\LARGE CharmDB}}}

\usepackage{enumitem}

\setlist[description]{%
  topsep=30pt,               % space before start / after end of list
  itemsep=5pt,               % space between items
  font={\bfseries\sffamily}, % set the label font
%  font={\bfseries\sffamily\color{red}}, % if colour is needed
}

\setlist[enumerate]{%
  topsep=3pt,               % space before start / after end of list
  itemsep=-2pt,               % space between items
  font={\bfseries\sffamily}, % set the label font
%  font={\bfseries\sffamily\color{red}}, % if colour is needed
}

%\usepackage{tcolorbox}

\newcommand{\slead}[1]{%
\noindent{\raisebox{2pt}{\relscale{1.15}{{{%
\fcolorbox{logoCyan!50!black}{logoGreen!5}{#1}
}}}}}\hspace{.5em}}


\let\OldLaTeX\LaTeX

\renewcommand{\LaTeX}{\resizebox{!}{8pt}{\color{orr!35!black}{\OldLaTeX}}}

\newcommand{\LargeLaTeX}{\resizebox{!}{8.5pt}{\color{orr!35!black}{\OldLaTeX}}}


\setlength\parindent{24pt}
%%\usepakage{newfile}

\usepackage{hyperref}

\usepackage{etoolbox}

\usepackage{zref-user}

\newwrite\sdiFile
\immediate\openout\sdiFile=\jobname.sdi.txt

\newcommand{\p}[1]{

\vspace{10pt}#1}

\newif\iftabng
\tabngfalse


\usepackage{letltxmacro}
\LetLtxMacro{\oldmmsemi}{\;}
\LetLtxMacro{\oldtbplus}{\+}
\LetLtxMacro{\oldtbgt}{\>}
\LetLtxMacro{\oldmmgt}{\+}

\newcommand{\+}{\iftabng\oldtbplus\else\sss\fi}

\renewcommand{\>}{\iftabng\oldtbplus\else
\ifmmode\oldmmgt\else\sse\sss\fi\fi}

%\renewcommand{\>}{\sse\sss}

\renewcommand{\;}{\relax\ifmmode\oldmmsemi\else\sse\fi}

\newcommand{\writeSDI}[1]{\immediate\write\sdiFile#1}

\newcommand{\emblink}[2]{\href{\#sdi:#1--#2}{\#sdi:#1--#2}}

%\newcount\sdiCounter
%\def\advsdiCounter{\global\advance\sdiCounter by1}

%\newcount\sdiCounterP
%\def\advsdiCounterP{\global\advance\sdiCounterP by1}

%\newcounter{sdiCounter}
\newcounter{sdiCounterP}[page]
\newcounter{sdiCounter}

\def\topt#1{\expandafter\the\dimexpr\dimexpr#1sp\relax\relax}

\makeatletter
%\catcode`\*=10
\newcommand{\sss}{%
\stepcounter{sdiCounterP}
\stepcounter{sdiCounter}
\pdfsavepos\write\sdiFile{!/ SDI_Sentence_Start} 
\write\sdiFile\expandafter{\expandafter$%
\expandafter i\expandafter:%
\expandafter\space\the\c@sdiCounter}
\write\sdiFile\expandafter{\expandafter$%
\expandafter o\expandafter:%
\expandafter\space\the\c@sdiCounterP}
\write\sdiFile\expandafter{\expandafter$%
\expandafter p\expandafter:%
\expandafter\space\thepage^^J%
$x: \topt\pdflastxpos^^J%
$y: \topt\pdflastypos^^J%
/!^^J%
<<>^^J%
}}
%\catcode`\%=14
\makeatother

\makeatletter
\newcommand{\sse}{%
\pdfsavepos\write\sdiFile{!/ SDI_Sentence_End} 
\write\sdiFile\expandafter{\expandafter$%
\expandafter i\expandafter:%
\expandafter\space\the\c@sdiCounter}
\write\sdiFile\expandafter{\expandafter$%
\expandafter o\expandafter:%
\expandafter\space\the\c@sdiCounterP}
\write\sdiFile\expandafter{\expandafter$%
\expandafter p\expandafter:%
\expandafter\space\thepage^^J%
$x: \topt\pdflastxpos^^J%
$y: \topt\pdflastypos^^J%
/!^^J%
<<>^^J%
}}
\makeatother



\newcommand{\lun}[1]{{\fontfamily{qcr}\selectfont{%
\LARGE{\textbf{\underline{#1}}}}}}

\newcommand{\inditem}{\itemindent10pt\item}


\usepackage{scrextend}
\newenvironment{mldescription}{%
  \begin{addmargin}[2em]{1em}
    \setlength{\parindent}{-1em}%
    \newcommand*{\mlitem}[1][]{\par\medskip\textbf{##1}\quad}\indent
}{%
  \end{addmargin}
  \medskip
}

\begin{document}
	
{\linespread{1.1}\selectfont

\vspace*{-7em}

\begin{center}
%{\relscale{1.2}{\fontfamily{qcr}\fontseries{b}\selectfont 
%{\colorbox{black}{\color{blue}{\llWC{} Database Engine \\and 
%\llMOSAIC{} Native Application Toolkit}}}}}

\colorlet{ctmp}{logoPeach!20!gray}
\colorlet{ctmpp}{ctmp!90!yellow}
\colorlet{ctmppp}{ctmpp!50!black}
\colorlet{ctmpppp}{ctmppp!90!logoRed}

\vspace{1em}

%{\colorbox{darkBlGreen!30!darkRed}{%
\begin{tcolorbox}
[
%%enhanced,
%%frame hidden,
%interior hidden
arc=2pt,outer arc=0pt,
enhanced jigsaw,
width=.7\textwidth,
colback=ctmpppp!60,
colframe=logoRed!30!darkRed,
drop shadow=logoPurple!50!darkRed,
%boxsep=0pt,
%left=0pt,
%right=0pt,
%top=2pt,
]
\begin{minipage}{\textwidth}	
\begin{center}		
{\setlength{\fboxsep}{18pt}
	\relscale{1.4}{{\fontfamily{qcr}\fontseries{b}\selectfont%
{Scientific Application Plugins for 
Education and Publishing (\sapp{})}}}}
\end{center}
\end{minipage}
\end{tcolorbox}
\end{center}

\vspace{-1em}

%\noindent\lun{Overview}
\fontfamily{ptm}\fontsize{13pt}{18pt}\selectfont
{\sectsp}
\p{This paper will summarize the \sapp{} plugin 
framework and discuss its 
applications for testing, education, and the 
development of test and test-preparation materials.  
The goal of \sapp{} is to augment scientific and 
technical applications with features that enhance 
their usefulness as teaching materials.  
\lsapp{} plugins can integrate desktop applications 
with course curricula, educational materials, 
and cloud services hosting student, instructor, 
and course information.   In general, 
\sapp{} refers not to a single plugin but rather a 
framework for creating plugins tailored 
to individual publishers, academic institutions, 
or educational software packages.  A given 
scientific or technical application may 
host multiple \sapp{} plugins, each tracking 
information structured according to the requirements 
of the plugin provider.}

\p{For sake of discussion, this paper will 
describe \sapp{} in terms of hypothetical 
ETS plugins developed expressly for Educational 
Testing Service.  To make the discussion more concrete, 
the paper will consider hypothetical ETS plugins for 
IQmol (a molecular visualization application), 
ParaView (data analysis and visualization software 
focused on statistical/quantitative data sets), 
MeshLab (a \ThreeD{} graphics engine), 
Octave (an open-source Matlab emulator),  
\Qt{} Creator (a \Cpp{} Integrated Development 
Environment), and XPDF (a \PDF{} viewer).}

\vspace{2em}
\noindent\lun{The Technological Role of \lssapp{} Plugins}
{\sectsp}

\p{A suite of inter-related \sapp{} plugins 
--- for example, an ETS plugin suite --- would contribute 
two kinds of functionality to their host 
applications: (1) tracking and presenting 
information specific to individual students, courses, 
and instructors; and (2) allowing multiple applications 
which each have ETS plugins to interoperate.  
To explain the features of the plugin suite, consider 
the following scenarios: 

\begin{mldescription}
\mlitem[%\parbox{14cm}{
Scenario 1: A student reads textbooks, articles, 
or test-preparation materials which may be enhanced 
with multi-media content%}
]  To augment the reading 
experience, educational texts may be supplemented 
with files describing visual, interactive materials 
which require specialized software.  For example, 
texts about chemistry (e.g., study materials for 
the chemistry \GRE{} exams) may include \ThreeD{} 
models of chemical compouns which can be viewed 
with IQmol; biology texts may be illustrated 
with \ThreeD{} tissue models which can be viewed 
in MeshLab; physics texts may describe 
equations or empirical data which may be visualized 
via ParaView, or Matlab simulations that could be 
executed through Octave.  Publications could then 
embed these supplemental 
materials directly, or else include links from which 
the multi-media files may be downloaded.  If the 
documents are viewed with an e-reader which itself 
hosts an ETS plugin --- take XPDF as a case-study --- 
then the viewer would identify the locations in the 
text where the multi-media files are relevant and, 
when the student is reading that part of the text, 
notify him or her of the option to automatically 
launch the proper application with which to access the content.

\pseudoIndent{} A hypothetical XPDF plugin, for example, 
would identify the correct application to use to view 
multi-media content based on the file type.  This feature 
could also be refined via a cloud service --- information 
provided by instructors could include notation of the 
kinds of software used for any particular course.  
This cloud-hosted data might, for instance, indicate 
that a specific course is using IQmol as a pedagogical 
tool, and indicate that cheminformatic files linked 
to teaching materials for the course should always be 
viewed with IQmol.  Once the XPDF plugin identifies 
the proper multi-media software to use, it can 
lanch the application on the student's computer and 
--- assuming the target application also has an ETS 
plugin --- send that application signals identifying 
which files to load into the application session.   
The XPDF plugin may also send 
information about the current student and 
class/curriculum, which the target application may use to 
personalize the User Interface according to students' 
or instructors' preferances (see the next scenario 
for more about personalization).

\pseudoIndent{} For multi-media content which is 
not specific to specialized technical softwar 
--- that is, generic multi-media formats such as 
audio, video, or panoramic-photography --- 
\sapp{} plugins for PDF or ePub viewers can present 
this content directly, in separate windows detached 
from the principle document viewer, rather than 
routing the files to external software.   

\mlitem[Scenario 2: A student launches a scientific 
application which is used as a pedagogical tool]  
Teachers often instruct students to download 
and install software relevant to course curriculum, 
and this software can potentially be an essential 
part of the course content.  Instructors may 
(1) use the visualization capabilities of these 
domain-specific applications to help students 
understand the concepts covered in class; 
(2) provide instruction in how to use the 
software as part of the curriculum; 
(3) evaluate students' understanding of the 
software as part of their assessment of 
students' mastery of the curriculum; or 
(4) use applications' analytic features as an 
overview of analytic or quantitative methodologies 
relevant to the course's subject matter.  In 
the case of IQmol, features such as 
energy minimization, plotting orbitals, 
calculating vibrational frequencies, and many 
other chemphysical computations provide an 
overview of scientific concepts which 
might be covered in a Chemistry class. 
 
\pseudoIndent{} To facilitate the 
use of scientific applications as teaching tools, 
\sapp{} plugins help instructors personalize the 
applications which their students use in 
conjunction with course curricula.  This personalization 
can have several dimenions, including: 
(1) manipulating the User Interface to prioritize 
concepts pertinent to each course; (2) 
enabling the application to present 
course-specific instructions to the student, 
such as instructions and assignments; 
(3) tracking a suite of resources curated 
for the specific class; and potentially (4) 
allowing students to send questions to the 
instructor with screenshots and application-state 
information.  Again using IQmol as a case-study,  
an ETS plugin could show students a list of molecular 
examples discussed in class (based on data 
provided by the instructor) and allow students 
to view the corresponding \ThreeD{} molecular 
graphics accordingly; instructors could also 
rearrange the IQmol menus and toolbars, to foreground 
those analytic tasks which are relevant to the 
course curriculum.  

\mlitem[Scenario 3: A student launches an Integrated Development Environment 
(\IDE{}) which is used as a teaching tool]  
Using \IDE{}s is a prerequite for most courses 
in computer science and computer programming, and 
in this context \sapp{} plugins may be used 
as with other specialized software.  An ETS 
plugin for \Qt{} Creator, for example, could 
load source and project files curated for 
individual classes, and display instructions 
or assignments for students based on instructor 
input.  The use-cases for \IDE{} plugins, however, 
extend beyond computer science proper, and include 
any scenario where students would write computer 
code as a learning aid or part of an assignment.  
Physics students might write algorithms 
to approximate answers to equations which lack  
closed-form solutions; biology students might 
write code to examine genetic patterns; 
chemistry students might develop simulations 
of materials' behavior in different force 
fields.  In these situations \sapp{} 
plugins would allow students to get information 
from instructors, within the \IDE{} itself, 
about the goals and requirements for a 
code-writing exercise. 

\pseudoIndent{} A further use-case for \IDE{} 
plugins is for building other \sapp{}-enabled 
applications.  For example, many scientific 
applications can be built from source on 
students' computers, with the aid of 
\IDE{}s such as \Qt{} Creator.  In these 
cases instructors can provide students 
with application code, perhaps modified 
according to the course curriculum 
(including with their own \sapp{} plugins).   
For example, \Qt{}-based applications such 
as IQmol, MeshLab, and XPDF can be built 
directly from \Qt{} Creator.  An 
ETS plugin for \Qt{} Creator could then be 
the first tool which students use at the 
start of a course, with that plugin obtaining 
information from a Cloud service about which 
applications are needed for the course.  
Behind-the-scenes tasks such as defining
project files and setting up build environments 
can then be performed automatically via the 
plugin, helping ensure that the student's 
system has the necessary prerequisites to 
run all the course-related software. 
\end{mldescription}
}


\p{

}




\p{\lHTXN{} is a new format and protocol for 
representing publications.  The central goal 
of \HTXN{} is to support a new generation of 
publishing technologies, where conventional 
document formats are increasingly being supplanted 
by digital, multi-media reader experiences.  
In the contemporary publishing paradigm, 
individual publications are often linked 
with other forms of digital content: multi-media 
resources, research data sets, machine-readable 
representations of document text, and 
domain-specific sofware applications 
(used to study or visualize the case-studies or 
research findings discussed in publications).  
The conventional manuscript (the \q{primary} 
resource which is cited and downloaded) 
is then networked with a package of 
supplemental (or \q{secondary}) resources.  
The \HTXN{} protocol is designed to 
rigorously document these multi-media networks, enabling 
e-readers and domain-specific applications to be 
integrated so that readers may easily access and 
experience multi-media content.}

\vspace{2em}
\noindent\lun{\lsHTXN{} for Multi-Media}
{\sectsp}
\p{The generic term \q{multi-media content} actually 
encompasses multiple phenomena:

\vspace{-1em}
\begin{description}[leftmargin=2pt,
	labelindent=-2pt,labelsep=12pt]
\item[Multimedia Files]  Individual 
files representing audio, video, or 3D graphics content.  
These files may be linked from specific locations in 
the primary manuscript, or even embedded within manuscripts 
when they are published in \PDF{} format.
 
\item[Data Sets and Data Visualization]  Publishers 
increasingly emphasize sharing research data alongside 
texts, so readers can verify or even attempt to 
replicate claimed results.  Data sets are also a form 
of multimedia content because, apart from being aggregates 
of raw data, data sets are almost always accompanied 
by interactive, visual content: charts, diagrams, or 
plots to visualize the information holistically, or 
interactive tools to examine or navigate through the 
data set at finer scales.  

\item[Application Networks]  Another genre of multi-media 
content involves resources which may only be experienced 
through specialized software.  This classification encompasses 
content from particular scientific or technical domains, 
which is encoded in domain-specific formats: representations 
of molecular structures, archaeological sites, image-processing 
data, wave-forms for signal processing, sentence-parses for 
linguistic analysis, and so forth.  To conveniently access 
this kind of multi-media, readers need to use software 
which can send signals to the specialized applications 
having the capability to recognize the domain-specific 
formats and translate them to interactive, 
visual presentations.  In short, publication viewers 
(e.g., e-readers) need to participate in multi-application 
networks, where data can be sent and received between 
each component.  Publishers can provide this 
functionality to readers by implementing special 
e-readers and, in addition, writing plugins 
(or collaborating with external application developers) 
to ensure that applications networked with e-readers 
are properly aligned with the e-readers themselves.

\item[Publications-as-Applications]  In some 
cases, publications themselves are a form of 
multi-layered multi-media content.  This applies 
to publications which are not simply read from 
start to end, but instead naturally lend 
themselves to a reading process which navigates 
back and forth between different sections of 
the text, or juxtaposes different sections to be 
visible at the same time.  A canonical example 
of such layered reading is testing materials 
and test preparation, where exam questions, 
instructions, supplemental materials (such 
as passages for reading-comprehension 
assessment), and comments or analyses about 
answers (in the case of prep materials), 
each form different layers which 
students may wish to view side-by-side.  
In these cases, e-readers cannot simply 
treat the publication as one single ePub 
or PDF file.  Instead, the manuscript needs 
to identify text segments which can be factored 
into different layers, and the e-reader 
needs to implement text-viewers which allow 
each layer to be viewed in separate windows, 
with readers able to juxtapose and position the 
windows as desired. 
\end{description}\vspace{-2em}
}


\p{\lHTXN{} represents publication manuscripts 
using structures which rigorously document 
publications' multi-media content and multi-application 
networking requirements.  This detailed 
multi-media support has several dimensions: 

\begin{enumerate}[leftmargin=2pt]
\inditem{} Defining points in the manuscript where 
multi-media files are linked or embedded: this 
involves annotating locations in the manuscript 
with hyper-references to multi-media files 
(audio, video, etc.) which readers should be 
able to access when they reach the corresponding 
point in the text.

\inditem{} Establishing granular cross-references between 
publications and multi-media content: this is a 
more complex case where manuscript locations have to 
link \textit{to} or \textit{from} limited 
\textit{portions} of the corresponding multi-media 
resources.  For example, a passage in the manuscript 
may discuss a single sample within a data set; or 
may explicate a particular facet of the data set, 
such as an individual column in a tabular 
information space, or a specific set of 
statistical parameters against which quantitative 
operations are performed.  These scenarios 
call for bi-directional cross-references between the 
data set and the publication, wherein the granular 
data-set facet topically relevant to the 
corresponding manuscript location (the sample, 
table-column, parameters, etc.) is formally 
isolated and declared as a reference-target.  

\inditem{} Cross-referances may also be 
defined between publications' non-textual or 
non-paragraph content and corresponding 
multi-media resources.  For example, tables 
or diagrams visually presented in a manuscript 
may be liked to statistical data from which 
the figures are derived.  A similar situation 
applies when visuals inluded in a publication 
are linked to multi-media resources which 
represent the same information in a different 
experiential register: a PDF document may 
include a two-dimensional graphic which is created 
by taking a camera shot of a \ThreeD{} model, 
which readers may also experience with a 
\ThreeD{} graphics engine; or a publication may 
reproduce a graph or scatter-plot derived 
from a data set, where data visualization 
software can represent the same information 
in a more interactive medium, with parameters 
plotted as curves or surfaces in a \ThreeD{} 
ambient space, or where systems are visualized 
as systems evolving over time.  
\end{enumerate}
}

\vspace{2em}
\noindent\lun{\lsHTXN{} for Teaching Materials}
{\sectsp}
\p{
\begin{description}[leftmargin=2pt,
	labelindent=-2pt,labelsep=12pt]
\item[ETS Plugin Framework]
The proposed ETS Plugin Framework would create 
a common code base that can be used to 
implement ETS-specific plugins to 
applications spanning a range of academic 
disciplines and subject areas.  The primary 
goal of these plugins would be to connect 
e-reader software --- applications 
for viewing test-preparation and course 
materials --- with domain-specific software 
which instructors may use as teaching aids.  
With respect to GRE exams in fields 
such as biology, chemistry, or physics, 
domain-specific applications might include 
bioinformatics, molecular visualization, 
or physical simulation tools, respectively.  
ETS plugins would help scientific applications 
become more valuable as classroom tools.  In 
particular, with ETS plugins these applications 
can (1) receive signals from document viewers 
(such as PDF viewers) to automatically display 
multi-media content and (2) display course 
and instructional materials.  As a 
concrete example, consider molecularal 
visualization software such as IQmol.  Via 
an ETS plugin, IQmol could (1) render a \ThreeD{} 
image given data, in a chemical file format, 
received from an e-reader with a similar plugin; 
and (2) display instructions, questions, 
assignments, or course-related content 
(such as graphics for chemical compounds discussed 
in class) provided by instructors. 

\item[Cloud Support for ETS Plugins]  
Supplementing the functionality described in the 
preceding paragraph, Cloud Services can be 
used to connect individual applications 
to content and curricula specific to individual 
courses.  Because most scientific applications 
are implemented as desktop software, the hosting 
of the cloud back-ends to support these features 
would be a natural fit for LTS's \q{Native Cloud/Native} 
(NCN) protocol, which is specifically designed 
to integrate desktop front-ends with Cloud/Native 
services.  According to this architecture, 
NCN services would host information specific to 
individual courses, students, and instructors.  
When a student launches an application with an 
ETS plugin, the plugin would retrieve data 
pertaining to that student and to his or her classes.  
As appropriate, the plugin could then instruct the 
host application to load specific content, and/or 
present questions or instructions supplied 
by the instructor.  For example, IQmol could 
load a list of molecular files corresponding to 
chemicals studied in the students' class.  
The ETS Plugin Cloud back-end would also be 
used by document (e.g., PDF) viewers to obtain 
information needed to properly route signals 
to other applications using ETS Plugins.

\item[Student Dashboards]  
One feature which should be common to all ETS 
Plugins is a \q{dashboard}, or a separate window 
aggregating the student's information.  The 
dashboard's design would be roughly as follows: 
student information would be divided into 
four or perhaps five tabs, with labels such 
as \q{My Courses}, \q{My Library}, 
perhaps \q{My Tests}, \q{My Applications}, and 
\q{My Account}.  Under the Courses tab, 
students select one course to focus on, 
which would cause the Library and Tests 
tabs to prioritize tests, test preparation, and 
readings assigned for that course.  Under 
the Applications tab, students could see 
a list of all applications on their computer 
which have the ETS Plugin installed, or 
which are used as pedagogical tools for their 
courses, or which they are instructed to install 
(the plugin could identify required software 
which students have not yet installed based 
on information provided by the ETS Cloud Service).

     

\end{description}
}

\end{document}


