\documentclass[10pt,letterpaper]{article}

\usepackage{eso-pic}

\AddToShipoutPictureBG{%

\ifnum\value{page}>0{
\AtTextUpperLeft{
\makebox[18.5cm][r]{
\raisebox{-2.3cm}{%
{\transparent{0.3}{\includegraphics[width=0.29\textwidth]{e-logo.png}}	}} } }
}\fi
}

\AddToShipoutPicture{%
{
 {\color{blGreen!70!red}\transparent{0.9}{\put(0,0){\rule{.55cm}{\paperheight}}}}%
 {\color{darkRed!70!purple}\transparent{1}\put(6,0){{\rule{.3cm}{\paperheight}}}}
% {\color{logoPeach!80!cyan}\transparent{0.5}{\put(0,700){\rule{1cm}{.6cm}}}}%
% {\color{darkRed!60!cyan}\transparent{0.7}\put(0,706){{\rule{1cm}{.6cm}}}}
% \put(18,726){\thepage}
% \transparent{0.8}
}
}



\AddToShipoutPicture{%

\ifnum\value{page}>0


{\color{blGreen!70!red}\transparent{0.9}{\put(300,8){\rule{0.5\paperwidth}{.3cm}}}}%
{\color{inOne}\transparent{0.8}{\put(300,10){\rule{0.5\paperwidth}{.3cm}}}}%
{\color{inTwo}\transparent{0.3}\put(300,13){{\rule{0.5\paperwidth}{.3cm}}}}

\put(301,16){%
\transparent{0.7}{
\includegraphics[width=0.2\textwidth]{logo.png}} }

{\color{blGreen!70!red}\transparent{0.9}{\put(5.6,5){\rule{0.5\paperwidth}{.4cm}}}}%
{\color{inOne}\transparent{1}{\put(5.6,10){\rule{0.5\paperwidth}{.4cm}}}}%
{\color{inTwo}\transparent{0.3}\put(5.6,15){{\rule{0.5\paperwidth}{.4cm}}}}

\fi
}

%\pagestyle{empty} % no page number
%\parskip 7.2pt    % space between paragraphs
%\parindent 12pt   % indent for new paragraph
%\textwidth 4.5in  % width of text
%\columnsep 0.8in  % separation between columns

\setlength{\footskip}{23pt}

\usepackage[paperheight=14.5in,paperwidth=9.1in]{geometry}
\geometry{left=.9in,top=1.1in,right=.6in,bottom=1.4in} %margins

\newcommand{\sectsp}{\vspace{12pt}}

\usepackage{graphicx}
\usepackage{color,framed}

\usepackage{float}

\usepackage{mdframed}


\usepackage{setspace}
\newcommand{\rpdfNotice}[1]{\begin{onehalfspacing}{

\Large #1

}\end{onehalfspacing}}

\usepackage{xcolor}

\usepackage[hyphenbreaks]{breakurl}
\usepackage[hyphens]{url}

\usepackage{hyperref}
\newcommand{\rpdfLink}[1]{\href{#1}{\small{#1}}}
\newcommand{\dblHref}[1]{\href{#1}{\small{\burl{#1}}}}
\newcommand{\browseHref}[2]{\href{#1}{\Large #2}}

\hypersetup{
    colorlinks=true,
    linkcolor=cyan,
    filecolor=magenta,
    urlcolor=blue,
}

\urlstyle{same}

\definecolor{blGreen}{rgb}{.2,.7,.3}
\definecolor{darkRed}{rgb}{.2,.0,.1}

\definecolor{darkBlGreen}{rgb}{.1,.3,.2}

\definecolor{oldBlColor}{rgb}{.2,.7,.3}

\definecolor{blColor}{rgb}{.1,.3,.2}

\definecolor{elColor}{rgb}{.2,.1,0}
\definecolor{flColor}{rgb}{0.7,0.3,0.3}

\definecolor{logoOrange}{RGB}{108, 18, 30}
\definecolor{logoGreen}{RGB}{85, 153, 89}
\definecolor{logoPurple}{RGB}{200, 208, 30}

\definecolor{logoBlue}{RGB}{4, 2, 25}
\definecolor{logoPeach}{RGB}{255, 159, 102}
\definecolor{logoCyan}{RGB}{66, 206, 244}
\definecolor{logoRed}{rgb}{.3,0,0}

\definecolor{inOne}{rgb}{0.122, 0.435, 0.698}% Rule colour
\definecolor{inTwo}{rgb}{0.122, 0.698, 0.435}% Rule colour

\definecolor{outOne}{rgb}{0.435, 0.698, 0.122}% Rule colour
\definecolor{outTwo}{rgb}{0.698, 0.435, 0.122}% Rule colour

\usepackage[many]{tcolorbox}% http://ctan.org/pkg/tcolorbox

\usepackage{transparent}

\newenvironment{cframed}{\begin{mdframed}[linecolor=logoPeach,linewidth=0.4mm]}{\end{mdframed}}

\newenvironment{ccframed}{\begin{mdframed}[backgroundcolor=logoGreen!5,linecolor=logoCyan!50!black,linewidth=0.4mm]}{\end{mdframed}}

\usepackage{aurical}
\usepackage[T1]{fontenc}

\usepackage{relsize}

\newcommand{\pseudoIndent}{

\vspace{1pt}\hspace{8pt}}

\newcommand{\YPDFI}{{\fontfamily{fvs}\selectfont YPDF-Interactive}}

%
\newcommand{\deconum}[1]{{\protect\raisebox{-1pt}{{\LARGE #1}}}}



\newcommand{\VersatileUX}{{\color{red!85!black}{\Fontauri Versatile}}%
{{\fontfamily{qhv}\selectfont\smaller UX}}}

\newcommand{\NDPCloud}{{\color{red!15!black}%
{\fontfamily{qhv}\selectfont {\smaller NDP C{\smaller LOUD}}}}}

\newcommand{\lfNDPCloud}{{\color{red!15!black}%
{\fontfamily{qhv}\selectfont N{\smaller DP C{\smaller LOUD}}}}}

\newcommand{\textds}[1]{{\fontfamily{lmdh}\selectfont{%
\raisebox{-1pt}{#1}}}}

\newcommand{\dsC}{{\textds{ds}{\fontfamily{qhv}\selectfont \raisebox{-1pt}
{\color{red!15!black}{C}}}}}

\newcommand{\HTXN}{\resizebox{!}{8pt}{\AcronymText{HTXN}}}
\newcommand{\lHTXN}{\resizebox{!}{8.5pt}{\AcronymText{HTXN}}}

\newcommand{\lMOSAIC}{\resizebox{!}{8.5pt}{\AcronymText{MOSAIC}}}

\newcommand{\XML}{\resizebox{!}{8pt}{\AcronymText{XML}}}
\newcommand{\RDF}{\resizebox{!}{8pt}{\AcronymText{RDF}}}

\newcommand{\CLang}{\resizebox{!}{8pt}{\AcronymText{C}}}

\newcommand{\HNaN}{\resizebox{!}{8pt}{\AcronymText{HN%
\textsc{a}N}}}


\newcommand{\MeshLab}{\resizebox{!}{8pt}{\AcronymText{MeshLab}}}
\newcommand{\IQmol}{\resizebox{!}{8pt}{\AcronymText{IQmol}}}

\newcommand{\GUI}{\resizebox{!}{8pt}{\AcronymText{GUI}}}

\newcommand{\API}{\resizebox{!}{8pt}{\AcronymText{API}}}

\newcommand{\IDE}{\resizebox{!}{8pt}{\AcronymText{IDE}}}

\newcommand{\ThreeD}{\resizebox{!}{8pt}{\AcronymText{3D}}}

\newcommand{\FAIR}{\resizebox{!}{8pt}{\AcronymText{FAIR}}}

\newcommand{\QNetworkManager}{\resizebox{!}{8pt}{\AcronymText{QNetworkManager}}}
\newcommand{\QTextDocument}{\resizebox{!}{8pt}{\AcronymText{QTextDocument}}}
\newcommand{\QWebEngineView}{\resizebox{!}{8pt}{\AcronymText{QWebEngineView}}}
\newcommand{\HTTP}{\resizebox{!}{8pt}{\AcronymText{HTTP}}}


\newcommand{\lAcronymTextNC}[2]{{\fontfamily{fvs}\selectfont {\Large{#1}}{\large{#2}}}}

\newcommand{\AcronymTextNC}[1]{{\fontfamily{fvs}\selectfont {\large #1}}}


\colorlet{orr}{orange!60!red}

\newcommand{\textscc}[1]{{\color{orr!35!black}{{%
						\fontfamily{Cabin-TLF}\fontseries{b}\selectfont{\textsc{\scriptsize{#1}}}}}}}


\newcommand{\textsccserif}[1]{{\color{orr!35!black}{{%
				\scriptsize{\textbf{#1}}}}}}


\newcommand{\AcronymText}[1]{{\textscc{#1}}}

\newcommand{\AcronymTextser}[1]{{\textsccserif{#1}}}


\newcommand{\mAcronymText}[1]{{\textscc{\normalsize{#1}}}}

\newcommand{\TeXMECS}{\resizebox{!}{8pt}{\AcronymText{TeXMECS}}}

\newcommand{\NGML}{\resizebox{!}{8pt}{\AcronymText{NGML}}}

\newcommand{\Cpp}{\resizebox{!}{8.5pt}{\AcronymText{C++}}}

\newcommand{\WhiteDB}{\resizebox{!}{8pt}{\AcronymText{WhiteDB}}}

\colorlet{drp}{darkRed!70!purple}

%\newcommand{\MOSAIC}{{\color{drp}{\AcronymTextNC{\scriptsize{MOSAIC}}}}}

\newcommand{\MOSAIC}{\resizebox{!}{8pt}{\AcronymText{MOSAIC}}}


\newcommand{\mMOSAIC}{{\color{drp}{\AcronymTextNC{\normalsize{MOSAIC}}}}}

\newcommand{\MOSAICVM}{\mMOSAIC-\mAcronymText{VM}}

\newcommand{\sMOSAICVM}{\resizebox{!}{8pt}{\MOSAICVM}}
\newcommand{\sMOSAIC}{\resizebox{!}{8pt}{\MOSAIC}}


%\newcommand{\lMOSAIC}{{\color{drp}{\lAcronymTextNC{M}{OSAIC}}}}
\newcommand{\lfMOSAIC}{\resizebox{!}{9pt}{{\color{drp}{\lAcronymTextNC{M}{OSAIC}}}}}

\newcommand{\Mosaic}{\resizebox{!}{8pt}{\MOSAIC}}
\newcommand{\MosaicPortal}{{\color{drp}{\AcronymTextNC{MOSAIC Portal}}}}

\newcommand{\RnD}{\resizebox{!}{7.5pt}{\AcronymText{R\&D}}}
\newcommand{\QtCpp}{\resizebox{!}{8.5pt}{\AcronymText{Qt/C++}}}
\newcommand{\Qt}{\resizebox{!}{9pt}{\AcronymText{Qt}}}
\newcommand{\QtSQL}{\resizebox{!}{8pt}{\AcronymText{QtSQL}}}

\newcommand{\HTML}{\resizebox{!}{8pt}{\AcronymText{HTML}}}
\newcommand{\PDF}{\resizebox{!}{8pt}{\AcronymText{PDF}}}

\newcommand{\p}{

\vspace{1.2em}}

\newcommand{\q}[1]{{\fontfamily{qcr}\selectfont ``}#1{\fontfamily{qcr}\selectfont ''}} 

%\newcommand{\deconum}[1]{{\textcircled{#1}}}


\renewcommand{\thesection}{\protect\mbox{\deconum{\Roman{section}}}}
\renewcommand{\thesubsection}{\arabic{section}.\arabic{subsection}}

\newcommand{\llMOSAIC}{\mbox{{\LARGE MOSAIC}}}
%\newcommand{\lfMOSAIC}{\mbox{M\small{OSAIC}}}

\newcommand{\llMosaic}{\llMOSAIC}
\newcommand{\lMosaic}{\lMOSAIC}
\newcommand{\lfMosaic}{\lfMOSAIC}


\newcommand{\llWC}{\mbox{{\LARGE WhiteCharmDB}}}

\newcommand{\llwh}{\mbox{{\LARGE White}}}
\newcommand{\llch}{\mbox{{\LARGE CharmDB}}}

\usepackage{enumitem}

\setlist[description]{%
  topsep=30pt,               % space before start / after end of list
  itemsep=5pt,               % space between items
  font={\bfseries\sffamily}, % set the label font
%  font={\bfseries\sffamily\color{red}}, % if colour is needed
}

\setlist[enumerate]{%
  topsep=3pt,               % space before start / after end of list
  itemsep=-2pt,               % space between items
  font={\bfseries\sffamily}, % set the label font
%  font={\bfseries\sffamily\color{red}}, % if colour is needed
}

%\usepackage{tcolorbox}

\newcommand{\slead}[1]{%
\noindent{\raisebox{2pt}{\relscale{1.15}{{{%
\fcolorbox{logoCyan!50!black}{logoGreen!5}{#1}
}}}}}\hspace{.5em}}


\let\OldLaTeX\LaTeX

\renewcommand{\LaTeX}{\resizebox{!}{8pt}{\color{orr!35!black}{\OldLaTeX}}}

\newcommand{\LargeLaTeX}{\resizebox{!}{8.5pt}{\color{orr!35!black}{\OldLaTeX}}}


\setlength\parindent{24pt}
%%\usepakage{newfile}

\usepackage{hyperref}

\usepackage{etoolbox}

\usepackage{zref-user}

\newwrite\sdiFile
\immediate\openout\sdiFile=\jobname.sdi.txt

\newcommand{\p}[1]{

\vspace{10pt}#1}

\newif\iftabng
\tabngfalse


\usepackage{letltxmacro}
\LetLtxMacro{\oldmmsemi}{\;}
\LetLtxMacro{\oldtbplus}{\+}
\LetLtxMacro{\oldtbgt}{\>}
\LetLtxMacro{\oldmmgt}{\+}

\newcommand{\+}{\iftabng\oldtbplus\else\sss\fi}

\renewcommand{\>}{\iftabng\oldtbplus\else
\ifmmode\oldmmgt\else\sse\sss\fi\fi}

%\renewcommand{\>}{\sse\sss}

\renewcommand{\;}{\relax\ifmmode\oldmmsemi\else\sse\fi}

\newcommand{\writeSDI}[1]{\immediate\write\sdiFile#1}

\newcommand{\emblink}[2]{\href{\#sdi:#1--#2}{\#sdi:#1--#2}}

%\newcount\sdiCounter
%\def\advsdiCounter{\global\advance\sdiCounter by1}

%\newcount\sdiCounterP
%\def\advsdiCounterP{\global\advance\sdiCounterP by1}

%\newcounter{sdiCounter}
\newcounter{sdiCounterP}[page]
\newcounter{sdiCounter}

\def\topt#1{\expandafter\the\dimexpr\dimexpr#1sp\relax\relax}

\makeatletter
%\catcode`\*=10
\newcommand{\sss}{%
\stepcounter{sdiCounterP}
\stepcounter{sdiCounter}
\pdfsavepos\write\sdiFile{!/ SDI_Sentence_Start} 
\write\sdiFile\expandafter{\expandafter$%
\expandafter i\expandafter:%
\expandafter\space\the\c@sdiCounter}
\write\sdiFile\expandafter{\expandafter$%
\expandafter o\expandafter:%
\expandafter\space\the\c@sdiCounterP}
\write\sdiFile\expandafter{\expandafter$%
\expandafter p\expandafter:%
\expandafter\space\thepage^^J%
$x: \topt\pdflastxpos^^J%
$y: \topt\pdflastypos^^J%
/!^^J%
<<>^^J%
}}
%\catcode`\%=14
\makeatother

\makeatletter
\newcommand{\sse}{%
\pdfsavepos\write\sdiFile{!/ SDI_Sentence_End} 
\write\sdiFile\expandafter{\expandafter$%
\expandafter i\expandafter:%
\expandafter\space\the\c@sdiCounter}
\write\sdiFile\expandafter{\expandafter$%
\expandafter o\expandafter:%
\expandafter\space\the\c@sdiCounterP}
\write\sdiFile\expandafter{\expandafter$%
\expandafter p\expandafter:%
\expandafter\space\thepage^^J%
$x: \topt\pdflastxpos^^J%
$y: \topt\pdflastypos^^J%
/!^^J%
<<>^^J%
}}
\makeatother



\newcommand{\lun}[1]{{\fontfamily{qcr}\selectfont{%
\LARGE{\textbf{\underline{#1}}}}}}


\begin{document}
	
{\linespread{1.1}\selectfont

\vspace*{-7em}

\begin{center}
%{\relscale{1.2}{\fontfamily{qcr}\fontseries{b}\selectfont 
%{\colorbox{black}{\color{blue}{\llWC{} Database Engine \\and 
%\llMOSAIC{} Native Application Toolkit}}}}}

\colorlet{ctmp}{logoPeach!20!gray}
\colorlet{ctmpp}{ctmp!90!yellow}
\colorlet{ctmppp}{ctmpp!50!black}
\colorlet{ctmpppp}{ctmppp!90!logoRed}

\vspace{1em}

%{\colorbox{darkBlGreen!30!darkRed}{%
\begin{tcolorbox}
[
%%enhanced,
%%frame hidden,
%interior hidden
arc=2pt,outer arc=0pt,
enhanced jigsaw,
width=.7\textwidth,
colback=ctmpppp!60,
colframe=logoRed!30!darkRed,
drop shadow=logoPurple!50!darkRed,
%boxsep=0pt,
%left=0pt,
%right=0pt,
%top=2pt,
]
\begin{minipage}{\textwidth}	
\begin{center}		
{\setlength{\fboxsep}{18pt}
	\relscale{1.4}{{\fontfamily{qcr}\fontseries{b}\selectfont%
{Hypergraph Text Encoding Protocol (\HTXN{})}}}}
\end{center}
\end{minipage}
\end{tcolorbox}
\end{center}

\vspace{-1em}

%\noindent\lun{Overview}
\fontfamily{ptm}\fontsize{13pt}{18pt}\selectfont
{\sectsp}
\p{The Hypergraph Text 
Encoding Protocol (\HTXN{}) is a new publication 
format which can generate documents in 
a variety of different representations 
(such as {\LaTeX}, {\XML}, {\HTML}, and 
{\RDF}).  \lHTXN{} enables publication workflows 
to combine two or more 
existing formats (e.g., {\LaTeX} and {\XML}).  
A document with \HTXN{} encoding will 
generate {\LaTeX} or {\XML} \q{views} 
which can each be constructed for use at different stages 
in the document-preparation pipeline.  \lHTXN{} is 
also designed to work with publications that are 
embedded in a larger data or multi-media ecosystem 
--- in particular, documents that are paired with 
data sets or multi-media content that need 
specialized software.  

\vspace{2.25em}
\noindent\lun{Benefits}
\vspace{-1em}
\begin{description}[leftmargin=13pt,labelsep=15pt,itemsep=12pt]
\item[Mix-and-Match Formats] \lHTXN{} employs 
\q{stand-off} annotation, which allows multiple markup 
protocols to be defined simultaneously on the same 
document.  In addition, \lHTXN{} uses a flexible 
character-encoding protocol that ensures 
compatibility with diverse text representations
(such as {\XML}, {\LaTeX}, Unicode, and {\Qt}/QString).  
Via secondary documents or \q{views} (generated from the primary 
\HTXN{} document), \HTXN{} can be used wherever 
{\XML} or {\LaTeX} (or other formats) are desired.   
   
\item[Fine-Grained Character Encoding] The \HTXN{} 
protocol includes more detailed document structure 
information compared to conventional 
markup.  For example, \HTXN{} identifies 
sentence boundaries and punctuation features, 
disambiguating logically distinct 
but often visually identical 
characters (such as \makebox{dots/periods or dashes/hyphens, 
which may have different structural meanings in different contexts}).   

\item[Multi-Media and Data Set Integration] 
\lHTXN{} is optimized for sophisticated 
publishing environments which combine 
conventional natural-language texts with 
interactive data sets and software.  
Contemporary publishing increasingly sees 
academic texts as part 
of a multi-faceted ecosystem, where  
publications, research data, and interactive 
software can be packaged into multi-media 
presentations.  \lHTXN{} ensures that 
documents can be properly experienced within these 
augmented publishing platforms.
\end{description}
}

\vspace{-.5em}
\noindent\lun{Complex Anchors and Interoperability}
\p{\lHTXN{} introduces a variation on micro-citations 
and conventional {\HTML} or {\LaTeX} anchors/labels, 
which we call \q{complex anchors}, to facilitate 
interoperability with research data sets 
and external software.  Complex anchors permit concepts, 
terms, and data structures to be tracked and cross-referenced 
between publications and their associated data sets or 
supplemental software.  Consider a table-structured 
data set whose columns represent scientific 
concepts or statistical parameters that are discussed in an associated publication (e.g. article or book chapter), 
or where data from a specific table is 
presented graphically as a figure within the publication.  Via 
complex anchors, these kinds of thematic relationships between 
data-set elements and corresponding points in 
the research text may be made explicit, 
so that data-set applications (including software 
expressly implemented for a newly published data set) 
can recognize references 
\q{pointing in} to the publication.  In the context of tabular data, 
data-set software could provide 
context-menu actions associated with 
each column (or, in the case of figure illustrations, the overall table), linking to an anchor in the text 
where the relevant concepts, terminology, 
or visualizations are explained.    
}

\p{This level of 
interoperation can be achieved by implementing 
document viewers (such as a \PDF{} viewer) embedded in 
scientific software or software expressly designed for 
a data set.  With embedded viewers, 
the operation of opening associated publications to 
specific anchor-points (such as figure illustrations) 
can be integrated as an interactive extension to 
their host applications, triggered by menu items 
or other {\GUI} components.  To support this 
interoperability, {\HTXN} provides anchors whose 
identifiers and embedded data can be customized 
to work with domain-specific host applications. 
}

\vspace{2em}
\noindent\lun{Complex Exolinks and Interoperability}
\p{\HTXN{} \q{exolinks} refer to data or multi-media content \textit{outside} the text (so they 
are effectively the converse of 
complex anchors, which support 
references pointing \textit{inside} the text from 
external software components).  Complex 
exolinks designate 
data or multi-media content --- such as video, audio, or \ThreeD{} (scenes or panoramic-photography) graphics --- which are outside the text,  
but potentially part of a larger publication package 
wherein an article is accompanied by data sets and/or 
multimedia content.  
Traditional external/{\HTTP} references in {\HTML} or {\PDF} 
are designed to reference resources on the 
World Wide Web, but not content downloaded in a 
package along with publications themselves.
Therefore, special reference formats are required 
for document viewers to properly handle links in the 
text that point to non-web content (which is a technical 
limitation present in conventional document encoding 
formats).  {\lHTXN} will ensure that, when accessing multi-media content, properly encoded signals will be sent between the 
document-viewer components 
and applications which support the relevant 
multimedia file types (potentially host applications 
where the document viewers themselves may be embedded).}

\p{As a concrete example, consider 
a chemistry paper which  
is paired with {\ThreeD} descriptions of a 
particular molecule.  In order for readers to view this content, 
a signal needs to be sent to compatible molecular-visualization 
software (such as {\IQmol} --- to cite a 
\Qt{}-based, open-source 
case-study).  Analogously, when reading a paper concerning {\ThreeD} radiology, signals may be sent to --- or be read on components 
embedded within --- {\ThreeD} viewers such as  
{\MeshLab} (again \Qt{}/open-source).
To support interoperability requirements as shown 
by these examples, the \HTXN{} specification includes an 
\HTXN{}-specific \q{Native Application Network} (\HNaN{}) protocol to connect document viewers with 
external and/or host applications; 
\lHTXN{} exolinks encode data which ensures that 
the concordant \HNaN{} signals are properly generated.
}

\p{\lHTXN{} has numerous features for 
interoperation with software applications where document 
viewers may be embedded, or  
which support \HNaN{} via plugins.  In particular, 
\HTXN{} parsers and generators are provided via 
a \QtCpp{} library that can be dropped in to the 
source code of native \CLang{}/\Cpp{} applications, 
so it is straightforward to add \HNaN{} to existing 
software.  Based on \Qt{} (a \Cpp{} \GUI{} and 
application-development framework), 
\HTXN{} can leverage the full range of 
\Qt{} networking or front-end features (for instance, access to publishers' \API{}s can be 
introduced as a processing stage during 
document preparation).  In addition, \HTXN{}'s graph-based 
parsing framework and hypergraph-based annotation 
model can be customized and paired with project-specific 
\GUI{} components for viewing and accessing data 
sets.  Via such customizations, \HTXN{} can be 
adapted to provide a domain-specific publication 
platform for specific research projects or 
environments (individual journals, 
book series, research labs and academic departments, etc.).  
\lHTXN{}'s \q{reference implementation} includes a 
specially-designed \q{Next-Generation} markup language 
(\NGML{}), as one way to build \HTXN{} files.  The 
\NGML{} library is transparent and distributed 
in source-code form, so it can be modified 
to create domain-specific markup styles.  A 
customized version of \NGML{} can thereby be used 
to construct \HTXN{} documents according to each individual 
lab's or journal's specifications.
}

\end{document}

