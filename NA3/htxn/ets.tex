\documentclass[10pt,letterpaper]{article}

\usepackage{eso-pic}

\setlength\parindent{0pt}

\AddToShipoutPictureBG{%

\ifnum\value{page}>1{
\AtTextUpperLeft{
\makebox[18.5cm][r]{
\raisebox{-2.3cm}{%
{\transparent{0.3}{\includegraphics[width=0.29\textwidth]{e-logo.png}}	}} } }
}\fi
}

\AddToShipoutPicture{%
{
 {\color{blGreen!70!red}\transparent{0.9}{\put(0,0){\rule{3pt}{\paperheight}}}}%
 {\color{darkRed!70!purple}\transparent{1}\put(3,0){{\rule{4pt}{\paperheight}}}}
% {\color{logoPeach!80!cyan}\transparent{0.5}{\put(0,700){\rule{1cm}{.6cm}}}}%
% {\color{darkRed!60!cyan}\transparent{0.7}\put(0,706){{\rule{1cm}{.6cm}}}}
% \put(18,726){\thepage}
% \transparent{0.8}
}
}



\AddToShipoutPicture{%

\ifnum\value{page}>0


{\color{blGreen!70!red}\transparent{0.9}{\put(300,8){\rule{0.5\paperwidth}{.3cm}}}}%
{\color{inOne}\transparent{0.8}{\put(300,10){\rule{0.5\paperwidth}{.3cm}}}}%
{\color{inTwo}\transparent{0.3}\put(300,13){{\rule{0.5\paperwidth}{.3cm}}}}

\put(301,16){%
\transparent{0.7}{
\includegraphics[width=0.2\textwidth]{logo.png}} }

{\color{blGreen!70!red}\transparent{0.9}{\put(5.6,5){\rule{0.5\paperwidth}{.4cm}}}}%
{\color{inOne}\transparent{1}{\put(5.6,10){\rule{0.5\paperwidth}{.4cm}}}}%
{\color{inTwo}\transparent{0.3}\put(5.6,15){{\rule{0.5\paperwidth}{.4cm}}}}

\fi
}

%\pagestyle{empty} % no page number
%\parskip 7.2pt    % space between paragraphs
%\parindent 12pt   % indent for new paragraph
%\textwidth 4.5in  % width of text
%\columnsep 0.8in  % separation between columns

\setlength{\footskip}{23pt}

\usepackage[paperheight=17.25in,paperwidth=9.1in]{geometry}
\geometry{left=1.3in,top=1.1in,right=.6in,bottom=1.6in} %margins

\newcommand{\sectsp}{\vspace{12pt}}

\usepackage{graphicx}
\usepackage{color,framed}

\usepackage{float}

\usepackage{mdframed}


\usepackage{setspace}
\newcommand{\rpdfNotice}[1]{\begin{onehalfspacing}{

\Large #1

}\end{onehalfspacing}}

\usepackage{xcolor}

\usepackage[hyphenbreaks]{breakurl}
\usepackage[hyphens]{url}

\usepackage{hyperref}
\newcommand{\rpdfLink}[1]{\href{#1}{\small{#1}}}
\newcommand{\dblHref}[1]{\href{#1}{\small{\burl{#1}}}}
\newcommand{\browseHref}[2]{\href{#1}{\Large #2}}

\colorlet{blCyan}{cyan!50!blue}

\definecolor{darkRed}{rgb}{.2,.0,.1}

\hypersetup{
    colorlinks=true,
    linkcolor=blCyan!80!darkRed,
    filecolor=magenta,
    urlcolor=blue,
}

\urlstyle{same}

\definecolor{blGreen}{rgb}{.2,.7,.3}

\definecolor{darkBlGreen}{rgb}{.1,.3,.2}

\definecolor{oldBlColor}{rgb}{.2,.7,.3}

\definecolor{blColor}{rgb}{.1,.3,.2}

\definecolor{elColor}{rgb}{.2,.1,0}
\definecolor{flColor}{rgb}{0.7,0.3,0.3}

\definecolor{logoOrange}{RGB}{108, 18, 30}
\definecolor{logoGreen}{RGB}{85, 153, 89}
\definecolor{logoPurple}{RGB}{200, 208, 30}

\definecolor{logoBlue}{RGB}{4, 2, 25}
\definecolor{logoPeach}{RGB}{255, 159, 102}
\definecolor{logoCyan}{RGB}{66, 206, 244}
\definecolor{logoRed}{rgb}{.3,0,0}

\definecolor{inOne}{rgb}{0.122, 0.435, 0.698}% Rule colour
\definecolor{inTwo}{rgb}{0.122, 0.698, 0.435}% Rule colour

\definecolor{outOne}{rgb}{0.435, 0.698, 0.122}% Rule colour
\definecolor{outTwo}{rgb}{0.698, 0.435, 0.122}% Rule colour

\usepackage[many]{tcolorbox}% http://ctan.org/pkg/tcolorbox

\usepackage{transparent}

\newenvironment{cframed}{\begin{mdframed}[linecolor=logoPeach,linewidth=0.4mm]}{\end{mdframed}}

\newenvironment{ccframed}{\begin{mdframed}[backgroundcolor=logoGreen!5,linecolor=logoCyan!50!black,linewidth=0.4mm]}{\end{mdframed}}

\usepackage{aurical}
\usepackage[T1]{fontenc}

\usepackage{relsize}

\newcommand{\bref}[1]{\hspace*{1pt}\textbf{\ref{#1}}}

\newcommand{\pseudoIndent}{

\vspace{10pt}\hspace*{12pt}}

\newcommand{\YPDFI}{{\fontfamily{fvs}\selectfont YPDF-Interactive}}

%
\newcommand{\deconum}[1]{{\protect\raisebox{-1pt}{{\LARGE #1}}}}



\newcommand{\VersatileUX}{{\color{red!85!black}{\Fontauri Versatile}}%
{{\fontfamily{qhv}\selectfont\smaller UX}}}

\newcommand{\NDPCloud}{{\color{red!15!black}%
{\fontfamily{qhv}\selectfont {\smaller NDP C{\smaller LOUD}}}}}

\newcommand{\lfNDPCloud}{{\color{red!15!black}%
{\fontfamily{qhv}\selectfont N{\smaller DP C{\smaller LOUD}}}}}

\newcommand{\textds}[1]{{\fontfamily{lmdh}\selectfont{%
\raisebox{-1pt}{#1}}}}

\newcommand{\dsC}{{\textds{ds}{\fontfamily{qhv}\selectfont \raisebox{-1pt}
{\color{red!15!black}{C}}}}}

\definecolor{tcolor}{RGB}{24,52,61}

\newcommand{\HTXN}{\resizebox{!}{8pt}{\AcronymText{HTXN}}}
\newcommand{\lHTXN}{\resizebox{!}{8.5pt}{\AcronymText{HTXN}}}
\newcommand{\lsHTXN}{\resizebox{!}{9.5pt}{\AcronymText{\textcolor{tcolor}{HTXN}}}}


\usepackage{mdframed}

\newcommand{\cframedboxpanda}[1]{\begin{mdframed}[linecolor=yellow!70!blue,linewidth=0.4mm]#1\end{mdframed}}


\newcommand{\PVD}{\resizebox{!}{8pt}{\AcronymText{PVD}}}

\newcommand{\sapp}{\resizebox{!}{8pt}{\AcronymText{Sapien+}}}
\newcommand{\lsapp}{\resizebox{!}{8.5pt}{\AcronymText{Sapien+}}}
\newcommand{\lssapp}{\resizebox{!}{9.5pt}{\AcronymText{Sapien+}}}

\newcommand{\ePub}{\resizebox{!}{8pt}{\AcronymText{ePub}}}

%\lsLPF



\newcommand{\LPF}{\resizebox{!}{8pt}{\AcronymText{LPF}}}
\newcommand{\lLPF}{\resizebox{!}{8.5pt}{\AcronymText{LPF}}}
\newcommand{\lsLPF}{\resizebox{!}{9.5pt}{\AcronymText{LPF}}}

\makeatletter

\newcommand*\getX[1]{\expandafter\getX@i#1\@nil}

\newcommand*\getY[1]{\expandafter\getY@i#1\@nil}
\def\getX@i#1,#2\@nil{#1}
\def\getY@i#1,#2\@nil{#2}
\makeatother
	
\newcommand{\rectann}[9]{%
\path [draw=#1,draw opacity=#2,line width=#3, fill=#4, fill opacity = #5, even odd rule] %
(#6) rectangle(\getX{#6}+#7,\getY{#6}+#8)
({\getX{#6}+((#7-(#7*#9))/2)},{\getY{#6}+((#8-(#8*#9))/2)}) rectangle %
({\getX{#6}+((#7-(#7*#9))/2)+#7*#9},{\getY{#6}+((#8-(#8*#9))/2)+#8*#9});}


\definecolor{pfcolor}{RGB}{94, 54, 73}

\newcommand{\EPF}{\resizebox{!}{8pt}{\AcronymText{ETS{\color{pfcolor}pf}}}}
\newcommand{\lEPF}{\resizebox{!}{8.5pt}{\AcronymText{ETS{\color{pfcolor}pf}}}}
\newcommand{\lsEPF}{\resizebox{!}{9.5pt}{\AcronymText{ETS{\color{pfcolor}pf}}}}


\newcommand{\XPDF}{\resizebox{!}{8.5pt}{\AcronymText{XPDF}}}

\newcommand{\GRE}{\resizebox{!}{8.5pt}{\AcronymText{GRE}}}

\newcommand{\lMOSAIC}{\resizebox{!}{8.5pt}{\AcronymText{MOSAIC}}}

\newcommand{\XML}{\resizebox{!}{8pt}{\AcronymText{XML}}}
\newcommand{\RDF}{\resizebox{!}{8pt}{\AcronymText{RDF}}}
\newcommand{\DOM}{\resizebox{!}{8pt}{\AcronymText{DOM}}}

\newcommand{\CLang}{\resizebox{!}{8pt}{\AcronymText{C}}}

\newcommand{\HNaN}{\resizebox{!}{8pt}{\AcronymText{HN%
\textsc{a}N}}}


\newcommand{\MeshLab}{\resizebox{!}{8pt}{\AcronymText{MeshLab}}}
\newcommand{\IQmol}{\resizebox{!}{8pt}{\AcronymText{IQmol}}}

\newcommand{\SGML}{\resizebox{!}{8pt}{\AcronymText{SGML}}}

\newcommand{\GUI}{\resizebox{!}{8pt}{\AcronymText{GUI}}}

\newcommand{\API}{\resizebox{!}{8pt}{\AcronymText{API}}}

\newcommand{\SDI}{\resizebox{!}{8pt}{\AcronymText{SDI}}}

\newcommand{\IDE}{\resizebox{!}{8pt}{\AcronymText{IDE}}}

\newcommand{\ThreeD}{\resizebox{!}{8pt}{\AcronymText{3D}}}

\newcommand{\FAIR}{\resizebox{!}{8pt}{\AcronymText{FAIR}}}

\newcommand{\QNetworkManager}{\resizebox{!}{8pt}{\AcronymText{QNetworkManager}}}
\newcommand{\QTextDocument}{\resizebox{!}{8pt}{\AcronymText{QTextDocument}}}
\newcommand{\QWebEngineView}{\resizebox{!}{8pt}{\AcronymText{QWebEngineView}}}
\newcommand{\HTTP}{\resizebox{!}{8pt}{\AcronymText{HTTP}}}


\newcommand{\lAcronymTextNC}[2]{{\fontfamily{fvs}\selectfont {\Large{#1}}{\large{#2}}}}

\newcommand{\AcronymTextNC}[1]{{\fontfamily{fvs}\selectfont {\large #1}}}


\colorlet{orr}{orange!60!red}

\newcommand{\textscc}[1]{{\color{orr!35!black}{{%
						\fontfamily{Cabin-TLF}\fontseries{b}\selectfont{\textsc{\scriptsize{#1}}}}}}}


\newcommand{\textsccserif}[1]{{\color{orr!35!black}{{%
				\scriptsize{\textbf{#1}}}}}}


\newcommand{\iXPDF}{\resizebox{!}{8pt}{\textsccserif{%
\textit{XPDF}}}}

\newcommand{\iEPF}{\resizebox{!}{8pt}{\textsccserif{%
\textit{ETSpf}}}}

\newcommand{\iSDI}{\resizebox{!}{8pt}{\textsccserif{%
\textit{SDI}}}}

\newcommand{\iHTXN}{\resizebox{!}{8pt}{\textsccserif{%
\textit{HTXN}}}}


\newcommand{\AcronymText}[1]{{\textscc{#1}}}

\newcommand{\AcronymTextser}[1]{{\textsccserif{#1}}}


\newcommand{\mAcronymText}[1]{{\textscc{\normalsize{#1}}}}

\newcommand{\TeXMECS}{\resizebox{!}{8pt}{\AcronymText{TeXMECS}}}

\newcommand{\NGML}{\resizebox{!}{8pt}{\AcronymText{NGML}}}

\newcommand{\Cpp}{\resizebox{!}{8.5pt}{\AcronymText{C++}}}

\newcommand{\WhiteDB}{\resizebox{!}{8pt}{\AcronymText{WhiteDB}}}

\colorlet{drp}{darkRed!70!purple}

%\newcommand{\MOSAIC}{{\color{drp}{\AcronymTextNC{\scriptsize{MOSAIC}}}}}

\newcommand{\MOSAIC}{\resizebox{!}{8pt}{\AcronymText{MOSAIC}}}


\newcommand{\mMOSAIC}{{\color{drp}{\AcronymTextNC{\normalsize{MOSAIC}}}}}

\newcommand{\MOSAICVM}{\mMOSAIC-\mAcronymText{VM}}

\newcommand{\sMOSAICVM}{\resizebox{!}{8pt}{\MOSAICVM}}
\newcommand{\sMOSAIC}{\resizebox{!}{8pt}{\MOSAIC}}

\newcommand{\LDOM}{\resizebox{!}{8pt}{\AcronymText{LDOM}}}

\newcommand{\LXCR}{\resizebox{!}{8pt}{\AcronymText{LXCR}}}
\newcommand{\lLXCR}{\resizebox{!}{8.5pt}{\AcronymText{LXCR}}}
\newcommand{\lsLXCR}{\resizebox{!}{9.5pt}{\AcronymText{LXCR}}}

%\newcommand{\lMOSAIC}{{\color{drp}{\lAcronymTextNC{M}{OSAIC}}}}
\newcommand{\lfMOSAIC}{\resizebox{!}{9pt}{{\color{drp}{\lAcronymTextNC{M}{OSAIC}}}}}

\newcommand{\Mosaic}{\resizebox{!}{8pt}{\MOSAIC}}
\newcommand{\MosaicPortal}{{\color{drp}{\AcronymTextNC{MOSAIC Portal}}}}

\newcommand{\RnD}{\resizebox{!}{7.5pt}{\AcronymText{R\&D}}}
\newcommand{\QtCpp}{\resizebox{!}{8.5pt}{\AcronymText{Qt/C++}}}
\newcommand{\Qt}{\resizebox{!}{9pt}{\AcronymText{Qt}}}
\newcommand{\QtSQL}{\resizebox{!}{8pt}{\AcronymText{QtSQL}}}

\newcommand{\HTML}{\resizebox{!}{8pt}{\AcronymText{HTML}}}
\newcommand{\PDF}{\resizebox{!}{8pt}{\AcronymText{PDF}}}

\newcommand{\lGRE}{\resizebox{!}{8pt}{\AcronymText{GRE}}}

\newcommand{\p}[1]{

\vspace{.85em}#1}

\newcommand{\q}[1]{{\fontfamily{qcr}\selectfont ``}#1{\fontfamily{qcr}\selectfont ''}} 

%\newcommand{\deconum}[1]{{\textcircled{#1}}}


\renewcommand{\thesection}{\protect\mbox{\deconum{\Roman{section}}}}
\renewcommand{\thesubsection}{\arabic{section}.\arabic{subsection}}

\newcommand{\llMOSAIC}{\mbox{{\LARGE MOSAIC}}}
%\newcommand{\lfMOSAIC}{\mbox{M\small{OSAIC}}}

\newcommand{\llMosaic}{\llMOSAIC}
\newcommand{\lMosaic}{\lMOSAIC}
\newcommand{\lfMosaic}{\lfMOSAIC}


\newcommand{\llWC}{\mbox{{\LARGE WhiteCharmDB}}}

\newcommand{\llwh}{\mbox{{\LARGE White}}}
\newcommand{\llch}{\mbox{{\LARGE CharmDB}}}

\usepackage{enumitem}

\setlist[description]{%
  topsep=30pt,               % space before start / after end of list
  itemsep=5pt,               % space between items
  font={\bfseries\sffamily}, % set the label font
%  font={\bfseries\sffamily\color{red}}, % if colour is needed
}

\setlist[enumerate]{%
  topsep=3pt,               % space before start / after end of list
  itemsep=-2pt,               % space between items
  font={\bfseries\sffamily}, % set the label font
%  font={\bfseries\sffamily\color{red}}, % if colour is needed
}

%\usepackage{tcolorbox}

\newcommand{\slead}[1]{%
\noindent{\raisebox{2pt}{\relscale{1.15}{{{%
\fcolorbox{logoCyan!50!black}{logoGreen!5}{#1}
}}}}}\hspace{.5em}}


\let\OldLaTeX\LaTeX

\renewcommand{\LaTeX}{\resizebox{!}{8pt}{\color{orr!35!black}{\OldLaTeX}}}

\let\OldTeX\TeX

\renewcommand{\TeX}{\resizebox{!}{8pt}{\color{orr!35!black}{\OldTeX}}}


\newcommand{\LargeLaTeX}{\resizebox{!}{8.5pt}{\color{orr!35!black}{\OldLaTeX}}}


%\setlength\parindent{24pt}
%%\usepakage{newfile}

\usepackage{hyperref}

\usepackage{etoolbox}

\usepackage{zref-user}

\newwrite\sdiFile
\immediate\openout\sdiFile=\jobname.sdi.txt

\newcommand{\p}[1]{

\vspace{10pt}#1}

\newif\iftabng
\tabngfalse


\usepackage{letltxmacro}
\LetLtxMacro{\oldmmsemi}{\;}
\LetLtxMacro{\oldtbplus}{\+}
\LetLtxMacro{\oldtbgt}{\>}
\LetLtxMacro{\oldmmgt}{\+}

\newcommand{\+}{\iftabng\oldtbplus\else\sss\fi}

\renewcommand{\>}{\iftabng\oldtbplus\else
\ifmmode\oldmmgt\else\sse\sss\fi\fi}

%\renewcommand{\>}{\sse\sss}

\renewcommand{\;}{\relax\ifmmode\oldmmsemi\else\sse\fi}

\newcommand{\writeSDI}[1]{\immediate\write\sdiFile#1}

\newcommand{\emblink}[2]{\href{\#sdi:#1--#2}{\#sdi:#1--#2}}

%\newcount\sdiCounter
%\def\advsdiCounter{\global\advance\sdiCounter by1}

%\newcount\sdiCounterP
%\def\advsdiCounterP{\global\advance\sdiCounterP by1}

%\newcounter{sdiCounter}
\newcounter{sdiCounterP}[page]
\newcounter{sdiCounter}

\def\topt#1{\expandafter\the\dimexpr\dimexpr#1sp\relax\relax}

\makeatletter
%\catcode`\*=10
\newcommand{\sss}{%
\stepcounter{sdiCounterP}
\stepcounter{sdiCounter}
\pdfsavepos\write\sdiFile{!/ SDI_Sentence_Start} 
\write\sdiFile\expandafter{\expandafter$%
\expandafter i\expandafter:%
\expandafter\space\the\c@sdiCounter}
\write\sdiFile\expandafter{\expandafter$%
\expandafter o\expandafter:%
\expandafter\space\the\c@sdiCounterP}
\write\sdiFile\expandafter{\expandafter$%
\expandafter p\expandafter:%
\expandafter\space\thepage^^J%
$x: \topt\pdflastxpos^^J%
$y: \topt\pdflastypos^^J%
/!^^J%
<<>^^J%
}}
%\catcode`\%=14
\makeatother

\makeatletter
\newcommand{\sse}{%
\pdfsavepos\write\sdiFile{!/ SDI_Sentence_End} 
\write\sdiFile\expandafter{\expandafter$%
\expandafter i\expandafter:%
\expandafter\space\the\c@sdiCounter}
\write\sdiFile\expandafter{\expandafter$%
\expandafter o\expandafter:%
\expandafter\space\the\c@sdiCounterP}
\write\sdiFile\expandafter{\expandafter$%
\expandafter p\expandafter:%
\expandafter\space\thepage^^J%
$x: \topt\pdflastxpos^^J%
$y: \topt\pdflastypos^^J%
/!^^J%
<<>^^J%
}}
\makeatother




\newcommand{\lun}[1]{\raisebox{-4pt}{\fontfamily{qcr}\selectfont{%
\LARGE{\textbf{\textcolor{tcolor}{#1}}}}}\vspace{-2pt}}

\newcommand{\inditem}{\itemindent10pt\item}

\usepackage{soul}

\definecolor{hlcolor}{RGB}{114, 54, 203}
\colorlet{hlcol}{hlcolor!35}
\sethlcolor{hlcol}

\makeatletter
\def\SOUL@hlpreamble{%
	\setul{}{3ex}%         !!!change this value!!! default is 2.5ex
	\let\SOUL@stcolor\SOUL@hlcolor
	\SOUL@stpreamble
}
\makeatother

\usepackage{scrextend}
%\vspace*{3em}
\newenvironment{mldescription}{\vspace{1em}%
  \begin{addmargin}[4pt]{1em}
    \setlength{\parindent}{-1em}%
    \newcommand*{\mlitem}[1][]{\vspace{5pt}\par\medskip%
%\colorbox{hlcolor}{\textbf{##1}}\quad}\indent
\hl{ \textbf{##1} }\quad}\indent
}{%
  \end{addmargin}
  \medskip
}

\usepackage{marginnote}

\newcommand{\mnote}[1]{%
\vspace*{-2em}
\reversemarginpar
\raisebox{1em}{\marginnote{\parbox{4em}{%
\begin{mdframed}[innerleftmargin=4pt,
	innerrightmargin=1pt,innertopmargin=1pt,
	linecolor=red!20!cyan,userdefinedwidth=4em,
	topline=false,
	rightline=false]
{{\fontfamily{ppl}\fontsize{12}{0}\selectfont
		\textit{#1}}}
\end{mdframed}}
	}[3em]}}

\newcommand{\mnotel}[1]{%
\vspace*{-2em}
\reversemarginpar
\raisebox{-4em}{\marginnote{\parbox{4em}{%
\begin{mdframed}[innerleftmargin=4pt,
	innerrightmargin=1pt,innertopmargin=1pt,
	linecolor=red!20!cyan,userdefinedwidth=4em,
	topline=false,
	rightline=false]
{{\fontfamily{ppl}\fontsize{12}{0}\selectfont
		\textit{#1}}}
\end{mdframed}}
	}[3em]}}

\newcommand{\mnoteh}[3]{%
	\vspace*{#1}
	\reversemarginpar
	\raisebox{#2}{\marginnote{\parbox{4em}{%
				\begin{mdframed}[innerleftmargin=4pt,
					innerrightmargin=1pt,innertopmargin=1pt,
					linecolor=red!20!cyan,userdefinedwidth=4em,
					topline=false,
					rightline=false]
					{{\fontfamily{ppl}\fontsize{12}{0}\selectfont
							\textit{#3}}}
				\end{mdframed}}
			}[3em]}}


\newcommand{\mnoteb}[1]{%
	\vspace*{1em}
	\reversemarginpar
	\raisebox{1em}{\marginnote{\parbox{4em}{%
				\begin{mdframed}[innerleftmargin=4pt,
					innerrightmargin=1pt,innertopmargin=1pt,
					linecolor=red!20!cyan,userdefinedwidth=4em,
					topline=false,
					rightline=false]
					{{\fontfamily{ppl}\fontsize{12}{0}\selectfont
							\textit{#1}}}
				\end{mdframed}}
			}[3em]}}
	
\usepackage{wrapfig}

\usetikzlibrary{arrows, decorations.markings}
\usetikzlibrary{shapes.arrows}

\newcommand{\curicon}[2]{%
	\node at (#1,#2) [
	draw=black,
	%minimum width=2ex,
	inner sep=.7pt,
	fill=white,
	single arrow,
	single arrow head extend=3pt,
	single arrow head indent=1.5pt,
	single arrow tip angle=45,
	line join=bevel,
	minimum height=4.6mm,
	rotate=115
	] {};
}

\begin{document}
	
{\linespread{1.1}\selectfont

\vspace*{-7em}

\begin{center}
%{\relscale{1.2}{\fontfamily{qcr}\fontseries{b}\selectfont 
%{\colorbox{black}{\color{blue}{\llWC{} Database Engine \\and 
%\llMOSAIC{} Native Application Toolkit}}}}}

\colorlet{ctmp}{logoPeach!20!gray}
\colorlet{ctmpp}{ctmp!90!yellow}
\colorlet{ctmppp}{ctmpp!50!black}
\colorlet{ctmpppp}{ctmppp!90!logoRed}

\vspace{1em}


%{\colorbox{darkBlGreen!30!darkRed}{%
\begin{tcolorbox}
[
%%enhanced,
%%frame hidden,
%interior hidden
arc=2pt,outer arc=0pt,
enhanced jigsaw,
width=.984\textwidth,
colback=ctmpppp!30,
colframe=logoRed!30!darkRed,
drop shadow=logoPurple!50!darkRed,
%boxsep=0pt,
%left=0pt,
%right=0pt,
%top=2pt,
]
\begin{minipage}{\textwidth}	
\begin{center}		
{\setlength{\fboxsep}{19pt}
	\relscale{1.4}{{\fontfamily{qcr}\fontseries{b}\selectfont%
{An ETS Plugin Framework for Enhanced Test Preparation}}}}
\end{center}
\end{minipage}
\end{tcolorbox}
\end{center}

\vspace{-1.5em}

%\noindent\lun{Overview}
\fontfamily{ptm}\fontsize{13pt}{18pt}\selectfont
{\sectsp}
\p{We are proposing a novel ETS Plugin Framework (\EPF{}) 
whose purpose is to enhance document viewers, 
in tandem with scientific and multimedia applications, 
to improve test preparation.
In so doing, this plugin framework would allow 
document viewers to launch and share 
data with a diverse array of applications 
that exist for both scientific and social 
science/humanities disciplines, such as 
chemistry, physics, biology, 
medicine, linguistics, sociology, and literature. 
Document viewers would therefore be able to support an 
interactive, multimedia reading/studying 
experience to an unprecedented degree.  
In particular, students preparing for exams 
would have at their 
disposal stimulating multimedia presentations that offer sophisticated 
data visualization and \ThreeD{} graphics tools, 
customized for individual subjects: e.g., 
\ThreeD{} molecular models for chemistry, 
or \ThreeD{} tissue models for biology.
In addition to offering multimedia features,
ETS plugins could likewise enhance document viewers 
with instructional features that are 
supplemental to the documents which students 
are reading; for example, review questions, 
assignment instructions, or definitions of important 
concepts.}

\vspace{1.5em}
\noindent\lun{ETS\textsc{pf} for Scientific and Technical Applications}

%{\sectsp}
\vspace{.5em}
\mnote{The \iEPF{} toolkit}
\p{\lEPF{} does not refer to a 
single plugin, but to a 
toolkit for implementing multiple ETS plugins to be 
embedded in many different scientific and 
social-scientific applications.  
These plugins should be sufficiently similar to 
one another so that students or instructors familiar with an 
ETS plugin in one context  
would quickly understand how to use ETS plugins found 
in a different context.  One important feature 
of this framework is that distinct 
ETS plugins would be able to 
communicate with one another.  For example, 
plugins for document viewers would send data to 
plugins for scientific or multimedia applications.  
In this way, students would be able to access multimedia content 
linked to the documents (viz., the test-preparation materials) 
that they are currently studying.}

\setlength{\belowcaptionskip}{10pt}
\mnote{How \iEPF{} enables multi-appli- cation 
networking and\\inter-\\operability}%\\\vspace{27em}}
\p{\begin{wrapfigure}[11]{r}{110mm}
\vspace{-1.4em}\caption{A Thionyl Chloride Question in XpdfReader}
\label{fig:tc}
\begin{tikzpicture}

\node[inner sep=0pt] (x1) at (0,0)
    {\includegraphics[width=110mm, 
    	trim={104mm 10mm 1mm 34mm},clip]
    	{xScreenShot2.png}};

\curicon{0.89}{-2.16}

\rectann{darkRed!50}{0.7}{2pt}{logoCyan}{0.5}{-1.3,-2.41}{6.45}{3.1}{1.1}

\end{tikzpicture}


\end{wrapfigure}For a concrete example of advanced functionality 
that can be achieved by connecting two distinct 
EPF plugins, consider a student reading through the ETS \GRE{} 
Chemistry practice test.  This book 
has sample multiple-choice questions such as 
(on page 11, number 4), 
\q{\textbf{The molecular geometry of thionyl chloride, 
SOCl$_2$, is best described as} \textit{(A) 
trigonal planar, (B) T-shaped, (C) tetrahedral, 
(D) trigonal pyramidal, or (E) linear}}.
To understand 
this question and its corresponding multiple-choice answers, it would help students to be able to view a 
\ThreeD{} model of thionyl chloride, which can be done 
with the aid of molecular visualization software, such as IQmol.  To support this functionality, 
our plugin within the document-viewer application 
(here \XPDF{})
would launch IQmol, sending data through a corresponding 
ETS plugin embedded in IQmol.  
Specifically, question \textit{4} in the practice test  
may be associated with a Molecular Data file for 
SOCl$_2$; the \XPDF{} plugin would  \makebox{launch IQmol, sending 
along a data package identifying this 
SOCl$_2$ file to IQmol's own plugin, with}

\setlength{\belowcaptionskip}{4pt}
\begin{wrapfigure}[11]{l}{110mm}
\vspace{-1em}

%\begin{figure}[h]
\caption{Thionyl Chloride in IQmol}
\label{fig:i1}

\begin{tikzpicture}

\node[inner sep=0pt] (x1) at (0,0)
    {\includegraphics[width=110mm, 
    	trim={0mm 25mm 6mm 0mm},clip]
    	{iScreenShot1.png}};
    
%\curicon{0}{10}

\end{tikzpicture}
%\end{figure}

\end{wrapfigure}\vspace{.2em}\noindent{}%

\vspace{-1.7em}instructions 
to load the file into an IQmol session 
(see Figure~\bref{fig:i1}).  The 
student could initiate this process by 
selecting a context-menu option 
(called \q{3D thionyl chloride Viewer} in 
Figure~\bref{fig:tc}, the highlighted option where the 
cursor is pointing).
The end result, then,  
would be that the student, with a single 
click, has access 
to an interactive \ThreeD{} graphic representing 
thionyl chloride.  The same 
functionality would be available for any chemical 
compound which has associated data in formats such as Molecular Data, Protein Data Bank, 
or Chemical Markup Language.}

\vspace{-3em}
\mnotel{\\\iEPF{} features\\for\\keeping track 
of\\students' previous activity.}
\p{%
\setlength{\belowcaptionskip}{4pt}%
\begin{wrapfigure}[11]{r}{110mm}
\vspace{-1.5em}
\begin{tikzpicture}

\node[inner sep=0pt] (x1) at (0,0)
    {\includegraphics[width=110mm, 
    	trim={60mm 40mm 20mm 5mm},clip]
    	{xScreenShot1.png}};

\curicon{2.6}{.94}

\rectann{darkRed!50}{0.7}{2pt}{logoCyan}{0.5}{-4.2,-3.4}{6.25}{2.25}{0.9}

\end{tikzpicture}


\end{wrapfigure}The previous case-study involving Thionyl Chloride exemplified 
a simple data structure transmitted between ETS plugins: 
specifically, the name of a single file to open.  
However, in other cases, the information sent between plugins might 
be more complex and detailed.  To accommodate this, all ETS plugins would require a 
\textit{common} vocabulary for representing multi-part data 
structures.  
For example, if a student views a \textit{second} 
molecule in IQmol, the document viewer should 
identify not only \textit{that} file, but any 
\textit{previous} files they had viewed, so that 
the student could conveniently refer back to 
those previous files as desired.  This 
would be the case where a student reading through the GRE Chemistry 
practice exam chooses to launch IQmol a second time --- perhaps in 
conjunction with a later question 
about the molecular structure of lactose, 
such as question number \textit{95} in the test (see 
Figure~\ref{fig:x2} above).
In this case, the plugin would send 
information not only about the present 
(lactose) request but also 
about the SOCl$_2$ (Thionyl Chloride) 
file that they had 
viewed earlier.  This is 
visible within the Model View panel at 
the top-left on 
Figure~\bref{fig:i2}, where the SOCl$_2$  
file is listed above the checked lactose file (the lactose file is checked because it is the one currently seen in the view-port).  
As this example illustrates, the plugins sharing data about a student's prior 
actions makes both applications more 
interactive, ensuring that students 
benefit from a flexible and responsive User 
Experience.}
%\begin{figure}[h]
\caption{Lactose in IQmol}
\label{fig:i2}

\begin{tikzpicture}

\node[inner sep=0pt] (x1) at (0,0)
    {\includegraphics[%width=110mm, 
    	trim={0mm 0 0 0mm},clip]
    	{iScreenShot2.png}};
    \colorlet{cyg}{cyan!70!gray}
 \node [anchor=west,top color=cyg!20,
 bottom color=red!40!brown, shading angle=230, 
 inner sep=3, text width=13.7cm]
 (longnote) at (-9.25,4.8) {\vspace{-8pt}%  %{\color{rb!85!red}{
 {\cframedboxpanda{\vspace{1pt}\large \textbf{%
\makebox{14641-93-1} is the \makebox{Chemical} Abstracts Service 
code for Lactose; 
\makebox{7719-09-7} is the corresponding code 
for Thionyl Chloride.}\vspace{0pt}}}};

\end{tikzpicture}
\end{figure}


%\vspace{-.75em}
\p{%
\setlength{\belowcaptionskip}{4pt}%
\begin{wrapfigure}[14]{l}{115mm}\vspace{-1.5em}

\begin{tikzpicture}

\node[inner sep=0pt] (x1) at (0,0)
    {\includegraphics[width=115mm, 
    	trim={54mm 0 0 0mm},clip]
    	{xScreenShot.png}};

\curicon{3.57}{-1.1}

\end{tikzpicture}
%
\end{wrapfigure}In general, the functionality 
	provided by each ETS plugin would 
	depend in part on the domain of the host application in which the 
	plugin is embedded.  For example, an IQmol plugin would load 
	cheminformatic files and may activate IQmol's analytic 
	capabilities in the domain of chemistry, whereas 
	a plugin for applications in the domain of quantitative/statistical analysis and data visualization, such as ParaView, would load quantitative data sets 
	(with 2D or 3D views via surfaces, scatter-plots, 
	bar charts, etc.). 
Nevertheless, certain functionality would be shared 
	among all ETS plugins, which would include 
	common data-sharing vocabulary (as mentioned above), as well as dialog windows to 
	show basic plugin information 
(see Figure~\bref{fig:xss}) 
	alongside a more detailed review of the data 
	that has been transmitted 
	between applications via plugins.  
	Specifically, the \q{Request/Launch Info} tab would allow students, instructors, 
	and plugin developers to see information about the 
	request which prompted the current application to be 
	launched and/or to open a specific file (see Figure~\bref{fig:i6}).}
\setlength{\belowcaptionskip}{0pt}
\begin{figure}[h]
\caption{Lactose in IQmol}
\label{fig:i2}

\begin{tikzpicture}

\node[inner sep=0pt] (x1) at (0,0)
    {\includegraphics[%width=110mm, 
    	trim={0mm 0 0 0mm},clip]
    	{iScreenShot2.png}};
    \colorlet{cyg}{cyan!70!gray}
 \node [anchor=west,top color=cyg!20,
 bottom color=red!40!brown, shading angle=230, 
 inner sep=3, text width=13.7cm]
 (longnote) at (-9.25,4.8) {\vspace{-8pt}%  %{\color{rb!85!red}{
 {\cframedboxpanda{\vspace{1pt}\large \textbf{%
\makebox{14641-93-1} is the \makebox{Chemical} Abstracts Service 
code for Lactose; 
\makebox{7719-09-7} is the corresponding code 
for Thionyl Chloride.}\vspace{0pt}}}};

\end{tikzpicture}
\end{figure}

%

\vspace{2em}
\noindent\lun{ETSpf Tools for Composing Test-Preparation Materials}
{\sectsp}
\vspace{-6.25em}

\mnoteh{-2em}{-1em}{\iEPF{} data\\in em-\\bedded files}\vspace{4.25em}
\p{\vspace{.4em}In most cases, ETS plugins for document viewers such as 
\XPDF{} would draw information from 
\PDF{} files (or files in other formats, 
e.g. \ePub{} or \HTML{}) to implement 
teaching enhancements, such as integration 
with scientific and multimedia applications.  
This \EPF{}-specific data can be placed in a separate 
file embedded in \PDF{} or \ePub{} documents, or 
(in \HTML{}) inserted as non-display contents.
When a document is opened, the ETS plugin would then 
extract the embedded file so as to read 
\EPF{}-specific data about the document --- in 
particular, to identify \PDF{} coordinates for 
document elements requiring special \EPF{} actions.  For questions \textit{4} and \textit{95} as 
illustrated above, the relevant 
\EPF{} action would be an option to view the 
question-specific molecular files in IQmol.  
\lEPF{} data is needed in order to 
map the textual boundaries of the 
question (and its multiple-choice answers) 
to on-screen coordinates, so that  
context menus can be customized 
for each question.}  

\setlength{\textfloatsep}{10pt}\setlength{\belowcaptionskip}{4pt}\begin{figure}[h!]
\caption{Request Information in IQmol}
\label{fig:i6}

\begin{tikzpicture}

\node[inner sep=0pt] (x1) at (0,0)
    {\includegraphics[width=\textwidth, 
    	trim={0mm 5mm 0mm 0mm},clip]
    	{iScreenShot.png}};

\curicon{8.2}{-6.1}

\end{tikzpicture}
\end{figure}


%\vspace{-.5em}
\mnote{Semantic Document Infosets 
(\iSDI{}s)}
\p{To support these capabilities, \EPF{} would 
include tools to help compose publications 
(such as test-preparation materials) that 
embed what we term a \q{Semantic Document Infoset} 
(\SDI{}), which effectively divides manuscripts into 
textual units (subsections, paragraphs, sentences, 
quotations, bullet lists, etc.) and identifies 
document elements such 
as technical terms (which may be compiled into a 
glossary) and figure illustrations.
\EPF{} code can then examine a publication's 
\SDI{} to generate machine-readable 
structural representations of publication manuscripts, 
which document viewers may use to augment the 
underlying document with additional  
instructional and/or multimedia features --- review questions, 
student instructions, glossaries, reading assignments, 
and so forth.  The \SDI{} can be used 
to guide \EPF{} plugins when sharing data 
between applications --- in 
Figure~\bref{fig:tc}, for instance, 
selecting the Molecular Data file to 
send to IQmol based on the screen 
coordinates of the context menu --- 
but also to enhance the 
presentation of content within the host application.  
For example, Figure~\bref{fig:qa} shows how 
an \EPF{} plugin could provide an alternative 
interface for viewing practice-test questions, 
where readers can consider one question at a time, 
isolated in its own window, 
which may help them focus 
attention on each question in turn.}
\setlength{\intextsep}{8pt}\setlength{\belowcaptionskip}{4pt}
\begin{figure}[h]
\caption{A Sample Practice-Test Question within XpdfReader}
\label{fig:qa}

\begin{tikzpicture}

\node[inner sep=0pt] (x1) at (0,0)
    {\includegraphics[%width=110mm, 
    	trim={7mm 26mm 3mm 1mm},clip]
    	{ScreenShot.png}};

\node[inner sep=0pt] (x1) at (0,-6)
{\includegraphics[%width=110mm, 
	trim={7mm 7mm 3mm 130mm},clip]
	{ScreenShot.png}};

\curicon{2.6}{-4.49}

\end{tikzpicture}

\end{figure}


%\mnote{Writing exams with \LaTeX{} and \iXPDF{} plugins.}
\vspace{-.5em}
\mnote{Using \LaTeX{} to\\generate \iSDI{} info- sets}
\p{\EPF{} implementations can include 
\LaTeX{} packages which automate the creation 
of \SDI{} data (placed 
as an embedded file in the generated \PDF{} document).  
This embedded data can then be read by 
\EPF{} plugins to compose multi-application networking 
requests, populate question/answer windows, or introduce 
other kinds of teaching content: review questions, 
glossaries, discussions of figure illustrations, etc. 
In documents where 
questions are printed as part of the publication 
text (for example, the ETS \GRE{} practices), 
the \LaTeX{} code can store the \PDF{} coordinates 
for the questions so that the document 
automatically scrolls while 
students work their way through a practice test session.  
Alternatively, the same techniques can be 
used to add review questions and answers to 
documents which are not expressly designed 
as test-prep materials, such as textbooks and 
research papers.  In this latter case, 
question/answer windows may be synced, 
using the \SDI{}, to 
sentences or paragraphs in those publications 
which are relevant to the review question that 
the student is currently reading/studying.}

\vspace*{-3.25em}
\mnotel{\iHTXN{} (Hypergraph Text Encoding Protocol) Specifications}  
\p{As an additional feature, \EPF{} plugins 
would implement a protocol which we call 
\HTXN{} (for \q{Hypergraph Text Encoding}).  
The goal of \HTXN{} is to enable a new generation of 
publishing technologies which aspire to support 
multimedia reader experiences.  
In so doing, the traditional manuscript --- the \q{primary} 
resource which is cited and downloaded --- 
would then be networked with a package of 
supplemental (or \q{secondary}) resources.  
However, at present, even when documents have 
supplemental files, it can be very difficult to 
transition from the primary to the secondary 
resource.  To address this problem, \lHTXN{} is designed to 
rigorously document these multimedia networks, enabling 
e-readers and domain-specific applications to be 
integrated so that users may easily access multimedia content.  
The \HTXN{} protocol uses \q{standoff annotation} (i.e., character 
encoding and document structure are 
defined in isolation from one another), and 
can be employed to encode manuscripts in 
different markup formats (both \LaTeX{} and \XML{}, 
for instance).}

\vspace*{-.25em}
\p{LTS can provide a demo with a more detailed overview of 
\EPF{}, additional use-cases, technical 
information about plugin code, and sample 
\HTXN{}-encoded documents.}
\end{document}


