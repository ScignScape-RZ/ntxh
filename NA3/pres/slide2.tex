
\begin{frame}{\ft{The NCN (Native Cloud/Native) Protocol}}
	\section{NCN}
\vspace{-3em}	
{\Large\fontfamily{uhv}\selectfont
\vspace{1em}
\begin{center}
\begin{minipage}{.9\textwidth}
\vspace{1em}
\fcolorbox{lqboutercolor}{lqbinnercolor}{\begin{minipage}{\textwidth}%
\begin{lightquadblock}{Cloud/Native Components as back-ends 
for native software}
\begin{center}\begin{minipage}{.98\textwidth}
{\Large \begin{itemize}
\item \q{Native Cloud/Native} refers to native application front-ends paired with 
Cloud/Native container instances.\vspace{1em}
\item  Share code libraries and data representation 
across both endpoints.\vspace{1em}
\item  Common representation on both server- and client-side 
streamlines network communications (no need to marshal data between 
different formats).\vspace{1em}
\item  This presentation will focus on NA3's default 
Qt{} implementation, though the technology can be 
ported to other application frameworks 
(wxWidgets, XCode, MFC, etc.).
\\\vspace{1.5em}
\end{itemize}}\end{minipage}
\end{center}
\end{lightquadblock}
\end{minipage}}
\fcolorbox{lqboutercolor}{lqbinnercolor}{\begin{minipage}{\textwidth}%
\begin{lightquadblock}{How Cloud Back-Ends Enhance Native Front Ends}
{\Large\begin{itemize}
\item  Cloud Backup; Share data between users; Collaborative Editing \vspace{1em}
\item  Persist users' application state across different 
computers (home/school/office)\vspace{1em}
\item  Upgrade running applications without re-compile
\vspace{1em}
\end{itemize} }
\end{lightquadblock}
\end{minipage}}

\end{minipage}
\end{center}
}

\end{frame}
