
\begin{frame}{\ft{The Qt Ecosystem}}
\section{Qt}
\vspace{.5em}	

{\begin{minipage}[c]{\textwidth}\Large\centering\color{slidePartHeadColor} 	
{\color[rgb]{.3,.1,0}{\Huge\fontfamily{bch}\fontseries{eb}\selectfont Qt is 
the most popular native, cross-platform application-development framework.}}\\
\vspace{1em}	
{\LARGE \MyDiamond{} \texttildelow{}1 million active developers \MyDiamond{}  Over 5,000 client companies \MyDiamond{}  
Worldwide \\\q{Qt Partners} Ecosystem  
\MyDiamond{} \texttildelow{}US \$250 million overall market }\end{minipage}}
\vspace{1em}
	
{\Large\fontfamily{uhv}\selectfont
\vspace{1em}
\begin{center}
\begin{minipage}{.9\textwidth}
\vspace{1em}
\fcolorbox{lqboutercolor}{lqbinnercolor}{\begin{minipage}{\textwidth}%
\begin{lightquadblock}{However ... Limited Qt Cloud Integration Support}
\begin{center}\begin{minipage}{.98\textwidth}
{\LARGE \setlength{\leftmargini}{30pt}\begin{itemize}
\item \hspace{.3em} \q{Qt Cloud Services} Discontinued in 2016. \vspace{1em}
\item \hspace{.3em} \parbox{14cm}{Currently there is no standard model for accessing 
\\Cloud services from Qt applications.}\vspace{1em}
\item \hspace{.3em} \parbox{14cm}{Nor is there a standard Qt-based Cloud/Native \\container architecture.}
\\\vspace{1.5em}
\end{itemize}}\end{minipage}
\end{center}
\end{lightquadblock}
\end{minipage}}


\end{minipage}
\end{center}
}

\end{frame}
