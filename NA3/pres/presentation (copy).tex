
\documentclass{beamer}

\usepackage{amssymb}
\usepackage{textcomp}
\usepackage{tikz}

\usetikzlibrary{arrows, positioning, shapes}

\usetheme{Singapore}
\setbeamertemplate{blocks}[rounded][shadow=true]
\setbeamertemplate{frametitle}[default][center]
\setbeamertemplate{caption}[numbered]

\setlength{\paperwidth}{10.in}
\setlength{\paperheight}{7.5in}
\setlength{\textwidth}{9.in}
\setlength{\textheight}{6.5in}

\newcommand{\cfbox}[2]{%
	\colorlet{currentcolor}{.}%
	{\color[rgb]{#1}%
		\fbox{\color{currentcolor}#2}}%
}


\usecolortheme{spruce}
%\setbeamercolor{block title}{red}   
%\setbeamercolor{block title}{red}   
\usecolortheme{orchid}
%\usepackage{float}

%\setbeamerfont{itemize/enumerate body}{size=\small}
%\setbeamerfont{normaltext}{size=\}
%\setbeamerfont{block body}{size=\small}
\setbeamerfont{block title}{size=\large}
%\setbeamerfont{block body example}{size=\tiny}

\newcommand{\colortriangle}{{\color[rgb]{0.45, 0.4, 0.28}$\blacktriangleright$}}
	
\newcommand\FourQuad[4]{%
	
	\begin{minipage}[b][.45\textheight][t]{.47\textwidth}#1\end{minipage}\hfill%
	\begin{minipage}[b][.45\textheight][t]{.47\textwidth}#2\end{minipage}\\[0.5em]
	\begin{minipage}[b][.45\textheight][t]{.47\textwidth}#3\end{minipage}\hfill
	\begin{minipage}[b][.45\textheight][t]{.47\textwidth}#4\end{minipage}%
}

\newcommand\ThreeQuad[3]{%
	\hspace{2pt}\begin{minipage}[b][.23\textheight][c]{.99\textwidth}#1\end{minipage}\hfill%
	\begin{minipage}[b][.75\textheight][t]{.55\textwidth}#2\end{minipage}\hfill
	\begin{minipage}[b][.75\textheight][t]{.43\textwidth}#3\end{minipage}%
}

\newcommand\TwoQuad[2]{%
	\begin{minipage}[b][.75\textheight][t]{.55\textwidth}#1\end{minipage}\hfill
	\begin{minipage}[b][.75\textheight][t]{.43\textwidth}#2\end{minipage}%
}

\newenvironment{quadblock}[1]{\begin{block}{\begin{center}\Large{%
\colorbox[rgb]{0.25,0.1,0.25}{\color{yellow}{\textbf{#1}}}
}\end{center}}%
\begin{minipage}[c]{.97\textwidth}\vspace{1em}
}{
\end{minipage}
\end{block}}


\newenvironment{mpblock}[1]{\begin{block}{\begin{center}\Large{%
\colorbox[rgb]{0.25,0.1,0.25}{\color{yellow}{\textbf{#1}}}
}\end{center}}%
\begin{minipage}[c]{.97\textwidth}\vspace{1em}
\begin{textsf}}{\end{textsf}
\end{minipage}
\end{block}}



%\newcommand{\doubleCenteredFramebox}[1]{%
%\begin{center}%
%\framebox{%
%\begin{minipage}{0.85\textwidth}\begin{center}%
%#1\end{minipage}\end{center}}\end{center}}

%\setlength{\fboxsep}{0.15em}



\newcommand{\doubleCenteredFramebox}[1]{%
\setlength{\fboxsep}{2em}\setlength{\fboxrule}{2pt}
\begin{center}\cfbox{0.1, 0.5, 0.8}{\begin{minipage}{0.85\textwidth}\begin{center}\setlength{\fboxsep}{.25em}#1
	\end{center}\end{minipage}}\end{center}}

\newcommand{\sfitem}[1]{\item \LARGE{\textbf{\textsf{#1}}}}				
\newcommand{\scitem}[1]{\item \large{\textsc{#1}}}			
\newcommand{\embitem}[1]{\item \textbf{#1}}			


%\begin{minipage}[.45\textheight][t]}{\end{minipage}\end{block}}

\newcommand{\manyasciimacron}{\textasciimacron\textasciimacron\textasciimacron%
\textasciimacron\textasciimacron\textasciimacron\textasciimacron\textasciimacron\textasciimacron%
\textasciimacron\textasciimacron\textasciimacron\textasciimacron\textasciimacron\textasciimacron%
\textasciimacron\textasciimacron\textasciimacron\textasciimacron\textasciimacron\textasciimacron%
}

\newcommand{\manyemdash}{\texttwelveudash\texttwelveudash\texttwelveudash\texttwelveudash\texttwelveudash%
 \texttwelveudash\texttwelveudash\texttwelveudash\texttwelveudash\texttwelveudash\texttwelveudash\texttwelveudash%
 \texttwelveudash\texttwelveudash\texttwelveudash\texttwelveudash\texttwelveudash\texttwelveudash\texttwelveudash%
}

%\newcommand{\conversationPatterns}{\framebox{%
%Conver\-\\
%sation\\
%Patterns}}

\newcommand{\conversationPatterns}{	
\begin{minipage}{35pt}\centering{\begin{footnotesize}\textbf{Convers\-ation\\Patterns}\end{footnotesize}} 
\end{minipage}}


\definecolor{blGreen}{rgb}{.2,.7,.3}
\definecolor{darkRed}{rgb}{.2,.0,.1}

\begin{document}

\begin{frame}{\Huge{\textbf{Language Techology, Threat Mitigation}}}
{\Huge
\vspace{2em}{\color{darkRed}
\colorbox{pink}
{\begin{minipage}{.55\textwidth}\centering 
Identify Recruitment Patterns and Imminent Threats \\ with 
Multilevel Language Analysis \\ --- Semantics, Sequence Packages, 
and Conversation Patterns --- \\ Integrated with Structured Data
\end{minipage}}}\vspace{2em}\\
\colorbox{white}{\setlength{\fboxsep}{1.5em}
\framebox{\begin{minipage}{.85\textwidth}\centering
Linguistic Technology Systems \\
POC: Amy Neustein, Ph.D. \\
Founder and CEO \\
amy.neustein@verizon.net \\
201-224-5096 \vspace{1em}\\
Lead Software Architect: Nathaniel Christen
\end{minipage}}}
}
\end{frame}

\begin{frame}{\Huge{\textbf{PROFILING SUSPECTS}}}

\begin{minipage}[b][.99\textheight][t]{.47\textwidth}
\begin{mpblock}{Threat Assessment from Natural Language}
{\LARGE	
\textbf{\texttt{Language Clues can Assess Terror Suspects on Three Axes:}}
\vspace{.25em}
\begin{list}{\textborn}{}
\sfitem{Familiarity with recruiters and/or their family members and friends}
\sfitem{Interest in colluding with terrorist \\ groups}
\sfitem{Level of collaboration with terrorist \\ entities}
\end{list}
\vspace{2em}
{\hrule}
\vspace{1.75em}
\textbf{\texttt{Conversation Patterns can Reveal Imminent Threats:}}
\vspace{.25em}
\begin{list}{\textborn}{}
\sfitem{Unusual speech patterns that \\ suggest speakers using coded language}
\sfitem{Matching conversations against \\ locations / activities to pinpoint \\ recruitment campaigns}
\sfitem{Conversations in their Natural Language contexts can be matched 
	against Threat Databases and Data Models (``Threat Scenarios")}
\end{list}}
\end{mpblock}
\end{minipage}\hfill%
\begin{minipage}[b][.99\textheight][t]{.50\textwidth}
\begin{minipage}[b][.50\textheight][t]{.99\textwidth}
\begin{mpblock}{Language Data Repositories}
{\LARGE	
  \textbf{\texttt{Data Sources for Language Clues Include:}}
\begin{list}{\textcolor{brown}{$\Box$}}{}
\sfitem{Online Social Media Posts and Tweets}
\sfitem{Email Communications} 
\sfitem{Recorded Phone Calls and/or Meetings}
\sfitem{Interviews with Residents in Communities Seeing Recruitment Actions}
\sfitem{Regional databases of idioms, colloquial speech, Social Media conventions ...}
\sfitem{Government-Fabricated Recordings of Terror Leaders
 \footnote{\begin{minipage}{.95\textwidth}These recordings would be mined for flaws, resulting in an 
 improved simulation of the conversational patterns of terror leaders.\end{minipage}}} 	
\end{list} 
}
\end{mpblock}
\end{minipage}
\begin{minipage}[t][.38\textheight][b]{.99\textwidth}
\begin{mpblock}{Object Data Models}
{\LARGE	
\begin{list}{\textborn}{}
\sfitem{Match Conversations ({\normalsize Visually and Algorithmically}) to Locations, Events, and Times}
\sfitem{Synchronize Natural Language Databases with Threat Scenarios ({\normalsize Object-Oriented Models}) }
\sfitem{Project Sources \underline{from} Multiple Human Languages \underline{to} 
	Unified Semantic Models (who/what/when/where ...)}
\end{list}
}
\end{mpblock}
\end{minipage}
\end{minipage}

\end{frame}\label{}
\begin{frame}{\Huge{\textbf{The CIGQL Database Stack}}}
	
\tikzstyle{nlComponent}=[double,draw=brown!120,fill=
{blue!50,thin,inner sep=.2cm]
\tikzstyle{sdComponent}=[double,rounded corners,draw=brown!120,fill=blue!50,thin,inner sep=.2cm]
\tikzstyle{cnvComponent}=[double,shape=diamond,rounded corners,draw=brown!120,fill=blue!50,thin,inner sep=0cm]
\tikzstyle{cfStyle}=[fill=red!50,draw,shape=ellipse,inner sep=.15cm]

\TwoQuad
\begin{quadblock}{The CIGQL Query Stack}
%\raisebox{.1em}{\begin{minipage}{\textwidth}
\begin{figure}
\doubleCenteredFramebox{%
\begin{tikzpicture}

\path 
 node (nlpNode) at (0,0) [nlComponent] {Natural Language Data}
 node (hiddenArcLeft) [above=of nlpNode] {}
 node (hiddenCenter) [right=of nlpNode] {}
 node (siNode) [sdComponent, right=of hiddenCenter] {Structured Information}
 node (hiddenArcRight) [above=of siNode] {}
 node (caNode) [cnvComponent, above=of hiddenCenter] {\conversationPatterns}
 node (dataModel) [rounded corners,double,draw=darkRed,above=of caNode] {(Data Models)}
% node (hiddenArcTopLeft) [left=.1cm of hiddenArc] {}
% node (hiddenArcTopRight) [right=.1cm of hiddenArc] {}
 node (cfNode) [cfStyle, above=of dataModel] {CIGQL-F}
 node (feedbackNode) [cnvComponent, left=of cfNode,inner sep=.15cm, draw=pink!120] {Feedback}
 
 node (cnNode) [double,draw=pink!120,fill=gray!50,inner sep=.5em,above=of cfNode] {{\hspace*{.5em}}CIGQL-N\hspace*{1em}};

\draw[ |-,-|, <->, thick, double equal sign distance, >= stealth, shorten <= .25cm, shorten >= .25cm ] 
(nlpNode.east) to node 
{\raisebox{2.5em}{\begin{minipage}{4em}%
\centering{\begin{footnotesize}\color{darkRed}{Matched Object Fields}\end{footnotesize}}
\end{minipage}}} 
(siNode.west);


\draw[thick, shorten <= .25cm, shorten >= .25cm ] 
(nlpNode.north)
to node {%
\raisebox{4em}{%	
\colorbox{white}{%	
\begin{minipage}[t]{3.5em}	 
\centering{\begin{footnotesize}
\color{blGreen}{Sequence Package Analysis}
\end{footnotesize}}
\end{minipage}}}}
(caNode);

\draw[ |-,-|, ->>, very thick, >= triangle 90, shorten <= .25cm, shorten >= .25cm ]
 (cnNode.south) to node {{\manyemdash}\textit{compiles to}...} (cfNode.north);
 
\draw[ |-,-|, ->, very thick, >= latex', shorten >= .25cm ]
(cfNode.south)  to node {{\manyasciimacron}\raisebox{.5em}{\textit{queries}...}}  (dataModel.north);

\draw[->, thick, >= angle 90, shorten >= .25cm ]
 (dataModel) edge [bend right=100,looseness=1,draw=purple] (hiddenArcLeft.west);
\draw[-, thick, shorten >= .25cm ]
 (dataModel) edge [bend right=100,looseness=1.05,draw=black] (hiddenArcLeft.west);
\draw[-, thick, shorten >= .25cm ]
(dataModel) edge [bend right=100,looseness=1.1,draw=purple] (hiddenArcLeft.west);

\draw[->, thick, >= angle 90, shorten >= .25cm ]
 (dataModel) edge [bend left=100,looseness=1,draw=purple] (hiddenArcRight.east);
\draw[->, thick, >= angle 90, shorten >= .25cm ]
(dataModel) edge [bend left=100,looseness=1.05] (hiddenArcRight.east);
\draw[->, thick, >= angle 90, shorten >= .25cm ]
(dataModel) edge [bend left=100,looseness=1.1,draw=purple] (hiddenArcRight.east);

\draw[ <->>, thick, >= triangle 60, shorten <= .15cm, shorten >= 1.5cm ]
 (dataModel) to node 
 {\raisebox{8.5em}{\begin{footnotesize}\hspace*{.75em}%
\color{darkRed}{Object Oriented}\end{footnotesize}}} (siNode);
 
\draw[ ->, >= angle 90, shorten <= .15cm, draw=cyan ] 
  (caNode) to node 
{%
 \raisebox{-6em}{\begin{footnotesize}\begin{minipage}{6em}%
 \color{blGreen}{Query as \\Conversation}
 \end{minipage}\end{footnotesize}}} 
(feedbackNode);

\draw[ ->, >= angle 90, shorten >= .15cm, draw=cyan ] 
  (feedbackNode) -- (cnNode);

\draw[ |-,-|, ->, thick, >= latex', shorten <= .25cm, shorten >= .25cm ]
(caNode)
to node 
{\colorbox{white}{\begin{minipage}{5em}\begin{footnotesize}\color{blGreen}{Actionable}\end{footnotesize}
		\end{minipage}}} 
(siNode.north) [double];
 
%\draw[->, thick, >= angle 90, shorten >= .25cm ]
%  (hiddenArcRight.east)
%   edge [bend left=100,looseness=1]  (hiddenCenter);

%\draw[ <->, thick, >= latex', shorten >= .25cm ]
% (hiddenArcLeft.north) edge [bend left=-100hiddenArc, bend right=100, looseness=1] (hiddenArcRight.north);

\end{tikzpicture}}
\caption{CIGQL Stack}\label{fig:CIGQLStack}
\end{figure}	
%\end{minipage}{\textwidth}}
\end{quadblock}
}
{
	\begin{quadblock}{Key}
		\large{{\color[rgb]{0.25,0.25,0.1}{\textbf{Figure~\ref{fig:CIGQLStack}}}}
			(left) outlines major components for storing CIGQL-aware data and executing CIGQL Queries.  
			Note: \begin{itemize}
\item Components modeled as Natural Language are shown as Rectangles (with straight edges and corners)
\item Components modeled as Conversations are shown as Diamonds
\item Components modeled as Structured are shown as Rectangles (with rounded corners)
			 \end{itemize}}  
			 
\vspace{.5em}			 
\hrule		 
\vspace{.25em}			 

		\begin{list}{\colortriangle}{}
			\scitem{CIGQL-F: A Formal Query Language integrating Language, Conversation, and Structured Data}  		
			\scitem{CIGQL-N: A way to write CIGQL-F queries using Natural Language} 
			\scitem{``Query as Conversation": Database User Experience, inspired by Natural Language}
		\end{list}	
	\end{quadblock}	
}			
\end{frame}

\begin{frame}{\Huge{\textbf{CIGQL Information}}}	
{
	\begin{quadblock}{CIGQL: The Conversational Intelligence-Gathering Query Language}
		{\LARGE A high level query language that integrates \textbf{Sequence Package Analysis} 
			algorithms with \textbf{Natural Language Processing} and 
			\textbf{Structured Information} (who/what/when/where).} \\ 
		
		\vspace{1em}
		{\colortriangle 
			%\begin{minipage}{.9\textwidth}
			\centering{\LARGE Analyzing Natural Language in terms of both \textit{Conversation} 
				and \textit{Structured Data} helps 
				convert observations to actionable intelligence}
			
		} 
	\end{quadblock}
	
	\begin{quadblock}{Using CIGQL}
			
			\begin{list}{\textborn}{}
				\scitem{CIGQL-F is a Formal Query Language to find signatures of} 
				\begin{itemize}
					\embitem{Conversation patterns 
						$\rightarrow$ {\normalsize often unconscious; detected by using Sequence Package Analysis} \\%
						\makebox{\hspace{-.45em}(example $\equiv$ covert relationships between speakers)}}
					\embitem{Natural Language patterns \\\makebox{\hspace{-.45em}(}example $\equiv$ word meanings, parts of speech)} 
					\embitem{Structured Data patterns \\\makebox{\hspace{-.45em}(}annotating Object-Oriented data models to 
						connect data fields with words and phrases)}  
				\end{itemize}
				\scitem{CIGQL-F extends existing technology:} 
				\begin{itemize}
                 \embitem{Modern implementation of SBQL (the Stack-Based Query Language, designed to 
                 	query complex Application and Object-Oriented data)}
                 \embitem{Partial implementation of Functor Query Lanuage (a mathematical 
                 	framework used in Categorical Informatics)}
                 \embitem{Secure Access Layer based on the E programming language}                 
				\end{itemize}
				
				\scitem{CIGQL-N is a way to write CIGQL-F queries using Natural Language 
					(extensible to support multiple languages and alphabets).}
				\scitem{``Query as Conversation": An \nobreakdash interface \nobreakdash designed to 
					show and review query results in 
					\nobreakdash conversational ways, modeling 
					user/data interaction via \nobreakdash \textbf{Sequence Package Analysis}.}
			\end{list}	
		\end{quadblock}	
	
}


\begin{frame}{\Huge{\textbf{Costs}}}	
\begin{minipage}[b][.33\textheight][t]{.99\textwidth}
\begin{mpblock}{Research Team and Time Span}
{\LARGE	
{\color{orange}Time Span:} 18 months (with deliverables every quarter, please see below)\vspace{1em}\\
{\color{orange}Team:} 9 full-time researchers in Speech Engineering, Natural Language Processing, Conversation Analysis, Computational Linguistics, and Statistical Mathematics;\vspace{1em}\\ 
5 veterans who know the regional dialects in the areas where they served\vspace{1em}\\
{\color{orange}Costs:} 1.175 Million\\
}
\end{mpblock}
\end{minipage}
\end{frame}

\begin{frame}{Deliverables}
\begin{minipage}[b][.65\textheight][t]{.99\textwidth}
\begin{mpblock}{Six Quarters starting mid-2015}
{\large 
{\LARGE 2015 – 3rd Quarter} 
\begin{itemize}
\item Improved Government-Fabricated Voice Recordings 
\item Framework for Database Integration between \textit{Natural Language} and 
\textit{Structured Information} Layers (for example, Threat Scenarios)
\end{itemize}\vspace{.25em}
{\LARGE 2015 – 4th Quarter} 
\begin{itemize}
\item Language tool to assess familiarity of suspects with Recruiters and/or their Family Members and Friends
\item Extending the Integration Framework to interrelate \textit{Conversation Pattern Observations} with 
\textit{Actionable Structured Information} based on their shared \textit{Natural Language Semantic Context}
\end{itemize}\vspace{.25em}
{\LARGE 2016 – 1st Quarter} 
\begin{itemize}
\item Language tool to assess suspects\textsc{'} interest in colluding with terrorist groups
\item Formal Specification, Parsers, and Intermediate Representation for CIGQL-F as a formal query language  
\end{itemize}\vspace{.25em}
{\LARGE 2016 – 2nd Quarter} 
\begin{itemize}
\item Language tool to assess suspects\textsc{'} level of collaboration with terrorist organizations
\item Implementation of CIGQL-F and underlying database engines 
{\color{blue}{$\bullet$}} Prototype of CIGQL-N for writing queries in Natural Language  
\end{itemize}\vspace{.25em}
{\LARGE 2016 – 3rd Quarter} 
\begin{itemize}
\item Apply suite of language tools to new geographic regions, by modifying generic algorithms with region-specific conversational markers		
\item Full Implementation of CIGQL-N and the \textsc{``}Queries as Conversation\textsc{"} User Experience model	
\end{itemize}
}\end{mpblock}
\end{minipage}
\end{frame}

\end{document}