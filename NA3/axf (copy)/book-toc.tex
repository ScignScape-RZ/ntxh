
\documentclass{statsoc}

\usepackage[a4paper, outer=1cm, inner=1cm,top=.8cm,bottom=.8cm]{geometry}

\usepackage[T1]{fontenc}

\usepackage{tgbonum}

\usepackage{graphicx}
\usepackage[textwidth=8em,textsize=small]{todonotes}
\usepackage{amsmath}
\usepackage{natbib}

\newcommand{\p}[1]{

\vspace{.75em}#1}

\setlength{\parindent}{60pt}

\colorlet{orr}{orange!60!red}
\newcommand{\textscc}[1]{{\color{orr!35!black}{{%
						\fontfamily{Cabin-TLF}\fontseries{b}\selectfont{\textsc{\scriptsize{#1}}}}}}}
\newcommand{\AcronymText}[1]{{\textscc{#1}}}


\newcommand{\q}[1]{{\fontfamily{qcr}\selectfont ``}#1{\fontfamily{qcr}\selectfont ''}} 
\newcommand{\API}{\resizebox{!}{7pt}{\AcronymText{API}}}

\newcommand{\PACS}{\resizebox{!}{7pt}{\AcronymText{PACS}}}
\newcommand{\EMR}{\resizebox{!}{7pt}{\AcronymText{EMR}}}

\newcommand{\HCI}{\resizebox{!}{7pt}{\AcronymText{HCI}}}

\newcommand{\visavis}{vis-\`a-vis}

\usepackage{enumitem}

\setlist[itemize]{topsep=20pt,before=\leavevmode\vspace{-1.5em}}


\colorlet{dsl}{purple!20!brown}
\colorlet{dslr}{dsl!50!blue}

\setlist[description]{%
  topsep=10pt,
  labelsep=12pt,
  itemsep=12pt,               % space between items
  font=\normalfont\bfseries\color{dslr!50!black}, % if colour is needed
  style=nextline 
}

\colorlet{rgrey}{red!30!grey}


\newif\ifSummaryInText
\SummaryInTexttrue

\makeatletter
\long\def\grabsummary#1#2\end{%
  \applydraftsummary{#2}
  \end}

\long\def\applydraftsummary#1{%
\vspace{-12pt}
  \textcolor{rgrey!50!purple}{\hrule width \textwidth\kern4pt} %
  %\textbf{Summary}%
  \vspace{2pt}#1\vspace{1pt}
  %\begin{itemize}#1\end{itemize}%
  \textcolor{rgrey!50}{\hrule width \textwidth\medskip} \vspace{10pt}}%
\newenvironment{summary}[1][0]{\let\BEGIN\begin\let\END\end\grabsummary{#1}}{}%
\makeatother


\title[Data Modeling and Text Mining for Covid-19]{Advances in Data Modeling and Text Mining for Covid-19 Research}
\author[Amy Neustein]{Amy Neustein}
\author[Amy Neustein]{Nathaniel Christen}
%\address{Linguistic Technology Systems}

  % 

\begin{document}

\vspace{1em}
\noindent{}Approx 250 pages, 15 chapters\\
Manuscript Submission Date: August 31, 2020\\


%\begin{abstract}
%Abstracts are meant to give a brief flavour of the article.
%\ldots\ something here just to end the sentence.
%\end{abstract}


\section{Table of Contents}
\vspace{1em}

\begin{description}

\item[Foreword (Invited)]

\item[Authors' Introduction]


\item[Part I: Architecting Data Models for Scientific Disciplines Associated with Covid-19]

\begin{itemize}
\item Chapter 1: Data Structures for Molecular Biology and Virology

\item Chapter 2: How Genomic Data is Stored and Analyzed in the Coronavirus Context

\item Chapter 3: Structuring of Radiographic and Diagnostic Data in the Context of Covid-19

\item Chapter 4: Reviewing Epidemiological Structures and Methodology for SARS-Cov-2 Research

\item Chapter 5: Modeling Clinical Data in the Covid-19 Patient Population  

\end{itemize}

\item[Part II: Creating a Cross-Disciplinary Ecosystem for Covid-19]

\begin{itemize}

\item Chapter 6: \hspace{3pt} Approaches for Merging Heterogeneous Data Sets: Ontologies and Hypergraphs

\item Chapter 7: \hspace{3pt} Scientific Workflows and Inter-Application Networking: Reviewing data pipelines commonly used in Covid-19 research

\item Chapter 8: \hspace{0.5pt} Formal Procedural Models: Representing computational procedures applicable to Covid-19

\item Chapter 9: \hspace{2pt} Integrating Procedural and Data Models

\item Chapter 10: Type Theories for Procedural Data Modeling

\end{itemize}

\item[Part III: Text and Data Mining for Covid-19]

\begin{itemize}

\item Chapter 11: Applying Data Mining Techniques to Covid-19 Research Corpora

\item Chapter 12: Text Mining of Covid-19 Publication Archives

\item Chapter 13: Human-Computer Interaction Approaches for Covid-19 Software 

\item Chapter 14: Using Text-Mining Tools to Extract Medical History from Clinical Narratives 

\item Chapter 15: Annotating Patient Narratives for Emerging 
Covid-19 Symptomatology 


\end{itemize}

\end{description}


\end{document}


