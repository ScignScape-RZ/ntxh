
\documentclass{statsoc}

\usepackage[a4paper]{geometry}
\usepackage{graphicx}
\usepackage[textwidth=8em,textsize=small]{todonotes}
\usepackage{amsmath}
\usepackage{natbib}

\colorlet{orr}{orange!60!red}
\newcommand{\textscc}[1]{{\color{orr!35!black}{{%
						\fontfamily{Cabin-TLF}\fontseries{b}\selectfont{\textsc{\scriptsize{#1}}}}}}}
\newcommand{\AcronymText}[1]{{\textscc{#1}}}


\newcommand{\q}[1]{{\fontfamily{qcr}\selectfont ``}#1{\fontfamily{qcr}\selectfont ''}} 
\newcommand{\API}{\resizebox{!}{7pt}{\AcronymText{API}}}

\newcommand{\visavis}{vis-`a-vis}


\title[Short title]{Advances in Data Modeling and Text Mining for Covid-19 Research}
\author[Amy Neustein]{Amy Neustein}
\address{Linguistic Technology Systems,
Fort Lee, NJ.}
\email{amy.neustein@verizon.net}
%\\\vspace*{1em}
\author[Nathaniel Christen]{Nathaniel Christen}
\address{Linguistic Technology Systems,
Fort Lee, NJ.}

\begin{document}

%\begin{abstract}
%Abstracts are meant to give a brief flavour of the article.
%\ldots\ something here just to end the sentence.
%\end{abstract}


\section{Introduction}

The Covid-19 pandemic has inspired an unprecedented 
convergence of scientific research, driven in part 
by publishers choosing to allow open public access 
to many research papers and published data 
relevant to Covid-19 (the disease) and SARS-CoV-2 
(the viral agent).  The sheer volume of the data 
presents both a practical challenge --- to help 
scientists find the information most relevant to them 
--- and a valuable case-study in the challenges and 
possibilities for curating multi-disiplinary 
information spaces unified around a common 
scientific theme and research project (in this case, 
determining how best to treat Covid-19 and to 
mitigate the global SARS-CoV-2 pandemic).  

In a matter of mere weeks, the scientific and publication 
community has essentially engineered the origination of 
an entirely new data ecosystem, unified around the 
challenges of studying and predicting properties 
of Covid-19 and SARS-CoV-2 at the molecular, genomic, 
epidemiological, clinical, and public-health levels.  
In many respects, this ecosystem has emerged 
haphazardly, with the rush to publicly share 
text and data taking priority over rigorous 
curation and accuracy.  In this context, 
scientists and policymakers may benefit from a 
volume which provides a critical and analytic 
overview of the Covid-19 data ecosystem: the 
different genres of data which are marshalled 
toward scientific investigation of Covid-19 
and SARS-CoV-2; how this data is obtained, 
consumed, analyzed, and interpreted; how 
research data supports scientific claims 
pertaining to Covid-19's biological and 
epidemiological mechanisms and trajectory; 
and how these scientific claims should 
translate into public policy, taking into 
consideration both the importance of 
basing government actions on empirical data and 
the gaps in scientific knowledge which make 
such data inexact and provisional. 

Against this background, our proposed volume 
will be directed at two distinct audiences.  
At one level, we intend to examine the 
operational logistics of Covid-19 data: its 
structures, protocols, analytic methodology, 
and empirical significance.  That is, we 
intend to identify the distinct scientific 
disciplines which each concern one facet 
of Covid-19 research --- molecular biology, 
genomics, radiology, epidemiology, clinical 
informatics, and their variatious subfields 
--- and, for each of these disciplines, 
review their distinct paradigms for 
data acquisition, analysis, and modeling.  
The point of these expositions is to bring 
the reader from conceiving \q{data} as something 
abstract and amorphous, to understand 
data as the building-blocks of scientific 
research and scientific claims.  We 
want readers to have some awareness, when 
hearing or assessing claims made by either 
domain experts or public officials, the 
provenance and history of the data behind these 
claims.  One way to supply this backstory 
is to examine Covid-19 data from the viewpoint 
of software engineering --- to demonstrate 
the methodology for data acquisition and 
management from the perspective of programmers 
implementing software which manipulates Covid-19 data.  
This explication would therefore examine the 
data structures, file formats, \API{} protocols, 
and other technical details intrinsic to writing 
code which works with Covid-19 data as a 
digital artifact.  Such an exposition might be 
of interest to programmers who actually are 
writing Covid-19-related code, but the primary 
goal of these discussions will be to help 
scientists (who may be well-versed in data 
structures relevant to their specialization 
but less so \visavis{} other disciplines), 
policymakers, and the general public understand 
the technical chains which transform 
Covid-19 information from the realm of 
laboratories and experiments to the realm of 
public health and policy.

Apart from that topical focus --- the analysis 
of Covid-19 data as a concrete ecosystem 
manifest in standardized formats, global 
identifiers, and other concrete information 
artifacts --- we also intend to write for a 
more theoretical audience for whom the 
pandemic is a unique case-study in data 
curation and integration.  From this perspective, 
the Covid-19 data ecosystem is a concrete example 
through which theories of data integration can 
be presented and put to the test.

There are two distinct phenomena which render 
inter-disciplinary data integration significant for 
Covid-19.   
   
 

 



Although Open-Access research data and text provides 
useful resources for scientists and government 
officials confronting the pandemic, this material 
is often shared and curated in a relatively unstructured 
and imprecise manner.  


\section{Some \LaTeX{} Examples}
\label{sec:examples}

\subsection{How to Leave Comments}

Comments can be added to the margins of the document using the \todo{Here's a comment in the margin!} todo command, as shown in the example on the right. You can also add inline comments:

\todo[inline, color=green!40]{This is an inline comment.}

\subsection{How to Include Figures}

\subsection{How to Make Tables}

Use the table and tabular commands for basic tables --- see Table~\ref{tab:widgets}, for example.

\begin{table}
%% Caption MUST come immediately after \begin{table}
\caption{\label{tab:widgets}An example table.}
\centering
\begin{tabular}{l|r}
Item & Quantity \\\hline
Widgets & 42 \\
Gadgets & 13
\end{tabular}
\end{table}

\subsection{How to Write Mathematics}

Architecting Data Models for Scientific Disciplines Associated with Covid-19

Data structures for molecular biology and virology
How genomic data is stored and analyzed in the coronavirus context
Structuring of radiographic data in the context of Covid-19
Reviewing epidemiological structures and methodology for SARS-Cov-2 research
Modeling clinical data in the Covid-19 patient population  
 

Creating a cross-disciplinary eco f c-19

Chapter:  Approaches for merging heterogeneous data sets: Ontologies and Hypergraphs
Chapter:  Scientific Workflows and Inter-Application Networking: reviewing data pipelines commonly used in Covid-19 research: 
Chapter:  Formal Procedural Models (representing computational procedures applicable to Covid-19)
Chapter:  Integrating Procedural and Data Models
Chapter:  Type Theorys for Procedural Data Modeling


Part Three:  Software Dev meth for Covid-19 

Chapter:  Applying Data Mining techniques to Covid-19 Research Corpora
Chapter:  Text Mining of Covid-19 Publication Archives
Chapter:  HCI Approaches for Covid-19 software 
Chapter:  Software dev and testing methods for Covid-19 software 

\end{document}


