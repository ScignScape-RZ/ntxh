
\documentclass{statsoc}

\usepackage[a4paper, outer=1cm, inner=1cm,top=2cm,bottom=1.8cm]{geometry}

\usepackage[T1]{fontenc}

\usepackage{tgbonum}

\usepackage{graphicx}
\usepackage[textwidth=8em,textsize=small]{todonotes}
\usepackage{amsmath}
\usepackage{natbib}

\newcommand{\p}[1]{

\vspace{.75em}#1}

\setlength{\parindent}{0pt}

\colorlet{orr}{orange!60!red}
\newcommand{\textscc}[1]{{\color{orr!35!black}{{%
						\fontfamily{Cabin-TLF}\fontseries{b}\selectfont{\textsc{\scriptsize{#1}}}}}}}
\newcommand{\AcronymText}[1]{{\textscc{#1}}}


\newcommand{\q}[1]{{\fontfamily{qcr}\selectfont ``}#1{\fontfamily{qcr}\selectfont ''}} 
\newcommand{\API}{\resizebox{!}{7pt}{\AcronymText{API}}}

\newcommand{\PACS}{\resizebox{!}{7pt}{\AcronymText{PACS}}}
\newcommand{\EMR}{\resizebox{!}{7pt}{\AcronymText{EMR}}}

\newcommand{\EHR}{\resizebox{!}{7pt}{\AcronymText{EHR}}}
\newcommand{\sEHR}{\resizebox{!}{6.5pt}{\AcronymText{EHR}}}

\newcommand{\HCI}{\resizebox{!}{7pt}{\AcronymText{HCI}}}
\newcommand{\TDM}{\resizebox{!}{7pt}{\AcronymText{TDM}}}

\newcommand{\sHCI}{\resizebox{!}{6.5pt}{\AcronymText{HCI}}}
\newcommand{\sTDM}{\resizebox{!}{6.5pt}{\AcronymText{TDM}}}


\newcommand{\visavis}{vis-\`a-vis}

\usepackage{enumitem}
\usepackage{setspace}

\setlist[itemize]{topsep=20pt,before=\leavevmode\vspace{-1.5em}}


\colorlet{dsl}{purple!20!brown}
\colorlet{dslr}{dsl!50!blue}

\setlist[description]{%
  topsep=10pt,
  labelsep=12pt,
  itemsep=12pt,               % space between items
  font=\normalfont\bfseries\color{dslr!50!black}, % if colour is needed
  style=nextline 
}

\colorlet{rgrey}{red!30!grey}


\newif\ifSummaryInText
\SummaryInTexttrue

\makeatletter
\long\def\grabsummary#1#2\end{%
  \applydraftsummary{#2}
  \end}

\long\def\applydraftsummary#1{%
\vspace{-2pt}
  \noindent\hfil\textcolor{rgrey!50!purple}{\rule{0.5\textwidth}{.4pt}}\hfil 

  \vspace{-2pt}{%
{\fontfamily{phv}\fontsize{9}{11}\selectfont #1}
}\\\vspace{-20pt}
  {\begin{center}\textcolor{rgrey!50}{\rule{0.5\textwidth}{.4pt}}\end{center}}}

\newenvironment{summaryx}[1][0]{\let\BEGIN\begin\let\END\end\grabsummary{#1}}{}%
\makeatother


\newenvironment{summary}{\\\vspace{-4pt}%
%
%
\noindent\hfil\textcolor{rgrey!50!purple}{\rule{0.5\textwidth}{.4pt}}\hfil

\hspace{-2cm}\begin{minipage}{1.02\textwidth}\fontfamily{phv}\fontsize{9}{11}\selectfont}%
{\\\vspace{-1em}\end{minipage}
%\vspace*{-2em}
{\begin{center}\textcolor{rgrey!50!yellow}{\rule{0.5\textwidth}{.4pt}}\end{center}}
\vspace{2em}}%


\title[Cross-Disciplinary Data Integration for Covid-19]{Cross-Disciplinary Data Integration Models \\\makebox{for the New Covid-19 Data Ecosystem}}
\author[Amy Neustein]{Amy Neustein}
\author[Amy Neustein]{Nathaniel Christen}

\begin{document}

\vspace{1em}
\noindent{}Approx 250 pages: 15 chapters\\
Manuscript Submission Date: October 15, 2020\\

{\fontsize{10}{14}\selectfont
\section{PROJECT SUMMARY}

\subsection{Purpose/Summary of the Project}
\p{The Covid-19 pandemic has inspired an unprecedented 
convergence of scientific research, driven in part 
by publishers choosing to allow open public access 
to many research papers and data sets 
relevant to Covid-19 (the disease) and SARS-CoV-2 
(the viral agent).  The sheer volume of this data 
presents both a practical challenge --- how should 
scientists find the information most relevant to them 
--- and a valuable case-study in the obstacles presented 
in the curation of multi-disciplinary 
information spaces, unified around a common 
scientific theme: in this case, 
determining how best to approach Covid-19 and to 
mitigate the global SARS-CoV-2 pandemic.}

\p{In a matter of weeks, the scientific and publishing 
community has essentially engineered the origination of 
an entirely new data ecosystem, centered around the 
challenges of studying and predicting properties 
of Covid-19 and SARS-CoV-2 at the molecular, genomic, 
epidemiological, clinical, and public-health levels.  
In many respects, this ecosystem has emerged somewhat 
haphazardly: there appears to be a rush to publicly 
share text and data, which has taken priority over 
rigorous curation and accuracy.  In this context, 
we are proposing a volume for scientists, 
biostatisticians, computer 
scientists, and policymakers who may benefit from a 
volume which provides a critical and analytic 
overview of the Covid-19 data ecosystem: the 
different genres of data which are marshaled 
toward scientific investigation of Covid-19 
and SARS-CoV-2; how this data is obtained, 
consumed, analyzed, and interpreted; how 
research data supports scientific claims 
pertaining to Covid-19's biological and 
epidemiological mechanisms and trajectory; 
and how these scientific claims should 
translate into public policy.} 

%\subsection{Audience}
\p{At one level, we intend to examine the 
operational logistics of Covid-19 data: its 
structures, protocols, analytic methodology, 
and empirical significance.  That is, we 
intend to identify the distinct scientific 
disciplines, each of which is aimed at one 
particular facet of Covid-19 research --- molecular 
biology, genomics, radiology, epidemiology, clinical 
informatics, and their various subfields 
--- and, for each of these disciplines, 
review their distinct paradigms for 
data acquisition, analysis, and modeling.  
The point of these expositions is to bring 
the reader from conceiving \q{data} as something 
abstract and amorphous, to understanding 
data as the building-blocks of scientific 
research and biomedical claims.  One way to supply this backstory 
is to examine Covid-19 data from the viewpoint 
of software engineering: to demonstrate 
the methodology for data acquisition and 
management from the perspective of programmers 
implementing software which manipulates Covid-19 data.  
This explication would therefore examine the 
data structures, file formats, \API{} protocols, 
and other technical details intrinsic to writing 
code which works with Covid-19 data as a 
digital artifact.  Though such an exposition might be 
of interest to programmers who actually are 
writing Covid-19-related code, the primary 
goal of these discussions will be to help 
scientists (who may be well-versed in data 
structures relevant to their specialization 
but less so \visavis{} other disciplines), 
policymakers, and a discerning general public understand 
the technical chains which transform 
Covid-19 information from the realm of 
laboratories and experiments to the realm of 
public policy and public health.}

%\p{Our volume will be directed at two distinct audiences.  

\p{Apart from that topical focus geared toward scientists and 
policymakers, this book will also satisfy a 
more theoretical and IT-focused audience for whom the 
pandemic is a unique case-study in data 
curation and integration.  From this perspective, 
the Covid-19 data ecosystem is a concrete example 
through which theories of cross-disciplinary data integration 
can be presented.}

%\subsection{Data Integration in the Covid-19 Context}

\p{There are two distinct phenomena which render 
inter-disciplinary data integration significant for 
Covid-19 in particular, and clinical/biomedical practices 
in general.  First, certain forms of analysis explicitly 
combine information or statistical parameters from 
distinct subject areas.  For example, in addition 
to epidemiological models of SARS-Cov-2 within an 
entire population, it is important to study the 
present or projected spread of the disease among 
different social groups, identified by age, gender, 
race, economic status, and so forth.  This form 
of analysis will therefore merge epidemiological 
and sociodemographic data and methods.  As such, it is an 
instance of analyses wherein it is explicitly necessary 
to pool data that is typically represented 
via different schemas --- and accessed via different 
protocols --- into a single algorithmic or computational 
environment.  This volume will therefore 
examine cross-disciplinary analysis along these 
lines as case studies of data integration on a procedural 
level: how computer code can obtain and marshal heterogeneous 
data into a common form suitable for 
qualitative and quantitative analyses.} 
   
\p{The second context where multi-disciplinary integration 
becomes relevant operates at a higher level: the development 
of heterogeneous information spaces which can absorb 
data from many environments, evincing a variety of 
disciplinary orientations.  The rationale for such heterogeneous 
repositories is often practical and logistical: institutions 
have operational reasons for curating a single, comprehensive data ecosystem 
that is shared by multiple information producers and consumers, such as 
a \q{Semantic Data Lake.}  In these situations, one 
large central repository will take the place of numerous 
narrower, domain-specific databases.  A central repository 
may be subdivided into smaller components implementing 
narrower protocols --- e.g., a clinical software network may provide 
diagnostic images via a \PACS{} (Picture Archiving and Communication 
System) service, and treatment/outcome data via an 
\EMR{} (Electronic Medical Record) architecture.  It is understood 
that the structure and use of data in these two environments 
(\PACS{} and \EMR{}) is very different.  Nevertheless, 
institutions will often unify these systems into 
a single data platform for logistical reasons: it 
is more convenient for doctors and researchers to 
have a single access point, a single login account, 
a single query framework, etc., which accesses the 
totality of information used across the organization's 
activities.}

\p{These institutional repositories present 
challenges which are different from granular 
syntheses of heterogeneous data into a single 
procedural/algorithmic context.  Disparate data structures 
in a heterogeneous archive, such as a \q{Data Lake,} may 
never be directly combined in a single computation.  
Nevertheless, Data Lakes and their kin seek to 
provide a single software, query, and accession infrastructure 
which can presumably accommodate a diversity of data models, and 
this diversity certainly presents technological challenges.  
And in fact, Covid-19 demonstrates the problems 
engendered by these complexities in a tangible way, 
insofar as health and governmental officials have criticized 
the lack of integrated data across disciplinary and 
jurisdictional boundaries --- poor coordination between 
city, state, and federal governments in the US, for 
example, as well as between medical and governmental institutions.  
Covid-19 therefore offers a case-study in the challenges of implementing large-scale heterogeneous
data repositories; this volume will offer theoretical 
analyses and practical recommendations which
could potentially improve such technology in the future.} 

\p{The volume will accordingly be organized structurally in a 
format which progresses from domain-specific to integrative 
styles of analysis: the first part will focus on data 
models and protocols within individual disciplines, 
while the second part will discuss 
cross-disciplinary integration at both a theoretical and 
practical level.  The third part will then delve deeper 
into integrative paradigms in several areas, particularly 
text mining and software development.}  

\subsection{Coverage and Approach}

\p{As just outlined above, this volume will examine 
Covid-19 research on two levels: one that is  
empirically focused, and one that is more theoretical.  As 
such, this volume would be of interest 
to two distinct audiences.  On an empirical level, 
scientists and policymakers could benefit 
from a broad overview of Covid-19 research, 
one which uses empirical case-studies to 
illustrate how Covid-19 data is accessed and 
analyzed within the disparate disciplines that 
collectively contribute to our knowledge 
about the disease.  On a more theoretical 
level, the book would be of interest to 
computer scientists and software engineers 
who will find new theoretical models and 
type systems with which to investigate 
data-integration problems.  The new 
theories and methods explicated in this book have practical 
applications to fields such as database implementation and 
Software Language Engineering.  Such applications will be 
documented via the supporting code available 
within publicly-accessible data sets, helping 
to further the goals of the Research Object Protocol.}

\section{Table of Contents}

\begin{description}

\item[Foreword (Invited)]

\item[Authors' Introduction]

\item[Part I: Architecting Data Models for Scientific Disciplines Associated with Covid-19]

\begin{itemize}
\item Chapter 1: Data structures for molecular biology and virology

\begin{summary}
This chapter will consider data pertaining to scientists' 
investigation of SARS-Cov-2's viral mechanisms.  It will 
consider how data modeling the pathogen's proteins, 
physical structure, and interactions with human cells 
is collected and utilized.  Emphasis will be placed 
on \makebox{cheminformatic} pipelines and protocols for 
working with molecular-biological information.   
\end{summary}

\item Chapter 2: How Genomic Data is Stored and Analyzed in the Coronavirus Context

\begin{summary}
This chapter will examine the genomic data ecosystem with SARS-Cov-2 
as a case-study.  It will focus on the challenges presented 
by the large scale of genomic data, and how these challenges 
are addressed through data acquisition protocols and 
distributed software networks. 
\end{summary}

\item Chapter 3: Structuring of Radiographic and Diagnostic Data in the Context of Covid-19

\begin{summary}
This chapter will focus on information germane to identifying Covid-19 infections.  
It will be centered on diagnostic imaging, but will also consider 
the structure of data generated by Covid-19 tests or 
biometric indicators of possible SARS-Cov-2 infection.
\end{summary}

\item Chapter 4: Reviewing Epidemiological Structures and Methodology for SARS-Cov-2 Research

\begin{summary}
This chapter will consider epidemiological modeling both \textit{a posteriori}
(via empirical clinical data) and via simulations (which try to 
predict different possible trajectories for the Covid-19 pandemic).  
The chapter will review fundamental epidemiological measures such 
as infection and transmission rate, mortality rate, and R$_O$ --- 
documenting how these magnitudes are assessed both empirically and theoretically. 
\end{summary}

\item Chapter 5: Modeling Clinical Data in the Covid-19 Patient Population  

\begin{summary}
This chapter will examine medical records, in the context of 
Covid-19 as a case-study: formats for representing 
diagnostic, treatment, and outcome data in a clinical setting, 
and the construction of patient cohorts.  Focus will be 
placed on one specific technology --- the 
Clinical Looking Glass software application and Object Model --- 
as a representative example of how patient-centered 
data is structured and queried.
\end{summary}

\end{itemize}

\item[Part II: Creating a Cross-Disciplinary Ecosystem for Covid-19]

\begin{itemize}

\item Chapter 6: Approaches for Merging Heterogeneous Data Sets: Ontologies and Hypergraphs

\begin{summary}
This chapter will review the use of ontologies to model 
specific information domains and, via ontology integration, 
to construct unified knowledge systems that encompass 
multiple domains.  The chapter will present varieties 
of hypergraphs as generic data containers applicable 
to heterogeneous domains, with the goal of identifying 
sufficiently generic data structures appropriate 
for cross-disciplinary integrated ontologies.   
\end{summary}

\item Chapter 7: Scientific Workflows and Inter-Application Networking: 
Reviewing data pipelines commonly used in Covid-19 research

\begin{summary}
This chapter will focus on \API{}s, command-line interfaces, 
and related technologies through which software components 
communicate with one another.  The purpose of this review is 
to direct attention not at specific procedures which are 
implemented within a given software component, but rather 
to show how components expose functionality to the 
\q{outside world.}  This review sets the 
stage for a more internal focus (implementation 
details within a software component) in the following chapter. 
\end{summary}

\item Chapter 8: Formal Procedural Models: Representing 
computational procedures applicable to Covid-19

\begin{summary}
This chapter will concentrate more rigorously on individual 
procedures implemented within software components.  
The goal is to use external interface models (discussed 
in the previous chapter) as entry-points to analyzing 
components at a procedural level: viz., a good method for 
analyzing software functionality is to examine how 
internal procedures provide the capabilities exposed 
to an external interface.  On this basis we will 
develop a general model of procedural properties 
which adequately represents computational logic 
across different programming languages and 
software-development methodologies.   
\end{summary}

\item Chapter 9: Integrating Procedural and Data Models

\begin{summary}
This chapter will unify the discussion within the prior 
three chapters, which will have yielded, first, a 
general and cross-domain model of data structures and, 
second, a paradigm-agnostic model of procedural logic.  
The current chapter will merge both models into an 
overarching paradigm, exploiting the fact that 
procedural types and requirements are representable 
as data structures in their own right. 
\end{summary}

\item Chapter 10: Type Theories for Procedural Data Modeling

\begin{summary}
This chapter will provide a formal structure supporting 
the analysis developed in Chapter 9.  The priority 
in this chapter is codifying a hypergraph-based 
type theory, one which exploits the hypergraph context 
to concretely anchor a type system: each inhabited 
type is by definition an attribute of one or more 
hypernodes.  This starting-point, along with the 
specification of a particular genre of 
\q{channelized} hypergraphs (introduced preliminarily in 
Nathaniel Christen's chapter in Amy Neustein's 
just-published \textit{Advances in Ubiquitous Computing} 
volume), permits the construction of a general-purpose 
type theory applicable to most programming environments.  
\end{summary}

\end{itemize}

\item[Part III: Text and Data Mining for Covid-19]

\begin{itemize}

\item Chapter 11: Applying Data Mining Techniques to Covid-19 Research Corpora

\begin{summary}
This chapter will take a concrete look at research data published 
in conjunction with Covid-19 data-sharing initiatives.  
The machinery of the prior chapters will be leveraged 
to examine this existing data in a rigorous fashion, 
using formally described data types and hypergraph 
meta-models as methodological tools in surveying the 
Covid-19 data ecosystem.
\end{summary}

\item Chapter 12: Text Mining of Covid-19 Publication Archives

\begin{summary}
This chapter will examine Covid-19 corpora with a concrete 
focus that overlaps with that of Chapter 11, but with an emphasis 
on text rather than data mining.  The goal of the 
present chapter is to review publishers' efforts to 
provide open-access document corpora to support 
Covid-19 research, and to demonstrate how text 
mining tools can make these corpora more valuable.
\end{summary}

\item Chapter 13: Human-Computer Interaction Approaches for Covid-19 Software 

\begin{summary}
This chapter will apply hypergraph-based type theory 
(as developed in Part II) to \sHCI{} methodology, 
with Covid-19 Text and Data Mining (\sTDM{}) as a case-study.  
The goal of this chapter is to consider \sTDM{} software 
targeting Covid-19 corpora as concrete examples illustrating 
how \sHCI{} concerns may be analyzed through the lens 
of the data models and type systems presented earlier in 
the volume. 
\end{summary}

\item Chapter 14: Using Text-Mining Tools to Extract Medical History 
from Clinical Narratives  

\begin{summary}
This chapter will examine how text-mining technology considered 
in Chapter 12 may be applied to clinical narratives 
for the purpose of extracting relevant patient data, 
pharmacological data, and epidemiological data, to improve 
patient care.  
\end{summary}


\item Chapter 15: Annotating Patient Narratives for Emerging 
Covid-19 Symptomatology 

\begin{summary}
This chapter will introduce techniques for encoding 
rhetorical structures identified within patient 
narratives where patients describe their first-hand 
experiential symptomology associated with Covid-19.  
These subjective descriptions can be found 
on social media websites and in the patient histories 
uploaded to portals which are made part of the overall 
Electronic Health Record (\sEHR{}) system.
\end{summary}


\end{itemize}

\end{description}


\section{About the Market}

\subsection{Secondary Markets}
\p{Epidemiology; Computation Epidemiology; Biostatistics; 
Public Policy; Software Engineering; 
Quantitative Analytics/Statistics; 
Data Visualization; and Text Mining}

\subsection{Audience}
\p{Public Policymakers; Government Health Officials; 
Healthcare Administrators; Statisticians; Epidemiologists; 
and Software Engineers}

\subsection{Competition}
\p{The structure and presentation of this book is 
inspired in part by Alexandru Telea's 
\textit{Data Visualization --- Principles and Practice}.  
In particular, we incorporate the idea of providing 
sample data sets associated with individual chapters, 
and recommending software applications that readers 
may use to open and explore these data sets.  
In fact, we intend to provide customized versions of 
several applications in a downloadable package 
accompanying the text (reusing code we developed 
in conjunction with Amy Neustein's just-published 
\textit{Advances in Ubiquitous Computing}).  In 
contrast to \textit{Data Visualization}, however, 
the current volume is focused on data modeling 
rather than data visualization, and of course 
uses Covid-19 research as a case study for 
all data-modeling examples.}


%\end{spacing}
%}

\end{document}


