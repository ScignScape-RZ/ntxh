
\p{Patient-centered research in the radiological 
context focuses on improving the precision of 
diagnostic-imaging techniques and the corresponding 
clinical interventions.  The goal of contemporary 
radiology is not only to confirm a diagnosis, 
but also to extract diatnostic cues from 
medical images that suggest which course of 
treatment has the highest probability of 
favorable treatment outcomes.  A related 
goal is curating collections of diagnostic 
images so as to improve our ability to 
identify such diagnostic cues, potentially 
using Machine Learning and/or Artificial 
Intelligence applied to large-scale 
image repositories.}

\p{The goal of building \q{searchable} image repositories 
has inspired projects such as the Semantic Dicom Ontology 
(\SeDI{})\footnote{See \bhref{https://bioportal.bioontology.org/ontologies/SEDI}.} and the \ViSion{} \q{structured 
reporting} system.\footnote{See \bhref{https://epos.myesr.org/esr/viewing/
index.php?module=viewing_poster&task=&pi=155548}.}  
As explained in the context of \SeDI{}: 
\q{if a user has a CT scan, and wants to retrieve the [corresponding] radiation treatment plan ... he has to search for the 
RTSTRUCT object based on the specific CT scan, and from there 
search for the RTPLAN object based on the RTSTRUCT object.  This is an inefficient operation because all RTSTRUCT [and] RTPLAN files for the patient need to be processed to find the correct treatment 
plan.}\footnote{See \bhref{https://www.ncbi.nlm.nih.gov/pmc/articles/PMC5119276/}  Even relatively simple queries such as 
\q{display all patients with a
bronchial carcinoma bigger than 50 cm$^3$} cannot 
be processed by \PACS{} systems: \q{although there are various powerful clinical applications to process image data and image data series to create significant clinical analyses, none of these analytic results can be merged with the clinical data of a single patient.}\footnote{See 
\bhref{https://semantic-dicom.com/starting-page/}.}  These 
limitations partly reflect the logistics of how 
information is transferred between clinical institutions 
and radiology labs.  So as to advance the science 
of diagnostic image-analysis, organizations such as 
the Radiological Society of North America 
(\RSNA{}) have curated open-access data sets encompassing 
medical images as well as image-annotations (encoding feature 
vectors) which can serve as reference sets and test 
corpora for investigating analytic methods.  Such 
repositories are designed to integrate data from 
multiple hospitals and multiple laboratories.}

\p{This renewed focus on patient outcomes 
has important consequences for the scope and 
requirements of diagnostic-imaging software.  
In particular, the domain of radiological applications 
is no longer to limited to \PACS{} workstations 
where pathologists perform their diagnostic analysis, 
with the results transferred back to the referring institution 
(and subsequently available only through that institution's 
medical records, if at all).  In the 
conventional workflow, radiographic images are requested by 
some medical institution for diagnostic purposes.  
Relevant information is therefore shared between 
two end-points: the institution which prescribes 
a diagostic evaluation and the radiologist or 
laboratory which analyzes the resulting images.  
Building radiographic data repositories complicates 
this workflow because a third entity becomes 
involved --- the organization responsible 
for aggregating images and analyses is generally 
distinct from both the prescribing institution and 
the radiologists themselves.  As a result, both radiologists and prescribing 
institutions, upon participation in the formation 
of the target repository, must identify which image 
series and which patient data are proper candidates 
for the relevant repository.}

\p{For a concrete example, \RSNA{} has announced the 
forthcoming publication of an open-access image 
repository devoted to 
Covid-19.\footnote{See \bhref{https://www.rsna.org/covid-19}.}  
This repository is being curated in collaboration with 
multiple European, Asian, and South American organizations 
so as to collect data from hospitals treating 
Covid-19 patients.  Such a collaboration requires 
protocols both for data submission and for patient privacy 
and security.  As this example demonstrates, 
these kinds of data-sharing initiatives present 
new requirements for radiological software, which must 
not only allow for the presentation, annotation, and 
analysis of medical images, but also for participation 
in data-sharing initiatives adhering to rigorous 
modeling and operational protocols.}

\p{Simultaneously, the science of diagnostic imaging 
is also expanding as new image-analytic techniques 
prove to be effective at detecting signals 
within image data, often complementing the work 
of human radiologists.  The proliferation of 
image-analysis methodologies places a new emphasis 
on \textit{extensibility,} where radiological 
software becomes more powerful and flexible 
because new analytic modules may be pluggin in 
to a central \PACS{} system.  A good example of 
this new paradigm is the Cancer Imaging Phenomics 
Toolkit (\CaPTk{}), developed by the Center for Biomedical 
Image Computing and Analytics (\CBICA{}) at the 
University of Pennsylvania's Perelman School of Medicine.
The \CaPTk{} project provides a central application 
which provides a centralized User Interface and 
takes responsibility for acquiring and loading 
radiographic images.  The \CaPTk{} core application is 
then paired with multiple \q{peer} applications 
which can be launched from \CaPTk{}'s main window, 
each peer focused on implementing specific 
algorithms so as to transform and/or to extract feature vectors 
from images sent between \CaPTk{} and its plugins.}

\p{Both the patient-outcomes focus in building image 
repositories and the integration of novel 
Computer Vision algorithms depend, at their core, 
on rigorous data sharing.  Taking the 
\RSNA{} Covid-19 repository as a case study for 
promoting research into post-diagnostic outcomes, 
this repository is possible because an international 
team of hospitals and institutions have agreed to 
pool radiological data relevant to SARS-CoV-2 infection 
according to a common protocol.  Taking \CaPTk{} as a 
case-study in multi-modal image analysis, this 
system is likewise possible because analytic modules 
can be wrapped into a plugin mechanism which allows 
many different algorithms to be bundled into a common 
software platform.  Of course, these two areas 
of data-sharing overlap: one mission of repositories 
such as the \RSNA{}'s is to permit many different 
analyses to be performed on the common image 
assets.  The results of these analyses then become 
additional information which enlarges the 
repository proportionately.  If \CaPTk{} modules 
are used to analyze the \RSNA{} Covid-19 images, 
for example, there needs to be a mechanism for 
exporting the resulting data outside the \CaPTk{} 
system, so that the analyses may be integrated into 
the repository either directly or as a supplemental 
resource.}
  
\p{This example demonstrates how software such as 
\CaPTk{} may be extended to support the curation 
of image repositories dedicated to Patient Outcomes 
and Comparative Effectiveness Research (\CER{}), insofar 
as analytic data generated by \CaPTk{} components 
can asquire the capability to share data according 
to repository protocols.  A further level of 
integration between \CaPTk{} and \CER{} initiatives 
can be achieved if one observes that clinical 
outcomes may be part of the analytic parameters 
used by \CaPTk{} modules.  As presently constituted, 
\CaPTk{} analytic tools are focused on extracting 
quantitative (or quantifiable) features from 
image themselves, without considering additional 
patient-centered context.  There is no technical 
limitation, however, which would prevent the 
\CaPTk{} system from sharing more detailed clinical 
information with its modules, allowing these 
analytic components to cross-reference image features 
with clinical or patient information.  Insofar as 
the image analysis is often retroactive --- not 
entertained in the course of a present diagnosis but 
examining images from which a diagnosis has already 
been rendered --- information about treatment 
protocols and outcomes can also be shared between 
the \CaPTk{} components, assuming this information 
is provided along with images themselves in the 
context of an image repository and/or a \q{semantic} 
\DICOM{} system.}

\p{We propose, therefore, to implement enhancements to 
\CaPTk{} allowing for clinical and outcomes data 
to be shared between \CaPTk{} components, and allowing 
for \CaPTk{} modules to participate in data-sharing 
initiatives devoted to integrating image analysis 
with outcomes research.  Moreover, we believe that 
the data-sharing protocol used internally by 
\CaPTk{} can be formalized and generalized to 
serve as a prototype for integrating diagnostic 
imaging with clinical outcomes in broader contexts.  


 }


