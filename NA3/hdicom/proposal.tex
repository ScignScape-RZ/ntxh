
\p{Patient-centered research in the radiological 
context focuses on improving the precision of 
diagnostic-imaging techniques and the corresponding 
clinical interventions.  The goal of contemporary 
radiology is not only to confirm a diagnosis, 
but also to extract diatnostic cues from 
medical images that suggest which course of 
treatment has the highest probability of 
favorable treatment outcomes.  A related 
goal is curating collections of diagnostic 
images so as to improve our ability to 
identify such diagnostic cues, potentially 
using Machine Learning and/or Artificial 
Intelligence applied to large-scale 
image repositories.}

\p{The goal of building \q{searchable} image repositories 
has inspired projects such as the Semantic Dicom Ontology 
(\SeDI{})\footnote{See \bhref{https://bioportal.bioontology.org/ontologies/SEDI}.} and the \ViSion{} \q{structured 
reporting} system.\footnote{See \bhref{https://epos.myesr.org/esr/viewing/
index.php?module=viewing_poster&task=&pi=155548}.}  
As explained in the context of \SeDI{}: 
\q{if a user has a CT scan, and wants to retrieve the [corresponding] radiation treatment plan ... he has to search for the 
RTSTRUCT object based on the specific CT scan, and from there 
search for the RTPLAN object based on the RTSTRUCT object.  This is an inefficient operation because all RTSTRUCT [and] RTPLAN files for the patient need to be processed to find the correct treatment 
plan.}\footnote{See \bhref{https://www.ncbi.nlm.nih.gov/pmc/articles/PMC5119276/}  Even relatively simple queries such as 
\q{display all patients with a
bronchial carcinoma bigger than 50 cm$^3$} cannot 
be processed by \PACS{} systems: \q{although there are various powerful clinical applications to process image data and image data series to create significant clinical analyses, none of these analytic results can be merged with the clinical data of a single patient.}\footnote{See 
\bhref{https://semantic-dicom.com/starting-page/}.}  These 
limitations partly reflect the logistics of how 
information is transferred between clinical institutions 
and radiology labs.  So as to advance the science 
of diagnostic image-analysis, organizations such as 
the Radiological Society of North America 
(\RSNA{}) have curated open-access data sets encompassing 
medical images as well as image-annotations (encoding feature 
vectors) which can serve as reference sets and test 
corpora for investigating analytic methods.  Such 
repositories are designed to integrate data from 
multiple hospitals and multiple laboratories.}

\p{This renewed focus on patient outcomes 
has important consequences for the scope and 
requirements of diagnostic-imaging software.  
In particular, the domain of radiological applications 
is no longer to limited to \PACS{} workstations 
where pathologists perform their diagnostic analysis, 
with the results transferred back to the referring institution 
(and subsequently available only through that institution's 
medical records, if at all).  In the 
conventional workflow, radiographic images are requested by 
some medical institution for diagnostic purposes.  
Relevant information is therefore shared between 
two end-points: the institution which prescribes 
a diagostic evaluation and the radiologist or 
laboratory which analyzes the resulting images.  
Building radiographic data repositories complicates 
this workflow because a third entity becomes 
involved --- the organization responsible 
for aggregating images and analyses is generally 
distinct from both the prescribing institution and 
the radiologists themselves.  As a result, both radiologists and prescribing 
institutions, upon participation in the formation 
of the target repository, must identify which image 
series and which patient data are proper candidates 
for the relevant repository.}

\p{For a concrete example, \RSNA{} has announced the 
forthcoming publication of an open-access image 
repository devoted to 
Covid-19.\footnote{See \bhref{https://www.rsna.org/covid-19}.}  
This repository is being curated in collaboration with 
multiple European, Asian, and South American organizations 
so as to collect data from hospitals treating 
Covid-19 patients.  Such a collaboration requires 
protocols both for data submission and for patient privacy 
and security.  As this example demonstrates, 
these kinds of data-sharing initiatives present 
new requirements for radiological software, which must 
not only allow for the presentation, annotation, and 
analysis of medical images, but also for participation 
in data-sharing initiatives adhering to rigorous 
modeling and operational protocols.}

\p{Simultaneously, the science of diagnostic imaging 
is also expanding as new image-analytic techniques 
prove to be effective at detecting signals 
within image data, often complementing the work 
of human radiologists.  The proliferation of 
image-analysis methodologies places a new emphasis 
on \textit{extensibility,} where radiological 
software becomes more powerful and flexible 
because new analytic modules may be pluggin in 
to a central \PACS{} system.  A good example of 
this new paradigm is the Cancer Imaging Phenomics 
Toolkit (\CaPTk{}), developed by the Center for Biomedical 
Image Computing and Analytics (\CBICA{}) at the 
University of Pennsylvania's Perelman School of Medicine.
The \CaPTk{} project provides a central application 
which provides a centralized User Interface and 
takes responsibility for acquiring and loading 
radiographic images.  The \CaPTk{} core application is 
then paired with multiple \q{peer} applications 
which can be launched from \CaPTk{}'s main window, 
each peer focused on implementing specific 
algorithms so as to transform and/or to extract feature vectors 
from images sent between \CaPTk{} and its plugins.}

\p{Both the patient-outcomes focus in building image 
repositories and the integration of novel 
Computer Vision algorithms depend, at their core, 
on rigorous data sharing.  Taking the 
\RSNA{} Covid-19 repository as a case study for 
promoting research into post-diagnostic outcomes, 
this repository is possible because an international 
team of hospitals and institutions have agreed to 
pool radiological data relevant to SARS-CoV-2 infection 
according to a common protocol.  Taking \CaPTk{} as a 
case-study in multi-modal image analysis, this 
system is likewise possible because analytic modules 
can be wrapped into a plugin mechanism which allows 
many different algorithms to be bundled into a common 
software platform.  Of course, these two areas 
of data-sharing overlap: one mission of repositories 
such as the \RSNA{}'s is to permit many different 
analyses to be performed on the common image 
assets.  The results of these analyses then become 
additional information which enlarges the 
repository proportionately.  If \CaPTk{} modules 
are used to analyze the \RSNA{} Covid-19 images, 
for example, there needs to be a mechanism for 
exporting the resulting data outside the \CaPTk{} 
system, so that the analyses may be integrated into 
the repository either directly or as a supplemental 
resource.}
  
\p{This example demonstrates how software such as 
\CaPTk{} may be extended to support the curation 
of image repositories dedicated to Patient Outcomes 
and Comparative Effectiveness Research (\CER{}), insofar 
as analytic data generated by \CaPTk{} components 
can asquire the capability to share data according 
to repository protocols.  A further level of 
integration between \CaPTk{} and \CER{} initiatives 
can be achieved if one observes that clinical 
outcomes may be part of the analytic parameters 
used by \CaPTk{} modules.  As presently constituted, 
\CaPTk{} analytic tools are focused on extracting 
quantitative (or quantifiable) features from 
image themselves, without considering additional 
patient-centered context.  There is no technical 
limitation, however, which would prevent the 
\CaPTk{} system from sharing more detailed clinical 
information with its modules, allowing these 
analytic components to cross-reference image features 
with clinical or patient information.  Insofar as 
the image analysis is often retroactive --- not 
entertained in the course of a present diagnosis but 
examining images from which a diagnosis has already 
been rendered --- information about treatment 
protocols and outcomes can also be shared between 
the \CaPTk{} components, assuming this information 
is provided along with images themselves in the 
context of an image repository and/or a \q{semantic} 
\DICOM{} system.}

\p{We propose, therefore, to implement enhancements to 
\CaPTk{} allowing for clinical and outcomes data 
to be shared between \CaPTk{} components, and allowing 
for \CaPTk{} modules to participate in data-sharing 
initiatives devoted to integrating image analysis 
with outcomes research.  Moreover, we believe that 
the data-sharing protocol used internally by 
\CaPTk{} can be formalized and generalized to 
serve as a prototype for integrating diagnostic 
imaging with clinical outcomes in broader contexts.  
The protocols adopted by \CaPTk{} reflect how the 
\CaPTk{} system is designed: each \CaPTk{} 
component is a semi-autonomous software application 
providing specific analytic capabilities.  
By default, each \CaPTk{} component is a \textit{stand-alone} 
and \textit{native} application which is operationally 
independent of \CaPTk{}, apart from receiving image data 
from the main \CaPTk{} application.  Integration 
between \CaPTk{} and its peer applications therefore 
operates on several different levels, including 
that of building and compiling the applications 
alongside \CaPTk{} and registering the modules with 
the central system, so that users can launch the 
peer application while running the main \CaPTk{} program.  
A variation of these protocols applies to 
components deployed by some means other than 
source-code level inclusion in a \CaPTk{} system; 
e.g. as pre-built binaries or as Docker 
images.\footnote{Our proposed enhancements will allow 
an additional form of plugin using the ReproZip 
framework; see \bhref{https://www.reprozip.org/}.}  
In each of these cases, a rigorous formalization 
of the \CaPTk{} protocol has to consider 
compiler, executable, and workflow-related 
integration between the components, not only 
shared data formats or Common Vocabularies.}

\p{Architecturally, this pattern of organization is 
analysis to the collection of software components 
that may share access to a data repository or 
a research-data corpus.  The purpose of 
research data archives --- particularly when 
they embrace contemporary open-access standards 
such as \FAIR{} (Findable, Accessible, Interoperable, 
Reusable\footnote{See \bhref{https://www.researchgate.net/publication/331775411_FAIRness_in_Biomedical_Data_Discovery}.}) 
and the Research Object 
Protocol\footnote{See \bhref{https://pages.semanticscholar.org/coronavirus-research}} --- is to promote reuse and reproduction of 
published data and findings, such that multiple subsequent research 
projects may be based on data originally published 
to accompany one book or article.  As a result, it is 
expected that numerous projects may overlap in their 
use of a common underlying data set, which potentially 
means a diversity of software components implementing 
a diversity of analytic techniques, each offering a unique 
perspective on the underlying data.  Open-access data 
publishing, as a result, is conducive to a software 
engineering paradigm that favors the implementation 
of autonomous software agents which, 
their autonomy notwithstanding, can interoperate to the degree 
that they are collectively oriented to a shared data source.  
Such a multi-application network requires integration 
not only at the level of data structures, but also 
at the level of Human-Computer Interaction and 
inter-application messaging --- in the optimal 
case users can switch between applications 
based on each components' respective capabilities.  
In sum, the \CaPTk{} architecture --- consisting 
of numerous autonomous, stand-alone, native 
applications federated into a decentralized 
but unified --- is similar logistically 
to the kinds of application networks 
appropriate for the technology supporting 
archives of research date (including 
diagnostic-imaging repositories).}

\p{Given this architectural correlation, we propose 
that the \CaPTk{} architecture can be used 
as a prototype for implementing multi-application 
networks such as those applicable to 
research data repositories.  As a starting point 
for modeling such multi-application networks, we 
propose to concentrate on software which is 
implementationally similar to \CaPTk{} within 
the medical-imaging domain --- for example, 
\medInria{} (a general-purpose radiology 
platform) and \ThreeDimViewer{} (a tool which generates 
\ThreeD{} tissue models from \TwoD{} image series) 
are both open-source \Cpp{} applications built 
on top of the \Qt{} application development 
platform, so they occupy essentially the 
same niche in the software engineering 
ecosystem as \CaPTk{}.  Our proposed \CaPTk{}  
extensions therefore apply to these related 
projects, insofar as we can implement plugins 
extending all three applications with a more 
rigorous data-integration protocol.  Each 
of these applications likewise lends distinct 
capabilities to a multi-application radiological 
platform --- whereas \CaPTk{} is focused on 
image analysis and \AI{}, \medInria{} functions 
more as a conventional \PACS{} workstation for 
manual image-annotation and diagnostic reporting, 
whereas \ThreeDimViewer{} is specifically 
focused on three-dimensional tissue modeling and 
reconstruction.  A robust data-sharing protocol 
would ensure that each of these applications can 
interoperate, allowing users the choice to 
switch which program they are using depending on 
the specific analytic tasks they wish to take on 
for a given session.}

\p{Along with these radiology-specific applications, 
the application network we are hereby describing 
can moreover include other components commonly 
used by researchers in a radiological context: 
software such as \XPDF{} (a \PDF{} viewer), 
\IQmol{} (a molecular-visualization program), 
\ParaView{} (a data-visualization platform), 
\Octave{} (an open-source Matlab alternative), 
and \Qt{} Creator (a \Cpp{} Integrated Development 
Environment) are all built with the same 
\Qt{}/\Cpp{} foundation as the three radiology 
tools mentioned above, and each of these 
applications provide capabilities sometimes needed 
for curating and conducting medical-imaging research.  
Therefore, by extending each of these applications 
with a common data-sharing and operational-integration 
protocol, we can provide scientists with a flexible 
research platform which synthesizes the capabilities 
of many popular scientific-computing applications. 
As an initial framework in which to implement this 
platform, we propose to focus on software 
tools for accessing Covid-19 research data included 
in the \RSNA{} repository, as well as other 
Covid-19 archives, such as the \Cnineteen{} 
collection of academic publications concerning 
Covid-19 and SARS-CoV-2.\footnote{See https://pages.semanticscholar.org/coronavirus-research}.}}

\p{Initiatives such as Research Objects and 
\FAIR{} advocate for a technological infrastructure 
characterized by a diverse software ecosystem 
paired with a open-access research data sets.  
Scientific data repositories, linked to 
academic publishing platforms, have been engineered 
to help scientists locate and re-use data sets 
which are relevant to their research projects.  
Although formats such as Research Objects have been 
standardized over the last decade, there has not 
been a comparable level of attention given to 
formalizing how mutiple software applications 
should interoperate when manipulating 
overlapping data.  The Common Workflow Language 
(\CWL{}), which has been explicitly included in the 
Research Object 
model\footnote{see \bhref{http://www.researchobject.org/scopes/}},
documents one layer of inter-application messaging, 
including the encoding of parameters via command-line 
arguments; \CaPTk{} provides a \Cpp{} implementation of 
\CWL{} and uses it to pass initial data between 
modules.  Serializing larger-scale data structures 
is of course a generic task of canonical encoding 
formats such as \JSON{}, \XML{}, \RDF{}, and 
Protocol Buffers --- not to mention text or binary 
resources serialized directly from runtime objects 
via, for instance, QTextStream and QDataStream.  
This means that some level of inter-application 
communications is enabled via \CWL{}, and 
that essentially any computationally tractable 
data structure can be encoded via formats such 
as \XML{}.  These solutions, however, are 
sub-optimal: \XML{} (as well as \JSON{} and analogous 
formats) is limited because it takes additional 
development effort to compose the code that marshals 
data between runtime and serial formats.  
Similarly, although \CWL{} can model information 
passed between applications, it provides 
only an indirect guide to programmers implementing 
each applications \q{operational semantics} --- 
viz., the procedures which must be executed 
before and after the event wherein data is 
actually passed between endpoints.}

\p{In the context of \CaPTk{}, for example, 
integrating peer modules with the \CaPTk{} core 
application involves more than simply 
ensuring that these endpoints communicate 
via a standardized data-serialization format: 
the plugins must be \textit{registered} 
with the core application, which affects the 
core in several areas, including the build/compile 
process and construction of the main \GUI{} 
window.  Modeling the interconnections between 
semi-autonomous modules, as \CaPTk{} demonstrates, 
therefore require more detail than simply 
modeling their shared data encodings; it 
is furthermore necessary to represent all 
procedural and User Interface requirements 
in each component that may be affected by 
the others.  Despite the standardization 
efforts that have been invested in 
Research Objects and the Common Workflow Language, 
we contend that this fully detailed 
protocol for multi-application interop has not 
yet been rigorously formalized.  As part of 
our project for extending \CaPTk{} and then 
generalizing this extension to scientific 
research platforms (not necessarily restricted 
to radiology), we propose to ameliorate 
this situation by implementing a more rigorous 
protocol for multi-application networking, 
which we will call the 
\q{Hypergraph Multi-Application Configuration Language} 
(\HMCL{}).  This language is paired with a new 
data-exchange protocol that we are also implementing, 
the Hypergraph Exchange Format (\HGXF{}).}

\p{}

\p{}


