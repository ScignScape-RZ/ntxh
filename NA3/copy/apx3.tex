\section{Image Analysis and GIS Imaging in Vaccine and Immunology Research}
\p{As with radiological outcomes and Clinical Effectiveness 
Research, vaccine ontologies are multi-modal because 
the scope of vaccine research spans numerous scientific 
disciplines, including fields where quantitative or 
image-based analysis is methodologically significant.  
As such, data generated during clinical trials is only one 
part of our overall understanding of the 
immunological properties of different vaccines.  
Vaccine data can be partitioned into information generated 
\textit{before} a vaccine is proven effective (that is, 
data generated during clinical trials, including 
challenge trials) and information 
generated \textit{after} a vaccine is adopted for immunization 
campaigns.  Data in the clinical trial phase tends to fit 
the structure of patient cohorts, demographics, and outcomes, 
whereas data in the vaccine \q{deployment} phase can be more 
epidemiological/sociological by virtue of post-vaccination 
analysis identifying the effects on the population 
writ large.  In particular, 
modeling immunization drives often depends on gathering 
geographical and political information --- that is, researchers 
need to identify localities where vaccinations are performed 
and to be able to model the governmental/organizational initiatives 
which support immunization.}

\p{Aside from these geographic, epidemiological, and clinical 
data profiles, vaccine research also addresses the 
immunological mechanisms which determine how a particular 
vaccine works biologically, as well as the level of 
immunity which the vaccine provides for individual 
patients.  Understanding these phenomena requires 
observing the immunological response of patients upon 
their receiving the vaccine, which generally involves blood and 
serum analysis --- using a variety of modalities, including 
chemical, genomic, and molecular imaging.  
Consequently, a general-purpose vaccine and 
immunization software suite needs  
the capacity to represent a broad spectrum of 
data which may be presented as evidence of a 
vaccine's immunological effectiveness.  
Biomedical imaging at the cellular and molecular 
level has become an increasingly important part of 
vaccine research, insofar as new imaging techniques 
allow scientists to directly observe immunological 
responses at the scale of antigens, adjuvants, and 
immune cells:  
\q{monitoring vaccine components, such as antigen 
or adjuvants, and immune cell dynamics at the site of
vaccination or draining lymph nodes can provide important
information to understand more about the vaccine response.} 
As such, \q{a variety of imaging modalities including bioluminescence
imaging, nuclear medicine imaging (such as positron 
emission tomography [PET], single photon emission computed
tomography [SPECT]), and magnetic resonance imaging (MRI)
can provide in vivo non-invasive imaging for visualizing 
immune cell kinetics} 
(see \bhref{https://pubmed.ncbi.nlm.nih.gov/31406689/}).}

\p{The evolution of vaccine research is, in effect, an inversion 
of that of diagnostic imaging.  That is, in the context of radiology, 
image-based technology have become established in previous 
decades and have only quite recently been augmented with 
data-integration models that allow images to be cross-referenced 
with clinical and outcomes data.  Conversely, in the vaccine 
context, the \textit{clinical} and epidemiological/sociodemographic 
dimensions of vaccine effectiveness have been modeled 
for decades, but only recently has sophisticated 
medical imaging become part of the vaccine engineering 
arsenal.  Despite these inverted paths, 
one can observe that contemporary radiological and vaccine research 
have a similar technological infrastructure, characterized 
by a digital synthesis of graphical/quantitative and 
clinical information.}

\p{Given the interdisciplinary nature of 
vaccine research, software engineers developing tools 
for curating data sets involving vaccine research 
should anticipate a similar diversity in the 
information present within repositories which 
include vaccine-related research.  That is 
to say, data-set software relevant in the 
vaccinology context should be able to read 
data in \textit{diverse formats,} which represent the 
scope of vaccine investigations (clinical trials, 
cytometry, blood/serum analysis, bioimaging, etc.).  
Similarly, developers should provide a suite of 
software tools offering interactive visualization 
of this data spectrum, including imaging tools.  
The tools for \CRtwo{} will therefore 
include components to help developers 
create software supporting vaccine/immunology 
research in several distinct areas, two 
examples of which are reviewed here in 
greater detail.}

\subsection{A Pure-C++ Library for Cytometry}

\p{The \CRtwo{} code libraries will include 
components to manipulate file and data 
types associated with cellular, biomolecular, 
and immunological imaging.  Examination 
of cellular-scale processes has several 
different roles in vaccine/immunization 
research, including identification of 
viral proteins and observing the immune 
system response to viral infection and/or 
vaccination.  Moreover, sudies of 
cellular processes can occur either by 
directly visualizing phenomena on this 
scale or by indirect analytic methods, 
such as cytometry.  In \q{flow cytometry}, 
for instance, a stream of cells is 
investigated by passing them through 
laser beams where the cells' material 
absords and/or scatters the light waves, 
yielding patterns of light intensity 
along two axes --- one parallel to the 
laser source, and one at right angles to 
the source which detects light scattering.  
These patterns, in turn, can be analyzed 
to ascertain properties of cells, proteins, 
and other molecular entities in the sample 
source.  The actual images generated 
in this process are not literal photograph 
or microscope-enhanced pictures, but rather 
are composite representations of light 
intensity formed by plotting 
parallel and scattered light patterns 
on a two-dimensional axis (effectively, a
high-density scatter plot).  While 
these representations are technically 
diagrams rather than images, certain 
image-analysis concepts have analogs 
in this context; for instance, \q{gating} 
--- selecting image regions of interest 
(e.g. as measures of cell count for different 
varieties of cells) --- is analogous 
to image segmentation.}

\p{Flow cytometry data is usually presented 
via \FCS{} (Flow Cytometry Standard) files.  
The most comprehensive package for working with 
the \FCS{} format is based on \R{}, although 
one \Cpp{} library (developed by the 
Fred Hutchinson institute) has been isolated from 
the surrounding \R{} package into a library 
that can be compiled standalone.  Outside the 
\R{} context, however, this library has limited 
functionality for applying predefined gating 
schemes or image visualization; one goal 
of \CRtwo{}, then, is to enhance the 
relevant \Cpp{} package with added tools 
allowing the code to be used in application 
contexts other than \R{}.}

\p{Meanwhile, other biomolecular visualization 
technology --- including \q{image cytometry,} 
an older but still useful alternative 
to flow cytometry --- require more conventional 
image-analysis techniques, so that \CaPTk{}-style 
image-analysis pipelines are appropriate in 
these contexts.  Different imaging modalities, 
including bioluminescence, \PET{} scans, and 
\MRI{}s, have been used for immunological 
cellular-scale imaging.\footnote{See 
\bhref{https://www.ncbi.nlm.nih.gov/pmc/articles/PMC6689497/}.}
A multi-disciplinary toolkit for immunology-related 
image processing should include the framework 
for incorporating Machine Vision algorithms specific 
to image processing in the context of 
immune-cell visualization as well as indirect 
cellular analysis enabled by methods 
such as cell cytometry.  The 
\CRtwo{} code will try to move in that 
direction by pairing cytometric libraries 
with an overarching image-analysis context.}


\subsection{Vaccine Data and Geographic Information Systems}

\p{Another area where vaccination data intersects with 
image-processing is that of using 
Geographic Information Systems (\GIS{}) to document 
vaccination campaigns.  Geospatial mapping 
has been used to support vaccination logistics 
in several ways.  The most straightforward 
is to map where vaccines have been 
administered, though some projects described 
by groups such as \GAVI{} (the Global Alliance for 
Vaccines and Immunization) have used mapping 
technology to represent the location of clinics, 
and to visualize demographic or jurisdictional data 
that factors in to advance planning for vaccination 
campaigns.}

\p{Geospatial mapping typically starts with 
an underlying image --- specifically a 
geographic map, which may be constructed by 
software, derived from satellite or 
street-view photography, or some combination 
--- and then superimpose on the map-image 
a data structure identifying geospatial 
points or regions of interest.  Numerous 
software packages exist to construct such 
annotated maps, such as (in the \Qt{} 
context) \ArcGIS{} and \QGIS{}.  In general, 
using these libraries involves writing plugins 
to import domain-specific data which includes 
geotagging, along with relevant details for 
each location.  These plugins therefore 
provide data structures that supply 
\GIS{} coordinates for sites of interest, 
along with structured data about the 
significance of each site; when 
the map is visualized, this supplemental 
data is converted to markings or text 
viewed as superimposed on the map image, 
providing helpful information to those 
studying the map (either for navigational 
purposes or as a visualization of 
geospatial data).}  

\p{In the vaccination context, geotagged 
annotations might involve color overlays 
documenting how extensively a vaccination 
campaigned covered each locale within its 
target area; or they might identify 
fixed locations, such as schools or 
clinics where a vaccine was administered.  
Several such maps have been developed 
for campaigns in particular countries, or 
other geographic areas, but there does 
not appear to be a general-purpose \Cpp{} 
library for modeling and then geotagging 
vaccination data so that the resulting 
data collection can be used to populate a 
\GIS{} plugin.  Therefore, the 
vaccination/immunology toolkit described 
in the previous subsection, providign code 
for integrating immulogical and image-processing 
data, will also include the framework 
for vaccination-related \GIS{} plugins.}


%, providing analogous 
%models of application data as well as inter-application 
%networking protocols.  }

%As such, software published 
%as part of the \CRtwo{} and \AIMConc{} will then use these 
%ontologies to guide the operations of 
%vaccinology and radiology research tools.}


