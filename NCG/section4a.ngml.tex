\vspace{-.1em}
\subsection{Dependent Types and Co-Constructors}
\p{To see why the metaconstructor problem 
determines how extensively Dependent Types are 
supported in a type system, consider a variation 
on the range \ZeroToOneHundred{} type.  
In lieu of a fixed range, consider a procedure 
taking a (variable) \Tvar{}-range \rRan{} and a number \xSym{}
\i{which must be in that range}.  Here \xSym{}
\q{depends} on \rRan{} \mdash{} its \i{type} is \rRan{}
seen as its own \TVOneToVTwo{} type \mdash{} so 
\xSym{} can vary among many 
range-types, only being fixed at runtime.  Defining a 
\i{type} for procedures meeting those 
\i{specifications} is a classic problem of 
Dependent Type theory.
}
\p{Using the \rRan{}-type as before, the type of \fFun{}'s second parameter
would then be \Tvar{} restricted to the \rRan{} interval, but here
\rRan{} is not fixed in \fFun{}'s declaration but rather passed in to
\fFun{} as a parameter.  Unless we know \i{a priori} that only
a specific set of \rRan{}s in the first parameter will ever be encountered,
the compiler has to be prepared for \xSym{} being assigned any one 
of many different range types, depending on the \fFun{}'s first argument.  
In particular, the compiler cannot know ahead of time which 
constructor to call for \xSym{}.  More precisely, it is impossible 
for the compiler to have \i{separate} constructors for millions 
of possible range types.  Instead, the compiler must either 
\q{create} a constructor \q{on the fly} or else have 
some generic constructor which services many range-types, 
but then requires extra information to establish 
\i{which} range is desired. 
}
\p{Assuming we use co-constructors to wrap constructors, 
these two options for compiler writers correspond to 
the choice of \i{either} creating ad-hoc co-constructors 
\i{or} designing co-constructors as a 
compound data structure.  We could certainly write a function that takes a range and a value and
ensures that the value fits the range \mdash{} perhaps by throwing an
exception if not, or mapping the value to the range's closest point.
Such a function would provide common functionality for a family of
constructors each associated with a given range.  But a function (\cfFun{}, say)
providing \q{common functionality} for value constructors is not necessarily
itself a value constructor.\footnote{Here I say \q{value constructor} to clarify that I am not 
commenting on \i{type constructors}, which derive specialized 
types from generic ones.
}
To treat such a function as a
\i{real} value constructor we would have to add contextual modifiers:
\cfFun{} is a value constructor for range-type \rRan{} in the 
presence of a \Tvar{}-pair to specify \rRan{} at runtime.
The co-constructor for a range type \TrRan{} is accordingly the
\q{common functionality} base function \i{plus} \Tvar{}'s 
passed to it \mdash{} some sort of \Cfr{} compound data structure,
again by analogy to \inc{} and
\addOne{} (see footnote \ref{fofgplausible}, above).  Here again, though, 
the co-constructor is a temporary data structure, 
created on-the-fly to model the desired value constructor 
for an \xSym{} whose type (and therefore whose constructor) 
is not known until runtime.  I contend, on examples like 
these, that Dependant Typing for a type system \TyS{} is thus logically 
equivalent to the possibility of \TyS{} co-constructors 
being temporary values.   
}
\p{But value constructors (and by extension co-constructors) 
are not just any function-value: they have a privileged
status \visavis{} types, and may be invoked whenever an appropriately-typed
value is used.  Many constructors are called behind-the-scenes: 
in \Cpp{}, the standard function-call mechanism is
\q{pass by value}, wherein values are \i{copied} when passed 
between procedures; but any copy can potentially 
invoke a so-called \q{copy constructor}.  Indeed, programmers 
use certain constructors as \q{hooks} to silently 
insert logic into normal program flow (usually this is 
to make complex types behave like built-in-types from 
client code's point of view).  Allowing large type families (like one type
for each \int{} or each two-number range \rRan{} \mdash{} similar to \q{inductive typing} as
discussed by Edwin Brady in the context of the Idris language
\cite[p. 14]{EdwinBradyImpl}) \mdash{} could easily conflict with 
user-defined constructor overrides: users (meaning, in this context, 
library developers) would need not only to write their own 
(e.g., copy) constructors, but to hook into a complex 
run-time mechanism for creating constructors ad-hoc as temporary values.
Conversely, forcing co-constructors to be
addressable prohibits \q{large} type families \mdash{} like types indexed
over other (non-enumerative) types
(see e.g. \cite[p. 4]{BernardyEtAl}) \mdash{} at least as \i{actual} types.
This apparently precludes full-fledged Dependent Types, since
dependent-typed values invariably require in general some extra
contextual data \mdash{} not just a function-pointer \mdash{} to designate the
desired value constructor at the point where a value,
attributed to the relevant dependent type,
is needed.  It may be infeasible to add the requisite contextual
information at every point where a dependent-typed value has to be constructed
\mdash{} unless, perhaps, a description of the context can be packaged and
carried around with the value, sharing the value's lifetime.
}
\p{As I will now review, this analysis in the realm of 
Dependent Types carries over into \i{typestate}, 
which is another mechanism intended to model 
coding requirements via type-checkable specifications.
}
