\section{Case Studies}
\phantomsection\label{sTwo}
\p{To motivate the themes I will emphasize going forward, 
this section will examine some concrete data models 
which are used or proposed in various Cyber-Physical contexts. 
I hope this discussion will lay out parameters on device 
behavior or shared data to illustrate typical 
modeling patterns and their corresponding safety or 
validation requirements.  As an initial overview, the 
following are some examples of data profiles that might 
be wedded to deployed Cyber-Physical devices 
(my comments here are also summarized in 
Table~\ref{table:profiles} on page \pageref{table:profiles}): 
\begin{description}\item[Heart-Rate Monitor]  A heart-rate sensor generates 
continuously-sampled integer values
whose understood Dimension of Measurement is in \q{beats per minute}
and whose maximum sensible range (inclusive of both
rest and exercise) corresponds roughly
to the \ftytwoh{} interval.  Interpreting heart-rate data depends on 
whether the person is resting or exercising.  Therefore, a 
usable data structure might join a beats-per-minute dimension 
with a field indicating (or measuring) exertion, either a 
two-valued discrimination between \q{rest} and \q{exercise} 
or a more granular sampling of a person's movement cotemporous 
with the heart-rate calculations.
\item[Accelerometers]  These devices measure object's 
or people's rate of 
movement (see \cite{MarcoAltini}, \cite{VincasBenevicius},
\cite{JamesKnight}, \cite{HyunwooLee}, etc.),
and therefore can be paired with heart-rate sensors 
to quantify how heart rate is affected by exercise 
(likewise for other biometric instruments, such as those 
calculating respiration rate).  Outside the biomedical context, 
accelerometers are important for 
Smart Cities (or factories, and so forth) for modeling 
the integrity of buildings, bridges, and industrial 
areas or structures (see e.g. \cite{Wetherington}, 
\cite{LiZhu}).  
\pseudoIndent{}
An accelerometer
presents data as voltage changes in two or three directional
axes, data which may only produce signals when a change occurs
(and therefore is not continuously varying), and which is
mathematically converted to yield information about physical
objects' (including a person's) movement and incline.  
Mechanically, that is, accelerometers actually 
measure \i{voltage}, from which quantitative reports 
of movement and incline can be derived.  Accelerometers are 
classified as \i{biaxial} or \i{triaxial} depending on 
whether they sample forces in two or three spatial 
dimensions.  
\pseudoIndent{}
The pairwise combination of heart-rate and acceleration data
(common in wearable devices) is then a mixture of these
two measurement profiles \mdash{} partly continuous and
partly discrete sampling, with variegated axes and
inter-dimensional relationships.  
\item[Remote Medical Diagnosis]  An emerging application of 
Cyber-Physical technology involves medical equipment 
deployed outside conventional clinical settings 
\mdash{} in remote areas with little electricity, refugee 
camps, temporary ad-hoc medical units (established 
to contain potential epidemics, for instance), and 
so forth.  These settings have limited diagnostic 
capabilities, so data is often transmitted to distant 
locations in lieu of on-site laboratories.  
\pseudoIndent{}
A good case-study derives from \q{medical whole slide imaging} 
(\mWSI{}) \cite{Auguste}, where a mobile 
phone attached to an ordinary microscope, 
by subtle modifications of camera position and microscope 
resolution, allows many views to be made on one slide.  
Positional data (the configuration of the phone and microscope) 
then merges with image segmentation 
computations characteristic of 
conventional whole slide imaging (see, e.g., 
\cite{Farahani}); and by extension diagnostic pathology 
in general, which 
is concerned with isolating medically significant image 
features and identifying diagnostically indicative 
anomalies (such as cell shapes 
suggesting cancer). 
\pseudoIndent{}
Segmentation, in turn, 
generates multiple forms of geometric data: in 
\cite{KaleAksoy}, for instance, segments are 
identified as approximations to ellipse shapes, 
and features are tracked across scales of resolution, 
so geometric data merges ellipse dimensions with 
positional data (in the image) and a metric 
of feature persistence across scales.  
(Features which are detectable at many scales of 
resolution are more likely to be empirically 
significant rather than visual \q{noise}; calculating 
cross-scale \q{persistence} is an applied 
methodology within Statistical Topology \mdash{} 
see e.g. \cite{EdelsbrunnerHarer}, 
\cite{HaneyMaxwell}, \cite{HarryStrange}).  
Merged with \mWSI{} configuration info and patient data, the 
whole data package integrates geometric, Cyber-Physical, 
and health-record aspects. 
\item[Speech Sampling]  Audio sensors can be used to 
isolate different people's speech episodes (see Alluri
and Vuppala, this volume; and Vuddagiri \textit{et. al.}, this volume).  Feature
extraction cancels background noise and partitions the foreground 
audio into different segments, individuated (potentially) by 
differences between speakers as well as each 
speaker's separated conversation turns.  Such data can 
then be employed in several ways.
The two chapters just mentioned present methodology for
estimating speakers' emotional states and
identifying samples' spoken language or dialect,
respectively; while the chapter by Teixera \textit{et. al.}
(which I quoted earlier) discusses
speech-activated User Interfaces.
\pseudoIndent{}
The data profile germane to 
an audio processor will be determined by the system's 
overarching goals.  For example, \cite{JongyoonChoi} describes 
techniques for measuring emotional stress via heart-rate signals.  
Combined with speech-derived data, a system might accordingly 
be designed around emotional profiles, merging linguistic and 
biometric evidence.  For those 
use-cases, programming would emphasize 
signs of emotional changes (reinforced by both metrics), 
and secondarily isolating times and locations, 
which factor into proper software responses to
users' moods.  
\pseudoIndent{}
On the other hand, a voice-based 
User Interface might similarly model speakers' identity 
and location, but perform Natural Language Processing 
to translate speech patterns into models of user 
requests.  Or, the use-case in Vuddagiri 
\textit{et. al.} in this volume profiles speech data 
for language classification (viz., matching voices to the 
language or dialect spoken) as part of a \q{smart city} network.  
The priority here is not necessarily identifying individual 
speakers, or mapping vocalizations to semantic or syntactic 
parses (although dialect-identification is obviously a first
step for those capabilities); even without further Natural Language Processing, 
tagged samples can yield a 
geospatial model of language-use in a given area. 
\item[Bioacoustic Sampling]  Similar to speech sampling 
(at least up to the point where acoustical analysis 
gives way to syntax and semantics), 
audio samples can be used to track and identify 
species (Ganchev, this volume; and Boulmaiz, \textit{et. al.},
this volume).  Here again feature 
extraction foregrounds certain noise patterns, but the 
main analytic objective is to map audio samples to 
the species of the animal that produced them.  
Sensor networks can then build a geospatial/temporal 
model of species' distribution in the area covered by 
the network: which species are identified, their prevalence, 
their concentration in different smaller areas, and so forth.  
These measurements can be employed in the study of 
species populations and behavioral patterns, and can 
also add data to urban-planning or ecological models.  
For example, precipitous decline of a species in some 
location can signal environmental degradation in that vicinity.   
\pseudoIndent{}
Data sets such as those accompanying \cite{JustinSalamon} 
(the smallest, labeled \CLOds{}, is profiled within
this chapter's data set)   
provide a good overview of data generated during 
species identification: in addition to audio 
samples themselves (in \WAV{} format), the data set 
includes \NPY{} (Numerical Python) files representing 
different spectral analysis methods applied 
to the bird songs, as well as a metadata file 
summarizing species-level data (such as the count 
of samples identified for each species).  
Every species also acquires a 4-letter identifier then used
as part of the \WAV{} and \NPY{} file names, which thereby 
themselves serve a classifying role, 
semantically linking each sample to its species.  
These three levels of information are a good 
example of the contrast in granularity 
\mdash{} and the mechanisms of information acquisition 
\mdash{} between raw Cyber-Physical input (the audio files), midstream 
processing (the spectral representations), and 
summarial overviews (species counts and labels;   
other avian data sets might also recognize 
geospatial coordinates obtained via noting sensor placement, 
as a further metadata dimension).
 
\item[Facial Recognition]  Given a frontal (or, potentially, 
partial) view, software can 
rather reliably match faces to a preexisting database or 
track faces across different locales (\cite{WeiLunChao},
\cite{YueqiDuan}, \cite{GaryHuang}, 
\cite{KalaiselviNithya}, \cite{FengLu}, etc.).  
The most common methodology depends 
on normalizing each foreground image segment (corresponding to 
one face) into a rectangle, whose axes then establish vector 
components for any features inside the segment.  Feature extraction 
then isolates anatomical features like eyes, nose, mouth, and 
chin, quantifying their position and distances, yielding a 
collection of numeric values which can statistically 
identify a person with relatively small error rates.  
\pseudoIndent{}
Given privacy concerns, enterprise or government use 
of this data is controversial: should analyses be performed 
on every person, or only on exceptional circumstances 
(crime investigation, say)?  Can facial-recognition outcomes 
be anonymized so that faces would be tracked across locations 
but not tied to specific persons without extra (normally 
inaccessible) data?  When and by whom should face data be 
obtainable, and under what legal or commercial circumstances?  
Should stores be allowed to use these methods to 
prevent shoplifting, for example?  What about searching 
for a missing or kidnapped child, or keeping tabs on an 
elderly patient?  When does surveillance cross the line 
from benevolent (protecting personal or public safety) 
to privacy-invasive and authoritarian?
\end{description}
}
\p{Of course, there are many other examples of Cyber-Physical 
devices and capabilities that could be enumerated.  But 
these cases illustrate certain noteworthy 
themes.  One observation is that a gap often exists between how 
devices physically operate and how they are conceptualized \mdash{} 
accelerometers, for instance, mechanically measure voltage, not 
acceleration or incline; but their data exposed to 
client software is constructed to be applied as vectors indicating 
persons' or objects' movement.  Moreover, multiple processing 
steps may be needed between raw physical inputs and usable  
software-facing data structures.  Such processing may generate 
a large amount of intermediate data; feature extraction 
from audio or image samples, say, can yield numeric aggregates with 
tens or hundreds of different fields.  Further processing usually 
reduces these structures to narrower summaries: an audio sample 
might be consolidated to a spatial location and temporal timestamp, 
along with a mapping to an individual person speaking 
(perhaps along with a text transcription), or 
human language spoken, or animal species.  Engineers then 
have to decide what level of detail to expose across a 
software network.  Another issue is integrating data from 
multiple sources: most of the more futuristic scenarios 
envision multi-modal Ubiquitous Computing spaces 
where, e.g., speech and biometric inputs are cross-referenced.
}
\p{Different levels of data resolution also 
intersect with privacy concerns: simpler data structures 
are more likely to employ private or sensitive 
information as an organizing instrument, heightening security 
and surveillance concerns.  For example, a simple 
facial-recognition system would match faces against known 
residents of or visitors to the relevant municipalities.  
This is less technologically challenging than anonymized 
systems which would persist more mid-processing data in order 
to complete the algorithmic cycle \mdash{} matching faces to 
concrete individuals \mdash{} only under exceptional circumstances; 
of course, though, it is also a greater invasion of privacy.
}
\p{Analogously, syncing speech technology with personal health 
data would be simplified by directly matching speaker identifications 
to biosensor devices wearers.  Again, though, using 
personal identities as an anchor for disparate data points makes 
the overall system more vulnerable to intrusive or inappropriate use.  
In total, security concerns might call for more complex data structures 
wherein shared data excludes the more condensed summaries 
wherever they may expose private details, and relies more 
on multipart, mid-level processing structures.  Rather than 
organize face-recognition around a database of persons, for example, 
the basic units might be numeric profiles paired with probabilistic 
links notating that a face detected at one time and place matches 
a face analyzed elsewhere, but without that similarity being 
anchored in a personal identifier.           
}
\p{Other broad issues raised by these CybePhysical case-studies 
include (1) testing and quality assurance and (2) data 
interoperability.  In the case of testing, many of the scenarios 
outlined above require complex 
computational transformations to convert raw physical data into 
usable software artifacts.  In \cite{FadelAdib}, for example, 
the authors present technology to measure heart rate from a 
distance, based on subtle analysis of physical motions 
associated with blood circulation and breathing.  The analytic 
protocols leverage feature extraction from wireless signal 
patterns.  As with feature extraction in audio and 
image-analysis (e.g. face recognition) settings, algorithms need 
to be rigorously tested to guard against false inferences 
or erroneous generated data.  In \cite{FadelAdib} the ultimate goal is to 
introduce heart and breathing monitors within a Smart Home environment,  
with computations performed on embedded Operating Systems.  
However, testing and prototyping of the technology should be 
conducted in a desktop Operating System environment so as to 
generate or leverage test data, document algorithmic revisions, 
and in general prove the system's trustworthiness in a 
controlled setting (including a software environment which 
transparently windows onto computational processes) before 
this kind of network is physically deployed.
}
\p{With respect to data integration, notice how projects mentioned 
here often anticipate pooled or overlapping information.  For 
instance, Smart Homes are envisioned to embed sensors analyzing 
speech, biomedical data-points like heart rate, atmospheric 
measurements (temperature, say) and appliance or architectural 
states (windows, doors, or refrigerator doors being open, 
ovens or stove burners being turned on, heaters/coolers being active, 
and so forth).  
In some cases this data would be cross-referenced, so that 
e.g. a voice command would close a window or turn off a stove. 
Analogously, \cite{MohamedSoltane} (one of whose co-authors 
is also a coauthor of this volume's 
chapter on bird species) describes a combination of 
face-recognition and speech analysis for \q{multi-modal biometric 
authentication}; here again a component supplying 
image-processing data and one supplying speech metrics 
will need to transmit data to a hub where the two inputs 
can be pooled.  Or, as I pointed out in the case of 
Mobile Whole Slide Imaging, image-segmentation, 
Cyber-Physical, and personal-health information fields 
may all be integrated into a holistic diagnostic platform.
}
\p{Overall, future Cyber-Physical systems may be integrated 
not only with respect to their empirical domain but in term 
of the environs where they are deployed \mdash{} Smart Homes, 
Smart Cities (or factories or industrial plants), hospitals and 
medical offices, schools and children's activities centers, 
refugee or displaced-persons camps/campuses, and so forth.  
I'll take Smart Homes as a case in point. 
We can imagine future \mbox{homes/apartments} provisioned with a panoply 
of devices evincing a broad spectrum of scientific backgrounds, 
from biology and medicine to ecology and industrial manufacturing.      
}
\vspace{-.1em}
\subsectiontwolinealt{How Internet of Things Interoperability
Affects Data Modeling Priorities}{How 
\q{Internet of Things} Interoperability Affects Data Modeling Priorities}
\p{So, let's imagine the following scenario: homeowners have a choice
of applications that they may install on their computers, 
supplied by multiple vendors or institutions, which access 
the myriad of Smart Home devices they've installed around the 
property.  Cyber-Physical products are engineered to 
interoperate with such hub applications \mdash{} and therefore 
with third-party components \mdash{} rather than just 
networking with their own proprietary access points, like phone apps.
}
\p{In this eventuality, technology inside 
the home is charged with pooling data from many kinds of devices 
into a comprehensive Smart Home platform, where users can see a 
broad overview, access disparate device data from a central location, 
and where cross-device data will be merged into aggregate models: 
e.g., cross-referencing speech and biometric inputs.  
Devices must be designed to broadcast 
data to third-party software platforms, and software 
hubs must wrangle received data
into a common format permitting integration algorithms \mdash{}
e.g. syncing speech and biometrics \mdash{} to operate properly.
Received device data must therefore be systematically 
mapped to appropriate transform procedures and 
\GUI{} components.  This is important because
we are no longer considering data models from the viewpoint 
of devices' own capabilities (viz., their specific physical
measurements and parameters).
More technically, the
key libraries associated with devices are no longer
merely low-level drivers or \IoT{} signal processors.
Instead, the technology stack would include an 
intermediate semantic layer acting between \q{smart objects}
and corresponding hub applications.
}
\p{The data models at this semantic midlayer, moreover,
would no longer be device-centric.  Instead, they would
be assessed in a software-centric milieu: how do we
route device data to proper interpretive procedures?  
How do we consolidate device data into \GUI{} presentations?  
This means that common representational formats for 
wireless data, or standardized \IoT{} Ontologies 
\mdash{} however much these may technically embody 
semantic middleware \mdash{} are insufficient as \i{high-level}, 
\i{software-centric} semantic layers.
}
\p{In the case of a Smart Home, once we commit to aggregating devices
via a software hub, the key organizing principle is 
a mesh of procedures implemented in the hub application 
that can pool all relevant devices into one information 
space.  What needs to be standardized then are not 
so much data \i{formats}, but in fact data-handling
\i{procedures}.  
}
\p{To put it differently, device manufacturers would now be dealing 
with an ecosystem in which hub applications receive and 
aggregate their data, affording users access points to and
overviews of device data and state.  Hub applications may be 
provided by different companies and iterations, 
their inner workings opaque to devices themselves.  What 
can be standardized, however, are the \i{procedures} 
implemented within hub software to receive and properly 
act upon device data.  Software might guarantee, for example, 
that so long as devices are supplying signals in 
their documented formats, the software has capabilities 
to receive the signals, unpack the data, and internally 
represent the data in a manner suitable for device particulars.  
Devices can then specify what kind of internal representations 
are appropriate for their specific data, essentially 
specifying conditions on software procedures and data types.  
}
\p{In short, the key units of mutual trust and verification among 
and between Cyber-Physical devices and Cyber-Physical software 
are not, in theory, data structures themselves, but instead 
\i{procedures} for processing relevant data structures.  
Robust Cyber-Physical ecosystems can be developed by 
reinforcing \i{procedural alignment} wherever possible, 
including by curating substantial collections of 
reusable software libraries, either for direct 
application or as prototypes and testing tools.  
Procedural \i{alignment} means that disparate components 
become integrated by virtue of each components' 
anticipating the procedures which the others have 
available for sending, receiving, and manipulating 
data \mdash{} an integration strategy I'll dub 
\q{\POSC{}}\phantomsection\label{POSC} 
(Procedure-Oriented and Software-Centric) 
by contract to \i{data format} or even \i{Ontological} alignment.
}
\p{Suppose many Cyber-Physical sensors were paired 
with open-source code libraries which illustrate how 
to process the data each device broadcasts.  
Commercial products could use those libraries directly, or, 
if they want to substitute closed-source alternatives, 
might be required to document that their data management 
emulates the open-source prototypes.  Test suites and 
testing technology can then be implemented against the 
open-source libraries and reused for stress-testing 
analogous proprietary components.  This appears to be 
the most likely path to ensuring interoperable, 
high-quality Cyber-Physical technology that 
serves the ultimate goal of integrated Smart Home 
(and Smart City, etc.) solutions. 
}
\p{That hypothesis notwithstanding,
there are a lot more academic papers on
Cyber-Physical Ontologies or common signal/message formats, 
like \CoAP{} and \MQTT{} (\cite{FirasAlbalas}, 
\cite{BadisDjamaa}, \cite{RiccardoGiambona}, 
\cite{CenkGundogan}, \cite{JussiHaikara}, \cite{AlejandroRodriguez}, 
etc.) than there are open-source libraries 
which prototype device data, its validation, parameters, 
and proper transformations.\footnote{The rationale for emphasizing standard data formats is 
probably that these formats constrain any procedure 
which operates on the data, so standardization in 
data representation indirectly leads to standardization 
in data-management procedures, or what I am calling 
\q{procedural alignment}.  This accommodates the fact 
that shared data may be used in many different 
software environments \mdash{} components implemented in 
different programming languages and coding styles.   
However, more detailed 
guarantees can be engineered by grounding standardization 
on procedures rather than data representations themselves.  
To accommodate multiple programming languages and paradigms, 
data models can be supplemented with a 
\q{reference implementation} which prototypes 
proper behavior \visavis{} conformant data; components 
in different languages can then emulate the prototype, 
serving both as an implementation guide and a 
criterion for other developers to accept a new 
implementation as trustworthy.  There are several examples 
of a reference implementation used as a standardizing 
tool, analogous to an Ontology, such as the \librets{} 
(Real Estate Transaction Standard) library, 
servers and clients for \FHIR{} (Fast Healthcare Interoperability 
Resources), and clients for \DICOM{} (Digital Imaging and 
Communications in Medicine), e.g. for Whole Slide Imaging 
(see \url{https://www.orthanc-server.com/static.php?page=wsi}).
}  A good case-in-point can be found in 
\cite[pages 4 ff.]{PatroBanerjee} using \CLang{} 
structures to model \CoAP{} meta-data: while it is 
reasonable, even expected, for low-level driver code 
to be implemented in \CLang{}, this code should also 
be the basis of data models implemented in a language 
like \Cpp{} where dimensions and ranges can be made explicit 
in the data types.  Nevertheless, many engineers will 
instead focus on describing the more nuanced 
data modeling dimensions through Ontologies and 
other semantic specifications wholly separate from 
programming type systems.  This has the effect of 
scattering the data models into different artifacts: 
low-level implementations with relatively little 
semantic expressiveness, and expressive 
Ontologies which, due to a lack of type-level 
correspondence between modeling and programming 
paradigms, are siloed from implementations themselves.
}
\p{Conversely, in lieu of \q{data-centric}
Ontologies whose mission is to standardize how information 
is mapped to a \i{representation}, in this chapter
I will consider \q{Procedural}
Ontologies: ones which focus on procedural capabilities and 
requirements that indicate whether software components 
are properly managing (e.g., Cyber-Physical) data.  The 
idea is that proving procedural conformance should be a 
central step, and an organizing groundwork, for 
showing that software intended for production deployment
is trustworthy and complies with technical and legal specifications.
}
\p{In practical terms, the above discussion mentioned the \CLOds{}
(avian) data set and then the chapter by Vuddagiri \i{et. al.},
which (as noted below) builds off
the \APds{} \q{challenge} corpus.  I will also
be referring to recent \CoNLL{} corpora.  So, together,
these constitute three representative examples of
data sets tangibly applicable to Cyber-Physical and/or 
\NLP{} Research and Development: one audio-based 
(for species identification); one audio/speech;
and one linguistic.  Based on the idea that 
\RnD{} data sets should germinate deployment data models, 
we should look to data sets like these 
to provide semantic and type-theoretic encapsulations 
of their information spaces and analytic methods 
(e.g., spectral waveform analysis, or Dependency Grammar 
parsing).  Accompanying materials for this chapter 
provide profiles of the three aforementioned data sets, 
which consolidate their various files and formats into 
a streamlined \q{\POSC{}} representational paradigm.
The demo code presents examples 
of procedures and data types targeting these 
data sets, as well as an architecture for 
deploying data sets in a procedure-oriented and 
software-centric manner, in terms of how files
and the information they contain are organized,
and in terms of employment of data-publishing standards
such as the \q{Research Object} model 
(\cite{KhalidBelhajjame}, \cite{KhalidBelhajjameWorkflow},
\cite{PhilipEBourne}, 
\cite{KatrinaFenlon},
\cite{MarkDWilkinson}). 
}
\p{At present, to make these issues more concrete
with further case-studies,
in this introductory discussion I will examine in
more detail the specific case of speech and language 
data structures.
}
\subsection{Linguistic Case-Study}
\p{Establishing data models for deployed technology 
is certainly part of the Research and Development 
cycle, which means that data profiles tend 
to emerge within the scientific process of 
formulating and refining technical and algorithmic 
designs.  This is particularly true for complex, 
computationally nuanced challenges such as image segmentation 
or (to cite one above example) measuring heartbeats 
and breathing patterns via subtle waveform analysis.  
We can assume that every \RnD{} phase will itself leave
behind an ecosystem of testing data and code which 
can be decisive for consolidating data models, directly 
or indirectly influencing production code for systems 
(even if their deployment and commercialization is 
well after the \RnD{} period).       
}
\p{In the case of speech and language technology, 
a research-oriented data infrastructure has been 
systematically curated, in several subdisciplines, 
by academic or industry collaborations.  The 
Conference on Natural Language Learning (\CoNLL{}), for 
example, invites participants to develop 
Natural Language Processing techniques targeted at 
a common \q{challenge} dataset, updated each year.  These 
data sets, along with the data formats and code libraries 
which allow software to use that data, thereby serve 
as a reference-point for Computational Linguistics 
researchers in general.  Similarly, this volume's 
chapter on Language Identification describes research 
targeting a multilingual data set (labeled \APds{}) curated for an
annual \q{Oriental Language Challenge} conference 
dedicated to language/dialect classification for 
languages spoken around East Asia (from  
East Asian language families and also Russian).  
}
\p{Technically, curated and publicly accessible data 
sets are a different genre of information space than 
real-time data generated by Cyber-Physical technology 
(e.g. voices picked up by microphones in a Smart Home).  
That is to say, software developed to access speech 
and language data sets like the \CoNLL{}'s or the 
Oriental Language Challenge has different requirements 
than software responding to voice requests in real time 
\mdash{} or other deployment use-cases, such as medical 
transcription, or identifying dialects spoken in an 
urban community.  However, data models derived from 
publicly shared test corpora \i{do} translate over 
to realtime data: we can assume that \RnD{} 
data sets are collections of signals or information 
granules which are structurally similar to those 
produced by operating Cyber-Physical devices.  
As a result, portions of the software targeting 
\RnD{} data sets \mdash{} specifically, the procedures 
for acquiring, transforming, validating, and 
interactively displaying individual samples \mdash{} 
remain useful as components or prototypes for 
deployed products.  Code libraries employed in 
\RnD{} cycles should typically be the basis 
for data models guiding the implementation of 
production software. 
}
\p{To make this discussion more concrete, I will use 
the example of \CoNLL{} data sets.  This chapter's 
demo includes samples from the most recent 
collection of \CoNLL{} files and 
conference challenge tasks (at the time of writing) 
as well as demo code which operates on such 
data via techniques described in this chapter.  
The \CoNLL{} format is representative of the 
kinds of linguistic parsing requisite for 
using Natural Language content (such as 
speech input) in Cyber-Physical settings.
}
\p{This volume's chapter by Teixera \i{et. al.}   
considers voice-activated 
Cyber-Physical interfaces; here Natural Language 
segments become the core elements in translating 
user queries to actionable software responses.  The proposed 
systems analyze speech patterns to build textual 
reconstructions of speakers' communications, then 
parses the text as Natural Language content, before 
eventually (if all goes well) interpreting the 
parsed and analyzed text as an instruction the 
software can follow.  Text data can then be 
supplemented with metrics measuring vocal 
patterns, syntactic and semantic information, speaker's 
spatial location, and other information that can 
help interpret speakers' wishes insofar as software 
can respond to them.  
}
\p{The authors discuss, for instance, the possibility 
of annotating language samples (after speech-to-text 
translation) with Dependency Grammar parses.\footnote{Adequately describing Dependency Grammar
is outside the scope of this chapter, but, in a nutshell, 
Dependency Grammar models syntax in terms of 
word-to-word relations rather than via phrase 
hierarchies; as such, Dependency parses yields 
directed, labeled graphs (node labels are words and 
edge labels are drawn from an inventory of 
syntactic inter-word connections),  
which are structurally similar to Semantic Web 
graphs.  See also \cite{AbromeitChiarcos}, 
\cite{KongRushSmith}, \cite{Nivre},
\cite{OsborneMaxwell}, \cite{SivaReddy}, 
\cite{Schneider}, 
\cite{XiaPalmer}, etc.; variants include 
Link Grammar \cite{GoertzelPLN}, \cite{ErwanMoreau}  
and Extensible Dependency 
Grammar \cite{DebusmannThesis}, 
\cite{DebusmannDuchierRossberg},
\cite{MichaelGasser}.
}  Textual content can also be annotated with 
models of intonation, stress patterns, and other 
acoustic features (because the original inputs are 
audio-based) that can help an \NLP{} engine to 
properly parse sentences (for instance by 
noting which words or syllables are vocally emphasized).  
We can assume that audio processing 
as well as \NLP{} technology would supply intermediary 
processing somewhere between acoustic devices and 
the centralized application. 
}
\p{Different kinds of linguistic details require different 
data models.  Dependency parses, for instance, 
are often notated via some version of a specialized 
\CoNLL{} format, which textually serializes parse and 
lexical data, usually one word per line.  
The most recent standard (dubbed \CoNLLU{}) recognizes 
ten fields for each word, identifying, in particular, 
Parts of Speech and syntactic connections with other 
words (see e.g. \cite{ChiarcosSchenk}, 
\cite{JohannesGraen}, \cite{DanielHershcovich},  
\cite{AmirMore}, \cite{MilanStraka}).  
As a custom format, \CoNLLU{} requires its 
own parser, such as the \UDPipe{} library for 
\Cpp{} (a slightly modified version of this library 
is published with this chapter's data set).  
So, for hub applications, a reasonable assumption 
is that these programs compile in the \UDPipe{} library 
or an alternative with similar capabilities, 
in order for them to handle parsed \NLP{} data.
}
\p{Suppose, then, that a hub application will periodically receive a 
data package comprising an audio sample along with 
text transcriptions and \NLP{}-generated, e.g., 
\CoNLLU{} data.  To make sense of linguistic content, 
the software would presumably pair the \NLP{}-specific 
information with extra details, such as, 
the identity of the speaker (if a Smart Home system 
knows of specific users), where and when each sentence 
or request was formulated, and perhaps the original 
audio input (allowing functionality such as users playing 
back instructions they uttered in the past).  A relevant 
data model might thereby comprise: (1) \CoNLLU{} data 
itself; (2) location info, such as spatial position and 
which room a speaker is found in; (3) timestamps; 
(4) speaker info, if available; (5) audio files; 
and maybe (6) extra acoustical or intonation data. 
(Those extra details could include annotations based on how 
conversation analysts notate speech patterns, or 
might be waveform features derived from initial 
processing of speech samples.)   
}
\p{This data model would presumably translate to multiple 
data types: we can envision (1) a class for \UDPipe{} 
sentences obtained from \CoNLLU{}; (2) a class for audio 
samples; (3) speaker and time/location information; plus 
versions of these classes appropriate for 
\GUI{}s and database persistence.  And, in addition, 
these data requirements for speech and text samples only 
considers obtaining a valid parse for the text; 
to actually react to speech input, an application would 
need to map lexical data to terms and actions the software 
itself, 
in the context of its own capabilities,  
can recognize.  For instance, \i{close the window} 
would map to an identifier for which window is intended 
(inferred perhaps from speaker location) and a 
\i{close} operation, which could be available via 
actuators embedded in the window area.  All of the 
objects that users might semantically reference 
in voice commands therefore need their own data 
models, which must be interoperable with linguistic 
parses.  So along with data types specific to 
linguistic elements we can consider \q{bridge} types 
connecting linguistic data (e.g., lexemes) to 
data types modeling physical objects.  
}
\p{Likewise, we 
can anticipate the procedures which speech and/or language 
data types need to implement: correctly decoding 
\CoNLLU{} files; mapping time/location data points to a 
spatial model of the Smart Home (which room is 
targeted by the location and also if that point is 
close to a door, window, appliance, and so forth; 
and perhaps matching the location to a \ThreeD{} 
or panoramic-photography graphics 
model for visualization); audio-playback procedures, 
along with interactive protocols for this process such 
as users pausing and restarting playback; procedures 
to map identified speakers to user profiles known to 
the Smart Home system.  The audio devise makers and 
\NLP{} providers \mdash{} assuming those products are 
delivered as one or several suites separate and apart from the 
Smart Home hub\footnote{For this discussion I assume that hub applications do 
not perform \NLP{} internally, but rather receive 
post-\NLP{} input from \NLP{} technologies designed 
with particular attention to vocal commands for \q{smart} objects. 
} \mdash{} can mandate that hub applications 
demonstrate procedural implementations that satisfy 
these requirements as a precondition for accessing 
their broadcast data.  Conversely, hub applications can 
stipulate the procedural mandates they are prepared 
to honor as a guide to how devices and their drivers 
and middleware components should be configured for 
an integrated Smart Home ecosystem. 
}
\p{The essential point here is that procedural requirements 
and validation becomes the essential glue 
that unifies the diverse Smart Home components, and 
allows products designed by different companies, 
with different goals, to become interoperable.  
Once again, procedural alignment and predictability is 
more important than standardized data formats.   
}
\p{We can also consider representative criteria for 
testing procedures; in particular, preconditions that 
procedures need to recognize.  In \CoNLLU{}, 
individual words can be extracted from a parse-model, 
but the numeric index for the word must fall within a 
fixed range (based on word count for the 
relevant sentence).  In audio playback, time intervals 
are only meaningful in the context of the length of 
the audio sample in (e.g.) seconds or milliseconds.  
Similarly, features extracted from an audio 
sample (of human speech or, say, a bird song) are localized 
by time points which have to reside within a sample window; 
and image features are localized in rectangular coordinates 
that need to fit inside the surrounding image.  
Therefore, procedures engaged with these data 
structures should be checked to ensure that they 
honor these ranges and properly respond to 
faulty data outside them.  This is an example 
of the kind of localized procedural testing which, 
cumulatively, establishes software as trustworthy.
}
\p{I will discuss similar procedural-validity issues 
for the remainder of this section before 
developing more abstract or theoretical models 
of procedures, as formal constructions, subsequently 
in the chapter. 
}
\vspace{-.1em}
%\spsubsection{Proactive Design}
\subsection{Proactive Design}
\p{I have thus far argued that applications
which process Cyber-Physical data should rigorously organize their functionality
around specific devices' data profiles.  The procedures that directly interact
with devices \mdash{} receiving data from and perhaps sending instructions
to each one \mdash{} will in many instances be \q{fragile} in the sense
I invoke in this chapter.  Each of these procedures may make assumptions
legislated by the relevant device's
specifications, to the extent that using any one procedure too broadly
constitutes a system error.  Furthermore, Cyber-Physical devices may
exhibit errors due to mechanical malfunction, hostile attacks,
or one-off errors in electrical-computing operations, causing
performance anomalies which look like software mistakes even if the code is
entirely correct (see \cite{MichaelEngel} and
\cite{LavanyaRamapantulu}, for example).  As a
consequence, \i{error classification} is especially
important \mdash{} distinguishing kinds of software errors
and even which problems are software errors to begin with.
}
\p{Summarizing the case studies from earlier in this section, 
Table~\ref{table:profiles} identifies several details 
about dimensions, parameters of operation, data fields, 
and other pieces of information relevant to 
implementing procedures and data types capturing 
Cyber-Physical data.  These types may derive from 
Cyber-Physical input directly or may model artifacts 
constructed from Cyber-Physical input midstream, such 
as audio or image files, or text transcriptions 
representing speech input.  The summaries are not 
rigorous profiles, just suggestive cues 
about what sort of details engineers should consider 
when formalizing data models.  In general, detailed 
models should be defined for any input source 
(including those transformed by middleware 
components, such as \NLP{} engines), thereby 
profiling both Cyber-Physical devices and also 
\q{midstream} artifacts such as audio or image files 
\mdash{} i.e., aggregates, derived from Cyber-Physical 
input, that can be shared between software components 
(within hub applications and/or between hubs and 
middleware).
}
\begin{table}
\label{table:profiles}
\begin{flushleft}
{\footnotesize
\begin{tabularx}{9.5cm}{>{\raggedright}p{21mm}|p{31mm}|p{30mm}}
\toprule
\textbf{Blood Pressure} & Beats per Minute &  Systolic $>$ Diastolic \\
\midrule
\textbf{Accelerator} & Incline; Movement in 2 or 3 Dimensions &  Biaxial or Triaxial \\
\midrule
\textbf{Audio Sample} & Binary data/file and/or feature set &  Length (in time) (1D) \\
\midrule
\textbf{Audio Feature} & Time-interval inside sample &  Subinterval of sample (1D) \\
\midrule
\textbf{Image} & Matrix of values in a colior model: {\smRGB}, {\smRGBa}, 
 {\smHSV}, etc. &  Matrix Dimensions (2D) \\
\midrule
\textbf{Image Segment} & Rectangular Coordinates; geometric 
characterization (e.g. ellipse dimensions); scale of resolution & Subregion of image (2D) \\
\midrule
\textbf{Bioacoustic Sample} & Audio Sample plus species identifier; geospatial 
and timestamp coordinates  & Well-formed spacetime coordinates \\
\midrule
\textbf{Speech Sample} & Audio Sample plus text transcription; spacetime coodinates; 
  Identify language/dialic and/or speaker & Valid dialect and/or speaker identifier \\
\midrule
\textbf{Dependency-Parsed Text} & Text transcription plus parse serialization; 
  audio metadata & Valid and accessible metadata \\
\midrule
\textbf{Lexical Text Component} & Index into parse serialization; Part of Speech 
tag; lexical/semantic classification & Valid index \\
\bottomrule
\end{tabularx}
}
\end{flushleft}
\caption{Example Data Profiles}
\end{table}


\p{These data profiles need to be integrated with Cyber-Physical code from a
perspective that cuts across multiple dimensions of project scale and
lifetime.  Do we design for biaxial or triaxial accelerometers, or both,
and may this change?  Is heart rate to be sampled in a context where
the range considered normal is based on \q{resting} rate or is it
expanded to factor in subjects who are exercising?  These kinds
of questions point to the multitude of subtle and project-specific
specifications that have to be established when implementing and then
deploying software systems in a domain like Ubiquitous Computing.
It is unreasonable to expect that all relevant standards will be
settled \i{a priori} by sufficiently monolithic and comprehensive
data models.  Instead,
developers and end-users need to acquire trust in a development process
which is configured to make standardization questions become apparent
and capable of being followed-up in system-wide ways.
}
\p{For instance, the hypothetical questions I pondered in
the last paragraph \mdash{} about biaxial vs.
triaxial accelerometers and about at-rest vs. exercise
heart-rate ranges \mdash{} would not
necessarily be evident to software engineers or project architects when the
system is first conceived.  These are the kind of modeling questions that tend
to emerge as individual procedures and datatypes are
implemented.  For this reason, code development serves a role beyond just
concretizing a system's deployment software.
The code at fine-grained scales also reveals questions that need to be
asked at larger scales, and then the larger answers reflected back in the
fine-grained coding assumptions, plus annotations
and documentation.  The overall
project community needs to recognize software implementation as a crucial
source for insights into the specifications that have to be established
to make the deployed system correct and resilient.
}
\p{For these reasons, code-writing \mdash{} especially at the smallest scales \mdash{}
should proceed via paradigms disposed to maximize
the \q{discovery of questions} effect
(see also, as a case study, \cite[pages 6-10]{Arantes}).  Systems in operation will be
more trustworthy when and insofar as their software bears witness to a project
evolution that has been well-poised to unearth questions
that could otherwise diminish the system's trustworthiness.
}
\p{\q{Proactiveness}, like transparency and trustworthiness, has been
identified as a core \USH{} principle, referring (again in
the series intro, as above)
to \q{data transmission to healthcare providers
... \i{to enable necessary interventions}} (my emphasis).  In
other words \mdash{} or so this language implies, as an
unstated axiom \mdash{}
patients need to be confident in deployed \USH{} products
to such degree that they are comfortable with clinical/logistical
actions \mdash{} the functional design of medical spaces; decisions about
course of treatment \mdash{} being grounded in part on data generated from
a \USH{} ecosystem.  This level of trust, or so I would argue,
is only warranted if patients feel
that the preconceived notions of a \USH{} project have been vetted against
operational reality \mdash{} which can happen through the interplay between
the domain experts who germinally envision a project and the programmers
(software and software-language engineers) who, in the end, produce its
digital substratum.
}
\p{I have argued that hub \i{libraries} \mdash{} intermediaries between Cyber-Physical 
devices and hub applications \mdash{} are the key components where  
diverse requirements may be exercised.  Hub libraries therefore need 
capabilities for documenting coding assumptions and requirements,  
such that their corresponding applications garner users' trust 
and acceptance.  Hub applications, in short, would be deemed 
trustworthy insofar as their hub libraries are 
properly engineered.  These, then, are the practical concerns 
driving the code-documentation proposals I will develop 
in the next two sections.  Hub libraries are an environment where 
these techniques may be especially applicable.   
}
