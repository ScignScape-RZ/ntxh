\p{In its first incarnation, Conceptual Space Theory was rooted in 
linguistics, or secondarily perhaps in philosophy \mdash{} insofar 
as any \q{theory of concepts} should be of philosophical 
interest.  This theory's year zero was 2000, via Peter \Gardenfors{}'s 
\i{Conceptual Spaces: The Geometry of Thought} (\cite{Gardenfors}).  
Soon, though, the theory migrated from linguistics to realms of 
science and technology: as a tool for modeling the evolution of 
scientific theories (\cite{Zenker}, \cite{Strle}), for describing scientific/technical 
data (\cite{RaubalAdams}, \cite{MartinRaubal}), for bridging 
cognitive and computational linguistics (\cite{KennethHolmqvist}), 
and in general for bringing a more nuanced understanding of 
cognition into computational settings (\cite{RaubalAdamsMore}).  
Advocates for Conceptual Space Theory saw \Gardenfors{}'s perspective 
as more faithful to human language and conceptualization, 
more humanistically accurate, than reductive \q{mind as computer} 
metaphors that often dominate research at the boundaries between 
cognitive and computer science \mdash{} Artificial Intelligence, 
certainly, but also Knowledge Engineering, Data Modeling, 
and technological fields where representing human concepts 
become imporant.  Concepts become \i{technological} 
subject matter when we have to build computational 
platforms \mdash{} software, databases \mdash{} to classify facts 
and documents for human's interactive use, in a manner 
sensitive to human concepts and their influence on 
how humans will understand (and therefore want to interact 
with) an information-space, such as a collection of 
research paper, or scientific research data.      
}
\p{Renewed interest in Conceptual Space Theory is evinced by 
recent projects integrating this theory with 
Hypergraph Categories (\cite{Fong}, \cite{Dagger}), with the 
core notion of merging a Hypergraph-based grammar with a 
Conceptual Space semantics.  Bob Cocke's 2015 paper (\i{et. al.}, 
\cite{BobCocke}) is the most direct statement of this new 
paradigm: essentially a strategy for co-theorizing syntax 
and semantics in a strongly formal, mathematically oriented 
spirit but also receptive to Conceptual Space Theory's cognitive 
and linguistic nuance.  That is, the proposed Hypergraph/Conceptual Space 
hybrid can orient both formal research (in fields like 
computer science and formal language theory) and more 
informal (or at least not logicomathematically formal) 
methodologies in linguistics and general \q{theories of 
meaning}.
}
\p{I believe the Hypergraph/Conceptual Space hybrid (which I'll 
nickname \q{\HCS{}}) is an important breakthrough, especially 
because it follows analogous research unifying theories 
of cognition/conceptualization with computational and 
technological endeavors (at a practical level, not only 
theoretical).  The Semantic Web, for example \mdash{} grounded 
in graph-based representations of technical data \mdash{} 
embodies an applied technological project but also 
the influence of philosophy and cognitive science, particularly 
through the notion of \i{Ontologies}, or formal models 
of some domain of information or conceptualization, used 
to specify the morhology (i.e., possible morphologie) of 
data structures which carry information relevant to 
the corresponding domains.  Although the technical 
restructuring of data \mdash{} for communication between different 
computing environments \mdash{} is a mostly bare-bones algorithmic 
problem, Ontologies also capture how formal systems have 
to model scientific concepts and norms that structure 
technical data in an overarching, gestalt fashion: 
notions like space, time, causality, points vs. regions, 
punctuality vs. perdurance, different spatial relations 
(inside, around, connected to, etc.), and so forth.  
Formulating logical models for these foundational conceptual 
building-blocks became essential for comprehensive models 
of scientific and technical domains, insofar as 
technical data is to be digitally archived and shared, 
so that the representation of human \q{phenomenological} 
concepts within structured, digital information systems 
becomes a pertinent question.   
}
\p{So largely speculative and theoretical philosophical 
territory like Ontology and Phenomenology started to become, 
particularly in this century, an origin for technical, 
digital-engineering-style research (see e.g. \cite{Donnelly}, 
\cite{SmithPetitot}, etc.)  Certainly only a minority of 
programmers constructing, say, the Semantic Web were 
actively engaged with these more philosophical 
foundations, but the collision between speculative 
traditions and contemporary digital applications did 
engender a certain interdisciplinary circle.  This milieu, 
in turn, overlapped with Cognitive Linguistics; as 
\Gardenfors{} puts it: 
.
}
\p{Meanwhile, researchers like Jean Petitot and Barry Smith 
were trying to absorb Husserlian Phenomenology into a 
computational and technological environment, partly 
through data-modeling Ontologies (though also via 
analyses of mathematical/computational models of 
vision and perception, such as in the realm of 
3D computer graphics, supplying approximate sketches 
of human perception \mdash{} e.g. in Petitot's works 
like \cite{PetitotSyntaxe}, \cite{Petitot} and in the 
broad literature on Mereotopoloy: ... ).  As 
Maxwell James Ramstead puts it in a 2015 master's thesis:
\begin{dquote}Now, the \q{science of salience}
proposed by Petitot and Smith (1997) illustrates the
kind of formalized analysis made possible through the direct
mathematization of phenomenological descriptions.
Its aim is to account for the invariant descriptive
structures of lived experience (what Husserl called \q{essences})
through formalization, providing a descriptive geometry of
macroscopic phenomena, a \q{morphological eidetics} of the
disclosure of objects in conscious experience (in Husserl's
words, the \q{constitution} of objects).
Petitot employs differential geometry and morphodynamics
to model phenomenal experience, and Smith uses formal structures from
mereotopology (the theory of parts, wholes, and their boundaries)
to a similar effect. \cite[p. 38]{Ramstead}
\end{dquote} 
}
\p{In principle, the Semantic Web and its accompanying technologies 
(like \RDF{} \mdash{} the Resource Description Format \mdash{} and 
\OWL{} \mdash{} Web Ontology Language) were the primary vehicle for 
operationalizing this \q{phenomenological} background.  
In practical terms, Barry Smith, for example, spearheaded 
applied projects like the \OBO{} (Open Biomedical Ontology) 
foundry and the National Center for Ontological Research 
(\NCOR{}).  Meanwhile, though, other scholars in the 
cognitive-linguistic tradition \mdash{} not least \Gardenfors{} 
himself \mdash{} objected to Semantic Web paradigms being 
oversimplistic, or too far removed from the patterns 
and priorities of actual human reasoning (see 
\cite{GardenforsSemanticWeb}).  Applications like 
Conceptual Space Markup Language were, to some degree, 
explicitly conceived as alternatives to Semantic Web 
formats like \RDF{} and \OWL{}. 
}
\p{The \q{\HCS{}} model adds a wrinkle to this intellectual 
history, some 15 years after debates like the 
implicit contrast between \Gardenfors{} and Barry Smith 
arose.  Writings of Barry Smith or Jean Petitot 
reveal how technological applications can be one venue for 
extending or consolidating philosophical frameworks: 
in the admittedly artificial and not-fully-human digital 
ecosystem we can still see a revealing trace of 
philosophical intuitions, the analytic reach of 
their assumptions, the potential for isolating 
within accounts of \i{human} meaning and reason 
certain gestalts where are intrinsic to 
\i{meaning} or \i{reason} as such, even without 
human subtlety and sociality/situatedness/embodiment.  
The Semantic Web is retroactively interesting 
to philosophy, as if \Gardenfors{}-style critiques thereof.     
}
\p{With the emergence of the Hypergraph-based \HCS{} model, which to 
some degree presents its own contrast to the 
(graph-based) Semantic Web, there is another iteration of 
Conceptual Space approaches that potentially address  
critiques that \q{the Semantic Web is not very semantic} 
(\Gardenfors{}) or \q{The Semantic Web Needs More Cognition} 
(Martin Raubal and Benjamin Adams).  Against the precedent 
that Semantic Web theory (or its alternatives) retroactively 
shed light on philosophical, cognitive, and phenomenological 
issues, we can see the Hypergraph-Conceptual Space hybrid as a 
new technological model that has its own ramifications for 
philosophy, and generally outside the narrow 
mathematical/computer-science fields.  In short, 
a comprehensive, multi-disciplinary actualization of the 
\HCS{} would become an intellectual construct 
stretching between highly formal theories such as 
Hypergraph Categories and more humanistic contexts 
such as Cognitive Linguistics and Cognitive Phenomenology. 
}
\p{With that said, I propose to develop the \HCS{} model in a 
manner that departs somewhat from both the Hypergraph 
analysis (on the syntactic side) and \Gardenfors{}'s 
original Conceptual Space theory (on the semantic side).  
My own formal Hypergraph model will be oriented more 
toward software engineering and applied type theory 
than toward pure mathematics \mdash{} while automated 
proofs can sometimes bring strong guarantees about 
the correctness of some software component, in the 
more typical applications we need well-structured 
data models to guide the implementation of 
software and data-sharing platforms.  As a result, 
formal models may be more valuable insofar as 
they facilitate computational integration, rather than 
drive mathematical theorems or analyses (e.g., via 
Category Theory).  I will clarify notions like 
\q{integration} and \q{implementation}, as I see them, 
below.  On this basis, I will both add some structure 
to the original \HCS{} framework and also present it in 
a more casual, less mathematical fashion. 
}
\p{Meanwhile, on the semantic side of things, I 
will attend specifically to the constrast between 
applications of Conceptual Space theory to \i{formal} 
(or technical/technological) semantics or, respectively, 
to (natural) linguistics.  Computational applications 
of \Gardenfors{}'s models are often motivated a sense 
that \Gardenfors{} has isolated conceptual structures 
underpinning semantics that transcend the distinction 
between \q{human} (or natural-language) and artificial 
semantics.  This trope seems to reappear in numerous 
papers (many I cited earlier) adapting Conceptual Space Theory 
to computational settings, as if \Gardenfors{} has unmasked 
a well-behaved core amidst the seeming chaos of human 
concepts, on which basis we can more confidently move toward a 
paradigm of computers understanding language on our own 
terms.   
} 
\p{In this area I find the applications of Conceptual Spaces 
still a little too reductionistic.  I think there 
certainly \i{are} both formal/technological and 
philosophical/linguistic applications of Conceptual Space 
theory, which I will try to summarize here, but 
as I see it the theory blossoms differently in the 
two contexts.  It is interesting to speculate on 
what structures or paradigms tie together the two 
branches, and what implications that might have, 
but in this paper I will largely treat 
Conceptual Space models as branching into two 
technically distinct concretizations depending 
on whether we are analyzing natural language or 
specifying codifications for data models and formal 
semantics.  I will start by reviewing the 
landscape from the Natural Language perspective 
before developing my own continuation of the 
underlying Conceptual Space theory \mdash{} particularly 
via a specific account of applied type theory \mdash{} later in the paper.
}
\p{}
\p{}
