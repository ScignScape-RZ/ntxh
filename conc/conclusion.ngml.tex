\section{Conclusion}
\p{When Conceptual Space Theory migrated from a natural-language and 
philosophical environment to a more technical and scientific 
foundation \mdash{} as a basis for modeling scientific data 
and analyzing scientific theories and theory-formation 
\mdash{} it also picked up certain evident practical applications.  
For example, \CSML{} was a concrete proposal for technical 
data modeling whose exlplicit goal was to be more 
conceptually expressive and scientifically rigorous than 
conventional \mdash{} or even \q{Semantic Web} \mdash{} data 
sharing tactics.  So one obvious domain for concrete 
applying Conceptual Space Theory lies in the 
communicating and annotating of scientific 
(and other technical research) data.  
This use-case could certainly benefit from the 
added structure of Hypergraph syntactic models 
(which can engender hypergraph serialization formats) 
and hypergraph-based type theories.
 }
\p{So a Hypergraph/Conceptual Space hybrid can readily be 
imagined as a kind of next-generation 
extension of the Semantic Web or reincarnation of 
\CSML{}, with an emphasis on sharing scientific 
data in a format conducive to capturing the 
theoretical context within which data is generated.  
This is still removed from the \i{natural language} 
origins of Conceptual Spaces, but it would mark 
a further step in the evolution of 
\Gardenfors{}'s theory from a linguistic to 
a metascientific paradigm.  
}
\p{But going even a step further, a data-sharing 
framework emerging in the scientific context 
may retroactively be utilized in a more 
humanistic context as well; so an \HCS{} hybrid 
may find applications in the conveying 
of \i{humanities} data \mdash{} natural language 
structures (parse trees or graphs, lexicons, 
and so forth), sociological/demographic 
data sets, digitized artifacts (art, 
archaeology, museum science), etc.    
In this scenario Conceptual Spaces might 
be relevant to, say, Cognitive Linguistics 
on two level \mdash{} a practical, software-oriented 
tool for linguistic research in its digital 
practice, alongside a paradigm for natural language 
semantic at the theoretical level.  These two 
modes of application may not have fully aligned 
theoretical commitments, but they would 
reveal a core Conceptual Space theory diversify, 
branching, and adapting to different practical 
and theoretical requirements. 
}
