\section{Conclusion}
\p{To regard data modeling as just a practical, behind-the-scenes 
endeavor is to underestimate the scientific richness and 
importance of data modeling paradigms as theoretical 
constructs.  In science, data models delineate the 
structure of information generated during scientific 
investigations (e.g., experiments and field work), and 
so their structure concretizes scientific theories and/or 
experimental protocols.  Meanwhile, data modeling 
has to balance the complexity of human concepts with 
a predictability conducive to software and computational 
treatments; data modeling thereby helps 
expose the boundary, in human cognition, between 
what is mechanical and what is not \mdash{} between 
the \q{mind as computer} metaphor and the  
philosophies of \q{situational} and \q{embodied} cognition 
which push against it.    
}
\p{With its resonance against these larger themes, data modeling is 
just as an operational prerequisite for scientific 
or technology research \mdash{} and then the  
conversion of new discoveries into new technologies with 
practical benefits \mdash{} but also an interdisciplinary 
nexus informed by and relevant to Computer Science, 
mathematics, Philosophy of Science, Sociology of Knowledge, 
formal semantics, and so forth.
}
\p{One key interdisciplinary question is: how can 
data models be expressive enough to represent 
cultural and scientific ideas and artifacts 
\mdash{} without any sense of conceptual 
mismatch or simplification \mdash{} but also serve 
a software ecosystem?  To be employed, that is, 
in a context where data structures require 
sufficient stability and classifiability that they 
are amenable to algorithms and mutations to accommodate 
different software roles, such as database and 
\GUI{} presentations?  Data models should be 
\i{systematic} so that they engender safe, 
reliable code.  On the other hand, digital resources 
should be expressive enough to represent 
complex concepts without \q{dehumanizing} 
their structure \mdash{} failing to recognize connections 
or distinctions which are part of human conceptualization, 
even if they are technically challenging to model computationally.    
}
\p{Achieving all of these goals involves a certain balancing 
act, where data repositories are modeled via expressive, 
fine-grained prototypes without becoming too unstructured, 
or too heterogeneous, for rigorous software implementations.  
The technical terrain of Ontology-based or type-theoretic 
modeling can therefore be seen as a drive to 
expand models' expressiveness as far as possible, but without 
losing models' underlying formal rigor and tractability.  
In terms of data models, this can be reflected in the 
evolution from fixed-field structures (like spreadsheets and relational 
databases) to labeled-graph Ontologies to Hypergraphs and 
other multi-scale graph paradigms.  Parallel to the emergence 
of Semantic Web technology there is also a body of research in 
Scientific Computing, where expressiveness translates to 
modeling strategies which encapsulate scientific theories and 
workflows \mdash{} cf. Object-Oriented simulations 
(\cite{Telea}, \cite{TeleaWijk} being a good case-study) and 
such formats or approaches as Conceptual Space theory 
(in science and linguistics) and Conceptual Space Markup Language 
(\cite{RaubalAdams}, \cite{RaubalAdamsCSML},  
\cite{RaubalAdamsMore}, \cite{IgorDouven}, \cite{Zenker},
\cite{HolmqvistDimensions},
\cite{MartinRaubal}, \cite{Strle}).
Meanwhile, in type theory, a similar 
impetus leads from the simple type systems of Typed Lambda 
Calculi through to Dependent Types, typestate, effect systems,
Object-Orientation, and other properties of modern programming 
environments.
}
\p{Whatever their features, data models are ultimately 
only as usable as the software that receives them.  
Applications may be importing CyberPhysical measurements 
\q{in real time} or affording access to archived 
research data sets, but in each case the structured 
formats of shared and/or persisted information must 
be transformed into interactive (usually \GUI{}-based)  
presentations if applications are to qualify as 
productive viewers onto the relevant information space.  
This is how we should understand 
the criterion of expressiveness: expressiveness at the 
modeling level is a means to an end; the ultimate 
goal is \q{expressive} software, i.e., software whose 
layout, visual presentations, and interactive features/responsiveness 
render applications effective vehicles for interfacing 
with complex, nuanced digital content.
Ultimately, then, data models are effective to the 
extent that they promote effective software engineering 
for the applications that transform modeled data into 
user-facing digital content.  
}
\p{On the other hand, this leaves room for differences in 
what is prioritized: data models can be targeted at 
a narrow, specialized set of software end-points, or 
can be designed flexibly to work with a diversity of 
software products, in the present and going forward.  
Broader application-scope is desirable in theory, 
but practically speaking a data model which is open-ended 
enough to work with a range of software components 
is potentially too provisional, or insufficiently 
detailed, to promote the highest-quality software. 
}
\p{Information Technology in the last one or two decades 
evidently has favored general-purpose data models 
\mdash{} or at least serialization techniques \mdash{} which 
exist in isolation from applications that work 
with them.  Canonical examples would be \JSON{}, \XML{}, 
and \RDF{}.  Conceptually, however, data models' 
most important manifestation are in the software 
components where they are shared \mdash{} sent (perhaps 
indirectly via a generated archive) and received. 
To the degree that multi-purpose formats like 
\XML{} are beneficial, there merits are in part that 
developers can anticipate the code that generated 
and/or will receive the data: while programmers do not 
necessarily just write code off of an \XML{} sample 
(or corresponding Document Type Declaration), any 
\XML{} document or \DTD{} gives us a rough idea of 
what its client code would look like. 
}
\p{Nevertheless, for robust software engineering we should 
aspire to something more rigorous than that.  In effect, 
we should consider documentation of components which 
send and/or receive data structures to be an intrinsic 
aspect of rigorous data modeling itself: description of 
the procedures which construct, serialize/deserialize, 
validate, and transform data structures, particularly 
those procedures supplying functionality determinative 
of their components' ability to be part of a 
conformant data-sharing network.  In this
sense data and code modeling coincide.  In 
particular, characterization of individual 
procedures \mdash{} their types, assumptions, and 
requirements \mdash{} is an essential building-block 
of data models generally.  Data structures 
can be indirectly systematized in terms of 
the procedures which act upon them.  
}
\p{With this background, the code archive supplementing 
this chapter operationalizes the notion of  
\q{Procedural Hypergraph Ontologies}, combining 
features of Procedural Ontologies and of 
Directed Hypergraphs that I have presented in 
this chapter.  Procedural Hypergraph Ontologies 
extend (or diverge from) conventional 
Semantic Web Ontology partly by orienting toward 
Hypergraphs, but more substantially by centering 
on this procedural dimension: the role of an
Ontology being to describe components' procedural
interface as well as their targeted data structures.  
Using a phrase I introduced above (page \hyperref[POSC]{\pageref{POSC}}), 
this technology aspires to promote a \i{procedure-oriented, software-centric} 
(\q{\POSC{}}) understanding of data-sharing pipelines. 
}
\p{In particular, the demo presents both a hypergraph serialization 
format and methodology for generating interface descriptions, 
based on channel complexes.  The demo code shows a compilation process 
which works with channel groups, branching off into a runtime 
engine which actually evaluates channel packages and, separately, 
algorithms to compile information about procedure signatures and 
function calls.  This last capability can be a point for embedding 
more detailed Interface Definition metadata, including via the non-standard 
channel protocols I have discussed in this chapter.  
Both static data structures and compiled channel groups 
translate to a Hypergraph format, which thereby serves as 
a common denominator between code and data.  
}
\p{Architecturally, then, the demo includes several data sets 
repackaged in a hypergraph-serialization format, and, 
simultaneously, application-level code which bridges 
the data sets to \GUI{} components.  The code base parses 
serialized hypergraphs to in-memory hypergraphs and then 
traverses them, using a kind of visitor pattern, to 
build in-memory \Cpp{} objects, which in turn are mapped 
to \GUI{} objects.  So this chain of processing steps 
models hypergraph-based data representations and 
the logistics of incoporating them at the application level.  
At the same time, \Cpp{} objects reconstituted from 
hypergraphs \mdash{} as well as the \GUI{} components 
which receive them \mdash{} are documented with an 
Interface Description Language that employs channels 
for articulating procedural signatures, an \IDL{} 
which in turn compiles to hypergraph structures 
leveraged for various code-analysis tasks 
(for example, generating a testing mechanism integrated 
with the application code).    
}
\p{The techniques thereby demonstrated can be practically adopted 
in several ways.  On the one hand, concepts like channels 
and preconstructors can be applied to mainstream programming 
languages such as \Cpp{}, becoming 
new design patterns or new coding practices that, over 
a large code base, can help produce components which are 
statically analyzable and (by systematically documenting and 
validating of coding assumptions) priotize safety at runtime.  
On the other hand (as profiled via special languages and 
Intermediate Representations in the demo), the techniques 
I have outlined can be used for new data and code models 
which guide, test, and/or retroactively analyze software 
components (e.g., using Channel Algebra in a fine-grained 
Interface Definition Language).       
}
\p{Aside from these practical applications, moreover, 
I contend that channels and Channelized Hypergraphs 
can be of interest in topics like linguistics and 
the Philosophy of Science as well \mdash{} insofar 
as data models and representations of 
information flow ensapsulate the 
structure of scientific theories, and the 
conceptual networks that lie behind 
Natural Language as well as formal semantics. 
}
